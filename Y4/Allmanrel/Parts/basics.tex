\section{Basic Concepts}

\paragraph{A Note on Minkowski Space}
In special relativity we work with Minkowski space, which is an affine space with a so-called pseudo-metric. This is a metric which is not positive definite, but instead a metric which has only non-zero eigenvalues (and is thus termed non-degenerate). We will work with the signature $(1, 3)$, meaning that there are three eigenvalues of $-1$ and one eigenvalue $1$.

\paragraph{Problems With Gravitation}
Newton's law of gravitation states that
\begin{align*}
	\vb{a} = \frac{1}{m_{\text{I}}}\vb{f}_{\text{G}} = -\frac{m_{\text{G}}}{m_{\text{I}}}\grad{\Phi},\ \Phi = 4\pi G\rho.
\end{align*}
There are a few possible problems and peculiarities with this, namely:
\begin{itemize}
	\item It has no explicit time dependence, and can therefore not hold by itself in special relativity.
	\item As the mass density does not transform as a Lorentz scalar, we cannot generalize gravitation to special relativity like we did with electromagnetism.
	\item The ratio between gravitational and inertial mass is the same for all particles. This makes gravitation stand out as a fundamental force.
\end{itemize}

\paragraph{The Equivalence Principle}
The equivalence principles states that in a freely-falling, non-rotating- spatially small laboratory, the laws of physics are those of special relativity.

\paragraph{An Extended Description of Spacetime}
Einstein proposed a solution to the issue of gravitation that also incorporates the equivalence principle.

Einstein's idea was to extend the special theory of relativity to a curved spacetime and propose that this spacetime is bent by matter. This would mean that free particles move along geodesics in this spacetime, explaining the equality of gravitational and inertial mass. For the equivalence principle to hold, i.e. for it to be possible to locally construct a Minkowski spacetime, spacetime must be a pseudo-Riemannian manifold.

This transition is in fact not as big a leap as one might expect. As we have seen in special relativity, simultaneity is no longer guaranteed by the physics, so surfaces of constant time could at least be slanted. As the introduction of spacetime made it possible in principle to introduce curvilinear coordinates on time as well, we have yet to add anything. The only addition in GR is the extension of spacetime from an affine space to a manifold.

In conclusion, we will describe spacetime as a $4$-dimensional manifold immersed in $5$-dimensional spacetime with a pseudometric of signature $(1, 3)$ and the Levi-Civita connection imposed on it.

\paragraph{Comoving Observers}
A comoving observer is one that has fixed spatial coordinates.

\paragraph{Kinematics of Test Particles}
A test particle is a particle that itself does not affect the spacetime. Such particles can generally move through spacetime, along curves called world lines. With this motion comes the $4$-velocity $V$, defined as the normalized tangent to a world line. In special relativity we could also define a proper acceleration by differentiating with respect to proper time. In general relativity we replace this with the $4$-acceleration $A = \dcov{V}{V} = \dcov{\dot{\gamma}}{\dot{\gamma}}$.  We may also define the proper acceleration $\alpha$, which satisfies $\alpha^{2} = -A^{2} = -g(A, A)$. As $g(V, V) = 1$, we have
\begin{align*}
	\dcov{\mu}{g(V, V)} = 2g(V, A) = 0,
\end{align*}
which implies that $A$ is space-like. Note that the curve parameter we use is $\tau$, which is the proper time and a measure of length in spacetime.

\paragraph{Free Particles}
A free particle moves along geodesics, meaning $A = 0$.

\paragraph{$4$-Momentum and $4$-Force}
We also define the $4$-momentum $P = mV$ and the $4$-force $F = \dcov{V}{P}$.

\paragraph{Simultaneity and Distance}
Two events are simultaneous if they are on the same hypersurface of constant $t$. As this depends very much on the choice of coordinates on spacetime, this notion is not at all well-defined.

Similarly, distances are defined along simultaneities and are equally ill-defined. It turns out that the only sense of distance that all observers can agree on is proper time.

\paragraph{Static and Stationary Spacetime}
If there exists a stime-like Killing field of a spacetime, it is stationary. If the spacetime is also orthogonal to a family of $3$-surfaces, the spacetime is static. The consequences of the latter is that the metric has no off-diagonal components.

\paragraph{The Einstein Field Equations}
The equations describing the metric are the Einstein field equations. We will derive them from a variational principle, starting with the field equations for vacuum. The action from which we will obtain the field equations is the Einstein-Hilbert action
\begin{align*}
	S_{\text{EH}} = -\frac{M_{\alpha}^{2}}{2}\integ[4]{}{}{\chi}{\R\sqrt{-\det(g)}}.
\end{align*}

To derive a set of equations describing the extremum of this action, we will need to differentiate the two factors. We choose to do so with respect to the dual metric components. Considering the metric determinant first, we have
\begin{align*}
	\pdv{g^{\mu\nu}}(g^{\rho\sigma}g_{\rho\sigma}) = \pdv{g^{\rho\sigma}}{g^{\mu\nu}}g_{\rho\sigma} + g^{\rho\sigma}\pdv{g^{\rho\sigma}}{g_{\mu\nu}} = 0.
\end{align*}
Next, we study the quantity $\tr(\ln(g))$. We have
\begin{align*}
	\pdv{\tr(\ln(g))}{g^{\mu\nu}} = \tr(\pdv{\ln(g)}{g^{\mu\nu}}) = \tr(g^{-1}\pdv{g}{g^{\mu\nu}}).
\end{align*}
Using the identity $\tr(\ln(g)) = \ln(\det(g))$, we have
\begin{align*}
	\tr(g^{-1}\pdv{g}{g^{\mu\nu}}) = \frac{1}{\det(g)}\pdv{\det(g)}{g^{\mu\nu}},
\end{align*}
hence
\begin{align*}
	\pdv{\det(g)}{g^{\mu\nu}} = \det(g)g^{\rho\sigma}\pdv{g_{\rho\sigma}}{g^{\mu\nu}} = -\det(g)g_{\rho\sigma}\pdv{g^{\rho\sigma}}{g^{\mu\nu}}
\end{align*}
and
\begin{align*}
	\pdv{\sqrt{-\det(g)}}{g^{\mu\nu}} = \frac{1}{2\sqrt{-\det(g)}}\det(g)g^{\rho\sigma}\pdv{g^{\rho\sigma}}{g^{\mu\nu}} = -\frac{1}{2}\sqrt{-\det(g)}g_{\rho\sigma}\pdv{g^{\rho\sigma}}{g^{\mu\nu}}.
\end{align*}

For the Ricci scalar we have $\R = g^{\mu\nu}R_{\mu\nu}$, hence
\begin{align*}
	\pdv{\R}{g^{\mu\nu}} = R_{\mu\nu} + g^{\rho\sigma}\pdv{R_{\rho\sigma}}{g^{\mu\nu}}.
\end{align*}
To differentiate the Ricci tensor, we first use the fact that differences in connection coefficients, and thus their derivatives, transform as tensors. Thus
\begin{align*}
	\dcov{\alpha}{\pdv{\chris{\beta}{\gamma}{\delta}}{g^{\mu\nu}}} = \del{\alpha}{\pdv{\chris{\beta}{\gamma}{\delta}}{g^{\mu\nu}}} + \chris{\beta}{\alpha}{\rho}\pdv{\chris{\rho}{\gamma}{\delta}}{g^{\mu\nu}} - \chris{\rho}{\alpha}{\gamma}\pdv{\chris{\beta}{\rho}{\delta}}{g^{\mu\nu}} - \chris{\rho}{\alpha}{\delta}\pdv{\chris{\beta}{\gamma}{\rho}}{g^{\mu\nu}}.
\end{align*}
Now we have
\begin{align*}
	\pdv{R_{\rho\sigma}}{g^{\mu\nu}} &= \pdv{R^{\alpha}_{\rho\alpha\sigma}}{g^{\mu\nu}} \\
	                                 &= \pdv{g^{\mu\nu}}\left(\del{\alpha}{\chris{\alpha}{\sigma}{\rho}} - \del{\sigma}{\chris{\alpha}{\alpha}{\rho}} + \chris{\gamma}{\sigma}{\rho}\chris{\alpha}{\alpha}{\gamma} - \chris{\gamma}{\alpha}{\rho}\chris{\alpha}{\sigma}{\gamma}\right) \\
	                                 &= \pdv{\del{\alpha}{\chris{\alpha}{\sigma}{\rho}}}{g^{\mu\nu}} - \pdv{\del{\sigma}{\chris{\alpha}{\alpha}{\rho}}}{g^{\mu\nu}} + \chris{\gamma}{\sigma}{\rho}\pdv{\chris{\alpha}{\alpha}{\gamma}}{g^{\mu\nu}} + \chris{\alpha}{\alpha}{\gamma}\pdv{\chris{\gamma}{\sigma}{\rho}}{g^{\mu\nu}} - \chris{\gamma}{\alpha}{\rho}\pdv{\chris{\alpha}{\sigma}{\gamma}}{g^{\mu\nu}} - \chris{\alpha}{\sigma}{\gamma}\pdv{\chris{\gamma}{\alpha}{\rho}}{g^{\mu\nu}} \\
	                                 &= \pdv{\del{\alpha}{\chris{\alpha}{\sigma}{\rho}}}{g^{\mu\nu}} + \chris{\alpha}{\alpha}{\gamma}\pdv{\chris{\gamma}{\sigma}{\rho}}{g^{\mu\nu}} - \chris{\gamma}{\alpha}{\rho}\pdv{\chris{\alpha}{\gamma}{\sigma}}{g^{\mu\nu}} - \chris{\gamma}{\alpha}{\sigma}\pdv{\chris{\alpha}{\rho}{\gamma}}{g^{\mu\nu}} - \dcov{\sigma}{\chris{\alpha}{\alpha}{\rho}} \\
	                                 &= \dcov{\alpha}{\chris{\alpha}{\sigma}{\rho}} - \dcov{\sigma}{\chris{\alpha}{\alpha}{\rho}}.
\end{align*}
As the Levi-Civita connection is metric compatible, we have
\begin{align*}
	\pdv{\R}{g^{\mu\nu}} = R_{\mu\nu} + \dcov{\alpha}{(g^{\rho\sigma}\chris{\alpha}{\sigma}{\rho})} - \dcov{\sigma}{(g^{\rho\sigma}\chris{\alpha}{\alpha}{\rho})}.
\end{align*}
The latter term may be ignored by converting it to a contribution at the boundary.

There are now two ways to proceed. The first is to note that the Lagrangian density has no functional dependence on the derivatives of the metric, implying that its derivatives with respect to the metric are zero, implying
\begin{align*}
	\pdv{g^{\mu\nu}}(\R\sqrt{-\det(g)}) = R_{\mu\nu}\sqrt{-\det(g)} - \frac{1}{2}\R\sqrt{-\det(g)}g_{\rho\sigma}\pdv{g^{\rho\sigma}}{g^{\mu\nu}} = 0,
\end{align*}
yielding the Einstein field equations
\begin{align*}
	R_{\mu\nu} - \frac{1}{2}\R g_{\mu\nu} = G_{\mu\nu} = 0.
\end{align*}
The other is to consider variations of the metric and their impact on the action. This will yield
\begin{align*}
	\var{S} = -\frac{M_{\alpha}^{2}}{2}\integ[4]{}{}{\chi}{\pdv{\R\sqrt{-\det(g)}}{g^{\mu\nu}}\var{g^{\mu\nu}}} = -\frac{M_{\alpha}^{2}}{2}\integ[4]{}{}{\chi}{\sqrt{-\det(g)}\left(R_{\mu\nu} - \frac{1}{2}\R g_{\mu\nu}\right)\var{g^{\mu\nu}}},
\end{align*}
and extremization yields the same equations.

\paragraph{The Energy-Momentum Tensor}
To obtain the field equations for other cases we will add terms to the action. In general, by adding a term $S_{\text{matter}}$ to the action with the coefficient $\frac{M_{\alpha}^{2}}{2}$ baked into it, we obtain
\begin{align*}
	R_{\mu\nu} - \frac{1}{2}g_{\mu\nu}\R = 8\pi GT_{\mu\nu},
\end{align*}
where
\begin{align*}
	T_{\mu\nu} = \frac{2}{\sqrt{-\det(g)}}\fdv{S_{\text{matter}}}{g^{\mu\nu}}
\end{align*}
is called the energy-momentum tensor.

\paragraph{Alternative Field Equations}
Starting from the Einstein field equation, we may contract with the metric according to
\begin{align*}
	g^{\mu\nu}R_{\nu\sigma} - \frac{1}{2}\kdelta{\sigma}{\mu}\R = 8\pi Gg^{\mu\nu}T_{\nu\sigma} = 8\pi G\tensor{T}{^{\mu}_{\sigma}},
\end{align*}
and setting $\mu = \sigma$ and summing yields
\begin{align*}
	8\pi G\tensor{T}{^{\mu}_{\mu}} = g^{\mu\nu}R_{\nu\mu} - 2\R = -\R,
\end{align*}
which we may insert back into the Einstein field equations to obtain
\begin{align*}
	R_{\mu\nu} = 8\pi G\left(T_{\mu\nu} - \frac{1}{2}g_{\mu\nu}\tensor{T}{^{\sigma}_{\sigma}}\right).
\end{align*}

\paragraph{The Schwarzschild Solution}
The Schwarzschild solution is the simplest solution for a spherically symmetric metric.

To derive it, we will use spherical coordinates to describe the space part, implying that we construct spacetime as a combination of spherical shells. The metric will then take the form
\begin{align*}
	\dd{s}^{2} = f(t, r)\dd{t}^{2} - g(t, r)\dd{r}^{2} - r^{2}\dd{\Omega}^{2},\ \dd{\Omega}^{2} = \dd{\theta}^{2} + \sin[2](\theta)\dd{\phi}^{2}
\end{align*}
in units where $c = 1$. The proceeding work will involve dividing the functions $f$ and $g$, so we define $f = e^{2\alpha},\ g = e^{2\beta}$. As the metric is diagonal we can write
\begin{align*}
	\chris{\sigma}{\mu}{\nu} = \frac{1}{2g_{\sigma\sigma}}(\del{\nu}{g_{\mu\sigma}} + \del{\mu}{g_{\sigma\nu}} - \del{\sigma}{g_{\mu\nu}}),\ \text{no sum over $\sigma$.}
\end{align*}
Again, as the metric is diagonal, at least two of the indices must be equal, immediately identifying any Christoffel symbol with three different indices as zero. The diagonality of the metric and the angle independence of the time and radial components also implies
\begin{align*}
	\chris{t}{t}{\theta} = \chris{t}{\theta}{t} = \chris{t}{t}{\phi} = \chris{t}{\phi}{t} = \chris{r}{r}{\theta} = \chris{r}{\theta}{r} = \chris{r}{r}{\phi} = \chris{r}{\phi}{r} = \chris{\theta}{t}{t} = \chris{\theta}{r}{r} = \chris{\phi}{t}{t} = \chris{\phi}{r}{r} = 0.
\end{align*}
The time independence of the solid angle also implies $\chris{t}{\theta}{\theta} = \chris{t}{\phi}{\phi} = \chris{t}{\theta}{t} = \chris{t}{t}{\theta} = \chris{t}{\phi}{t} = \chris{t}{t}{\phi} = \chris{\theta}{t}{t} = \chris{\phi}{t}{t} = \chris{\theta}{\theta}{t} = \chris{\theta}{t}{\theta} = \chris{\phi}{\phi}{t} = \chris{\phi}{t}{\phi} = 0$. Next, as the metric is independent of $\phi$, we must also have $\chris{\theta}{\theta}{\phi} =	\chris{\theta}{\phi}{\theta} = \chris{\phi}{\theta}{\theta} = \chris{\phi}{\phi}{\phi} = 0$. We also find $\chris{\theta}{\theta}{\theta} = 0$ as the $\theta$ component is independent of $\theta$. The possibly non-zero ones are
\begin{align*}
	\chris{t}{t}{t} &= \frac{1}{2}e^{-2\alpha}\cdot 2\del{t}{\alpha}e^{2\alpha} = \del{t}{\alpha}, \ \chris{t}{t}{r} = \chris{t}{r}{t} = \frac{1}{2}e^{-2\alpha}\cdot 2\del{r}{\alpha}e^{2\alpha} = \del{r}{\alpha},\ \chris{t}{r}{r} = -\frac{1}{2}e^{-2\alpha}\cdot -2\del{t}{\beta}e^{2\beta} = \del{t}{\beta}e^{2(\beta - \alpha)}, \\
	\chris{r}{t}{t} &= -\frac{1}{2}e^{-2\beta}\cdot -2\del{r}{\alpha}e^{2\alpha} = \del{r}{\alpha}e^{2(\alpha - \beta)},\ \chris{r}{t}{r} = \chris{r}{r}{t} = -\frac{1}{2}e^{-2\beta}\cdot -2\del{t}{\beta}e^{2\beta} = \del{t}{\beta},\ \chris{r}{r}{r} = -\frac{1}{2}e^{-2\beta}\cdot -2\del{r}{\beta}e^{2\beta} = \del{r}{\beta}, \\
	\chris{r}{\theta}{\theta} &= -\frac{1}{2}e^{-2\beta}\cdot 2r = -re^{-2\beta},\ \chris{r}{\phi}{\phi} = -\frac{1}{2}e^{-2\beta}\cdot 2r\sin[2](\theta) = -r\sin[2](\theta)e^{-2\beta},\ \chris{\theta}{r}{\theta} = \chris{\theta}{\theta}{r} = -\frac{1}{2r^{2}}\cdot -2r = \frac{1}{r}, \\
	\chris{\theta}{\phi}{\phi} &= -\frac{1}{2r^{2}}\cdot 2r^{2}\sin(\theta)\cos(\theta) = -\sin(\theta)\cos(\theta),\ \chris{\phi}{r}{\phi} = \chris{\phi}{\phi}{r} = -\frac{1}{2r^{2}\sin[2](\theta)}\cdot -2r\sin[2](\theta) = \frac{1}{r}, \\
	\chris{\phi}{\theta}{\phi} &= \chris{\phi}{\phi}{\theta} = -\frac{1}{2r^{2}\sin[2](\phi)}\cdot -2r^{2}\sin(\theta)\cos(\theta) = \cot(\theta).
\end{align*}

We will now need to solve the Einstein field equations. We will use the formulation purely in terms of the energy-momentum tensor, which yields $R_{\mu\nu} = 0$ in vacuum. We have
\begin{align*}
	R_{\mu\nu} = \del{\rho}{\chris{\rho}{\nu}{\mu}} - \del{\nu}{\chris{\rho}{\rho}{\mu}} + \chris{\sigma}{\nu}{\mu}\chris{\rho}{\rho}{\sigma} - \chris{\sigma}{\rho}{\mu}\chris{\rho}{\nu}{\sigma}.
\end{align*}
Let us commence by finding restrictions on the functions $\alpha$ and $\beta$. Cheating by looking into the future I find
\begin{align*}
	R_{tr} =& \del{\rho}{\chris{\rho}{r}{t}} - \del{r}{\chris{\rho}{\rho}{t}} + \chris{\sigma}{r}{t}\chris{\rho}{\rho}{\sigma} - \chris{\sigma}{\rho}{t}\chris{\rho}{r}{\sigma} \\
	       =& \del{t}{\del{r}{\alpha}} + \del{r}{\del{t}{\beta}} - \del{r}{\del{t}{\alpha}} - \del{r}{\del{t}{\beta}} + \chris{t}{r}{t}(\chris{t}{t}{t} + \chris{r}{r}{t}) + \chris{r}{r}{t}(\chris{t}{t}{r} + \chris{r}{r}{r} + \chris{\theta}{\theta}{r} + \chris{\phi}{\phi}{r}) - \chris{t}{t}{t}\chris{t}{r}{t} - \chris{t}{r}{t}\chris{r}{r}{t} \\
	        &- \chris{r}{t}{t}\chris{t}{r}{r} - \chris{r}{r}{t}\chris{r}{r}{r} - \chris{r}{\theta}{t}\chris{\theta}{r}{r} - \chris{r}{\phi}{t}\chris{\phi}{r}{r} - \chris{\theta}{\theta}{t}\chris{\theta}{r}{\theta} - \chris{\phi}{\phi}{t}\chris{\phi}{r}{\phi} \\
	       =& \chris{r}{r}{t}(\chris{t}{t}{r} + \chris{\theta}{\theta}{r} + \chris{\phi}{\phi}{r}) - \chris{r}{t}{t}\chris{t}{r}{r} \\
	       =& \frac{2}{r}\del{t}{\beta} = 0,
\end{align*}
hence $\beta$ is a function of $r$ only. Next:
\begin{align*}
	R_{\theta\theta} =& \del{\rho}{\chris{\rho}{\theta}{\theta}} - \del{\theta}{\chris{\rho}{\rho}{\theta}} + \chris{\sigma}{\theta}{\theta}\chris{\rho}{\rho}{\sigma} - \chris{\sigma}{\rho}{\theta}\chris{\rho}{\theta}{\sigma} \\
	                 =& \del{r}{\chris{r}{\theta}{\theta}} - \del{\theta}{\chris{\phi}{\phi}{\theta}} + \chris{r}{\theta}{\theta}(\chris{t}{t}{r} + \chris{r}{r}{r} + \chris{\theta}{\theta}{r} + \chris{\phi}{\phi}{r}) - \chris{\phi}{\phi}{\theta}\chris{\phi}{\theta}{\phi} - \chris{r}{\theta}{\theta}\chris{\theta}{\theta}{r} - \chris{\theta}{r}{\theta}\chris{r}{\theta}{\theta} \\
	                 =& -\del{r}{(re^{-2\beta})} - \del{\theta}{\cot(\theta)} - re^{-2\beta}\left(\del{r}{\alpha} + \del{r}{\beta} + \frac{2}{r}\right) - \cot[2](\theta) + 2e^{-2\beta} \\
	                 =& -e^{-2\beta} + 2r\del{r}{\beta}e^{-2\beta} + \csc[2](\theta) - re^{-2\beta}\left(\del{r}{\alpha} + \del{r}{\beta} + \frac{2}{r}\right) - \cot[2](\theta) + 2e^{-2\beta} \\
	                 =& 1 + e^{-2\beta}\left(-1 + 2r\del{r}{\beta} - r\del{r}{\alpha} - r\del{r}{\beta} - 2 + 2\right) \\
	                 =& 1 + e^{-2\beta}\left(r\del{r}{\beta} - 1 - r\del{r}{\alpha}\right) = 0.
\end{align*}
On its own it provides little information, but differentiating this with respect to $t$ yields
\begin{align*}
	\del{t}{\del{r}{\alpha}} = 0,
\end{align*}
with solution $\alpha = u(t) + v(r)$.

The function $u$ may be eliminated by a change of variables such that $\dd{t}\to e^{-2u(t)}\dd{t}$, which yields
\begin{align*}
	\dd{s}^{2} = e^{2v(r)}\dd{t}^{2} - 2e^{\beta}\dd{r}^{2} - r^{2}\dd{\Omega}^{2}.
\end{align*}
Note that with these simplifications we have $\chris{t}{t}{t} = \chris{t}{r}{r} = \chris{r}{r}{t} = \chris{r}{t}{r} = 0$ and no remaining time-dependent components. Next we compute the components
\begin{align*}
	R_{tt} =& \del{\rho}{\chris{\rho}{t}{t}} - \del{t}{\chris{\rho}{\rho}{t}} + \chris{\sigma}{t}{t}\chris{\rho}{\rho}{\sigma} - \chris{\sigma}{\rho}{t}\chris{\rho}{t}{\sigma} \\
	       =& \del{r}{\chris{r}{t}{t}} + \chris{r}{t}{t}\chris{r}{r}{r} + \chris{r}{t}{t}\chris{\theta}{\theta}{r} + \chris{r}{t}{t}\chris{\phi}{\phi}{r} - \chris{\sigma}{t}{t}\chris{t}{t}{\sigma} - \chris{\sigma}{\sigma}{t}\chris{\sigma}{t}{\sigma} \\
	       =& \del{r}{(\del{r}{\alpha}e^{2(\alpha - \beta)})} + \del{r}{\alpha}e^{2(\alpha - \beta)}\del{r}{\beta} + \del{r}{\alpha}e^{2(\alpha - \beta)}\frac{1}{r} + \del{r}{\alpha}e^{2(\alpha - \beta)}\frac{1}{r} - \del{r}{\alpha}e^{2(\alpha - \beta)}\del{r}{\alpha} \\
	       =& e^{2(\alpha - \beta)}\del[2]{r}{\alpha} + 2\del{r}{\alpha}(\del{r}{\alpha} - \del{r}{\beta})e^{2(\alpha - \beta)} + e^{2(\alpha - \beta)}\del{r}{\alpha}\del{r}{\beta} + e^{2(\alpha - \beta)}\frac{2\del{r}{\alpha}}{r} - (\del{r}{\alpha})^{2}e^{2(\alpha - \beta)} \\
	       =& e^{2(\alpha - \beta)}\left(\del[2]{r}{\alpha} + \del{r}{\alpha}\left(\del{r}{\alpha} - \del{r}{\beta} + \frac{2}{r}\right)\right) = 0, \\
	R_{rr} =& \del{\rho}{\chris{\rho}{r}{r}} - \del{r}{\chris{\rho}{\rho}{r}} + \chris{\sigma}{r}{r}\chris{\rho}{\rho}{\sigma} - \chris{\sigma}{\rho}{r}\chris{\rho}{r}{\sigma} \\
	       =& \del{r}{\chris{r}{r}{r}} - \del{r}{\chris{t}{t}{r}} - \del{r}{\chris{r}{r}{r}} - \del{r}{\chris{\theta}{\theta}{r}} - \del{r}{\chris{\phi}{\phi}{r}} + \chris{r}{r}{r}(\chris{t}{t}{r} + \chris{r}{r}{r} + \chris{\theta}{\theta}{r} + \chris{\phi}{\phi}{r}) - \chris{r}{r}{r}\chris{r}{r}{r} - \chris{t}{t}{r}\chris{t}{r}{t} - \chris{r}{r}{r}\chris{r}{r}{r} \\
	        &- \chris{\theta}{\theta}{r}\chris{\theta}{r}{\theta} - \chris{\phi}{\phi}{r}\chris{\phi}{r}{\phi} \\
	       =& -\del[2]{r}{\alpha} - \del{r}{\frac{1}{r}} - \del{r}{\frac{1}{r}} + \del{r}{\beta}\left(\del{r}{\alpha} + \del{r}{\beta} + \frac{2}{r}\right) - (\del{r}{\beta})^{2} - (\del{r}{\alpha})^{2} - \frac{2}{r^{2}} \\
	       =& -\del[2]{r}{\alpha} + \del{r}{\beta}\left(\del{r}{\alpha} + \frac{2}{r}\right) - (\del{r}{\alpha})^{2} = 0,
\end{align*}
hence
\begin{align*}
	(\del{r}{\alpha} + \del{r}{\beta})\left(\del{r}{\alpha} + \frac{2}{r}\right) - \del{r}{\alpha}\del{r}{\beta} - (\del{r}{\alpha})^{2} = \frac{2}{r}(\del{r}{u} + \del{r}{\beta}) = 0,
\end{align*}
and $\beta = C - v$. By rescaling $r\to e^{-\frac{1}{2}C}r$ we may set $C = 0$, and find $\beta = -v$. Finally we dig up an old equation and find
\begin{align*}
	1 + e^{-2\beta}\left(r\del{r}{\beta} - 1 - r\del{r}{\alpha}\right) = 1 + e^{-2\beta}\left(2r\del{r}{\beta} - 1\right) = 0,
\end{align*}
which we may solve and write as
\begin{align*}
	\del{r}{(re^{-2\beta})} = 1.
\end{align*}
Integrating this yields
\begin{align*}
	re^{-2\beta} = r - R_{\text{S}},\ e^{-2\beta} = 1 - \frac{R_{\text{S}}}{r}.
\end{align*}

Finally we may put this together to find
\begin{align*}
	\dd{s}^{2} = \left(1 - \frac{R_{\text{S}}}{r}\right)\dd{t}^{2} - \left(1 - \frac{R_{\text{S}}}{r}\right)^{-1}\dd{r}^{2} - r^{2}\dd{\Omega}^{2},
\end{align*}
where $R_{\text{S}}$ is the Schwarzschild radius and we work in units where $c = 1$. To reobtain Newtonian gravity at large distances we would need $R_{\text{S}} = 2MG$.

This metric has singularities at $r = R_{\text{S}}$ and $r = 0$. If you study the curvature invariant $R_{\mu\nu\lambda\sigma}R^{\mu\nu\lambda\sigma}$, however, you find that it is finite at $r = R_{\text{S}}$ and diverges at $r = 0$. This would imply that there exists a smart choice of coordinates in which the singularity at $R_{\text{S}}$ would be eliminated.

This set of coordinates, called Eddington-Finkelstein coordinates, replaces $t$ with a coordinate that has the light-like geodesics as coordinate lines. For a purely radial path the requirement $\dd{s}^{2} = 0$ for such a geodesic yields
\begin{align*}
	\dd{t}^{2} = \left(1 - \frac{R_{\text{S}}}{r}\right)^{-2}\dd{r}^{2},
\end{align*}
with solution
\begin{align*}
	t = u - r - R_{\text{S}}\ln(\frac{r}{R_{\text{S}}} - 1),
\end{align*}
where $u$ is an integration constant labelling the geodesics. This will be the new coordinate. In these coordinates we obtain
\begin{align*}
	\dd{s}^{2} = \left(1 - \frac{R_{\text{S}}}{r}\right)\dd{u}^{2} - 2\dd{u}\dd{r} - r^{2}\dd{\Omega}^{2}.
\end{align*}
Notably, there is now only a singularity at $r = 0$.

For radial light cones in these coordinates, one obtains
\begin{align*}
	\left(\left(1 - \frac{R_{\text{S}}}{r}\right)\dd{u}^{2} - 2\dd{r}\right)\dd{u} = 0,
\end{align*}
with solutions $\dd{u} = 0$ and $\dv{u}{r} = \frac{2}{1 - \frac{R_{\text{S}}}{r}}$. The first case is as discussed above. The second has $\dv{u}{r} > 0$ for $r < R_{\text{S}}$, meaning that world lines moving towards the future are drawn to the singularity at the origin when within the Schwarzschild radius.

To describe space-like world lines, we can use Kruskal-Szekeres coordinates
\begin{align*}
	U = \abs{\frac{r}{R_{\text{S}}} - 1}^{\frac{1}{2}}e^{\frac{r}{2R_{\text{S}}}}\sinh(\frac{t}{2R_{\text{S}}}),\ V = \abs{\frac{r}{R_{\text{S}}} - 1}^{\frac{1}{2}}e^{\frac{r}{2R_{\text{S}}}}\cosh(\frac{t}{2R_{\text{S}}}).
\end{align*}
One finds that the metric is
\begin{align*}
	\dd{s}^{2} = \frac{4R_{\text{S}}^{3}}{r}e^{-\frac{r}{R_{\text{S}}}}(\dd{U}^{2} - \dd{V}^{2}) - r^{2}\dd{\Omega}^{2}.
\end{align*}
In these coordinates geodesics are hyperbolae.

\paragraph{Symmetries and Conserved Quantities}
Assume that a spacetime has a Killing field $K$, and consider a geodesic of the spacetime with tangent $U$. Along the geodesic we then have
\begin{align*}
	\dd{\tau}g_{\mu\nu}K^{\mu}U^{\nu} = \dcov{U}{g_{\mu\nu}K^{\mu}U^{\nu}} = (\dcov{U}{g_{\mu\nu}})K^{\mu}U^{\nu} + g_{\mu\nu}\dcov{U}{K^{\mu}U^{\nu}}.
\end{align*}
As the Levi-Civita connection is metric-compatible, the former term vanishes, and we are left with
\begin{align*}
	\dd{\tau}g_{\mu\nu}K^{\mu}U^{\nu} = g_{\mu\nu}U^{\nu}\dcov{U}{K^{\mu}} + g_{\mu\nu}K^{\mu}\dcov{U}{U^{\nu}}.
\end{align*}
As the path is a geodesic, the latter term vanishes and all that is left is
\begin{align*}
	\dd{\tau}g_{\mu\nu}K^{\mu}U^{\nu} = g_{\mu\nu}U^{\nu}\dcov{U}{K^{\mu}} = g_{\mu\nu}U^{\nu}U^{\sigma}\dcov{\sigma}{K^{\mu}} = U^{\nu}U^{\sigma}\dcov{\sigma}{K_{\nu}}.
\end{align*}
The two first factors are symmetric under permutation of indices, whereas the latter is not, hence this is just $0$. Thus the quantity $g_{\mu\nu}K^{\mu}U^{\nu}$ is a constant of motion along the path.

\paragraph{Symmetries of the Schwarzschild Solution}
We note that $\del{t}{}$ and $\del{\phi}{}$ are both Killing fields of the Schwarzschild solution. For $r > R_{\text{S}}$ one find that $\del{t}{}$ is time-like.

For a general path we define
\begin{align*}
	\sqrt{2E} &= g(\del{t}{}, \dot{\gamma}) = \left(1 - \frac{R_{\text{S}}}{r}\right)\dot{t}, \\
	L         &= g(\del{\phi}{}, \dot{\gamma}) = r^{s}\sin[2](\theta)\dot{\phi}, \\
	\alpha    &= g(\dot{\gamma}, \dot{\gamma}).
\end{align*}
The latter is $1$ for a time-like path and $0$ for a light-like path. By definition we have
\begin{align*}
	\alpha &= \left(1 - \frac{R_{\text{S}}}{r}\right)\dot{t}^{2} - \frac{1}{1 - \frac{R_{\text{S}}}{r}}\dot{r}^{2} - r^{2}\sin[2](\theta)\dot{\phi}^{2} = \frac{2E}{1 - \frac{R_{\text{S}}}{r}} - \frac{1}{1 - \frac{R_{\text{S}}}{r}}\dot{r}^{2} - \frac{L^{2}}{r^{2}},
\end{align*}
which may be written as
\begin{align*}
	E - \frac{1}{2}\dot{r}^{2} = \frac{1}{2}\left(\alpha + \frac{L^{2}}{r^{2}}\right)\left(1 - \frac{R_{\text{S}}}{r}\right).
\end{align*}
This looks like the relation describing the potential energy of a particle in classical mechanics.

\paragraph{Frequency Shift}
A wave generally has a phase which depends on both position and time. We define the frequency of a wave as $\omega = \dv{\phi}{t}$. For a general world line we define
\begin{align*}
	\omega = \dv{\phi}{\tau} = \dot{\chi}^{a}\del{a}{\phi} = V\phi = \df{\phi}(V).
\end{align*}
$\df{\phi}$ is the dual of the $4$-frequency $N^{\mu}$. It can be shown that $\dcov{N}{\df{\pi}} = 0$, and thus $\dcov{N}{N} = 0$. Rays with tangent $N$ are thus light-like geodesics.

In general an observer will measure a frequency $\df{\phi}(V)$.

\paragraph{Gravitational Time Dilation and Redshift}
Consider a static space time. This means that there exists coordinates $\chi^{a}$ such that $\dd{s}^{2} = \alpha^{2}\dd{t}^{2} - h_{ij}\dd{\chi}^{i}\dd{\chi}^{j}$, where the components of the metric are functions of the spatial coordinates only. Now consider two static observers in this spacetime observing two events. As the geodesics have time translation symmetry due to the metric being static, the two observers must observe the same time difference $t_{0}$. Each observer observes an elapsed proper time
\begin{align*}
	\tau = \integ{t}{t + t_{0}}{t}{\sqrt{g_{tt}}} = \alpha t_{0}.
\end{align*}
This means that
\begin{align*}
	\frac{\tau_{A}}{\tau_{B}} = \frac{\alpha_{A}}{\alpha_{B}}.
\end{align*}

From this the gravitational redshift follows. If the two events are successive crests of a light pulse, we find
\begin{align*}
	\frac{f_{B}}{f_{A}} = \frac{\alpha_{A}}{\alpha_{B}}.
\end{align*}

\paragraph{Ideal Fluids}
An ideal fluid is a substance such that its energy-momentum tensor is of the form
\begin{align*}
	T^{\mu\nu} = (\rho_{0} + p)u^{\mu}u^{\nu} - pg^{\mu\nu},
\end{align*}
where $u^{\mu}$ is the $4$-velocity of the rest frame of the fluid and $\rho_{0}$ and $p$ are the energy density and pressure of the fluid as measured in the rest frame.

\paragraph{The Weak Field Limit}
We will now study the Einstein field equations in a limit where the effects of general relativity are weak, in the hopes of finding some limit that reproduces Newtonian gravity. We will do this by linearizing the metric according to $g_{\mu\nu} = \eta_{\mu\nu} + h_{\mu\nu}$, where $\eta$ is the Minkowski metric and all components of $h$ are small. To raise its components, we use the fact that the metric should produce the Kronecker delta to obtain
\begin{align*}
	\kdelta{\mu}{\sigma} = g^{\mu\nu}g_{\nu\sigma} = (\eta^{\mu\nu} + f^{\mu\nu})(\eta_{\nu\sigma} + h_{\nu\sigma}) = \eta^{\mu\nu}\eta_{\nu\sigma} + \eta^{\mu\nu}h_{\nu\sigma} + f^{\mu\nu}\eta_{\nu\sigma},
\end{align*}
where the perturbations of the inverse metric must also necessarily be small. This yields
\begin{align*}
	f^{\mu\nu} = -\eta^{\sigma\nu}\eta^{\rho\mu}h_{\rho\sigma}.
\end{align*}

When computing the curvature tensor, we will need squares of the Christoffel symbols. These are linear in the perturbations and may therefore be neglected, yielding
\begin{align*}
	\tensor*{R}{^{\mu}_{\nu\lambda\sigma}} = \del{\lambda}{\chris{\mu}{\sigma}{\nu}} - \del{\sigma}{\chris{\mu}{\lambda}{\nu}}
\end{align*}
and
\begin{align*}
	R_{\nu\sigma} = \frac{1}{2}(\del{\nu}{\del{\mu}{\tensor{h}{^{\mu}_{\sigma}}}} - \square h_{\nu\sigma} - \del{\nu}{\del{\sigma}{h}} + \del{\sigma}{\del{\mu}{\tensor{h}{^{\mu}_{\nu}}}}),
\end{align*}
where we have introduced the d'Alembertian $\eta^{\mu\nu}\del{\mu}{\del{\nu}{}}$ and $h = \tensor{h}{^{\mu}_{\mu}}$. The Einstein tensor is
\begin{align*}
	G_{\nu\sigma} = \frac{1}{2}(\del{\nu}{\del{\mu}{\tensor{h}{^{\mu}_{\sigma}}}} - \square h_{\nu\sigma} - \del{\nu}{\del{\sigma}{h}} + \del{\sigma}{\del{\mu}{\tensor{h}{^{\mu}_{\nu}}}} + \eta_{\nu\sigma}\square h).
\end{align*}

To proceed, we choose coordinates such that $\square\chi^{\mu} = 0$. It may be shown that this leads to
\begin{align*}
	g^{\mu\nu}\chris{\lambda}{\mu}{\nu} = 0 \implies \del{\mu}{\tensor{h}{^{\mu}_{\lambda}}} = \frac{1}{2}\del{\lambda}{h}.
\end{align*}
This will yield the Einstein field equations
\begin{align*}
	G_{\nu\sigma} = -\frac{1}{2}(\square h_{\nu\sigma} - \frac{1}{2}\eta_{\nu\sigma}\square h) = 8\pi GT_{\nu\sigma},
\end{align*}
or alternatively, by defining $\bar{h}_{\nu\sigma} = h_{\nu\sigma} - \frac{1}{2}\eta_{\nu\sigma}h$:
\begin{align*}
	\square\bar{h}_{\mu\nu} = -16\pi GT_{\mu\nu}.
\end{align*}

To proceed we say a little more about the Newtonian limit. It should be found at low speeds, meaning that the energy-momentum tensor will be dominated by $T_{00}$, which is the energy density, or equivalently the mass density. The corresponding component of $\bar{h}$ will thus also dominate the other side. It should also be obtained when the involved masses are small. If the situation also evolves slowly, we may neglect time derivatives to obtain
\begin{align*}
	\laplacian{\bar{h}_{00}} = 16\pi G\rho.
\end{align*}
We may thus identify $\bar{h}_{00} = 4\Phi$, where $\Phi$ is the Newtonian gravitational potential. The corresponding spacetime length is
\begin{align*}
	\dd{s}^{2} = (1 + 2\Phi)\dd{t}^{2} - (1 - 2\Phi)\dd{x^{i}}\dd{x^{i}}.
\end{align*}
It can also be shown that the geodesics of this metric corresponds to Newton's laws for such a case.

\paragraph{Gravitational Waves}
We have seen that in the weak field limit the perturbations to the metric in vacuum satisfy the wave equation. The solution is of the form
\begin{align*}
	\bar{h}_{\mu\nu} = A_{\mu\nu}e^{-ik^{\sigma}x_{\sigma}}.
\end{align*}
This is an eigenfunction of the d'Alembertian with eigenvalue $k^{\mu}k_{\mu}$, hence the wavevector must be light-like and waves in the metric travel at the speed of light.

With the chosen gauge for the coordinates we still have some freedom - namely, the coordinate transform $x^{\mu} \to x^{\mu} + B^{\mu}e^{-k^{\sigma}x_{\sigma}}$ gives the same gauge. We may also choose $B^{\mu}$ such that $\tensor{A}{^{\mu}_{\mu}} = 0$ and $A_{0\mu} = 0$ in some frame. Furthermore, in this gauge we have $\del{\mu}{\bar{h}^{\mu\nu}} = k_{\mu}A^{\mu\nu} = 0$. For a wave propagating in the third Cartesian direction the only non-zero amplitudes are $A^{11} = -A^{22} = A_{+}$ and $A^{12} = A^{21} = A_{\cross}$.

Gravitational waves propagate between geodesics, so we need to consider geodesic deviation. To do this, consider two events close to each other on a simultaneity. Evolution in proper time will then take the events to a new simultaneity. Let $X$ be the vector field that is tangent to the simultaneity and pointing in the direction of the other event for each $\tau$, and $U$ the tangent of the geodesic. We then have:
\begin{align*}
	\ddot{X} = \dcov{U}{\dcov{U}{X}} = R(U, X)U,
\end{align*}
where we have chosen the torsion such that $\comm{X}{U} = 0$. In component form:
\begin{align*}
	\ddot{X}^{\mu} = \tensor*{R}{^{\mu}_{\lambda\sigma\nu}}U^{\lambda}U^{\sigma}X^{\nu}.
\end{align*}
For a slowly moving test particle we obtain
\begin{align*}
	\ddot{X}^{\mu} = \tensor*{R}{^{\mu}_{00\nu}}X^{\nu}.
\end{align*}
As we have $\tensor*{R}{_{\mu 00\nu}} = \frac{1}{2}\del[2]{0}{h_{\mu\nu}}$, the solution is
\begin{align*}
	X^{1} = X^{1}(0)\left(1 + \frac{1}{2}A_{+}e^{-ik^{\sigma}x_{\sigma}}\right),\ X^{2} = X^{2}(0)\left(1 - \frac{1}{2}A_{+}e^{-ik^{\sigma}x_{\sigma}}\right).
\end{align*}
Hence spatial distances oscillate due to the oscillating metric. I think.

\paragraph{Gravitational Lensing}
Consider an object and an observer separated by a distribution of matter producing a gravitational potential $\Phi$. In the weak-field limit the metric is described by
\begin{align*}
	\dd{s}^{2} = (1 + 2\Phi)\dd{t}^{2} - (1 - 2\Phi)\dd{\vb{x}}^{2},
\end{align*}
and the path of a ray of light travelling between the two is described by
\begin{align*}
	(1 + 2\Phi)\dot{t}^{2} - (1 - 2\Phi)\dot{\vb{x}}^{2} = 0.
\end{align*}
In the weak-field limit we expect $\Phi \ll 1$, and thus $\dot{t}^{2} \approx \dot{\vb{x}}^{2}$.

In the case where the potential is time-independent, the geodesic equation for the time coordinate yields
\begin{align*}
	(1 + 2\Phi)\dot{t} = c,
\end{align*}
and we are free to choose $c = 1$ by rescaling the parameter. Next, the equation for the spatial coordinates is
\begin{align*}
	-\ddot{\vb{x}} + 2(\dot{\vb{x}}\cdot\grad{\Phi})\dot{\vb{x}} = (\dot{t}^{2} + \dot{\vb{x}}^{2})\grad{\Phi}.
\end{align*}

Choosing a coordinate system such that $\dot{\vb{x}} = \vb{e}_{x}$ initially. In the weak-field we expect
\begin{align*}
	\dot{\vb{x}} = (1 + \dots)\vb{e}_{x} + v^{2}\vb{e}_{y} + v^{3}\vb{e}_{z},
\end{align*}
where the new velocity components are small and the dots represent perturbations of the path of higher order than the other velocity components. Inserting this into the geodesic equation yields
\begin{align*}
	-\dot{v}^{2} = 2(\grad{\Phi})^{2},
\end{align*}
where we refer to vector components on either side. We can solve this differential equation by integrating along the unperturbed path. In particular, by moving both the observer and the object to infinity we find
\begin{align*}
	\dot{v}^{2} = -\integ{-\infty}{\infty}{t}{2(\grad{\Phi})^{2}}.
\end{align*}

\paragraph{Experimental Evidence for General Relativity}
The first example evidence validating general relativity is the perihelion precession of Mercury. Namely, the elliptic axis of the orbit of Mercury precedes with a very long period. Much of this could be attributed to many-body effects, but the remaining contribution matched very closely with the prediction from general relativity.

Next there is strong gravitational lensing, exemplified in an observation of a solar eclipse in 1919. Due to the gravitation from the Sun, a star was visible on Earth despite being behind the Sun.

Then there is so-called Shapiro delay - namely, it takes more time for light signals to travel due to gravitation.

Furthermore there is the Pound-Rebka experiments, which demonstrates gravitational time dilation.

Finally among the major examples discussed is the presence of gravitational effects in binary systems. As they emit gravitational systems they lose energy, causing their orbits to slow. When two stars merge, relatively strong waves are formed. The observation of gravitational waves from such a phenomenon was recently discovered and awarded the Nobel prize.

In addition there are gyroscopic effects and effects due to strong gravitational lensing.