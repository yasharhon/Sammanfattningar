\section{Basic Concepts}

\paragraph{A Note on Minkowski Space}
In special relativity we work with Minkowski space, which is an affine space with a so-called pseudo-metric. This is a metric which is not positive definite, but instead a metric which has only non-zero eigenvalues (and is thus termed non-degenerate). We will work with the signature $(1, 3)$, meaning that there are three eigenvalues of $-1$ and one eigenvalue $1$.

\paragraph{Problems With Gravitation}
Newton's law of gravitation states that
\begin{align*}
	\vb{a} = \frac{1}{m_{\text{I}}}\vb{f}_{\text{G}} = -\frac{m_{\text{G}}}{m_{\text{I}}}\grad{\Phi},\ \Phi = 4\pi G\rho.
\end{align*}
There are a few possible problems and peculiarities with this, namely:
\begin{itemize}
	\item It has no explicit time dependence, and can therefore not hold by itself in special relativity.
	\item As the mass density does not transform as a Lorentz scalar, we cannot generalize gravitation to special relativity like we did with electromagnetism.
	\item The ratio between gravitational and inertial mass is the same for all particles. This makes gravitation stand out as a fundamental force.
\end{itemize}

\paragraph{The Equivalence Principle}
The equivalence principles states that in a freely-falling, non-rotating- spatially small laboratory, the laws of physics are those of special relativity.

\paragraph{An Extended Description of Spacetime}
Einstein proposed a solution to the issue of gravitation that also incorporates the equivalence principle.

Einstein's idea was to extend the special theory of relativity to a curved spacetime and propose that this spacetime is bent by matter. This would mean that free particles move along geodesics in this spacetime, explaining the equality of gravitational and inertial mass. For the equivalence principle to hold, i.e. for it to be possible to locally construct a Minkowski spacetime, spacetime must be a pseudo-Riemannian manifold.

This transition is in fact not as big a leap as one might expect. As we have seen in special relativity, simultaneity is no longer guaranteed by the physics, so surfaces of constant time could at least be slanted. As the introduction of spacetime made it possible in principle to introduce curvilinear coordinates on time as well, we have yet to add anything. The only addition in GR is the extension of spacetime from an affine space to a manifold.

In conclusion, we will describe spacetime as a $4$-dimensional manifold immersed in $5$-dimensional spacetime with a pseudometric of signature $(1, 3)$ and the Levi-Civita connection imposed on it.

\paragraph{Static and Stationary Spacetime}
If there exists a time-like Killing field of a spacetime, it is stationary. If the spacetime is also orthogonal to a family of $3$-surfaces, the spacetime is static. The consequences of the latter is that the metric is block diagonal, with one time block and one space block.

\paragraph{Comoving Observers}
A comoving observer is one that has fixed spatial coordinates.

\paragraph{Kinematics of Test Particles}
A test particle is a particle that itself does not affect the spacetime. Such particles can generally move through spacetime, along curves called world lines. With this motion comes the $4$-velocity $V$, defined as the normalized tangent to a world line. In special relativity we could also define a proper acceleration by differentiating with respect to proper time. In general relativity we replace this with the $4$-acceleration $A = \dcov{V}{V} = \dcov{\dot{\gamma}}{\dot{\gamma}}$.  We may also define the proper acceleration $\alpha$, which satisfies $\alpha^{2} = -A^{2} = -g(A, A)$. As $g(V, V) = 1$, we have
\begin{align*}
	\dcov{V}{g(V, V)} = 2g(V, A) = 0,
\end{align*}
which implies that $A$ is space-like. Note that the curve parameter we use is $\tau$, which is the proper time and a measure of length in spacetime.

\paragraph{Free Particles}
A free particle moves along geodesics, meaning $A = 0$. The geodesic equations could in principle be obtained both from a variational principle described by a Lagrangian $\lag = g_{\mu\nu}\dot{\chi}^{\mu}\dot{\chi}^{\nu}$ or by computing Christoffel symbols, of which both approaches will be used.

\paragraph{$4$-Momentum and $4$-Force}
We also define the $4$-momentum $P = mV$ and the $4$-force $F = \dcov{V}{P}$.

\paragraph{Simultaneity and Distance}
Two events are simultaneous if they are on the same hypersurface of constant $t$. As this depends very much on the choice of coordinates on spacetime, this notion is not at all well-defined.

Similarly, distances are defined along simultaneities and are equally ill-defined. It turns out that the only sense of distance that all observers can agree on is proper time.

\paragraph{The Einstein Field Equations}
The equations describing the metric are the Einstein field equations. We will derive them from a variational principle, starting with the field equations for vacuum. The action from which we will obtain the field equations is the Einstein-Hilbert action
\begin{align*}
	S_{\text{EH}} = -\frac{M_{\alpha}^{2}}{2}\integ[4]{}{}{\chi}{\R\sqrt{-\det(g)}}.
\end{align*}

To derive a set of equations describing the extremum of this action, we will need to differentiate the two factors. We choose to do so with respect to the dual metric components. Considering the metric determinant first, we have
\begin{align*}
	\pdv{g^{\mu\nu}}(g^{\rho\sigma}g_{\rho\sigma}) = \pdv{g^{\rho\sigma}}{g^{\mu\nu}}g_{\rho\sigma} + g^{\rho\sigma}\pdv{g^{\rho\sigma}}{g_{\mu\nu}} = 0.
\end{align*}
Next, we study the quantity $\tr(\ln(g))$. We have
\begin{align*}
	\pdv{\tr(\ln(g))}{g^{\mu\nu}} = \tr(\pdv{\ln(g)}{g^{\mu\nu}}) = \tr(g^{-1}\pdv{g}{g^{\mu\nu}}).
\end{align*}
Using the identity $\tr(\ln(g)) = \ln(\det(g))$, we have
\begin{align*}
	\tr(g^{-1}\pdv{g}{g^{\mu\nu}}) = \frac{1}{\det(g)}\pdv{\det(g)}{g^{\mu\nu}},
\end{align*}
hence
\begin{align*}
	\pdv{\det(g)}{g^{\mu\nu}} = \det(g)g^{\rho\sigma}\pdv{g_{\rho\sigma}}{g^{\mu\nu}} = -\det(g)g_{\rho\sigma}\pdv{g^{\rho\sigma}}{g^{\mu\nu}}
\end{align*}
and
\begin{align*}
	\pdv{\sqrt{-\det(g)}}{g^{\mu\nu}} = \frac{1}{2\sqrt{-\det(g)}}\det(g)g^{\rho\sigma}\pdv{g^{\rho\sigma}}{g^{\mu\nu}} = -\frac{1}{2}\sqrt{-\det(g)}g_{\rho\sigma}\pdv{g^{\rho\sigma}}{g^{\mu\nu}}.
\end{align*}

For the Ricci scalar we have $\R = g^{\mu\nu}R_{\mu\nu}$, hence
\begin{align*}
	\pdv{\R}{g^{\mu\nu}} = R_{\mu\nu} + g^{\rho\sigma}\pdv{R_{\rho\sigma}}{g^{\mu\nu}}.
\end{align*}
To differentiate the Ricci tensor, we first use the fact that differences in connection coefficients, and thus their derivatives, transform as tensors. Thus
\begin{align*}
	\dcov{\alpha}{\pdv{\chris{\beta}{\gamma}{\delta}}{g^{\mu\nu}}} = \del{\alpha}{\pdv{\chris{\beta}{\gamma}{\delta}}{g^{\mu\nu}}} + \chris{\beta}{\alpha}{\rho}\pdv{\chris{\rho}{\gamma}{\delta}}{g^{\mu\nu}} - \chris{\rho}{\alpha}{\gamma}\pdv{\chris{\beta}{\rho}{\delta}}{g^{\mu\nu}} - \chris{\rho}{\alpha}{\delta}\pdv{\chris{\beta}{\gamma}{\rho}}{g^{\mu\nu}}.
\end{align*}
Now we have
\begin{align*}
	\pdv{R_{\rho\sigma}}{g^{\mu\nu}} &= \pdv{R^{\alpha}_{\rho\alpha\sigma}}{g^{\mu\nu}} \\
	                                 &= \pdv{g^{\mu\nu}}\left(\del{\alpha}{\chris{\alpha}{\sigma}{\rho}} - \del{\sigma}{\chris{\alpha}{\alpha}{\rho}} + \chris{\gamma}{\sigma}{\rho}\chris{\alpha}{\alpha}{\gamma} - \chris{\gamma}{\alpha}{\rho}\chris{\alpha}{\sigma}{\gamma}\right) \\
	                                 &= \pdv{\del{\alpha}{\chris{\alpha}{\sigma}{\rho}}}{g^{\mu\nu}} - \pdv{\del{\sigma}{\chris{\alpha}{\alpha}{\rho}}}{g^{\mu\nu}} + \chris{\gamma}{\sigma}{\rho}\pdv{\chris{\alpha}{\alpha}{\gamma}}{g^{\mu\nu}} + \chris{\alpha}{\alpha}{\gamma}\pdv{\chris{\gamma}{\sigma}{\rho}}{g^{\mu\nu}} - \chris{\gamma}{\alpha}{\rho}\pdv{\chris{\alpha}{\sigma}{\gamma}}{g^{\mu\nu}} - \chris{\alpha}{\sigma}{\gamma}\pdv{\chris{\gamma}{\alpha}{\rho}}{g^{\mu\nu}} \\
	                                 &= \pdv{\del{\alpha}{\chris{\alpha}{\sigma}{\rho}}}{g^{\mu\nu}} + \chris{\alpha}{\alpha}{\gamma}\pdv{\chris{\gamma}{\sigma}{\rho}}{g^{\mu\nu}} - \chris{\gamma}{\alpha}{\rho}\pdv{\chris{\alpha}{\gamma}{\sigma}}{g^{\mu\nu}} - \chris{\gamma}{\alpha}{\sigma}\pdv{\chris{\alpha}{\rho}{\gamma}}{g^{\mu\nu}} - \dcov{\sigma}{\chris{\alpha}{\alpha}{\rho}} \\
	                                 &= \dcov{\alpha}{\chris{\alpha}{\sigma}{\rho}} - \dcov{\sigma}{\chris{\alpha}{\alpha}{\rho}}.
\end{align*}
As the Levi-Civita connection is metric compatible, we have
\begin{align*}
	\pdv{\R}{g^{\mu\nu}} = R_{\mu\nu} + \dcov{\alpha}{(g^{\rho\sigma}\chris{\alpha}{\sigma}{\rho})} - \dcov{\sigma}{(g^{\rho\sigma}\chris{\alpha}{\alpha}{\rho})}.
\end{align*}
The latter terms may be ignored by converting it to a contribution at the boundary.

There are now two ways to proceed. The first is to note that the Lagrangian density has no functional dependence on the derivatives of the metric, implying that its derivatives with respect to the metric are zero, implying
\begin{align*}
	\pdv{g^{\mu\nu}}(\R\sqrt{-\det(g)}) = R_{\mu\nu}\sqrt{-\det(g)} - \frac{1}{2}\R\sqrt{-\det(g)}g_{\rho\sigma}\pdv{g^{\rho\sigma}}{g^{\mu\nu}} = 0,
\end{align*}
yielding the Einstein field equations
\begin{align*}
	R_{\mu\nu} - \frac{1}{2}\R g_{\mu\nu} = G_{\mu\nu} = 0.
\end{align*}
The other is to consider variations of the metric and their impact on the action. This will yield
\begin{align*}
	\var{S} = -\frac{M_{\alpha}^{2}}{2}\integ[4]{}{}{\chi}{\pdv{\R\sqrt{-\det(g)}}{g^{\mu\nu}}\var{g^{\mu\nu}}} = -\frac{M_{\alpha}^{2}}{2}\integ[4]{}{}{\chi}{\sqrt{-\det(g)}\left(R_{\mu\nu} - \frac{1}{2}\R g_{\mu\nu}\right)\var{g^{\mu\nu}}},
\end{align*}
and extremization yields the same equations.

\paragraph{The Energy-Momentum Tensor}
To obtain the field equations for other cases we will add terms to the action. In general, by adding a term $S_{\text{matter}}$ to the action with the coefficient $\frac{M_{\alpha}^{2}}{2}$ baked into it, we obtain
\begin{align*}
	R_{\mu\nu} - \frac{1}{2}g_{\mu\nu}\R = 8\pi GT_{\mu\nu},
\end{align*}
where
\begin{align*}
	T_{\mu\nu} = \frac{2}{\sqrt{-\det(g)}}\fdv{S_{\text{matter}}}{g^{\mu\nu}}
\end{align*}
is called the energy-momentum tensor.

\paragraph{Alternative Field Equations}
Starting from the Einstein field equation, we may contract with the metric according to
\begin{align*}
	g^{\mu\nu}R_{\nu\sigma} - \frac{1}{2}\kdelta{\sigma}{\mu}\R = 8\pi Gg^{\mu\nu}T_{\nu\sigma} = 8\pi G\tensor{T}{^{\mu}_{\sigma}},
\end{align*}
and setting $\mu = \sigma$ and summing yields
\begin{align*}
	8\pi G\tensor{T}{^{\mu}_{\mu}} = g^{\mu\nu}R_{\nu\mu} - 2\R = -\R,
\end{align*}
which we may insert back into the Einstein field equations to obtain
\begin{align*}
	R_{\mu\nu} = 8\pi G\left(T_{\mu\nu} - \frac{1}{2}g_{\mu\nu}\tensor{T}{^{\sigma}_{\sigma}}\right).
\end{align*}

\paragraph{Example Calculations of the Energy-Momentum Tensor}
The definition of the energy-momentum tensor is by itself not very elucidating. I therefore show two examples, both to illustrate the results one could obtain and how to do the computation in practice.

The first example is for an electromagnetic field in vacuum, defined by the action and the Lagrangian
\begin{align*}
	S = \integ[4]{}{}{\chi}{\sqrt{-\det(g)}\lag},\ \lag = -\frac{1}{4}F_{\mu\nu}F^{\mu\nu} = -\frac{1}{4}g^{\mu\rho}g^{\nu\sigma}F_{\mu\nu}F_{\rho\sigma}.
\end{align*}
Because electromagnetic theory is a field theory, writing the Lagrangian in a proper way is easy, and as the Lagrangian does not depend on derivatives of the metric we have
\begin{align*}
	\fdv{S}{g^{\mu\nu}} &= \pdv{\lag}{g^{\mu\nu}} = -\frac{1}{2}\sqrt{-\det(g)}g_{\mu\nu}\cdot -\frac{1}{4}g^{\alpha\sigma}g^{\beta\sigma}F_{\alpha\beta}F_{\rho\sigma} + \sqrt{-\det(g)}\cdot -\frac{1}{4}F_{\alpha\beta}F_{\rho\sigma}(\kdelta{\alpha}{\mu}\kdelta{\rho}{\nu}g^{\beta\sigma} + g^{\alpha\rho}\kdelta{\beta}{\mu}\kdelta{\sigma}{\nu}) \\
	                    &= \frac{1}{4}\sqrt{-\det(g)}\left(\frac{1}{2}g_{\mu\nu}g^{\alpha\rho}g^{\beta\sigma}F_{\alpha\beta}F_{\rho\sigma} - (F_{\mu\beta}F_{\nu\sigma}g^{\beta\sigma} + F_{\alpha\mu}F_{\rho\nu}g^{\alpha\rho})\right).
\end{align*}
As the Faraday tensor is antisymmetric, we may rewrite this as
\begin{align*}
	\fdv{S}{g^{\mu\nu}} &= \frac{1}{4}\sqrt{-\det(g)}\left(\frac{1}{2}g_{\mu\nu}g^{\alpha\rho}g^{\beta\sigma}F_{\alpha\beta}F_{\rho\sigma} - (F_{\mu\beta}F_{\nu\sigma}g^{\beta\sigma} + F_{\mu\alpha}F_{\nu\rho}g^{\alpha\rho})\right) \\
	                    &= \frac{1}{4}\sqrt{-\det(g)}\left(\frac{1}{2}g_{\mu\nu}g^{\alpha\rho}g^{\beta\sigma}F_{\alpha\beta}F_{\rho\sigma} - 2F_{\mu\beta}F_{\nu\sigma}g^{\beta\sigma}\right) \\
	                    &= \frac{1}{4}\sqrt{-\det(g)}\left(\frac{1}{2}g_{\mu\nu}F^{\rho\sigma}F_{\rho\sigma} - 2F_{\beta\mu}\tensor{F}{^{\beta}_{\nu}}\right),
\end{align*}
and arrive at the final expression
\begin{align*}
	T_{\mu\nu} = \frac{1}{4}g_{\mu\nu}F^{\rho\sigma}F_{\rho\sigma} - F_{\beta\mu}\tensor{F}{^{\beta}_{\nu}}.
\end{align*}
This expression coincides exactly with one derived from other principles in the context of special relativity.

The next example is for a massive free particle, defined by the action and the Lagrangian
\begin{align*}
	S = \integ{}{}{\tau}{\lag},\ \lag = m\sqrt{g_{\mu\nu}\dot{\chi}^{\mu}\dot{\chi}^{\nu}}.
\end{align*}
An obvious difficulty with this case is that the theory describing the motion of free particles is not a field theory, so we will have to write it as one. We do that by performing the rewrite
\begin{align*}
	S = \integ{}{}{\tau}{\integ[4]{}{}{\eta}{m\delta^{4}(\eta - \chi(\tau))\sqrt{g_{\mu\nu}\dot{\eta}^{\mu}\dot{\eta}^{\nu}}}},
\end{align*}
where we have introduced the generalized Dirac delta such that the integral over all spacetime of $\delta^{4}(\eta)$ is $1$. To proceed we will need to integrate out the affine parameter $\tau$. We can perform this elimination by integrating out $\chi^{0}$ under the assumption that this particular coordinate is also usable as a parameter for the world line. This will yield
\begin{align*}
	S = \integ[4]{}{}{\eta}{m\delta^{3}(\eta - \chi(\tau))\frac{1}{\dot{\chi^{0}}}\sqrt{g_{\mu\nu}\dot{\eta}^{\mu}\dot{\eta}^{\nu}}},
\end{align*}
where $\tau$ may in this expression be taken as the inverse function of $\chi^{0}$. Having obtained a proper kind of action, we may now compute
\begin{align*}
	\fdv{S}{g_{\mu\nu}} &= \frac{m\delta^{3}(\eta - \chi(\tau))\dot{\eta}^{\mu}\dot{\eta}^{\nu}}{2\dot{\chi^{0}}\sqrt{g_{\mu\nu}\dot{\eta}^{\mu}\dot{\eta}^{\nu}}},
\end{align*}
and
\begin{align*}
	T^{\mu\nu} = \frac{m\delta^{3}(\eta - \chi(\tau))\dot{\eta}^{\mu}\dot{\eta}^{\nu}}{\dot{\chi^{0}}\sqrt{-\det(g)g_{\mu\nu}\dot{\eta}^{\mu}\dot{\eta}^{\nu}}}.
\end{align*}
In particular, choosing proper time as the affine parameter we have
\begin{align*}
	T^{\mu\nu} = \frac{m\delta^{3}(\eta - \chi(\tau))\dot{\eta}^{\mu}\dot{\eta}^{\nu}}{\dot{\chi^{0}}\sqrt{-\det(g)}}.
\end{align*}

To make it even simpler to study, we choose Cartesian coordinates, for which we find
\begin{align*}
	T^{00} = m\gamma\delta^{3}(\eta - \chi(\tau)) = E\delta^{3}(\eta - \chi(\tau)),\ T^{0i} = m\delta^{3}(\eta - \chi(\tau))\dot{\eta}^{i} = p^{i}\delta^{3}(\eta - \chi(\tau)),
\end{align*}
in other words describing a system with the exact energy and momentum of the free particle located at the position of the particle.

\paragraph{Ideal Fluids}
An ideal fluid is a substance such that its energy-momentum tensor is of the form
\begin{align*}
	T^{\mu\nu} = (\rho_{0} + p)U^{\mu}U^{\nu} - pg^{\mu\nu},
\end{align*}
where $u^{\mu}$ is the $4$-velocity of the rest frame of the fluid and $\rho_{0}$ and $p$ are the energy density and pressure of the fluid as measured in the rest frame.

\paragraph{The Weak Field Limit}
We will now study the Einstein field equations in a limit where the effects of general relativity are weak, in the hopes of finding some limit that reproduces Newtonian gravity. We will do this by linearizing the metric according to $g_{\mu\nu} = \eta_{\mu\nu} + h_{\mu\nu}$, where $\eta$ is the Minkowski metric and all components of $h$ are small. To raise its components, we use the fact that the metric should produce the Kronecker delta to obtain
\begin{align*}
	\kdelta{\mu}{\sigma} = g^{\mu\nu}g_{\nu\sigma} = (\eta^{\mu\nu} + f^{\mu\nu})(\eta_{\nu\sigma} + h_{\nu\sigma}) = \eta^{\mu\nu}\eta_{\nu\sigma} + \eta^{\mu\nu}h_{\nu\sigma} + f^{\mu\nu}\eta_{\nu\sigma},
\end{align*}
where the perturbations of the inverse metric must also necessarily be small, allowing us to neglect higher-order terms. This yields
\begin{align*}
	&\eta^{\mu\nu}h_{\nu\sigma} + f^{\mu\nu}\eta_{\nu\sigma} = 0, \\
	&f^{\mu\nu} = -\eta^{\sigma\nu}\eta^{\mu\rho}h_{\rho\sigma}.
\end{align*}

When computing the curvature tensor, we will need the Christoffel symbols, which are given by
\begin{align*}
	\chris{\lambda}{\mu}{\nu} &= \frac{1}{2}(\eta^{\lambda\rho} - \eta^{\sigma\rho}\eta^{\lambda\gamma}h_{\gamma\sigma})(\del{\nu}{h_{\mu\rho}} + \del{\mu}{h_{\rho\nu}} - \del{\rho}{h_{\mu\nu}}) \\
	                          &= \frac{1}{2}(\eta^{\lambda\rho}\del{\nu}{h_{\mu\rho}} + \eta^{\lambda\rho}\del{\mu}{h_{\rho\nu}} - \eta^{\lambda\rho}\del{\rho}{h_{\mu\nu}}).
\end{align*}
We will also need products of the Christoffel symbols. As the symbols themselves are linear in the magnitude of the perturbations, we may however neglect these terms, yielding
\begin{align*}
	\tensor{R}{^{\mu}_{\nu\lambda\sigma}} = \del{\lambda}{\chris{\mu}{\sigma}{\nu}} - \del{\sigma}{\chris{\mu}{\lambda}{\nu}}
\end{align*}
and
\begin{align*}
	R_{\nu\sigma} &= \del{\mu}{\chris{\mu}{\sigma}{\nu}} - \del{\sigma}{\chris{\mu}{\mu}{\nu}} \\
	              &= \frac{1}{2}\left(\del{\mu}{(\eta^{\mu\rho}\del{\nu}{h_{\sigma\rho}} + \eta^{\mu\rho}\del{\sigma}{h_{\rho\nu}} - \eta^{\mu\rho}\del{\rho}{h_{\sigma\nu}})} - \del{\sigma}{(\eta^{\mu\rho}\del{\nu}{h_{\mu\rho}} + \eta^{\mu\rho}\del{\mu}{h_{\rho\nu}} - \eta^{\mu\rho}\del{\rho}{h_{\mu\nu}})}\right) \\
	              &= \frac{1}{2}\left(\del{\mu}{(\del{\nu}{\tensor{h}{_{\sigma}^{\mu}}} + \del{\sigma}{\tensor{h}{^{\mu}_{\nu}}} - \eta^{\mu\rho}\del{\rho}{h_{\sigma\nu}})} - \del{\sigma}{(\del{\nu}{\tensor{h}{^{\mu}_{\mu}}} + \del{\mu}{\tensor{h}{^{\mu}_{\nu}}} - \eta^{\mu\rho}\del{\rho}{h_{\mu\nu}})}\right) \\
	              &= \frac{1}{2}\left(\del{\mu}{\del{\nu}{\tensor{h}{_{\sigma}^{\mu}}}} - \del{\mu}{\eta^{\mu\rho}\del{\rho}{h_{\sigma\nu}}} - \del{\sigma}{\del{\nu}{\tensor{h}{^{\mu}_{\mu}}}} + \del{\sigma}{\eta^{\mu\rho}\del{\rho}{h_{\mu\nu}}}\right) \\
	              &= \frac{1}{2}\left(\del{\mu}{\del{\nu}{\tensor{h}{^{\mu}_{\sigma}}}} - \del{\mu}{\del[\mu]{}{h_{\sigma\nu}}} - \del{\sigma}{\del{\nu}{h}} + \del{\sigma}{\del[\mu]{}{h_{\mu\nu}}}\right) \\
	              &= \frac{1}{2}(\del{\nu}{\del{\mu}{\tensor{h}{^{\mu}_{\sigma}}}} - \square h_{\nu\sigma} - \del{\nu}{\del{\sigma}{h}} + \del{\sigma}{\del{\mu}{\tensor{h}{^{\mu}_{\nu}}}}),
\end{align*}
where we have introduced the trace of the perturbation $h = \tensor{h}{^{\mu}_{\mu}}$. We can rewrite this further by introducing the d'Alembertian $\square = \dcov[\mu]{}{\dcov{\mu}{}}$. When applied to the perturbations of the metric, the only terms that are linear in its magnitude are the derivative terms, hence $\square h_{\mu\nu} = \del[\sigma]{}{\del{\sigma}{h_{\mu\nu}}}$, and
\begin{align*}
	R_{\nu\sigma} &= \frac{1}{2}(\del{\nu}{\del{\mu}{\tensor{h}{^{\mu}_{\sigma}}}} - \square h_{\nu\sigma} - \del{\nu}{\del{\sigma}{h}} + \del{\sigma}{\del{\mu}{\tensor{h}{^{\mu}_{\nu}}}}).
\end{align*}
Next, the Ricci scalar is
\begin{align*}
	\R &= \frac{1}{2}(\eta^{\nu\sigma} -\eta^{\sigma\nu}\eta^{\mu\rho}h_{\rho\sigma})(\del{\nu}{\del{\mu}{\tensor{h}{^{\mu}_{\sigma}}}} - \square h_{\nu\sigma} - \del{\nu}{\del{\sigma}{h}} + \del{\sigma}{\del{\mu}{\tensor{h}{^{\mu}_{\nu}}}}) \\
	   &= \frac{1}{2}\eta^{\nu\sigma}(\del{\nu}{\del{\mu}{\tensor{h}{^{\mu}_{\sigma}}}} - \square h_{\nu\sigma} - \del{\nu}{\del{\sigma}{h}} + \del{\sigma}{\del{\mu}{\tensor{h}{^{\mu}_{\nu}}}}) \\
	   &= \frac{1}{2}(\del[\sigma]{}{\del{\mu}{\tensor{h}{^{\mu}_{\sigma}}}} - \square h - \square h + \del[\nu]{}{\del{\mu}{\tensor{h}{^{\mu}_{\nu}}}}) \\
	   &= \del[\sigma]{}{\del{\mu}{\tensor{h}{^{\mu}_{\sigma}}}} - \square h \\
	   &= \del{\mu}{\del{\sigma}{\tensor{h}{^{\mu\sigma}}}} - \square h,
\end{align*}
and the Einstein tensor is
\begin{align*}
	G_{\nu\sigma} &= R_{\nu\sigma} - \frac{1}{2}g_{\nu\sigma}\R \\
	              &= \frac{1}{2}(\del{\nu}{\del{\mu}{\tensor{h}{^{\mu}_{\sigma}}}} - \square h_{\nu\sigma} - \del{\nu}{\del{\sigma}{h}} + \del{\sigma}{\del{\mu}{\tensor{h}{^{\mu}_{\nu}}}}) - \frac{1}{2}(\eta_{\nu\sigma} + h_{\nu\sigma})(\del{\mu}{\del{\lambda}{\tensor{h}{^{\mu\lambda}}}} - \square h) \\
	              &= \frac{1}{2}(\del{\nu}{\del{\mu}{\tensor{h}{^{\mu}_{\sigma}}}} - \square h_{\nu\sigma} - \del{\nu}{\del{\sigma}{h}} + \del{\sigma}{\del{\mu}{\tensor{h}{^{\mu}_{\nu}}}} - \eta_{\nu\sigma}(\del{\mu}{\del{\lambda}{\tensor{h}{^{\mu\lambda}}}} - \square h)) \\
	              &= \frac{1}{2}(\del{\nu}{\del{\mu}{\tensor{h}{^{\mu}_{\sigma}}}} - \square h_{\nu\sigma} - \del{\nu}{\del{\sigma}{h}} + \del{\sigma}{\del{\mu}{\tensor{h}{^{\mu}_{\nu}}}} - \eta_{\nu\sigma}\del{\mu}{\del{\lambda}{\tensor{h}{^{\mu\lambda}}}} + \eta_{\nu\sigma}\square h).
\end{align*}

To proceed, we choose coordinates such that $\square\chi^{\mu} = 0$. In these coordinates we have
\begin{align*}
	\square \chi^{\mu} &= \dcov[\nu]{}{\dcov{\nu}{\chi^{\mu}}} \\
	                   &= \dcov[\nu]{}{(\del{\nu}{\chi^{\mu}})} \\
	                   &= \del[\nu]{}{(\del{\nu}{\chi^{\mu}})} - g^{\sigma\nu}\chris{\gamma}{\nu}{\sigma}\del{\gamma}{\chi^{\mu}} \\
	                   &= -g^{\nu\sigma}\chris{\mu}{\nu}{\sigma} = 0,
\end{align*}
and using the previously obtained Christoffel symbols we find
\begin{align*}
	\frac{1}{2}\eta^{\nu\sigma}(\eta^{\mu\rho}\del{\sigma}{h_{\nu\rho}} + \eta^{\mu\rho}\del{\nu}{h_{\rho\sigma}} - \eta^{\mu\rho}\del{\rho}{h_{\nu\sigma}}) &= \frac{1}{2}(\del[\nu]{}{\tensor{h}{_{\nu}^{\mu}}} + \del{\nu}{\tensor{h}{^{\mu\nu}}} - \del[\mu]{}{h}) = 0,
\end{align*}
or
\begin{align*}
	\del[\mu]{}{h} = \del[\nu]{}{\tensor{h}{_{\nu}^{\mu}}} + \del{\nu}{\tensor{h}{^{\mu\nu}}} = 2\del{\nu}{h^{\nu\mu}}.
\end{align*}
Lowering the free index yields
\begin{align*}
	\del{\nu}{\tensor{h}{^{\nu}_{\mu}}} = \frac{1}{2}\del{\mu}{h}.
\end{align*}

The Einstein tensor is now
\begin{align*}
	G_{\nu\sigma} &= \frac{1}{2}(\del{\nu}{\del{\mu}{\tensor{h}{^{\mu}_{\sigma}}}} - \square h_{\nu\sigma} - \del{\nu}{\del{\sigma}{h}} + \del{\sigma}{\del{\mu}{\tensor{h}{^{\mu}_{\nu}}}} - \eta_{\nu\sigma}\del{\mu}{\del{\lambda}{\tensor{h}{^{\mu\lambda}}}} + \eta_{\nu\sigma}\square h) \\
	              &= \frac{1}{2}\left(\frac{1}{2}\del{\nu}{\del{\sigma}{h}} - \square h_{\nu\sigma} - \del{\nu}{\del{\sigma}{h}} + \frac{1}{2}\del{\sigma}{\del{\nu}{h}} - \eta_{\nu\sigma}\del[\lambda]{}{\del{\mu}{\tensor{h}{^{\mu}_{\lambda}}}} + \eta_{\nu\sigma}\square h\right) \\
	              &= \frac{1}{2}\left(-\square h_{\nu\sigma} - \frac{1}{2}\eta_{\nu\sigma}\del[\lambda]{}{\del{\lambda}{h}} + \eta_{\nu\sigma}\square h\right) \\
	              &= \frac{1}{2}\left(-\square h_{\nu\sigma} + \frac{1}{2}\eta_{\nu\sigma}\square h\right) \\
	              &= -\frac{1}{2}\left(\square h_{\nu\sigma} - \frac{1}{2}\eta_{\nu\sigma}\square h\right).
\end{align*}
The Einstein field equations are thus
\begin{align*}
	G_{\nu\sigma} = -\frac{1}{2}\left(\square h_{\nu\sigma} - \frac{1}{2}\eta_{\nu\sigma}\square h\right) = 8\pi GT_{\nu\sigma},
\end{align*}
or alternatively, by defining $\bar{h}_{\nu\sigma} = h_{\nu\sigma} - \frac{1}{2}\eta_{\nu\sigma}h$, the compatibility of the metric implies
\begin{align*}
	\square\bar{h}_{\mu\nu} = -16\pi GT_{\mu\nu}.
\end{align*}

To proceed we say a little more about the Newtonian limit. It should be found at low speeds, meaning that the energy-momentum tensor will be dominated by $T_{00}$, which is the energy density, or equivalently the mass density. The corresponding component of $\bar{h}$ will thus also dominate the other side. It should also be obtained when the involved masses are small. If the situation also evolves slowly, we may neglect time derivatives to obtain
\begin{align*}
	\laplacian{\bar{h}_{00}} = 16\pi G\rho.
\end{align*}
We may thus identify $\bar{h}_{00} = 4\Phi$, where $\Phi$ is the Newtonian gravitational potential. Next, the trace of $\bar{h}$ is
\begin{align*}
	\bar{h} = h - \frac{1}{2}\kdelta{\mu}{\mu}h = -h,
\end{align*}
which according to the above should be dominated by the time component. Thus
\begin{align*}
	h_{\nu\sigma} = \bar{h}_{\nu\sigma} + \frac{1}{2}\eta_{\nu\sigma}h = \bar{h}_{\nu\sigma} - \frac{1}{2}\eta_{\nu\sigma}\bar{h} = 4\Phi\left(\kdelta{0}{\mu}\kdelta{0}{\nu} - \frac{1}{2}\eta_{\mu\nu}\right)
\end{align*}
and
\begin{align*}
	h_{00} = 2\phi,\ h_{ij} = 2\Phi.
\end{align*}
The corresponding spacetime length is
\begin{align*}
	\dd{s}^{2} = (1 + 2\Phi)\dd{t}^{2} - (1 - 2\Phi)\dd{x^{i}}\dd{x^{i}}.
\end{align*}

Next we study geodesics in this spacetime. For example, in a spherically symmetric case we may choose
\begin{align*}
	\lag = (1 + 2\Phi)\dot{t}^{2} - (1 - 2\Phi)\left(\dot{r}^{2} + r^{2}(\sin[2](\theta)\dot{\phi}^{2} + \dot{\theta}^{2})\right).
\end{align*}
The geodesic equations are
\begin{align*}
	&\dv{\tau}((1 + 2\Phi)\dot{t}) = 0, \\
	&2\del{r}{\Phi}\dot{t}^{2} + 2\del{r}{\Phi}\left(\dot{r}^{2} + r^{2}(\sin[2](\theta)\dot{\phi}^{2} + \dot{\theta}^{2})\right) - (1 - 2\Phi)\left(2r(\sin[2](\theta)\dot{\phi}^{2} + \dot{\theta}^{2})\right) - \dv{\tau}(-2(1 - 2\Phi)\dot{r}) = 0, \\
	&\dv{\tau}(r^{2}\sin[2](\theta)\dot{\phi}) = 0, \\
	&(1 - 2\Phi)\left(2r^{2}\sin(\theta)\cos(\theta)\dot{\phi}^{2}\right) - \dv{\tau}(-2(1 - 2\Phi)r^{2}\dot{\theta}) = 0.
\end{align*}
We assume to be in a weak-field situation, hence the first equation implies $\dot{t}$ being constant. We may take this constant to be $1$, essentially involving a parametrization in terms of $t$. Next we may by symmetry choose $\theta$ to be constant, this constant necessarily being $\frac{\pi}{2}$, to find
\begin{align*}
	&2\del{r}{\Phi} + 2\del{r}{\Phi}\left(\dot{r}^{2} + \frac{L^{2}}{r^{2}}\right) - 2(1 - 2\Phi)\frac{L^{2}}{r^{3}} + \dv{\tau}(2(1 - 2\Phi)\dot{r}) = 0, \\
	&r^{2}\dot{\phi} = L.
\end{align*}
Expanding the derivatives yields
\begin{align*}
	\del{r}{\Phi} + \del{r}{\Phi}\left(\dot{r}^{2} + \frac{L^{2}}{r^{2}}\right) - (1 - 2\Phi)\frac{L^{2}}{r^{3}} + (1 - 2\Phi)\ddot{r} - \del{r}{\Phi}\dot{r}^{2} = \del{r}{\Phi}\left(1 + \frac{L^{2}}{r^{2}} - \dot{r}^{2}\right) - (1 - 2\Phi)\frac{L^{2}}{r^{3}} + (1 - 2\Phi)\ddot{r} = 0.
\end{align*}
Assuming slow motion, such that $L$ and $\dot{r}$ are all much smaller than $1$, we may simplify the above to
\begin{align*}
	\del{r}{\Phi} - \frac{L^{2}}{r^{3}} + \ddot{r} = \ddot{r} + \del{r}{\Phi} - r\dot{\phi}^{2} = 0,
\end{align*}
which is the radial equation of motion for a central potential.

\paragraph{Gravitational Waves}
We have seen that in the weak field limit the perturbations to the metric in vacuum satisfy the wave equation. The solution is of the form
\begin{align*}
	\bar{h}_{\mu\nu} = A_{\mu\nu}e^{-ik^{\sigma}x_{\sigma}}.
\end{align*}
This is an eigenfunction of the d'Alembertian with eigenvalue $-k^{\mu}k_{\mu}$, hence the wavevector must be light-like and waves in the metric travel at the speed of light.

With the chosen gauge for the coordinates we still have some freedom - namely, the coordinate transform $x^{\mu} \to x^{\mu} + B^{\mu}e^{-ik^{\sigma}x_{\sigma}}$ does not change the gauge condition. We may also choose $B^{\mu}$ such that $\tensor{A}{^{\mu}_{\mu}} = 0$ and $A_{0\mu} = 0$ in some frame. Furthermore, in this gauge we have
\begin{align*}
	\del{\mu}{\bar{h}^{\mu\nu}} = \del{\mu}{h^{\mu\nu}} - \frac{1}{2}\del{\mu}{(\eta^{\mu\nu}h)} = \del{\mu}{h^{\nu\mu}} - \frac{1}{2}\del[\nu]{}{h} = 0.
\end{align*}
This implies $k_{\mu}A^{\mu\nu} = 0$.

Specializing to a wave propagating in the third Cartesian direction, the only non-zero amplitudes are $A^{11} = -A^{22} = A_{+}$ and $A^{12} = A^{21} = A_{\cross}$. As gravitational waves propagate between geodesics, we need to consider geodesic deviation. To do this, consider two events close to each other on a simultaneity. Evolution in proper time will then take the events to a new simultaneity. Let $X$ be the vector field that is tangent to the simultaneity and pointing in the direction of the other event for each $\tau$, and $U$ the tangent of one of the geodesics. We then have
\begin{align*}
	\ddot{X} = \dcov{U}{\dcov{U}{X}} = R(U, X)U,
\end{align*}
where we have chosen the torsion such that $\comm{X}{U} = 0$. In component form:
\begin{align*}
	\ddot{X}^{\mu} = \tensor{R}{^{\mu}_{\lambda\sigma\nu}}U^{\lambda}U^{\sigma}X^{\nu}.
\end{align*}
For a slowly moving test particle we obtain
\begin{align*}
	\ddot{X}^{\mu} \approx \tensor{R}{^{\mu}_{00\nu}}X^{\nu}.
\end{align*}
We have
\begin{align*}
	\tensor{R}{_{\mu 00\nu}} &= (\eta_{\mu\sigma} + h_{\mu\sigma})(\del{0}{\chris{\sigma}{\nu}{0}} - \del{\nu}{\chris{\sigma}{0}{0}}) \\
	                         &= \frac{1}{2}\eta_{\mu\sigma}\left(\del{0}{(\eta^{\sigma\rho}\del{0}{h_{\nu\rho}} + \eta^{\sigma\rho}\del{\nu}{h_{\rho 0}} - \eta^{\sigma\rho}\del{\rho}{h_{\nu 0}})} - \del{\nu}{(\eta^{\sigma\rho}\del{0}{h_{0\rho}} + \eta^{\sigma\rho}\del{0}{h_{\rho 0}} - \eta^{\sigma\rho}\del{\rho}{h_{00}})}\right) \\
	                         &= \frac{1}{2}\eta_{\mu\sigma}\left(\eta^{\sigma\rho}\del{0}{\del{0}{h_{\nu\rho}}} + \eta^{\sigma\rho}\del{0}{\del{\nu}{h_{\rho 0}}} - \eta^{\sigma\rho}\del{0}{\del{\rho}{h_{\nu 0}}} - \eta^{\sigma\rho}\del{\nu}{\del{0}{h_{0\rho}}} - \eta^{\sigma\rho}\del{\nu}{\del{0}{h_{\rho 0}}} + \eta^{\sigma\rho}\del{\nu}{\del{\rho}{h_{00}}}\right) \\
	                         &= \frac{1}{2}\kdelta{\rho}{\mu}\left(\del[2]{0}{h_{\nu\rho}} - \del{0}{\del{\rho}{h_{\nu 0}}} - \del{\nu}{\del{0}{h_{\rho 0}}} + \del{\nu}{\del{\rho}{h_{00}}}\right) \\
	                         &= \frac{1}{2}\left(\del[2]{0}{h_{\nu\mu}} - \del{0}{\del{\mu}{h_{\nu 0}}} - \del{\nu}{\del{0}{h_{\mu 0}}} + \del{\nu}{\del{\mu}{h_{00}}}\right) \\
	                         &= \frac{1}{2}\del[2]{0}{h_{\mu\nu}},
\end{align*}
where the last equality follows from the particular geometry. Writing this out we find
\begin{align*}
	\ddot{X}^{\mu} = \frac{1}{2}X^{\nu}\del[2]{0}{\tensor{h}{^{\mu}_{\nu}}} = \frac{1}{2}X^{\nu}\tensor{A}{^{\mu}_{\nu}}\del[2]{0}{e^{-ik^{\sigma}x_{\sigma}}}.
\end{align*}
In the case of $A_{\cross} = 0$ we can simplify this to
\begin{align*}
	\ddot{X}^{1} = \frac{1}{2}X^{\nu}\tensor{A}{^{1}_{\nu}}\del[2]{0}{e^{-ik^{\sigma}x_{\sigma}}} = -\frac{1}{2}X^{1}A_{+}\del[2]{0}{e^{-ik^{\sigma}x_{\sigma}}},\ \ddot{X}^{2} =  \frac{1}{2}X^{2}A_{+}\del[2]{0}{e^{-ik^{\sigma}x_{\sigma}}}.
\end{align*}
To leading order in the amplitude, we thus find
\begin{align*}
	X^{1} = X^{1}(0)\left(1 - \frac{1}{2}A_{+}e^{-ik^{\sigma}x_{\sigma}}\right),\ X^{2} = X^{2}(0)\left(1 + \frac{1}{2}A_{+}e^{-ik^{\sigma}x_{\sigma}}\right).
\end{align*}
I believe the idea is that because the components of the tangent vector oscillate, so do the distances between the events on the two geodesics. In this case the oscillations are in the two coordinate directions. For the case $A_{+} = 0$ we instead have
\begin{align*}
	\ddot{X}^{1} = -\frac{1}{2}X^{2}A_{+}\del[2]{0}{e^{-ik^{\sigma}x_{\sigma}}},\ \ddot{X}^{2} = -\frac{1}{2}X^{1}A_{+}\del[2]{0}{e^{-ik^{\sigma}x_{\sigma}}}.
\end{align*}
This produces the same behaviour, but in a coordinate system rotate $45^{\circ}$ relative to the original system.

\paragraph{Gravitational Lensing}
Consider a light source and an observer separated by a distribution of matter producing a gravitational potential $\Phi$. In the weak-field limit the metric is described by
\begin{align*}
	\dd{s}^{2} = (1 + 2\Phi)\dd{t}^{2} - (1 - 2\Phi)\dd{\vb{x}}^{2},
\end{align*}
and the path of a ray of light travelling between the two is described by
\begin{align*}
	(1 + 2\Phi)\dot{t}^{2} - (1 - 2\Phi)\dot{\vb{x}}^{2} = 0.
\end{align*}
In the weak-field limit we expect $\Phi \ll 1$, and thus $\dot{t}^{2} \approx \dot{\vb{x}}^{2}$.

In the case where the potential is time-independent, the geodesic equation for the time coordinate yields
\begin{align*}
	(1 + 2\Phi)\dot{t} = c,
\end{align*}
and we are free to choose $c = 1$ by rescaling. Next, the equation for the spatial coordinates is
\begin{align*}
	-\ddot{\vb{x}} + 2(\dot{\vb{x}}\cdot\grad{\Phi})\dot{\vb{x}} = (\dot{t}^{2} + \dot{\vb{x}}^{2})\grad{\Phi}.
\end{align*}

Choosing a coordinate system such that $\dot{\vb{x}} = \vb{e}_{x}$ initially, we expect in the weak-field limit
\begin{align*}
	\dot{\vb{x}} = (1 + \dots)\vb{e}_{x} + v^{2}\vb{e}_{y} + v^{3}\vb{e}_{z},
\end{align*}
where the new velocity components are small and the dots represent perturbations of the path of higher order than the other velocity components. Inserting this into the geodesic equation yields
\begin{align*}
	-\dot{v}^{2} = 2(\grad{\Phi})^{2},
\end{align*}
where we refer to vector components on either side. We can solve this differential equation by integrating along the unperturbed path. In particular, by moving both the observer and the source to infinity we find
\begin{align*}
	\dot{v}^{2} = -\integ{-\infty}{\infty}{t}{2(\grad{\Phi})^{2}}.
\end{align*}

\paragraph{The Schwarzschild Solution}
The Schwarzschild solution is the simplest solution for a spherically symmetric metric.

To derive it, we will use spherical coordinates to describe the space part, implying that we construct spacetime as a combination of spherical shells. The metric will then take the form
\begin{align*}
	\dd{s}^{2} = f(t, r)\dd{t}^{2} - g(t, r)\dd{r}^{2} - r^{2}\dd{\Omega}^{2},\ \dd{\Omega}^{2} = \dd{\theta}^{2} + \sin[2](\theta)\dd{\phi}^{2}
\end{align*}
in units where $c = 1$. The proceeding work will involve dividing the functions $f$ and $g$, so we define $f = e^{2\alpha},\ g = e^{2\beta}$. As the metric is diagonal we can write
\begin{align*}
	\chris{\sigma}{\mu}{\nu} = \frac{1}{2g_{\sigma\sigma}}(\del{\nu}{g_{\mu\sigma}} + \del{\mu}{g_{\sigma\nu}} - \del{\sigma}{g_{\mu\nu}}),\ \text{no sum over $\sigma$.}
\end{align*}
Again, as the metric is diagonal, at least two of the indices must be equal, immediately identifying any Christoffel symbol with three different indices as zero. The diagonality of the metric and the angle independence of the time and radial components also implies
\begin{align*}
	\chris{t}{t}{\theta} = \chris{t}{\theta}{t} = \chris{t}{t}{\phi} = \chris{t}{\phi}{t} = \chris{r}{r}{\theta} = \chris{r}{\theta}{r} = \chris{r}{r}{\phi} = \chris{r}{\phi}{r} = \chris{\theta}{t}{t} = \chris{\theta}{r}{r} = \chris{\phi}{t}{t} = \chris{\phi}{r}{r} = 0.
\end{align*}
The time independence of the solid angle also implies $\chris{t}{\theta}{\theta} = \chris{t}{\phi}{\phi} = \chris{\theta}{t}{t} = \chris{\phi}{t}{t} = \chris{\theta}{\theta}{t} = \chris{\theta}{t}{\theta} = \chris{\phi}{\phi}{t} = \chris{\phi}{t}{\phi} = 0$. Next, as the metric is independent of $\phi$, we must also have $\chris{\theta}{\theta}{\phi} =	\chris{\theta}{\phi}{\theta} = \chris{\phi}{\theta}{\theta} = \chris{\phi}{\phi}{\phi} = 0$. We also find $\chris{\theta}{\theta}{\theta} = 0$ as the $\theta$ component is independent of $\theta$. The possibly non-zero ones are
\begin{align*}
	\chris{t}{t}{t} &= \frac{1}{2}e^{-2\alpha}\cdot 2\del{t}{\alpha}e^{2\alpha} = \del{t}{\alpha}, \ \chris{t}{t}{r} = \chris{t}{r}{t} = \frac{1}{2}e^{-2\alpha}\cdot 2\del{r}{\alpha}e^{2\alpha} = \del{r}{\alpha},\ \chris{t}{r}{r} = -\frac{1}{2}e^{-2\alpha}\cdot -2\del{t}{\beta}e^{2\beta} = \del{t}{\beta}e^{2(\beta - \alpha)}, \\
	\chris{r}{t}{t} &= -\frac{1}{2}e^{-2\beta}\cdot -2\del{r}{\alpha}e^{2\alpha} = \del{r}{\alpha}e^{2(\alpha - \beta)},\ \chris{r}{t}{r} = \chris{r}{r}{t} = -\frac{1}{2}e^{-2\beta}\cdot -2\del{t}{\beta}e^{2\beta} = \del{t}{\beta},\ \chris{r}{r}{r} = -\frac{1}{2}e^{-2\beta}\cdot -2\del{r}{\beta}e^{2\beta} = \del{r}{\beta}, \\
	\chris{r}{\theta}{\theta} &= -\frac{1}{2}e^{-2\beta}\cdot 2r = -re^{-2\beta},\ \chris{r}{\phi}{\phi} = -\frac{1}{2}e^{-2\beta}\cdot 2r\sin[2](\theta) = -r\sin[2](\theta)e^{-2\beta},\ \chris{\theta}{r}{\theta} = \chris{\theta}{\theta}{r} = -\frac{1}{2r^{2}}\cdot -2r = \frac{1}{r}, \\
	\chris{\theta}{\phi}{\phi} &= -\frac{1}{2r^{2}}\cdot 2r^{2}\sin(\theta)\cos(\theta) = -\sin(\theta)\cos(\theta),\ \chris{\phi}{r}{\phi} = \chris{\phi}{\phi}{r} = -\frac{1}{2r^{2}\sin[2](\theta)}\cdot -2r\sin[2](\theta) = \frac{1}{r}, \\
	\chris{\phi}{\theta}{\phi} &= \chris{\phi}{\phi}{\theta} = -\frac{1}{2r^{2}\sin[2](\phi)}\cdot -2r^{2}\sin(\theta)\cos(\theta) = \cot(\theta).
\end{align*}

We will now need to solve the Einstein field equations. We will use the formulation purely in terms of the energy-momentum tensor, which yields $R_{\mu\nu} = 0$ in vacuum. We have
\begin{align*}
	R_{\mu\nu} = \del{\rho}{\chris{\rho}{\nu}{\mu}} - \del{\nu}{\chris{\rho}{\rho}{\mu}} + \chris{\sigma}{\nu}{\mu}\chris{\rho}{\rho}{\sigma} - \chris{\sigma}{\rho}{\mu}\chris{\rho}{\nu}{\sigma}.
\end{align*}
Let us commence by finding restrictions on the functions $\alpha$ and $\beta$. Cheating by looking into the future I find
\begin{align*}
	R_{tr} =& \del{\rho}{\chris{\rho}{r}{t}} - \del{r}{\chris{\rho}{\rho}{t}} + \chris{\sigma}{r}{t}\chris{\rho}{\rho}{\sigma} - \chris{\sigma}{\rho}{t}\chris{\rho}{r}{\sigma} \\
	       =& \del{t}{\del{r}{\alpha}} + \del{r}{\del{t}{\beta}} - \del{r}{\del{t}{\alpha}} - \del{r}{\del{t}{\beta}} + \chris{t}{r}{t}(\chris{t}{t}{t} + \chris{r}{r}{t}) + \chris{r}{r}{t}(\chris{t}{t}{r} + \chris{r}{r}{r} + \chris{\theta}{\theta}{r} + \chris{\phi}{\phi}{r}) - \chris{t}{t}{t}\chris{t}{r}{t} - \chris{t}{r}{t}\chris{r}{r}{t} \\
	        &- \chris{r}{t}{t}\chris{t}{r}{r} - \chris{r}{r}{t}\chris{r}{r}{r} - \chris{r}{\theta}{t}\chris{\theta}{r}{r} - \chris{r}{\phi}{t}\chris{\phi}{r}{r} - \chris{\theta}{\theta}{t}\chris{\theta}{r}{\theta} - \chris{\phi}{\phi}{t}\chris{\phi}{r}{\phi} \\
	       =& \chris{r}{r}{t}(\chris{t}{t}{r} + \chris{\theta}{\theta}{r} + \chris{\phi}{\phi}{r}) - \chris{r}{t}{t}\chris{t}{r}{r} \\
	       =& \frac{2}{r}\del{t}{\beta} = 0,
\end{align*}
hence $\beta$ is a function of $r$ only. Next:
\begin{align*}
	R_{\theta\theta} =& \del{\rho}{\chris{\rho}{\theta}{\theta}} - \del{\theta}{\chris{\rho}{\rho}{\theta}} + \chris{\sigma}{\theta}{\theta}\chris{\rho}{\rho}{\sigma} - \chris{\sigma}{\rho}{\theta}\chris{\rho}{\theta}{\sigma} \\
	                 =& \del{r}{\chris{r}{\theta}{\theta}} - \del{\theta}{\chris{\phi}{\phi}{\theta}} + \chris{r}{\theta}{\theta}(\chris{t}{t}{r} + \chris{r}{r}{r} + \chris{\theta}{\theta}{r} + \chris{\phi}{\phi}{r}) - \chris{\phi}{\phi}{\theta}\chris{\phi}{\theta}{\phi} - \chris{r}{\theta}{\theta}\chris{\theta}{\theta}{r} - \chris{\theta}{r}{\theta}\chris{r}{\theta}{\theta} \\
	                 =& -\del{r}{(re^{-2\beta})} - \del{\theta}{\cot(\theta)} - re^{-2\beta}\left(\del{r}{\alpha} + \del{r}{\beta} + \frac{2}{r}\right) - \cot[2](\theta) + 2e^{-2\beta} \\
	                 =& -e^{-2\beta} + 2r\del{r}{\beta}e^{-2\beta} + \csc[2](\theta) - re^{-2\beta}\left(\del{r}{\alpha} + \del{r}{\beta} + \frac{2}{r}\right) - \cot[2](\theta) + 2e^{-2\beta} \\
	                 =& 1 + e^{-2\beta}\left(-1 + 2r\del{r}{\beta} - r\del{r}{\alpha} - r\del{r}{\beta} - 2 + 2\right) \\
	                 =& 1 + e^{-2\beta}\left(r\del{r}{\beta} - 1 - r\del{r}{\alpha}\right) = 0.
\end{align*}
On its own it provides little information, but differentiating this with respect to $t$ yields
\begin{align*}
	\del{t}{\del{r}{\alpha}} = 0,
\end{align*}
with solution $\alpha = u(t) + v(r)$.

The function $u$ may be eliminated by a change of variables such that $\dd{t}\to e^{-2u(t)}\dd{t}$, which yields
\begin{align*}
	\dd{s}^{2} = e^{2v(r)}\dd{t}^{2} - 2e^{\beta}\dd{r}^{2} - r^{2}\dd{\Omega}^{2}.
\end{align*}
Note that with these simplifications we have $\chris{t}{t}{t} = \chris{t}{r}{r} = \chris{r}{r}{t} = \chris{r}{t}{r} = 0$ and no remaining time-dependent components. Next we compute the components
\begin{align*}
	R_{tt} =& \del{\rho}{\chris{\rho}{t}{t}} - \del{t}{\chris{\rho}{\rho}{t}} + \chris{\sigma}{t}{t}\chris{\rho}{\rho}{\sigma} - \chris{\sigma}{\rho}{t}\chris{\rho}{t}{\sigma} \\
	       =& \del{r}{\chris{r}{t}{t}} + \chris{r}{t}{t}\chris{r}{r}{r} + \chris{r}{t}{t}\chris{\theta}{\theta}{r} + \chris{r}{t}{t}\chris{\phi}{\phi}{r} - \chris{\sigma}{t}{t}\chris{t}{t}{\sigma} - \chris{\sigma}{\sigma}{t}\chris{\sigma}{t}{\sigma} \\
	       =& \del{r}{(\del{r}{\alpha}e^{2(\alpha - \beta)})} + \del{r}{\alpha}e^{2(\alpha - \beta)}\del{r}{\beta} + \del{r}{\alpha}e^{2(\alpha - \beta)}\frac{1}{r} + \del{r}{\alpha}e^{2(\alpha - \beta)}\frac{1}{r} - \del{r}{\alpha}e^{2(\alpha - \beta)}\del{r}{\alpha} \\
	       =& e^{2(\alpha - \beta)}\del[2]{r}{\alpha} + 2\del{r}{\alpha}(\del{r}{\alpha} - \del{r}{\beta})e^{2(\alpha - \beta)} + e^{2(\alpha - \beta)}\del{r}{\alpha}\del{r}{\beta} + e^{2(\alpha - \beta)}\frac{2\del{r}{\alpha}}{r} - (\del{r}{\alpha})^{2}e^{2(\alpha - \beta)} \\
	       =& e^{2(\alpha - \beta)}\left(\del[2]{r}{\alpha} + \del{r}{\alpha}\left(\del{r}{\alpha} - \del{r}{\beta} + \frac{2}{r}\right)\right) = 0, \\
	R_{rr} =& \del{\rho}{\chris{\rho}{r}{r}} - \del{r}{\chris{\rho}{\rho}{r}} + \chris{\sigma}{r}{r}\chris{\rho}{\rho}{\sigma} - \chris{\sigma}{\rho}{r}\chris{\rho}{r}{\sigma} \\
	       =& \del{r}{\chris{r}{r}{r}} - \del{r}{\chris{t}{t}{r}} - \del{r}{\chris{r}{r}{r}} - \del{r}{\chris{\theta}{\theta}{r}} - \del{r}{\chris{\phi}{\phi}{r}} + \chris{r}{r}{r}(\chris{t}{t}{r} + \chris{r}{r}{r} + \chris{\theta}{\theta}{r} + \chris{\phi}{\phi}{r}) - \chris{r}{r}{r}\chris{r}{r}{r} - \chris{t}{t}{r}\chris{t}{r}{t} - \chris{r}{r}{r}\chris{r}{r}{r} \\
	        &- \chris{\theta}{\theta}{r}\chris{\theta}{r}{\theta} - \chris{\phi}{\phi}{r}\chris{\phi}{r}{\phi} \\
	       =& -\del[2]{r}{\alpha} - \del{r}{\frac{1}{r}} - \del{r}{\frac{1}{r}} + \del{r}{\beta}\left(\del{r}{\alpha} + \del{r}{\beta} + \frac{2}{r}\right) - (\del{r}{\beta})^{2} - (\del{r}{\alpha})^{2} - \frac{2}{r^{2}} \\
	       =& -\del[2]{r}{\alpha} + \del{r}{\beta}\left(\del{r}{\alpha} + \frac{2}{r}\right) - (\del{r}{\alpha})^{2} = 0,
\end{align*}
hence
\begin{align*}
	(\del{r}{\alpha} + \del{r}{\beta})\left(\del{r}{\alpha} + \frac{2}{r}\right) - \del{r}{\alpha}\del{r}{\beta} - (\del{r}{\alpha})^{2} = \frac{2}{r}(\del{r}{u} + \del{r}{\beta}) = 0,
\end{align*}
and $\beta = C - v$. By rescaling $r\to e^{-\frac{1}{2}C}r$ we may set $C = 0$, and find $\beta = -v$. Finally we dig up an old equation and find
\begin{align*}
	1 + e^{-2\beta}\left(r\del{r}{\beta} - 1 - r\del{r}{\alpha}\right) = 1 + e^{-2\beta}\left(2r\del{r}{\beta} - 1\right) = 0,
\end{align*}
which we may solve and write as
\begin{align*}
	\del{r}{(re^{-2\beta})} = 1.
\end{align*}
Integrating this yields
\begin{align*}
	re^{-2\beta} = r - R_{\text{S}},\ e^{-2\beta} = 1 - \frac{R_{\text{S}}}{r}.
\end{align*}

Finally we may put this together to find
\begin{align*}
	\dd{s}^{2} = \left(1 - \frac{R_{\text{S}}}{r}\right)\dd{t}^{2} - \left(1 - \frac{R_{\text{S}}}{r}\right)^{-1}\dd{r}^{2} - r^{2}\dd{\Omega}^{2},
\end{align*}
where $R_{\text{S}}$ is the Schwarzschild radius. To reobtain Newtonian gravity at large distances we would need $R_{\text{S}} = 2MG$.

This metric has singularities at $r = R_{\text{S}}$ and $r = 0$. If you study the curvature invariant $R_{\mu\nu\lambda\sigma}R^{\mu\nu\lambda\sigma}$, however, you find that it is finite at $r = R_{\text{S}}$ and diverges at $r = 0$. This would imply that there exists a smart choice of coordinates in which the singularity at $R_{\text{S}}$ would be eliminated.

This set of coordinates, called Eddington-Finkelstein coordinates, replaces $t$ with a coordinate that has the light-like geodesics as coordinate lines. For a purely radial path the requirement $\dd{s}^{2} = 0$ for such a geodesic yields
\begin{align*}
	\dd{t}^{2} = \left(1 - \frac{R_{\text{S}}}{r}\right)^{-2}\dd{r}^{2},
\end{align*}
with solution
\begin{align*}
	t = u - r - R_{\text{S}}\ln(\frac{r}{R_{\text{S}}} - 1),
\end{align*}
where $u$ is an integration constant labelling the geodesics. This will be the new coordinate. In these coordinates we obtain
\begin{align*}
	\dd{s}^{2} = \left(1 - \frac{R_{\text{S}}}{r}\right)\dd{u}^{2} - 2\dd{u}\dd{r} - r^{2}\dd{\Omega}^{2}.
\end{align*}
Notably, there is now only a singularity at $r = 0$.

For radial light cones in these coordinates, one obtains
\begin{align*}
	\left(\left(1 - \frac{R_{\text{S}}}{r}\right)\dd{u}^{2} - 2\dd{r}\right)\dd{u} = 0,
\end{align*}
with solutions $\dd{u} = 0$ and $\dv{u}{r} = \frac{2}{1 - \frac{R_{\text{S}}}{r}}$. The first case is as discussed above. The second has $\dv{u}{r} > 0$ for $r < R_{\text{S}}$, meaning that world lines moving towards the future are drawn to the singularity at the origin when within the Schwarzschild radius.

To describe space-like world lines, we can use Kruskal-Szekeres coordinates
\begin{align*}
	U = \abs{\frac{r}{R_{\text{S}}} - 1}^{\frac{1}{2}}e^{\frac{r}{2R_{\text{S}}}}\sinh(\frac{t}{2R_{\text{S}}}),\ V = \abs{\frac{r}{R_{\text{S}}} - 1}^{\frac{1}{2}}e^{\frac{r}{2R_{\text{S}}}}\cosh(\frac{t}{2R_{\text{S}}}).
\end{align*}
One finds that the metric is
\begin{align*}
	\dd{s}^{2} = \frac{4R_{\text{S}}^{3}}{r}e^{-\frac{r}{R_{\text{S}}}}(\dd{U}^{2} - \dd{V}^{2}) - r^{2}\dd{\Omega}^{2}.
\end{align*}
In these coordinates geodesics are hyperbolae.

\paragraph{Geometric Symmetries and Conserved Quantities}
Assume that a spacetime has a Killing field $K$, and consider a geodesic of the spacetime with tangent $U$. Along the geodesic we then have
\begin{align*}
	\dv{\tau}(g_{\mu\nu}K^{\mu}U^{\nu}) = \dcov{U}{(g_{\mu\nu}K^{\mu}U^{\nu})} = (\dcov{U}{g_{\mu\nu}})K^{\mu}U^{\nu} + g_{\mu\nu}\dcov{U}{K^{\mu}U^{\nu}}.
\end{align*}
As the Levi-Civita connection is metric-compatible, the former term vanishes, and we are left with
\begin{align*}
	\dv{\tau}(g_{\mu\nu}K^{\mu}U^{\nu}) = g_{\mu\nu}U^{\nu}\dcov{U}{K^{\mu}} + g_{\mu\nu}K^{\mu}\dcov{U}{U^{\nu}}.
\end{align*}
As the path is a geodesic, the latter term vanishes and all that is left is
\begin{align*}
	\dv{\tau}(g_{\mu\nu}K^{\mu}U^{\nu}) = g_{\mu\nu}U^{\nu}\dcov{U}{K^{\mu}} = g_{\mu\nu}U^{\nu}U^{\sigma}\dcov{\sigma}{K^{\mu}} = U^{\nu}U^{\sigma}\dcov{\sigma}{K_{\nu}}.
\end{align*}
The two first factors are symmetric under permutation of indices. The other, on the other hand, is antisymmetric as $K$ is a Killing field, hence $\dv{\tau}(g_{\mu\nu}K^{\mu}U^{\nu}) = 0$. Thus the quantity $g_{\mu\nu}K^{\mu}U^{\nu}$ is constant along geodesics.

\paragraph{Symmetries of the Schwarzschild Solution}
We note that $\del{t}{}$ and $\del{\phi}{}$ are both Killing fields of the Schwarzschild solution. For $r > R_{\text{S}}$ one has
\begin{align*}
	g(\del{t}{}, \del{t}{}) = 1 - \frac{R_{\text{S}}}{r} > 0,
\end{align*}
and $\del{t}{}$ is time-like.

For a geodesic we define
\begin{align*}
	\sqrt{2E} &= g(\del{t}{}, \dot{\gamma}) = \left(1 - \frac{R_{\text{S}}}{r}\right)\dot{t}, \\
	L         &= -g(\del{\phi}{}, \dot{\gamma}) = r^{2}\sin[2](\theta)\dot{\phi}, \\
	\alpha    &= g(\dot{\gamma}, \dot{\gamma}),
\end{align*}
which are all constants along the path. The latter is $1$ for a time-like geodesic and $0$ for a light-like geodesic. By definition we have
\begin{align*}
	\alpha &= \left(1 - \frac{R_{\text{S}}}{r}\right)\dot{t}^{2} - \frac{1}{1 - \frac{R_{\text{S}}}{r}}\dot{r}^{2} - r^{2}\dot{\theta}^{2}  - r^{2}\sin[2](\theta)\dot{\phi}^{2} = \frac{2E}{1 - \frac{R_{\text{S}}}{r}} - \frac{1}{1 - \frac{R_{\text{S}}}{r}}\dot{r}^{2} - r^{2}\dot{\theta}^{2} - \frac{L^{2}}{r^{2}}.
\end{align*}
Noting that $\theta = \frac{\pi}{2}$ solves the geodesic equations, we may limit our considerations to such a solution by symmetry. We then have
\begin{align*}
	E - \frac{1}{2}\dot{r}^{2} = \frac{1}{2}\left(\alpha + \frac{L^{2}}{r^{2}}\right)\left(1 - \frac{R_{\text{S}}}{r}\right).
\end{align*}
This looks like the relation describing a classical particle moving in a potential.

\paragraph{Frequency Shift}
A wave generally has a phase $\phi$ which depends on both position and time. For a general world line with tangent $V$ we define
\begin{align*}
	\omega = \dcov{V}{\phi} = V^{\mu}\del{\mu}{\phi} = \df{\phi}(V).
\end{align*}
By raising the indices of $\df{\phi}$, one obtains the $4$-frequency $N^{\mu}$.

Consider the case of light-like surfaces of constant phase, implying $g^{\mu\nu}(\df{\phi})_{\mu}(\df{\phi})_{\nu} = g^{\mu\nu}\del{\mu}{\phi}\del{\nu}{\phi} = 0$. We then find
\begin{align*}
	(\dcov{N}{\df{\phi}})_{\mu} &= N^{\nu}(\del{\nu}{\del{\mu}{\phi}} - \chris{\sigma}{\nu}{\mu}\del{\sigma}{\phi}) \\
	                            &= g^{\nu\rho}(\del{\rho}{\phi}\del{\nu}{\del{\mu}{\phi}} - \chris{\sigma}{\nu}{\mu}\del{\rho}{\phi}\del{\sigma}{\phi}) \\
	                            &= \frac{1}{2}g^{\nu\rho}\left(\del{\rho}{\phi}\del{\nu}{\del{\mu}{\phi}} - \chris{\sigma}{\nu}{\mu}\del{\rho}{\phi}\del{\sigma}{\phi} + \del{\nu}{\phi}\del{\rho}{\del{\mu}{\phi}} - \chris{\sigma}{\rho}{\mu}\del{\nu}{\phi}\del{\sigma}{\phi}\right) \\
	                            &= \frac{1}{2}g^{\nu\rho}\left(\del{\mu}{(\del{\rho}{\phi}\del{\nu}{\phi})} - \chris{\sigma}{\nu}{\mu}\del{\rho}{\phi}\del{\sigma}{\phi} - \chris{\sigma}{\rho}{\mu}\del{\nu}{\phi}\del{\sigma}{\phi}\right) \\
	                            &= \frac{1}{2}g^{\nu\rho}\dcov{\mu}{(\del{\rho}{\phi}\del{\nu}{\phi})} \\
	                            &= \frac{1}{2}\dcov{\mu}{(g^{\nu\rho}\del{\rho}{\phi}\del{\nu}{\phi})} \\
	                            &= 0.
\end{align*}
In other words, $\df{\phi}$ is parallel along the world line defined by $N$, implying that these world lines are light-like geodesics.

In general an observer will measure a frequency $f = g(N, V) = \df{\phi}(V)$. The emitted frequency will simply be this inner product, while the frequency observed at a different point will be found by the same process after having parallel transported $N$. Computing this parallel transport is usually cumbersome, so we will avoid applying it directly in favor of other arguments.

\paragraph{Gravitational Time Dilation and Redshift}
Consider a static space time with spacetime interval $\dd{s}^{2} = \alpha^{2}\dd{t}^{2} - h_{ij}\dd{\chi}^{i}\dd{\chi}^{j}$, where the components of the metric are functions of the spatial coordinates only. Now consider two comoving observers in this spacetime observing two events. As $\del{t}{}$ is a Killing field, the geodesics have time translation symmetry, hence the two observers must observe the same time difference $t_{0}$. Each observer observes an elapsed proper time
\begin{align*}
	\tau = \integ{t}{t + t_{0}}{t}{\sqrt{g_{tt}}} = \alpha t_{0}.
\end{align*}
This means that the elapsed proper time dilates according to
\begin{align*}
	\frac{\tau_{A}}{\tau_{B}} = \frac{\alpha_{A}}{\alpha_{B}}.
\end{align*}

In particular, if the two events are successive crests of a light pulse, we find the gravitational redshift formula
\begin{align*}
	\frac{f_{B}}{f_{A}} = \frac{\alpha_{A}}{\alpha_{B}}.
\end{align*}

\paragraph{Experimental Evidence for General Relativity}
The first example evidence validating general relativity is the perihelion precession of Mercury. Namely, the elliptic axis of the orbit of Mercury precedes with a very long period. Much of this could be attributed to many-body effects, but the remaining contribution matched very closely with the prediction from general relativity.

Next there is strong gravitational lensing, exemplified in an observation of a solar eclipse in 1919. Due to the gravitation from the Sun, a star was visible on Earth despite being behind the Sun.

Then there is so-called Shapiro delay - namely, it takes more time for light signals to travel due to gravitation.

Furthermore there is the Pound-Rebka experiments, which demonstrates gravitational time dilation.

Finally among the major examples discussed is the presence of gravitational effects in binary systems. As they emit gravitational systems they lose energy, causing their orbits to slow. When two stars merge, relatively strong waves are formed. The observation of gravitational waves from such a phenomenon was recently discovered and awarded the Nobel prize.

In addition there are gyroscopic effects and effects due to strong gravitational lensing.