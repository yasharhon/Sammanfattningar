\section{Basic Concepts}

\paragraph{A Note on Minkowski Space}
In special relativity we work with Minkowski space, which is an affine space with a so-called pseudo-metric. This is a metric which is not positive definite, but instead a metric which has only non-zero eigenvalues (and is thus termed non-degenerate). We will work with the signature $(1, 3)$, meaning that there are three eigenvalues of $-1$ and one eigenvalue $1$.

\paragraph{The Description of Spacetime}
In general relativity we will describe spacetime as a $4$-dimensional manifold with a pseudometric of signature $(1, 3)$ with a Levi-Civita connection imposed on it.

\paragraph{Kinematics of Test Particles}
A test particle is a particle that itself does not affect the spacetime. Such particles can generally move through spacetime, along curves called world lines. With this motion comes the $4$-velocity $V$, defined as the normalized tangent to a world line. In special relativity we could also define a proper acceleration by differentiating with respect to proper time. In general relativity we replace this with the $4$-acceleration $A = \dcov{V}{V} = \dcov{\dot{\gamma}}{\dot{\gamma}}$.  We may also define the proper acceleration $\alpha$, which satisfies $\alpha^{2} = -A^{2} = -g(A, A)$, and it can be shown that if $V$ is time-like, then $A$ is space-like. Note that the curve parameter we use is $\tau$, which is the proper time and a measure of length in spacetime.

\paragraph{Free Particles}
A free particle in special relativity is a particle for which $A = 0$. We take this definition to apply in general relativity as well. This implies that free test particles travel along spacetime geodesics.

\paragraph{$4$-Momentum and $4$-Force}
We also define the $4$-momentum $P = mV$ and the $4$-force $F = \dcov{V}{P}$.

\paragraph{Simultaneity}
Two events are simultaneous if they are on the same hypersurface of constant $t$. As this depends very much on the choice of coordinates on spacetime, this notion is not at all well-defined.

\paragraph{Static and Stationary Spacetime}
If there exists a stime-like Killing field of a spacetime, it is stationary. If the spacetime is also orthogonal to a family of $3$-surfaces, the spacetime is static. The consequences of the latter is that the metric has no off-diagonal components.

\paragraph{The Einstein Field Equations}
The equations describing the metric are the Einstein field equations. We will derive them from a variational principle, starting with the field equations for vacuum. The action from which we will obtain the field equations is the Einstein-Hilbert action
\begin{align*}
	S_{\text{EH}} = -\frac{M_{\alpha}^{2}}{2}\integ[4]{}{}{\chi}{\R\sqrt{-\det(g)}}.
\end{align*}
%TODO: Derive
The final equations are
\begin{align*}
	R_{\mu\nu} - \frac{1}{2}\R = 0.
\end{align*}

To obtain the field equations for other cases we will add terms to the action. In general, by adding a term $S_{\text{matter}}$ to the action, we obtain
\begin{align*}
	R_{\mu\nu} - \frac{1}{2}\R = 8\pi GT_{\mu\nu},
\end{align*}
where
\begin{align*}
	T_{\mu\nu} = \frac{2}{\sqrt{-\det(g)}}\fdv{S_{\text{matter}}}{g^{\mu\nu}}
\end{align*}
is called the energy-momentum tensor.

\paragraph{Frequency Shift}
A wave generally has a phase which depends on both position and time. We define the frequency of a wave as $\omega = \dv{\phi}{t}$. For a general world line we define
\begin{align*}
	\omega = \dv{\phi}{\tau} = \dot{\chi}^{a}\del{a}{\phi} = V\phi = \df{\phi}(V).
\end{align*}
$\df{\phi}$ is the dual of the $4$-frequency $N^{\mu}$. It can be shown that $\dcov{N}{\df{\pi}} = 0$, and thus $\dcov{N}{N} = 0$. Rays with tangent $N$ are thus light-like geodesics.

In general an observer will measure a frequency $\df{\phi}(V)$.

\paragraph{The Schwarzschild Solution}
The Schwarzschild solution is the simplest solution for a spherically symmetric metric. It is of the form
\begin{align*}
	\dd{s}^{2} = \left(1 - \frac{R_{\text{S}}}{r}\right)\dd{t}^{2} - \left(1 - \frac{R_{\text{S}}}{r}\right)^{-1}\dd{r}^{2} - r^{2}\dd{\Omega}^{2},
\end{align*}
where $R_{\text{S}}$ is the Schwarzschild radius and we work in units where $c = 1$. To reobtain Newtonian gravity at large distances we would need $R_{\text{S}} = 2MG$.

This metric has singularities at $r = R_{\text{S}}$ and $r = 0$. If you study the curvature invariant $R_{\mu\nu\lambda\sigma}R^{\mu\nu\lambda\sigma}$, however, you find that it is finite at $r = R_{\text{S}}$ and diverges at $r = 0$. This would imply that there exists a smart choice of coordinates in which the singularity at $R_{\text{S}}$ would be eliminated.

This set of coordinates, called Eddington-Finkelstein coordinates, replaces $t$ with a coordinate that has the light-like geodesics as coordinate lines. For a purely radial path the requirement $\dd{s}^{2} = 0$ for such a geodesic yields
\begin{align*}
	\dd{t}^{2} = \left(1 - \frac{R_{\text{S}}}{r}\right)^{-2}\dd{r}^{2},
\end{align*}
with solution
\begin{align*}
	t = u - r - R_{\text{S}}\ln(\frac{r}{R_{\text{S}}} - 1),
\end{align*}
where $u$ is an integration constant labelling the geodesics. This will be the new coordinate. In these coordinates we obtain
\begin{align*}
	\dd{s}^{2} = \left(1 - \frac{R_{\text{S}}}{r}\right)\dd{u}^{2} - 2\dd{u}\dd{r} - r^{2}\dd{\Omega}^{2}.
\end{align*}
Notably, there is now only a singularity at $r = 0$.

For radial light cones in these coordinates, one obtains
\begin{align*}
	\left(\left(1 - \frac{R_{\text{S}}}{r}\right)\dd{u}^{2} - 2\dd{r}\right)\dd{u} = 0,
\end{align*}
with solutions $\dd{u} = 0$ and $\dv{u}{r} = \frac{2}{1 - \frac{R_{\text{S}}}{r}}$. The first case is as discussed above. The second has $\dv{u}{r} > 0$ for $r < R_{\text{S}}$, meaning that world lines moving towards the future are drawn to the singularity at the origin when within the Schwarzschild radius.

To describe space-like world lines, we can use Kruskal-Szekeres coordinates
\begin{align*}
	U = \abs{\frac{r}{R_{\text{S}}} - 1}^{\frac{1}{2}}e^{\frac{r}{2R_{\text{S}}}}\sinh(\frac{t}{2R_{\text{S}}}),\ V = \abs{\frac{r}{R_{\text{S}}} - 1}^{\frac{1}{2}}e^{\frac{r}{2R_{\text{S}}}}\cosh(\frac{t}{2R_{\text{S}}}).
\end{align*}
One finds that the metric is
\begin{align*}
	\dd{s}^{2} = \frac{4R_{\text{S}}^{3}}{r}e^{-\frac{r}{R_{\text{S}}}}(\dd{U}^{2} - \dd{V}^{2}) - r^{2}\dd{\Omega}^{2}.
\end{align*}
In these coordinates geodesics are hyperbolae.

\paragraph{Symmetries and Conserved Quantities}
Assume that a spacetime has a Killing field $K$, and consider a geodesic of the spacetime with tangent $U$. Along the geodesic we then have
\begin{align*}
	\dd{\tau}g_{\mu\nu}K^{\mu}U^{\nu} = \dcov{U}{g_{\mu\nu}K^{\mu}U^{\nu}} = (\dcov{U}{g_{\mu\nu}})K^{\mu}U^{\nu} + g_{\mu\nu}\dcov{U}{K^{\mu}U^{\nu}}.
\end{align*}
As the Levi-Civita connection is metric-compatible, the former term vanishes, and we are left with
\begin{align*}
	\dd{\tau}g_{\mu\nu}K^{\mu}U^{\nu} = g_{\mu\nu}U^{\nu}\dcov{U}{K^{\mu}} + g_{\mu\nu}K^{\mu}\dcov{U}{U^{\nu}}.
\end{align*}
As the path is a geodesic, the latter term vanishes and all that is left is
\begin{align*}
	\dd{\tau}g_{\mu\nu}K^{\mu}U^{\nu} = g_{\mu\nu}U^{\nu}\dcov{U}{K^{\mu}} = g_{\mu\nu}U^{\nu}U^{\sigma}\dcov{\sigma}{K^{\mu}} = U^{\nu}U^{\sigma}\dcov{\sigma}{K_{\nu}}.
\end{align*}
The two first factors are symmetric under permutation of indices, whereas the latter is not, hence this is just $0$. Thus the quantity $g_{\mu\nu}K^{\mu}U^{\nu}$ is a constant of motion along the path.

\paragraph{Symmetries of the Schwarzschild Solution}
We note that $\del{t}{}$ and $\del{\phi}{}$ are both Killing fields of the Schwarzschild solution. For $r > R_{\text{S}}$ one find that $\del{t}{}$ is time-like.

For a general path we define
\begin{align*}
	\sqrt{2E} &= g(\del{t}{}, \dot{\gamma}) = \left(1 - \frac{R_{\text{S}}}{r}\right)\dot{t}, \\
	L         &= g(\del{\phi}{}, \dot{\gamma}) = r^{s}\sin[2](\theta)\dot{\phi}, \\
	\alpha    &= g(\dot{\gamma}, \dot{\gamma}).
\end{align*}
The latter is $1$ for a time-like path and $0$ for a light-like path. By definition we have
\begin{align*}
	\alpha &= \left(1 - \frac{R_{\text{S}}}{r}\right)\dot{t}^{2} - \frac{1}{1 - \frac{R_{\text{S}}}{r}}\dot{r}^{2} - r^{2}\sin[2](\theta)\dot{\phi}^{2} = \frac{2E}{1 - \frac{R_{\text{S}}}{r}} - \frac{1}{1 - \frac{R_{\text{S}}}{r}}\dot{r}^{2} - \frac{L^{2}}{r^{2}},
\end{align*}
which may be written as
\begin{align*}
	E - \frac{1}{2}\dot{r}^{2} = \frac{1}{2}\left(\alpha + \frac{L^{2}}{r^{2}}\right)\left(1 - \frac{R_{\text{S}}}{r}\right).
\end{align*}
This looks like the relation describing the potential energy of a particle in classical mechanics.

\paragraph{Gravitational Time Dilation and Redshift}
Consider a static space time. This means that there exists coordinates $\chi^{a}$ such that $\dd{s}^{2} = \alpha^{2}\dd{t}^{2} - h_{ij}\dd{\chi}^{i}\dd{\chi}^{j}$, where the components of the metric are functions of the spatial coordinates only. Now consider two static observers in this spacetime observing two events. As the geodesics have time translation symmetry due to the metric being static, the two observers must observe the same time difference $t_{0}$. Each observer observes an elapsed proper time
\begin{align*}
	\tau = \integ{t}{t + t_{0}}{t}{\sqrt{g_{tt}}} = \alpha t_{0}.
\end{align*}
This means that
\begin{align*}
	\frac{\tau_{A}}{\tau_{B}} = \frac{\alpha_{A}}{\alpha_{B}}.
\end{align*}

From this the gravitational redshift follows. If the two events are successive crests of a light pulse, we find
\begin{align*}
	\frac{f_{B}}{f_{A}} = \frac{\alpha_{A}}{\alpha_{B}}.
\end{align*}

\paragraph{Ideal Fluids}
An ideal fluid is a substance such that its energy-momentum tensor is of the form
\begin{align*}
	T^{\mu\nu} = (\rho_{0} + p)u^{\mu}u^{\nu} - pg^{\mu\nu},
\end{align*}
where $u^{\mu}$ is the $4$-velocity of the rest frame of the fluid and $\rho_{0}$ and $p$ are the energy density and pressure of the fluid as measured in the rest frame.