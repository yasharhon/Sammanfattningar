\section{Differential Geometry}

For details on much of this, notably the early parts on Euclidean space, please consult my summary of SI2360 Analytical Mechanics and Classical Field Theory.

\paragraph{Euclidean and Affine Spaces}
A Euclidean space is a set of points such that there to each point can be assigned a position vector. To such spaces we may assign a set of $n$ coordinates $\chi^{a}$ which together uniquely describe each point in the space locally.

\paragraph{Tangent and Dual Bases}
The tangent and dual bases are defined by
\begin{align*}
	\tb{a} = \del{\chi^{a}}{\vb{r}} = \del{a}{\vb{r}},\ \db{a} = \grad{\chi^{a}}.
\end{align*}
Using such bases, we may write
\begin{align*}
	\vb{v} = v^{a}\tb{a} = v_{a}\db{a}.
\end{align*}
The components of these vectors are called contravariant and covariant components respectively.

\paragraph{Christoffel Symbols}
When computing the derivative of a vector quantity, one must account both for the change in the quantity itself and the change in the basis vectors. We define the Christoffel symbols according to
\begin{align*}
	\del{b}{\tb{a}} = \chris{c}{b}{a}\tb{c}.
\end{align*}
These can be computed according to
\begin{align*}
	\db{c}\cdot\del{b}{\tb{a}} = \db{c}\cdot\chris{d}{b}{a}\tb{d} = \kdelta{d}{c}\chris{d}{b}{a} = \chris{c}{b}{a}.
\end{align*}
Note that
\begin{align*}
	\del{a}{\tb{b}} = \del{a}{\del{b}{\vb{r}}} = \del{b}{\del{a}{\vb{r}}} = \del{b}{\tb{a}},
\end{align*}
which implies
\begin{align*}
	\chris{c}{b}{a} = \chris{c}{a}{b}.
\end{align*}
Similarly, we might want to consider $\del{b}{\db{a}}$, which might introduce new symbols. We find, however, that
\begin{align*}
	\del{a}{\db{b}\cdot\tb{c}} = \db{b}\cdot\del{a}{\tb{c}} + \tb{c}\cdot\del{a}{\db{b}} = 0,
\end{align*}
which implies
\begin{align*}
	\del{a}{\db{b}} = -\chris{b}{a}{c}\db{c}.
\end{align*}

\paragraph{Covariant Derivatives}
Covariant derivatives are defined by
\begin{align*}
	\dcov{a}{v^{b}} = \del{a}{v^{b}} + \chris{b}{a}{c}v^{c},
\end{align*}
and thus satisfy
\begin{align*}
	\del{a}{\vb{v}} = \tb{b}\dcov{a}{v^{b}}.
\end{align*}

\paragraph{Tensors}
To define tensors, we first define tensors of the kind $(0, n)$ as maps from $n$ vectors to scalars. Using this, we define tensors of the kind $(n, m)$ as linear maps from $(0, n)$ tensors to $(0, m)$ tensors.

\paragraph{Manifolds}
Manifolds are sets which are locally isomorphic to an open subset of $\R^{n}$.

\paragraph{Tangent and Dual Bases}
The tangent basis for a manifold is $\tb{a} = \del{a}{}$. The corresponding dual basis, denoted $\df{\chi^{a}}$, is defined such that $\df{f}(X) = X^{a}\del{a}{f}$.

\paragraph{Tensors}
A general $(n, m)$ tensor is constructed by taking the tensor product of tangent and dual basis vectors.

\paragraph{Pushforwards and Pullbacks}