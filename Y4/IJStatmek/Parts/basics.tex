\section{Basic Concepts}

\paragraph{Stochastic Processes}
A stochastic process is a function of a random variable. We use the notation $f(X, t)$, and outcomes of $X$ corresponds to functions
\begin{align*}
	f(x, t) = f_{x}(t).
\end{align*}
The moments of a stochastic process are
\begin{align*}
	\expval{\prod\limits_{i}f(X, t_{i})} = \integ{}{}{x}{P(x)\prod\limits_{i}f_{x}(t_{i})}.
\end{align*}

\section{Diffusion and Random Walks}
Macroscopic diffusion is governed by the diffusion equation
\begin{align*}
	\del{t}{\rho} = D\laplacian{\rho}.
\end{align*}
In particular, with the initial condition $\rho(\vb{x}, t_{0}) = \delta(\vb{x} - \vb{x}_{0})$ we find
\begin{align*}
	\rho(\vb{x}, t) = \left(4\pi D(t - t_{0})\right)^{-\frac{d}{2}}e^{-\frac{\abs{\vb{x} - \vb{x}_{0}}^{2}}{4D(t - t_{0})}}.
\end{align*}

Can we re-obtain this result from microscopic dynamics? One way to do this is to consider a random walk with coordinates
\begin{align*}
	x^{i}(t + \Delta t) = x^{i}(t) + \Delta x_{n}^{i},
\end{align*}
where the $\Delta x_{n}^{i}$ are random variables assumed to be independent and from the same probability distribution. It turns out that all the information we need is contained in the first two moments
\begin{align*}
	\expval{\Delta x_{n}^{i}} = 0,\ \expval{(\Delta x_{n}^{i})^{2}} = \sigma^{2},
\end{align*}
as with this recursion we find
\begin{align*}
	x^{i}(t) = x^{i}_{0} + \sum\limits_{n}\Delta x_{n}^{i}.
\end{align*}
For a large number of steps $N$ the coordinates then follow a normal distribution. Its expectation value is
\begin{align*}
	\expval{x^{i}(t)} = \expval{x^{i}_{0} + \sum\limits_{n}\Delta x_{n}^{i}} = x^{i}_{0}.
\end{align*}
Its variance is
\begin{align*}
	\expval{(x^{i}(t) - \expval{x^{i}(t)})^{2}} &= \expval{\left(\sum\limits_{n}\Delta x_{n}^{i}\right)^{2}} \\
	                                            &= \sum\limits_{m, n}\expval{\Delta x_{m}^{i}\Delta x_{n}^{i}} \\
	                                            &= \sum\limits_{m = n}\expval{\Delta x_{m}^{i}\Delta x_{n}^{i}} + \sum\limits_{m \neq n}\expval{\Delta x_{m}^{i}\Delta x_{n}^{i}} \\
	                                            &= N\sigma^{2},
\end{align*}
where we have used the fact that the steps are independent and the covariance therefore is $0$, implying
\begin{align*}
	\expval{(\Delta x_{m}^{i} - \expval{\Delta x_{m}^{i}})(\Delta x_{n}^{i} - \expval{\Delta x_{n}^{i}})} &= \expval{\Delta x_{m}^{i}\Delta x_{n}^{i}} - \expval{\Delta x_{m}^{i}\expval{\Delta x_{n}^{i}}} - \expval{\expval{\Delta x_{m}^{i}}\Delta x_{n}^{i}} + \expval{\expval{\Delta x_{m}^{i}}\expval{\Delta x_{n}^{i}}} \\
	             &= \expval{\Delta x_{m}^{i}\Delta x_{n}^{i}} = 0.
\end{align*}
To rewrite this in the appropriate form, we introduce the time as $t = N\Delta t$ and define
\begin{align*}
	\frac{\sigma^{2}}{\Delta t}t = 2Dt,
\end{align*}
meaning that things have the right form somehow.

\paragraph{Irreversibility and Classical Mechanics}
The example with random walks showed that the diffusion result was produced under the assumption that the microscopic process did not have time reversal symmetry. Classical mechanics, however, does. How do we reconcile the two?

One ad hoc answer is that classical dynamics, while allowing for dynamics with time reversal, will not necessarily admit this with a high probability, accounting for why it is not observed.

A more computational example might be found by defining the statistical entropy
\begin{align*}
	S = -\kb\integ{}{}{\vb{x}}{f(\vb{x}, t)\ln(f(\vb{x}, t))}.
\end{align*}
$f$ is the probability density in phase space, and according to Liouville's theorem we have
\begin{align*}
	\del{t}{f} = \pb{\ham}{f},
\end{align*}
where \ham is the Hamiltonian. Defining the velocity field $\vb{v} = \dv{\vb{x}}{t}$ in phase space and using Hamilton's equations we may alternatively write the above as
\begin{align*}
	\del{t}{f} = -\vb{v}\cdot\grad_{\vb{x}}{f}.
\end{align*}
As
\begin{align*}
	\div_{\vb{x}}{\vb{v}} = \del{q^{i}}{\del{p_{i}}{\ham}} - \del{p^{i}}{\del{q_{i}}{\ham}} = 0,
\end{align*}
we have
\begin{align*}
	\del{t}{f} = -\div_{\vb{x}}{(f\vb{v})}.
\end{align*}
The derivative of the entropy is thus
\begin{align*}
	\del{t}{S} &= -\kb\integ{}{}{\vb{x}}{\del{t}{f}\ln(f) + \del{t}{f}} \\
	           &= \kb\integ{}{}{\vb{x}}{\div_{\vb{x}}{(f\vb{v})}(\ln(f) + 1)}.
\end{align*}
The latter term may be converted to a surface term using Gauss' theorem, and choosing boundaries at infinity it produces no contribution. Similarly for the other term we may write
\begin{align*}
	\del{t}{S} &= \kb\integ{}{}{\vb{x}}{\div_{\vb{x}}{(f\ln(f)\vb{v})} - f\vb{v}\cdot\grad_{\vb{x}}{\ln(f)}} \\
	           &= -\kb\integ{}{}{\vb{x}}{f\vb{v}\cdot\frac{1}{f}\grad_{\vb{x}}{f}} \\
	           &= -\kb\integ{}{}{\vb{x}}{\vb{v}\cdot\grad_{\vb{x}}{f}}.
\end{align*}
Repeating a previous step, we find that this term gives no contribution either, meaning $\del{t}{S} = 0$.

\paragraph{Brownian Motion and the Langevin Equation}
To describe Brownian motion we consider particles much heavier than the molecules in the medium in which they are immersed and perform the anzats
\begin{align*}
	m\dv{v^{i}}{t} = R_{i}(t),
\end{align*}
where the $R_{i}(t)$ are stochastic processes with the underlying random variable being the initial position of the particle (I think). In order to produce Brownian motion we expect $\expval{R_{i}(t)} = 0$. Next we introduce the autocorrelation function
\begin{align*}
	k(t, t\p) = \expval{(R_{i}(t) - \expval{R_{i}(t)})(R_{i}(t\p) - \expval{R_{i}(t\p)})} = \expval{R_{i}(t)R_{i}(t\p)} - \expval{R_{i}(t)}\expval{R_{i}(t\p)}.
\end{align*}
It measures for how long deviations from the mean force remain correlated, and thus the time scales of the fluctuations and typical amplitudes of the force. In the case of a single time scale being relevant, we name it the autocorrelation time and denote it $\tau_{\text{c}}$. For Brownian motion this time scale is the typical time between collisions. In the relevant limits in which we are working we expect
\begin{align*}
	k(t, t\p) = \expval{R_{i}(t)R_{i}(t\p)} = \Gamma\delta(t - t\p),
\end{align*}
as the autocorrelation time is assumed to be unresolvable on the relevant time scales, and may therefore be taken to be zero. $\sqrt{\Gamma}$ is a typical intensity of the fluctuations in the force. If we assume the stochastic processes to be Gaussian, we have now completely specified them.

Does the anzats work? No! By computing expectation values, we see that constant drift is allowed, which is certainly not what we expect. The liquid will in fact resist the motion of the particle with a velocity-dependent force, which we will remedy by adding a first-order correction. By also incorporating an external force $\vb{F}$ we arrive at the Langevin equation
\begin{align*}
	m\dv{v^{i}}{t} = R_{i}(t) + F_{i} - \alpha v_{i}.
\end{align*}

What kind of behaviour is produced now? To begin with, let us assume there are no external forces. The expectation value then evolves according to
\begin{align*}
	m\dv{\expval{v^{i}}}{t} =  -\alpha\expval{v_{i}},
\end{align*}
with solution
\begin{align*}
	\expval{v_{i}} = v_{i, 0}e^{-\frac{t}{\tau_{v}}},\ \tau_{v} = \frac{1}{\gamma} = \frac{m}{\alpha}.
\end{align*}
To be consistent with our assumptions, we thus require $\tau_{v} \gg \tau_{\text{c}}$.

In a general case we have
\begin{align*}
	v_{i}(t) = e^{-\gamma t}v_{i, 0} + \frac{1}{m}\integ{0}{t}{s}{e^{-\gamma(t - s)}(R(s) + F(s))}.
\end{align*}
Expecting
\begin{align*}
	\frac{1}{2}m\expval{v^{2}} = \frac{d}{2}\kb T,
\end{align*}
we find
\begin{align*}
	\Gamma = 2dm\gamma\kb T.
\end{align*}
We can also introduce the velocity correlation function
\begin{align*}
	C(t, t\p) = \expval{v(t)v(t\p)} - \expval{v(t)}\expval{v(t\p)},
\end{align*}
and we find
\begin{align*}
	C(t, t\p) = \frac{\Gamma}{2\gamma m^{2}}e^{-\gamma(t + t\p)}(e^{2\gamma t\p} - 1).
\end{align*}
This can be used to show
\begin{align*}
	\frac{1}{\alpha} = \frac{1}{2\kb T}\integ{-\infty}{\infty}{t}{C(t)}.
\end{align*}

Next, the position is
\begin{align*}
	x_{i}(t) = x_{i, 0} + \frac{v_{i, 0}}{\gamma}(1 - e^{-\gamma t}) + \frac{1}{m}\integ{0}{t}{s}{\integ{0}{s}{s\p}{R(s\p)e^{-\gamma(s - s\p)}}}.
\end{align*}
We can thus find
\begin{align*}
	\expval{(x_{i}(t) - \expval{x_{i}(t)})^{2}} = 2Dt.
\end{align*}