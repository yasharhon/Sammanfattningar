\section{Quantum Systems}

\paragraph{The Idea}
For this consideration of quantum systems, the idea is to have the system interact with some bath containing an enormous number of degrees of freedom relative to those of the system.

\paragraph{The Reduced Density Matrix}
The reduced density matrix is found by tracing out a particular set of degrees of freedom from a total density matrix.

\paragraph{Product Operators}
We recall that for a tensor product, the expectation value of the product is the product of the expectation values. This means that expectation values of system observables make sense.

\paragraph{The Schrödinger Langevin Equation}
We will now perform a similar consideration as that done in the classical case, namely extending the Schrödinger equation to
\begin{align*}
	i\dv{t}\ket{\Psi} = (H - iW + i\eta(t)V)\ket{\Psi}.
\end{align*}
$W$ is a Hermitian damping term and $\eta$ is a random process such that
\begin{align*}
	\expval{\expval{\eta(t)}} = 1,\ \expval{\expval{\eta(t)\eta\cc(t\p)}} = \delta(t - t\p),
\end{align*}
where we are now referring to averages over the random process rather than quantum mechanical expectation values. This term then corresponds to a noise term. A time step $\delta t$ yields
\begin{align*}
	\ket{\Psi}_{t + \delta t} = \left(1 - (iH + W)\delta t + \integ{t}{t + \delta t}{t\p}{\eta(t\p)V}\right)\ket{\Psi}_{t}.
\end{align*}
Using this we may compute the norm of $\ket{\Psi}$ after this time step according to
\begin{align*}
	\braket{\Psi}_{t + \delta t} =& \expval{\left(1 + (iH - W)\delta t + \integ{t}{t + \delta t}{t\p}{\eta\cc(t\p)V\adj}\right)\left(1 - (iH + W)\delta t + \integ{t}{t + \delta t}{t\p}{\eta(t\p)V}\right)}{\Psi} \\
	                             =& \expval{\left(1 + (iH - W)\delta t\right)\left(1 - (iH + W)\delta t\right)}{\Psi} \\
	                              &+ \expval{\left(1 + (iH - W)\delta t\right)\left(\integ{t}{t + \delta t}{t\p}{\eta(t\p)V}\right)}{\Psi} + \expval{\left(\integ{t}{t + \delta t}{t\p}{\eta\cc(t\p)V\adj}\right)\left(1 - (iH + W)\delta t\right)}{\Psi} \\
	                              &+ \expval{\integ{t}{t + \delta t}{t\p}{\integ{t}{t + \delta t}{t^{\prime\prime}}{\eta(t\p)\eta\cc(t^{\prime\prime})V\adj V}}}{\Psi}.
\end{align*}
Averaging over the random process removes the two terms in the middle. To first order in the time step we therefore have
\begin{align*}
	\braket{\Psi}_{t + \delta t} &= \expval{\left(1 + -2W\delta t\right)}{\Psi} + \expval{\integ{t}{t + \delta t}{t\p}{\integ{t}{t + \delta t}{t^{\prime\prime}}{\delta(t\p - t^{\prime\prime})V\adj V}}}{\Psi} \\
	                             &= \expval{\left(1 - 2W\delta t\right)}{\Psi} + \expval{\integ{t}{t + \delta t}{t\p}{V\adj V}}{\Psi} \\
	                             &= \expval{\left(1 - 2W\delta t\right)}{\Psi} + \delta t\expval{V\adj V}{\Psi},
\end{align*}
and requiring the norm to be preserved by the time step implies
\begin{align*}
	W = \frac{1}{2}V\adj V.
\end{align*}
In summary, the Schrödinger Langevin equation is
\begin{align*}
	i\dv{t}\ket{\Psi} = \left(H - \frac{i}{2}V\adj V + i\eta(t)V\right)\ket{\Psi}.
\end{align*}

\paragraph{The Lindblad Equation}
Using the Schrödinger Langevin equation we may derive a time evolution equation for the density matrix. We have the time evolution operator
\begin{align*}
	U(t, t_{0}) = e^{-i\left(\left(H - \frac{i}{2}V\adj V\right)(t - t_{0}) + i\integ{t_{0}}{t}{t\p}{\eta(t\p)V}\right)},
\end{align*}
and thus
\begin{align*}
	\rho(t) = U(t, t_{0})\rho(t_{0})U\adj(t, t_{0}).
\end{align*}
In particular, for a small time step $\delta t$ we may expand to find
\begin{align*}
	\rho(t + \delta t) =& \left(1 - i\left(\left(H - \frac{i}{2}V\adj V\right)\delta t + i\integ{t}{t + \delta t}{t\p}{\eta(t\p)V}\right)\right)\rho(t)\left(1 + i\left(\left(H + \frac{i}{2}V\adj V\right)\delta t - i\integ{t}{t + \delta t}{t\p}{\eta\cc(t\p)V\adj}\right)\right) \\
	                   =& \rho(t) - i\left(\left(H - \frac{i}{2}V\adj V\right)\delta t + i\integ{t}{t + \delta t}{t\p}{\eta(t\p)V}\right)\rho(t) + i\rho(t)\left(\left(H + \frac{i}{2}V\adj V\right)\delta t - i\integ{t}{t + \delta t}{t\p}{\eta\cc(t\p)V\adj}\right) \\
	                    &+ \integ{t}{t + \delta t}{t\p}{\integ{t}{t + \delta t}{t^{\prime\prime}}{\eta(t\p)\eta\cc(t^{\prime\prime})V\rho(t)V\adj}}.
\end{align*}
Averaging over the noise yields
\begin{align*}
	\rho(t + \delta t) &= \rho(t) - i\left(\left(H - \frac{i}{2}V\adj V\right)\delta t\right)\rho(t) + i\rho(t)\left(\left(H + \frac{i}{2}V\adj V\right)\delta t\right) + \integ{t}{t + \delta t}{t\p}{\integ{t}{t + \delta t}{t^{\prime\prime}}{\delta(t\p - t^{\prime\prime})V\rho(t)V\adj}} \\
	                   &= \rho(t) + i\delta t\left(-\left(H - \frac{i}{2}V\adj V\right)\rho(t) + \rho(t)\left(H + \frac{i}{2}V\adj V\right)\right) + \integ{t}{t + \delta t}{t\p}{V\rho(t)V\adj} \\
	                   &= \rho(t) + \delta t\left(-i\comm{H}{\rho(t)} - \frac{1}{2}\acomm{V\adj V}{\rho(t)} + V\rho(t)V\adj\right),
\end{align*}
and in the limit of infinitesimal time steps we obtain the Lindblad equation
\begin{align*}
	\dv{t}\rho = -i\comm{H}{\rho(t)} - \frac{1}{2}\acomm{V\adj V}{\rho(t)} + V\rho(t)V\adj.
\end{align*}
It may be generalized to a set of interactions with the surroundings by applying a linearity and choosing a basis for the space of operators such that
\begin{align*}
	\dv{t}\rho = -i\comm{H}{\rho(t)} + \sum\limits_{\alpha}V_{\alpha}\rho(t)V_{\alpha}\adj - \frac{1}{2}\acomm{V_{\alpha}\adj V_{\alpha}}{\rho(t)}.
\end{align*}

\paragraph{Quantum Quenching}
We will now develop a Monte Carlo method for solving the Lindblad equation by noting that the jump operators represent jumps whereas the remaining terms in the Schrödinger Langevin equation represent continuous time evolution given by the operator
\begin{align*}
	J = H - \frac{i}{2}\sum\limits_{\alpha}L_{\alpha}\adj L_{\alpha}.
\end{align*}
Time evolving for a short time and assuming no jumps to occur we find
\begin{align*}
	\braket{\Psi}_{t + \delta t} = \braket{\Psi}_{t} + \expval{-iJ\delta t + iJ\adj\delta t}{\Psi} = 1 + i\delta t\expval{J\adj - J}{\Psi} = 1 - \delta t\expval{\sum\limits_{\alpha}L_{\alpha}\adj L_{\alpha}}{\Psi},
\end{align*}
assuming the Hamiltonian to be self-adjoint and the state to initially be normalized. To interpret this, we define
\begin{align*}
	p_{0} = 1 - \delta t\expval{\sum\limits_{\alpha}L_{\alpha}\adj L_{\alpha}}{\Psi}
\end{align*}
and
\begin{align*}
	p_{\alpha} = \delta t\expval{L_{\alpha}\adj L_{\alpha}}{\Psi},
\end{align*}
which satisfy
\begin{align*}
	p_{0} + \sum\limits_{\alpha}p_{\alpha} = 1,
\end{align*}
and our interpretation is thus that $p_{0}$ is the probability of no jump happening and $p_{\alpha}$ being the probability that the jump given by $L_{\alpha}$ happened. This defines a Monte Carlo algorithm to time evolve a quantum state:
\begin{enumerate}
	\item Given some initial state, a system Hamiltonian and jump operators, compute $p_{0}$ and all $p_{\alpha}$.
	\item Get a random number between $0$ and $1$.
	\item If $r < p_{0}$, no jump has occured, meaning that one time evolves using $J$. Otherwise, a jump occured. To determine which jump occured, identify the smallest $\nu$ such that $\sum\limits_{\alpha = 1}^{\nu}p_{\alpha}  > r$ and evolve the state by simply multiplying by $L_{\nu}$.
	\item Normalize the state.
\end{enumerate}
The Monte Carlo approach consists of performing a set of such evolutions from some initial state and compute the density matrix from a statistical overage over different runs.

\paragraph{Quantum Quenches}
In a quantum quench, one prepares a system in an eigenstate $\ket{\Psi_{0}}$ of some Hamiltonian $H_{0}$, and then switch the Hamiltonian to $H_{1}$ and let the system evolve. These two Hamiltonians generally do not share eigenstates, hence the system has non-trivial dynamics. This switch can be performed globally or locally.

\paragraph{A Conundrum of Thermalization}
Quantum mechanics preserves the purity of the density matrix, but the thermalized density matrix $\rho = \frac{1}{Z}e^{-\beta H}$ is always mixed. How, then, can quantum mechanics produce thermalization? It will turn out that this is the same problem as the one we faced with classical mechanics.

Von Neumann provided a suggestion for how to handle this in a previously-forgotten paper. In it he suggested that states themselves do not thermalize, but that expectation values of local observables do. This would require that the reduced density matrix for a system in consideration approaches that predicted by statistical mechanics.

\paragraph{Random Matrix Theory}