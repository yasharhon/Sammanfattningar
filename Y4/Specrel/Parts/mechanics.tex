\section{Relativistic Mechanics}

\paragraph{$4$-Momentum and $4$-Force}
To formulate Newton's laws in a relativistic manner, we start with Newton's second law. We try extending it to relativistic mechanics by introducing the $4$-momentum
\begin{align*}
	p^{\mu} = m_{0}u^{\mu}
\end{align*}
for some scalar $m_{0}$, as well as the $4$-force
\begin{align*}
	F^{\mu} = \dv{\tau}p^{\mu}.
\end{align*}

\paragraph{A First Postulate}
Any laws of classical mechanics must result from these definitions in the limit of $\gamma_{u}$ approaching $1$. In this limit we have $P^{\mu} = (m_{0}c, m_{0}\vb{u})$, imploring us to recognize $m_{0}$ as the mass of the particle measured at rest and thus the space components to be those of the classical momentum.

To generalize mechanics we thus postulate that $p^{\mu}$ is conserved. This will lead to the conservation of both the spatial components $m_{0}\gamma_{u}\vb{u}$, which is termed the relativistic $3$-momentum $\vb{p}$, and the conversion of $m_{0}\gamma_{u}\vb{u}$, to be discussed.

\paragraph{Relativistic Energy}
In the classical limit we obtain
\begin{align*}
	m_{0}\gamma_{u}c^{2} \approx m_{0}c^{2} + \frac{1}{2}m_{0}u^{2},
\end{align*}
imploring us to define relativistic kinetic energy as
\begin{align*}
	T = m_{0}c^{2}(\gamma_{u} - 1)
\end{align*}
and relativistic total energy as
\begin{align*}
	E = m_{0}\gamma_{u}c^{2}.
\end{align*}

Having done this, we may generally write $p^{\mu} = \left(\frac{E}{c}, \vb{p}\right)$. In particular, we obtain in the rest frame that $p_{\mu}p^{\mu} = m_{0}c$, and thus
\begin{align*}
	E^{2} = (m_{0}c^{2})^{2} + (c\vb{p})^{2}.
\end{align*}

\paragraph{Potential Energy}

\paragraph{A Note on Relativistic Mass}
I make a brief note of how the above is modified by defining the relativistic mass $m = m_{0}\gamma_{u}$. In this case you would obtain $E = mc^{2}$ and $\vb{p} = m\vb{u}$. In this context $m_{0}$ is termed the rest mass. This way of doing this is generally not preferred in modern contexts.

\paragraph{Massless Particles}
The expression for the energy may be extended to massless particles. In these cases we have $E = \abs{\vb{p}}c$.

\paragraph{de Broglie}
de Broglie noted that photons, which had been discovered to be particles, could be attributed a momentum $\abs{p} = \hbar\abs{\vb{k}}$ as a consequence of the above. He extended this to all particles, revealing the wave-particle duality of matter to the world.