\section{Electrodynamics}

\paragraph{Transformation of Fields}
By applying a Lorentz boost to the Faraday tensor, one obtains
\begin{align*}
	\vb{e}_{\parallel}\p = \vb{e}_{\parallel},\ \vb{e}_{\perp}\p = \gamma(\vb{e}_{\perp} + c\vb*{\beta}\times\vb{b}),\ \vb{b}_{\parallel}\p = \vb{b}_{\parallel},\ \vb{b}_{\perp}\p = \gamma(\vb{b}_{\perp} - \frac{1}{c}\vb*{\beta}\times\vb{e}).
\end{align*}

\paragraph{The Dual Field Strength}
We may construct a dual field strength $\tilde{F}^{\mu\nu} = -\frac{1}{2}\varepsilon^{\mu\nu\rho\sigma}F_{\rho\sigma}$, where $\varepsilon$ is the completely antisymmetric tensor with the convention $\varepsilon_{0123} = 1$.

\paragraph{Invariants From the Field Strength}
We may now construct Lorentz scalars by contracting indices of the field strength and its dual. We have:
\begin{align*}
	\frac{1}{2}F^{\mu\nu}F_{\mu\nu} = \frac{1}{2}\tilde{F}^{\mu\nu}\tilde{F}_{\mu\nu} = c^{2}\vb{b}^{2} - \vb{e}^{2},\ \frac{1}{2}F^{\mu\nu}\tilde{F}_{\mu\nu} = c\vb{e}\cdot\vb{b}.
\end{align*}

\paragraph{Maxwell's Equations}
We will now try to formulate Maxwell's equations using the introduced formalism. We start by introducing the $4$-current $J^{\mu} = (c\rho, \vb{j})$. We then have $\del{\mu}{J^{\mu}} = \del{t}{\rho} - \div{\vb{j}}$. Maxwell's equations imply that this quantity is zero, and we will take that to be true. We also take the $4$-current to be a $4$-vector.

A general $4$-current may be written as
\begin{align*}
	J^{\mu} = \sum\limits_{i}(c\rho_{i}, \rho_{i}\vb{u}_{i}),
\end{align*}
where the charge densities $\rho_{i}$ are defined in frames such that the corresponding current densities are zero.

We try to recreate Maxwell's equation with an anzats
\begin{align*}
	\del{\mu}{F^{\mu\nu}} = kJ^{\nu},\ \del{\mu}{\tilde{F}^{\mu\nu}} = kJ_{\text{m}}^{\nu}.
\end{align*}
$J_{\text{m}}$ is a quantity that will correspond to magnetic charge and current, but as such a thing has not been found in nature, this quantity must be zero. $k$ is a constant that makes the dimensions match those of Maxwell's equation, and turns out to be $\frac{1}{c\varepsilon_{0}}$. Note that as the field strength is antisymmetric, this form guarantees charge conservation. This turns out to be the correct form of Maxwell's equations.

\paragraph{The $4$-Potential}
We define the $4$-potential as $\Phi^{\mu} = (c\phi, \vb{a})$, where $\phi$ is the electric potential and $\vb{a}$ is the magnetic potential. Using this, the Maxwell equation $\del{\mu}{\tilde{F}^{\mu\nu}} = 0$ is automatically satisfied. With this particular choice of $4$-potential, the other will as well. Namely, we infer from the above that 
\begin{align*}
	F_{\mu\nu} = \del{\mu}{\Phi_{\nu}} - \del{\nu}{\Phi_{\mu}},
\end{align*}
and the other Maxwell equation takes the form
\begin{align*}
	\del{\mu}{\del[\mu]{}{\Phi^{\nu}}} - \del{\mu}{\del[\nu]{}{\Phi^{\mu}}} = \frac{1}{c\varepsilon_{0}}J^{\nu}.
\end{align*}

\paragraph{Gauge}
The choice of $4$-potential is not unique - namely, if two $4$-potentials differ by a term $\del{\mu}{\Psi}$, then they produce the same field strength. This means that there is a so-called gauge degree of freedom in the choice of the $4$-potential.

\paragraph{The Lorentz Gauge}
The Lorentz gauge is the gauge such that $\del{\mu}{\Phi^{\mu}} = 0$.

\paragraph{Solutions in Vacuum}
In vacuum and in the Lorentz gauge the second set of Maxwell equations is simply $\del{\mu}{\del[\mu]{}{\Phi^{\nu}}} = 0$, with the solution
\begin{align*}
	\Phi^{\mu} = \varepsilon^{\mu}e^{-ik^{\nu}x_{\nu}},
\end{align*}
where $k^{\mu}$ is necessarily a light-like wavevector. Reinserting this into the definition of the Lorentz gauge we obtain $k^{\mu}\varepsilon_{\mu} = 0$. The vector $\varepsilon^{\mu}$ is called the $4$-polarization, and it may always be constructed in a given frame such that $\varepsilon^{0} = 0$.

\paragraph{General Solutions}
In vases where there are sources present, the general solution is constructed using the Green's function
\begin{align*}
	G_{x_{0}^{\mu}}(x^{\mu}) = \frac{1}{4\pi\abs{\vb{r} - \vb{r}_{0}}}\delta\left(t - t_{0 - }\frac{1}{c}\abs{\vb{r} - \vb{r}_{0}}\right).
\end{align*}

\paragraph{Fields from Uniformly Moving Charges}
From a uniformly moving charge moving at speed $v$ in the $x$-direction one can find
\begin{align*}
	\Phi^{0} = \frac{\gamma_{v}q}{4\pi\varepsilon_{0}r\p},\ \Phi^{1} = \frac{\beta_{v}\gamma_{v}q}{4\pi\varepsilon_{0}r\p},\ r\p = \sqrt{\gamma_{v}^{2}(x - vt)^{2} + y^{2} + z^{2}}.
\end{align*}
The corresponding electric and magnetic field is
\begin{align*}
	\vb{e} = \frac{q}{4\pi\varepsilon_{0}r\p}(\vb{r} - \vb{v}t),\ \vb{b} = \frac{1}{c^{2}}\vb{v}\times\vb{e}.
\end{align*}

\paragraph{Maxwell's Equations From a Variational Principle}
It can be shown that the Lagrangian density
\begin{align*}
	\lag = -\frac{1}{4}\varepsilon_{0}F^{\mu\nu}F_{\mu\nu} - \frac{1}{c}\Phi_{\mu}J^{\mu}
\end{align*}
reproduces Maxwell's equation.

\paragraph{The Electromagnetic Energy Tensor}
In the case of continuous test charge distributions, the distribution experiences a force described by a force density
\begin{align*}
	K_{\mu} = \frac{1}{c}F_{\mu\nu}U^{\nu},
\end{align*}
where $U$ is the $4$-velocity field. Furthermore, it can be shown that $K_{\mu} = -\del{\nu}{\tensor{M}{_{\mu}^{\nu}}}$ for
\begin{align*}
	\tensor{M}{_{\mu}^{\nu}} = -\varepsilon_{0}F_{\mu\sigma}F^{\nu\sigma} + \frac{1}{4}\kdelta{\nu}{\mu}F^{\rho\sigma}F_{\rho\sigma}.
\end{align*}
When elevating indices to the same height, we find that this tensor is symmetric.