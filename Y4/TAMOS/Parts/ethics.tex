\section{Ethics}

\paragraph{Morality}
Engaging with morality involves asking oneself questions about what one may or ought to do. The answers may be permissions, prohibition and obligations.

The descriptive view of morality describes it as a code of conduct accepted by a society, group or individual. This view is limited, however, as it carries no natural distinction from, for instance, etiquette, law and instrumental rationality. Furthermore, it carries a high degree of relativism.

The normative view of morality, which will be the focus of the rest of the discussion, views morality as a code of conduct that all rational persons would endorse.

\paragraph{Ethics}
Ethics is a theory that offers normatively valid reasons for rationally endorsing some code of behaviour. Depending on the choice of ethical theory, people might agree on a particular choice but disagree on why this is the correct choice.

\paragraph{Consequentialism}
Consequentialism is an ethical framework in which moral assessment is based on the consequences of the action. This begs the question of how to assess the consequences of an action. Subtheories of consequentialism specify this further, utilitarianism being a noteworthy example.

Some problems with consequentialism include:
\begin{itemize}
	\item It is extremely demanding, requiring all possible consequences to be evaluated. As a consequence of the complexity, all actions are seemingly either required or forbidden.
	\item It permits some intuitively wrong actions.
\end{itemize}

\paragraph{Deontology}
Deontology is an ethical framework in which an action is moral if it fulfills relevant rules or duties.

Compared to consequentialism, deontology is simpler, includes the morally gray and the amoral and incorporates the preference of the agents into ethical considerations. However, its issues include:
\begin{itemize}
	\item The possibility of allowing disastrous consequences in the name of principle.
	\item The fact that rights and duties are categorical, limiting flexibility.
\end{itemize}

\paragraph{Virtue Ethics}
Virtue ethics asserts that morality consists of having and exemplifying certain good character traits, or virtues.

Compared to consequentialism, it lacks the overdemanding nature, and compared to deontology, it explains the motivation for people to be moral. This system has problems too, however, including:
\begin{itemize}
	\item Providing no clear guidance for how to act in any particular situation.
	\item Making it hard to resolve conflicts.
\end{itemize}

\paragraph{Morals and Experimental Design}
In experimental design moral questions often arise, in particular when the subjects are human. What is important to maintain the morality of the study and how important different aspects are will naturally vary according to different ethical frameworks.

\paragraph{Informed Consent}
Informed consent of a human subject to being experimented upon is consent that is
\begin{itemize}
	\item informed - the subject has received all relevant info about the purpose of the project, how it will be carried out and the effects on themselves.
	\item voluntary - not forced and sufficiently free from negative influence.
	\item decisionally-capacitated - the subject can assess the provided information, appreciate how it concerns them and make and communicate their decision.
\end{itemize}

According to a consequentialist, informed consent might be important because individuals know best what is good for them and because it is good for science as an institution to establish and maintain trust. According to a deontologist, one should treat others as autonomous beings and not lie. According to a virtue ethicist, it is good to show sincerity, reflexivity and respectfulness.

\paragraph{Scientific Misconduct}
Scientific misconduct may manifest as
\begin{itemize}
	\item fabrication - making up data or results.
	\item falsification - manipulating materials, equipment or process, or changing or omitting data or results, such that the research is not accurately presented.
	\item plagiarism - appropriation of other people's ideas, processes, results or words without credit.
\end{itemize}

A consequentialist might argue that misconduct undermines the trust of the institution of science or wastes resources. A deontologist might argue that lying violates some fundamental principle. A virtue ethicist might argue that the moral path would be to show sincerity and humility, and thus not perform such acts.

\paragraph{Whistleblowers}
A whistleblower is someone who exposes scientific misconduct.

\paragraph{Authorship}
A person is assigned authorship of a work if they substantially contributed to the conception of the work, drafted the final work, gave final approval and agreed to be held accountable to all aspects of the work.

Why is the proper assignment of authorship important? A consequentialist might argue that the impartiality of science, public trusti in science, incentives for authors and the saving of resources are important consequences of this. A deontologist might argue that proper authorship preserves intellectual property rights and properly assigns responsibility. A virtue ethicist might argue that proper authorship reflects and allows authors to display sincerity, humility, resoluteness and reflexivity.

\paragraph{Ghost Authors and Gift Authors}
A ghost author is someone who contributes to a work according to the authorship criteria but is not listed as an author. The opposite is a gift author, or honorary author.

\paragraph{The Precautionary Principle}
The precautionary principle states that if an activity raises threats to human health or the environment, precautionary measures should be taken even if all causal relations are not established.

\paragraph{Consequentialism and Risk}
Consequentialism has a few approaches to the issue of risk. One is the actualist approach, which measures the goodness of an action solely based on the actual outcome. This approach is clearly insufficient. Another is the maximization of expected utility, defined as the probability,weighted average of the utility.

Criticisms of the consequentialist approach include:
\begin{itemize}
	\item one might take it as important to avoid disproportionate disaster, but the consequentialist approach provides no obvious way to take this into account.
	\item one might want to avoid disproportionately exposing certain individuals, but this does not emerge naturally from consequentialism.
\end{itemize}

\paragraph{Deontology and Risk}
When assessing risk using deontology, a fundamental issue is how small a probability of a risk violation actually counts as a violation. This issue is extremely complicated, some complications including distinctions between intentional and unintentional risks, voluntary and imposed risks, self-produced and externally produced risks and the existence and absence of benefits.