\section{General Philosophy of Science}

\paragraph{Lexical Definitions}
A lexical definition is a definition based on a common understanding of a word or phrase.

\paragraph{Stipulative Definitions}
A stipulative definition is a definition based on reasons and arguments for choosing it in a particular way.

\paragraph{Knowledge}
Knowledge is true, justified belief. These three conditions are necessary, but not always sufficient. We will nevertheless proceed with this definition.

This definition is compatible with uncertainty, and recognizing uncertain knowledge is in fact vital in science. Naturally there are degrees of uncertainty.

In science, the justification of belief is central in the social aspects of its practice. An ideally justified belief requires consideration of all relevant reasons for that belief, and science is publically defended exactly to adhere to this ideal.

\paragraph{Definitions Pertaining to Phenomena}
A phenomenon $X$ is
\begin{itemize}
	\item predicted if one can know at what time $X$ will occur.
	\item explained if the causes of $X$ are known.
	\item designable if it is known to satisfy certain properties.
\end{itemize}

\paragraph{Instrumentalism and Realism}
Do scientific theories constitute claims that are true or false? Realists say yes, whereas instrumentalists say that theory only orders observations and do not in themselves contain truth.