\section{General Philosophy of Science}

\paragraph{Definitions}
Definitions explain the meaning of words. They consist of a definiendum (what is to be defined), a definiens (that which defines the definiendum) and a defining connective (which describes how the two are connected. Words such as ``is'' as well as logical comparisons are in this category). In addition, there may be delimiters which explain that the definiendum is an aspect of something else. For instance, in the definition ``A system is spherically symmetric if it looks the same from all directions,'' ``spherically symmetric'' is the definiendum and ``A system is'' is the delimiter.

\paragraph{Lexical Definitions}
A lexical definition is a definition based on a common understanding of a word or phrase.

\paragraph{Stipulative Definitions}
A stipulative definition is a definition based on reasons and arguments for choosing it in a particular way.

\paragraph{Ambiguity}
Ambiguity is the presence of two possible meanings of the same word.

\paragraph{Vagueness}
Vagueness is the unclarity of the limits of what is within the scope of a definition.

\paragraph{Narrowness and Broadness}
A definition is too narrow if it does not apply to something that it should, and too broad if it applies to something that it should not.

\paragraph{Knowledge}
Knowledge is true, justified belief. These three conditions are necessary, but not always sufficient. We will nevertheless proceed with this definition.

This definition is compatible with uncertainty, which is important to us as recognizing uncertain knowledge is vital in science. Naturally there are degrees of uncertainty, and our idea of knowledge should capture that.

In science, the justification of belief is central in the social aspects of its practice. An ideally justified belief requires consideration of all relevant reasons for that belief, and science is publically defended exactly to adhere to this ideal.

\paragraph{Inference}
Inference is the logical process of going from a set of facts and/or evidence to some conclusion.

\paragraph{Inductive and Deductive Inference}
Inductive reasoning is the process of going from a set of observations to a general claim. Such reasoning is an attempt at extending our knowledge beyond what can be directly observed. Deductive reasoning is the process of explicating or extracting knowledge about something particular from previously possessed knowledge.

Inductive reasoning is fallible - even the best of efforts at induction might still overlook something for which one has no evidence. Deductive reasoning, on the other hand, is infallible - deduction necessitates that you have sufficient knowledge, and proper reasoning based on sufficient knowledge preserves the truth of your statements.

\paragraph{Examples of Deduction and Induction}
A few ways of inductive inference are
\begin{itemize}
	\item Direct inference, where a property is extrapolated from a finite set of observations.
	\item Projection, where a sequence or trend is used to extrapolate a future observation.
	\item Generalization, where a hypothesis about a population is formed based on a finite set of observations.
\end{itemize}

A few ways of deductive inference are
\begin{itemize}
	\item Modus ponens, in which a set of assumptions $A$ is combined with the idea that if $A$ is true then their implications $B$ are true to infer that $B$ is true.
	\item Modus tollens, in which a conditional $A\to B$ is combined with the observation that $B$ is false to infer that $A$ is false. 
\end{itemize}

\paragraph{Ill-Formed Inferences}
An ill-formed inference is a logically inconsistent inference.

\paragraph{The Goals of Science}
The goals of science are to predict, explain and design phenomena. A phenomenon $X$ is
\begin{itemize}
	\item predicted if one can know at what time or in what way $X$ will occur.
	\item explained if the causes of $X$ are known.
	\item designable if it is known to satisfy certain properties.
\end{itemize}

\paragraph{Instrumentalism and Realism}
Do scientific theories constitute claims that are true or false? Realists say yes, whereas instrumentalists say that theory only orders observations and do not in themselves contain truth.

\paragraph{Hume's Problem}
As there exists an infinite set of inference rules, in particular inductive inference rules, the conclusions of any particular inference are only justified if the choice of inference rule is justified. Hume's problem discusses the justification of inference rules.

Hume's argument was the following:
\begin{enumerate}
	\item Assume that inference is either deductive or inductive.
	\item Assume that in order to justify an inference rule, it has to be inferred from some set of principles.
	\item Assume that inference rules cannot be deduced, as past and previous inference need not be connected,
	\item Thus, inference rules are arrived at inductively.
\end{enumerate}
The implication of the latter statement is that there must be a set of inference rules allowing you to infer the validity of the rule you want use. But this set of inference rules is not necessarily valid in itself. The result is a so-called infinite regress of the same problem of justification, which according to Hume could only be resolved if no inference rule is justified.

\paragraph{Solving Hume's Problem}
Hume's problem would imply that the practice of science is itself irrational. This would be a complete disaster, and must therefore be resolved.

One way to solve this came from Karl Popper, who denied the premise that science rests on using inference rules. This will be discussed later.

Another solution is to deny the impact of Hume's argument by studying what offers justification, in the hope that this is not destroyed by the infinite regress. Foundationalism is one attempt at this that asserts that there exists a foundation upon which justification is built. Coherentism is an alternative, which desires that claims be justified from some coherent system containing the set of previously accepted claims. In this view, inference rules are good if they are coherent with the rest of scientific knowledge. The result is that inference rules are merely manifestations of coherence and a way to connect different practices in a coherent manner.

\paragraph{The Hypothetico-Deductive Method}
The hypothetico-deductive method consists of
\begin{enumerate}
	\item Formulating a hypothesis.
	\item Deducing consequences of the hypothesis.
	\item Testing whether the consequences are true.
	\item If the consequences are false, reject the hypothesis deductively. Otherwise, strengthen your confidence in the hypothesis inductively.
\end{enumerate}

\paragraph{Quality Criteria for Hypotheses}
A good hypothesis must be
\begin{itemize}
	\item Either true or false.
	\item Not a tautology, i.e. something that is necessarily true or false by definition.
	\item Contain some generalization or discussion of the unobservable.
\end{itemize}

A good set of consequences of a hypothesis must be
\begin{itemize}
	\item Observable.
	\item Follow from valid deduction.
	\item Relevant for the hypothesis. This part informs what kinds of experiment you want to do. For instance, if the hypothesis is $A$ implies $B$, then you would need to find cases where $A$ is true and $B$ is false at the same time to reject the hypothesis. In other words, studying cases where $A$ is false or $B$ can at best be a tool for verifying the validity of the deductions made.
\end{itemize}

\paragraph{Falsifiability and Falsification}
A theory or hypothesis is falsifiable if it has implications the truth or falsehood of which can be observed. The act of observing an implication to be false is called falsification.

\paragraph{Popper's Falsificationism}
Popper attempted to circumvent Hume's problem by rejecting the ability of science to strengthen hypotheses, leaving it only with the ability to deductively falsify them. While this seems to solve the problem, some critiques of this solution are:
\begin{itemize}
	\item It provides no way to give hypotheses confidence, in contrast to what seems reasonable and what is scientific practice.
	\item Along with an observation comes auxillary hypotheses about the validity of the experiment. A particular falsification can only falsify the conjuction of the fundamental hypothesis and the auxillary hypotheses, limiting our ability to perform modus tollens on the fundamental hypothesis. This is called the Duhem-Quine thesis.
	\item Hypotheses may be modified ad hoc solely to fit particular observations. Popper countered this by claiming that modifications that reduce falsifiability are ad hoc and should not be used, in contrast to other modifications, which might be necessary.
\end{itemize}

\paragraph{Confidence}
Confidence in a hypothesis may be described either qualitatively or quantitatively.

\paragraph{Bayesianism and Frequentism}
A possible qualitative approach is to assign a probability of truth to a hypothesis and try to find it from observations combined with the probability of these observations given the hypothesis. This way of assigning probabilities is characteristic of the Bayesian approach to probability.

The frequentist view of probability, on the other hand, is that it is the frequency with which some event occurs. As hypotheses are not events, it follows from this view that probabilities cannot be meaningfully assigned to hypotheses, limiting the possibility of quantitatively strengthening confidence.

\paragraph{Strengthening Confidence}
How can our confidence in a hypothesis be strengthened, and why does observations of consequences of a hypothesis increase our confidence in the hypothesis itself?

One answer might be found in the compatibility of the hypothesis and its consequences. However, this is not strong enough on its own. Relevance is also needed - otherwise, many observations might seemingly verify far-reaching conclusions.

The issue of under-determination is also prominent, as multiple hypotheses might be compatible with the same observations. Given the observation of some consequences, the question then becomes which hypotheses should have strengthened confidence. One solution to this, often used in science, is Occam's razor, where the simplest hypothesis is selected. Another solution is to assign some relevance to the consequences of a hypothesis and only strengthen one's confidence upon the observation of relevant consequences.

Another requirement, stronger than the one previously presented, is that the conclusions be unlikely if the hypothesis is false. A test such that a given consequence is unlikely if the hypothesis is false is called severe testing. While this is an improvement, one must first of all avoid double-counting evidence - if you use some piece of evidence to construct a hypothesis, you cannot use that evidence to verify your hypothesis. In addition, if there is low confidence in the hypothesis to begin with, severe testing will not sufficiently strengthen it.

\paragraph{Observation}
Loosely speaking, observation is the event of experience through the senses. It is key to all scientific practice. Observations can be divided into categories:
\begin{itemize}
	\item Direct observation, which is unaided sensory experience.
	\item Aided direct observation, which uses instruments to amplify the senses.
	\item Indirect observation, in which an event is not directly observed and the event is experienced through its effect on its surroundings.
\end{itemize}

\paragraph{Empiricism}
Empiricism is the thesis that sensory knowledge is the ultimate basis for knowledge.

\paragraph{Logical Empiricism and its Refutation}
Logical empiricism is the position that theory and experiment can be separated, that is that theory does not inform experiment. If this were true, then any scientific theory would rest on a solid and independent foundation of experimental observation.

This position has been refuted. Its refutation is as follows: First of all, indirect observation always depends on theory, as theory dictates how to interpret how the indirect observations pertain to the underlying phenomena. Next, aided direct observation sometimes requires theory as instruments may distort the phenomena they are used to observe, and separating between instrumental effects and actual phenomena requires theory. Thus theory and observation cannot be separated.

\paragraph{The Problem of Nomic Measurement}
The circular relation between theory and observation is formulated in the problem of nomic measurement. Its statement is as follows:

\begin{itemize}
	\item Assume that you want to measure property $X$, which is not directly observable.
	\item Assume that $X$ may be inferred from $Y$, which is directly observable.
	\item For the inference to be performed, one needs some relation $X = f(Y)$. This relation cannot be obtained experimentally obtained as it would require $X$ and $Y$ to be known at the same time, violating the assumption.
	\item As $f$ cannot be identified from measurement alone, it must come from theory. Thus theory and experiment are inextricably connected.
\end{itemize}

\paragraph{Operationalization}
Direct observation, while powerful in a sense, is generally error-prone and can often only give qualitative descriptions, hence indirect observational methods are usually preferred. The process of linking properties (a so-called property of interest) to directly observable effects is called operationalization of that property. If the two are linked by some stable relation (a so-called hypothesized causal chain), that guarantees that when the property is present, so is the observed effect. Using operationalization we may infer information about the property from observing the effect.

\paragraph{Quality Criteria for Operationalizations}
The quality criteria for an operationalization are
\begin{itemize}
	\item That the underlying property be well-defined in order to allow for proper inference.
	\item That the relation between the property and the effect is valid.
	\item That the relation is sufficiently stable.
	\item That the effect is publically observable with sufficient precision.
\end{itemize}

\paragraph{Operationalism}
Operationalism is the position that everything is defined through the operations by which  we observe or measure them. Its main appeal is to remove the issue of theory-dependence.

\paragraph{Criticisms of Operationalism}
The criticisms of operationalism are:
\begin{itemize}
	\item Operationalism prohibits two differently performed measurements to pertain to the same underlying concept.
	\item Operationalism makes it impossible to criticize a measurement for not properly capturing an underlying concept.
\end{itemize}

\paragraph{Measurement}
The process of measurement consists of the following:
\begin{enumerate}
	\item Define the concept that you would like to measure.
	\item Operationalize the concept.
	\item Specify a measure and define units of comparison.
	\item Represent the results with numbers.
\end{enumerate}

\paragraph{Quality Criteria for Measurements}
The quality criteria for a measurement are:
\begin{itemize}
	\item There must be a unit of comparison.
	\item The unit must be sufficiently stable.
	\item Anyone must have access to the same measure.
\end{itemize}

\paragraph{Scales}
The representation part of a measurement implies the need for scale. There are five kinds of scales:
\begin{enumerate}
	\item Nominal scales, in which samples are given ID numbers.
	\item Ordinal scales, in which objects are ordered according to some qualitative measure.
	\item Interval scales, which allow the comparison of distances.
	\item Ratio scales, which allow the comparison of both distances and ratios.
	\item Absolute scales, which allow the comparison of absolute numbers, i.e. where the absolute numbers have a significance by themselves.
\end{enumerate}

Different scales of the same type made to represent the same property must represent the same empirical structure. Each set of scales is defined by the set of allowed transformations, and combined with the previous this implies that what can be inferred from a measurement is limited by the allowed transformations. The allowed transformations are:
\begin{itemize}
	\item Transformations that preserve uniqueness for nominal scales.
	\item Positive monotone transformations for ordinal scales.
	\item Positive linear transformations for interval scales.
	\item Positive scaling transformations for ratio scales.
	\item None for absolute scales.
\end{itemize}
Note that any scale also allows the transforms below where it appears in the list.

\paragraph{Measurement Error}
The measurement error is defined as the difference between the measured and true value of some property. It comes in two forms: Systematic and random error.

Before introducing the two, we introduce the notions of precision and accuracy. Precision is the absence of variation, and accuracy is closeness to the true value. Systematic errors are thus inaccuracies, and random errors are imprecision.

\paragraph{Convergent and Divergent Validity}
Convergent validity is achieved if different methods with causally independent operationalizations measure the same target under the same conditions and yield the same result. Divergent validity is achieved if an operationalization applied to substantially different targets under different conditions yield different results.

\paragraph{Experiments}
An experiment is a controlled observation in which the observer manipulates the real variables that are believed to influence the outcome, both for the purpose of intervention and control. Its purpose is to justify accepting or rejecting a hypothesis. Its characteristics are:
\begin{itemize}
	\item manipulation.
	\item intervention with the independent variables.
	\item control of disturbing factors.
	\item observation.
\end{itemize}

\paragraph{Mill's Method of Difference}
Mill's method of difference describes the logic and procedure of experiments. It has the following steps:
\begin{enumerate}
	\item Ask what causes a phenomenon $E$.
	\item Conjecture that $C$ is the cause.
	\item Produce situations $S_{1}$ and $S_{2}$ in which neither $C$ nor $E$ occur and such that all relevant causal factors are the same.
	\item Activate $C$ in $S_{1}$ only.
	\item Observe $E$ in $S_{1}$ only.
	\item Assert that something causes $E$ in $S_{1}$ and that nothing causes $E$ in $S_{2}$.
	\item As the two only differ by $C$, conclude that $C$ causes $E$.
\end{enumerate}

Baked into these steps are a number of assertions and ideas which might be difficult to realize - for instance, it might be hard to know what factors are causal. This necessitates good design.

\paragraph{Repeatability, Reproducibility and Replicability}
If an experiment is described such that it contains enough information that others can repeat it, the experiment is repeatable. If a competent repetition of the experiment yields the same result, the experiment is reproducible. If a competent independent experiment using independent data, methods and experimental infrastructure yields the same results, the experiment is said to be replicable. These categories will be important when looking for errors in experimental design.

\paragraph{Internal and External Validity}
If the relation between intervention and observed effect inferred from an experiment is true and not clouded by uncontrolled background factors, it is internally valid. If such an inference also holds for different targets and not just the same experiment, it is externally valid.

\paragraph{Models}
Models are important tools in science used to help us describe the world. They have different aspects, and may be approached through any of them. These are:
\begin{itemize}
	\item Representations, where the models attempt to represent some target in the real world to some extent. This is useful when studying the real-world target might be impossible or infeasible, morally or ethically prohibited or cognitively difficult.
	\item Idealizations, where the model contains only the relevant aspects of the real-world target. Models are generally analogies of the real-world target. These may be positive (containing similar aspects), negative (containing idealizations) or neutral (containing descriptions of things that cannot be known in the target).
	\item Purpose-dependent tools - some models describe certain properties better than others.
	\item Things to be manipulated. The kind of analogy provided by the model limits the available set of manipulations - if some property is idealized away in the model, then investigating that property in the model has no bearing on the target. Manipulations may also be used to reveal neutral analogies as either positive or negative.
\end{itemize}

\paragraph{Quality Criteria for Models}
Models do not necessarily have quality criteria that apply equally well in all contexts, and in fact these often trade off with one another. Nevertheless, some criteria are:
\begin{itemize}
	\item Accuracy. A model is accurate if it is similar to the target with respect to relevant properties.
	\item Robustness. A model is robust with respect to some condition if changing it does not change the result.
	\item Parameter precision. A model has higher parameter precision than another if the specification of the former implies the specification of the latter.
	\item Simplicity. A model is simpler than another if it contains fewer variables and parameters than the other.
	\item Tractability. A model is tractable with respect to some principles if the results can be obtained by applying the set of principles to the model.
	\item Transparency. A model is transparent if the user is capable of understanding how the results are produced.
\end{itemize}

\paragraph{Models and Experiments}
Models and experiments are similar in that they both contain a set of variables and parameters, as well as both entailing the manipulation of something and the observation of the effect. However, they differ in the errors that can be made. The greatest issues with experiments concern internal validity, whereas the greatest issues with models concern whether the relevant analogies of the model hold.

\paragraph{The Structure of Explanations}
Explanations consists of an explanandum - what is to explained - and one or more explanans - statements that increase understanding. An important ingredient in explanations is contrasting the explanandum to some other scenario, and this contrast might affect the answer you seek or obtain.

\paragraph{Scientific Knowledge and Explanations}
The process of explanation is vital in fulfilling the goals of science - namely:
\begin{itemize}
	\item making predictions by providing reasons for expecting a phenomenon to occur in a particular way or at a particular time.
	\item explaining phenomena by providing lawful reasons why a phenomenon occurs.
	\item designing phenomena by providing reasons for expecting why a certain manipulation satisfies certain functions.
\end{itemize}
Our description of explanation seemingly makes it very similar to prediction - one looks to the future, the other to the past.

\paragraph{Understanding}
Understanding a phenomenon can be taken to be the ability to answer questions about what would happen if things were different. The process of understanding thus involves tracing so-called productive relations, which describes what features of a system produces what others.

\paragraph{Causation}
We try to clarify the notion of causation by defining some associated terms.

$X$ is a direct cause of $Y$ with respect to a set of background variables $V$ if and only if there exists an intervention on $X$ that changes $Y$ with $V$ held constant. Built into this notion is the idea that $X$ causes $Y$ with respect to some background variables, which is a model. Therefore one cannot say that $X$ causes $Y$ generally.

$X$ is a contributing cause of $Y$ with respect to a set of background variables $V$ if and only if there exists a causal chain comprised of direct causes extending from $X$ to $Y$.

\paragraph{The Deductive-Nomological Account}
The deductive-nomological account is way of providing an explanation. It answers the question of why some particular phenomenon occurred by deducing it from a set of natural laws and circumstances.

\paragraph{Critiques of the Deductive-Nomological Account}
If we modify our concept of explanation to be the process of providing understanding, we see that explanation must entail identifying productive relations. This is a better notion of explanation because it helps us distinguish between explanation and prediction. The deductive-nomological account does not distinguish between the producer and what is produced, and is therefore insufficient to provide explanations. Furthermore, the deductive-nomological account does not identify the relevance of the different circumstances.

Another critique comes from considering so-called singular causal explanations. These are explanations based purely on recounting a sequence of events. In many practical contexts such explanations are sufficient, but according to the deductive-nomological account they are not. Hence the deductive-nomological account does not cover everything that we would like to consider an explanation.

\paragraph{Causal Explanations and their Quality Criteria}
Causal explanations are explanations that identify difference-making contributing causes to an explanandum.

\paragraph{Quality Criteria for Causal Explanations}
The quality criteria of causal explanations are:
\begin{itemize}
	\item Accuracy - the ability of the explanans to describe the state or properties. The explanans only needs to identify the difference-making contributions.
	\item Precision - the more precisely the explanandum states a contrast, the better.
	\item Difference-making - the explanans must identify all contributing causes.
	\item Non-sensitivity - the explanans must have low sensitivity to background causes.
	\item Cognitive salience - the explanation should be fit to its audience. This typically means making it simpler.
\end{itemize}