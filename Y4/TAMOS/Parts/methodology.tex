\section{Methodology}

Methodology is the study of the design of methods for gaining knowledge. It is different from general philosophy of science in that it does not contain, for instance, comparison, choice and justification of methods.

\paragraph{Choice of Method}
To answer the question of what methods should be chosen, three approaches are typically found:
\begin{itemize}
	\item The conventional approach, in which you choose the same methods as your teachers or peers.
	\item The outcome-oriented approach, in which you choose the method that gives the best results.
	\item The reason-based approach, in which you choose the method for which you have the overall best reasons. Reasons here means considerations of the method with respect to a set of goals that you want to achieve.
\end{itemize}

Their advantages and disadvantages may be summarized as follows:

\begin{table}[!ht]
	\centering
	\begin{tabular}{| l | p{2in} | p{2in} |}
		\hline
		\textbf{Approach} & \textbf{Advantages} & \textbf{Disadvantages} \\
		\hline
		Conventional      & \begin{itemize}
			\item Makes choices easy
		\end{itemize} & \begin{itemize}
			\item Leaves you less open to correcting methodological mistakes and being critical of your own results
			\item Makes it hard to collaborate with others due to inflexibililty
	    \end{itemize} \\
    	\hline
    	Outcome-oriented & \begin{itemize}
    		\item Nada
    	\end{itemize} & \begin{itemize}
    		\item Is too vague - what are the best results, for instance, and who judges this?
    		\item Science often involves long-term planning, so the optimal choice of method might be hard to know a priori
	    \end{itemize} \\
    	\hline
    	Reason-based & \begin{itemize}
    		\item Flexible
    	\end{itemize} & \begin{itemize}
    		\item Nada
    	\end{itemize} \\
    	\hline
	\end{tabular}
	\caption{Advantages and disadvantages of approaches to method choice.}
\end{table}