\section{Perturbation Theory}

\paragraph{A Field Transformation Operator}
$\phi$ and $\phi_{\text{in}}$, together with their conjugate momenta, are complete fields that satisfy the same commutation relations. This implies that there is a unitary operator $U(t)$ such that
\begin{align*}
	\phi = U^{-1}\phi_{\text{in}}U,\ \pi = U^{-1}\pi_{\text{in}}U.
\end{align*}
To find this operator, we will use the Heisenberg equation. We note that the total Hamiltonian is a sum of the free-field Hamiltonian $H_{\text{in}}$ and the interaction Hamiltonian $H_{\text{I}}$. We have:
\begin{align*}
	\del{}{t}\phi_{\text{in}} = \dot{U}\phi U^{-1} + U\del{}{t}\phi U^{-1} + U\phi \dot{U}^{-1}.
\end{align*}
The unitarity of $U$ yields
\begin{align*}
	\dot{U}\phi U^{-1} + U\phi\dot{U}^{-1} = 0 \iff \dot{U}^{-1} = -U^{-1}\dot{U}U^{-1},
\end{align*}
hence
\begin{align*}
	\del{}{t}\phi_{\text{in}} &= \dot{U}\phi U^{-1} + U\del{}{t}\phi U^{-1} - U\phi U^{-1}\dot{U}U^{-1} \\
	                          &= \dot{U}U^{-1}\phi_{\text{in}} + iU\comm{H(\phi_{\text{in}}, \pi_{\text{in}})}{\phi} U^{-1} - \phi_{\text{in}}\dot{U}U^{-1} \\
	                          &= \comm{\dot{U}U^{-1}}{\phi_{\text{in}}} + i\comm{H(\phi_{\text{in}}, \pi_{\text{in}})}{\phi_{\text{in}}},
\end{align*}
assuming the Hamiltonian to commute with $H$. Now, the Heisenberg equation dictates that the time evolution of the asymptotic field is dictated only by the free-field Hamiltonian, hence
\begin{align*}
	\comm{\dot{U}U^{-1} + iH_{\text{I}}(\phi_{\text{in}}, \pi_{\text{in}})}{\phi_{\text{in}}} = 0.
\end{align*}
Because this commutator is zero, presumably for any free field, we must have
\begin{align*}
	\dot{U}U^{-1} + iH_{\text{I}} = -iE_{0},
\end{align*}
yielding the differential equation
\begin{align*}
	i\dot{U} = (H_{\text{I}} + E_{0})U = H_{\text{I}}\p U.
\end{align*}

We note that the above equation is very similar to that describing the time evolution operator in non-relativistic quantum mechanics. We therefore introduce the operator
\begin{align*}
	U(t, t\p) = U(t)U^{-1}(t\p),
\end{align*}
which satisfies $U(t, t\p) = U(t, \tau)U(\tau, t\p)$ and
\begin{align*}
	i\del{}{t}U(t, t\p) = H_{\text{I}}\p U(t, t\p).
\end{align*}
We must have $U(t\p, t\p) = 1$, and we thus obtain the integral equation
\begin{align*}
	U(t, t\p) = 1 -i\integ{t\p}{t}{t_{1}}{H_{\text{I}}\p(t_{1})U(t_{1}, t\p)}.
\end{align*}
This can be solved using an iteration scheme
\begin{align*}
	U(t, t\p) =& 1 -i\integ{t\p}{t}{t_{1}}{H_{\text{I}}\p(t_{1})} + (-i)^{2}\integ{t\p}{t}{t_{1}}{\integ{t\p}{t_{1}}{t_{2}}{H_{\text{I}}\p(t_{1})H_{\text{I}}\p(t_{2})}} \\
	           &+ (-i)^{n}\integ{t\p}{t}{t_{1}}{\dots\integ{t\p}{t_{n - 1}}{t_{n}}{\prod\limits_{i = 1}^{n}H_{\text{I}}\p(t_{i})}} + \dots.
\end{align*}
Now we note that this setup is such that the different times appear in order, meaning we can use the time-ordered product. We can show that
\begin{align*}
	\integ{t\p}{t}{t_{1}}{\integ{t\p}{t_{1}}{t_{2}}{\torp{H_{\text{I}}\p(t_{1})}{H_{\text{I}}\p(t_{2})}}} = \frac{1}{2}\integ{t\p}{t}{t_{1}}{\integ{t\p}{t}{t_{2}}{\torp{H_{\text{I}}\p(t_{1})}{H_{\text{I}}\p(t_{2})}}},
\end{align*}
and we thus end up with
\begin{align*}
	U(t, t\p) = 1 + \sum\limits_{n = 1}^{\infty}\frac{(-i)^{n}}{n!}\integ{t\p}{t}{t_{1}}{\dots\integ{t\p}{t}{t_{n}}{\torp\prod\limits_{i = 1}^{n}H_{\text{I}}\p(t_{i})}} = \torp\exp(-i\integ{t\p}{t}{\tau}{H_{\text{I}}\p(\tau)}) = \torp\exp(-i\integ[4]{}{}{x}{H_{\text{I}}\p(\phi_{\text{in}}, \pi_{\text{in}})}),
\end{align*}
where we have introduced the time-ordered exponential.

We now return to the issue of computing vacuum expectation values. Consider the expectation value
\begin{align*}
	\tau(x_{1}, \dots, x_{n}) = \expval{\torp\prod\limits_{i = 1}^{n}\phi(x_{i})}{\Omega}.
\end{align*}
We have
\begin{align*}
	\tau(x_{1}, \dots, x_{n}) &= \expval{\torp\prod\limits_{i = 1}^{n}U^{-1}(t_{i})\phi_{\text{in}}(x_{i})U(t_{i})}{\Omega} \\
	                          &= \expval{\torp U^{-1}(t_{1})\phi_{\text{in}}(x_{1})U(t_{1}, t_{2})\phi_{\text{in}}(x_{2})\dots U(t_{n - 1}, t_{n})\phi_{\text{in}}(x_{n})U(t_{n})}{\Omega}.
\end{align*}
Introducing a $t$ that is greater than all involved times (implying that $-t$ is smaller than all times) we have
\begin{align*}
	\tau(x_{1}, \dots, x_{n}) = \expval{\torp U^{-1}(t)U(t, t_{1})\phi_{\text{in}}(x_{1})U(t_{1}, t_{2})\phi_{\text{in}}(x_{2})\dots U(t_{n - 1}, t_{n})\phi_{\text{in}}(x_{n})U(t_{n}, -t)U(-t)}{\Omega}.
\end{align*}
In the limit of $t\to\infty$ we somehow find
\begin{align*}
	\tau(x_{1}, \dots, x_{n}) &= \expval{U^{-1}(t)\torp\left(\phi_{\text{in}}(x_{1})\dots \phi_{\text{in}}(x_{n})U(t, -t)\right)U(-t)}{\Omega} \\
	                          &= \expval{U^{-1}(t)\torp\left(\phi_{\text{in}}(x_{1})\dots \phi_{\text{in}}(x_{n})e^{-i\integ{-t}{t}{\tau}{H_{\text{I}\p(\tau)}}}\right)U(-t)}{\Omega}
\end{align*}

Next we want to show that $\ket{\Omega}$ is an eigenstate of $U(-t)$. We therefore look at the state $\mel{\alpha, \vb{p}, \text{in}}{U(-t)}{\Omega}$. It can be shown that as $t$ goes to infinity, this matrix element approaches zero, meaning
\begin{align*}
	\lim\limits_{t\to\infty}U(\pm t)\ket{\Omega} = \lambda_{\pm}\ket{\Omega}.
\end{align*}
This also means that for such states, adding the completeness relation amounts to just multiplying by $\op{\Omega}$. In the limit we thus have
\begin{align*}
	\tau(x_{1}, \dots, x_{n}) &= \expval{U^{-1}(t)\torp\left(\phi_{\text{in}}(x_{1})\dots \phi_{\text{in}}(x_{n})e^{-i\integ{-t}{t}{\tau}{H_{\text{I}\p(\tau)}}}\right)U(-t)}{\Omega} \\ 
	                          &=\expval{U^{-1}(t)}{\Omega}\expval{\torp\phi_{\text{in}}(x_{1})\dots \phi_{\text{in}}(x_{n})e^{-i\integ{-t}{t}{\tau}{H_{\text{I}\p(\tau)}}}}{\Omega}\expval{U(-t)}{\Omega} \\
	                          &= \lambda_{+}\cc\lambda_{-}\expval{\torp\phi_{\text{in}}(x_{1})\dots \phi_{\text{in}}(x_{n})e^{-i\integ{-t}{t}{\tau}{H_{\text{I}\p(\tau)}}}}{\Omega}.
\end{align*}
To prepare for the final step, we note that
\begin{align*}
	\lambda_{+}\cc\lambda_{-} = \expval{U(-t, t)}{\Omega}
\end{align*}
according to the definition, hence
\begin{align*}
	\tau(x_{1}, \dots, x_{n}) = \frac{\expval{\torp\phi_{\text{in}}(x_{1})\dots \phi_{\text{in}}(x_{n})e^{-i\integ{-t}{t}{\tau}{H_{\text{I}\p(\tau)}}}}{\Omega}}{\expval{\torp e^{-i\integ{-t}{t}{\tau}{H_{\text{I}\p(\tau)}}}}{\Omega}}.
\end{align*}

\paragraph{The Different Vacuua}
We have thus far been using the physical and free vacuua without being clear on what they are. The physical vacuum $\ket{\Omega}$ is the ground state of the total Hamiltonian $H$, while the free vacuum $\ket{0}$ is the ground state of the free Hamiltonian $H_{\text{in}}$. These two are not orthogonal in the case of small perturbations. Expanding in eigenstates of the total Hamiltonian we have
\begin{align*}
	e^{-iHt}\ket{0} = e^{-iE_{n}t}\ket{\Omega}\braket{\Omega}{0} + \sum\limits_{n > 0}e^{-iE_{n}t}\ket{n}\braket{n}{0}.
\end{align*}
Because the energies are ordered, we somehow find that all the exponentials corresponding to excited states approach zero faster than the others, meaning
\begin{align*}
	\ket{\Omega} = \lim\limits_{t\to\infty}\frac{1}{\braket{\Omega}{0}}e^{iE_{0}t}e^{-iHt}\ket{0}.
\end{align*}
In this limit we may add a small time shift without changing the result, yielding
\begin{align*}
	\ket{\Omega} &= \lim\limits_{t\to\infty}\frac{1}{\braket{\Omega}{0}}e^{iE_{0}(t + t_{0})}e^{-iH(t + t_{0})}\ket{0} \\
	             &= \lim\limits_{t\to\infty}\frac{1}{\braket{\Omega}{0}}e^{iE_{0}(t - (-t_{0}))}e^{-iH(t - (-t_{0}))}e^{-iH_{\text{in}}(t - (-t_{0}))}\ket{0} \\
	             &= \lim\limits_{t\to\infty}\frac{1}{\braket{\Omega}{0}}e^{iE_{0}(t - (-t_{0}))}U(t_{0}, -t)\ket{0}.
\end{align*}
Returning to the issue of the time-ordered product we have
\begin{align*}
	\tau(x_{1}, \dots, x_{n}) &= \lim\limits_{t\to\infty}\frac{1}{\abs{\braket{\Omega}{0}}^{2}}e^{2iE_{0}t}\expval{U(t, 0)U^{-1}(t)\torp(\phi_{\text{in}}(x_{1})\dots\phi_{\text{in}}(x_{n}))U(t, -t)U(-t)U(0, t)}{0} \\
	                          &= \lim\limits_{t\to\infty}\frac{1}{\abs{\braket{\Omega}{0}}^{2}}e^{2iE_{0}t}\expval{\torp \phi_{\text{in}}(x_{1})\dots\phi_{\text{in}}(x_{n})}{0}.
\end{align*}
Requiring the physical vacuum to be normalized we find
\begin{align*}
	\lim\limits_{t\to\infty}\frac{1}{\abs{\braket{\Omega}{0}}^{2}}e^{2iE_{0}t}\expval{U(t, -t)}{0} = 1,
\end{align*}
and thus
\begin{align*}
	\tau(x_{1}, \dots, x_{n}) = \lim\limits_{t\to\infty}\frac{\expval{\torp \phi_{\text{in}}(x_{1})\dots\phi_{\text{in}}(x_{n})}{0}}{\expval{U(t, -t)}{0}}.
\end{align*}

\paragraph{Wick's Theorem}
To compute expectation values of the form $\expval{\torp\phi_{\text{in}}(x_{1})\dots\phi_{\text{in}}(x_{n})}{0}$, we divide the fields into two terms $\phi_{\text{in}}(x) = \phi_{\text{in}, +}(x) + \phi_{\text{in}, }(x_{1})$ where the two terms contain all positive and negative frequencies respectively, i.e. all annihilation and creation operators. Assuming $x_{1}^{0} > x_{2}^{0}$ we have:
\begin{align*}
	\torp\phi_{\text{in}}(x_{1})\phi_{\text{in}}(x_{2}) =& \phi_{\text{in}, +}(x_{1})\phi_{\text{in}, +}(x_{2}) + \phi_{\text{in}, +}(x_{1})\phi_{\text{in}, -}(x_{2}) + \phi_{\text{in}, -}(x_{1})\phi_{\text{in}, +}(x_{2}) + \phi_{\text{in}, -}(x_{1})\phi_{\text{in}, -}(x_{2}).
\end{align*}
The second term is not normal-ordered, complicating things. We remedy this by writing
\begin{align*}
	\torp\phi_{\text{in}}(x_{1})\phi_{\text{in}}(x_{2}) =& \phi_{\text{in}, +}(x_{1})\phi_{\text{in}, +}(x_{2}) + \comm{\phi_{\text{in}, +}(x_{1})}{\phi_{\text{in}, -}(x_{2})} + \phi_{\text{in}, -}(x_{2})\phi_{\text{in}, +}(x_{1}) \\
	                                                     &+ \phi_{\text{in}, -}(x_{1})\phi_{\text{in}, +}(x_{2}) + \phi_{\text{in}, -}(x_{1})\phi_{\text{in}, -}(x_{2}).
\end{align*}
To proceed we define the contraction
\begin{align*}
	\contraction{}{\phi_{\text{in}}(x_{1})}{}{\phi_{\text{in}}(x_{2})}\phi_{\text{in}}(x_{1})\phi_{\text{in}}(x_{2}) =
	\begin{cases}
		\comm{\phi_{\text{in}, +}(x_{1})}{\phi_{\text{in}, -}(x_{2})},\ x_{1}^{0} > x_{2}^{0} \\
		\comm{\phi_{\text{in}, +}(x_{2})}{\phi_{\text{in}, -}(x_{1})},\ x_{2}^{0} > x_{1}^{0}.
	\end{cases}
\end{align*}
Wick's theorem states that
\begin{align*}
	\torp\phi_{\text{in}}(x_{1})\phi_{\text{in}}(x_{2}) = \nop{\phi_{\text{in}}(x_{1})\phi_{\text{in}}(x_{2})} + \contraction{}{\phi_{\text{in}}(x_{1})}{}{\phi_{\text{in}}(x_{2})}\phi_{\text{in}}(x_{1})\phi_{\text{in}}(x_{2}),
\end{align*}
or in general
\begin{align*}
	\torp\phi_{\text{in}}(x_{1})\dots\phi_{\text{in}}(x_{n}) = \nop{\torp\phi_{\text{in}}(x_{1})\dots\phi_{\text{in}}(x_{n})} + \text{ all possible contractions}.
\end{align*}