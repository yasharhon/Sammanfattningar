\section{Basic Concepts}

\paragraph{Casimir Operators}
A Casimir operator is an operator that is constructed from the generators of a group and commutes with all generators.

\paragraph{Casimir Operators of the Poincare Group}
The Casimir operators of the Poincare group are
\begin{align*}
	P^{2} = P^{\mu}P_{\mu},\ w^{2} = w^{\mu}w_{\mu},
\end{align*}
where we have introduced the Pauli-Lubanski vector
\begin{align*}
	w_{\mu} = \frac{1}{2}\varepsilon_{\mu\nu\rho\sigma}M^{\nu\rho}P^{\sigma}. 
\end{align*}
It can be shown that
\begin{align*}
	w_{0} = \vb{P}\cdot\vb{J},\ \vb{w} = P_{0}\vb{J} + \vb{P}\times\vb{K}.
\end{align*}

\paragraph{The Wigner Classification}
As we will consider unitary representations of the Poincare group acting on states and the representations can be decomposed into irreps, we will find that we can reduce our considerations to a set of fundamental systems, termed particles. The classification, divided according to the eigenvalues of $P^{2}$ and $w^{2}$, is according to the Wigner system:
\begin{enumerate}
	\item $P^{2} > 0$, with subclasses:
	\begin{itemize}
		\item $P^{0} < 0$.
		\item $P^{0} > 0$.
	\end{itemize}
	\item $P^{2} = 0$, with subclasses:
	\begin{itemize}
		\item $P^{0} < 0$.
		\item $P^{0} > 0$.
	\end{itemize}
	\item $P^{2} = 0$ and $P^{0} = 0$.
	\item $P^{2} < 0$, corresponding to tachyons.
\end{enumerate}

\paragraph{Lorentz Covariance and the Schrödinger Equation}
Using the 4-momentum $P^{\mu} = (E, \vb{p})$ and the correspondence principle $P^{\mu} = i\del{\mu}{}{}$, the quantization of the classical energy $E = \frac{1}{2m}\vb{p}^{2}$ of a free particle is
\begin{align*}
	i\del{}{t}{\Psi} = -\frac{1}{2m}\laplacian{\Psi}.
\end{align*}
This does not in general respect Lorentz transformations, which one might expect given that it is not taken from a Lorentz covariant expression. In other words, the Schrödinger equation is not Lorentz covariant.

The quantization of the relativistic $E^{2} = m^{2} + \vb{p}^{2}$ is instead
\begin{align*}
	-\del{2}{t}{\phi} = m^{2}\phi - \laplacian{\phi}.
\end{align*}
By introducing the d'Alembertian $\dalem{} = \del{}{\mu}{\del{\mu}{}{}}$ we can write the above as
\begin{align*}
	\dalem{\phi} + m^{2}\phi = 0.
\end{align*}
This is the Klein-Gordon equation, which is an appropriate quantization of a spinless particle.

\paragraph{A Conserved Current}
Corresponding to the Klein-Gordon equation there exists a density and a current
\begin{align*}
	\rho = \frac{i}{2m}(\cc{\phi}\del{}{0}{\phi} - \phi\del{}{0}{\cc{\phi}}),\ \vb{j} = \frac{1}{2im}(\cc{\phi}\grad{\phi} - \phi\grad{\cc{\phi}})
\end{align*}
such that
\begin{align*}
	\del{}{t}{\rho} + \div{\vb{j}} = 0.
\end{align*}
Alternatively, by combining the two into a 4-current $J^{\mu} = (\rho, \vb{j})$ we find
\begin{align*}
	\del{}{\mu}{J^{\mu}} = 0.
\end{align*}

\paragraph{Problems With Stationary States}
A stationary state is a state such that
\begin{align*}
	P^{0}\phi = E\phi.
\end{align*}
For such a state we have
\begin{align*}
	J^{0} = \frac{E}{m}\abs{\phi}^{2}.
\end{align*}
In the classical limit we have $\frac{E}{m} \approx 1$, whereas in the general case we have $E = \pm\sqrt{m^{2} + \vb{p}^{2}}$, meaning that $J^{0}$ is not positive definite and the conserved Nöether cannot be interpreted as conservation of probability density. This implores us to reinterpret the Klein-Gordon equation as a general field equation.

\paragraph{Plane-Wave Solutions}
Plane-wave solutions of the Klein-Gordon equation are of the form
\begin{align*}
	\phi = Ne^{-iP_{\mu}x^{\mu}}.
\end{align*}
In order for these to be solutions, we require
\begin{align*}
	P^{0} = \pm\sqrt{m^{2} + \abs{\vb{p}}}.
\end{align*}
This does not pose a problem in non-interacting cases, as the solutions maintain their signs.

\paragraph{Charged Particles}
When treating charged particles in external electromagnetic fields, we employ the minimal coupling scheme and perform the replacement $P^{\mu}\to P^{\mu} - qA^{\mu}$. The Klein-Gordon equation then becomes
\begin{align*}
	((\del{}{\mu}{} + iqA_{\mu})(\del{\mu}{}{} + iqA^{\mu}) + m^{2})\phi = 0.
\end{align*}
This will cause additional terms
\begin{align*}
	J^{\mu} \to J^{\mu} - \frac{q}{m}\abs{\phi}^{2}A^{\mu}
\end{align*}
in the Nöether current, further destroying our hopes of creating a one-particle theory.

\paragraph{The Klein Paradox}
Consider scattering after normal incidence on a step potential described by $A^{\mu} = (V\theta(x), \vb{0})$. Performing an anzats similar to that in the non-relativistic case, the Klein-Gordon equation predicts the same behaviour as the Schrödinger equation, except for the case where $V > E + m$. In this case the transmitted 4-momentum has a negative space component. Furthermore, the transmission probability becomes negative, but still preserving $T + R = 1$. This peculiar behaviour is known as Klein's paradox.

\paragraph{The Dirac Equation}
We will now try to develop a theory that remedies the problems with the Klein-Gordon equation. The hope is that this equations has a positive-definite conserved density. An important source of the bad time was the second-order time derivative, so we will try to remedy this with a first-order time derivative. We also make the space derivatives first-order, perhaps because of Lorentz stuff. This leads us to the anzats
\begin{align*}
	\del{t}{\Psi} + (\vb*{\alpha}\cdot\grad{})\Psi + im\beta\Psi = 0,
\end{align*}
where $\beta$ and $\alpha^{i}$ are matrices and $\Psi$ is a vector. The sizes of these are yet to be determined. The corresponding equation for $\Psi\adj$ is
\begin{align*}
	\del{t}{\Psi\adj} + (\grad{\Psi\adj})\cdot\vb*{\alpha}\adj - im\Psi\adj\beta\adj = 0.
\end{align*}
Considering the quantity $\Psi\adj\Psi$ we have
\begin{align*}
	\del{t}{(\Psi\adj\Psi)} &= (\del{t}{\Psi\adj})\Psi + \Psi\adj\del{t}{\Psi} \\
	                        &= (im\Psi\adj\beta\adj - (\grad{\Psi\adj})\cdot\vb*{\alpha})\Psi + \Psi\adj(-(\vb*{\alpha}\cdot\grad{})\Psi - im\beta\Psi) \\
	                        &= im\Psi\adj(\beta\adj - \beta) - (\grad{\Psi\adj})\cdot\vb*{\alpha}\adj\Psi - \Psi\adj(\vb*{\alpha}\cdot\grad{})\Psi.
\end{align*}
We really want the right-hand side to be the 3-divergence of some 3-current. To do that, we may choose $\alpha^{i}$ and $\beta$ to be Hermitian, yielding
\begin{align*}
	(\grad{\Psi\adj})\cdot\vb*{\alpha}\adj\Psi + \Psi\adj(\vb*{\alpha}\cdot\grad{})\Psi = \div{\Psi\adj\vb*{\alpha}\Psi}.
\end{align*}
The conserved 4-current is thus $j^{\mu} = (\Psi\adj\Psi, \Psi\adj\vb*{\alpha}\Psi)$.

To reobtain something like the 4-vector norm we had when discussing the Klein-Gordon equation, we apply the operator $\del{t}{} - (\vb*{\alpha}\cdot\grad{}) - im\beta\Psi$ to our anzats to find
\begin{align*}
	(\del{t}{} - (\vb*{\alpha}\cdot\grad{}) - im\beta\Psi)(\del{t}{\Psi} + (\vb*{\alpha}\cdot\grad{})\Psi + im\beta\Psi) = 0.
\end{align*}
As the derivatives commute with the matrices, the cross terms vanish, yielding
\begin{align*}
	\del[2]{t}{\Psi} - (\vb*{\alpha}\cdot\grad{})^{2}\Psi - (\vb*{\alpha}\cdot\grad{})im\beta\Psi - im\beta(\vb*{\alpha}\cdot\grad{})\Psi + m^{2}\beta^{2}\Psi &= 0,
\end{align*}
or more explicitly
\begin{align*}
	\del[2]{t}{\Psi} - (\alpha^{i}\del{i}{})(\alpha^{j}\del{j}{})\Psi - im((\alpha^{i}\del{i}{})\beta + \beta\alpha^{i}\del{i}{})\Psi + m^{2}\beta^{2}\Psi = \del[2]{t}{\Psi} - \alpha^{i}\alpha^{j}\del{i}{\del{j}{\Psi}} - im(\alpha^{i}\beta + \beta\alpha^{i})\del{i}{\Psi} + m^{2}\beta^{2}\Psi = 0.
\end{align*}
We can symmetrize the second term to find
\begin{align*}
	\del[2]{t}{\Psi} - \frac{1}{2}(\alpha^{i}\alpha^{j} + \alpha^{j}\alpha^{i})\del{i}{\del{j}{\Psi}} - im(\alpha^{i}\beta + \beta\alpha^{i})\del{i}{\Psi} + m^{2}\beta^{2}\Psi = 0.
\end{align*}
This produces the same 4-vector norm if
\begin{align*}
	\acomm{\alpha^{i}}{\alpha^{j}} = 2\delta^{ij},\ \beta^{2} = 1,\ \alpha^{i}\beta + \beta\alpha^{i} = 0.
\end{align*}
Computing the determinant of the last equation, we find
\begin{align*}
	\det(\alpha^{i}\beta) = (-1)^{N}\det(\beta\alpha^{i}),
\end{align*}
where $N$ is the length of $\Psi$. The only way for the above equations to be solvable is then that $N$ be odd. It can also be shown that $\alpha^{i}$ and $\beta$ are all traceless. By a series of arguments we find that $N = 4$ is correct.

To complete our discussion, we define $\gamma^{0} = \beta,\ \gamma^{i} = \beta\alpha^{i}$. With this we multiply our anzats by $-i\beta$ and find
\begin{align*}
	(-i\beta\del{0}{} - i(\beta\vb*{\alpha}\cdot\grad{}))\Psi + m\Psi = 0.
\end{align*}
Defining the inner product $\gamma^{\mu}A_{\mu} = \fsl{A}$ we arrive at the Dirac equation
\begin{align*}
	i\fsl{\del{}{}}\Psi - m\Psi = 0.
\end{align*}

\paragraph{Properties of the $\gamma^{\mu}$}
The $\gamma^{\mu}$ satisfy
\begin{align*}
	(\gamma^{\mu})\adj = \gamma^{0}\gamma^{\mu}\gamma^{0}.
\end{align*}

Defining the matrix $\gamma^{5} = i\gamma^{0}\gamma^{1}\gamma^{2}\gamma^{3}$, we find that it must be Hermitian.

We also have
\begin{align*}
	\acomm{\gamma^{5}}{\gamma^{\mu}} = 0.
\end{align*}

We have
\begin{align*}
	\tr(\prod\limits_{i = 1}^{n}\gamma^{\mu_{i}}) &= 0,\ n\text{ odd}, \\
	\tr(\gamma^{5})                               &= 0.
\end{align*}

\paragraph{The Dirac Algebra}
The $\gamma^{\mu}$ are a representation of the Dirac algebra
\begin{align*}
	\acomm{\gamma^{\mu}}{\gamma^{\nu}} = 2g^{\mu\nu}.
\end{align*}
We may compute explicit representations of this algebra as
\begin{align*}
	\gamma^{0} = 
	\mqty[
		1 & 0 \\
		0 & -1
	],\ 
	\gamma^{i} = 
	\mqty[
		0           & \sigma^{i} \\
		-\sigma^{i} & 0
	].
\end{align*}
In this representation we have
\begin{align*}
	\gamma^{5} =
	\mqty[
		0 & 1 \\
		1 & 0
	].
\end{align*}

\paragraph{The Dirac Adjoint}
The Dirac adjoint is defined as
\begin{align*}
	\bar{A} = A\adj\gamma^{0}.
\end{align*}

\paragraph{Rewriting the 4-Current}
We may use the properties of the $\gamma^{\mu}$ to write
\begin{align*}
	j^{\mu} = \Psi\adj\gamma^{0}\gamma^{\mu}\Psi = \bar{\Psi}\gamma^{\mu}\Psi.
\end{align*}

\paragraph{Plane-Wave Solutions}
Plane-wave solutions of the Dirac equation are of the form
%TODO: Add negative energy solutions
\begin{align*}
	\Psi_{P} = e^{-iP_{\mu}x^{\mu}}u(P^{\mu}),
\end{align*}
where $u(P^{\mu})$ is a so-called spinor. Inserting it into the Dirac equation we find
\begin{align*}
	(-\fsl{P} + m)u(P^{\mu}) = 0.
\end{align*}
Multiplying by $\fsl{P} + m$ we find
\begin{align*}
	(-\fsl{P}^{2} + m^{2})u(P^{\mu}) = (-\gamma^{\mu}\gamma^{\nu}P_{\mu}P_{\nu} + m^{2})u(P^{\mu}) = 0.
\end{align*}
We can symmetrize the first term and use the anticommutation relations of the $\gamma^{\mu}$ to find
\begin{align*}
	(-P^{2} + m^{2})u(P^{\mu}) = 0.
\end{align*}
In other words, the solution satisfies the relativistic energy-momentum relation. This also implies that for non-trivial solutions, the 4-momentum is time-like.

In the corresponding rest frame, there are four independent spinor solutions. These are $u_{\pm}$ and $v_{\pm}$, and with this representation they are as you would expect.

\paragraph{A Hamiltonian}
As the plane-wave solutions have time-like 4-momenta, there is a corresponding rest frame. In this rest frame, the Hamiltonian, which is generally
\begin{align*}
	\ham = \beta m + \vb*{\alpha}\cdot\vb{p},
\end{align*}
reduces to
\begin{align*}
	\ham = \beta m.
\end{align*}
Defining
\begin{align*}
	\vb*{\Sigma} =
	\mqty[
		\vb*{\sigma} & 0 \\
		0            & \vb*{\sigma}
	],
\end{align*}
we have $\comm{\ham}{\vb*{\Sigma}} = 0$.