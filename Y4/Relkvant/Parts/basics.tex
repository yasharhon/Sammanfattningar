\section{Basic Concepts}

\paragraph{Casimir Operators}
A Casimir operator is an operator that is constructed from the generators of a group and commutes with all generators.

\paragraph{Casimir Operators of the Poincare Group}
The Casimir operators of the Poincare group are
\begin{align*}
	P^{2} = P^{\mu}P_{\mu},\ w^{2} = w^{\mu}w_{\mu},
\end{align*}
where we have introduced the Pauli-Lubanski vector
\begin{align*}
	w_{\mu} = \frac{1}{2}\varepsilon_{\mu\nu\rho\sigma}M^{\nu\rho}P^{\sigma}. 
\end{align*}
It can be shown that
\begin{align*}
	w_{0} = \vb{P}\cdot\vb{J},\ \vb{w} = P_{0}\vb{J} + \vb{P}\times\vb{K}.
\end{align*}

\paragraph{The Wigner Classification}
As we will consider unitary representations of the Poincare group acting on states and the representations can be decomposed into irreps, we will find that we can reduce our considerations to a set of fundamental systems, termed particles. The classification, divided according to the eigenvalues of $P^{2}$ and $w^{2}$, is according to the Wigner system:
\begin{enumerate}
	\item $P^{2} > 0$, with subclasses:
	\begin{itemize}
		\item $P^{0} < 0$.
		\item $P^{0} > 0$.
	\end{itemize}
	\item $P^{2} = 0$, with subclasses:
	\begin{itemize}
		\item $P^{0} < 0$.
		\item $P^{0} > 0$.
	\end{itemize}
	\item $P^{2} = 0$ and $P^{0} = 0$.
	\item $P^{2} < 0$, corresponding to tachyons.
\end{enumerate}

\paragraph{Lorentz Covariance and the Schrödinger Equation}
Using the 4-momentum $P^{\mu} = (E, \vb{p})$ and the correspondence principle $P^{\mu} = i\del{\mu}{}{}$, the quantization of the classical energy $E = \frac{1}{2m}\vb{p}^{2}$ of a free particle is
\begin{align*}
	i\del{}{t}{\Psi} = -\frac{1}{2m}\laplacian{\Psi}.
\end{align*}
This does not in general respect Lorentz transformations, which one might expect given that it is not taken from a Lorentz covariant expression. In other words, the Schrödinger equation is not Lorentz covariant.

The quantization of the relativistic $E^{2} = m^{2} + \vb{p}^{2}$ is instead
\begin{align*}
	-\del{2}{t}{\phi} = m^{2}\phi - \laplacian{\phi}.
\end{align*}
By introducing the d'Alembertian $\dalem{} = \del{}{\mu}{\del{\mu}{}{}}$ we can write the above as
\begin{align*}
	\dalem{\phi} + m^{2}\phi = 0.
\end{align*}
This is the Klein-Gordon equation, which is an appropriate quantization of a spinless particle.

\paragraph{A Conserved Current}
Corresponding to the Klein-Gordon equation there exists a density and a current
\begin{align*}
	\rho = \frac{i}{2m}(\cc{\phi}\del{}{0}{\phi} - \phi\del{}{0}{\cc{\phi}}),\ \vb{j} = \frac{1}{2im}(\cc{\phi}\grad{\phi} - \phi\grad{\cc{\phi}})
\end{align*}
such that
\begin{align*}
	\del{}{t}{\rho} + \div{\vb{j}} = 0.
\end{align*}
Alternatively, by combining the two into a 4-current $J^{\mu} = (\rho, \vb{j})$ we find
\begin{align*}
	\del{}{\mu}{J^{\mu}} = 0.
\end{align*}

\paragraph{Problems With Stationary States}
A stationary state is a state such that
\begin{align*}
	P^{0}\phi = E\phi.
\end{align*}
For such a state we have
\begin{align*}
	J^{0} = \frac{E}{m}\abs{\phi}^{2}.
\end{align*}
In the classical limit we have $\frac{E}{m} \approx 1$, whereas in the general case we have $E = \pm\sqrt{m^{2} + \vb{p}^{2}}$, meaning that $J^{0}$ is not positive definite and the conserved Nöether cannot be interpreted as conservation of probability density. This implores us to reinterpret the Klein-Gordon equation as a general field equation.

\paragraph{Plane-Wave Solutions}
Plane-wave solutions of the Klein-Gordon equation are of the form
\begin{align*}
	\phi = Ne^{-iP_{\mu}x^{\mu}}.
\end{align*}
In order for these to be solutions, we require
\begin{align*}
	P^{0} = \pm\sqrt{m^{2} + \abs{\vb{p}}}.
\end{align*}
This does not pose a problem in non-interacting cases, as the solutions maintain their signs.

\paragraph{Charged Particles}
When treating charged particles in external electromagnetic fields, we employ the minimal coupling scheme and perform the replacement $P^{\mu}\to P^{\mu} - qA^{\mu}$. The Klein-Gordon equation then becomes
\begin{align*}
	((\del{}{\mu}{} + iqA_{\mu})(\del{\mu}{}{} + iqA^{\mu}) + m^{2})\phi = 0.
\end{align*}
This will cause additional terms
\begin{align*}
	J^{\mu} \to J^{\mu} - \frac{q}{m}\abs{\phi}^{2}A^{\mu}
\end{align*}
in the Nöether current, further destroying our hopes of creating a one-particle theory.

\paragraph{The Klein Paradox}
Consider scattering after normal incidence on a step potential described by $A^{\mu} = (V\theta(x), \vb{0})$. Performing an anzats similar to that in the non-relativistic case, the Klein-Gordon equation predicts the same behaviour as the Schrödinger equation, except for the case where $V > E + m$. In this case the transmitted 4-momentum has a negative space component. Furthermore, the transmission probability becomes negative, but still preserving $T + R = 1$. This peculiar behaviour is known as Klein's paradox.

\paragraph{The Dirac Equation}
We will now try to develop a theory that remedies the problems with the Klein-Gordon equation. The hope is that this equations has a positive-definite conserved density. An important source of the bad time was the second-order time derivative, so we will try to remedy this with a first-order time derivative. We also make the space derivatives first-order, perhaps because of Lorentz stuff. This leads us to the anzats
\begin{align*}
	\del{t}{\Psi} + (\vb*{\alpha}\cdot\grad{})\Psi + im\beta\Psi = 0,
\end{align*}
where $\beta$ and $\alpha^{i}$ are matrices and $\Psi$ is a vector. The sizes of these are yet to be determined. The corresponding equation for $\Psi\adj$ is
\begin{align*}
	\del{t}{\Psi\adj} + (\grad{\Psi\adj})\cdot\vb*{\alpha}\adj - im\Psi\adj\beta\adj = 0.
\end{align*}
Considering the quantity $\Psi\adj\Psi$ we have
\begin{align*}
	\del{t}{(\Psi\adj\Psi)} &= (\del{t}{\Psi\adj})\Psi + \Psi\adj\del{t}{\Psi} \\
	                        &= (im\Psi\adj\beta\adj - (\grad{\Psi\adj})\cdot\vb*{\alpha})\Psi + \Psi\adj(-(\vb*{\alpha}\cdot\grad{})\Psi - im\beta\Psi) \\
	                        &= im\Psi\adj(\beta\adj - \beta) - (\grad{\Psi\adj})\cdot\vb*{\alpha}\adj\Psi - \Psi\adj(\vb*{\alpha}\cdot\grad{})\Psi.
\end{align*}
We really want the right-hand side to be the 3-divergence of some 3-current. To do that, we may choose $\alpha^{i}$ and $\beta$ to be Hermitian, yielding
\begin{align*}
	(\grad{\Psi\adj})\cdot\vb*{\alpha}\adj\Psi + \Psi\adj(\vb*{\alpha}\cdot\grad{})\Psi = \div{\Psi\adj\vb*{\alpha}\Psi}.
\end{align*}
The conserved 4-current is thus $j^{\mu} = (\Psi\adj\Psi, \Psi\adj\vb*{\alpha}\Psi)$.

To reobtain something like the 4-vector norm we had when discussing the Klein-Gordon equation, we apply the operator $\del{t}{} - (\vb*{\alpha}\cdot\grad{}) - im\beta\Psi$ to our anzats to find
\begin{align*}
	(\del{t}{} - (\vb*{\alpha}\cdot\grad{}) - im\beta\Psi)(\del{t}{\Psi} + (\vb*{\alpha}\cdot\grad{})\Psi + im\beta\Psi) = 0.
\end{align*}
As the derivatives commute with the matrices, the cross terms vanish, yielding
\begin{align*}
	\del[2]{t}{\Psi} - (\vb*{\alpha}\cdot\grad{})^{2}\Psi - (\vb*{\alpha}\cdot\grad{})im\beta\Psi - im\beta(\vb*{\alpha}\cdot\grad{})\Psi + m^{2}\beta^{2}\Psi &= 0,
\end{align*}
or more explicitly
\begin{align*}
	\del{2}{t}{\Psi} - (\alpha^{i}\del{}{i}{})(\alpha^{j}\del{}{j}{})\Psi - im((\alpha^{i}\del{}{i}{})\beta + \beta\alpha^{i}\del{}{i}{})\Psi + m^{2}\beta^{2}\Psi = \del{2}{t}{\Psi} - \alpha^{i}\alpha^{j}\del{}{i}{\del{}{j}{\Psi}} - im(\alpha^{i}\beta + \beta\alpha^{i})\del{}{i}{\Psi} + m^{2}\beta^{2}\Psi = 0.
\end{align*}
We can symmetrize the second term to find
\begin{align*}
	\del{2}{t}{\Psi} - \frac{1}{2}(\alpha^{i}\alpha^{j} + \alpha^{j}\alpha^{i})\del{}{i}{\del{}{j}{\Psi}} - im(\alpha^{i}\beta + \beta\alpha^{i})\del{}{i}{\Psi} + m^{2}\beta^{2}\Psi = 0.
\end{align*}
This produces the same 4-vector norm if
\begin{align*}
	\acomm{\alpha^{i}}{\alpha^{j}} = 2\delta^{ij},\ \beta^{2} = 1,\ \alpha^{i}\beta + \beta\alpha^{i} = 0.
\end{align*}
Computing the determinant of the last equation, we find
\begin{align*}
	\det(\alpha^{i}\beta) = (-1)^{N}\det(\beta\alpha^{i}),
\end{align*}
where $N$ is the length of $\Psi$. The only way for the above equations to be solvable is then that $N$ be odd. It can also be shown that $\alpha^{i}$ and $\beta$ are all traceless. By a series of arguments we find that $N = 4$ is correct.

To complete our discussion, we define $\gamma^{0} = \beta,\ \gamma^{i} = \beta\alpha^{i}$. With this we multiply our anzats by $-i\beta$ and find
\begin{align*}
	(-i\beta\del{0}{} - i(\beta\vb*{\alpha}\cdot\grad{}))\Psi + m\Psi = 0.
\end{align*}
Defining the inner product $\gamma^{\mu}A_{\mu} = \fsl{A}$ we arrive at the Dirac equation
\begin{align*}
	i\fsl{\del{}{}{}}\Psi - m\Psi = 0.
\end{align*}

\paragraph{Properties of the $\gamma^{\mu}$}
The $\gamma^{\mu}$ satisfy
\begin{align*}
	(\gamma^{\mu})\adj = \gamma^{0}\gamma^{\mu}\gamma^{0}.
\end{align*}

Defining the matrix $\gamma^{5} = i\gamma^{0}\gamma^{1}\gamma^{2}\gamma^{3}$, we find that it must be Hermitian.

We also have
\begin{align*}
	\acomm{\gamma^{5}}{\gamma^{\mu}} = 0.
\end{align*}

We have
\begin{align*}
	\tr(\prod\limits_{i = 1}^{n}\gamma^{\mu_{i}}) &= 0,\ n\text{ odd}, \\
	\tr(\gamma^{5})                               &= 0.
\end{align*}

\paragraph{The Dirac Algebra}
The $\gamma^{\mu}$ are a representation of the Dirac algebra
\begin{align*}
	\acomm{\gamma^{\mu}}{\gamma^{\nu}} = 2g^{\mu\nu}.
\end{align*}
We may compute explicit representations of this algebra as
\begin{align*}
	\gamma^{0} = 
	\mqty[
		1 & 0 \\
		0 & -1
	],\ 
	\gamma^{i} = 
	\mqty[
		0           & \sigma^{i} \\
		-\sigma^{i} & 0
	].
\end{align*}
In this representation we have
\begin{align*}
	\gamma^{5} =
	\mqty[
		0 & 1 \\
		1 & 0
	].
\end{align*}

\paragraph{The Dirac Adjoint}
The Dirac adjoint is defined as
\begin{align*}
	\bar{A} = A\adj\gamma^{0}.
\end{align*}

\paragraph{Rewriting the 4-Current}
We may use the properties of the $\gamma^{\mu}$ to write
\begin{align*}
	j^{\mu} = \Psi\adj\gamma^{0}\gamma^{\mu}\Psi = \bar{\Psi}\gamma^{\mu}\Psi.
\end{align*}

\paragraph{Plane-Wave Solutions}
Plane-wave solutions of the Dirac equation are of the form
%TODO: Add negative energy solutions
\begin{align*}
	\Psi_{P} = e^{-iP_{\mu}x^{\mu}}u(P^{\mu}),
\end{align*}
where $u(P^{\mu})$ is a so-called spinor. Inserting it into the Dirac equation we find
\begin{align*}
	(-\fsl{P} + m)u(P^{\mu}) = 0.
\end{align*}
Multiplying by $\fsl{P} + m$ we find
\begin{align*}
	(-\fsl{P}^{2} + m^{2})u(P^{\mu}) = (-\gamma^{\mu}\gamma^{\nu}P_{\mu}P_{\nu} + m^{2})u(P^{\mu}) = 0.
\end{align*}
We can symmetrize the first term and use the anticommutation relations of the $\gamma^{\mu}$ to find
\begin{align*}
	(-P^{2} + m^{2})u(P^{\mu}) = 0.
\end{align*}
In other words, the solution satisfies the relativistic energy-momentum relation. This also implies that for non-trivial solutions, the 4-momentum is time-like.

In the corresponding rest frame, there are four independent spinor solutions. These are $u_{\pm}$ and $v_{\pm}$, and with this representation they are as you would expect.

\paragraph{A Hamiltonian}
As the plane-wave solutions have time-like 4-momenta, there is a corresponding rest frame. In this rest frame, the Hamiltonian, which is generally
\begin{align*}
	\ham = \beta m + \vb*{\alpha}\cdot\vb{p},
\end{align*}
reduces to
\begin{align*}
	\ham = \beta m.
\end{align*}
Defining
\begin{align*}
	\vb*{\Sigma} =
	\mqty[
		\vb*{\sigma} & 0 \\
		0            & \vb*{\sigma}
	],
\end{align*}
we have $\comm{\ham}{\vb*{\Sigma}} = 0$.

\paragraph{Lorentz Transformation of Spinors}
Suppose that $\Psi$ is a solution to the Dirac equation. Under a Lorentz transform it should transform according to
\begin{align*}
	\Psi\p(x\p) = S(\Lambda)\Psi(x).
\end{align*}
We would like to identify the transformation matrix $S$.

Because the Dirac equation is Lorentz covariant, it looks the same in all frames. We can use the chain rule to find
\begin{align*}
	\del{\mu}{} = \tensor{\Lambda}{^{\mu\p}_{\mu}}\del{\mu\p}{}
\end{align*}
and thus
\begin{align*}
	(-i\gamma^{\mu}\del{\mu}{} + m)\Psi = (-i\gamma^{\mu}\tensor{\Lambda}{^{\mu\p}_{\mu}}\del{\mu\p}{} + m)S^{-1}\Psi\p = 0.
\end{align*}
Note that as $m$ is a Lorentz scalar, it is also the same in all frames. Multiplication by $S$ yields
\begin{align*}
	(-i\tensor{\Lambda}{^{\mu\p}_{\mu}}S\gamma^{\mu}S^{-1}\del{\mu\p}{} + m)\Psi\p = 0.
\end{align*}
Comparing this with the Dirac equation in the primed frame, we must therefore have
\begin{align*}
	\tensor{\Lambda}{^{\mu\p}_{\mu}}S\gamma^{\mu}S^{-1} = \gamma^{\mu\p} \iff \tensor{\Lambda}{^{\mu\p}_{\mu}}\gamma^{\mu} = S^{-1}\gamma^{\mu\p}S.
\end{align*}
To proceed, we apply the expansion
\begin{align*}
	\tensor{\Lambda}{^{\mu\p}_{\mu}} = \kdelta{\mu\p}{\mu} + \varepsilon\tensor{\omega}{^{\mu\p}_{\mu}},\ \omega^{\mu\nu} = -\omega^{\nu\mu},
\end{align*}
and expand $S$ in terms of $\varepsilon$ as
\begin{align*}
	S = 1 - \frac{i}{4}\varepsilon\omega^{\mu\nu}\sigma_{\mu\nu},\ \sigma_{\mu\nu} = -\sigma_{\nu\mu}
\end{align*}
for some set of matrices $\sigma_{\mu\nu}$. We see that its inverse to first order must be
\begin{align*}
	S^{-1} = 1 + \frac{i}{4}\varepsilon\omega^{\mu\nu}\sigma_{\mu\nu}.
\end{align*}
We now have
\begin{align*}
	(\kdelta{\mu\p}{\mu} + \varepsilon\tensor{\omega}{^{\mu\p}_{\mu}})\gamma^{\mu} &= \left(1 + \frac{i}{4}\varepsilon\omega^{\mu\nu}\sigma_{\mu\nu}\right)\gamma^{\mu\p}\left(1 - \frac{i}{4}\varepsilon\omega^{\mu\nu}\sigma_{\mu\nu}\right) \\
	&= \gamma^{\mu\p} + \frac{i}{4}\varepsilon\omega^{\mu\nu}(\sigma_{\mu\nu}\gamma^{\mu\p} - \gamma^{\mu\p}\sigma_{\mu\nu}),
\end{align*}
which is somehow
\begin{align*}
	\comm{\gamma^{\mu\p}}{\sigma_{\mu\nu}} = 2i(\tensor{g}{^{\mu\p}_{\mu}}\gamma_{\nu} - \tensor{g}{^{\mu\p}_{\nu}}\gamma_{\mu}),
\end{align*}
with the solution
\begin{align*}
	\sigma_{\mu\nu} = \frac{i}{2}\comm{\gamma_{\mu}}{\gamma_{\nu}}.
\end{align*}

\paragraph{Charged Particles}
When introducing charge, we employ the minimal coupling scheme. The Dirac equation then becomes
\begin{align*}
	(-\gamma^{\mu}(i\del{}{\mu}{} - qA_{\mu}) + m)\Psi = (\fsl{p} - q\fsl{A} - m)\Psi = 0.
\end{align*}
To proceed we write the solution as
\begin{align*}
	\Psi = (\fsl{p} - q\fsl{A} + m)\chi,
\end{align*}
yielding
\begin{align*}
	(\fsl{p} - q\fsl{A} - m)(\fsl{p} - q\fsl{A} + m)\chi = ((\fsl{p} - q\fsl{A})^{2} - m^{2})\chi = 0.
\end{align*}

We investigate the operator appearing in this modified solution as
\begin{align*}
	(\fsl{p} - q\fsl{A})^{2} &= \gamma^{\mu}\gamma^{\nu}(p_{\mu} - qA_{\mu})(p_{\nu} - qA_{\nu}) \\
	                         &= \frac{1}{2}\left(\gamma^{\mu}\gamma^{\nu}(p_{\mu} - qA_{\mu})(p_{\nu} - qA_{\nu}) + \gamma^{\nu}\gamma^{\mu}(p_{\nu} - qA_{\nu})(p_{\mu} - qA_{\mu})\right) \\
	                         &= \frac{1}{2}\left(\gamma^{\mu}\gamma^{\nu}(p_{\mu}p_{\nu} - q(p_{\mu}A_{\nu} + A_{\mu}p_{\nu}) + q^{2}A_{\mu}A_{\nu}) + \gamma^{\nu}\gamma^{\mu}(p_{\nu}p_{\mu} - q(p_{\nu}A_{\mu} + A_{\nu}p_{\mu}) + q^{2}A_{\nu}A_{\mu})\right) \\
	                         &= \frac{1}{2}\left(\acomm{\gamma^{\mu}}{\gamma^{\nu}}(p_{\mu}p_{\nu} + q^{2}A_{\mu}A_{\nu}) - q(\gamma^{\mu}\gamma^{\nu}(p_{\mu}A_{\nu} + A_{\mu}p_{\nu}) + \gamma^{\nu}\gamma^{\mu}(p_{\nu}A_{\mu} + A_{\nu}p_{\mu}))\right) \\
	                         &= \frac{1}{2}\left(\acomm{\gamma^{\mu}}{\gamma^{\nu}}(p_{\mu}p_{\nu} - q(p_{\mu}A_{\nu} + A_{\mu}p_{\nu}) + q^{2}A_{\mu}A_{\nu}) - q\gamma^{\nu}\gamma^{\mu}(p_{\nu}A_{\mu} + A_{\nu}p_{\mu} - p_{\mu}A_{\nu} - A_{\mu}p_{\nu})\right) \\
	                         &= \frac{1}{2}\left(2g^{\mu\nu}(p_{\mu}p_{\nu} - q(p_{\mu}A_{\nu} + A_{\mu}p_{\nu}) + q^{2}A_{\mu}A_{\nu}) - q\gamma^{\nu}\gamma^{\mu}(\comm{p_{\nu}}{A_{\mu}} - \comm{p_{\mu}}{A_{\nu}}\right).
\end{align*}
We have
\begin{align*}
	\comm{p_{\mu}}{A_{\nu}} = i((\del{}{\mu}{A_{\nu}}) + A_{\nu}\del{}{\mu}{} - A_{\nu}\del{}{\mu}{}) = i(\del{}{\mu}{A_{\nu}}),
\end{align*}
and thus
\begin{align*}
	(\fsl{p} - q\fsl{A})^{2} &= \frac{1}{2}\left(2g^{\mu\nu}(p_{\mu}p_{\nu} - q(p_{\mu}A_{\nu} + A_{\mu}p_{\nu}) + q^{2}A_{\mu}A_{\nu}) - iq\gamma^{\nu}\gamma^{\mu}(\del{}{\nu}{A_{\mu}} - \del{}{\mu}{A_{\nu}})\right) \\
	                         &= (p - qA)^{2} + \frac{i}{2}q\gamma^{\nu}\gamma^{\mu}F_{\mu\nu}.
\end{align*}
As
\begin{align*}
	\gamma^{\nu}\gamma^{\mu} = \frac{1}{2}(\comm{\gamma^{\nu}}{\gamma^{\mu}} + \acomm{\gamma^{\nu}}{\gamma^{\mu}}),
\end{align*}
where the first term is symmetric and the second is antisymmetric, we have
\begin{align*}
	(\fsl{p} - q\fsl{A})^{2} &= (p - qA)^{2} + \frac{1}{2}q\sigma^{\nu\mu}F_{\mu\nu} = (p - qA)^{2} - \frac{1}{2}q\sigma^{\mu\nu}F_{\mu\nu}.
\end{align*}
It can be shown that
\begin{align*}
	\frac{1}{2}q\sigma^{\mu\nu}F_{\mu\nu} = -q(\vb*{\Sigma}\cdot\vb{B} - i\vb*{\alpha}\cdot\vb{E}),
\end{align*}
where the first term is interpreted as a magnetic dipole contribution and the second as an electric monopole contribution.

\paragraph{The Hydrogenic Atom}
A system with a Coulomb potential $V(r) = -\frac{Ze^{2}}{4\pi r}$ is an example of an exactly solvable model with the Dirac formalism. The corresponding Hamiltonian is
\begin{align*}
	H = \beta m + \vb*{\alpha}\cdot\vb{p} + V(r),
\end{align*}
which commutes with $\vb{J}$ and the parity operator $P$. We thus seek simultaneous eigenstates of $H, \vb{J}^{2}, J^{3}$ and $P$. The corresponding eigenvalues are $E, j(j + 1), m$ and $(-1)^{j + \frac{\bar{\omega}}{2}}$, where
\begin{align*}
	\bar{\omega} =
	\begin{cases}
		 1,\ P = (-1)^{j + \frac{1}{2}}, \\
		-1,\ P = (-1)^{j - \frac{1}{2}}.
	\end{cases}
\end{align*}
As the problem is divided into blocks, we write the desired states as
\begin{align*}
	\Psi = 
	\mqty[
		\phi \\
		\chi
	].
\end{align*}
The angular part may be separated out, and corresponding to them are two sets of solutions $\mathcal{Y}^{m}_{lj}$, where $l$ is the eigenvalue of $L$ and takes on values $j \pm \frac{1}{2}\bar{\omega}$ for the two sets of solutions. Explicitly the solutions are
\begin{align*}
	\mathcal{Y}^{m}_{lj} =
	\mqty[
		\sqrt{\frac{j + m}{2j}}Y_{j - \frac{1}{2}, m - \frac{1}{2}} \\
		\sqrt{\frac{j - m}{2j}}Y_{j - \frac{1}{2}, m + \frac{1}{2}}
	],\ j = l + \frac{1}{2}
\end{align*}
and
\begin{align*}
	\mathcal{Y}^{m}_{lj} =
	\mqty[
		\sqrt{\frac{j - m + 1}{2(j + 1)}}Y_{j + \frac{1}{2}, m - \frac{1}{2}} \\
		-\sqrt{\frac{j + m + 1}{2(j + 1)}}Y_{j + \frac{1}{2}, m + \frac{1}{2}}
	],\ j = l - \frac{1}{2},\ l > 0.
\end{align*}

To proceed, we make the anzats
\begin{align*}
	\phi = \frac{1}{r}F(r)\mathcal{Y}^{m}_{lj},\ \chi = \frac{1}{r}G(r)\mathcal{Y}^{m}_{lj}.
\end{align*}
We also introduce
\begin{align*}
	\vb{r}\cdot\vb{p} = -ir\del{}{r}{},\ p_{r} = -\frac{i}{r}\del{}{r}{} = \frac{1}{r}(\vb{r}\cdot\vb{p} - i),
\end{align*}
as well as
\begin{align*}
	\alpha_{r} = \frac{1}{r}\vb*{\alpha}\cdot\vb{r},
\end{align*}
from which one can show
\begin{align*}
	\vb*{\alpha}\cdot\vb{p} = \alpha_{r}(p_{r} + \frac{i}{r}\beta K),
\end{align*}
where $K = \beta(\vb*{\Sigma}\cdot\vb{L} + 1)$, with eigenvalues $-\bar{\omega}\left(j + \frac{1}{2}\right)$. The Hamiltonian can now be written as
\begin{align*}
	H = \alpha_{r}(p_{r} - \frac{i\bar{\omega}\left(j + \frac{1}{2}\right)}{r}\beta) + \beta m + V(r).
\end{align*}
The eigenvalue equation then becomes
\begin{align*}
	\left(\mqty[
		0           & -\del{}{r}{} \\
		\del{}{r}{} & 0
	] + \mqty[
		0                                                   & \frac{\bar{\omega}\left(j + \frac{1}{2}\right)}{r} \\
		\frac{i\bar{\omega}\left(j + \frac{1}{2}\right)}{r} & 0
	]\right)\mqty[
		F \\
		G
	] = (E - m - V(r))\mqty[
		F \\
		G
	].
\end{align*}

In the particular case of the hydrogenic atom, we introduce the following notation:
\begin{align*}
	\kappa = \sqrt{m^{2} - E^{2}},\ \rho = \kappa r,\ \tau = \bar{\omega}\left(j + \frac{1}{2}\right),´ \nu = \sqrt{\frac{m - E}{m + E}},\ Z\p = \frac{Ze^{2}}{4\pi}
\end{align*}
to find
\begin{align*}
	\left(\mqty[
		0              & -\del{}{\rho}{} \\
		\del{}{\rho}{} & 0
	] + \mqty[
		0                                                   & \frac{\tau}{\rho} \\
		\frac{\tau}{\rho} & 0
	]\right)\mqty[
		F \\
		G
	] = \mqty[
		-\nu + \frac{Z\p}{\rho} & 0 \\
		0                       & \frac{1}{\nu} + \frac{Z\p}{\rho}
	]\mqty[
		F \\
		G
	].
\end{align*}
We can show that close to the origin the functions behave like power laws, and at infinity they decay exponentially. We thus make the anzats
\begin{align*}
	F = e^{-\rho}f,\ G = e^{-\rho}g
\end{align*}
and introduce
\begin{align*}
	w = \mqty[
		f \\
		g
	],\ A = \mqty[
		-\tau & Z\p \\
		Z\p & \tau
	],\ B = \mqty[
		1   & \frac{1}{\nu} \\
		\nu & 1
	],
\end{align*}
from which we obtain
\begin{align*}
	\rho\del{}{\rho}{w} = (A + \rho B)w.
\end{align*}

We solve this problem with a Frobenius anzats
\begin{align*}
	w = \rho^{\mu}\sum\limits_{s = 0}^{N}w_{s}\rho^{s},
\end{align*}
where $N$ is some number to be determined. The recursion relation one obtains is
\begin{align*}
	\mu w_{0} = Aw_{0},\ (s + \lambda - A)w_{s} = Bw_{s - 1},
\end{align*}
where $\lambda = \sqrt{\tau^{2} - (Z\p)^{2}}$ is the magnitude of the eigenvalues of $A$. It can be shown that $w$ converges and that $N$ is finite. More specifically it holds that
\begin{align*}
	N + \lambda - \frac{1}{2}Z\p\left(\frac{1}{\nu} - \nu\right) = 0,
\end{align*}
implying
\begin{align*}
	\frac{E}{m} = \frac{1}{\sqrt{1 + \frac{Z^{2}\alpha^{2}}{N + \sqrt{\left(j + \frac{1}{2}\right)^{2} - (Z\alpha)^{2}}}}},\ N + j + \frac{1}{2} = 1, \dots, n,
\end{align*}
where $n$ is termed the little quantum number and $\alpha$ is the fine structure constant.

Expanding this we find
\begin{align*}
	\frac{E}{m} \approx 1 - \frac{1}{2}\frac{Z^{2}\alpha^{2}}{\left(j + \frac{1}{2}\right)^{2}} + \dots = 1 - \frac{1}{2}\frac{Z^{2}\alpha^{2}}{n^{2}}\left(1 + \frac{Z^{2}\alpha^{2}}{n^{2}}\left(\frac{n}{j + \frac{1}{2}} - \frac{3}{4}\right)\right),
\end{align*}
which is similar to the result found with perturbation theory of the non-relativistic problem, except for the angular momentum-dependence of the perturbation.