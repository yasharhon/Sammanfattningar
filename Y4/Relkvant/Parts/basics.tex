\section{Basic Concepts}

\paragraph{Casimir Operators}
A Casimir operator is an operator that is constructed from the generators of a group and commutes with all generators.

\paragraph{Casimir Operators of the Poincare Group}
The Casimir operators of the Poincare group are
\begin{align*}
	P^{2} = P^{\mu}P_{\mu},\ w^{2} = w^{\mu}w_{\mu},
\end{align*}
where we have introduced the Pauli-Lubanski vector
\begin{align*}
	w_{\mu} = \frac{1}{2}\varepsilon_{\mu\nu\rho\sigma}M^{\nu\rho}P^{\sigma}. 
\end{align*}
It can be shown that
\begin{align*}
	w_{0} = \vb{P}\cdot\vb{J},\ \vb{w} = P_{0}\vb{J} + \vb{P}\times\vb{K}.
\end{align*}

\paragraph{The Wigner Classification}
As we will consider unitary representations of the Poincare group acting on states and the representations can be decomposed into irreps, we will find that we can reduce our considerations to a set of fundamental systems, termed particles. The classification, divided according to the eigenvalues of $P^{2}$ and $w^{2}$, is according to the Wigner system:
\begin{enumerate}
	\item $P^{2} > 0$, with subclasses:
	\begin{itemize}
		\item $P^{0} < 0$.
		\item $P^{0} > 0$.
	\end{itemize}
	\item $P^{2} = 0$, with subclasses:
	\begin{itemize}
		\item $P^{0} < 0$.
		\item $P^{0} > 0$.
	\end{itemize}
	\item $P^{2} = 0$ and $P^{0} = 0$.
	\item $P^{2} < 0$, corresponding to tachyons.
\end{enumerate}