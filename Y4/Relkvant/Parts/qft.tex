\section{Introductory Quantum Field Theory}

\paragraph{Quantum Field Theory}
The basic idea of quantum field theory is that operators are generated by operator-valued fields, meaning that operators are localized in this formalism. The interpretation of these fields is that they are creation and annihilation operators at different points. The plan is to obtain a Lagrangian density for a particular classical theory, find the equations of motions and solve the inhomogenous versions to find the propagators (to be discussed).

\paragraph{Propagators}
The propagator is the amplitude of probability from travelling between two points in spacetime. They act as Green's functions for the equations of motion to which they correspond.

One propagator is the Klein-Gordon propagator, defined as
\begin{align*}
	i\kgprop{F}{x} = \expval{\torp{\phi(x)}{\phi(0)}}{0}.
\end{align*}
It can be represented as a Feynman diagram as below
\begin{figure}[!ht]
	\centering
	\feynmandiagram[horizontal = a to b]{
		a[particle = $0$] --[scalar] b[particle = $x$],
	};
\end{figure}
and is interpreted as a particle being created at $0$, propagated to $x$ and annihilated there. Equivalently, it may be interpreted as an antiparticle doing the same in the opposite order.

In Fourier space we may also introduce a propagator according to
\begin{align*}
	i\kgprop{F}{x} = \integ[4]{}{}{k}{\frac{1}{(2\pi)^{4}}e^{-ik^{\mu}x_{\mu}}\frac{i}{k^{2} - m^{2} - i0}},
\end{align*}
and thus define
\begin{align*}
i\kgprop{F}{k} = \frac{i}{k^{2} - m^{2} - i0}.
\end{align*}
Note the constant $i0 = \lim\limits_{\varepsilon = 0}\varepsilon i$. In a Feynman diagram it looks like this:
\begin{figure}[!ht]
	\centering
	\feynmandiagram[horizontal = a to b]{
		a[particle = $0$] --[scalar, momentum' = $k$] b[particle = $x$],
	};
\end{figure}

Another is the Dirac propagator, defined as
\begin{align*}
	i\dirprop{\alpha}{\beta}{x} = \expval{\torp{\psi_{\alpha}(x)}{\bar{\Psi}_{\beta}(0)}}{0}
\end{align*}
in real space and
\begin{align*}
	i\dirprop{\alpha}{\beta}{k} = \frac{i}{k^{2} - m^{2} + i0}(\fsl{k} + m)_{\alpha\beta}
\end{align*}
which is illustrated as
\begin{figure}[!ht]
	\centering
	\feynmandiagram[horizontal = a to b]{
		a[particle = {0, \beta}] --[fermion] b[particle = {x, \alpha}],
	};
\end{figure}
in real space and
\begin{figure}[!ht]
	\centering
	\feynmandiagram[horizontal = a to b]{
		a[particle = {0, \beta}] --[fermion, momentum' = $k$] b[particle = {x, \alpha}],
	};
\end{figure}
in momentum space.

Finally there is the photon propagator
\begin{align*}
	i\phprop{\mu}{\nu}{x} = \expval{\torp{A_{\mu}(x)}{A_{\nu}(0)}}{0},
\end{align*}
in real space and
\begin{align}
	i\phprop{\mu}{\nu}{x} = = -\frac{i}{k^{2} + i0}g_{\mu\nu}
\end{align}
in momentum space. Note that in other gauges one will have to perform the replacement
\begin{align*}
	g_{\mu\nu} \to g_{\mu\nu} + k_{\mu}F_{\nu}(x),\ k_{\nu}F_{\mu}(x),
\end{align*}
where the new terms depend on the gauge. These are represented by
\begin{figure}[!ht]
	\centering
	\feynmandiagram[horizontal = a to b]{
		a[particle = {0, \nu}] --[photon] b[particle = {x, \mu}],
	};
\end{figure}
in real space and
\begin{figure}[!ht]
	\centering
	\feynmandiagram[horizontal = a to b]{
		a[particle = {0, \nu}] --[photon, momentum' = $k$] b[particle = {0, \mu}],
	};
\end{figure}
in momentum space.

\paragraph{Gauge Interactions}
There are three fundamental gauge interactions we will consider: electromagnetic, weak and strong interactions. These will all be mediated by spin-1 particles. We will use a minimal coupling scheme with a general coupling parameter $g$. We will generally find that this adds an interaction term to the Lagrangian density. As an example, for the Dirac Lagrangian density we will find
\begin{align*}
	\lag \to \lag - g\bar{\Psi}\gamma_{\mu}\Psi A^{\mu}.
\end{align*}

\paragraph{The Normal-Ordered Product}
The normal-ordered product of a set of fields is the product rearranged such that for each term, creation operators are found to the right. This implies that the normal-ordered product of fermionic fields carry sign information, whereas bosonic fields do not.

\paragraph{The Time-Ordered Product}
The time-ordered product of a set of fields is the product rearranged such that they appear in order of decreasing time from the left. This implies that the time-ordered product of fermionic fields carry sign information, whereas bosonic fields do not.

\paragraph{The Non-Relativistic String}
Consider a non-relativistic string with a coordinate $x$ along its equilibrium position. To study it, we impose periodic boundary conditions and discretize the string, dividing it into $N$ points separated by a distance $a$. Denoting the displacements at each point as $\phi_{i}$ and assigning each point a mass $m$, the total kinetic energy of the string is
\begin{align*}
	T = \frac{1}{2}m\sum\limits_{i = 0}^{N - 1}\left(\dv{\phi_{i}}{t}\right)^{2}.
\end{align*}
Implementing a simple Hook-like tension in the string we also have a total potential energy
\begin{align*}
	V = \frac{1}{2}k\sum\limits_{i = 0}^{N - 1}\left(\phi_{i + 1} - \phi_{i}\right)^{2}.
\end{align*}

We will now consider a continuum limit, where $N$ goes to infinity and $a$ to zero such that $l = Na$ is constant. Introducing
\begin{align*}
	\mu = \frac{m}{a},\ \tau = ka
\end{align*}
and relabelling the displacements by noting that
\begin{align*}
	\phi_{i}(t) = \phi(t, x_{i}) = \phi(t, ia),
\end{align*}
allowing us to use the label $\phi(t, x)$ in the continuum limit, we find that the kinetic energy is
\begin{align*}
	T = \frac{1}{2}\mu\sum\limits_{i = 0}^{N - 1}a\left(\dv{\phi_{i}}{t}\right)^{2} \to \frac{1}{2}\mu\integ{0}{l}{z}{\left(\del{}{t}{\phi}\right)^{2}}
\end{align*}
and the potential energy is
\begin{align*}
	V = \frac{1}{2}\frac{\tau}{a}\sum\limits_{i = 0}^{N - 1}a^{2}\left(\frac{\phi_{i + 1} - \phi_{i}}{a}\right)^{2} \to \frac{1}{2}\tau\integ{0}{l}{x}{(\del{}{x}{\phi})^{2}}.
\end{align*}
The system is then described by a Lagrangian density
\begin{align*}
	\lag = \frac{1}{2}\mu\left(\del{}{t}{\phi}\right)^{2} - \frac{1}{2}\tau(\del{}{x}{\phi})^{2}.
\end{align*}
Rescaling the field by a factor $\frac{1}{\sqrt{\tau}}$ and introducing $v^{2} = \frac{\tau}{\mu}$ we have
\begin{align*}
	\lag = \frac{1}{2}\left(\frac{1}{v^{2}}\left(\del{}{t}{\phi}\right)^{2} - (\del{}{x}{\phi})^{2}\right).
\end{align*}

The equations of motion is the discrete case are
\begin{align*}
	-k(-\phi_{i + 1} + 2\phi_{i} - \phi_{i - 1}) - m\dv[2]{\phi_{i}}{t} = 0,
\end{align*}
which can be written as
\begin{align*}
	\frac{m}{k}\dv[2]{\phi_{i}}{t} - \left(\phi_{i + 1} - 2\phi_{i} + \phi_{i - 1}\right) = \frac{m}{k}\dv[2]{\phi_{i}}{t} - a^{2}\frac{1}{a}\left(\frac{\phi_{i + 1} - \phi_{i}}{a} - \frac{\phi_{i} - \phi_{i - 1}}{a}\right) = 0.
\end{align*}
We note that
\begin{align*}
	\frac{m}{k} = \frac{a^{2}\mu}{\tau} = \frac{a^{2}}{v^{2}},
\end{align*}
hence
\begin{align*}
	\frac{1}{v^{2}}\dv[2]{\phi_{i}}{t} - \frac{1}{a}\left(\frac{\phi_{i + 1} - \phi_{i}}{a} - \frac{\phi_{i} - \phi_{i - 1}}{a}\right) = 0.
\end{align*}
The equations for the continuum limit are
\begin{align*}
	\frac{1}{v^{2}}\del{2}{t}{\phi} - \del{2}{x}{\phi} = 0,
\end{align*}
which are indeed the continuum limits of the discretized solutions.

The solutions of the equations of motion are so-called normal modes. They are denoted as having positive or negative frequency (or equivalently, energy). The solutions are
\begin{align*}
	\phi_{n} = \frac{1}{\sqrt{l}}e^{i(k_{n}x - \omega_{n}t)},\ \phi_{n}\cc = \frac{1}{\sqrt{l}}e^{-i(k_{n}x - \omega_{n}t)},\ k_{n} = \frac{2\pi n}{l},\ \omega_{n}^{2} = v^{2}k_{n}^{2}.
\end{align*}
Their normalization is
\begin{align*}
	\integ{0}{l}{x}{\phi_{n}\cc\phi_m} = \delta_{nm}.
\end{align*}
A general solution can then be written as
\begin{align*}
	\phi = \sum\limits_{n = -\infty}^{\infty}\frac{c_{n}}{\sqrt{l}}\left(a_{n}(t)e^{ik_{n}x} + a_{n}\cc(t)e^{-ik_{n}x}\right),\ a_{n}(t) = a_{n}(0)e^{-i\omega_{n}t}.
\end{align*}
The two terms correspond to positive and negative frequencies. The expansion coefficients thus satisfy
\begin{align*}
	\dv[2]{a_{n}}{t} + \omega_{n}^{2}a_{n} = 0,
\end{align*}
and we treat them as simple harmonic oscillators.

This treatment is a bit unclear, so let us one-up it by computing the Lagrangian using the above expansion. We have
\begin{align*}
	L &= \integ{0}{l}{x}{\frac{1}{2}\left(\frac{1}{v^{2}}\left(\sum\limits_{n = -\infty}^{\infty}\frac{c_{n}}{\sqrt{l}}\left(\dot{a}_{n}(t)e^{ik_{n}x} + \dot{a}_{n}\cc(t)e^{-ik_{n}x}\right)\right)^{2} - \left(\sum\limits_{n = -\infty}^{\infty}\frac{k_{n}c_{n}}{\sqrt{l}}\left(a_{n}(t)e^{ik_{n}x} - a_{n}\cc(t)e^{-ik_{n}x}\right)\right)^{2}\right)} \\
	 =& \frac{1}{2l}\integ{0}{l}{x}{\frac{1}{v^{2}}\sum\limits_{n = -\infty}^{\infty}\sum\limits_{m = -\infty}^{\infty}c_{n}c_{m}\left(\dot{a}_{n}(t)e^{ik_{n}x} + \dot{a}_{n}\cc(t)e^{-ik_{n}x}\right)\left(\dot{a}_{m}(t)e^{ik_{m}x} + \dot{a}_{m}\cc(t)e^{-ik_{,}x}\right) \\
	  &- \sum\limits_{n = -\infty}^{\infty}\sum\limits_{m = -\infty}^{\infty}k_{n}k_{m}c_{n}c_{m}\left(a_{n}(t)e^{ik_{n}x} - a_{n}\cc(t)e^{-ik_{n}x}\right)\left(a_{m}(t)e^{ik_{m}x} - a_{m}\cc(t)e^{-ik_{m}x}\right)} \\
     =& \frac{1}{2l}\integ{0}{l}{x}{\sum\limits_{n = -\infty}^{\infty}\sum\limits_{m = -\infty}^{\infty}\frac{c_{n}c_{m}}{v^{2}}\left(\dot{a}_{n}\dot{a}_{m}e^{i(k_{n} + k_{m})x} + \dot{a}_{n}\dot{a}_{m}\cc e^{i(k_{n} - k_{m})x} + \dot{a}_{n}\cc\dot{a}_{m}e^{-i(k_{m} - k_{n})x} + \dot{a}_{n}\cc\dot{a}_{m}\cc e^{-i(k_{m} + k_{n})x}\right) \\
      &- k_{n}k_{m}c_{n}c_{m}\left(a_{n}a_{m}e^{i(k_{n} + k_{m})x} - a_{n}a_{m}\cc e^{i(k_{n} - k_{m})x} - a_{m}a_{n}\cc e^{i(k_{m} - k_{n})x} + a_{n}\cc a_{m}\cc e^{-i(k_{n} + k_{m})x}\right)} \\
     =& \frac{1}{2}\sum\limits_{n = -\infty}^{\infty}\sum\limits_{m = -\infty}^{\infty}\frac{c_{n}c_{m}}{v^{2}}\left(\dot{a}_{n}\dot{a}_{m}\delta_{n, -m} + \dot{a}_{n}\dot{a}_{m}\cc\delta_{n, m} + \dot{a}_{n}\cc\dot{a}_{m}\delta_{n, m} + \dot{a}_{n}\cc\dot{a}_{m}\cc\delta_{n, -m}\right) \\
      &- k_{n}k_{m}c_{n}c_{m}\left(a_{n}a_{m}\delta_{n, -m} - a_{n}a_{m}\cc\delta_{n, m} - a_{m}a_{n}\cc\delta_{n, m} + a_{n}\cc a_{m}\cc\delta_{n, -m}\right) \\
     =& \frac{1}{2}\sum\limits_{n = -\infty}^{\infty}\frac{1}{v^{2}}\left(c_{n}c_{-n}\dot{a}_{n}\dot{a}_{-n} + 2c_{n}^{2}\dot{a}_{n}\dot{a}_{n}\cc + c_{n}c_{-n}\dot{a}_{n}\cc\dot{a}_{-n}\cc\right) - \left(k_{n}k_{-n}c_{n}c_{-n}a_{n}a_{-n} - 2k_{n}^{2}c_{n}^{2}a_{n}a_{n}\cc + k_{n}k_{-n}c_{n}c_{-n}a_{n}\cc a_{-n}\cc\right).
\end{align*}
Restricting ourselves to real solutions we find
\begin{align*}
	a_{-n} = a_{n}\cc,\ c_{-n} = c_{n},
\end{align*}
and the Lagrangian may be written as
\begin{align*}
	L =& \frac{1}{2}\sum\limits_{n = -\infty}^{\infty}\frac{4c_{n}^{2}}{v^{2}}\dot{a}_{n}\dot{a}_{-n} - 4k_{n}^{2}c_{n}^{2}a_{n}a_{-n} = 4\sum\limits_{n = 0}^{\infty}\frac{c_{n}^{2}}{v^{2}}\dot{a}_{n}\dot{a}_{-n} - k_{n}^{2}c_{n}^{2}a_{n}a_{-n},
\end{align*}
which is a Lagrangian in a set of discrete degrees of freedom. The corresponding canonical momenta are
\begin{align*}
	p_{\pm n} = \frac{4c_{n}^{2}}{v^{2}}\dot{a}_{\mp n},
\end{align*}
and the Hamiltonian is
\begin{align*}
	H = \sum\limits_{n = 0}^{\infty}\frac{4c_{n}^{2}}{v^{2}}\dot{a}_{\mp n} - 4\sum\limits_{n = 0}^{\infty}\frac{c_{n}^{2}}{v^{2}}\dot{a}_{n}\dot{a}_{-n} - k_{n}^{2}c_{n}^{2}a_{n}a_{-n} = 4\sum\limits_{n = 0}^{\infty}k_{n}^{2}c_{n}^{2}a_{n}a_{-n} = 4\sum\limits_{n = 0}^{\infty}k_{n}^{2}c_{n}^{2}a_{n}a_{n}\cc,
\end{align*}
where we switched to the notation of complex conjugates to emphasize the distinctness of the degrees of freedom. We move the dimensionanlity away from the degrees of freedom by choosing
\begin{align*}
	c_{n}^{2} = \frac{v}{2k_{n}},
\end{align*}
yielding
\begin{align*}
	H = \sum\limits_{n = -\infty}^{\infty}vk_{n}a_{n}\cc a_{n} = \sum\limits_{n = -\infty}^{\infty}\omega_{n}a_{n}\cc a_{n}.
\end{align*}
Introducing new canonical coordinates and momenta
\begin{align*}
	q_{n} = \frac{1}{\sqrt{2\omega_{n}}}(a_{n} + a_{n}\cc),\ p_{n} = i\sqrt{\frac{\omega_{n}}{2}}(a_{n} - a_{n}\cc),
\end{align*}
which invert to
\begin{align*}
	a_{n} = \frac{1}{\sqrt{2\omega_{n}}}(\omega_{n}q_{n} + ip_{n}),\ a_{n}\cc = \frac{1}{\sqrt{2\omega_{n}}}(\omega_{n}q_{n} - ip_{n}),
\end{align*}
we find
\begin{align*}
	H =& \sum\limits_{n = 0}^{\infty}\frac{1}{2}(\omega_{n}q_{n} + ip_{n})(\omega_{n}q_{n} - ip_{n}) \\ 
	  =& \frac{1}{2}\sum\limits_{n = 0}^{\infty}p_{n}^{2} + \omega_{n}^{2}q_{n}^{2},
\end{align*}
which is that of a system of harmonic oscillators. The coordinates and momenta satisfy canonical commutation relations, which means
\begin{align*}
	\pb{a_{n}}{a_{m}}      =& \frac{1}{2\sqrt{\omega_{n}\omega_{m}}}\left(\omega_{n}\omega_{m}\pb{q_{n}}{q_{m}} + i(\omega_{n}\pb{q_{n}}{p_{m}} + \omega_{m}\pb{p_{n}}{q_{m}}) - \pb{p_{n}}{p_{m}}\right)  \\
	                        =& \frac{i}{2\sqrt{\omega_{n}\omega_{m}}}(\omega_{n}\delta_{nm} -  \omega_{m}\delta_{nm}) = 0, \\
	\pb{a_{n}}{a_{m}\cc}    =& \frac{1}{2\sqrt{\omega_{n}\omega_{m}}}\left(\omega_{n}\omega_{m}\pb{q_{n}}{q_{m}} + i(-\omega_{n}\pb{q_{n}}{p_{m}} + \omega_{m}\pb{p_{n}}{q_{m}}) + \pb{p_{n}}{p_{m}}\right) \\
	                        =& \frac{i}{2\sqrt{\omega_{n}\omega_{m}}}\left(-\omega_{n}\delta_{nm} - \omega_{m}\delta_{nm}\right) = -i\delta_{nm}, \\
	\pb{a_{n}\cc}{a_{m}\cc} =& \frac{1}{2\sqrt{\omega_{n}\omega_{m}}}\left(\omega_{n}\omega_{m}\pb{q_{n}}{q_{m}} - i(\omega_{n}\pb{q_{n}}{p_{m}} + \omega_{m}\pb{p_{n}}{q_{m}}) - \pb{p_{n}}{p_{m}}\right) \\
	                        =& -\frac{i}{2\sqrt{\omega_{n}\omega_{m}}}\left(\omega_{n}\delta_{nm} - \omega_{m}\delta_{nm}\right) = 0,
\end{align*}

We also make a brief note of the Hamiltonian. As we know for the discrete system we have $\ham = T + V$, so we would find the Hamiltonian density to be
\begin{align*}
	\ham = \frac{1}{2}\mu\left(\del{}{t}{\phi}\right)^{2} + \frac{1}{2}\tau(\del{}{x}{\phi})^{2}.
\end{align*}
This doesn't really work, however. To do this properly, we will need the canonical momentum density
\begin{align*}
	\pi = \pdv{\lag}{\del{}{t}{\phi}} = \mu\del{}{t}{\phi}.
\end{align*}
The Hamiltonian density is thus
\begin{align*}
	\ham = \frac{1}{\mu}\pi^{2} - \frac{1}{2\mu}\pi^{2} + \frac{1}{2}\tau(\del{}{x}{\phi})^{2} = \frac{1}{2\mu}\pi^{2} + \frac{1}{2}\tau(\del{}{x}{\phi})^{2}.
\end{align*}
Note that these also have canonical Poisson brackets
\begin{align*}
	\pb{\phi(t, x)}{\phi(t, y)} = \pb{\pi(t, x)}{\pi(t, y)} = 0,\ \pb{\pi(t, x)}{\pi(t, y)} = \delta(x - y).
\end{align*}

\paragraph{Quantizing the Non-Relativistic String}
To quantize the relativistic string, we replace the expansion coefficients with operators and impose the canonical commutation relations - that is, the $p$ and $q$ commute and $\comm{q_{m}}{p_{m}} = i\delta_{nm}$. This leads to the $a$ and $a\adj$ commuting, and $\comm{a_{n}}{a_{m}\adj} = \delta_{nm}$. The so-called quantum field is then
\begin{align*}
	\phi = v\sum\limits_{n = -\infty}^{\infty}\frac{1}{\sqrt{2\omega_{n}l}}\left(a_{n}e^{i(k_{n}x - \omega_{n}t)} + a_{n}\adj e^{-i(k_{n}x - \omega_{n}t)}\right) = \phi\pl + \phi\mi,
\end{align*}
where the two terms contain only creation and annihilation operators respectively. The Hamiltonian is now somehow
\begin{align*}
	H = \sum\limits_{n = -\infty}^{\infty}\frac{1}{2}\omega_{n}(a_{n}\adj a_{n} + a_{n}a_{n}\adj) = \sum\limits_{n = -\infty}^{\infty}\omega_{n}\left(a_{n}\adj a_{n} + \frac{1}{2}\right).
\end{align*}
We may subtract the latter term, which is a vacuum energy contribution. Of course, this term does not come when directly quantizing the Hamiltonian we derived above. We may also introduce the number operators $\mathcal{N}_{n} = a_{n}\adj a_{n}$.

\paragraph{Quantizing the Klein-Gordon Field}
We start with the Klein-Gordon Lagrangian
\begin{align*}
	\lag = \frac{1}{2}(\del{}{\mu}{\phi}\del{\mu}{}{\phi} - m^{2}\phi).
\end{align*}
Expanding into eigenfunctions of the space parts, a general solution may be written as
\begin{align*}
	\phi = \sum\limits_{n}q_{n}(t)u_{n}(\vb{x}).
\end{align*}
As we saw in the case of the non-relativistic string, the expansion coefficients evolve according to some Lagrangian. We may thus impose the canonical commutation relations on these by also introducing
\begin{align*}
	p_{m} = \integ[3]{}{}{\vb{x}}{u_{m}\pi},
\end{align*}
where $\pi$ is the momentum density. From these we introduce creation and annihilation operators according to
\begin{align*}
	a_{n} = \frac{1}{\sqrt{2}}(q_{n} + ip_{m}),
\end{align*}
which satisfy the correct commutation relations.
%a\delsw{}{}b = a\del{}{}b - (\del{}{}a)b

To proceed we start with the expansion
\begin{align*}
	\phi = \integ[4]{}{}{k}{a(k)e^{-ikx}\delta(k^{2} - m^{2})}.
\end{align*}
We have introduced a Dirac delta to guarantee that only solutions of the Klein-Gordon equation are included. We can use the properties of the Dirac delta to obtain
\begin{align*}
	\phi = \integ[4]{}{}{k}{a(k)e^{-ikx}\frac{1}{\abs{2k^{0}}}\left(\delta(k^{0} - \omega_{\vb{k}}) + \delta(k^{0} + \omega_{\vb{k}})\right)},
\end{align*}
at least for the purposes of integration, where $\omega_{\vb{k}} = \sqrt{\vb{k}^{2} + m^{2}}$. Integrating we find
\begin{align*}
	\phi = \frac{1}{(2\pi)^{3}}\integ[3]{}{}{\vb{k}}{\frac{1}{2\omega_{\vb{k}}}\left(a_{k}(t)e^{i\vb{k}\cdot\vb{x}} + a_{k}\cc(t)e^{-i\vb{k}\cdot\vb{x}}\right)},
\end{align*}
where we have introduced a particular choice of normalization and done some renaming and relabelling.

We can obtain a nicer expression by introducing coefficients in terms of $3$-momentum. We will take $a_{k} = c_{k}a_{\vb{k}}$. The canonical Poisson bracket is
\begin{align*}
	\pb{a_{\vb{k}}}{a_{\vb{q}}\cc} = -i\delta^{3}(\vb{k} - \vb{q}).
\end{align*}
Taking this to be true, we compute the momentum density
\begin{align*}
	\pi = \frac{i}{(2\pi)^{3}}\integ[3]{}{}{\vb{k}}{\frac{1}{2}\left(-a_{k}(t)e^{i\vb{k}\cdot\vb{x}} + a_{k}\cc(t)e^{-i\vb{k}\cdot\vb{x}}\right)},
\end{align*}
and find
\begin{align*}
	\pb{\phi(t, \vb{x})}{\pi(t, \vb{y})} &= \frac{i}{(2\pi)^{6}}\integ[3]{}{}{\vb{k}}{\integ[3]{}{}{\vb{q}}{\frac{1}{4\omega_{\vb{k}}}\pb{a_{k}(t)e^{i\vb{k}\cdot\vb{x}} + a_{k}\cc(t)e^{-i\vb{k}\cdot\vb{x}}}{-a_{q}(t)e^{i\vb{q}\cdot\vb{y}} + a_{q}\cc(t)e^{-i\vb{q}\cdot\vb{y}}}}} \\
	&= \frac{i}{(2\pi)^{6}}\integ[3]{}{}{\vb{k}}{\integ[3]{}{}{\vb{q}}{\frac{1}{2\omega_{\vb{k}}}e^{i(\vb{k}\cdot\vb{x} - \vb{q}\cdot\vb{y})}\pb{a_{k}(t)}{a_{q}\cc(t)}}} \\
	&= \frac{1}{(2\pi)^{6}}\integ[3]{}{}{\vb{k}}{\frac{\abs{c_{k}}^{2}}{2\omega_{\vb{k}}}e^{i\vb{k}\cdot(\vb{x} - \vb{y})}},
\end{align*}
and the choice
\begin{align*}
	c_{k} = (2\pi)^{\frac{3}{2}}\sqrt{2\omega_{\vb{k}}}
\end{align*}
yields the correct Poisson bracket. We then rewrite the field as
\begin{align*}
	\phi = \frac{1}{(2\pi)^{\frac{3}{2}}}\integ[3]{}{}{\vb{k}}{\frac{1}{\sqrt{2\omega_{\vb{k}}}}\left(a_{\vb{k}}(t)e^{i\vb{k}\cdot\vb{x}} + a_{\vb{k}}\cc(t)e^{-i\vb{k}\cdot\vb{x}}\right)}.
\end{align*}

The above expressions are ripe for quantization. To separate the time dependence from the creation and annihilation operators, we note that the Hamiltonian does not act on functions. As the creation and annihilation operators create eigenstates of the Hamiltonian, we have
\begin{align*}
	a_{\vb{k}}(t) &= e^{iHt}a_{\vb{k}}(0)e^{-iHt}\ket{n_{\vb{k}},\ \text{rest}} \\
	              &= e^{iHt}a_{\vb{k}}(0)e^{-it(n_{\vb{k}}\omega_{\vb{k}} + \text{rest})}\ket{n_{\vb{k}},\ \text{rest}} \\
	              &= \sqrt{n_{\vb{q}}}e^{it((n_{\vb{k}} - 1)\omega_{\vb{k}} + \text{rest})}e^{-it(n_{\vb{k}}\omega_{\vb{k}} + \text{rest})}\ket{n_{\vb{k}} - 1,\ \text{rest}} \\
	              &= e^{-it\omega_{\vb{k}}}\ket{n_{\vb{k}},\ \text{rest}}.
\end{align*}
A similar argument can be performed for the annihilation operators, meaning that the creation and annihilation terms of the field are
\begin{align*}
	\phi_{+} = \integ[3]{}{}{\vb{k}}{\frac{1}{\sqrt{2\omega_{\vb{k}}}}a_{\vb{k}}e^{-ikx}},\ \phi_{-} = \integ[3]{}{}{\vb{k}}{\frac{1}{\sqrt{2\omega_{\vb{k}}}}a\adj_{\vb{k}}e^{ikx}}.
\end{align*}
It is now clear that the two terms commute, even when evaluated at different points. We also have
\begin{align*}
	\comm{\phi_{+}(x)}{\phi_{-}(y)} &= \frac{1}{(2\pi)^{3}}\integ[3]{}{}{\vb{k}}{\integ[3]{}{}{\vb{q}}{\frac{1}{2\sqrt{\omega_{\vb{k}}\omega_{\vb{q}}}}\comm{a(\vb{k})}{a\adj(\vb{q})}e^{-i(kx - qy)}}} \\
	                                &= \frac{1}{(2\pi)^{3}}\integ[3]{}{}{\vb{k}}{\integ[3]{}{}{\vb{q}}{\frac{1}{2\sqrt{\omega_{\vb{k}}\omega_{\vb{q}}}}\delta(\vb{k} - \vb{q})e^{-i(kx - qy)}}} \\
	                                &= \frac{1}{2(2\pi)^{3}}\integ[3]{}{}{\vb{k}}{\frac{1}{\omega_{\vb{k}}}e^{-ik(x - y)}},
\end{align*}
which we define as $\kgprop{+}{x - y}$. Similarly we define
\begin{align*}
	\comm{\phi_{-}(x)}{\phi_{+}(y)} = \kgprop{-}{x - y} = -\frac{1}{2(2\pi)^{3}}\integ[3]{}{}{\vb{k}}{\frac{1}{\omega_{\vb{k}}}e^{ik(x - y)}}.
\end{align*}
This implies
\begin{align*}
	\comm{\phi(x)}{\phi(y)} = -\frac{1}{2(2\pi)^{3}}\integ[3]{}{}{\vb{k}}{\frac{1}{k^{0}}\left(e^{-ik(x - y)} - e^{ik(x - y)}\right)},
\end{align*}
defined to be $i\Delta(x - y)$. It can also be shown that
\begin{align*}
	i\Delta(x) = -\frac{1}{2\pi}\text{sgn}(x^{0})(\delta(x^{\mu}x{\mu}) - \frac{1}{2}m^{2}\theta(x^{\mu}x{\mu})).
\end{align*}

The propagator solves the inhomogenous Klein-Gordon equation, with inhomogeneity $-\delta(x - y)$ for convention. We define its Fourier transform as
\begin{align*}
	G(x - y) = \frac{1}{(2\pi)^{4}}\integ[4]{}{}{k}{e^{-ik(x - y)}\tilde{G}(k)}.
\end{align*}
The Klein-Gordon equation implies
\begin{align*}
	\tilde{G} = \frac{1}{k^{2} - m^{2}}.
\end{align*}
This function has a pole at $k^{0} = \pm \omega$, where $\omega^{2} = \vb{k}^{2} + m^{2}$. To remedy this, we add a term $-i\varepsilon$ to the denominator, shifting the poles from the real axis. Writing
\begin{align*}
	G(x - y) = \frac{1}{(2\pi)^{4}}\integ[3]{}{}{\vb{k}}{e^{i\vb{k}\cdot(\vb{x} - \vb{y})}}\integ{}{}{k^{0}}{\frac{1}{(k^{0})^{2} - \omega^{2} - i\varepsilon}e^{-ik^{0}(x^{0} - y^{0})}}
\end{align*}
for $x^{0} > y^{0}$ and switching the sign of the last integral if the opposite is true, we find that the last integral is
\begin{align}
	-2\pi i\frac{e^{\pm ik^{0}(x^{0} - y^{0})}}{2k^{0}}
\end{align}
for the two cases. The two corresponding functions are dubbed $\pm i\Delta_{\pm}(x - y)$, and we finally have
\begin{align*}
	i\kgprop{F}{x - y} = i(\theta(x^{0} - y^{0})\Delta_{+}(x - y) - \theta(y^{0} - x^{0})\Delta_{+}(x - y)).
\end{align*}

Corresponding to the Klein-Gordon field is an energy-momentum tensor
\begin{align*}
	T^{\mu\nu} = \del{\mu}{}{\pi}\del{\nu}{}{\pi} + \frac{1}{2}g^{\mu\nu}(m^{2}\phi^{2} - \dalem{\phi}),
\end{align*}
which is symmetric and divergence-free. We define the 4-momentum
\begin{align*}
	P^{\mu} = \integ[3]{}{}{\vb{x}}{T^{0\mu}}.
\end{align*}
It can somehow be shown that
\begin{align*}
	P^{\mu} = \integ[3]{}{}{q}{\frac{1}{q^{0}}q^{\mu}a\adj(q)a(q)},
\end{align*}
where $\comm{a(\vb{p})}{a\adj(\vb{q})} = q^{0}\delta(\vb{p} - \vb{q})$. It can be shown that $a\adj(p)\ket{0}$ is an eigenstate of this operator.

\paragraph{Quantizing the Dirac Field}
We start with the Lagrangian of a free Dirac field
\begin{align*}
	\lag = -\frac{1}{2}\bar{\Psi}(-i\fsl{\del{}{}{}} + m)\Psi - \frac{1}{2}(i\del{}{\mu}{\bar{\Psi}}\gamma^{\mu} + m\bar{\Psi})\Psi.
\end{align*}
Another Lagrangian is
\begin{align*}
	\lag_{\text{D}} = \bar{\Psi}(i\fsl{\del{}{}{}} - m)\Psi.
\end{align*}
The corresponding momentum densities are
\begin{align*}
	\pi = \frac{i}{2}\Psi\adj,\ \bar{\pi} = -\frac{i}{2}\gamma^{0}\Psi
\end{align*}
for the first choice and
\begin{align*}
	\pi = i\Psi\adj,\ \bar{\pi} = 0
\end{align*}
for the second. The corresponding energy-momentum tensor is
\begin{align*}
	T^{\mu\nu} = \frac{i}{2}(\bar{\Psi}\gamma^{\mu}\gamma^{\nu}\Psi - (\del{\nu}{}{\Psi}\gamma^{\mu}\Psi)),
\end{align*}
with corresponding momentum densities
\begin{align*}
	P^{\mu} = i\integ[3]{}{}{\vb{x}}{\Psi\adj\gamma^{0}\gamma^{\mu}\Psi}.
\end{align*}

To quantize the field, we expand it as
\begin{align*}
	\Psi     &= \sum\limits_{s}\integ[3]{}{}{\vb{p}}{\frac{1}{(2\pi)^{\frac{3}{2}}}\sqrt{\frac{m}{E(\vb{p})}}(b(\vb{p}, s)u(\vb{p}, s)e^{-ipx} + d\adj(\vb{p}, s)v(\vb{p}, s)e^{ipx})}, \\
	\Psi\adj &= \sum\limits_{s}\integ[3]{}{}{\vb{p}}{\frac{1}{(2\pi)^{\frac{3}{2}}}\sqrt{\frac{m}{E(\vb{p})}}(b\adj(\vb{p}, s)u\adj(\vb{p}, s)e^{ipx} + d(\vb{p}, s)v\adj(\vb{p}, s)e^{-ipx})},
\end{align*}
where we have introduced the particle and antiparticle spinors and creation and annihilation operators $b$ and $d$ for particles and antiparticles. This leads to the Hamiltonian becoming
\begin{align*}
	\ham = \integ[3]{}{}{\vb{p}}{E(\vb{p})\sum\limits_{s}(b\adj b(\vb{p}, s) - dd\adj(\vb{p}, s))}.
\end{align*}
In order to produce positive values, we must therefore choose anticommutation relations
\begin{align*}
	\acomm{d(\vb{p}, s)}{d\adj(\vb{p}\p, s\p)} = \delta(\vb{p} - \vb{p}\p)\delta_{ss\p}.
\end{align*}

Inspired by previous work, we employ the anzats $\dirprop{}{}{x - y} = (i\fsl{\del{}{}{}}_{x} + m)F(x - y)$, implying
\begin{align*}
	(\dalem{} + m^{2})F(x - y) = -\delta(x - y),
\end{align*}
meaning $F$ is one of the propagators of the Klein-Gordon equation.

To introduce coupling to an electromagnetic field, the Dirac equation becomes
\begin{align*}
	(i\fsl{\del{}{}{}} - m)\Psi = q\fsl{A}\Psi
\end{align*}
in the minimal-coupling scheme. This somehow produces the solution
\begin{align*}
	\Psi = \Psi_{\text{in}} + q\integ[4]{}{}{y}{S_{\text{R}}(x - y)\fsl{A}(y)\Psi(y)},\ \Psi = \Psi_{\text{out}} + q\integ[4]{}{}{y}{S_{\text{A}}(x - y)\fsl{A}(y)\Psi(y)}.
\end{align*}
To simplify this we will employ a perturbation scheme
\begin{align*}
	\Psi = \Psi^{(0)} + q\Psi^{(1)} + q^{2}\Psi^{(2)} + \dots,
\end{align*}
which yields
\begin{align*}
	\Psi^{(0)} = \Psi_{\text{in}},\ \Psi^{(1)} = \integ[4]{}{}{y}{S_{\text{R}}(x - y)\fsl{A}(y)\Psi_{\text{in}}(y)},\ \dots
\end{align*}
We also introduce a unitary operator $S$ such that
\begin{align*}
	\Psi_{\text{out}} = S\adj\Psi_{\text{in}}S,
\end{align*}
and expand according to
\begin{align*}
	S = 1 + qS^{(1)} + q^{2}S^{(2)} + \dots
\end{align*}
The unitarity implies that $S^{(1)}$ is anti-Hermitian. Introducing the function $K = S_{\text{R}} - S_{\text{A}}$ we find
\begin{align*}
	\Psi_{\text{out}} = \Psi_{\text{in}} + q\integ[4]{}{}{y}{K(x - y)\fsl{A}\Psi}.
\end{align*}
A first-order expansion of the left-hand side yields
\begin{align*}
	\Psi_{\text{in}} + q((S^{(1)})\adj\Psi_{\text{in}} - \Psi_{\text{in}}(S^{(1)})\adj) = \Psi_{\text{in}} + q\comm{(S^{(1)})\adj}{\Psi_{\text{in}}},
\end{align*}
implying
\begin{align*}
	\comm{(S^{(1)})\adj}{\Psi_{\text{in}}} = \integ[4]{}{}{y}{K(x - y)\fsl{A}\Psi},
\end{align*}
with solution
\begin{align*}
	(S^{(1)})\adj = i\integ[4]{}{}{z}{\nop{\bar{\Psi}_{\text{in}}}{\fsl{A}\Psi_{\text{in}}}}.
\end{align*}

\paragraph{Quantizing the Electromagnetic Field}
The quantization of the electromagnetic field carries a few more difficulties with it. First, the photons, which build up the quantum theory, are massless. Second, photons have a vector character to them.

While the electric and magnetic fields are measurable, and should at first glance be the quantities we replace by operators, we instead do this to the 4-potential. The question of gauge then arises. Starting with Maxwell's equations, written in the two forms
\begin{align*}
	\del{}{\mu}{F^{\mu\nu}} = J^{\nu},\ \dalem{A^{\nu}} - \del{\nu}{}{\del{}{\mu}{A^{\mu}}} = J^{\nu}.
\end{align*}
One choice of gauge is $\del{}{\mu}{A^{\mu}} = 0$, called the Lorenz gauge. Another is the temporal gauge $A^{0} = 0$. Third there is the axial gauge $A^{3} = 0$. Finally there is the Coulomb gauge $\del{}{i}{A^{i}} = 0$. For a free electromagnetic field the temporal and Coulomb gauges can be satisfied simultaneously. The combination of these is known as the radiation gauge. In this gauge we have for a free field that
\begin{align*}
	\dalem{A^{\mu}} = 0.
\end{align*}
The Coulomb gauge requirement in momentum space is $\vb{k}\cdot\vb{A} = 0$. This means that this gauge has no longitudinal photons.

We will quantize the electromagnetic field in four attempts. The first is to employ the familiar non-relativistic scheme
\begin{align*}
	\vb{A} = \frac{1}{\sqrt{2(2\pi)^{3}}}\integ[3]{}{}{\vb{k}}{\frac{1}{\sqrt{\omega}}\sum\limits_{\eta = \pm 1}e^{i(\vb{k}\cdot\vb{x} - \omega t)}\vb*{\varepsilon}(\vb{k}, \eta)a_{\text{nr}}(\vb{k}, \eta) + e^{-i(\vb{k}\cdot\vb{x} - \omega t)}\vb*{\varepsilon}\cc(\vb{k}, \eta)a_{\text{nr}}\adj(\vb{k}, \eta)},
\end{align*}
which guarantees that $\vb{A}$ is Hermitian. The classical interpretation of this is that the field is composed of harmonic oscillators, with commutation relations
\begin{align*}
	\comm{a_{\text{nr}}(\vb{k}, \eta)}{a_{\text{nr}}\adj(\vb{k}\p, \eta\p)} = \delta_{\eta\eta\p}\delta(\vb{k} - \vb{k}\p).
\end{align*}

In our next attempt we introduce $a = \sqrt{k^{0}}a_{\text{nr}}$. This yields
\begin{align*}
	\vb{A} = \frac{1}{\sqrt{2(2\pi)^{3}}}\integ[3]{}{}{\vb{k}}{\frac{1}{k_{0}}\sum\limits_{\eta = \pm 1}e^{-ikx}\vb*{\varepsilon}(\vb{k}, \eta)a(\vb{k}, \eta) + e^{ikx}\vb*{\varepsilon}\cc(\vb{k}, \eta)a\adj(\vb{k}, \eta)},
\end{align*}
as well as the commutation relation
\begin{align*}
	\comm{a(\vb{k}, \eta)}{a\adj(\vb{k}\p, \eta\p)} = k^{0}\delta_{\eta\eta\p}\delta(\vb{k} - \vb{k}\p).
\end{align*}

These formulations have not been covariant. Strochi discovered that one cannot have a covariant quantization and a positive definite metric on Hilbert space at the same time. Our third attempt will be
\begin{align*}
	A^{\mu} = \frac{1}{\sqrt{2(2\pi)^{3}}}\integ[3]{}{}{\vb{k}}{\frac{1}{k_{0}}\sum\limits_{\eta = 1}^{2}\left(e^{-ikx}a(\vb{k}, \eta) + e^{ikx}a\adj(\vb{k}, \eta)\right)e^{\mu}(\vb{k}, \eta)}.
\end{align*}

Finally we try to extend this to something that has an inner product, with the result
\begin{align*}
	A^{\mu} = \frac{1}{\sqrt{2(2\pi)^{3}}}\integ[3]{}{}{\vb{k}}{\frac{1}{k_{0}}\sum\limits_{\lambda = 0}^{3}-g_{\lambda\lambda}\left(e^{-ikx}a(\vb{k}, \lambda) + e^{ikx}a\adj(\vb{k}, \lambda)\right)e^{\mu}(\vb{k}, \lambda)}.
\end{align*}
However, this contains creation and annihilation operators for longitudinal and scalar photons, which have not been found. We need to get rid of them. We also have the commutation relations
\begin{align*}
	\comm{a^{\mu}(\vb{k})}{(a^{\nu})\adj(\vb{k}\p)} = -g^{\mu\nu}k^{0}\delta(\vb{k} - \vb{k}\p)
\end{align*}
for the operators
\begin{align*}
	a^{\mu}(\vb{k}) = -\sum\limits_{\lambda = 0}^{3}g_{\lambda\lambda}e^{\mu}(\vb{k}, \lambda)a(\vb{k}, \lambda).
\end{align*}
This implies
\begin{align*}
	\abs{(a^{0})\adj\ket{0}} = -g^{00}k^{0}\delta(\vb{0}),
\end{align*}
which is negative. Fermi's solution was to switch the interpretations of $a(\vb{k}, 0)$ and $a\adj(\vb{k}, 0)$. Another idea is to just keep the indefinite metric.

\paragraph{More on the Electromagnetic Field}
The classical Lagrangian for an electromagnetic field is
\begin{align*}
	\lag = -\frac{1}{4}F_{\mu\nu}F^{\mu\nu} - j_{\mu}A^{\mu}.
\end{align*}
It turns out that if $j$ is a conserved current, this Lagrangian is invariant under gauge transformations $A^{\mu} \to A^{\mu} + \del{\mu}{}{\chi}$.

By adding a mass term to the Lagrangian, making it of the form
\begin{align*}
	\lag = -\frac{1}{4}F_{\mu\nu}F^{\mu\nu} + \frac{1}{2}M^{2}A_{\mu}A^{\mu} - j_{\mu}A^{\mu},
\end{align*}
we find the equations of motion to be
\begin{align*}
	\del{}{\mu}{F^{\mu\nu}} + M^{2}A^{\nu} = j^{\nu}.
\end{align*}
This implies that the current is only conserved in the Lorenz gauge. Proceeding with this gauge, the equation of motion becomes
\begin{align*}
	(\dalem{} + M^{2})A^{\mu} = j^{\mu}.
\end{align*}
Another possible extension is with a term
\begin{align*}
	\lag = G_{\mu\nu}G^{\mu\nu}, G_{\mu\nu} = \del{}{\mu}{A_{\nu}} + \del{}{\nu}{A_{\mu}},
\end{align*}
which also has gauge issues.

When performing a gauge transformation we assume the field and potentials to transform according to
\begin{align*}
	A_{\mu} \to A_{\mu} + \del{}{\mu}{\chi},\ \Psi \to e^{i\alpha(x)}\Psi,
\end{align*}
where $\alpha = -e\chi$. Introducing the so-called covariant derivative
\begin{align*}
	\cdel{}{\mu} = \del{}{\mu}{} + ieA_{\mu}.
\end{align*}
By applying the gauge transformation we find
\begin{align*}
	\cdel{}{\mu}\Psi \to e^{i\alpha}\cdel{}{\mu}\Psi,
\end{align*}
or $\cdel{}{\mu}\to e^{i\alpha}\cdel{}{\mu}$, which the ordinary derivative does not satisfy. We can also show that
\begin{align*}
	\comm{\cdel{}{\mu}}{\cdel{}{\nu}} = e^{i\alpha}\comm{\cdel{}{\mu}}{\cdel{}{\nu}}\Psi,
\end{align*}
although this commutator is not a covariant derivative. We also find
\begin{align*}
	\comm{\cdel{}{\mu}}{\cdel{}{\nu}} = ieF_{\mu\nu}.
\end{align*}

\paragraph{Classical Yang-Mills Theory}
The idea behind Yang-Mills theory is to construct a doublet of Dirac fields, from now termed $\Psi$, which transforms according to
\begin{align*}
	\Psi \to e^{\frac{i}{2}\sigma^{j}\alpha^{j}(x)}\Psi
\end{align*}
under gauge transformations. We introduce a covariant derivative
\begin{align*}
	\cdel{}{\mu} = \del{}{\mu}{} - \frac{i}{2}gA_{\mu}^{j}\sigma^{j},
\end{align*}
which satisfies
\begin{align*}
	\comm{\cdel{}{\mu}}{\cdel{}{\nu}} = - \frac{i}{2}gF_{\mu\nu}^{j}\sigma^{j},
\end{align*}
where
\begin{align*}
	\frac{i}{2}F_{\mu\nu}^{j}\sigma^{j} = \frac{1}{2}\left(\del{}{\mu}{A_{\nu}^{j}} - \del{}{\nu}{A_{\mu}^{j}}\right)\sigma^{j} - ig\comm{\frac{1}{2}A_{\mu}^{j}\sigma^{j}}{\frac{1}{2}A_{\nu}^{k}\sigma^{k}}.
\end{align*}
This can be solved to yield
\begin{align*}
	F_{\mu\nu}^{i} = \del{}{\mu}{A_{\nu}^{i}} - \del{}{\nu}{A_{\mu}^{i}} + g\varepsilon^{ijk}A_{\mu}^{j}A_{\nu}^{k},
\end{align*}
a so-called non-abelian field strength. It transforms according to
\begin{align*}
	\frac{i}{2}F_{\mu\nu}^{j}\sigma^{j} \to \frac{i}{2}F_{\mu\nu}^{j}\sigma^{j} + \comm{\frac{i}{2}\alpha^{j}\sigma^{j}}{\frac{1}{2}F_{\mu\nu}^{k}\sigma^{k}}.
\end{align*}
The Lagrangian corresponding to this theory is
\begin{align*}
	\lag = -\frac{1}{2}\tr(\left(\frac{1}{2}F_{\mu\nu}^{j}\sigma^{j}\right)^{2}) = -\frac{1}{4}F_{\mu\nu}^{j}F^{\mu\nu, j},
\end{align*}
which is gauge invariant.