\section{Introductory Quantum Field Theory}

\paragraph{The Non-Relativistic String}
Consider a non-relativistic string with a coordinate $x$ along its equilibrium position. To study it, we impose periodic boundary conditions and discretize the string, dividing it into $N$ points separated by a distance $a$. Denoting the displacements at each point as $\phi_{i}$ and assigning each point a mass $m$, the total kinetic energy of the string is
\begin{align*}
	T = \frac{1}{2}m\sum\limits_{i = 0}^{N - 1}\left(\dv{\phi_{i}}{t}\right)^{2}.
\end{align*}
Implementing a simple Hook-like tension in the string we also have a total potential energy
\begin{align*}
	V = \frac{1}{2}k\sum\limits_{i = 0}^{N - 1}\left(\phi_{i + 1} - \phi_{i}\right)^{2}.
\end{align*}

We will now consider a continuum limit, where $N$ goes to infinity and $a$ to zero such that $l = Na$ is constant. Introducing
\begin{align*}
	\mu = \frac{m}{a},\ \tau = ka
\end{align*}
and relabelling the displacements by noting that
\begin{align*}
	\phi_{i}(t) = \phi(t, x_{i}) = \phi(t, ia),
\end{align*}
allowing us to use the label $\phi(t, x)$ in the continuum limit, we find that the kinetic energy is
\begin{align*}
	T = \frac{1}{2}\mu\sum\limits_{i = 0}^{N - 1}a\left(\dv{\phi_{i}}{t}\right)^{2} \to \frac{1}{2}\mu\integ{0}{l}{z}{\left(\del{}{t}{\phi}\right)^{2}}
\end{align*}
and the potential energy is
\begin{align*}
	V = \frac{1}{2}\frac{\tau}{a}\sum\limits_{i = 0}^{N - 1}a^{2}\left(\frac{\phi_{i + 1} - \phi_{i}}{a}\right)^{2} \to \frac{1}{2}\tau\integ{0}{l}{x}{(\del{}{x}{\phi})^{2}}.
\end{align*}
The system is then described by a Lagrangian density
\begin{align*}
	\lag = \frac{1}{2}\mu\left(\del{}{t}{\phi}\right)^{2} - \frac{1}{2}\tau(\del{}{x}{\phi})^{2}.
\end{align*}
Rescaling the field by a factor $\frac{1}{\sqrt{\tau}}$ and introducing $v^{2} = \frac{\tau}{\mu}$ we have
\begin{align*}
	\lag = \frac{1}{2}\left(\frac{1}{v^{2}}\left(\del{}{t}{\phi}\right)^{2} - (\del{}{x}{\phi})^{2}\right).
\end{align*}

The equations of motion is the discrete case are
\begin{align*}
	-k(-\phi_{i + 1} + 2\phi_{i} - \phi_{i - 1}) - m\dv[2]{\phi_{i}}{t} = 0,
\end{align*}
which can be written as
\begin{align*}
	\frac{m}{k}\dv[2]{\phi_{i}}{t} - \left(\phi_{i + 1} - 2\phi_{i} + \phi_{i - 1}\right) = \frac{m}{k}\dv[2]{\phi_{i}}{t} - a^{2}\frac{1}{a}\left(\frac{\phi_{i + 1} - \phi_{i}}{a} - \frac{\phi_{i} - \phi_{i - 1}}{a}\right) = 0.
\end{align*}
We note that
\begin{align*}
	\frac{m}{k} = \frac{a^{2}\mu}{\tau} = \frac{a^{2}}{v^{2}},
\end{align*}
hence
\begin{align*}
	\frac{1}{v^{2}}\dv[2]{\phi_{i}}{t} - \frac{1}{a}\left(\frac{\phi_{i + 1} - \phi_{i}}{a} - \frac{\phi_{i} - \phi_{i - 1}}{a}\right) = 0.
\end{align*}
The equations for the continuum limit are
\begin{align*}
	\frac{1}{v^{2}}\del{2}{t}{\phi} - \del{2}{x}{\phi} = 0,
\end{align*}
which are indeed the continuum limits of the discretized solutions.

The solutions of the equations of motion are so-called normal modes. They are denoted as having positive or negative frequency (or equivalently, energy). The solutions are
\begin{align*}
	\phi_{n} = \frac{1}{\sqrt{l}}e^{i(k_{n}x - \omega_{n}t)},\ \cc{\phi_{n}} = \frac{1}{\sqrt{l}}e^{-i(k_{n}x - \omega_{n}t)},\ k_{n} = \frac{2\pi n}{l},\ \omega_{n}^{2} = v^{2}k_{n}^{2}.
\end{align*}
Their normalization is
\begin{align*}
	\integ{0}{l}{x}{\cc{\phi_{n}}\phi_m} = \delta_{nm}.
\end{align*}
A general solution can then be written as
\begin{align*}
	\phi = \sum\limits_{n = -\infty}^{\infty}\frac{c_{n}}{\sqrt{l}}\left(a_{n}(t)e^{ik_{n}x} + \cc{a_{n}}(t)e^{-ik_{n}x}\right),\ a_{n}(t) = a_{n}(0)e^{-i\omega_{n}t}.
\end{align*}
The expansion coefficients thus satisfy
\begin{align*}
	\dv[2]{a_{n}}{t} + \omega_{n}^{2}a_{n} = 0,
\end{align*}
and we treat them as simple harmonic oscillators.

We also make a brief note of the Hamiltonian. As we know for the discrete system we have $\ham = T + V$, so we would find the Hamiltonian density to be
\begin{align*}
	\ham = \frac{1}{2}\mu\left(\del{}{t}{\phi}\right)^{2} + \frac{1}{2}\tau(\del{}{x}{\phi})^{2}.
\end{align*}
This doesn't really work, however. To do this properly, we will need the canonical momentum density
\begin{align*}
	\pi = \pdv{\lag}{\del{}{t}{\phi}} = \mu\del{}{t}{\phi}.
\end{align*}
The Hamiltonian density is thus
\begin{align*}
	\ham = \frac{1}{\mu}\pi^{2} - \frac{1}{2\mu}\pi^{2} + \frac{1}{2}\tau(\del{}{x}{\phi})^{2} = \frac{1}{2\mu}\pi^{2} + \frac{1}{2}\tau(\del{}{x}{\phi})^{2}.
\end{align*}

When quantizing the non-relativistic strings, we will only quantize the simple harmonic oscillators. It can be shown that
\begin{align*}
	\ham = \sum\limits_{n = -\infty}^{\infty}c_{n}^{2}\frac{2\omega_{n}^{2}}{v^{2}}\cc{a}_{n}a_{n},
\end{align*}
which with an appropriate choice of $c_{n}$ becomes
\begin{align*}
	\ham = \sum\limits_{n = -\infty}^{\infty}\omega_{n}\cc{a}_{n}a_{n}.
\end{align*}
Alternatively, by introducing generalized coordinates and momenta
\begin{align*}
	q_{m} = \frac{1}{\sqrt{2\omega_{n}}}(a_{n} + \cc{a}_{n}),\ p_{m} = -i\sqrt{\frac{\omega_{n}}{2}}(a_{n} - \cc{a}_{n})
\end{align*}
the Hamiltonian becomes
\begin{align*}
	\ham = \frac{1}{2}\sum\limits_{n = -\infty}^{\infty}(p_{n}^{2} + \omega_{n}^{2}q_{n}^{2}).
\end{align*}

\paragraph{Quantizing the Non-Relativistic String}
To quantize the relativistic string, we replace the expansion coefficients with operators and impose the canonical commutation relations - that is, the $p$ and $q$ commute and $\comm{q_{m}}{p_{m}} = i\delta_{nm}$. This leads to the $a$ and $a\adj$ commuting, and $\comm{a_{n}}{a_{m}\adj} = \delta_{nm}$. The so-called quantum field is then
\begin{align*}
	\phi = v\sum\limits_{n = -\infty}^{\infty}\frac{1}{\sqrt{2\omega_{n}l}}\left(a_{n}e^{i(k_{n}x - \omega_{n}t)} + a_{n}\adj e^{-i(k_{n}x - \omega_{n}t)}\right) = \phi\pl + \phi\mi,
\end{align*}
where the two terms contain only creation and annihilation operators respectively. The Hamiltonian is now somehow
\begin{align*}
	H = \sum\limits_{n = -\infty}^{\infty}\frac{1}{2}\omega_{n}(a_{n}\adj a_{n} + a_{n}a_{n}\adj) = \sum\limits_{n = -\infty}^{\infty}\omega_{n}\left(a_{n}\adj a_{n} + \frac{1}{2}\right).
\end{align*}
We may subtract the latter term, which is a vacuum energy contribution. We may also introduce the number operators $\mathcal{N}_{n} = a_{n}\adj a_{n}$.

With the introduction of the canonical momentum density comes a canonical commutation relation
\begin{align*}
	\comm{\phi(x, t)}{\pi(x\p, t)} = i\delta(x - x\p).
\end{align*}

\paragraph{The Normal-Ordered Product}
The normal-ordered product is defined as
\begin{align*}
	N\phi^{1}\phi^{2} = \nop{\phi^{1}}{\phi^{2}} = \phi^{1}\phi^{2} - \expval{\phi^{1}\phi^{2}}{0}.
\end{align*}
One usually splits up the two fields in their creation and annihilation terms. In this division one makes sure to move annihilation operators to the left.

\paragraph{Quantum Field Theory}
The basic idea of quantum field theory is that operators are generated by operator-valued fields, meaning that operators are localized in this formalism. The interpretation of these fields is that they are creation and annihilation operators at different points.

\paragraph{The Time-Ordered Product}
The time-ordered product is defined as
\begin{align}
	\torp{\phi_{1}}{\phi_{2}} = \theta(t_{1} - t_{2})\phi_{1}\phi_{2} + \theta(t_{2} - t_{1})\phi_{2}\phi_{1}
\end{align}
for bosonic fields and
\begin{align*}
	\torp{\psi_{1}}{\psi_{2}} = \theta(t_{1} - t_{2})\psi_{1}\psi_{2} - \theta(t_{2} - t_{1})\psi_{2}\psi_{1}
\end{align*}
for fermionic fields. The time subscripts denote the time coordinates at which the two fields are evaluated.

\paragraph{Propagators}
The propagator is the amplitude of probability from travelling between two points in spacetime. They act as Green's functions for the equations of motion to which they correspond.

One propagator is the Klein-Gordon propagator, defined as
\begin{align*}
	i\kgprop{x} = \expval{\torp{\phi(x)}{\phi(0)}}{0}.
\end{align*}
It can be represented as a Feynman diagram according to
\begin{figure}[!ht]
	\centering
	\feynmandiagram[horizontal = a to b]{
		a[particle = $0$] --[scalar] b[particle = $x$],
	};
\end{figure}
and is interpreted as a particle being created at $0$, propagated to $x$ and annihilated there. Equivalently, it may be interpreted as an antiparticle doing the same in the opposite order.

In Fourier space we may also introduce a propagator according to
\begin{align*}
	i\kgprop{x} = \integ[4]{}{}{k}{\frac{1}{(2\pi)^{4}}e^{-ik^{\mu}x_{\mu}}\frac{i}{k^{2} - m^{2} - i0}},
\end{align*}
and thus define
\begin{align*}
	i\kgprop{k} = \frac{i}{k^{2} - m^{2} - i0}.
\end{align*}
Note the constant $i0 = \lim\limits_{\varepsilon = 0}\varepsilon i$. In a Feynman diagram it looks like this:
\begin{figure}[!ht]
	\centering
	\feynmandiagram[horizontal = a to b]{
		a[particle = $0$] --[scalar, momentum' = $k$] b[particle = $x$],
	};
\end{figure}

Another is the Dirac propagator, defined as
\begin{align*}
	i\dirprop{\alpha}{\beta}{x} = \expval{\torp{\psi_{\alpha}(x)}{\bar{\Psi}_{\beta}(0)}}{0}
\end{align*}
in real space and
\begin{align*}
	i\dirprop{\alpha}{\beta}{k} = \frac{i}{k^{2} - m^{2} + i0}(\fsl{k} + m)_{\alpha\beta}
\end{align*}
which is illustrated as
\begin{figure}[!ht]
	\centering
	\feynmandiagram[horizontal = a to b]{
		a[particle = {0, \beta}] --[fermion] b[particle = {x, \alpha}],
	};
\end{figure}
in real space and
\begin{figure}[!ht]
	\centering
	\feynmandiagram[horizontal = a to b]{
		a[particle = {0, \beta}] --[fermion, momentum' = $k$] b[particle = {x, \alpha}],
	};
\end{figure}
in momentum space.

Finally there is the photon propagator
\begin{align*}
	i\phprop{\mu}{\nu}{x} = \expval{\torp{A_{\mu}(x)}{A_{\nu}(0)}}{0},
\end{align*}
in real space and
\begin{align}
	i\phprop{\mu}{\nu}{x} = = -\frac{i}{k^{2} + i0}g_{\mu\nu}
\end{align}
in momentum space. Note that in other gauges one will have to perform the replacement
\begin{align*}
	g_{\mu\nu} \to g_{\mu\nu} + k_{\mu}F_{\nu}(x),\ k_{\nu}F_{\mu}(x),
\end{align*}
where the new terms depend on the gauge. These are represented by
\begin{figure}[!ht]
	\centering
	\feynmandiagram[horizontal = a to b]{
		a[particle = {0, \nu}] --[photon] b[particle = {x, \mu}],
	};
\end{figure}
in real space and
\begin{figure}[!ht]
	\centering
	\feynmandiagram[horizontal = a to b]{
		a[particle = {0, \nu}] --[photon, momentum' = $k$] b[particle = {0, \mu}],
	};
\end{figure}
in momentum space.

\paragraph{The Plan}
The plan is to now obtain a Lagrangian density for our theory, find the equations of motions and solve the inhomogenous versions to find the propagators.

\paragraph{Gauge Interactions}
There are three fundamental gauge interactions we will consider: electromagnetic, weak and strong interactions. These will all be mediated by spin-1 particles. We will use a minimal coupling scheme with a general coupling parameter $g$. We will generally find that this adds an interaction term to the Lagrangian density. As an example, for the Dirac Lagrangian density we will find
\begin{align*}
	\lag \to \lag - g\bar{\Psi}\gamma_{\mu}\Psi A^{\mu}.
\end{align*}