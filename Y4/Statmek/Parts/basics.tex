\section{Basic Concepts}

\paragraph{Phase Transitions}
Landau introduced the concept that phase transitions are defined by spontaneous symmetry breaking.

\paragraph{Order Parameters}
An order parameter describes spontaneous symmetry breaking. It is zero in one phase and non-zero in another.

\paragraph{Density Matrices}
The probability distribution is of the form
\begin{align*}
	p_{n} = \frac{1}{\sum\limits_{m}P_{m}}P_{n} = \frac{1}{Z}P_{n},
\end{align*}
where the summation is performed over some set of states. We now introduce the density matrix
\begin{align*}
	\rho = \frac{1}{Z}\sum\limits_{n}P_{n}\op{n}{n},
\end{align*}
yielding
\begin{align*}
	\expval{O} = \sum\limits_{n}p_{n}O_{nn} = \sum\limits_{n}\frac{1}{Z}P_{n}\expval{O}{n} = \frac{1}{Z}\sum\limits_{n}\sum\limits_{m}P_{n}\braket{n}{m}\mel{m}{O}{n} = \frac{1}{Z}\sum\limits_{n}\sum\limits_{m}\mel{n}{\rho}{m}\mel{m}{O}{n} = \tr(O\rho).
\end{align*}

As a side note, $\rho$ takes the form
\begin{align*}
	\rho = \frac{1}{Z}e^{-\beta K},
\end{align*}
where $K$ is the Hamiltonian in the canonical ensemble and $H - \mu N$ in the grand canonical ensemble. In these cases, the partition function $Z$ may be computed according to
\begin{align*}
	Z = \sum\limits_{m}P_{m} = \sum\limits_{m}e^{-\beta K_{m}} = \sum\limits_{m}\mel{m}{e^{-\beta K}}{m} = \tr(e^{-\beta K}).
\end{align*}

\paragraph{The Ising Model}
The Ising model is a simple model of magnets. In this model, a magnet is a collection of interacting spins on a lattice under the influence of an external field. Its generalized coordinates are $\sigma_{i}$, which may take the values $\pm 1$, signifying a particular spin pointing up or down. The Hamiltonian is
\begin{align*}
	\ham = -J\sum\limits_{i}\sum\limits_{j = \text{nn}(i)}\sigma_{i}\sigma_{j} - h\sum\limits_{i}\sigma_{i}.
\end{align*}
The inner summation is carried out over the nearest neighbours of site $i$ in the Ising model, but is generally a sum over the whole lattice. The order parameter defining its phase transition is $m = \expval{\sigma_{i}}$.

This model will be used to demonstrate core concepts in the course.

\paragraph{Exact Solution in One Dimension}
To solve the Ising model in one dimension we will impose periodic boundary conditions $\sigma_{N} = \sigma_{0}$. In the absence of an external magnetic field the Hamiltonian is
\begin{align*}
	\ham = -J\sum\limits_{i = 0}^{N - 1}\sigma_{i}\sigma_{i + 1} - h\sum\limits_{i = 0}^{N - 1}\sigma_{i}.
\end{align*}
This yields
\begin{align*}
	e^{-\beta\ham} = \prod\limits_{i = 0}^{N - 1}e^{\beta(J\sigma_{i}\sigma_{i + 1} + h\sigma_{i})}.
\end{align*}
Consider one particular factor $T_{\sigma_{i}\sigma_{i + 1}} = e^{\beta J\sigma_{i}\sigma_{i + 1}}$. If we could compute the trace of any one of these, that would directly give us the partition function. However, this rewriting does not decouple the Hamiltonian, making the trace hard to compute. To do that, let us now treat $T_{\sigma_{i}\sigma_{i + 1}}$ as an operator $T_{\sigma_{i + 1}}$ working solely on $\sigma_{i}$ and fix $\sigma_{i + 1}$. The matrix elements of $\sigma_{i}\sigma_{i + 1}$ are
\begin{align*}
	\mel{i}{\sigma_{i}\sigma_{i + 1}}{j} = ij\sigma_{i + 1}.
\end{align*}
The redefined transfer matrix may thus be written as
\begin{align*}
	T_{\sigma_{i + 1}} =
	\mqty[
		e^{\beta J \sigma_{i + 1} + h} & e^{-\beta J \sigma_{i + 1} - h} \\
		e^{\beta J \sigma_{i + 1} + h} & e^{-\beta J \sigma_{i + 1} - h}.
	]
\end{align*}
Its eigenvalues are the solution to
\begin{align*}
	\left(\lambda - e^{\beta J \sigma_{i + 1} + h}\right)\left(\lambda - e^{-\beta J \sigma_{i + 1} - h}\right) - 1 = 0.
\end{align*}
This sucks

Alternatively, we may write the partition function as
\begin{align*}
	Z = \sum\limits_{\sigma_{0} = \pm 1}\dots \sum\limits_{\sigma_{N - 1} = \pm 1}e^{\left(\beta J\sum\limits_{i = 0}^{N - 1}\sigma_{i}\sigma_{i + 1} + \frac{1}{2}h\sum\limits_{i = 0}^{N - 1}\sigma_{i} + \sigma_{i + 1}\right)}.
\end{align*}
Introducing $t_{\sigma\sigma^{\prime}} = e^{\beta J\sigma\sigma^{\prime} + \frac{1}{2}h(\sigma + \sigma^{\prime})}$ we have
\begin{align*}
	Z = \sum\limits_{\sigma_{0} = \pm 1}\dots \sum\limits_{\sigma_{N - 1} = \pm 1}\prod\limits_{i = 0}^{N - 1}t_{\sigma_{i}\sigma_{i + 1}}.
\end{align*}
Now consider some particular spin and perform the summation over this one first. We obtain
\begin{align*}
	\sum\limits_{\sigma_{j} = \pm 1}t_{\sigma_{j - 1}\sigma_{j}}t_{\sigma_{j}\sigma_{j + 1}} = t_{\sigma_{j - 1}\sigma_{j + 1}}^{2}.
\end{align*}
This process is repeated until you obtain
\begin{align*}
	Z = \tr(t^{N}).
\end{align*}
The matrix representation of the transfer matrix is
\begin{align*}
	t =
	\mqty[
		e^{\beta(J + h)}  & e^{-\beta J} \\
		e^{-\beta J} & e^{\beta(J - h)}
	].
\end{align*}
Its eigenvalues are the solutions to
\begin{align*}
	\left(\lambda - e^{\beta(J + h)}\right)\left(\lambda - e^{\beta(J - h)}\right) - e^{-2\beta J} = 0,
\end{align*}
and are given by
\begin{align*}
	\lambda^{2} - 2e^{\beta J}\cosh(\beta h)\lambda + 2\sinh(2\beta J) &= 0, \\
	\lambda_{\pm}                                                      &= e^{\beta J}\cosh(\beta h) \pm \sqrt{e^{2\beta J}\cosh[2](\beta h) - 2\sinh(2\beta J)} \\
	                                                                   &= e^{\beta J}\cosh(\beta h) \pm \sqrt{e^{2\beta J}\sinh[2](\beta h) + e^{-2\beta J}}.	
\end{align*}

Now that we have the eigenvalues, we identify the partition function as
\begin{align*}
	Z = \lambda_{+}^{N} + \lambda_{-}^{N}.
\end{align*}
This can be further simplified to
\begin{align*}
	Z = \lambda_{+}^{N}\left(1 + \left(\frac{\lambda_{-}}{\lambda_{+}}\right)^{N}\right) \approx \lambda_{+}^{N}.
\end{align*}

Next, the free energy is given by
\begin{align*}
	G = -\kb T\left(N\ln(\lambda_{+}) + \ln(1 + \left(\frac{\lambda_{-}}{\lambda_{+}}\right)^{N})\right).
\end{align*}
The magnetization is given by
\begin{align*}
	m &= -\frac{1}{N}\fix{\pdv{G}{\beta h}}{T} \\
	  &\approx \frac{e^{\beta J}\sinh(\beta h) + \frac{e^{2\beta J}\sinh(\beta h)\cosh(\beta h)}{\sqrt{e^{2\beta J}\sinh[2](\beta h) + e^{-2\beta J}}}}{e^{\beta J}\cosh(\beta h) + \sqrt{e^{2\beta J}\sinh[2](\beta h) + e^{-2\beta J}}} \\
	  &= \sinh(\beta h)\frac{1 + \frac{\cosh(\beta h)}{\sqrt{\sinh[2](\beta h) + e^{-4\beta J}}}}{\cosh(\beta h) + \sqrt{\sinh[2](\beta h) + e^{-4\beta J}}} \\
	  &= \frac{\sinh(\beta h)}{\sqrt{\sinh[2](\beta h) + e^{-4\beta J}}}.
\end{align*}
If $h = 0$ there is no spontaneous magnetization. However, at low temperatures a very small field will produce saturation magnetization. This corresponds to a phase transition at $T = 0$.

Next consider the pair distribution function
\begin{align*}
	g(j) = \expval{\sigma_{0}\sigma_{j}}.
\end{align*}
The error introduced by assuming uncorrelated spins, as will be done later, is
\begin{align*}
	\Gamma(j) = \expval{\sigma_{i}\sigma_{i + j}} - \expval{\sigma_{i}}\expval{\sigma_{i + j}}.
\end{align*}
In a general case with different couplings between spins and without an external field we have
\begin{align*}
	\expval{\sigma_{i}\sigma_{i + j}} &= \frac{1}{Z}\sum\sigma_{i}\sigma_{i + j}e^{\beta \sum\limits_{i = 0}^{N - 1}J_{i}\sigma_{i}\sigma_{i + 1}} \\
	                                  &= \frac{1}{Z}\sum\sigma_{i}\sigma_{i + 1}\sigma_{i + 1}\sigma_{i + 2}\dots\sigma_{i + j - 1}\sigma_{i + j}e^{\beta \sum\limits_{i = 0}^{N - 1}J_{i}\sigma_{i}\sigma_{i + 1}} \\
	                                  &= \frac{1}{Z}\pdv{\beta J_{i}}\dots\pdv{\beta J_{i + j}}Z.
\end{align*}
Using the fact that
\begin{align*}
	Z = 2\prod\limits_{i = 1}^{N}2\cosh(\beta J_{i})
\end{align*}
we obtain
\begin{align*}
	\expval{\sigma_{i}\sigma_{i + j}} = \prod\limits_{k = i}^{i + j}\tanh(\beta J_{k}).
\end{align*}
In one dimension there is no magnetization. In the case where all couplings are the same we obtain
\begin{align*}
	\Gamma(j) = \tanh[j](\beta J) = e^{-\frac{j}{\xi}},
\end{align*}
where we have introduced the correlation length
\begin{align*}
	\xi = -\frac{1}{\ln(\tanh(\beta J))}.
\end{align*}