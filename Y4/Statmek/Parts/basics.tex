\section{Basic Concepts}

\paragraph{Phase Transitions}
Landau introduced the concept that phase transitions are defined by spontaneous symmetry breaking.

\paragraph{Order Parameters}
An order parameter describes spontaneous symmetry breaking. It is zero in one phase and non-zero in another.

\paragraph{The Ising Model}
The Ising model is a simple model of magnets. In this model, a magnet is a collection of interacting spins on a lattice under the influence of an external field. Its generalized coordinates are $\sigma_{i}$, which may take the values $\pm 1$, signifying a particular spin pointing up or down. The Hamiltonian is
\begin{align*}
	\ham = -J\sum\limits_{i}\sum\limits_{j = \text{nn}(i)}\sigma_{i}\sigma_{j} - h\sum\limits_{i}\sigma_{i}.
\end{align*}
The inner summation is carried out over the nearest neighbours of site $i$ in the Ising model, but is generally a sum over the whole lattice. The order parameter defining its phase transition is $m = \expval{\sigma_{i}}$.

This model will be used to demonstrate core concepts in the course.