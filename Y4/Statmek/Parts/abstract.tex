This is a summary of SI2510 Statistical Mechanics.

Someone should probably give a brief statement about what this course is about. My attempt is the following: In the first course we only studied ideal systems in order to develop the machinery. Ideal systems generally have no phase transitions (bosonic gases being the exception, but quantum systems are a whole other can of worms). Having developed the machinery for describing ideal systems, we now wish to extend our studies to systems with interactions in order to describe phase transitions. This can generally not be done analytically as we have previously done, hence this course is dedicated to developing both a better language to understand phase transitions and approximate analytical methods.

I am sorry about these notes. They are unfinished and a mess in general. I blame poor teaching.

On that note, everyone seems very eager to tell you that statistical mechanics is so hard to understand, a proposition that I would counter with the following: No one seems interested in making an effort at teaching statistical mechanics. As an analogy, consider an infant swimming. If you look this up, you will see that babies have a remarkable affinity for swimming and being in water. However, a parent throwing their child in a pool and leaving it can hardly attribute the subsequent and unavoidable drowning to the intricacies of swimming. Likewise, if your teacher attributes their difficulties in teaching to the complexity of the subject matter, you can be sure that they have thrown you into the deep end with no regard for your learning. You are being gaslit. But do not let this discourage you. I have made an honest effort within my abilities to give comprehensive explanations both of what I am doing and why. If you are reading this, I hope that you will take the time to improve upon these notes, both for your own sake and that of future students.