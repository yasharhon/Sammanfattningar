\section{Percolation}

\paragraph{The Idea}
In many contexts the existence of certain behaviour depends on the structure of microscopic clusters. The study of such clusters is called percolation theory.

Percolation models come in two forms: site percolation, in which lattice sites are occupied with some probability $p$, and bond percolation, in which nearest-neighbour bonds are formed with some probability $p$.

\paragraph{Definition of Quantities}
We now define the following quantities for such models:

\begin{itemize}
	\item the occupation probability $p$.
	\item the site percolation probability $p_{\text{c}}$, which is the lowest probability such that infinite clusters form in the thermodynamic limit.
	\item $n_{s}(p)$, which is the number of clusters of size $s$ per lattice site given the occupation probability.
	\item $P(p)$, which is the fraction of occupied sites belonging to a spanning cluster (a cluster which goes from one end of the lattice to the other).
	\item $S(p)$, which is the mean size of finite clusters.
	\item the pair connectedness $C(p, r)$, which is the probability that two occupied sites separated by a distance $r$ belong to the same cluster.
\end{itemize}

Given these definitions, we may infer the following: First, the probability that a particular site is occupied and belongs to a cluster of size $s$ is equal to $sn_{s}(p)$. Next, we must have $P(1) = 1$ and $P(p) = 0$ for $p < p_{\text{c}}$ in the thermodynamic limit. In addition, all sites either belong to a spanning cluster or a cluster of finite size, implying
\begin{align*}
	pP(p) + \sum\limits_{s}sn_{s}(p) = p,
\end{align*}
where the summation is over clusters of finite size. Finally, we have
\begin{align*}
	S(p) = \frac{\sum\limits_{s}s^{2}n_{s}(p)}{\sum\limits_{s}sn_{s}(p)}.
\end{align*}

\paragraph{Scaling Theory}
For doing scaling theory, we postulate that the quantity
\begin{align*}
	G(p) = \sum\limits_{s}n_{s}(p)
\end{align*}
is analogous to the free energy per site in such models. According to this analogy, $P(p)$ plays the role of the order parameter, $S(p)$ the role of the susceptibility and $C(p, r)$ that of the correlation function. Based on this, we define critical exponents as previously. The interesting part is that the critical exponents will not depend on details of our work.

To this we add an extra piece of conjecture, namely that there exists a typical cluster size $s_{\xi}$ close to $p = p_{\text{c}}$ which dominates the critical behaviour of $G, P$ and $S$. We define its critical exponent according to
\begin{align*}
	s_{\xi} \propto \abs{p - p_{\text{c}}}^{-\frac{1}{\sigma}}.
\end{align*}
We also assume that
\begin{align*}
	n_{s}(p) = n_{s}(p_{\text{c}})f\left(\frac{s}{s_{\xi}}\right).
\end{align*}
$f$ must, by construction, approach zero as its argument increases to infinity and approach $1$ as its argument approaches zero. For large $s$, it has been observed that $n_{s}$ behaves as a power law in $s$ with power $-\tau$, which is dependent on dimensionality. Thus we have
\begin{align*}
	n_{s}(p) = s^{-\tau}\phi\left(s\abs{p - p_{\text{c}}}^{\frac{1}{\sigma}}\right).
\end{align*}
We may now write $G$ as
\begin{align*}
	G = \sum\limits_{s}s^{-\tau}\phi\left(s\abs{p - p_{\text{c}}}^{\frac{1}{\sigma}}\right) \approx \abs{p - p_{\text{c}}}^{\frac{\tau - 1}{\sigma}}\integ{0}{\infty}{x}{x\phi(x)}.
\end{align*}
Assuming the integral to converge, we have
\begin{align*}
	\alpha = 2 - \frac{\tau - 1}{\sigma}.
\end{align*}
%TODO: Show
In similar ways one obtains
\begin{align*}
	\gamma = \frac{3 - \tau}{\sigma},\ \beta = \frac{2 - \tau}{\sigma},
\end{align*}
and thus the scaling relationship $\alpha + 2\beta + \gamma = 2$ is satisfied.

%TODO: SHow
Next, we somehow expect that the integral of $C$ should be equal to $S$. This will yield $\gamma = \nu(2 - \eta)$. One may also obtain $d\nu = 2 - \alpha$.