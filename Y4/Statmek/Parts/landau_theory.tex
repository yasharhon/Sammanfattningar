\section{Landau Theory}

\paragraph{The Idea}
Landau's theory is a general theory of phase transitions. The core idea is to series expand the free energy in terms of the order parameter close to the phase transition. This isn't really valid, but this method works nevertheless.

The general form of the series expansion is
\begin{align*}
	G(m, T) = a_{0}(T) + \sum\limits_{i}\frac{1}{i}a_{i}(T)m^{i},
\end{align*}
where certain terms may be zero depending on the symmetry of the system. We assume the order parameter to be finite at equilibrium, meaning that the free energy must be bounded from below. One generally truncates this sum to make it possible to handle, and the free energy being bounded is guaranteed by the highest-order term to be even in $m$ and positive.

\paragraph{Landau Theory of the Ising Model}
For the Ising model we expect the system to be invariant with respect to flipping all spins, hence we have the series expansion
\begin{align*}
	G(m, T) = a_{0}(T) + \sum\limits_{i}\frac{1}{2i}a_{2i}(T)m^{2i}.
\end{align*}
At $T = 0$ you will have $\abs{\vb{m}} = 1$.

Suppose now that as the temperature is lowered, $a_{2}$ is the first coefficient to change sign (at least one coefficient must do this in order for a phase transition to exist). Close to the temperature $T_{\text{c}}$ at which it changes sign, it may be linearized as
\begin{align*}
	a_{2}(T) = a_{2, 0}(T - T_{\text{c}}).
\end{align*}
At equilibrium we have
\begin{align*}
	\pdv{G}{m} = \sum\limits_{i}a_{2i}(T)m^{2i - 1} = 0.
\end{align*}
Assuming that other coefficients are approximately constant close to the transition temperature and approaching $T_{\text{c}}$ from below, where $m$ is small but non-zero, we have
\begin{align*}
	a_{2, 0}(T - T_{\text{c}}) + \sum\limits_{i > 2}a_{2i}(T_{\text{c}})m^{2i - 2} = 0.
\end{align*}
We truncate this sum at order $2$ in $m$ to obtain
\begin{align*}
	m = \sqrt{\frac{a_{2, 0}}{a_{4}(T_{\text{c}})}(T_{\text{c}} - T)},
\end{align*}
which reproduces the correct critical exponent.

We see that the series expansion and our assumption about which coefficients change sign produce a theory with a second-order phase transition. This is profound, and both the nature of the phase transition and the critical exponent are general. This is part of the power of Landau theory. This is also expected as the truncated free energy series is expected to transition between having a single minimum and two minima.

The heat capacity is given by
\begin{align*}
	C = T\fix{\pdv{S}{T}}{h}.
\end{align*}
In the case of a second-order phase transition we have
\begin{align*}
	S = -\pdv{G}{T} = -\dv{a_{0}}{T} - \sum\limits_{i}\frac{1}{2i}\left(\dv{a_{2i}}{T}m^{2i} + a_{2i}\dv{m^{2i}}{T}\right).
\end{align*}
Close to and below the critical temperature we have
\begin{align*}
	C &= T\left(-\dv[2]{a_{0}}{T} - \sum\limits_{i}\frac{1}{2i}\left(\dv[2]{a_{2i}}{T}m^{2i} + \dv{a_{2i}}{T}\dv{m^{2i}}{T} + \dv{a_{2i}}{T}\dv{m^{2i}}{T} + a_{2i}\dv[2]{m^{2i}}{T}\right)\right) \\
	  &= T\left(-\dv[2]{a_{0}}{T} - \sum\limits_{i}\frac{1}{2i}\left(\dv[2]{a_{2i}}{T}m^{2i} + 2\dv{a_{2i}}{T}\dv{m^{2i}}{T} + a_{2i}\dv[2]{m^{2i}}{T}\right)\right).
\end{align*}
We have
\begin{align*}
	&\dv{m^{2i}}{T} = im^{2(i - 1)}\dv{m^{2}}{T} = -im^{2(i - 1)}\frac{a_{2, 0}}{a_{4}(T_{\text{c}})}, \\
	&\dv[2]{m^{2i}}{T} = i(i - 1)m^{2(i - 2)}\frac{a_{2, 0}^{2}}{a_{4}^{2}(T_{\text{c}})},
\end{align*}
and thus
\begin{align*}
	C =& T\left(-\dv[2]{a_{0}}{T} - \sum\limits_{i = 1}^{\infty}\frac{1}{2i}\left(\dv[2]{a_{2i}}{T}m^{2i} - 2i\dv{a_{2i}}{T}\frac{a_{2, 0}}{a_{4}(T_{\text{c}})}m^{2(i - 1)} + a_{2i}i(i - 1)m^{2(i - 2)}\frac{a_{2, 0}^{2}}{a_{4}^{2}(T_{\text{c}})}\right)\right) \\
	  =& -T\dv[2]{a_{0}}{T} + T\frac{a_{2, 0}^{2}}{a_{4}(T_{\text{c}})} - T\sum\limits_{i = 2}^{\infty}\frac{1}{2i}\left(\dv[2]{a_{2i}}{T}m^{2i} - 2i\dv{a_{2i}}{T}\frac{a_{2, 0}}{a_{4}(T_{\text{c}})}m^{2(i - 1)} + a_{2i}i(i - 1)m^{2(i - 2)}\frac{a_{2, 0}^{2}}{a_{4}^{2}(T_{\text{c}})}\right) \\
	  =& -T\dv[2]{a_{0}}{T} + T\frac{a_{2, 0}^{2}}{a_{4}(T_{\text{c}})} - \frac{1}{4}T\left(\dv[2]{a_{4}}{T}m^{4} - 4\dv{a_{4}}{T}\frac{a_{2, 0}}{a_{4}(T_{\text{c}})}m^{2} + 2a_{4}\frac{a_{2, 0}^{2}}{a_{4}^{2}(T_{\text{c}})}\right) \\
	   &- T\sum\limits_{i = 3}^{\infty}\frac{1}{2i}\left(\dv[2]{a_{2i}}{T}m^{2i} - 2i\dv{a_{2i}}{T}\frac{a_{2, 0}}{a_{4}(T_{\text{c}})}m^{2(i - 1)} + a_{2i}i(i - 1)m^{2(i - 2)}\frac{a_{2, 0}^{2}}{a_{4}^{2}(T_{\text{c}})}\right) \\
	  \approx& -T\dv[2]{a_{0}}{T} + T\frac{a_{2, 0}^{2}}{2a_{4}(T_{\text{c}})},
\end{align*}
where we have ignored terms containing the magnetization. Above the critical temperature the magnetization is instead identically zero, netting
\begin{align*}
	C = -T\dv[2]{a_{0}}{T}.
\end{align*}
Hence there is a step in the phase transition of
\begin{align*}
	-\frac{a_{2, 0}^{2}T_{\text{c}}}{2a_{4}(T_{\text{c}})}
\end{align*}
when approaching the phase transition from below.

Suppose instead that $a_{4}$ is the first coefficient to change sign. In this case the free energy might have several local minima, meaning that a discontinuous step in the order parameter might occur. To show that such a step exists, we need to show that $a_{2}(T_{\text{c}}) > 0$. We investigate this by comparing $G(m_{0}, T_{\text{c}})$ to $G(0, T_{\text{c}})$, where $m_{0}$ is the magnetization at the minimum. The phase transition occurs when the two are equal, i.e. when
\begin{align*}
	G(m_{0}, T_{\text{c}}) - G(0, T_{\text{c}}) = \sum\limits_{i}\frac{1}{2i}a_{2i}(T_{\text{c}})m_{0}^{2i} = 0.
\end{align*}
The magnetization corresponds to a minimum of $G$, and thus satisfies
\begin{align*}
	\sum\limits_{i}a_{2i}(T_{\text{c}})m_{0}^{2i - 1} = 0.
\end{align*}
Truncating at order $6$ we have
\begin{align*}
	a_{2}(T_{\text{c}})m_{0} + a_{4}(T_{\text{c}})m_{0}^{3} + a_{6}(T_{\text{c}})m_{0}^{5} = 0,\ \frac{1}{2}a_{2}(T_{\text{c}})m_{0}^{2} + \frac{1}{4}a_{4}(T_{\text{c}})m_{0}^{4} + \frac{1}{6}a_{4}(T_{\text{c}})m_{0}^{6} = 0.
\end{align*}
The non-trivial value of the order parameter satisfies
\begin{align*}
	a_{2}(T_{\text{c}}) + a_{4}(T_{\text{c}})m_{0}^{2} + a_{6}(T_{\text{c}})m_{0}^{4} = 0,\ \frac{1}{2}a_{2}(T_{\text{c}}) + \frac{1}{4}a_{4}(T_{\text{c}})m_{0}^{2} + \frac{1}{6}a_{4}(T_{\text{c}})m_{0}^{4} = 0.
\end{align*}
Combining the equation nets
\begin{align*}
	\frac{1}{2}a_{4}(T_{\text{c}})m_{0}^{2} + \frac{2}{3}a_{6}(T_{\text{c}})m_{0}^{4} &= 0, \\
	\frac{1}{2}a_{4}(T_{\text{c}}) + \frac{2}{3}a_{6}(T_{\text{c}})m_{0}^{2}          &= 0, \\
	m_{0}^{2}                                                                         &= -\frac{3a_{4}(T_{\text{c}})}{4a_{6}(T_{\text{c}})}.
\end{align*}
For this to work, we must have $a_{6}(T_{\text{c}}) > 0$ to keep the global minimum at finite magnetization and $a_{4}(T_{\text{c}}) < 0$ per our assumption of the existence of a local minimum, making the magnetization real. Inserting this into a previous expression yields
\begin{align*}
	a_{2}(T_{\text{c}}) - a_{4}(T_{\text{c}})\frac{3a_{4}(T_{\text{c}})}{4a_{6}(T_{\text{c}})} + a_{6}(T_{\text{c}})\frac{9a_{4}^{2}(T_{\text{c}})}{16a_{6}^{2}(T_{\text{c}})} = a_{2}(T_{\text{c}}) - \frac{3}{4}\frac{a_{4}^{2}(T_{\text{c}})}{a_{6}(T_{\text{c}})} + \frac{9}{16}\frac{a_{4}^{2}(T_{\text{c}})}{a_{6}(T_{\text{c}})} = a_{2}(T_{\text{c}}) - \frac{3}{16}\frac{a_{4}^{2}(T_{\text{c}})}{a_{6}(T_{\text{c}})} = 0,
\end{align*}
and thus
\begin{align*}
	a_{2}(T_{\text{c}}) = \frac{3}{16}\frac{a_{4}^{2}(T_{\text{c}})}{a_{6}(T_{\text{c}})} > 0,
\end{align*}
as we wanted to show.

\paragraph{Non-Symmetric Cases}
Suppose we have some case where this symmetry does not hold. Then we would instead use the series expansion
\begin{align*}
	G(m, T) = a_{0}(T) + \sum\limits_{i}\frac{1}{i}a_{i}(T)m^{i}.
\end{align*}
It might be of interest to remove linear terms. This can be done by introducing a new order parameter $\tilde{m} = m + \Delta$ (the tilde will be omitted from now) where $\Delta$ is chosen appropriately so that
\begin{align*}
	G(m, T) = a_{0}(T) + \sum\limits_{i = 2}\frac{1}{i}a_{i}(T)m^{i}.
\end{align*}
The coefficients have implicitly been modified as well. Truncating the sum at $a_{4}$, the equilibrium magnetization satisfies
\begin{align*}
	a_{2}(T_{\text{c}})m_{0} + a_{3}(T_{\text{c}})m_{0}^{2} + a_{4}(T_{\text{c}})m_{0}^{3} = 0.
\end{align*}
In addition, at the transition point the free energy is equal at the post-transition equilibrium and zero, yielding
\begin{align*}
	\frac{1}{2}a_{2}(T_{\text{c}})m_{0}^{2} + \frac{1}{3}a_{3}(T_{\text{c}})m_{0}^{3} + \frac{1}{4}a_{4}(T_{\text{c}})m_{0}^{4} = 0.
\end{align*}
The non-zero solution satisfies
\begin{align*}
	\frac{1}{3}a_{3}(T_{\text{c}}) + \frac{1}{2}a_{4}(T_{\text{c}})m_{0} = 0,\ m_{0} = -\frac{2}{3}\frac{a_{3}(T_{\text{c}})}{a_{4}(T_{\text{c}})}
\end{align*}
and
\begin{align*}
	a_{2}(T_{\text{c}}) - \frac{2}{3}\frac{a_{3}^{2}(T_{\text{c}})}{a_{4}(T_{\text{c}})} + \frac{4}{9}\frac{a_{3}^{2}(T_{\text{c}})}{a_{4}(T_{\text{c}})} = 0,\ a_{2}(T_{\text{c}}) = \frac{2}{9}\frac{a_{3}^{2}(T_{\text{c}})}{a_{4}(T_{\text{c}})}
\end{align*}
%TODO: Finish

\paragraph{Ginzburg-Landau Theory}
Landau theory characterizes a system in terms of a single order parameter. Ginzburg-Landau theory instead characterizes the system in terms of a field $m(\vb{r})$. This field could be thought of as at any particular point describing the order parameter when calculated based only on the vicinity of that point. In other words, it is a more resolved version of Landau theory.

The order parameter extremizes the free energy, which in this theory is given by
\begin{align*}
	F = \integ[d]{}{}{\vb{x}}{a_{0}(T) + \sum\limits_{i}\frac{1}{2i}a_{2i}(T)m^{2i} + \frac{1}{2}f(\grad{m})^{2}}.
\end{align*}
The series expansion generalize Landau theory, whereas the last term is a simple extra term that gives non-trivial behaviour of $m$. We assume $f > 0$ as fluctuations should add to the free energy.

The corresponding extensive variable (the external field in the Ising model) is in this theory given by
\begin{align*}
	h = \fdv{F}{m}.
\end{align*}
We have
\begin{align*}
	\var{F} = \integ[d]{}{}{\vb{x}}{\var{m}\sum\limits_{2i}a_{2i}(T)m^{2i - 1} + f\grad{(\var{m})}\cdot\grad{m}}.
\end{align*}
Fixing boundary conditions and integrating by parts yields
\begin{align*}
	\var{F} = \integ[d]{}{}{\vb{x}}{\var{m}\left(\sum\limits_{i}a_{2i}(T)m^{2i - 1} - f\laplacian{m}\right)},
\end{align*}
and finally
\begin{align*}
	h = \sum\limits_{i}a_{2i}(T)m^{2i - 1} - f\laplacian{m}.
\end{align*}
Close to a second-order phase transition we may re-obtain the result from Landau theory, namely
\begin{align*}
	m_{0}^{2} = -\frac{a_{2}(T)}{a_{4}(T)},
\end{align*}
by setting $h = 0$.

Ginzburg-Landau theory allow us to study fluctuations, which is what we will do next. Suppose we add some small perturbation $h_{0}\delta(\vb{x})$ from $h = 0$, which changes the field to $m_{0} + \phi$, where $m_{0}$ depends on temperature and $\phi$ on position. Truncating the sum at $i = 2$ we obtain
\begin{align*}
	h_{0}\delta(\vb{x}) = a_{2}(T)(m_{0} + \phi) + a_{4}(T)(m_{0} + \phi)^{3} - f\laplacian{m_{0}} - f\laplacian{\phi}.
\end{align*}
Neglecting higher-order terms in $\phi$ we obtain
\begin{align*}
	\laplacian{\phi} - \frac{a_{2}(T)}{f}\phi - \frac{a_{2}(T)}{f}m_{0} - \frac{3a_{4}(T)m_{0}^{2}}{f}\phi - \frac{a_{4}(T)}{f}m_{0}^{3} = -\frac{h_{0}}{f}\delta(\vb{x}).
\end{align*}
This simplifies to
\begin{align*}
	\laplacian{\phi} + \frac{2a_{2}(T)}{f}\phi = -\frac{h_{0}}{f}\delta(\vb{x})
\end{align*}
below the critical temperature and
\begin{align*}
	\laplacian{\phi} - \frac{a_{2}(T)}{f}\phi = -\frac{h_{0}}{f}\delta(\vb{x})
\end{align*}
above the critical temperature. The solution to these equations is
\begin{align*}
	\phi = \frac{h_{0}}{4\pi f}\frac{e^{-\frac{r}{\xi}}}{r},
\end{align*}
where
\begin{align*}
	\xi =
	\begin{cases}
		\sqrt{\frac{f}{a_{2}(T)}},\ T > T_{\text{c}}, \\
		\sqrt{-\frac{f}{2a_{2}(T)}},\ T < T_{\text{c}}.
	\end{cases}
\end{align*}
$\xi$ is the correlation length, and according to the linearization $a_{2} = a_{2, 0}(T - T_{\text{c}})$ it diverges when approaching the phase transition.

The perturbation $\phi$ may be related to a correlation function. To see this, we add a term
\begin{align*}
	-\integ[3]{}{}{\vb{x}}{mh}
\end{align*}
to the Hamiltonian, yielding
\begin{align*}
	\expval{m} = \frac{\tr(me^{-\beta\left(\ham - \integ[3]{}{}{\vb{x}}{mh}\right)})}{\tr(e^{-\beta\left(\ham - \integ[3]{}{}{\vb{x}}{mh}\right)})}.
\end{align*}
This implies
\begin{align*}
	\fdv{\expval{m}}{h_{0}} &= \frac{\beta \tr(m(\vb{0})me^{-\beta\left(\ham - \integ[3]{}{}{\vb{x}}{mh}\right)})\tr(e^{-\beta\left(\ham - \integ[3]{}{}{\vb{x}}{mh}\right)}) - \beta\tr(me^{-\beta\left(\ham - \integ[3]{}{}{\vb{x}}{mh}\right)})\tr(m(\vb{0})e^{-\beta\left(\ham - \integ[3]{}{}{\vb{x}}{mh}\right)})}{\left(\tr(e^{-\beta\left(\ham - \integ[3]{}{}{\vb{x}}{mh}\right)})\right)^{2}} \\
	                       &= \beta(\expval{m(\vb{0})m} - \expval{m(\vb{0})}\expval{m}) \\
	                       &= \beta\Gamma(\vb{r}).
\end{align*}
As $\phi$ contains the entire effect of the perturbation, this is equal to $\frac{\phi}{h_{0}}$, hence $\phi$ is an order parameter correlation function. The susceptibility is given by
\begin{align*}
	\chi = \integ[3]{}{}{\vb{x}}{\beta\Gamma(\vb{r})},
\end{align*}
and this implies that the mean-field result $\chi\propto\abs{T_{\text{c}} - T}^{-1}$ is obtained.

\paragraph{The Ginzburg Criterion}
The Ginzburg criterion is a self-consistency criterion for mean-field or Landau theories.

To obtain it, we generalize to $d$ dimensions. In such cases $\phi$ will generally have a different form, but we may still use the order-of-magnitude approximation
\begin{align*}
	\phi = \frac{e^{-\frac{r}{\xi}}}{r^{d - 2}}.
\end{align*}
We would like to crudely approximate the correlation function at large distances. This is expected to be valid if the correlation function is small compared to the overall order parameter, which is satisfied if
\begin{align*}
	\frac{\integ[d]{\Omega(\xi)}{}{\vb{x}}{\expval{m(\vb{0})m} - \expval{m(\vb{0})}\expval{m}}}{\integ[d]{\Omega(\xi)}{}{\vb{x}}{m_{0}^{2}}} \ll 1,
\end{align*}
where $\Omega(\xi)$ is the $d$-dimensional hypersphere of radius $\xi$. I believe this is the Ginzburg criterion.

We will now use the Ginzburg criterion to try to estimate the dimensionality for which Landau theory correctly predicts the critical behaviour. Using our estimate of the correlation function and introducing the critical exponent for the order parameter, we have
\begin{align*}
	\frac{\integ[d]{\Omega(\xi)}{}{\vb{x}}{\frac{e^{-\frac{r}{\xi}}}{r^{d - 2}}}}{\integ[d]{\Omega(\xi)}{}{\vb{x}}{\abs{T_{\text{c}} - T}^{2\beta}}} \ll 1
\end{align*}
when close to criticality. Computing this in spherical coordinates yields
\begin{align*}
	\frac{d\integ{0}{\xi}{r}{re^{-\frac{r}{\xi}}}}{\xi^{d}\abs{T_{\text{c}} - T}^{2\beta}} = \frac{d\xi^{2}\integ{0}{1}{u}{ue^{-u}}}{\xi^{d}\abs{T_{\text{c}} - T}^{2\beta}} = \xi^{2 - d}\abs{T_{\text{c}} - T}^{-2\beta}d\integ{0}{1}{u}{ue^{-u}}.
\end{align*}
Introducing the critical exponent $\nu$ for the correlation length, the left-hand side is proportional to
\begin{align*}
	\abs{T_{\text{c}} - T}^{2\beta + (d - 2)\nu}.
\end{align*}
The inequality is thus satisfied if and only if
\begin{align*}
	d > 2 + \frac{2\beta}{\nu}.
\end{align*}