\section{Ginzburg-Landau Theory}

\paragraph{Formulation of Ginzburg-Landau Theory}
Landau theory characterizes a system in terms of a single order parameter. Ginzburg-Landau theory instead characterizes the system in terms of a field $m(\vb{r})$. This field could be thought of as at any particular point describing the order parameter when calculated based only on the vicinity of that point. In other words, it is a more resolved version of Landau theory.

The order parameter extremizes the free energy, which in this theory is given by
\begin{align*}
	F = \integ[d]{}{}{\vb{x}}{a_{0}(T) + \sum\limits_{i}\frac{1}{2i}a_{2i}(T)m^{2i} + \frac{1}{2}f(\grad{m})^{2}}.
\end{align*}
The series expansion generalize Landau theory, whereas the last term is a simple extra term that gives non-trivial behaviour of $m$. We assume $f > 0$ as fluctuations should add to the free energy.

The corresponding extensive variable (the external field in the Ising model) is in this theory given by
\begin{align*}
	h = \fdv{F}{m}.
\end{align*}
We have
\begin{align*}
	\var{F} = \integ[d]{}{}{\vb{x}}{\var{m}\sum\limits_{2i}a_{2i}(T)m^{2i - 1} + f\grad{(\var{m})}\cdot\grad{m}}.
\end{align*}
Fixing boundary conditions and integrating by parts yields
\begin{align*}
	\var{F} = \integ[d]{}{}{\vb{x}}{\var{m}\left(\sum\limits_{i}a_{2i}(T)m^{2i - 1} - f\laplacian{m}\right)},
\end{align*}
and finally
\begin{align*}
	h = \sum\limits_{i}a_{2i}(T)m^{2i - 1} - f\laplacian{m}.
\end{align*}
Close to a second-order phase transition we may re-obtain the result from Landau theory, namely
\begin{align*}
	m_{0}^{2} = -\frac{a_{2}(T)}{a_{4}(T)},
\end{align*}
by setting $h = 0$.

Ginzburg-Landau theory allow us to study fluctuations, which is what we will do next. Suppose we add some small perturbation $h_{0}\delta(\vb{x})$ from $h = 0$, which changes the field to $m_{0} + \phi$, where $m_{0}$ depends on temperature and $\phi$ on position. Truncating the sum at $i = 2$ we obtain
\begin{align*}
	h_{0}\delta(\vb{x}) = a_{2}(T)(m_{0} + \phi) + a_{4}(T)(m_{0} + \phi)^{3} - f\laplacian{m_{0}} - f\laplacian{\phi}.
\end{align*}
Neglecting higher-order terms in $\phi$ we obtain
\begin{align*}
	\laplacian{\phi} - \frac{a_{2}(T)}{f}\phi - \frac{a_{2}(T)}{f}m_{0} - \frac{3a_{4}(T)m_{0}^{2}}{f}\phi - \frac{a_{4}(T)}{f}m_{0}^{3} = -\frac{h_{0}}{f}\delta(\vb{x}).
\end{align*}
This simplifies to
\begin{align*}
	\laplacian{\phi} + \frac{2a_{2}(T)}{f}\phi = -\frac{h_{0}}{f}\delta(\vb{x})
\end{align*}
below the critical temperature and
\begin{align*}
	\laplacian{\phi} - \frac{a_{2}(T)}{f}\phi = -\frac{h_{0}}{f}\delta(\vb{x})
\end{align*}
above the critical temperature. The solution to these equations is
\begin{align*}
	\phi = \frac{h_{0}}{4\pi f}\frac{e^{-\frac{r}{\xi}}}{r},
\end{align*}
where
\begin{align*}
	\xi =
	\begin{cases}
		\sqrt{\frac{f}{a_{2}(T)}},\ T > T_{\text{c}}, \\
		\sqrt{-\frac{f}{2a_{2}(T)}},\ T < T_{\text{c}}.
	\end{cases}
\end{align*}
$\xi$ is the correlation length, and according to the linearization $a_{2} = a_{2, 0}(T - T_{\text{c}})$ it diverges when approaching the phase transition.

The perturbation $\phi$ may be related to a correlation function. To see this, we add a term
\begin{align*}
	-\integ[3]{}{}{\vb{x}}{mh}
\end{align*}
to the Hamiltonian, yielding
\begin{align*}
	\expval{m} = \frac{\tr(me^{-\beta\left(\ham - \integ[3]{}{}{\vb{x}}{mh}\right)})}{\tr(e^{-\beta\left(\ham - \integ[3]{}{}{\vb{x}}{mh}\right)})}.
\end{align*}
This implies
\begin{align*}
	\fdv{\expval{m}}{h_{0}} &= \frac{\beta \tr(m(\vb{0})me^{-\beta\left(\ham - \integ[3]{}{}{\vb{x}}{mh}\right)})\tr(e^{-\beta\left(\ham - \integ[3]{}{}{\vb{x}}{mh}\right)}) - \beta\tr(me^{-\beta\left(\ham - \integ[3]{}{}{\vb{x}}{mh}\right)})\tr(m(\vb{0})e^{-\beta\left(\ham - \integ[3]{}{}{\vb{x}}{mh}\right)})}{\left(\tr(e^{-\beta\left(\ham - \integ[3]{}{}{\vb{x}}{mh}\right)})\right)^{2}} \\
	                       &= \beta(\expval{m(\vb{0})m} - \expval{m(\vb{0})}\expval{m}) \\
	                       &= \beta\Gamma(\vb{r}).
\end{align*}
As $\phi$ contains the entire effect of the perturbation, this is equal to $\frac{\phi}{h_{0}}$, hence $\phi$ is an order parameter correlation function. The susceptibility is given by
\begin{align*}
	\chi = \integ[3]{}{}{\vb{x}}{\beta\Gamma(\vb{r})},
\end{align*}
and this implies that the mean-field result $\chi\propto\abs{T_{\text{c}} - T}^{-1}$ is obtained.

\paragraph{The Ginzburg Criterion}
The Ginzburg criterion is a self-consistency criterion for mean-field or Landau theories.

To obtain it, we generalize to $d$ dimensions. In such cases $\phi$ will generally have a different form, but we may still use the order-of-magnitude approximation
\begin{align*}
	\phi = \frac{e^{-\frac{r}{\xi}}}{r^{d - 2}}.
\end{align*}
We would like to crudely approximate the correlation function at large distances. This is expected to be valid if the correlation function is small compared to the overall order parameter, which is satisfied if
\begin{align*}
	\frac{\integ[d]{\Omega(\xi)}{}{\vb{x}}{\expval{m(\vb{0})m} - \expval{m(\vb{0})}\expval{m}}}{\integ[d]{\Omega(\xi)}{}{\vb{x}}{m_{0}^{2}}} \ll 1,
\end{align*}
where $\Omega(\xi)$ is the $d$-dimensional hypersphere of radius $\xi$. I believe this is the Ginzburg criterion.

We will now use the Ginzburg criterion to try to estimate the dimensionality for which Landau theory correctly predicts the critical behaviour. Using our estimate of the correlation function and introducing the critical exponent for the order parameter, we have
\begin{align*}
	\frac{\integ[d]{\Omega(\xi)}{}{\vb{x}}{\frac{e^{-\frac{r}{\xi}}}{r^{d - 2}}}}{\integ[d]{\Omega(\xi)}{}{\vb{x}}{\abs{T_{\text{c}} - T}^{2\beta}}} \ll 1
\end{align*}
when close to criticality. Computing this in spherical coordinates yields
\begin{align*}
	\frac{d\integ{0}{\xi}{r}{re^{-\frac{r}{\xi}}}}{\xi^{d}\abs{T_{\text{c}} - T}^{2\beta}} = \frac{d\xi^{2}\integ{0}{1}{u}{ue^{-u}}}{\xi^{d}\abs{T_{\text{c}} - T}^{2\beta}} = \xi^{2 - d}\abs{T_{\text{c}} - T}^{-2\beta}d\integ{0}{1}{u}{ue^{-u}}.
\end{align*}
Introducing the critical exponent $\nu$ for the correlation length, the left-hand side is proportional to
\begin{align*}
	\abs{T_{\text{c}} - T}^{2\beta + (d - 2)\nu}.
\end{align*}
The inequality is thus satisfied if and only if
\begin{align*}
	d > 2 + \frac{2\beta}{\nu}.
\end{align*}