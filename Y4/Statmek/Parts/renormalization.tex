\section{Renormalization}

\paragraph{Decimation of the Ising Chain}
Before proceeding, we define the dimensionless operator
\begin{align*}
	H = -\beta\ham = K\sum\limits_{i}\sum\limits_{j = \text{nn}(i)}\sigma_{i}\sigma_{j} + \frac{1}{2}h\sum\limits_{i}\sigma_{i} + \sigma_{i + 1}.
\end{align*}
When computing the partition function, we may do this by first summing over spins with odd indices. For any one of these, we obtain the (partial) sum
\begin{align*}
	2\cosh(K(\sigma_{i - 1} + \sigma_{i + 1}) + h).
\end{align*}
Next we may write
\begin{align*}
	2e^{\frac{1}{2}h(\sigma_{i - 1} + \sigma_{i + 1})}\cosh(K(\sigma_{i - 1} + \sigma_{i + 1}) + h) &= e^{2g + K^{\prime}\sigma_{i - 1}\sigma_{i + 1} + \frac{1}{2}h^{\prime}(\sigma_{i - 1} + \sigma_{i + 1})}
\end{align*}
for quantities
\begin{align*}
	K^{\prime} &= \frac{1}{4}\ln(\frac{\cosh(2K + h)\cosh(2K - h)}{\cosh[2](h)}), \\
	h^{\prime} &= h + \frac{1}{2}\ln(\frac{\cosh(2K + h)}{\cosh(2K - h)}), \\
	g          &= \frac{1}{8}\ln(16\cosh(2K + h)\cosh(2K - h)\cosh[2](h)).
\end{align*}
Thus the trace over odd-numbered spins is
\begin{align*}
	e^{Ng + K^{\prime}\sum\limits_{i}\sigma_{2i}\sigma_{2i + 2} + h^{\prime}\sum\limits_{i}\sigma_{2i}}.
\end{align*}
When computing the trace over the even-numbered spins, we notice that it has the same structure as for the total chain, apart from the modified constants. We thus obtain
\begin{align*}
	Z(N, K, h) = e^{Ng(K, h)}Z\left(\frac{1}{2}N, K^{\prime}, h^{\prime}\right).
\end{align*}
We may repeat this procedure indefinitely. Hence we obtain
\begin{align*}
	-\frac{\beta G}{N} = \sum\limits_{j = 0}^{\infty}\frac{1}{2^{j}}g(K_{j}, h_{j}).
\end{align*}

\paragraph{A More General Discussion}
The more general approach is to consider a model on a $d$-dimensional lattice with a set of degrees of freedom $\sigma_{i}$ and coupling constants $K_{\alpha}$, and suppose that a transform of the kind we have discussed, now termed a renormalization transform, preserves the form of the Hamiltonian. We describe the system with the dimensionless quantity
\begin{align*}
	H = -\beta\ham = \sum\limits_{\alpha = 1}^{n}K_{\alpha}\psi_{\alpha}(\sigma)
\end{align*}
where the functions $\psi_{\alpha}(\sigma)$ describe one particular kind of interaction. As an example, the Ising model with coupling all over the lattice may have one function $\psi_{1}$ which contains all nearest-neighbour interactions, one function $\psi_{2}$ which describes next-to-nearest neighbours and so on. The renormalization transforms the Hamiltonian to
\begin{align*}
	H^{\prime} = Ng(K_{\alpha}) + \sum\limits_{\alpha = 1}^{n}K_{\alpha}^{\prime}\psi_{\alpha}(\sigma^{\prime}),
\end{align*}
where $\sigma^{\prime}$ is the set of reduced degrees of freedom, which must have the same algebraic property as the non-reduced ones, and we have new coupling coefficients which are functions of the old. Supposing that the number of degrees of freedom is reduced by a factor $b^{d}$ we obtain
\begin{align*}
	-\frac{\beta}{N}G\left(N, K_{\alpha}\right) = f\left(K_{\alpha}\right) = g(K_{\alpha}) + b^{-d}f\left(K_{\alpha}^{\prime}\right).
\end{align*}

We will now study the critical points of such model. To do this, consider a model described by two coupling constants. The set of critical points for this model will form a line in coupling constant space. Under a renormalization transform, points in this space are made to flow. This flow cannot approach the critical line, as the correlation length there is infinite and generally decreased by a renormalization transform. The long-range order may not be changed by such transforms either, hence the flow cannot make points cross the critical line. Certain points on the critical line may be fixed points, but generally all of them are not. To proceed, consider some particular fixed point $K_{1}^{\star}, K_{2}^{\star}$ and a flow close to this point. We introduce displacements $\Delta K$ from the critical point, and obtain to first order
\begin{align*}
	K_{1}^{\prime} = R_{1}(K_{1}^{\star} + \Delta K_{1}, K_{2}^{\star} + \Delta K_{2}) = K_{1}^{\star} + \Delta K_{1}\pdv{R_{1}}{K_{1}} + \Delta K_{2}\pdv{R_{1}}{K_{2}},
\end{align*}
with a similar expression for $K_{2}\prime$. Introducing the matrix
\begin{align*}
	M_{ij} = \pdv{R_{i}}{K_{j}}
\end{align*}
and the displacements
\begin{align*}
	\Delta K_{1}^{\prime} = K_{1}^{\prime} - K_{1}^{\star}
\end{align*}
we have
\begin{align*}
	\Delta K_{1}^{\prime} = M_{11}\Delta K_{1} + M_{12}\Delta K_{2},\ \Delta K_{2}^{\prime} = M_{21}\Delta K_{1} + M_{22}\Delta K_{2}.
\end{align*}
To continue, we change variables as determined by the eigenvectors of $M$. Naming these $\vb*{\phi}_{1}$ and $\vb*{\phi}_{2}$, the new variables, called scaling fields, are
\begin{align*}
	U_{i} = \phi_{1, i}\Delta K_{1} + \phi_{2, i}\Delta K_{2}.
\end{align*}
One somehow obtains
\begin{align*}
	U_{i}^{\prime} = b^{y_{i}}U_{i}.
\end{align*}

As we have assumed that points on the critical line flow towards the fixed point, one of the $y_{i}$ must be negative while the other is positive. This also means that one of the eigenvectors must be tangential to the critical line at the fixed point. The possibility $y_{i} = 0$ corresponds to a line of fixed points.

Returning to the energy function, we expect $g$ to be analytic, which somehow implies that $f$ satisfies the recursion relation
\begin{align*}
	f\left(K_{\alpha}\right) = b^{-d}f\left(K_{\alpha}^{\prime}\right).
\end{align*}
Close to the critical point, this may be written as
\begin{align*}
	f(U_{1}^{\prime}, U_{2}^{\prime}) = b^{-d}f\left(b^{y_{1}}U_{1}^{\prime}, b^{y_{2}}U_{2}^{\prime}\right)
\end{align*}
Suppose now that $U_{1}\propto t$ while $U_{2}$ is approximately unchanged by the temperature. Then
\begin{align*}
	f(t, U_{2}^{\prime}) = b^{-d}f\left(b^{y_{1}}t, b^{y_{2}}U_{2}^{\prime}\right).
\end{align*}
Hence $f\propto\abs{t}̂^{\frac{d}{y_{1}}}$.

Two things may be concluded from this. First, the critical exponents are determined by the character of the fixed point. Second, it turns out that the flow is independent of $U_{2}$. Hence, all systems whose Hamiltonians flow to the same critical fixed points have the same critical exponents, which is reminiscent of the previously studied universality arguments.

The previous analysis generalizes to higher-dimensional coupling constant spaces.