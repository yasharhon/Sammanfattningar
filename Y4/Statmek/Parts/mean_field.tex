\section{Mean-Field Theory}

\paragraph{The Idea}
The idea between mean-field theory is to separate the Hamiltonian by for any component of the system replace the ones with which it is interacting by an expectation value. Alternatively one may also replace the interactions themselves by mean values. This way a partition function may be computed more easily.

One usually requires self-consistency, meaning that expectation values should be computable from this partition function. This implies that the partition function implicitly gives the expectation values which were used to separate the Hamiltonian.

\paragraph{Mean-Field Theory of the Ising Model}
To construct a mean-field theory for the Ising model, we write the Hamiltonian as
\begin{align*}
	\ham = -\sum\limits_{i}\sigma_{i}\left(J\sum\limits_{j = \text{nn}(i)}\sigma_{j} + h\right).
\end{align*}
We note that the replacement $\sigma_{j}\to m$ will make the Hamiltonian separable, hence we propose the mean-field Hamiltonian
\begin{align*}
	\ham = -\sum\limits_{i}\sigma_{i}\left(J\sum\limits_{j = \text{nn}(i)}m + h\right) = -\sum\limits_{i}\sigma_{i}\left(qJm + h\right) = -h^{\prime}\sum\limits_{i}\sigma_{i}
\end{align*}
where we have introduced the number $q$ of nearest neighbours to any one site and the effective field
\begin{align*}
	h^{\prime} = h + qJm.
\end{align*}

Using this Hamiltonian, the partition function is now given by
\begin{align*}
	&Z = \tr(e^{-\beta\ham}) = \sum\limits_{\{\sigma\}}e^{\beta h^{\prime}\sum\limits_{i}\sigma_{i}} = \left(\sum\limits_{\sigma = \pm 1}e^{\beta h^{\prime}\sigma}\right)^{N} = 2^{N}\cosh[N](\beta h^{\prime}), \\
	&m = \frac{1}{N}\expval{\sum\limits_{i}\sigma_{i}} = \frac{1}{NZ}\sum\sum\limits_{i}\sigma_{i}e^{\beta h^{\prime}\sum\limits_{j}\sigma_{j}} = \frac{1}{N}\dv{\beta h^{\prime}}\ln(2^{N}\cosh[N](\beta h^{\prime})) = \tanh(\beta h^{\prime}).
\end{align*}
Note that this implies that all spins are expected to point in the same direction. Baked into the process there is a specific idea of the structure of the solution, and it is therefore important to make a good guess about this. We proceed with the ferromagnetic case, where the guess is good, and obtain
\begin{align*}
	m = \tanh(\beta\left(h + qJm\right)).
\end{align*}
This equation can be solved graphically to yield the magnetization, but we will discuss it qualitatively here. The expression above is anti-symmetric in $m$, meaning we may consider $m \geq 0$. Depending on the parameters, the number of solutions is between one and three. In the case where $h = 0$, one solution is $m = 0$, and two other solutions may be found at $m = \pm m_{0}$. At low temperatures the right-hand side approaches $\pm 1$, yielding $m_{0} = 1$. For $h = 0$ solutions are only found where the right-hand side grows faster than the left-hand side at zero. This is satisfied if
\begin{align*}
	\beta qJ > 1.
\end{align*}
This defines the critical temperature
\begin{align*}
	T_{\text{c}} = \frac{qJ}{\kb},
\end{align*}
above which no non-zero solutions are found.

As the temperature approaches the critical temperature, above which no spontaneous magnetism is found, $m_{0}$ is small (but non-zero) and we obtain
\begin{align*}
	m_{0}                              &\approx \beta qJm_{0} - \frac{1}{3}\left(\beta qJm_{0}\right)^{3}, \\
	\left(\frac{T_{\text{c}}}{T}\right)^{3}m_{0}^{2} &= 3\left(\frac{T_{\text{c}}}{T} - 1\right), \\
	m_{0}                              &= \left(\frac{T}{T_{\text{c}}}\right)^{\frac{3}{2}}\sqrt{3\left(\frac{T_{\text{c}}}{T} - 1\right)} \\
	                                   &= \sqrt{\frac{3}{T_{\text{c}}}\left(\frac{T}{T_{\text{c}}}\right)^{2}\left(T_{\text{c}} - T\right)}.
\end{align*}
This identifies the critical exponent $\beta = 1$. One of the uses of mean-field theory is determining asymptotic behaviour close to phase transitions.

One more thing should be mentioned, namely the assertion that there actually is spontaneous magnetization. After all, if three solutions are possible, who is to say that one of the non-zero ones are found? To do this, we consider the entropy of an ideal paramagnet, which can be shown to be
\begin{align*}
	S = -N\kb\left(\frac{1 - m}{2}\ln(\frac{1 - m}{2}) + \frac{1 + m}{2}\ln(\frac{1 + m}{2})\right).
\end{align*}
We see that the non-zero solutions maximize entropy - a relief.

\paragraph{Critical Behaviour}
Using the mean-field result, we may now study other quantities close to the phase transition. The susceptibility is given by
\begin{align*}
	\chi = \fix{\pdv{m}{h}}{T}.
\end{align*}
The implicit equation for the magnetization yields
\begin{align*}
	\chi &= \frac{\beta}{\cosh[2](\beta(qJm + h))}(qJ\chi + 1), \\
	\chi &= \frac{\beta}{\cosh[2](\beta(qJm + h)) - \beta qJ} = \frac{1}{\kb \left(T\cosh[2](\beta(qJm + h)) - \frac{qJ}{\kb}\right)}.
\end{align*}
Introducing the critical temperature, we write this as
\begin{align*}
	\chi = \frac{1}{\kb \left(T\cosh[2](\beta(qJm + h)) - T_{\text{c}}\right)}.
\end{align*}
In particular, for $h = 0$ and temperatures above $T_{\text{c}}$, where there is no magnetization, we obtain
\begin{align*}
	\chi = \frac{1}{\kb\left(T - T_{\text{c}}\right)}.
\end{align*}
When approaching the phase transition from below for $h = 0$, we use the asymptotic expression for the magnetization to obtain
\begin{align*}
	\chi &= \frac{1}{\kb \left(T\cosh[2](\beta qJ\sqrt{\frac{3}{T_{\text{c}}}\left(\frac{T}{T_{\text{c}}}\right)^{2}\left(T_{\text{c}} - T\right)}) - T_{\text{c}}\right)} \\
	     &= \frac{1}{\kb \left(T\cosh[2](\sqrt{3\left(1 - \frac{T}{T_{\text{c}}}\right)}) - T_{\text{c}}\right)} \\
	     &= \frac{1}{\kb T_{\text{c}}\left(\frac{T}{T_{\text{c}}}\left(1 + 3\left(1 - \frac{T}{T_{\text{c}}}\right)\right) - 1\right)} \\
	     &=	\frac{1}{\kb T_{\text{c}}\left(4\frac{T}{T_{\text{c}}} - 3\left(\frac{T}{T_{\text{c}}}\right)^{2} - 1\right)} \\
	     &= \frac{1}{\kb T_{\text{c}}\left(3\frac{T}{T_{\text{c}}}\left(1 - \frac{T}{T_{\text{c}}}\right) + \frac{T}{T_{\text{c}}} - 1\right)} \\
	     &\approx \frac{1}{\kb T_{\text{c}}\left(3\left(1 - \frac{T}{T_{\text{c}}}\right) + \frac{T}{T_{\text{c}}} - 1\right)} \\
	     &= \frac{1}{2\kb\left(T_{\text{c}} - T\right)}.
\end{align*}

\paragraph{The Bragg-Williams Approximation}
The Bragg-Williams approximation to mean-field theory starts with constructing the availability in terms of the order parameter. In the case of the Ising model, we introduce the numbers $N_{\pm}$ of spins with values $\pm 1$. Furthermore, we introduce the numbers $N_{\pm\pm}$ of spin pairs of any kind. The Hamiltonian is thus
\begin{align*}
	\ham = -J(N_{++} + N_{--} - N_{+-}) - h(N_{+} - N_{-}).
\end{align*}
Treating the spins as independent allows us to write
\begin{align*}
	S = -\kb\left(N_{+}\ln(N_{+}) + N_{-}\ln(N_{-})\right).
\end{align*}
%TODO: Show
The number of pairs is given by
\begin{align*}
	N_{\pm\pm} = \frac{qN_{\pm}^{2}}{2N},\ N_{+-} = \frac{qN_{+}N_{-}}{N}.
\end{align*}
To proceed, we re-express the spin numbers in terms of the order parameter by using $N = N_{+} + N_{-}$ and $\sigma = N_{+} - N_{-}$ to obtain
\begin{align*}
	N_{+} = \frac{1}{2}N(1 + m),\ N_{-} = \frac{1}{2}N(1 - m).
\end{align*}
The Hamiltonian is now given by
\begin{align*}
	\ham &= -\frac{qJ}{2N}(N_{+}^{2} + N_{-}^{2} - 2N_{+}N_{-}) - Nhm \\
	     &= -\frac{qJN}{8}((1 + m)^{2} + (1 - m)^{2} - 2(1 + m)(1 - m)) - Nhm \\
	     &= -\frac{qJN}{2}m^{2} - Nhm,
\end{align*}
and the free energy is somehow
\begin{align*}
	G(h, T) &= \ham - TS \\
	        &= -\frac{qJN}{2}m^{2} - Nhm + \frac{1}{2}N\kb T\left((1 + m)\ln(\frac{1}{2}N(1 + m)) + (1 - m)\ln(\frac{1}{2}N(1 - m))\right).
\end{align*}
Minimizing it with respect to the order parameter yields
\begin{align*}
	&-qJNm - Nh + \frac{1}{2}N\kb T\left(\ln(\frac{1}{2}N(1 + m)) + 1 - \ln(\frac{1}{2}N(1 - m)) - 1\right) = 0, \\
	&-qJm - h + \frac{1}{2}\kb T\ln(\frac{1 + m}{1 - m}) = 0.
\end{align*}
Its solution is
\begin{align*}
	\frac{1 + m}{1 - m} = e^{2\beta(qJm + h)}, \\
	1 + m = (1 - m)e^{2\beta(qJm + h)}, \\
	m(1 + e^{2\beta(qJm + h)}) = e^{2\beta(qJm + h)} - 1, \\
	m = \tanh(\beta(qJm + h)),
\end{align*}
as expected.

\paragraph{Inaccuracies of Mean-Field Theories}
The mean-field arguments predict the existence of a phase transition, but this cannot be the case in one dimension. To see this, consider a chain in its ground state and the set of excitations that flips all spins to the right of some spin $k$. The change in energy is $2J$, and the number of possible states corresponding to this energy is $N - 1$, hence the free energy changes by $2J - \kb\ln(N - 1)$. For large $N$ such states are thus always preferable. Their removal of translation invariance implies that there is no magnetization, in contradiction of the mean-field results.

A slightly better result is obtained for a $N\times N$ lattice in two dimensions. The set of excitations now consists of excitations that split the system in two distinct magnetic domains. Any particular excitation is described by a chain running through the bonds. Each segment crosses one bond, and the energy change due to the excitation is $2LJ$, where $L$ is the number of segments. The typical length is $2N$. The next segment may always be placed in at least two sites, neglecting the boundaries. Including the $N$ possible starting points, the multiplicity of the chain is $N2^{L}$, and the free energy change of the excitation is
\begin{align*}
	\Delta G = 4NJ - \kb T\ln(2^{2N}N).
\end{align*}
The phase transition occurs when this energy change is negative, i.e. when
\begin{align*}
	\kb T\ln(2^{2N}N) > 4NJ,\ \kb T(2N\ln(2) + \ln(N)) > 4NJ,\ T < T_{\text{c}} \approx \frac{2J}{\kb\ln(2)},
\end{align*}
which is decently close to the analytically obtained results.

\paragraph{Antiferromagnetism}
The case of $J < 0$ is another interesting case, and gives rise to some interesting phases. One phenomenon which may occur is frustration, where the lattice structure is such that for any group of spins, the state of lowest energy is degenerate and not such that all interactions are beneficial. Assuming this not to be the case, we find that the ground state corresponds to an ordered phase such that two lattice translations combined leave the system invariant. This symmetry is broken by the phase transition, hence we introduce the order parameter
\begin{align*}
	m = \frac{1}{N}\sum(-1)^{j}\sigma_{j},
\end{align*}
which may be written as
\begin{align*}
	m = \frac{1}{2}(m_{A} - m_{B}),
\end{align*}
where the lattice has been divided into two sublattices.

The mean-field Hamiltonian is
\begin{align*}
	\ham = \ham_{A} + \ham_{B} = -Jq\left(m_{B}\sum\limits_{A}\sigma_{i} + m_{A}\sum\limits_{B}\sigma_{i}\right),
\end{align*}
and by the same methods as previously, the mean-field partition function is
\begin{align*}
	Z = 2^{N}\cosh[\frac{1}{2}N](\beta Jqm_{B})\cosh[\frac{1}{2}N](\beta Jqm_{A}).
\end{align*}
From this we obtain
\begin{align*}
	m_{A} = \tanh(\beta Jqm_{B}).
\end{align*}
By construction we have $m_{A} = -m_{B}$, which yields
\begin{align*}
	m_{A} = -\tanh(\beta Jqm_{A}).
\end{align*}
This is the same as we obtained for the ferromagnetic case, and we may immediately identify the critical temperature
\begin{align*}
	T_{\text{c}} = -\frac{qJ}{\kb}.
\end{align*}