\section{Basic Concepts}

\paragraph{What is a Complex System?}
A complex system is a dynamical system characterized by at least one of the following:
\begin{itemize}
	\item Nonlinearity.
	\item High sensitivity to initial conditions - the butterfly effect.
	\item The existence of bifurcations.
	\item Emergent phenomena - the formation of patterns in the solution.
	\item Feedback.
	\item Dissipation.
\end{itemize}

\paragraph{The Interesting Aspects of Complex Systems}
The interesting aspects of complex systems are
\begin{itemize}
	\item long-term behaviour.
	\item dependence on initial conditions.
	\item parameter dependence.
\end{itemize}

\paragraph{Autonomous Systems}
An autonomous system is described by
\begin{align*}
	\dv{\vb{x}}{t} = \vb{f}(\vb{x})
\end{align*}
where $\vb{x}$ is the state vector of the system.

This is of course not a restriction to first-order systems, as all systems may be written in this form. It is neither a restriction to $f$ being independent of $t$, as one in this case can simply extend $\vb{x}$ to contain $t$.

\paragraph{Deterministic Systems}
A deterministic system is a system without random noise. Such systems are entirely specified by $\vb{f}$ and an initial condition.

\paragraph{Conservative and Dissipative Systems}
Conservative systems satisfy $\div{\vb{f}} = 0$. Dissipative systems satisfy $\div{\vb{f}} < 0$.

\paragraph{Orbits}
An orbit is a solution to an autonomous system corresponding to some particular initial value. The set of all orbits is the set of flow lines of $\vb{f}$. Because the position in phase space fully determines the future solution, flow lines never cross.

\paragraph{Fixed Points}
A fixed point is a point that satisfies $\vb{f}(\vb{x}) = \vb{0}$. Close to such points, non-zero first derivatives produce exponential behaviour locally and first derivatives equal to zero produce evolution slower than exponential.

\paragraph{Bifurcations}
A bifurcation is a qualitative in the structure of $\vb{f}$ as some parameter is varied.

\paragraph{Uniqueness of Solutions}
The weakest condition for the existence and uniqueness of a solution to
\begin{align*}
	\dv{\vb{x}}{t} = \vb{f}(\vb{x}),\ \vb{x}(t_{0}) = \vb{x}_{0}
\end{align*}
in a finite time interval around $t_{0}$, which we will assume to hold, is the Lipschitz condition
\begin{align*}
	\abs{\vb{f}(\vb{x}) - \vb{f}(\vb{y})} \leq \kappa\abs{\vb{x} - \vb{y}}
\end{align*}
for some finite $\kappa$. This entails that $\vb{f}$ should be continuous and have piecewise continuous derivatives. If this condition holds, the solution is continuous in $\vb{x}_{0}$.