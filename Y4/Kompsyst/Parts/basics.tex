\section{Basic Concepts}

\paragraph{What is a Complex System?}
A complex system is a dynamical system characterized by at least one of the following:
\begin{itemize}
	\item Nonlinearity.
	\item High sensitivity to initial conditions - the butterfly effect.
	\item The existence of bifurcations.
	\item Emergent phenomena - the formation of patterns in the solution.
	\item Feedback.
	\item Dissipation.
\end{itemize}

\paragraph{The Interesting Aspects of Complex Systems}
The interesting aspects of complex systems are
\begin{itemize}
	\item long-term behaviour.
	\item dependence on initial conditions.
	\item parameter dependence.
\end{itemize}

\paragraph{Autonomous Systems}
An autonomous system is described by
\begin{align*}
	\dv{\vb{x}}{t} = \vb{f}(\vb{x})
\end{align*}
where $\vb{x}$ is the state vector of the system.

This is of course not a restriction to first-order systems, as all systems may be written in this form. It is neither a restriction to $f$ being independent of $t$, as one in this case can simply extend $\vb{x}$ to contain $t$.

\paragraph{Deterministic Systems}
A deterministic system is a system without random noise. Such systems are entirely specified by $\vb{f}$ and an initial condition.

\paragraph{Conservative and Dissipative Systems}
Conservative systems satisfy $\div{\vb{f}} = 0$. Dissipative systems satisfy $\div{\vb{f}} < 0$.

\paragraph{Orbits}
An orbit is a solution to an autonomous system corresponding to some particular initial value. The set of all orbits is the set of flow lines of $\vb{f}$. Because the position in phase space fully determines the future solution, flow lines never cross.

\paragraph{Fixed Points}
A fixed point is a point that satisfies $\vb{f}(\vb{x}) = \vb{0}$. Close to such points, non-zero first derivatives produce exponential behaviour locally and first derivatives equal to zero produce evolution slower than exponential.

\paragraph{Bifurcations}
A bifurcation is a qualitative in the structure of $\vb{f}$ as some parameter is varied.

\paragraph{Uniqueness of Solutions}
The weakest condition for the existence and uniqueness of a solution to
\begin{align*}
	\dv{\vb{x}}{t} = \vb{f}(\vb{x}),\ \vb{x}(t_{0}) = \vb{x}_{0}
\end{align*}
in a finite time interval around $t_{0}$, which we will assume to hold, is the Lipschitz condition
\begin{align*}
	\abs{\vb{f}(\vb{x}) - \vb{f}(\vb{y})} \leq \kappa\abs{\vb{x} - \vb{y}}
\end{align*}
for some finite $\kappa$. This entails that $\vb{f}$ should be continuous and have piecewise continuous derivatives. If this condition holds, the solution is continuous in $\vb{x}_{0}$.

\paragraph{Periodic Motion}
For a system with a one-dimensional phase space, periodic motion is impossible on the real line or a subset of it. It is possible, however, on spaces with different topologies, such as a circle.

\paragraph{Numerical Integration}
Numerical integration methods are based around the Taylor expansion
\begin{align*}
	\vb{x}(t + \Delta t) = \vb{x}(t) + \eval{\dv{\vb{x}}{t}}_{t}\Delta t + \frac{1}{2!}\eval{\dv[2]{\vb{x}}{t}}_{t}(\Delta t)^{2} + \dots
\end{align*}
and specific schemes are usually obtained by truncating this expansion.

\paragraph{The Forward Euler Method}
The forward Euler method is obtained by truncating at the second step. For an autonomous system we have
\begin{align*}
	\vb{x}(t + \Delta t) = \vb{x}(t) + \vb{f}(t)\Delta t.
\end{align*}
The error in each step is of ordet $(\Delta t)^{2}$, and the total error after $N$ steps, which integrate a time $\tau$ forward, is of order $N(\Delta t)^{2} = \tau\Delta t$. Note that this is equivalent to the approximation
\begin{align*}
	\eval{\dv{\vb{x}}{t}}_{t} = \frac{1}{\Delta t}(\vb{x}(t + \Delta t) - \vb{x}(t)).
\end{align*}

\paragraph{Runge-Kutta Schemes}
An improved scheme starts with
\begin{align*}
	\frac{1}{\Delta t}(\vb{x}(t + \Delta t) - \vb{x}(t)) = \eval{\dv{\vb{x}}{t}}_{t + \frac{1}{2}\Delta t} = \vb{f}\left(t + \frac{1}{2}\Delta t\right) = \vb{f}\left(\vb{x}(t) + \frac{1}{2}\Delta t\vb{f}\left(\vb{x}(t)\right)\right) = \vb{f}\left(\vb{x}(t)\right) + \frac{1}{2}\Delta t\vb{f}\p(\vb{x}(t))\vb{f}\left(\vb{x}(t)\right) + \dots
\end{align*}
From this we devise the second-order Runge-Kutta method
\begin{align*}
	\vb{k}_{1} = \Delta t\vb{f}\left(\vb{x}(t)\right),\ \vb{k}_{2} = \Delta t\vb{f}\left(\vb{x}(t) + \frac{1}{2}\vb{k}_{1}\right),\ \vb{x}(t + \Delta t) = \vb{x}(t) + \vb{k}_{2}.
\end{align*}
Similarly there is a fourth-order scheme
\begin{align*}
	\vb{k}_{1} = \Delta t\vb{f}\left(\vb{x}(t)\right),\ \vb{k}_{2} = \Delta t\vb{f}\left(\vb{x}(t) + \frac{1}{2}\vb{k}_{1}\right),\ \vb{k}_{3} = \Delta t\vb{f}\left(\vb{x}(t) + \frac{1}{2}\vb{k}_{2}\right),\ \vb{k}_{4} = \Delta t\vb{f}\left(\vb{x}(t) + \vb{k}_{3}\right), \\
	\vb{x}(t + \Delta t) = \vb{x}(t) + \frac{1}{6}\left(\vb{k}_{1} + 2\vb{k}_{2} + 2\vb{k}_{3} + \vb{k}_{4}\right),
\end{align*}
with an accumulated error of order $(\Delta t)^{4}$.

\paragraph{Symplectic Methods}
Symplectic methods are numerical integration schemes that respect conservation laws.