\section{Quantum Field Theory}

\paragraph{A Comment on Normalization}
We will be using the convention where the integral containing the fields has a division by a factor $\sqrt{2\omega_{\vb{k}}}$.

\paragraph{Lorentz Transformations and Dirac Fields}
We have already established that Dirac fields transform as
\begin{align*}
	\psi(x) \to e^{-\frac{i}{4}\omega^{\mu\nu}\sigma_{\mu\nu}}\psi(x),
\end{align*}
where $\sigma_{\mu\nu} = \frac{i}{2}\comm{\gamma_{\mu}}{\gamma_{\nu}}$. This means that the adjoint transforms according to
\begin{align*}
	\psi\adj(x) \to \psi\adj(x)e^{\frac{i}{4}\omega^{\mu\nu}\sigma_{\mu\nu}\adj}.
\end{align*}
As $\sigma_{\mu\nu}$ are not self-adjoint, $\psi\adj\psi$ is not a Lorentz invariant and cannot be included in any field theory. The hope is, however, that there exists some matrix $M$ which we can put in the middle to solve that issue. For this to work we would need
\begin{align*}
	e^{\frac{i}{4}\omega^{\mu\nu}\sigma_{\mu\nu}\adj}M = Me^{\frac{i}{4}\omega^{\mu\nu}\sigma_{\mu\nu}}.
\end{align*}
Expanding around the identity we find
\begin{align*}
	\sigma_{\mu\nu}\adj M = M\sigma_{\mu\nu}.
\end{align*}

To proceed we need to know a little about the gamma matrices. It can be shown in the Dirac basis that $(\gamma^{\mu})\adj = \gamma^{0}\gamma^{\mu}\gamma^{0}$. Because different bases are related by a unitary similarity transform, this holds in all bases. This means that
\begin{align*}
	(\sigma^{\mu\nu})\adj = -\frac{i}{2}\gamma^{0}\comm{\gamma^{\nu}}{\gamma^{\mu}}\gamma^{0} = \gamma^{0}\sigma^{\mu\nu}\gamma^{0},
\end{align*}
and $\gamma^{0}M$ must commute with $\sigma_{\mu\nu}$. Two choices for $M$ are thus $\gamma^{5}$ and $\gamma^{0}$, the latter of which we will use to define the Dirac adjoint $\overline{\psi} = \psi\adj\gamma^{0}$, which transforms the right way under Lorentz transformations.

What about forms of the type $\overline{\psi}F\psi$? We will need a basis for the space of matrices, and can choose the identity, the gamma matrices, their commutators, $\gamma^{5}$ and $\gamma_{\mu}\gamma^{5}$. Of these, only the index-free versions will work. To see this, consider
\begin{align*}
	\overline{\psi}\gamma^{\rho}\psi &\to \overline{\psi}e^{\frac{i}{4}\omega_{\mu\nu}\sigma^{\mu\nu}}\gamma^{\rho}e^{-\frac{i}{4}\omega_{\mu\nu}\sigma^{\mu\nu}}\psi \\
	                                &\approx \overline{\psi}\left(1 + \frac{i}{4}\omega_{\mu\nu}\sigma^{\mu\nu}\right)\gamma^{\rho}\left(1 - \frac{i}{4}\omega_{\mu\nu}\sigma^{\mu\nu}\right)\psi \\
	                                &\approx \overline{\psi}\left(\gamma^{\rho} + \frac{i}{4}\omega_{\mu\nu}\comm{\sigma^{\mu\nu}}{\gamma^{\rho}}\right)\psi.
\end{align*}
The commutator can be simplified as
\begin{align*}
	\comm{\sigma^{\mu\nu}}{\gamma^{\rho}} &= \frac{i}{2}\left(\gamma^{\mu}\comm{\gamma^{\nu}}{\gamma^{\rho}} + \comm{\gamma^{\mu}}{\gamma^{\rho}}\gamma^{\nu} - \gamma^{\nu}\comm{\gamma^{\mu}}{\gamma^{\rho}} - \comm{\gamma^{\nu}}{\gamma^{\rho}}\gamma^{\mu}\right) \\
	                                         &= \frac{i}{2}\left(\gamma^{\mu}(\gamma^{\nu}\gamma^{\rho} - \gamma^{\rho}\gamma^{\nu}) + (\gamma^{\mu}\gamma^{\rho} - \gamma^{\rho}\gamma^{\mu})\gamma^{\nu} - \gamma^{\nu}(\gamma^{\mu}\gamma^{\rho} - \gamma^{\rho}\gamma^{\mu}) - (\gamma^{\nu}\gamma^{\rho} - \gamma^{\rho}\gamma^{\nu})\gamma^{\mu}\right) \\
	                                         &= \frac{i}{2}\left(2g^{\nu\rho}\gamma^{\mu} - 2g^{\rho\mu}\gamma^{\nu} + 2g^{\nu\rho}\gamma^{\mu} - 2g^{\rho\mu}\gamma^{\nu}\right) \\
	                                         &= 2i(g^{\nu\rho}\gamma^{\mu} - g^{\rho\mu}\gamma^{\nu}).
\end{align*}
Thus we have
\begin{align*}
	\omega_{\mu\nu}\comm{\sigma^{\mu\nu}}{\gamma^{\rho}} &= 2i\omega_{\mu\nu}(g^{\nu\rho}\gamma^{\mu} - g^{\rho\mu}\gamma^{\nu}) = -4i\tensor{\omega}{^{\rho}_{\mu}}\gamma^{\mu}
\end{align*}
and
\begin{align*}
	\overline{\psi}\gamma^{\rho}\psi &\approx \overline{\psi}\left(\kdelta{\rho}{\mu} + \tensor{\omega}{^{\rho}_{\mu}}\right)\gamma^{\mu}\psi,
\end{align*}
which is the same transformation rule as for a vector. Similarly objects with more indices transform like tensors, justifying the use of Lorentz indices despite the gamma matrices being the same in all frames.

\paragraph{Energy Change for Klein-Gordon Fields}
For a Klein-Gordon field the process with the non-relativistic string can be repeated to find
\begin{align*}
	\ham = \frac{1}{2}\integ[3]{}{}{\vb{p}}{\omega_{\vb{p}}(a_{\vb{p}}\adj a_{\vb{p}} + a_{\vb{p}}a_{\vb{p}}\adj)}.
\end{align*}
Using the commutation relations we find
\begin{align*}
	\comm{\ham}{a_{\vb{k}}\adj} =& \frac{1}{2}\integ[3]{}{}{\vb{p}}{\omega_{\vb{p}}(a_{\vb{p}}\adj\comm{a_{\vb{p}}}{a_{\vb{k}}\adj} + \comm{a_{\vb{p}}\adj}{a_{\vb{k}}\adj}a_{\vb{p}} + a_{\vb{p}}\comm{a_{\vb{p}}\adj}{a_{\vb{k}}\adj} + \comm{a_{\vb{p}}}{a_{\vb{k}}\adj}a_{\vb{p}}\adj)} \\
	                            =& \integ[3]{}{}{\vb{p}}{\delta^{3}(\vb{p} - \vb{k})\omega_{\vb{p}}a_{\vb{p}}\adj} \\
	                            =& \omega_{\vb{k}}a_{\vb{k}}\adj,
\end{align*}
and similarly $\comm{\ham}{a_{\vb{k}}\adj} = -\omega_{\vb{k}}a_{\vb{k}}\adj$. Thus these operators raise and lower the energies of eigenstates of the Hamiltonian, further justifying our interpretations.

\paragraph{The Charge of Charged Klein-Gordon Fields}
The Lagrangian for the charged Klein-Gordon field is
\begin{align*}
	\lag = \del{}{\mu}\phi\cc\del{\mu}{}\phi - m^{2}\phi\cc\phi.
\end{align*}
Corresponding to this Lagrangian is a symmetry $\var{\phi} = i\alpha\phi$ which leaves the action unchanged. The corresponding Noether current is
\begin{align*}
	j^{\mu} = i(\phi\cc\del{\mu}{}\phi - \phi\del{\mu}{}\phi\cc).
\end{align*}
Using the field expansion
\begin{align*}
	\phi = \frac{1}{\sqrt{(2\pi)^{3}}}\integ[3]{}{}{\vb{p}}{\frac{1}{\sqrt{2\omega_{\vb{p}}}}\left(a_{\vb{p}}e^{-ipx} + b_{\vb{p}}\cc e^{ipx}\right)}
\end{align*}
with distinct amplitudes $a$ and $b$ we find
\begin{align*}
	\rho &= -\frac{1}{2(2\pi)^{3}}\integ[3]{}{}{\vb{p}}{\integ[3]{}{}{\vb{k}}{\sqrt{\frac{\omega_{\vb{k}}}{\omega_{\vb{p}}}}\left(\left(a_{\vb{p}}\cc e^{ipx} + b_{\vb{p}} e^{-ipx}\right)\left(-a_{\vb{k}}e^{-ikx} + b_{\vb{k}}\cc e^{ikx}\right) - \left(a_{\vb{p}}e^{-ipx} + b_{\vb{p}}\cc e^{ipx}\right)\left(a_{\vb{k}}\cc e^{ikx} - b_{\vb{k}}e^{-ikx}\right)\right)}} \\
	     &= -\frac{1}{2(2\pi)^{3}}\integ[3]{}{}{\vb{p}}{\integ[3]{}{}{\vb{k}}{\sqrt{\frac{\omega_{\vb{k}}}{\omega_{\vb{p}}}}\left(-a_{\vb{p}}\cc a_{\vb{k}}e^{i(p - k)x} + b_{\vb{p}}b_{\vb{k}}\cc e^{i(k - p)x} - a_{\vb{p}}a_{\vb{k}}\cc e^{i(k - p)x} + b_{\vb{p}}\cc b_{\vb{k}}e^{i(p - k)x}\right)}}. 
\end{align*}
The quantized version is
\begin{align*}
	\rho &= -\frac{1}{2(2\pi)^{3}}\integ[3]{}{}{\vb{p}}{\integ[3]{}{}{\vb{k}}{\sqrt{\frac{\omega_{\vb{k}}}{\omega_{\vb{p}}}}\left(-a_{\vb{p}}\adj a_{\vb{k}}e^{i(p - k)x} + b_{\vb{p}}b_{\vb{k}}\adj e^{i(k - p)x} - a_{\vb{p}}a_{\vb{k}}\adj e^{i(k - p)x} + b_{\vb{p}}\adj b_{\vb{k}}e^{i(p - k)x}\right)}}.
\end{align*}
Using the commutation relations we find
\begin{align*}
	\rho &= -\frac{1}{(2\pi)^{3}}\integ[3]{}{}{\vb{p}}{\integ[3]{}{}{\vb{k}}{\sqrt{\frac{\omega_{\vb{k}}}{\omega_{\vb{p}}}}\left(b_{\vb{p}}\adj b_{\vb{k}}e^{i(p - k)x} - a_{\vb{p}}\adj a_{\vb{k}}e^{i(p - k)x}\right)}}.
\end{align*}
Computing the total charge amounts to integrating over space, adding a Dirac delta in momentum space and finally yielding
\begin{align*}
	Q &= \integ[3]{}{}{\vb{p}}{a_{\vb{p}}\adj a_{\vb{k}} - b_{\vb{p}}\adj b_{\vb{k}}}.
\end{align*}

\paragraph{A Trick for Computing Propagators}
A neat trick to find propagators, at least for the standard theories we will consider, is to write the Lagrangian as
\begin{align*}
	\lag = \overline{\phi}A\phi + \dots
\end{align*}
where $A$ is some operator and the dots are total derivatives. The bar signifies some relevant conjugate field. For real fields you want a prefactor of $\frac{1}{2}$ as well. The propagator is the inverse of $A$.

Let us show two examples. First, for a charged scalar field we have
\begin{align*}
	\lag = \del{}{\mu}\phi\adj\del{\mu}{}\phi - m^{2}\phi\adj\phi = \del{}{\mu}\left(\phi\adj\del{\mu}{}\phi\right) - \phi\adj\left(\dalem + m^{2}\right)\phi,
\end{align*}
hence $A = -\dalem - m^{2}$. In momentum space we thus find the familiar formula
\begin{align*}
	i\kgprop{\text{F}}{p} = \frac{i}{p^{2} - m^{2}}.
\end{align*}
Next, for a Dirac field we have
\begin{align*}
	\lag = i\overline{\psi}\fsl{\del{}{}}\psi - m\overline{\psi}\psi = \overline{\psi}\left(i\fsl{\del{}{}} - m\right).
\end{align*}
Again, in momentum space we have
\begin{align*}
	(\fsl{p} - m)\dirprop{}{}{p} = 1.
\end{align*}
Using gamma gymnastics we know that $(\fsl{p} - m)(\fsl{p} + m) = p^{2} - m^{2}$, hence we find
\begin{align*}
	i\dirprop{}{}{p} = \frac{i(\fsl{p} + m)}{p^{2} - m^{2}}.
\end{align*}

\paragraph{The Electromagnetic Field and its Complications}
For a general gauge the Lagrangian for the electromagnetic field is
\begin{align*}
	\lag = -\frac{1}{4}\tensor{F}{_{\mu\nu}}\tensor{F}{^{\mu\nu}} - \frac{1}{2\xi}(\del{}{\mu}A^{\mu})^{2}
\end{align*}
for some control parameter $\xi$. The second term, called the gauge fixing term, turns out to be really useful for doing calculations. To show that we expand the Lagrangian as
\begin{align*}
	\lag &= -\frac{1}{4}(\del{}{\mu}A_{\nu} - \del{}{\nu}A_{\mu})(\del{\mu}{}A^{\nu} - \del{\nu}{}A^{\mu}) - \frac{1}{2\xi}(\del{}{\mu}A^{\mu})(\del{}{\nu}A^{\nu}) \\
	     &= -\frac{1}{2}\left(\del{}{\mu}A_{\nu}(\del{\mu}{}A^{\nu} - \del{\nu}{}A^{\mu}) - \frac{1}{\xi}(\del{}{\mu}A^{\mu})(\del{}{\nu}A^{\nu})\right) \\
	     &= -\frac{1}{2}\left(\del{}{\mu}A_{\nu}(g^{\nu\rho}\del{\mu}{}A_{\rho} - g^{\mu\rho}\del{\nu}{}A_{\rho}) - \frac{1}{\xi}(\del{\mu}{}A_{\mu})(\del{\nu}{}A_{\nu})\right) \\
	     &= \frac{1}{2}\left(A_{\nu}\del{}{\mu}(g^{\nu\rho}\del{\mu}{}A_{\rho} - g^{\mu\rho}\del{\nu}{}A_{\rho}) + \frac{1}{\xi}A_{\mu}\del{\mu}{}\del{\nu}{}A_{\nu}\right) - \frac{1}{2}\del{}{\mu}\left(A_{\nu}(g^{\nu\rho}\del{\mu}{}A_{\rho} - g^{\mu\rho}\del{\nu}{}A_{\rho})\right) - \frac{1}{2\xi}\del{\mu}{}(A_{\mu}\del{\nu}{}A_{\nu}).
\end{align*}
Ignoring the total derivatives we are left with
\begin{align*}
	\lag &= \frac{1}{2}A_{\nu}\left(g^{\nu\rho}\del{}{\mu}\del{\mu}{}A_{\rho} - g^{\mu\rho}\del{}{\mu}\del{\nu}{}A_{\rho} + \frac{1}{\xi}\del{\mu}{}\del{\nu}{}A_{\mu}\right) \\
	     &= \frac{1}{2}A_{\nu}\left(g^{\nu\mu}\dalem A_{\mu} - \del{\mu}{}\del{\nu}{}A_{\mu} + \frac{1}{\xi}\del{\mu}{}\del{\nu}{}A_{\mu}\right) \\
	     &= \frac{1}{2}A_{\nu}\left(g^{\mu\nu}\dalem - \left(1 - \frac{1}{\xi}\right)\del{\mu}{}\del{\nu}{}\right)A_{\mu},
\end{align*}
hence we have found a matrix operator $A^{\mu\nu}$. Its inverse must be the corresponding inverse matrix. Calling the inverse $D_{\nu\rho}$ we require in Fourier space
\begin{align*}
	\left(-g^{\mu\nu}p^{2} + \left(1 - \frac{1}{\xi}\right)p^{\mu}p^{\nu}\right)D_{\nu\rho} = \kdelta{\mu}{\rho}.
\end{align*}
We can expand $D$ in terms of the metric and the momentum to find
\begin{align*}
	\left(-g^{\mu\nu}p^{2} + \left(1 - \frac{1}{\xi}\right)p^{\mu}p^{\nu}\right)(ag_{\nu\rho} + bp_{\nu}p_{\rho}) = -ap^{2}\kdelta{\mu}{\rho} - bp^{2}p^{\mu}p_{\rho} + \left(1 - \frac{1}{\xi}\right)ap^{\mu}p_{\rho} + b\left(1 - \frac{1}{\xi}\right)p^{2}p^{\mu}p_{\rho}.
\end{align*}
A first choice is $a = -\frac{1}{p^{2}}$. Given that, we would be left with
\begin{align*}
	-bp^{2} + \left(1 - \frac{1}{\xi}\right)\left(-\frac{1}{p^{2}} + bp^{2}\right) = -\frac{1}{\xi}bp^{2} - \frac{1}{p^{2}}\left(1 - \frac{1}{\xi}\right) = 0,
\end{align*}
with solution
\begin{align*}
	b = \frac{1}{(p^{2})^{2}}(1 - \xi).
\end{align*}
Thus we have
\begin{align*}
	iD_{\mu\nu} = -\frac{i}{p^{2}}\left(g_{\mu\nu} - (1 - \xi)\frac{p_{\mu}p_{\nu}}{p^{2}}\right).
\end{align*}
At this point we may simplify the propagator by simply choosing $\xi = 1$, netting us
\begin{align*}
	iD_{\mu\nu} = -\frac{ig_{\mu\nu}}{p^{2}}.
\end{align*}

\paragraph{Dirac Fields}
For Dirac fields we will use the normalization
\begin{align*}
	\overline{u}u = 2mc,\ \overline{v}v = -2mc
\end{align*}
and the completeness relation
\begin{align*}
	\sum\limits_{s}u_{s}\overline{u}_{s} = \fsl{p} + m,\ \sum\limits_{s}v_{s}\overline{v}_{s} = \fsl{p} - m.
\end{align*}

\paragraph{Quantum Electrodynamics}
The Dirac field is invariant under a $\text{U}(1)$ symmetry $\psi\to e^{-ieQ\theta}\psi$. The normalization in the exponent contains a parameter $e$ to be discussed, a parameter $Q$ specific to each particle and a parameter $\theta$ accounting for the rest. By convention we take $Q = -1$ for the electron. The corresponding Noether current is
\begin{align*}
	j^{\mu} = eQ\overline{\psi}\gamma^{\mu}\psi.
\end{align*}

Extending $\theta$ to be a function on spacetime breaks this symmetry for the standard Dirac Lagrangian, however. By extending the derivative operator to
\begin{align*}
	D_{\mu} = \del{}{\mu} + ieA_{\mu}
\end{align*}
and requiring $A_{\mu}\to A_{\mu} + \del{}{\mu}\theta$, however, we have a Lagrangian that respects this symmetry. This is called a gauge symmetry, and corresponds to a gauge theory realized by gauge bosons. We identify this field with the 4-potential and the corresponding transformation as a gauge transformation, adding its contribution to the total Lagrangian. To include the gauge field in the Lagrangian, we will need the field strength tensor. We already know how to get it, but I note the result
\begin{align*}
	\comm{D_{\mu}}{D_{\nu}} &= (\del{}{\mu} + ieA_{\mu})(\del{}{\nu} + ieA_{\nu}) - (\del{}{\nu} + ieA_{\nu})(\del{}{\mu} + ieA_{\mu}) \\
	                        &= ie(\del{}{\mu}A_{\nu} + A_{\mu}\del{}{\nu} - \del{}{\nu}A_{\mu} - A_{\nu}\del{}{\mu})  \\
	                        &= ie(\del{}{\mu}(A_{\nu}) + A_{\nu}\del{}{\mu} + A_{\mu}\del{}{\nu} - \del{}{\nu}(A_{\mu}) - A_{\mu}\del{}{\nu} - A_{\nu}\del{}{\mu}) \\
	                        &= ieF_{\mu\nu},
\end{align*}
which will be useful later.

When comparing to the non-interacting case, we find an interaction term
\begin{align*}
	\lag_{\text{int}} = -eQ\overline{\psi}\gamma^{\mu}\psi A_{\mu}.
\end{align*}
This is the interaction term defining quantum electrodynamics. It is a so-called gauge theory based on the $\text{U}(1)$ gauge group.

Let us now consider the corresponding interaction vertex. Note that a single of these does not represent a physical process. To see this, consider the spontaneous emission of a photon from a particle. Denoting the momenta of the particle before and after the emission as $p$ and $k$, and the photon momentum as $q$, we have
\begin{align*}
	q^{2} = 2p\cdot q.
\end{align*}
Letting the photon have frequency $\omega$ and the particle have energy $E$ we find
\begin{align*}
	\omega^{2} - \vb{q}^{2} = 2(E\omega - \vb{p}\cdot\vb{q}).
\end{align*}
By the properties of the scalar product we thus find
\begin{align*}
	E\omega - \abs{\vb{p}}\abs{\vb{q}} \leq \frac{1}{2}(\omega^{2} - \vb{q}^{2}) \leq E\omega + \abs{\vb{p}}\abs{\vb{q}}.
\end{align*}
Assuming the photon to be physical, the center part must be zero. But if the particle is massive, the left-hand side is positive, creating a contradiction. The only way this can be resolved is for the created photon to not be physical, and merely mediate between two vertices.

In terms of spins, one sums over incoming spins and averages over outgoing spins.

\paragraph{Casimir's Trick}
In QED, at least, we will need to compute expressions of the form
\begin{align*}
	\sum\limits_{s}(\overline{u}_{1}Fu_{2})\cc \overline{u}_{1}Gu_{2}
\end{align*}
for some spinor $u$ and some matrices $F$ and $G$. To calculate this we use the fact that
\begin{align*}
	(\overline{u}_{1}Fu_{2})\cc = u_{2}\adj F\adj\gamma_{0}u_{1} = \overline{u}_{2}\gamma_{0}F\adj\gamma_{0}u_{1}.
\end{align*}
Introducing $F^{\ddagger} = \gamma_{0}F\adj\gamma_{0}$ we have $(\overline{u}_{1}Fu_{2})\cc = \overline{u}_{2}F^{\ddagger}u_{1}$. We can now simplify the above as
\begin{align*}
	\sum\limits_{s}(\overline{u}_{1}Fu_{2})\cc \overline{u}_{1}Gu_{2} &= \sum\limits_{s}\overline{u}_{2, a}F^{\ddagger}_{ab}u_{1, b}\overline{u}_{1, c}G_{cd}u_{2, d} \\
	&= F^{\ddagger}_{ab}G_{cd}\sum\limits_{s_{1}}\sum\limits_{s_{2}}u_{1, b}\overline{u}_{1, c}u_{2, d}\overline{u}_{2, a} \\
	&= F^{\ddagger}_{ab}G_{cd}\sum\limits_{s_{1}}(u_{1}\overline{u}_{1})_{bc}\sum\limits_{s_{2}}(u_{2}\overline{u}_{2})_{da}.
\end{align*}
In the two spin sums we have elements of $u\overline{u}$, which are given by the normalization, hence
\begin{align*}
	\sum\limits_{s}(\overline{u}_{1}Fu_{2})\cc \overline{u}_{1}Gu_{2} &= F^{\ddagger}_{ab}G_{cd}(\fsl{p}_{1} + \eta_{1}m_{1})_{bc}(\fsl{p}_{2} + \eta_{2}m_{1})_{da} \\
	&= \tr((\fsl{p}_{1} + \eta_{1}m_{1})G(\fsl{p}_{2} + \eta_{2}m_{1})F^{\ddagger}),
\end{align*}
where the $\eta$ are signs which depend on the relevant fermions being particles or antiparticles. 

We will also need a mixing term of the form
\begin{align*}
	\sum\limits_{s}(\overline{u}_{1}Au_{2}\overline{u}_{3}Bu_{4})\cc\overline{u}_{3}Cu_{2}\overline{u}_{1}Du_{4} =& \sum\limits_{s}\overline{u}_{2, a}A_{ab}^{\ddagger}u_{1, b}\overline{u}_{4, c}B_{cd}^{\ddagger}u_{3, d}\overline{u}_{3, e}C_{ef}u_{2, f}\overline{u}_{1, g}D_{gh}u_{4, h} \\
	=& A_{ab}^{\ddagger}B_{cd}^{\ddagger}C_{ef}D_{gh}\sum\limits_{s_{1}}\sum\limits_{s_{2}}\sum\limits_{s_{3}}\sum\limits_{s_{4}}u_{1, b}\overline{u}_{1, g}u_{2, f}\overline{u}_{2, a}u_{3, d}\overline{u}_{3, e}u_{4, h}\overline{u}_{4, c} \\
	=& A_{ab}^{\ddagger}B_{cd}^{\ddagger}C_{ef}D_{gh}(\fsl{p}_{1} + \eta_{1}m_{1})_{bg}(\fsl{p}_{2} + \eta_{2}m_{2})_{fa}(\fsl{p}_{3} + \eta_{3}m_{3})_{de}(\fsl{p}_{4} + \eta_{4}m_{4})_{hc} \\
	=& \tr(A^{\ddagger}(\fsl{p}_{1} + \eta_{1}m_{1})D(\fsl{p}_{4} + \eta_{4}m_{4})B^{\ddagger}(\fsl{p}_{3} + \eta_{3}m_{3})C(\fsl{p}_{2} + \eta_{2}m_{2})).
\end{align*}
These are the formulae we need, and we will handle them using trace theorems.

\paragraph{Fundamental QED Processes}
The fundamental QED processes are the simplest processes we can make with QED vertices. By their nature, these are processes with two incoming and two outgoing particles. These are good for demonstrating Feynman diagrams as there are no undetermined momenta in the diagram, hence we don't have to perform any integrals. The first is electron-electron scattering, represented by
\begin{figure}[!ht]
	\centering
	\feynmandiagram[vertical=b to d]{
		a[particle = $e^{-}$] --[fermion, momentum = $p_{1}$] b --[fermion, momentum' = $p_{3}$] c[particle = $e^{-}$],
		b --[photon] d,
		e[particle = $e^{-}$] --[anti fermion, rmomentum' = $p_{4}$] d --[anti fermion, rmomentum = $p_{2}$] f[particle = $e^{-}$]
	};
	\caption{Electron-electron scattering Feynman diagram.}
	\label{fig:eescat}
\end{figure}

The Feynman rules dictate that
\begin{align*}
	iM = \overline{u}(p_{3})ie\gamma^{\mu}u(p_{1})\frac{-ig_{\mu\nu}}{(p_{1} - p_{3})^{2}}\overline{u}(p_{4})ie\gamma^{\nu}u(p_{2}) = ie^{2}\frac{\overline{u}(p_{3})\gamma^{\mu}u(p_{1})\overline{u}(p_{4})\gamma_{\mu}u(p_{2})}{(p_{1} - p_{3})^{2}}.
\end{align*}
This isn't quite correct, however - as
\begin{align*}
	iM = \mel{p_{3}p_{4}}{S}{p_{1}p_{2}}
\end{align*}
and $\ket{p_{3}p_{4}} = -\ket{p_{4}p_{3}}$, $M$ must be antisymmetric under exchange of the outgoing momenta. This amounts to an equivalent diagram adding to $M$, which means that we must subtract the two. We then find
\begin{align*}
	iM = ie^{2}\left(\frac{\overline{u}(p_{3})\gamma^{\mu}u(p_{1})\overline{u}(p_{4})\gamma_{\mu}u(p_{2})}{(p_{1} - p_{3})^{2}} - \frac{\overline{u}(p_{4})\gamma^{\mu}u(p_{1})\overline{u}(p_{3})\gamma_{\mu}u(p_{2})}{(p_{1} - p_{4})^{2}}\right).
\end{align*}
Using the Mandelstam variables we write this as
\begin{align*}
	iM = ie^{2}\left(\frac{T_{1}}{t} - \frac{T_{2}}{u}\right).
\end{align*}
Seeking the spin average of $\abs{M}^{2}$ we need to calculate spin sums of the products of the $T$s, which is handled using Casimir's trick. In the high energy regime, we can neglect the electron mass to find
\begin{align*}
	\sum\limits_{s}T_{1}\cc T_{1} &= \tr(\fsl{p}_{1}\gamma^{\mu}\fsl{p}_{3}\gamma^{\nu})\tr(\fsl{p}_{4}\gamma_{\mu}\fsl{p}_{2}\gamma_{\nu}) \\
	                              &= p_{1, \rho}p_{3, \sigma}\tr(\gamma^{\rho}\gamma^{\mu}\gamma^{\sigma}\gamma^{\nu})p_{2}^{\alpha}p_{4}^{\beta}\tr(\gamma_{\alpha}\gamma_{\mu}\gamma_{\beta}\gamma_{\nu}) \\
	                              &= 16p_{1, \rho}p_{3, \sigma}p_{2}^{\alpha}p_{4}^{\beta}(g^{\rho\mu}g^{\sigma\nu} - g^{\rho\sigma}g^{\mu\nu} + g^{\rho\nu}g^{\sigma\mu})(g_{\alpha\mu}g_{\beta\nu} - g_{\alpha\beta}g_{\mu\nu} + g_{\alpha\nu}g_{\beta\mu}) \\
	                              &= 16(p_{1}^{\mu}p_{3}^{\nu} - p_{1}^{\sigma}p_{3, \sigma}g^{\mu\nu} + p_{1}^{\nu}p_{3}^{\mu})(p_{2, \mu}p_{4, \nu} - p_{2, \beta}p_{4}^{\beta}g_{\mu\nu} + p_{2, \nu}p_{4, \mu}).
\end{align*}
The collision is satisfies the laws of relativistic dynamics, hence $p_{1}\cdot p_{2} = p_{3}\cdot p_{4}$ as well as $p_{1}\cdot p_{4} = p_{3}\cdot p_{2}$, and
\begin{align*}
	\sum\limits_{s}T_{1}\cc T_{1} =& 16((p_{1}\cdot p_{2})^{2} - (p_{1}\cdot p_{3})(p_{2}\cdot p_{4}) + (p_{1}\cdot p_{4})^{2} - (p_{1}\cdot p_{3})(p_{2}\cdot p_{4}) + 4(p_{1}\cdot p_{3})(p_{2}\cdot p_{4}) - (p_{1}\cdot p_{3})(p_{2}\cdot p_{4}) \\
	                               &+ (p_{1}\cdot p_{4})^{2} - (p_{1}\cdot p_{3})(p_{2}\cdot p_{4}) + (p_{1}\cdot p_{2})^{2}) \\
	                              =& 32((p_{1}\cdot p_{2})^{2} + (p_{1}\cdot p_{4})^{2}).
\end{align*}
Similarly, by exchanging $p_{3}$ and $p_{4}$ we find
\begin{align*}
	\sum\limits_{s}T_{2}\cc T_{2} =& 32((p_{1}\cdot p_{2})^{2} + (p_{1}\cdot p_{3})^{2}).
\end{align*}
Finally we have the mixed term, given by
\begin{align*}
	\sum\limits_{s}T_{1}\cc T_{2} =& \tr(\gamma^{\mu}\fsl{p}_{3}\gamma_{\nu}\fsl{p}_{2}\gamma_{\mu}\fsl{p}_{4}\gamma^{\nu}\fsl{p}_{1}) \\
	=& \tr(\fsl{p}_{4}\gamma^{\nu}\fsl{p}_{1}\gamma^{\mu}\fsl{p}_{3}\gamma_{\nu}\fsl{p}_{2}\gamma_{\mu}) \\
	=& -2\tr(\fsl{p}_{4}\fsl{p}_{3}\gamma^{\mu}\fsl{p}_{1}\fsl{p}_{2}\gamma_{\mu}) \\
	=& -8\tr(\fsl{p}_{4}\fsl{p}_{3}p_{1}\cdot p_{2}) \\
	=& -32(p_{1}\cdot p_{2})^{2}.
\end{align*}
This is real, and therefore equal to its complex conjugate. Finally, neglecting masses we have
\begin{align*}
	\overline{M}^{2} = 32e^{4}\left(\frac{s^{2} + u^{2}}{t^{2}} + \frac{s^{2} + t^{2}}{u^{2}} + \frac{2s^{2}}{tu}\right).
\end{align*}

\paragraph{Non-Abelian Gauge Theory}
The gauge theory of QED is an example of an Abelian gauge theory. For the non-Abelian case, consider a representation $R$ of \SU{N} with dimension $d_{R}$ acting on some multiplet $\Psi$. To add gauge theory, we suppose that the components of $\Psi$ have some internal structure, either that of bosonic or fermionic fields, all with the same mass. As we saw for the QED case, any constant unitary operator $U$ leaves the laws of physics invariant. Allowing $U$ to have spacetime dependence, however, messes with this, in particular the invariance of derivatives. To remedy this, we will introduce a covariant derivative
\begin{align*}
	D_{\mu} = \del{}{\mu} + igT_{a}A_{\mu}^{a} = \del{}{\mu} + ig\textbf{A}_{\mu},
\end{align*}
which contains some coupling constant $g$, the generators $T_{a}$ of \SU{N} and the gauge fields $A_{\mu}^{a}$. This derivative has a matrix structure, and in the final expression, this structure has been baked into the field. The laws of physics will be invariant if
\begin{align*}
	D_{\mu}\Psi \to UD_{\mu}\p \Psi
\end{align*}
under a transformation. We have
\begin{align*}
	D_{\mu}\p(U\Psi) &= (\del{}{\mu} + igT_{a}(A\p)_{\mu}^{a})\left(e^{-igT_{a}\theta^{a}}\Psi\right) \\
	               &= -igT_{a}\del{}{\mu}\theta^{a}e^{-igT_{a}\theta^{a}}\Psi + e^{-igT_{a}\theta^{a}}\del{}{\mu}\Psi + igT_{a}(A\p)_{\mu}^{a}e^{-igT_{a}\theta^{a}}\Psi \\
	               &= U\del{}{\mu}\Psi + \left(igT_{a}(A\p)_{\mu}^{a} - igT_{a}\del{}{\mu}\theta^{a}\right)U\Psi.
\end{align*}
For the transformation rule to be fulfilled, we therefore require
\begin{align*}
	\left(igT_{a}(A\p)_{\mu}^{a} - igT_{a}\del{}{\mu}\theta^{a}\right)U = igUT_{a}A_{\mu}^{a}.
\end{align*}
There are two ways to write the condition. The first is representation dependent, and found by combining the generators and the gauge fields to obtain matrix structure, yielding
\begin{align*}
	igA\p_{\mu}U + \del{}{\mu}U = igUA_{\mu},
\end{align*}
with solution
\begin{align*}
	\textbf{A}\p_{\mu} = U\textbf{A}_{\mu}U\adj + \frac{i}{g}(\del{}{\mu}U)U\adj.
\end{align*}
The other is to write the above without the matrix structure in the fields as
\begin{align*}
	T_{a}(A\p)_{\mu}^{a} = UT_{a}U\adj A_{\mu}^{a} - \frac{i}{g}(igT_{a}\del{}{\mu}\theta^{a})UU\adj = UT_{a}U\adj A_{\mu}^{a} + T_{a}\del{}{\mu}\theta^{a}.
\end{align*}
We can expand the first term according to
\begin{align*}
	UT_{a}U\adj &\approx \left(1 - igT_{b}\theta^{b}\right)T_{a}\left(1 + igT_{c}\theta^{c}\right) \\
	            &\approx T_{a} + ig\left(T_{a}T_{c}\theta^{c} - T_{b}T_{a}\theta^{b}\right) \\
	            &= T_{a} + ig\comm{T_{a}}{T_{b}}\theta^{b} \\
	            &= T_{a} - gf_{abc}T_{c}\theta^{b},
\end{align*}
hence
\begin{align*}
	T_{a}(A\p)_{\mu}^{a} &= T_{a}A_{\mu}^{a} - gf_{abc}T_{c}\theta^{b}A_{\mu}^{a} + T_{a}\del{}{\mu}\theta^{a} \\
	                     &= T_{a}\left(A_{\mu}^{a} + gf_{bca}\theta^{b}A_{\mu}^{c} + \del{}{\mu}\theta^{a}\right)
\end{align*}
and
\begin{align*}
	(A\p)_{\mu}^{a} = A_{\mu}^{a} + gf_{bca}\theta^{b}A_{\mu}^{c} + \del{}{\mu}\theta^{a}.
\end{align*}
We can invoke the adjoint representation, here denoted by elements with small $t$, to find
\begin{align*}
	(A\p)_{\mu}^{a} = A_{\mu}^{a} - ig(t_{b})_{ac}\theta^{b}A_{\mu}^{c} + \del{}{\mu}\theta^{a}.
\end{align*}
This is interesting, as it means that under global transformations, gauge fields always transform according to the adjoint representation.

The gauge fields should be included in the Lagrangian as well. We choose the same form for the Lagrangian as we did for $\text{U}(1)$, but will need an expression for the field strength. We showed a commutation relation previously, and we take that to be defining for the field strength. That is,
\begin{align*}
	igT_{a}F_{\mu\nu}^{a} &= \comm{D_{\mu}}{D_{\nu}}.
\end{align*}
One of the derivative products is given by
\begin{align*}
	(\del{}{\mu} + igT_{a}A_{\mu}^{a})(\del{}{\nu} + igT_{b}A_{\nu}^{b}) &= \del{}{\mu}\del{}{\nu} + ig(\del{}{\mu}T_{a}A_{\nu}^{a} + T_{b}A_{\mu}^{b}\del{}{\nu}) - g^{2}T_{a}T_{b}A_{\mu}^{a}A_{\nu}^{b},
\end{align*}
hence
\begin{align*}
	\comm{D_{\mu}}{D_{\nu}} &= igT_{a}(\del{}{\mu}A_{\nu}^{a} + A_{\mu}^{a}\del{}{\nu} - \del{}{\nu}A_{\mu}^{a} - A_{\nu}^{a}\del{}{\mu}) - g^{2}A_{\mu}^{a}A_{\nu}^{b}\comm{T_{a}}{T_{b}} \\
	                        &= igT_{a}(\del{}{\mu}(A_{\nu}^{a}) - \del{}{\nu}(A_{\mu}^{a})) - ig^{2}A_{\mu}^{a}A_{\nu}^{b}f_{abc}T_{c} \\
	                        &= igT_{a}\left(\del{}{\mu}(A_{\nu}^{a}) - \del{}{\nu}(A_{\mu}^{a}) - gf_{bca}A_{\mu}^{b}A_{\nu}^{c}\right),
\end{align*}
hence
\begin{align*}
	F_{\mu\nu}^{a} = \del{}{\mu}(A_{\nu}^{a}) - \del{}{\nu}(A_{\mu}^{a}) - gf_{bca}A_{\mu}^{b}A_{\nu}^{c}.
\end{align*}
Alternatively we may write this as
\begin{align*}
	\textbf{F}_{\mu\nu} = T_{a}F^{a}_{\mu\nu} = \del{}{\mu}(\textbf{A}_{\nu}) - \del{}{\nu}(\textbf{A}_{\mu}) + ig\comm{\textbf{A}_{\mu}}{\textbf{A}_{\nu}}
\end{align*}
Let us now consider its gauge transformation properties, best demonstrated with the representation-dependent formalism. We have
\begin{align*}
	\textbf{F}_{\mu\nu}\p &= \del{}{\mu}\left(U\textbf{A}_{\nu}U\adj + \frac{i}{g}(\del{}{\nu}U)U\adj\right) - \del{}{\nu}\left(U\textbf{A}_{\mu}U\adj + \frac{i}{g}(\del{}{\mu}U)U\adj\right) + ig\comm{U\textbf{A}_{\mu}U\adj + \frac{i}{g}(\del{}{\mu}U)U\adj}{U\textbf{A}_{\nu}U\adj + \frac{i}{g}(\del{}{\nu}U)U\adj}.
\end{align*}
Woah, that is a lot of terms. We can eliminate a few of them, however, using the following results:
\begin{align*}
	&\comm{U\textbf{A}_{\mu}U\adj}{U\textbf{A}_{\nu}U\adj} = U\comm{\textbf{A}_{\mu}}{\textbf{A}_{\nu}}U\adj, \\
	&\del{}{\mu}(UU\adj) = 0 \implies \del{}{\mu}(U\adj) = -U\adj\del{}{\mu}(U)U\adj.
\end{align*}
Next we have
\begin{align*}
	\del{}{\mu}\left(U\textbf{A}_{\nu}U\adj + \frac{i}{g}(\del{}{\nu}U)U\adj\right) &= \del{}{\mu}(U)\textbf{A}_{\nu}U\adj + U\del{}{\mu}(\textbf{A}_{\nu})U\adj - U\textbf{A}_{\nu}U\adj\del{}{\mu}(U)U\adj + \frac{i}{g}\left((\del{}{\mu}\del{}{\nu}U)U\adj + (\del{}{\nu}U)(\del{}{\mu}U\adj)\right) \\
	&= \del{}{\mu}(U)\textbf{A}_{\nu}U\adj + U\del{}{\mu}(\textbf{A}_{\nu})U\adj - U\textbf{A}_{\nu}U\adj\del{}{\mu}(U)U\adj + \frac{i}{g}\left((\del{}{\mu}\del{}{\nu}U)U\adj - (\del{}{\nu}U)U\adj(\del{}{\mu}U)U\adj\right), \\
	\comm{U\textbf{A}_{\mu}U\adj}{(\del{}{\nu}U)U\adj} &= U\textbf{A}_{\mu}U\adj(\del{}{\nu}U)U\adj - (\del{}{\nu}U)U\adj U\textbf{A}_{\mu}U\adj \\
	&= U\textbf{A}_{\mu}U\adj(\del{}{\nu}U)U\adj - (\del{}{\nu}U)\textbf{A}_{\mu}U\adj, \\
	\comm{(\del{}{\mu}U)U\adj}{(\del{}{\nu}U)U\adj} &= (\del{}{\mu}U)U\adj(\del{}{\nu}U)U\adj - (\del{}{\nu}U)U\adj(\del{}{\mu}U)U\adj.
\end{align*}
We can now look at the terms with and without gauge fields separately. The first add to
\begin{align*}
	 &\del{}{\mu}(U)\textbf{A}_{\nu}U\adj + U\del{}{\mu}(\textbf{A}_{\nu})U\adj - U\textbf{A}_{\nu}U\adj\del{}{\mu}(U)U\adj - \left(\del{}{\nu}(U)\textbf{A}_{\mu}U\adj + U\del{}{\nu}(\textbf{A}_{\mu})U\adj - U\textbf{A}_{\mu}U\adj\del{}{\nu}(U)U\adj\right) \\
	-& \left(U\textbf{A}_{\mu}U\adj(\del{}{\nu}U)U\adj - (\del{}{\nu}U)\textbf{A}_{\mu}U\adj - \left(U\textbf{A}_{\nu}U\adj(\del{}{\mu}U)U\adj - (\del{}{\mu}U)\textbf{A}_{\nu}U\adj\right)\right) \\
	=& U\left(\del{}{\mu}(\textbf{A}_{\nu}) - \del{}{\nu}(\textbf{A}_{\mu})\right)U\adj.
\end{align*}
Naturally the commutator also produces a term $U\comm{\textbf{A}_{\mu}}{\textbf{A}_{\nu}}U\adj$. The other terms add up to 
\begin{align*}
	&\frac{i}{g}\left((\del{}{\mu}\del{}{\nu}U)U\adj - (\del{}{\nu}U)U\adj(\del{}{\mu}U)U\adj - (\del{}{\nu}\del{}{\mu}U)U\adj + (\del{}{\mu}U)U\adj(\del{}{\nu}U)U\adj\right) - \frac{i}{g}\left((\del{}{\mu}U)U\adj(\del{}{\nu}U)U\adj - (\del{}{\nu}U)U\adj(\del{}{\mu}U)U\adj\right),
\end{align*}
which add up to zero. Thus we have
\begin{align*}
	\textbf{F}_{\mu\nu}\p = U\textbf{F}_{\mu\nu}U\adj.
\end{align*}
Constructing $\textbf{F}_{\mu\nu}\adj\textbf{F}^{\mu\nu}$ guarantees Lorentz invariance, but this object still has matrix structure, and it is not gauge invariant. We can, however, construct $\tr(\textbf{F}_{\mu\nu}\adj\textbf{F}^{\mu\nu})$, with tracing over the multiplet space, which is gauge invariant. We can write this as
\begin{align*}
	\tr(\textbf{F}_{\mu\nu}\adj\textbf{F}^{\mu\nu}) &= (F_{\mu\nu}^{a})\cc F^{b, \mu\nu}\tr(T_{a}T_{b}) \\
	                                            &= C(F_{\mu\nu}^{a})\cc F^{b, \mu\nu}\delta_{ab} \\
	                                            &= C(F_{\mu\nu}^{a})\cc F^{a, \mu\nu},
\end{align*}
hence we choose the free-field Lagrangian
\begin{align*}
	\lag = -\frac{1}{4}(F_{\mu\nu}^{a})\cc F^{a, \mu\nu}.
\end{align*}

In general there could be mass terms of the form $A_{\mu}^{a}A^{\mu, a}$ in the Lagrangian, but it turns out that these are not gauge invariant. Thus, as long as the gauge symmetry is exact, the gauge bosons must be massless.

The propagator for this kind of gauge field is generally not defined unless you specify a gauge. The Lagrangian has terms of the same form as in QED, but with some self-interaction terms as well. The free part is the same, however, meaning we can use a gauge fixing term
\begin{align*}
	\lag = -\frac{1}{2\xi}(\del{}{\mu}A^{\mu}_{a})\adj (\del{}{\nu}A^{\nu}_{a}).
\end{align*}
Now the propagator will be the same as in QED, as will other Feynman rules.

Intrinsic to this kind of field theory is interactions of the gauge field with itself. These may be both cubic and quartic. Introducing $E_{\alpha\beta\gamma\delta} = g_{\alpha\gamma}g_{\beta\delta} - g_{\alpha\delta}g_{\beta\gamma}$ the Feynman rules are
\begin{align*}
	gf_{ab\overline{c}}\left(p^{\alpha}E_{\alpha\mu\nu\lambda} + k^{\alpha}E_{\alpha\nu\lambda\mu} + q^{\alpha}E_{\alpha\lambda\mu\nu}\right)
\end{align*}
for the cubic vertex and
\begin{align*}
	-ig^{2}\left(f_{ab\overline{e}}f_{cde}E_{\mu\nu\lambda\rho} + f_{ac\overline{e}}f_{dbe}E_{\mu\lambda\rho\nu} + f_{ad\overline{e}}f_{bce}E_{\mu\rho\nu\lambda}\right)
\end{align*}
for the quartic vertex.

Let us now study the interactions of the gauge fields with matter. Taking the fermionic case first, the field multiplet $\Psi$ is described by
\begin{align*}
	\lag = \sum\limits_{i = 1}^{d_{R}}i\overline{\Psi}^{i}\fsl{\del{}{}}\Psi^{i} - m\overline{\Psi}^{i}\Psi^{i}.
\end{align*}
In terms of matrix notation we can write this as
\begin{align*}
	\lag = i\overline{\Psi}\fsl{\del{}{}}\Psi - m\overline{\Psi}\Psi.
\end{align*}
The Dirac matrices work on the individual fields in the multiplet, hence in this notation they are diagonal matrices. Replacing the partial derivative with the covariant derivative ensures gauge symmetry. Emerging from this Lagrangian is now an interaction term
\begin{align*}
	\lag_{\text{int}} = -g\overline{\Psi}\fsl{A}\Psi = -g\overline{\Psi}\gamma^{\mu}T_{a}\Psi A_{\mu}^{a}.
\end{align*}
This adds a vertex rule $-ig\gamma^{\mu}(T_{a})_{ij}$, where $ij$ are indices specifying the interacting fields.

Next we consider interactions with scalars. The Lagrangian is
\begin{align*}
	\lag &= (D_{\mu}\Phi)\adj D^{\mu}\Phi - M^{2}\Phi\adj\Phi \\
	     &= (\del{}{\mu}\Phi)\adj \del{\mu}{}\Phi + ig\left((\del{}{\mu}\Phi)\adj T_{a}A^{a, \mu}\Phi - \Phi\adj T_{a}\adj (A_{\mu}^{a})\adj\del{\mu}{}\Phi\right) + g^{2}\Phi\adj T_{a}\adj (A_{\mu}^{a})\adj T_{b}A^{b, \mu}\Phi - M^{2}\Phi\adj\Phi \\
	     &= (\del{}{\mu}\Phi)\adj \del{\mu}{}\Phi + ig\left((\del{}{\mu}\Phi)\adj T_{a}A^{a, \mu}\Phi - \Phi\adj T_{a}\adj (A_{\mu}^{a})\adj\del{\mu}{}\Phi\right) + g^{2}g_{\mu\nu}\Phi\adj T_{a}\adj (A_{\mu}^{a})\adj T_{b}A_{\nu}^{b}\Phi - M^{2}\Phi\adj\Phi.
\end{align*}
We thus see that there are two kinds of vertices, one with two particles and one gauge bosons, and one with two particles and two gauge bosons. The corresponding Feynman rules are $-ig(p + p\p)_{\mu}(T_{a})_{ij}$ and $ig^{2}g_{\mu\nu}(T_{a}T_{b} + T_{b}T_{a})_{ij}$. Note that the structure of the generators dictates which particles can interact in both cases.

\paragraph{Quantum Chromodynamics}
Quantum chromodynamics is the theory of the strong nuclear force. It is a quantized \SU{3} gauge theory. The number 3 is due to the presence of 3 color charges. We choose \SU{3} rather than \SU{2}, despite the latter having a triplet representation because we need a triplet representation for both the quarks and antiquarks, but \SU{2} has only real representations.

%TODO: Add something on Noether currents. See https://physics.stackexchange.com/questions/56886/why-is-color-conserved-in-qcd
The gauge bosons come in eight forms. They are called gluons.

\paragraph{The Weak Force}
The weak force started as an attempt to describe beta decay. The first attempt was to formulate a theory involving an interaction vertex with four fermions, where currents interacted. This turned out not to work, however, as it did not contain parity violation. The fact that neutrinos are only found with certain chiralities made clear the need to introduce this concept as well. Both of these could be remedied by adding $\gamma^{\mu}(1 - \gamma^{5})$ to the Dirac field bilinears. This will bring us closer, but still not leave us with a renormalizable theory, as the coupling constant would have mass dimension $-2$.

To remedy the above, one could strengthen the analogy to QED by adding a propagator to the interaction. This would remove the four-fermion vertex, replacing it with vertices involving a vector boson. This boson must be heavy, as it has never been observed. The propagator generally takes the form
\begin{align*}
	\frac{-ig_{\mu\nu} - \frac{q_{\mu}q_{\nu}}{M^{2}}}{q^{2} - m^{2}}.
\end{align*}

One important question is why the weak force is so weak. One way to understand this is to compare matrix elements with and without the mediating boson. The latter is of the form
\begin{align*}
	iM = i\frac{G}{\sqrt{2}}\left(\overline{u}_{3}\gamma^{\mu}(1 - \gamma^{5})u_{1}\right)\left(\overline{u}_{4}\gamma^{\mu}(1 - \gamma^{5})u_{2}\right),
\end{align*}
while the former is of the form
\begin{align*}
	iM = i\left(\frac{g}{\sqrt{2}}\overline{u}_{3}\gamma^{\mu}\frac{1}{2}(1 - \gamma^{5})u_{1}\right)\frac{1}{M^{2} - q^{2}}\left(\frac{g}{\sqrt{2}}\overline{u}_{4}\gamma^{\mu}\frac{1}{2}(1 - \gamma^{5})u_{2}\right),
\end{align*}
and in the low-energy regime
\begin{align*}
	\frac{G}{\sqrt{2}} = \frac{g^{2}}{8M^{2}},
\end{align*}
hence the heaviness of the gauge boson makes the effective interaction constant so small.