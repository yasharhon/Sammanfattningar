\section{Basic Concepts}

\paragraph{The Standard Model}
The standard model is a quantum field theory with the gauge group $\SU{3}\times\SU{2}\times\text{U}(1)$. It is comprised of
\begin{itemize}
	\item matter fields, which represent fermions.
	\item gauge bosons - spin $1$ bosons that mediate the fundamental forces.
	\item a scalar field, which represents the Higgs boson.
\end{itemize}

\paragraph{The Fundamental Forces}
The fundamental forces are the strong and weak nuclear interactions, the electromagnetic interaction and the gravitational interactions. Only the three former have a corresponding quantum field theory, hence this discussion (and in fact the standard model itself at time of writing) will be restricted to those.

An important aspect about the strong force is the fact that its coupling constant is rolling - that is, it varies in space. It is very large at distances comparable to those of nuclei, but very small at the scale of an individual nucleon. This means, for instance, that quarks in a proton are virtually free within the proton, but that the proton itself is stable.

\paragraph{Elementary Particles}
An elementary particle is a particle without substructure.

\paragraph{The Building Blocks of the Standard Model}
The fermions comprising the standard model come in two kinds: quarks and leptons. Each quark also has an antiquark. The quarks have colour charge red, green or blue, which is involved in the strong interaction. In nature they are not found as free particles, but are either joined with an antiquark to create a meson, or joined with two other quarks to make a baryon. Leptons, by contrast, have no colour charge. Both quarks and leptons may have electric charge, and they all have spin and interact with the weak interaction. They are also divided into three so-called generations.

\paragraph{Quantum Numbers}
In addition to charge and momentum, we introduce quantum numbers that must be conserved at vertices in Feynman diagrams. These are:
\begin{itemize}
	\item lepton number.
	\item generational lepton number.
	\item baryon number.
	\item strangeness (not conserved in weak interactions).
	\item colour.
\end{itemize}
These quantum numbers are flipped in the antiparticles.

\paragraph{QCD Vertices}
QCD, or quantum chromodynamics, is the field theory describing the strong force. Unlike QED, it has different kinds of vertices as the gluons carry colour charge. The one including fermions is exemplified as in figure \ref{fig:strong_vert_fer_ex}, and symbolizes an incoming up quark with blue colour charge interacting via a gluon and leaving the interaction with red colour charge. The gluon then carries one positive unit of red colour charge and one unit of negative red colour charge.

\begin{figure}[!ht]
	\centering
	\feynmandiagram[horizontal = a to b]{
		c --[fermion, edge label = $u(b)$] a --[fermion, edge label = $u(r)$] d,
		a --[gluon, edge label = $g({b, \bar{r}})$] b,
	};
	\caption{An example of a strong interaction vertex involving a fermion.}
	\label{fig:strong_vert_fer_ex}
\end{figure}

In addition there are two pure gluon vertices, shown in figure \ref{fig:strong_vert_glu}.

\begin{figure}[!ht]
	\centering
	\feynmandiagram[horizontal = a to b]{
		c --[gluon] a --[gluon] d,
		a --[gluon] b,
	};
	\feynmandiagram[vertical = a to b]{
		a --[gluon] e --[gluon] b,
		c --[gluon] e --[gluon] d,
	};
	\caption{The two strong interaction vertices only involving gluons.}
	\label{fig:strong_vert_glu}
\end{figure}

\paragraph{Weak Interaction Vertices}
The weak interaction is mediated by a charged $W$ boson and a neutral $Z$ boson. The corresponding vertices are shown in figure \ref{fig:weak_verts}.

\begin{figure}[!ht]
	\centering
	\feynmandiagram[horizontal = a to b]{
		c --[fermion, edge label = $u(b)$] a --[fermion, edge label = $u(r)$] d,
		a --[charged scalar, edge label = $W$] b,
	};
	\feynmandiagram[horizontal = a to b]{
		c --[fermion] a --[fermion] d,
		a --[scalar, edge label = $Z$] b,
	};
	\caption{The fundamental vertices of the weak interaction.}
	\label{fig:weak_verts}
\end{figure}