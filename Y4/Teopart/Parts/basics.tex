\section{Basic Concepts}

\paragraph{The Standard Model}
The standard model is a quantum field theory with the gauge group $\SU{3}\times\SU{2}\times\text{U}(1)$. It is comprised of
\begin{itemize}
	\item matter fields, which represent fermions.
	\item gauge bosons - spin $1$ bosons that mediate the fundamental forces.
	\item a scalar field, which represents the Higgs boson.
\end{itemize}

\paragraph{The Fundamental Forces}
The fundamental forces are the strong and weak nuclear interactions, the electromagnetic interaction and the gravitational interactions. Only the three former have a corresponding quantum field theory, hence this discussion (and in fact the standard model itself at time of writing) will be restricted to those.

\paragraph{Elementary Particles}
An elementary particle is a particle without substructure.

\paragraph{The Building Blocks of the Standard Model}
The fermions comprising the standard model come in two kinds: quarks and leptons. Each quark also has an antiquark. The quarks have colour charge red, green or blue, which is involved in the strong interaction. In nature they are not found as free particles, but are either joined with an antiquark to create a meson, or joined with two other quarks to make a baryon. Leptons, by contrast, have no colour charge. Both quarks and leptons may have electric charge, and they all have spin and interact with the weak interaction. They are also divided into three so-called generations.

\paragraph{Quantum Numbers}
In addition to charge and momentum, we introduce quantum numbers that must be conserved at vertices in Feynman diagrams. These are:
\begin{itemize}
	\item lepton number.
	\item generational lepton number.
	\item baryon number.
	\item strangeness (not conserved in weak interactions).
	\item colour.
\end{itemize}
These quantum numbers are flipped in the antiparticles.