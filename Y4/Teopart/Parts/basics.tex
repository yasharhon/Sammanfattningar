\section{Basic Concepts}

\paragraph{The Standard Model}
The standard model is a quantum field theory with the gauge group $\SU{3}\times\SU{2}\times\text{U}(1)$. It is comprised of
\begin{itemize}
	\item matter fields, which represent fermions.
	\item gauge bosons - spin $1$ bosons that mediate the fundamental forces.
	\item a scalar field, which represents the Higgs boson.
\end{itemize}

\paragraph{The Fundamental Forces}
The fundamental forces are the strong and weak nuclear interactions, the electromagnetic interaction and the gravitational interactions. Only the three former have a corresponding quantum field theory, hence this discussion (and in fact the standard model itself at time of writing) will be restricted to those.

An important aspect about the strong force is the fact that its coupling constant is rolling - that is, it varies in space. It is very large at distances comparable to those of nuclei, but very small at the scale of an individual nucleon. This means, for instance, that quarks in a proton are virtually free within the proton, but that the proton itself is stable.

\paragraph{Elementary Particles}
An elementary particle is a particle without substructure.

\paragraph{The Building Blocks of the Standard Model}
The fermions comprising the standard model come in two kinds: quarks and leptons. Each quark also has an antiquark. The quarks have colour charge red, green or blue, which is involved in the strong interaction. In nature they are not found as free particles, but are either joined with an antiquark to create a meson, or joined with two other quarks to make a baryon. In general, combinations of quarks are called hadrons. Leptons, by contrast, have no colour charge. Both quarks and leptons may have electric charge, and they all have spin and interact with the weak interaction. They are also divided into three so-called generations.

\paragraph{Quantum Numbers}
In addition to charge and momentum, we introduce quantum numbers that must be conserved at vertices in Feynman diagrams. These are:
\begin{itemize}
	\item lepton number.
	\item generational lepton number.
	\item baryon number.
	\item strangeness (not conserved in weak interactions).
	\item colour.
\end{itemize}
These quantum numbers are flipped in the antiparticles.

\paragraph{QCD Vertices}
QCD, or quantum chromodynamics, is the field theory describing the strong force. Unlike QED, it has different kinds of vertices as the gluons carry colour charge. The one including fermions is exemplified as in figure \ref{fig:strong_vert_fer_ex}, and symbolizes an incoming up quark with blue colour charge interacting via a gluon and leaving the interaction with red colour charge. The gluon then carries one positive unit of red colour charge and one unit of negative red colour charge.

\begin{figure}[!ht]
	\centering
	\feynmandiagram[horizontal = a to b]{
		c --[fermion, edge label = $u(b)$] a --[fermion, edge label = $u(r)$] d,
		a --[gluon, edge label = $g({b, \bar{r}})$] b,
	};
	\caption{An example of a strong interaction vertex involving a fermion.}
	\label{fig:strong_vert_fer_ex}
\end{figure}

In addition there are two pure gluon vertices, shown in figure \ref{fig:strong_vert_glu}.

\begin{figure}[!ht]
	\centering
	\feynmandiagram[horizontal = a to b]{
		c --[gluon] a --[gluon] d,
		a --[gluon] b,
	};
	\feynmandiagram[vertical = a to b]{
		a --[gluon] e --[gluon] b,
		c --[gluon] e --[gluon] d,
	};
	\caption{The two strong interaction vertices only involving gluons.}
	\label{fig:strong_vert_glu}
\end{figure}

\paragraph{Weak Interaction Vertices}
The weak interaction is mediated by a charged $W$ boson and a neutral $Z$ boson. The corresponding vertices are shown in figure \ref{fig:weak_verts}.

\begin{figure}[!ht]
	\centering
	\feynmandiagram[horizontal = a to b]{
		c --[fermion, edge label = $u(b)$] a --[fermion, edge label = $u(r)$] d,
		a --[charged scalar, edge label = $W$] b,
	};
	\feynmandiagram[horizontal = a to b]{
		c --[fermion] a --[fermion] d,
		a --[scalar, edge label = $Z$] b,
	};
	\caption{The fundamental vertices of the weak interaction.}
	\label{fig:weak_verts}
\end{figure}

\paragraph{The Jacobi Identity for Structure Constants}
Applying the Jacobi identity we find
\begin{align*}
	\comm{X_{a}}{\comm{X_{b}}{X_{c}}} + \comm{X_{b}}{\comm{X_{c}}{X_{a}}} + \comm{X_{c}}{\comm{X_{a}}{X_{b}}} &= i\left(f_{bcd}\comm{X_{a}}{X_{d}} + f_{cad}\comm{X_{b}}{X_{d}} + f_{abd}\comm{X_{c}}{X_{d}}\right) \\
	                                  &= i^{2}\left(f_{bcd}f_{ade} + f_{cad}f_{bde} + f_{abd}f_{cde}\right)X_{e} = 0,
\end{align*}
which means
\begin{align*}
	f_{bcd}f_{ade} + f_{cad}f_{bde} + f_{abd}f_{cde} = 0.
\end{align*}

\paragraph{Structure Constant Representations}
For a group, consider the structure constants
\begin{align*}
	\comm{X_{a}}{X_{b}} = if_{abc}X_{c}.
\end{align*}
Constructing the matrices
\begin{align*}
	(T_{a})_{bc} = -if_{abc}
\end{align*}
we find
\begin{align*}
	\comm{T_{a}}{T_{b}}_{cd} = -(f_{ace}f_{bed} - f_{bce}f_{aed}) = f_{cae}f_{bed} + f_{bce}f_{aed} = -f_{abe}f_{ced} = if_{abe}(T_{e})_{cd},
\end{align*}
and thus
\begin{align*}
	\comm{T_{a}}{T_{b}} = if_{abe}T_{e},
\end{align*}
meaning that the structure constants themselves provide a representation of the group. This is a real representation, i.e. is identical to its conjugate representation.

\paragraph{A Closer Look at \SU{2}}
The Lie algebra of \SU{2} is
\begin{align*}
	\comm{J_{i}}{J_{j}} = i\varepsilon_{ijk}J_{k}.
\end{align*}
Our goal is to block diagonalize the space on which elements in the group act, starting by looking at the generators. There are three generators, so we can only diagonalize the one. We choose $J_{3}$, which will net us the so-called spin representation. Taking the space to be finite dimensional (which in the context of angular momentum is explained by physical arguments), we consider the eigenstate $\ket{j, \alpha}$ with the highest eigenvalue $j$ of $J_{3}$. $\alpha$ contains other information, the existence of which is undetermined as of yet. Next we introduce the raising and lowering operators
\begin{align*}
	J_{\pm} = \frac{1}{\sqrt{2}}(J_{1} \pm iJ_{2}).
\end{align*}
We have
\begin{align*}
	&\comm{J_{+}}{J_{-}} = \frac{1}{2}i(-\comm{J_{1}}{J_{2}} + \comm{J_{2}}{J_{1}}) = J_{3}, \\
	&\comm{J_{3}}{J_{\pm}} = \frac{1}{\sqrt{2}}(\comm{J_{3}}{J_{1}} \pm i\comm{J_{3}}{J_{2}}) = \frac{1}{\sqrt{2}}(\pm J_{1} + iJ_{2}) = \pm J_{\pm}.
\end{align*}
For a general eigenstate we find
\begin{align*}
	J_{3}J_{\pm}\ket{m, \alpha} = (\pm J_{\pm} + J_{\pm}J_{3})\ket{m, \alpha} = (m \pm 1)J_{\pm}\ket{m, \alpha},
\end{align*}
and the raising and lowering operators do indeed work as expected. This implies $J_{+}\ket{j, \alpha} = 0$. Furthermore, we have
\begin{align*}
	\mel{j, \beta}{J_{-}\adj J_{-}}{j, \alpha} &= \mel{j, \beta}{J_{+}J_{-}}{j, \alpha} \\
	                                           &= \mel{j, \beta}{J_{+}J_{-} - J_{-}J_{+}}{j, \alpha} \\
	                                           &= \mel{j, \beta}{\comm{J_{+}}{J_{-}}}{j, \alpha} \\
	                                           &= \mel{j, \beta}{J_{3}}{j, \alpha} \\
	                                           &= j\delta_{\alpha\beta},
\end{align*}
as we assume the $j$ states to be orthonormal. This implies that the raising and lowering operators preserve orthogonality in terms of the other quantum numbers. Introducing $N_{j} = \sqrt{j}$ we may thus write
\begin{align*}
	J_{-}\ket{j, \alpha} = N_{j}\ket{j - 1, \alpha}.
\end{align*}
Similarly we find
\begin{align*}
	J_{+}\ket{j - 1, \alpha} &= \frac{1}{N_{j}}J_{+}J_{-}\ket{j, \alpha} \\
	                         &= \frac{1}{N_{j}}\comm{J_{+}}{J_{-}}\ket{j, \alpha} \\
	                         &= N_{j}\ket{j, \alpha}.
\end{align*}
We generalize this to
\begin{align*}
	J_{-}\ket{m, \alpha} = N_{m}\ket{m - 1, \alpha},\ J_{+}\ket{m - 1, \alpha} = N_{m}\ket{m, \alpha}.
\end{align*}
These satisfy a recursion relation which we can find by writing
\begin{align*}
	N_{m}^{2} &= \mel{m, \alpha}{J_{+}J_{-}}{m, \alpha} \\
	          &= \mel{m, \alpha}{\comm{J_{+}}{J_{-}} + J_{-}J_{+}}{m, \alpha} \\
	          &= m + N_{m + 1}^{2}\braket{m + 1, \alpha}{m + 1, \alpha} \\
	          &= m + N_{m + 1}^{2}.
\end{align*}
Repeating this we find
\begin{align*}
	N_{j - k}^{2} &= \sum\limits_{a = j - k}^{j}a \\
	              &= \sum\limits_{n = 1}^{k + 1}j - k + n - 1 \\
	              &= \frac{k + 1}{2}(2(j - k) + k) \\
	              &= \frac{(2j - k)(k + 1)}{2} \\
	              &= (j - (j - k) + 1)j - \frac{1}{2}(k + 1)k = \frac{1}{2}(k + 1)(2j - k),
\end{align*}
or
\begin{align*}
	N_{m} = \frac{1}{\sqrt{2}}\sqrt{(j - m + 1)(j + m)}.
\end{align*}
From this we can infer two things about the spectrum. First, by the assumption that Hilbert space is finite dimensional, there must exist a maximal number of lowerings $l$. This case should satisfy
\begin{align*}
	N_{j - l} = \frac{1}{\sqrt{2}}\sqrt{(l + 1)(2j - l)} = 0.
\end{align*}
As $l$ is an integer, the possible upper bounds of the spectrum must therefore be half-integer. Next, we can identify the limits of the eigenvalue spectrum: The upper limit corresponds to the fact that $N_{j + 1} = 0$. The lower limit comes from $N_{-j} = 0$, meaning the eigenstate at the bottom of the spectrum is $\ket{-j, \alpha}$. At this point we may ignore the other quantum numbers, as the representation we are working with breaks Hilbert space into invariant subspaces under \SU{2}. Or, rather, we will replace them simply by the value of $j$, which together with $m$ specifies the state. We now know that the dimension of each invariant Hilbert space is $2j + 1$.

We can now work out the particulars of this representation by computing the matrix elements
\begin{align*}
	&\mel{j, m}{J_{3}}{j, m\p} = m\delta_{m, m\p}, \\
	&\mel{j, m}{J_{+}}{j, m\p} = N_{m\p + 1}\delta_{m, m\p + 1} = \frac{1}{\sqrt{2}}\sqrt{(j - m\p)(j + m\p + 1)}\delta_{m, m\p + 1}, \\
	&\mel{j, m}{J_{-}}{j, m\p} = N_{m\p}\delta_{m, m\p - 1} = \frac{1}{\sqrt{2}}\sqrt{(j - m\p + 1)(j + m\p)}\delta_{m, m\p - 1},
\end{align*}
the latter of which can be used to yield
\begin{align*}
	&\mel{j, m}{J_{1}}{j, m\p} = \frac{1}{2}\left(\sqrt{(j - m\p)(j + m\p + 1)}\delta_{m, m\p + 1} + \sqrt{(j - m\p + 1)(j + m\p)}\delta_{m, m\p - 1}\right), \\
	&\mel{j, m}{J_{2}}{j, m\p} = -\frac{i}{2}\left(\sqrt{(j - m\p)(j + m\p + 1)}\delta_{m, m\p + 1} - \sqrt{(j - m\p + 1)(j + m\p)}\delta_{m, m\p - 1}\right).
\end{align*}
As an example, the representation on the $j = \frac{1}{2}$ subspace is
\begin{align*}
	J_{1} = \frac{1}{2}\mqty[
		0 & 1 \\
		1 & 0
	],\ J_{2} = \frac{1}{2}\mqty[
		0 & -i \\
		i & 0
	],\ J_{3} = \frac{1}{2}\mqty[
		1 & 0 \\
		0 & -1
	].
\end{align*}
This is the simplest representation of \SU{2}, and is therefore called the fundamental representation. Another example is found on the $j = 1$ subspace, which yields
\begin{align*}
	J_{1} = \mqty[
		0                  & \frac{1}{\sqrt{2}} & 0 \\
		\frac{1}{\sqrt{2}} & 0                  & \frac{1}{\sqrt{2}} \\
		0                  & \frac{1}{\sqrt{2}} & 0
	],\ J_{2} = \mqty[
		0                  & -\frac{i}{\sqrt{2}} & 0 \\
		\frac{i}{\sqrt{2}} & 0                   & -\frac{i}{\sqrt{2}} \\
		0                  & \frac{i}{\sqrt{2}}  & 0
	],\ J_{3} = \mqty[
		1 & 0 & 0 \\
		0 & 0 & 0 \\
		0 & 0 & -1
	].
\end{align*}

\paragraph{Tensor Product of \SU{2} Representations}
Taking the tensor product of two angular momentum Hilbert spaces corresponds to creating a new space of eigenstates of either angular momentum operator. As we know, however, it is possible to diagonalize this Hilbert space in terms of another operator, namely the total angular momentum. As an example, for two $j = \frac{1}{2}$, the total Hilbert space may be written as the tensor product of two two-dimensional subspaces, or as the direct sum of the eigenspaces of the total angular momentum. The latter has two eigenspaces, one with dimension $3$ and one with dimension $1$. We often denote this in the ridiculous form
\begin{align*}
	2\otimes 2 = 3\oplus 1.
\end{align*}

\paragraph{Isospin}
Early particle physicists, noting the very similar masses of the proton and neutron and studying nuclear energy levels, conjectured the existence of a symmetry of nuclei under swapping of the two. This symmetry would then need to be respected by the strong force. More specifically, they wrote the state of each nucleon as a proton-neutron doublet which transforms under action by \SU{2}. The generators of this transformation are called components of isospin. We will do a similar thing, but for quarks instead.

\paragraph{Parity}
Parity is a discrete symmetry which corresponds to spatial inversion. Strong and electromagnetic interactions respect parity, but weak interactions do not.

It turns out that the ground states of hadrons are eigenstates of the parity operator. It also turns out that the parity of fermions (taken to be positive) is opposite to that of their antiparticles, whereas they are the sames for bosons. In addition to this there comes parity from their orbital angular momentum.

\paragraph{Chirality and Helicity}
As has been discussed in other summaries, helicity and chirality depend on the combination of spin and momentum. It turns out that helicity is frame-dependent, but chirality is intrinsic to the particle. Furthermore, some particles are only found with certain chiralities.

\paragraph{Going From Particles to Antiparticles}
Consider the particle and antiparticle states
\begin{align*}
	e^{-ipx}\mqty[
		1 \\
		0 \\
		\frac{p_{x} - ip_{y}}{p_{0} + m} \\
		-\frac{p_{	z}}{p_{0} + m}
	],\ e^{ipx}\mqty[
		\frac{p_{z}}{p_{0} + m} \\
		\frac{p_{x} + ip_{y}}{p_{0} + m} \\
		1 \\
		0
	].
\end{align*}
We will try to devise a transformation between the two. The exponent is flipped by complex conjugation, so that will have to be included. Next we need a transformation matrix which is block off-diagonal. The lower left block being
\begin{align*}
	\mqty[
		0  & 1 \\
		-1 & 0
	] = -i\sigma^{2}
\end{align*}
will get the job done. The upper right block should similarly be $i\sigma^{2}$, meaning that the total matrix should be $-i\gamma^{2}$. The transformation is thus
\begin{align*}
	\Psi\p = -i\gamma^{2}\Psi\cc.
\end{align*}
This also reveals the need for complex conjugation in changing between and antiparticles.

\paragraph{Charge Conjugation}
Charge conjugation is a discrete symmetry which corresponds to changing all internal quantum numbers of a state, creating a corresponding antiparticle. It is written as
\begin{align*}
	C\ket{p} = \ket{\bar{p}}.
\end{align*}
It is respected by the strong and electromagnetic interactions, but not the weak one.

In particular, for fermions we can use the transformation rule we derived on the Dirac equation with minimal coupling. For the particle we have
\begin{align*}
	\gamma^{\mu}(\del{}{\mu} - ieA_{\mu})\Psi + im\Psi = 0.
\end{align*}
For the antiparticle we have
\begin{align*}
	-i\gamma^{2}(\gamma^{\mu})\cc(\del{}{\mu} + ieA_{\mu})\Psi\cc - i\gamma^{2}\cdot -im\Psi\cc = -i\gamma^{2}(\gamma^{\mu})\cc(\del{}{\mu} + ieA_{\mu})\Psi\cc - m\gamma^{2}\Psi\cc = 0.
\end{align*}
In the Dirac representation the only matrix that is changed by complex conjugation is $\gamma^{2}$, which changes sign. All the other gamma matrices anticommute, and for $\gamma^{2}$ we can simply reshuffle the complex conjugation to find
\begin{align*}
	\gamma^{\mu}(\del{}{\mu} + ieA_{\mu})\cdot i\gamma^{2}\Psi\cc + im\cdot i\gamma^{2}\Psi\cc = 0.
\end{align*}
This implies
\begin{align*}
	C\Psi = i\gamma^{2}\Psi\cc.
\end{align*}

Such symmetries can be used to determine possible reactions. A trivial example is $\pi^{0} \to \gamma + \gamma$, which balances charge conjugation, whereas $\pi^{0} \to \gamma + \gamma + \gamma$ does not, hence only the former is allowed.

\paragraph{Transformation of Antiquark Isospin}
We know that quark doublets transform under \SU{2}. An antiquark doublet, however, is related to the complex conjugate of the quark doublet. The complex conjugate itself does not transform the same way as the original doublet - more specifically, we have
\begin{align*}
	(\phi\p)\cc = e^{\frac{i}{2}\theta_{i}I_{i}\cc}\phi\cc,
\end{align*}
which is the wrong transformation rule. We therefore need to write the antiparticle doublet as $\varepsilon\phi\cc$, and choose $\varepsilon$ such that
\begin{align*}
	\varepsilon e^{\frac{i}{2}\theta_{i}I_{i}\cc} = e^{-\frac{i}{2}\theta_{i}I_{i}}\varepsilon,
\end{align*}
which produces the correct transformation rule. Expanding each side around the identity we find
\begin{align*}
	\varepsilon I_{i}\cc = -I_{i}\varepsilon.
\end{align*}
We can choose the fundamental representation, which leaves $I_{1}$ and $I_{3}$ invariant. Choosing $\varepsilon\propto\sigma_{2}$ the corresponding relations work out. This would also make the final relation work out. One choice is $\varepsilon = i\sigma_{2}$. In terms of the up and down quark we denote the two states
\begin{align*}
	\phi = \mqty[
		u \\
		d
	],\ \hat{\phi} = \mqty[
		\hat{d} \\
		-\hat{u}
	].
\end{align*}
With this we can construct mesons as eigenstates of total isospin.

\paragraph{$CP$}
$CP$ is the combination of charge conjugation and parity. This combination of symmetries allows for the theory to contain more reactions. For instance, consider
\begin{align*}
	\pi^{+} \to \mu^{+} + \nu_{\mu}.
\end{align*}
This reaction works provided that the muon and the neutrino have opposite helicities. A parity transformation, however, will reverse all helicities, producing a left-handed muon, which is impossible. Similarly, charge conjugation of this reaction involves a right-handed antineutrino, which is also not allowed. The combined $CP$ symmetry, however, produces the reaction
\begin{align*}
	\pi^{-} \to \mu^{-} + \bar{\nu}_{\mu},
\end{align*}
where the antineutrino is left-handed, which is allowed. The hope of particle physicists was that the weak force respected $CP$.

\paragraph{$CP$ Violation}
%TODO: Look up Griffiths on kaons
$CP$ violation turns out to be one way to explain the domination of matter in the universe.

\paragraph{Time Reversal}
Time reversal - or rather, motion reversal - is a discrete symmetry corresponding to running a process backwards. It is respected by the strong and electromagnetic interactions. It is hard to test it for the weak interaction.

\paragraph{$CPT$}
$CPT$ is a combination of the discrete symmetries we have studied.

We can use it to draw conclusions about time reversal. Because quantum field theory is based on $CPT$ invariance and $CP$ is violated by the weak interaction, we expect $T$ to be violated as well in order for $CPT$ to be respected.

\paragraph{Conservation Laws for Reactions}

\paragraph{Young Tableaux}
Young tableaux are a useful tool for analyzing the structure of representations. I will present it in the context of \SU{3}, and therefore need to talk very briefly about that group.

\SU{3} is generated by eight matrices $\frac{1}{2}\lambda_{a}$, which are normalized such that $\tr(\lambda_{a}\lambda_{b}) = 2\delta_{ab}$. Two of these commute, hence the group has rank 2.

Corresponding to each representation of \SU{3} is a so-called conjugate representation, found by taking the complex conjugate of the first. This representation is denoted by a bar. The generators of this representation are $-\frac{1}{2}\lambda_{a}\cc$. This representation has not been brought up in the context of \SU{2} because it is equivalent to the ones we discussed. This equivalence occurs if there exists a matrix $\varepsilon$ which satisfies
\begin{align*}
	-\lambda_{a}\cc = \varepsilon\lambda_{a}\varepsilon^{-1}.
\end{align*}
An easy way to treat a conjugate representation with tensor notation is to treat the action of the original representation as acting on contravariant indices and the action of the conjugate representation as acting on covariant indices.

Note that to each antisymmetric rank-$N - 1$ tensor in this representation there exists a vector according to
\begin{align*}
	t_{j} = \varepsilon_{i_{1}\dots i_{N - 1}J}T^{i_{1}\dots i_{N - 1}},
\end{align*}
hence we associate such tensors with the conjugate representation.

The next idea would be to take the tensor product of such representations. Young tableaux are useful tools for treating just this. The basic idea is to represent each index with a box, according to \ytableaushort{{}}. As the fundamental representation acts on states with single indices, the single box thus represents the fundamental representations. Next, if you have multiple indices, symmetric indices are put in the same row and antisymmetric indices in the same column. The two look like
\begin{align*}
	\ydiagram{1, 1},\ \ydiagram{2}.
\end{align*}
Higher-rank diagrams are arranged such that the number of boxes in any row is equal to or less than the number of boxes in the rows above.

The power of the Young tableaux is in computing tensor products of representations. This is done with the following steps:
\begin{enumerate}
	\item Draw tableaux corresponding to each representation.
	\item Mark the boxes in the right tableau according to the row it is in.
	\item Take one box at a time from the right tableau and attach it to the left one, making sure to respect the rules of the tableaux and not putting two boxes from the same row in the same column.
	\item Discard columns of $N$ boxes.
	\item For each possible unique combination, compute the direct sum.
\end{enumerate}

How many elements can there be in a column for a general \SU{N} tableau? The answer is $N - 1$, as there is only a single rank $N$ antisymmetric tensor - the Levi-Civita tensor - which looks the same in all frames and thus transforms analogously to a scalar.

What is the dimensionality of the product representation corresponding to each tableau? To compute this we index each box in the tableau by a row number $j$ and a column number $k$. Next, for \SU{N} we compute the numbers
\begin{align*}
	A_{jk} = N + k - j,\ B_{jk} = n_{j} + m_{k}  + 1 - j - k
\end{align*}
for every box, where $n_{j}$ is the number of boxes in the row and $m_{k}$ is the number of boxes in the column. For a tableau, $B_{jk}$ can also be calculated by drawing an L with the corner in the box in question and the legs extending all the way down and to the right and counting the number of boxes in the L. By combinatorics it can somehow be shown that
\begin{align*}
	d = \prod\frac{A_{jk}}{B_{jk}}.
\end{align*}

Let us now do some examples. The first is a simple one, namely $3\otimes 3$. Using tableaux we have
\begin{align*}
	\ydiagram{1} \otimes \ydiagram{1} = \ydiagram{1, 1} \oplus \ydiagram{2}.
\end{align*}
The first tableau has dimension $3$ and the second part has dimension $6$. The first one corresponds to the conjugate representation, hence we find $3\otimes 3 = 6 \oplus 3\cc$. Next we study $3\cc\otimes 3$, which in tableau form is
\begin{align*}
	\ydiagram{1, 1} \otimes \ydiagram{1} = \ydiagram{2, 1} + \bullet,
\end{align*}
where the bullet signifies an empty tableau - in other words, a scalar. We thus find $3\cc\otimes 3 = 8\oplus 1$. Finally, let us do the more involved $3\otimes 3 \otimes 3$. Using our previous work we find
\begin{align*}
	\ydiagram{1} \otimes \ydiagram{1} \otimes \ydiagram{1} = \ydiagram{1} \otimes \left(\ydiagram{1, 1} \oplus \ydiagram{2}\right) = \ydiagram{2, 1} \oplus \bullet \oplus \ydiagram{3} \oplus \ydiagram{2, 1}.
\end{align*}
Note that the same tableau appears twice due to the direct sum. We thus find $3\otimes 3 \otimes 3 = 10 \oplus 8 \oplus 8 \oplus 1$.

\paragraph{The Quark Model}
The quark model is a simplified model which only considers the up, down and strange quarks, combining them into a triplet which transforms under \SU{3}. Taking the generators to be $\frac{\lambda_{a}}{2}$, the generators of transformations on the antiparticle states is $-\frac{\lambda_{a}\cc}{2}$. We denote the particle representation as $3$ and the antiparticle representation as $\bar{3}$.