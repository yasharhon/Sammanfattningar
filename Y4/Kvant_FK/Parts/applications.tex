\section{Applications}

\paragraph{The Harmonic Oscillator}
The Hamiltonian of the harmonic oscillator is
\begin{align*}
	H = \frac{1}{2m}p^{2} + \frac{1}{2}m\omega^{2}x^{2}.
\end{align*}
To diagonalize it we introduce the lowering operator
\begin{align*}
	a = \frac{1}{\sqrt{2}}\left(\sqrt{\frac{m\omega}{\hbar}}x + \frac{i}{\sqrt{m\omega\hbar}}p\right)
\end{align*}
and its adjoint, the raising operator. Their commutator is
\begin{align*}
	\comm{a}{\adj{a}} &= \comm{\frac{1}{\sqrt{2}}\left(\sqrt{\frac{m\omega}{\hbar}}x + \frac{i}{\sqrt{m\omega\hbar}}p\right)}{\frac{1}{\sqrt{2}}\left(\sqrt{\frac{m\omega}{\hbar}}x - \frac{i}{\sqrt{m\omega\hbar}}p\right)} \\
	                  &= \comm{\frac{1}{\sqrt{2}}\sqrt{\frac{m\omega}{\hbar}}x}{-\frac{1}{\sqrt{2}}\frac{i}{\sqrt{m\omega\hbar}}p} + \comm{\frac{1}{\sqrt{2}}\frac{i}{\sqrt{m\omega\hbar}}p}{\frac{1}{\sqrt{2}}\sqrt{\frac{m\omega}{\hbar}}x} \\
	                  &= \frac{1}{2}\frac{i}{\hbar}\left(\comm{x}{-p} + \comm{p}{x}\right) \\
	                  &= 1.
\end{align*}
The definition of the raising and lowering operators may be inverted to obtain
\begin{align*}
	x = \frac{1}{\sqrt{2}}\sqrt{\frac{\hbar}{m\omega}}(\adj{a} + a),\ p = \frac{i}{\sqrt{2}}\sqrt{m\omega\hbar}(\adj{a} - a).
\end{align*}
The Hamiltonian may now be written in terms of these operators as
\begin{align*}
	H &= \frac{1}{2m}\cdot -\frac{1}{2}m\omega\hbar(\adj{a} - a)^{2} + \frac{1}{2}m\omega^{2}\frac{1}{2}\frac{\hbar}{m\omega}(\adj{a} + a)^{2} \\
	  &= -\frac{1}{4}\hbar\omega(\adj{a} - a)^{2} + \frac{1}{4}\hbar\omega(\adj{a} + a)^{2} \\
	  &= \frac{1}{4}\hbar\omega\left((\adj{a})^{2} + \adj{a}a + a\adj{a} + a^{2} - \left((\adj{a})^{2} - \adj{a}a - a\adj{a} + a^{2}\right)\right) \\
	  &= \frac{1}{2}\hbar\omega\left(\adj{a}a + a\adj{a}\right) \\
	  &= \hbar\omega\left(\adj{a}a + \frac{1}{2}\right).
\end{align*}
We now define the operator $n = \adj{a}a$. It is Hermitian, meaning that an orthonormal basis of its eigenvectors exists (fortunately, as it constitutes the Hamiltonian). These eigenvectors must be studied next. To do this, we use the commutation relations\footnote{What might inspire this? A suggestion might be the fact that if $n$ and the raising and lowering operators commuted, we would find that they share eigenvectors.}
\begin{align*}
	\comm{n}{a} = \adj{a}\comm{a}{a} + \comm{\adj{a}}{a}a = -a,\ \comm{n}{\adj{a}} = \adj{a}\comm{a}{\adj{a}} + \comm{\adj{a}}{\adj{a}}a = \adj{a}
\end{align*}
applied to some eigenvector $\ket{\nu}$ with eigenvalue $\nu$ to obtain
\begin{align*}
	na\ket{\nu} = (an - a)\ket{\nu} = (\nu - 1)a\ket{\nu}.
\end{align*}
Hence, if some eigenvalue $\nu$ exists, we can repeat this argument to show that $\nu - 1,\ \nu - 2, \dots$ are also eigenvalues, assuming no value in this sequence is zero. The length of these eigenvectors is given by
\begin{align*}
	\expval{\adj{a}a}{\nu} = \nu\braket{\nu} \geq 0,
\end{align*}
where the latter is due to the positivity of the inner product. In order for this to work, no negative eigenvalues may exist. This only fits with the previous sequence of eigenvalues if $\nu = 0$ is an eigenvalue.

Having established that, we rename the eigenvalues to $n$. Next, we have
\begin{align*}
	n\adj{a}\ket{n} = (\adj{a}n + \adj{a})\ket{n} = (n + 1)\adj{a}\ket{n}.
\end{align*}
Hence the sequence $n + 1, n + 2, \dots$ also consists of eigenvalues of $n$. The length of such vectors is
\begin{align*}
	\expval{a\adj{a}}{n} = \expval{\adj{a}a + 1}{n} = (n + 1)\braket{n} > 0.
\end{align*}
Now the eigenvalues of the Hamiltonian are found to be
\begin{align*}
	H_{n} = \hbar\omega\left(n + \frac{1}{2}\right),\ H\ket{n} = H_{n}\ket{n}.
\end{align*}

With respect to degeneracy, suppose there is a set of eigenvectors denoted by the index $k$ such that $a\ket{0, k} = 0$. In the coordinate basis we obtain
\begin{align*}
	\mel**{x}{\frac{1}{\sqrt{2}}\left(\sqrt{\frac{m\omega}{\hbar}}x + \frac{i}{\sqrt{m\omega\hbar}}p\right)}{0, k} = \frac{1}{\sqrt{2}}\left(\sqrt{\frac{m\omega}{\hbar}}x + \sqrt{\frac{\hbar}{m\omega}}\dv{x}\right)\Psi_{0, k} = 0.
\end{align*}
The solution to this differential equation is unique, hence the ground state is non-degenerate. The linearity of the raising operator therefore implies that the other eigenvalues are non-degenerate as well.

With respect to normalization, we may require all states to be normalized. Then
\begin{align*}
	\adj{a}\ket{n}      &= c_{n + 1}\ket{n + 1}, \\
	\abs{c_{n + 1}}^{2} &= \expval{a\adj{a}}{n} = \expval{n + 1}{n} = n + 1, \\
	c_{n}               &= \sqrt{n}.
\end{align*}
Next we have
\begin{align*}
	a\adj{a}\ket{n - 1} &= \sqrt{n}a\ket{n} \\
	n\ket{n - 1}        &= \sqrt{n}a\ket{n}, \\
	a\ket{n}            &= \sqrt{n}\ket{n - 1}.
\end{align*}

Finally, the excited states may be found according to
\begin{align*}
	\ket{n} = \frac{1}{\sqrt{n}}\adj{a}\ket{n - 1} = \dots = \frac{1}{\sqrt{n!}}(\adj{a})^{n}\ket{0},
\end{align*}
which when applied to the ground state will reproduce some special function.

\paragraph{Quantum Hall Effect}
The Hamiltonian of a charged particle in a magnetic field is
\begin{align*}
	H = \frac{1}{2m}(\vb{p} - q\vb{A})^{2}.
\end{align*}
We recall that gauge transformations of the electromagnetic field are defined according to $A^{\mu} \to A^{\mu} - \del[\mu]{}{\chi}$ for some function $\chi$. While Maxwell's equations are gauge invariant, Schrödinger's equation is not. We try to remedy this by combining the gauge transformation with a transformation
\begin{align*}
	\ket{\Psi} \to e^{i\frac{q}{\hbar}\chi}\ket{\Psi}.
\end{align*}
The inverse transformation yields
\begin{align*}
	(\vb{p} - q\vb{A})\ket{\Psi} = (\vb{p} - q(\vb{A} - \grad{\chi}))e^{-i\frac{q}{\hbar}\chi}\ket{\Psi},
\end{align*}
which is given by
\begin{align*}
	(\vb{p} - q(\vb{A} - \grad{\chi}))e^{-i\frac{q}{\hbar}\chi}\ket{\Psi} &= e^{i\frac{q}{\hbar}\chi}(\vb{p} - q\grad{\chi} - q(\vb{A} - \grad{\chi}))\ket{\Psi} = e^{-i\frac{q}{\hbar}\chi}(\vb{p} - q\vb{A})\ket{\Psi},
\end{align*}
implying that the transformation does not change the Schrödinger equation.

We will start by studying two-dimensional motion in a rectangular domain with a constant magnetic field in the $z$-direction. In the Landau gauge we choose the vector potential $\vb{A} = Bx\ub{x}$. The Hamiltonian is thus
\begin{align*}
	H = \frac{1}{2m}\left(p_{x}^{2} + (p_{y} - qBx)^{2}\right).
\end{align*}
We see that the Hamiltonian commutes with $p_{y}$, but not with $p_{x}$, implying that the solution is of the form
\begin{align*}
	\Psi(\vb{x}) = \frac{1}{\sqrt{2\pi\hbar}}e^{-ik_{y}y}f(x),
\end{align*}
where $k_{y} = \frac{1}{\hbar}p_{y}$ may be taken to have a definite value. We are thus left with
\begin{align*}
	H = \frac{1}{2m}\left(p_{x}^{2} + \hbar^{2}\left(k_{y} - \frac{1}{l_{B}^{2}}x\right)^{2}\right),\ l_{B}^{2} = \frac{\hbar}{qB}.
\end{align*}
We solve this by introducing raising and lowering operators
\begin{align*}
	a_{k_{y}} = \frac{1}{\sqrt{2}}\left(\frac{x - k_{y}l_{B}^{2}}{l_{B}} + i\frac{l_{B}}{\hbar}p_{x}\right)
\end{align*}
such that
\begin{align*}
	H = \hbar\omega(\adj{a_{k_{y}}}a_{k_{y}} + \frac{1}{2}),
\end{align*}
yielding a harmonic oscillator with the classical cyclotron frequency
\begin{align*}
	\omega = \frac{\hbar}{ml_{B}^{2}} = \frac{eB}{m}.
\end{align*}
The energy levels of this harmonic oscillator are called Landau levels.

To study the degeneracy, we impose periodic boundary conditions in the $y$-direction, implying
\begin{align*}
	k_{y} = \frac{2\pi}{L_{y}}m.
\end{align*}
If you have a large but finite sample of length $L_{x}$ in the $x$-direction, the fact that the state is localized around $k_{y}l_{B}^{2}$ implies that the maximum value of $m$ is
\begin{align*}
	N = \frac{L_{x}}{l_{B}^{2}\frac{2\pi}{L_{y}}} = \frac{qBA}{h}.
\end{align*}
In particular, for $q = e$ we have
\begin{align*}
	H = \frac{\Phi}{\Phi_{0}},
\end{align*}
where $\Phi_{0} = \frac{h}{e}$ is the flux quantum.

To study samples at the edge of the sample, add a potential to represent the edge and assume that it varies slowly when compared to the length scale $l_{B}$. In this case the eigenvalues are modified to
\begin{align*}
	E_{n} = \hbar\omega(n + \frac{1}{2}) + V(x = = -k_{y}l_{B}^{2})
\end{align*}
The edge states (perhaps) carry the current, meaning that such systems that we have studied will display steps in their Hall coefficient. This is the quantum Hall effect.

\paragraph{Aharanov-Bohm Effect}
To demonstrate the principle, consider a metal ring connected to two terminals with magnetic flux through the middle and suppose that current flows from one terminal to the other. The vector potential is non-zero in the ring, and may in fact be written as $\vb{A} = \grad{f}$ here. Performing a gauge transformation preserves the Schrödinger equation, as before. Now, as we may write
\begin{align*}
	f = \integ{\gamma}{}{\vb{x}}{\cdot\vb{A}},
\end{align*}
this implies that the phase of the state is determined by the path taken. Combining this with our knowledge of path integrals yields the transmittion probability
\begin{align*}
	T = \abs{t_{\text{upper}}}^{2} + \abs{t_{\text{lower}}}^{2} + 2\Re\left(t_{\text{upper}}\cc{t_{\text{upper}}}e^{iq\left(\integ{\gamma_{\text{lower}}}{}{\vb{x}}{\cdot\vb{A}} - \integ{\gamma_{\text{upper}}}{}{\vb{x}}{\cdot\vb{A}}\right)}\right) = \abs{t_{\text{upper}}}^{2} + \abs{t_{\text{lower}}}^{2} + 2\Re\left(t_{\text{upper}}\cc{t_{\text{upper}}}e^{-iq\Phi}\right).
\end{align*}
This oscillating transmission probability is the Aharanov-Bohm effect.

\paragraph{Entanglement}
Suppose you prepare a system of two particles with two possible states in some state $\frac{1}{\sqrt{2}}(\ket{ab} - \ket{ba})$ and introduce some operator $O_{A} = O\otimes 1$ in the frame of some observer $A$. The projection onto the $a$ eigenstates is given by
\begin{align*}
	P_{a} = \sum\limits_{\alpha}\abs{\braket{a\alpha}{\Psi}}^{2} = \frac{1}{2} = P_{b}.
\end{align*}
The same turns out to be true for any non-degenerate operator. Hence making any measurement only in frame $A$ does not reveal everything about the state.

To work around this, we introduce the reduced density matrix. Suppose the state is of the form
\begin{align*}
	\ket{\Psi} = \sum\limits_{\sigma_{A}, \sigma_{B}}\Psi_{\sigma_{A}\sigma_{B}}\ket{\sigma_{A}\sigma_{B}}.
\end{align*}
The expectation value of an operator of the form $O_{A}$ is given by
\begin{align*}
	\expval{O_{A}} &= \sum\limits_{\sigma_{A}, \sigma_{B}, \sigma_{A}^{\prime}, \sigma_{B}^{\prime}}\cc{\Psi}_{\sigma_{A}\sigma_{B}}\Psi_{\sigma_{A}^{\prime}\sigma_{B}^{\prime}}\mel{\sigma_{A}}{O}{\sigma_{A}^{\prime}}\braket{\sigma_{B}}{\sigma_{B}^{\prime}} \\
	               &= \sum\limits_{\sigma_{A}, \sigma_{B}, \sigma_{A}^{\prime}}\cc{\Psi}_{\sigma_{A}\sigma_{B}}\Psi_{\sigma_{A}^{\prime}\sigma_{B}}\mel{\sigma_{A}}{O}{\sigma_{A}^{\prime}} \\
	               &= \sum\limits_{\sigma_{A}^{\prime}}\mel{\sigma_{A}^{\prime}}{\sum\limits_{\sigma_{A},\sigma_{B}, \sigma_{A}^{\prime\prime}}\cc{\Psi}_{\sigma_{A}\sigma_{B}}\Psi_{\sigma_{A}^{\prime\prime}\sigma_{B}}}{\sigma_{A}^{\prime\prime}}\mel{\sigma_{A}}{O}{\sigma_{A}^{\prime}}.
\end{align*}
This defines the reduced density matrix
\begin{align*}
	\rho_{A} = \tr(\rho)_{B},
\end{align*}
allowing us to write
\begin{align*}
	\expval{O_{A}} &= \tr(\rho_{A}A)_{A}.
\end{align*}
For the given state we have
\begin{align*}
	\mel{a}{\rho_{A}}{a} &= \sum\limits_{d}\cc{\Psi}_{ad}\Psi_{ad} = \frac{1}{2}, \\
	\mel{a}{\rho_{A}}{b} &= \sum\limits_{d}\cc{\Psi}_{ad}\Psi_{bd} = 0, \\
	\mel{b}{\rho_{A}}{b} &= \sum\limits_{d}\cc{\Psi}_{bd}\Psi_{bd} = \frac{1}{2}.
\end{align*}
Note that such density matrices do not necessarily correspond to pure states.