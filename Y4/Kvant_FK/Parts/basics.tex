\section{Basic Concepts}

\paragraph{Observables}
An observable is a Hermitian operator whose orthonormal eigenvectors form a basis.

\paragraph{The Postulates of Quantum Mechanics}
The postulates of quantum mechanics are:
\begin{itemize}
	\item At any fixed time the state of a physical system is specified by a ket in Hilbert space.
	\item Every measurable physical quantity corresponds to an operator on Hilbert space. This is a Hermitian observable. The possible outcomes of a measurement are the eigenvalues of $A$.
	\item The probability of measuring the value $a$ of operator $A$ in a normalized state $\ket{\Psi}$ is $P(a) = \expval{P_{a}}{\Psi}$, where $P_{a}$ is the projector onto the subspace corresponding to the eigenvalue $a$ given by $P_{a} = \dyad{a}{a}$.
	\item If a measurement of an observable $A$ gives an outcome $a$, the state of the system immediately after the measurement is the projection of the state onto the subspace with eigenvalue $a$.
	\item The time evolution of a state is governed by the Schrödinger equation.
	\item Physical states in a many-body system are either completely symmetric or completely anti-symmetric with respect to particle exchange (to be discussed later).
\end{itemize}

\paragraph{Consequences of the Probability Picture}
The form of writing the projection operator implies $P(a) = \abs{\bra{a}\ket{\Psi}}^{2}$, or $P(a)\dd{a} = \abs{\bra{a}\ket{\Psi}}^{2}\dd{a}$ in the continuous case. In order for the probability interpretation to be consistent, i.e. for the sum of all probabilities to amount to $1$, it must hold that $\bra{\Psi}\ket{\Psi} = 1$.

\paragraph{Expectation Values}
Expectation values are given by
\begin{align*}
	\expval{A} = \sum aP(a) = \sum a\expval{P_{a}}{\Psi} = \bra{\Psi}\sum a\ket{a}\bra{a}\ket{\Psi} = \expval{A}{\Psi}.
\end{align*}

\paragraph{Physical States}
Modifying a state by a phase factor $e^{i\alpha}$ does not change any expectation values.

\paragraph{Pure and Mixed States}
Pure states are states with a well-defined state vector. Mixed states are states wherein the state vector is not well-defined.

\paragraph{Density Matrices}
The density matrix is defined as
\begin{align*}
	\rho = \op{\Psi}{\Psi}.
\end{align*}
It has some cool properties. For instance:
\begin{align*}
	&\tr{\rho} = \sum\limits_{n}\expval{\rho}{n} = \ev**{\sum\limits_{n}\op{n}{n}}{\psi} = \braket{\Psi} = 1, \\
	&\adj{\rho} = \rho, \\
	&\expval{A} = \sum\limits_{n, m}\braket{\Psi}{n}\mel{n}{A}{m}\braket{m}{\Psi} = \sum\limits_{n, m}\braket{m}{\Psi}\braket{\Psi}{n}\mel{n}{A}{m} = \sum\limits_{n, m}\mel{m}{\rho}{n}\mel{n}{A}{m} = \tr(\rho A), \\
	&\rho^{2} = \rho.
\end{align*}
%TODO: Introduce mixed states
Note that the latter is only true for pure states. Mixed states have a density matrix of the form
\begin{align*}
	\rho = \sum\limits_{j}P_{j}\op{\Psi_{j}}{\Psi_{j}},
\end{align*}
where the $P_{j}$ are the probability that the state of the system is $\ket{\Psi_{j}}$.

\paragraph{The Time Evolution Operator}
Consider the operator $u_{t'}(t)$ which evolves $\ket{\Psi(t')}$ to $\ket{\Psi(t)}$. Inserting it into the Schrödinger equation yields
\begin{align*}
	i\hbar\dv{t}u_{t^{\prime}}(t)\ket{\Psi(t^{\prime})} = Hu_{t^{\prime}}(t)\ket{\Psi(t^{\prime})}, \\
	i\hbar\del{t}{u_{t^{\prime}}} = Hu_{t^{\prime}}(t).
\end{align*}
It follows from In the case of a time-independent Hamiltonian, the solution must be of the form $u_{t^{\prime}}(t) = u(t - t^{\prime})$, and the equation above can be integrated to yield
\begin{align*}
	u_{t^{\prime}}(t) = e^{-i\frac{t - t^{\prime}}{\hbar}H}.
\end{align*}
Hence $H$ generates time translation - at least in time-independent cases.

\paragraph{Space and Space Translation}
We introduce the notion of space as a set of operators $x_{i}$ on the basis states. These operators are postulated to commute, as are their corresponding translations. The latter implies that their generators $k_{i}$ commute as well.

\paragraph{The Momentum Operator}
It turns out that the generators of space translations have a physical interpretation. TO understand this, we note that the generating function of a spatial translation in classical mechanics is
\begin{align*}
	F(\vb{x}, \vb{P}) = \vb{x}\cdot\vb{P} + \vb{p}\cdot\vb{x},
\end{align*}
which contains one term generating the identity and one causing the translation. We are therefore prompted to guess that $k_{i} \propto p_{i}$. When studying de Broglie waves, one finds that the constant of proportionality is $\hbar$. We thus arrive at the final analogue to the canonical commutation relations, namely
\begin{align*}
	\comm{x_{i}}{p_{j}} = i\hbar\delta_{ij}.
\end{align*}
One could of course have started with these as postulates instead, prompting a stronger analogy to classical mechanics. But the symmetry approach is nice too.

\paragraph{Spatial Inversion}
Let $\Pi$ be the spatial inversion operator, or parity operator, such that $\Pi\vb{x}\Pi^{-1} = -\vb{x}$. We then have
\begin{align*}
	\vb{x}\Pi\ket{\vb{x}^{\prime}} = -\Pi\vb{x}\ket{\vb{x}^{\prime}} = -\vb{x}^{\prime}\Pi\ket{\vb{x}^{\prime}},
\end{align*}
hence $\Pi\ket{\vb{x}^{\prime}} = \ket{-\vb{x}^{\prime}}$, as the phase provided by spatial inversion may be chosen freely. Next we have
\begin{align*}
	\Pi\Psi = \mel{\vb{x}}{\Pi}{\Psi} = \braket{-\vb{x}}{\Psi} = \Psi(-\vb{x}).
\end{align*}
Furthermore, as we chose the operator to provide no phase, we have $\Pi^{2} = 1$, hence $\Pi$ has eigenvalues $\pm 1$ and $\Pi^{-1} = \adj{\Pi} = \Pi$.

How does spatial inversion affect other quantities? The geometry of space implores us to require
\begin{align*}
	\Pi e^{-i\frac{1}{\hbar}\var{\vb{x}}\cdot\vb{p}} = e^{i\frac{1}{\hbar}\var{\vb{x}}\cdot\vb{p}}\Pi,
\end{align*}
namely a spatial translation followed by inversion is equal to inversion followed by an opposite translation. Expanding yields
\begin{align*}
	\Pi\left(1 -i\frac{1}{\hbar}\var{\vb{x}}\cdot\vb{p}\right)\adj{\Pi} = 1 + i\frac{1}{\hbar}\var{\vb{x}}\cdot\vb{p},
\end{align*}
and we may therefore identify $\Pi\vb{p}\adj{\Pi} = -\vb{p}$. Next, we require that rotations and inversions commute (that this must be is clear as the matrix representation of a rotation is diagonal). Hence we obtain
\begin{align*}
	\Pi\vb{J}\adj{\Pi} = \vb{J}
\end{align*}
by a similar procedure. Notably, this is true for spin as well - a result which we could only arrive at by enforcing these requirements.

\paragraph{Inversion Symmetry and Selection Rules}
Most observables are either even or odd under inversions, namely $\Pi A\Pi^{-1} = \pi_{A}A,\ \pi_{A} = \pm 1$. If we are working with states with definite parity, we have
\begin{align*}
	\mel{\Psi}{A}{\Phi} &= \pi_{A}\mel{\Psi}{\Pi A\Pi}{\Phi} \\
	                    &= \pi_{A}\pi_{\Psi}\pi_{\Phi}\mel{\Psi}{A}{\Phi},
\end{align*}
meaning that this matrix element is zero if $\pi_{a}\pi_{\Psi}\pi_{\Phi} = -1$. This is an example of a so-called selection rule.

\paragraph{Mirror Symmetries}
Other discrete spatial symmetries exist. One example is mirror symmetry. This particular example may be written as a composition of an inversion and a rotation.

\paragraph{Time Reversal}
Time reversal should preserve position and flip the sign of momentum. This implies, however, that
\begin{align*}
	T(i\hbar)T^{-1} = -i\hbar,
\end{align*}
which can only be satisfied if $T$ is anti-linear and anti-unitary. It can be shown that for spinless particles, the time reversal operator is simply the complex conjugation operator.

\paragraph{Kramer's Degeneracy}
Consider a system which is invariant under time reversal. It would seem that for any eigenvector $\ket{E}$, there must also exist an eigenvector $T\ket{E}$. However, in cases with spin-$\frac{1}{2}$, for which $T^{2} = -1$, we have
\begin{align*}
	\mel{E}{T}{E} &= \cc{(\bra{TE})(\ket{T^{2}E})} = -\cc{(\bra{TE})(\ket{E})} = -\mel{E}{T}{E},
\end{align*}
implying the two vectors are orthogonal. This is called Kramer's degeneracy.

\paragraph{Time Evolution of the Density Matrix}
The time evolution of the density matrix is given by
\begin{align*}
	\rho(t) = \sum P_{i}u_{t_{0}}(t)\ket{\Psi_{i}}\bra{\Psi_{i}}\adj{u_{t_{0}}(t)} = u_{t_{0}}(t)\rho(t_{0})\adj{u_{t_{0}}(t)}.
\end{align*}
This implies
\begin{align*}
	i\hbar\dv{t}\rho = Hu_{t_{0}}(t)\rho(t_{0})\adj{u_{t_{0}}(t)} - u_{t_{0}}(t)\rho(t_{0})\adj{u_{t_{0}}(t)}H = H\rho(t) - \rho(t)H = \comm{H}{\rho}.
\end{align*}

\paragraph{The Heisenberg Equation}
Heisenberg's outlook starts from preserving expectation values under time translations in such a way that all (total) time evolution is contained in the operators, arriving at the transformation rule
\begin{align*}
	A_{\text{H}} = \adj{u_{t_{0}}}(t)A_{\text{S}}u_{t_{0}}(t).
\end{align*}
$A_{\text{H}}$ is the operator according to Heisenberg and $A_{\text{S}}$ is the operator according to Schrödinger. We now have
\begin{align*}
	i\hbar\dv{t}\expval{A_{\text{H}}} &= -\adj{u_{t_{0}}}(t)HA_{\text{S}}u_{t_{0}}(t) + \adj{u_{t_{0}}}(t)(i\hbar\del{t}{A_{\text{S}}})u_{t_{0}}(t) + \adj{u_{t_{0}}}(t)A_{\text{S}}Hu_{t_{0}}(t) \\
	                                  &= -\adj{u_{t_{0}}}(t)Hu_{t_{0}}(t)\adj{u_{t_{0}}}(t)A_{\text{S}}u_{t_{0}}(t) + \adj{u_{t_{0}}}(t)(i\hbar\del{t}{A_{\text{S}}})u_{t_{0}}(t) + \adj{u_{t_{0}}}(t)A_{\text{S}}u_{t_{0}}(t)\adj{u_{t_{0}}}(t)Hu_{t_{0}}(t) \\
	                                  &= -H_{\text{H}}A_{\text{H}} + \adj{u_{t_{0}}}(t)(i\hbar\del{t}{A_{\text{S}}})u_{t_{0}}(t) + A_{\text{H}}H_{\text{H}} \\
	                                  &= -{H_{\text{H}}}\comm{A_{\text{H}}} + (i\hbar\del{t}{A_{\text{S}}})_{\text{H}}.
\end{align*}

\paragraph{Symmetries and Conserved Quantities}
Suppose that there exists some unitary transformation $u = e^{-i\frac{\varepsilon}{\hbar}A}$ such that $\adj{u}Hu = H$. Expanding the symmetry yields
\begin{align*}
	\left(1 + i\frac{\varepsilon}{\hbar}A + \dots\right)H\left(1 - i\frac{\varepsilon}{\hbar}A + \dots\right) = H + i\frac{\varepsilon}{\hbar}(-HA + AH) + \dots = H + i\frac{\varepsilon}{\hbar}\comm{A}{H} + \dots = H,
\end{align*}
implying that $A$ and $H$ commute. Assuming $A$ to have no explicit time dependence, Heisenberg's equations yield that $\expval{A}$ is conserved. This is essentially a kind of Nöether's theorem in quantum mechanics.

\paragraph{Propagators}
The probability amplitude at some point $x$ at time $t$ is given by
\begin{align*}
	\Psi(x, t) = \braket{x}{\Psi(t)} = \mel{x}{u_{0}(t)}{\Psi(0)} = \integ{}{}{x^{\prime}}{\mel{x}{u_{0}(t)}{x^{\prime}}\braket{x^{\prime}}{\Psi(0)}}.
\end{align*}
Defining the propagator $G_{x^{\prime}, t^{\prime}}(x, t) = \mel{x}{u_{t^{\prime}}(t)}{x^{\prime}}$, we arrive at
\begin{align*}
	\Psi(x, t) = \integ{}{}{x^{\prime}}{G_{x^{\prime}, 0}(x, t)\braket{x^{\prime}}{\Psi(0)}} = \integ{}{}{x^{\prime}}{G_{x^{\prime}, 0}(x, t)\Psi(x^{\prime}, 0)}.
\end{align*}
Hence the propagator acts as a Green's function with respect to time, in some sense.

\paragraph{Arriving at Path Integrals}
The general propagator of some state is given by
\begin{align*}
	G_{x^{\prime}, t^{\prime}}(x, t) = \sum\limits_{\gamma}G_{\gamma; x^{\prime}, t^{\prime}}(x, t),
\end{align*}
where the summation is performed over all possible paths $\gamma$ between the two points.

Suppose now that the time evolution is divided into steps such that
\begin{align*}
	u_{t^{\prime}}(t) = \prod\limits_{k = 1}^{n}u_{t_{k - 1}}(t_{k}),\ t_{0} = t^{\prime},\ t_{n} = t,\ t_{k} - t_{k - 1} = \delta t.
\end{align*}
Then
\begin{align*}
	G_{x^{\prime}, t^{\prime}}(x, t) = \mel**{x}{\prod\limits_{k = 1}^{n}u_{t_{k - 1}}(t_{k})}{x^{\prime}}.
\end{align*}
For every $k$ we now introduce an identity according to
\begin{align*}
	G_{x^{\prime}, t^{\prime}}(x, t) &= \mel**{x}{\prod\limits_{k = 1}^{n}u_{t_{k - 1}}(t_{k})\integ{}{}{x_{k}}{\op{x_{k - 1}}{x_{k - 1}}}}{x^{\prime}} \\
	                                 &= \mel**{x}{\prod\limits_{k = 1}^{n}\integ{}{}{x_{k}}{u_{t_{k - 1}}(t_{k})\op{x_{k - 1}}{x_{k - 1}}}}{x^{\prime}} \\
	                                 &= \int\prod\limits_{k = 1}^{n}\dd{x_{k}}\mel{x_{k}}{u_{t_{k - 1}}(t_{k})}{x_{k - 1}}.
\end{align*}
The time translation operator has the form $u_{t_{k - 1}}(t_{k}) = e^{-i\frac{\Delta t}{\hbar}H}$. For a Hamiltonian of the form $H = \frac{p^{2}}{2m} + V(\vb{x})$, the terms do not necessarily commute. However, to second order we have
\begin{align*}
	e^{\alpha A}e^{\alpha B} &= \left(1 + \alpha A + \frac{1}{2}\alpha^{2}A^{2} + \dots\right)\left(1 + \alpha B + \frac{1}{2}\alpha^{2}B^{2} + \dots\right), \\
	e^{\alpha(A + B)}  &= 1 + \alpha A + \alpha B + \frac{1}{2}\alpha^{2}(A^{2} + B^{2} + AB + BA) + \dots, \\
	           &= e^{A}e^{B}\left(1 - \frac{1}{2}\alpha^{2}AB + \frac{1}{2}\alpha^{2}BA + \dots\right) \\
	           &= e^{\alpha A}e^{\alpha B}e^{\frac{1}{2}\alpha^{2}\comm{A}{B}}.
\end{align*}
Ignoring the second-order term yields
\begin{align*}
	G_{x^{\prime}, t^{\prime}}(x, t) &= \int\prod\limits_{k = 1}^{n}\dd{x_{k}}\mel{x_{k}}{e^{-i\frac{\Delta t}{\hbar}(T + V)}}{x_{k - 1}} \\
	                                 &= \int\prod\limits_{k = 1}^{n}\dd{x_{k}}e^{-i\frac{\Delta t}{\hbar}V(x_{k - 1})}\mel{x_{k}}{e^{-i\frac{\Delta t}{\hbar}T}}{x_{k - 1}} \\
	                                 &= \int\prod\limits_{k = 1}^{n}\dd{x_{k}}e^{-i\frac{\Delta t}{\hbar}V(x_{k - 1})}\mel**{x_{k}}{e^{-i\frac{\Delta t}{\hbar}T}\integ{}{}{p_{k}}{\op{p_{k}}{p_{k}}}}{x_{k - 1}} \\
	                                 &= \int\prod\limits_{k = 1}^{n}\dd{x_{k}}e^{-i\frac{\Delta t}{\hbar}V(x_{k - 1})}\mel**{x_{k}}{\integ{}{}{p_{k}}{e^{-i\frac{\Delta t}{\hbar}T}\op{p_{k}}{p_{k}}}}{x_{k - 1}} \\
	                                 &= \int\prod\limits_{k = 1}^{n}\dd{x_{k}}e^{-i\frac{\Delta t}{\hbar}V(x_{k - 1})}\integ{}{}{p_{k}}{e^{-i\frac{\Delta t}{2m\hbar}p_{k}^{2}}\braket{x_{k}}{p_{k}}\braket{p_{k}}}{x_{k - 1}} \\
	                                 &= \int\prod\limits_{k = 1}^{n}\dd{x_{k}}e^{-i\frac{\Delta t}{\hbar}V(x_{k - 1})}\integ{}{}{p_{k}}{e^{-i\frac{\Delta t}{2m\hbar}p_{k}^{2}}\frac{1}{2\pi\hbar}e^{i\frac{p_{k}(x_{k} - x_{k - 1})}{\hbar}}} \\
	                                 &= \int\prod\limits_{k = 1}^{n}\dd{x_{k}}e^{-i\frac{\Delta t}{\hbar}V(x_{k - 1})}e^{i\frac{m}{2\hbar\Delta t}(x_{k} - x_{k - 1})^{2}}\frac{1}{2\pi\hbar}\integ{}{}{p_{k}}{e^{-i\frac{\Delta t}{2m\hbar}\left(p_{k} - \frac{m}{\Delta t}(x_{k} - x_{k - 1})\right)^{2}}} \\
	                                 &= \int\prod\limits_{k = 1}^{n}\dd{x_{k}}e^{-i\frac{\Delta t}{\hbar}V(x_{k - 1})}e^{i\frac{m}{2\hbar\Delta t}(x_{k} - x_{k - 1})^{2}}\sqrt{\frac{m}{2\pi^{2}\hbar\Delta ti}}\integ{}{}{v_{k}}{e^{-v_{k}^{2}}} \\
	                                 &= \int\prod\limits_{k = 1}^{n}\dd{x_{k}}e^{-i\frac{\Delta t}{\hbar}V(x_{k - 1})}e^{i\frac{m}{2\hbar\Delta t}(x_{k} - x_{k - 1})^{2}}\sqrt{\frac{m}{2\pi\hbar\Delta ti}} \\
	                                 &= \int\prod\limits_{k = 1}^{n}\dd{x_{k}}\sqrt{\frac{m}{2\pi\hbar\Delta ti}}e^{i\frac{1}{\hbar}\sum\limits_{k = 1}^{n}\left(\frac{1}{2}m\left(\frac{x_{k} - x_{k - 1}}{\Delta t}\right)^{2} - V(x_{k - 1})\right)\Delta t}.
\end{align*}
In the continuous limit the exponent becomes
\begin{align*}
	i\frac{1}{\hbar}\integ{}{}{t}{\frac{1}{2}m\dot{x}^{2} - V(x)} = i\frac{S}{\hbar}
\end{align*}
where $S$ is the action. The remaining factor, termed the measure, is
\begin{align*}
	D(x(t)) = \lim\limits_{\Delta t \to 0}\prod\limits_{k = 1}^{n}\dd{x_{k}}\sqrt{\frac{m}{2\pi\hbar\Delta ti}}.
\end{align*}
Finally the propagator is given by
\begin{align*}
	G_{x^{\prime}, t^{\prime}}(x, t) &= \int D(x(t))e^{-i\frac{S}{\hbar}}.
\end{align*}
This is termed the path integral.

As a side note, if the action is large compared to $\hbar$, the action varies strongly, causing destructive interference from all paths except for the one such that
\begin{align*}
	\fdv{S}{x} = 0.
\end{align*}
This is Hamilton's principle, the fundamental postulate of classical mechanics.

\paragraph{The Interaction Picture}
