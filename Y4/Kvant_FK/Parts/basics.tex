\section{Basic Concepts}

\paragraph{Observables}
An observable is a Hermitian operator whose orthonormal eigenvectors form a basis.

\paragraph{The Postulates of Quantum Mechanics}
The postulates of quantum mechanics are:
\begin{itemize}
	\item At any fixed time the state of a physical system is specified by a ket in Hilbert space.
	\item Every measurable physical quantity corresponds to an operator on Hilbert space. This is a Hermitian observable. The possible outcomes of a measurement are the eigenvalues of $A$.
	\item The probability of measuring the value $a$ of operator $A$ in a normalized state $\ket{\Psi}$ is $P(a) = \expval{P_{a}}{\Psi}$, where $P_{a}$ is the projector onto the subspace corresponding to the eigenvalue $a$ given by $P_{a} = \dyad{a}{a}$.
	\item If a measurement of an observable $A$ gives an outcome $a$, the state of the system immediately after the measurement is the projection of the state onto the subspace with eigenvalue $a$.
	\item The time evolution of a state is governed by the Schrödinger equation.
\end{itemize}

\paragraph{Consequences of the Probability Picture}
The form of writing the projection operator implies $P(a) = \abs{\bra{a}\ket{\Psi}}^{2}$, or $P(a)\dd{a} = \abs{\bra{a}\ket{\Psi}}^{2}\dd{a}$ in the continuous case. In order for the probability interpretation to be consistent, i.e. for the sum of all probabilities to amount to $1$, it must hold that $\bra{\Psi}\ket{\Psi} = 1$.

\paragraph{Expectation Values}
Expectation values are given by
\begin{align*}
	\expval{A} = \sum aP(a) = \sum a\expval{P_{a}}{\Psi} = \bra{\Psi}\sum a\ket{a}\bra{a}\ket{\Psi} = \expval{A}{\Psi}.
\end{align*}

\paragraph{Physical States}
Modifying a state by a phase factor $e^{i\alpha}$ does not change any expectation values.

\paragraph{Mixed States}

\paragraph{Density Matrix}
The density matrix is defined as
\begin{align*}
	\rho = \op{\Psi}{\Psi}.
\end{align*}
It has some cool properties. For instance:
\begin{align*}
	&\tr{\rho} = \sum\limits_{n}\expval{\rho}{n} = \ev**{\sum\limits_{n}\op{n}{n}}{\psi} = \braket{\Psi} = 1, \\
	&\adj{\rho} = \rho, \\
	&\expval{A} = \sum\limits_{n, m}\braket{\Psi}{n}\mel{n}{A}{m}\braket{m}{\Psi} = \sum\limits_{n, m}\braket{m}{\Psi}\braket{\Psi}{n}\mel{n}{A}{m} = \sum\limits_{n, m}\mel{m}{\rho}{n}\mel{n}{A}{m} = \tr(\rho A), \\
	&\rho^{2} = \rho.
\end{align*}
Note that the latter is only true for pure states. Mixed states have a density matrix of the form
\begin{align*}
	\rho = \sum\limits_{j}P_{j}\op{\Psi_{j}}{\Psi_{j}}.
\end{align*}

\paragraph{The Time Evolution Operator}
Suppose that there exists an operator $u_{t'}(t)$ which evolves $\ket{\Psi(t')}$ to $\ket{\Psi(t)}$. Such an operator should satisfy
\begin{itemize}
	\item $u_{t'}(t) = u_{t''}(t)u_{t'}(t'')$ for consistency.
	\item $u_{t'}(t)$ is unitary to preserve the normalization.
	\item $u_{t}(t) = 1$.
\end{itemize}
Inserting this into the Schrödinger equation yields
\begin{align*}
	i\hbar\dv{t}u_{t^{\prime}}(t)\ket{\Psi(t^{\prime})} = Hu_{t^{\prime}}(t)\ket{\Psi(t^{\prime})}, \\
	i\hbar\del{t}{t^{\prime}} = Hu_{t^{\prime}}(t).
\end{align*}
In the case of a time-independent Hamiltonian, the solution must be of the form $u_{t^{\prime}}(t) = u(t - t^{\prime})$, and the equation above can be integrated to yield
\begin{align*}
	u_{t^{\prime}}(t) = e^{-i\frac{t - t^{\prime}}{\hbar}H}.
\end{align*}

\paragraph{Symmetries in Quantum Mechanics}
A symmetry in a quantum mechanics context is any transformation acting on Hilbert space that leaves all probabilities invariant.

\paragraph{Wigner's Theorem}
Wigner's theorem states that any operator that is a symmetry is either unitary or anti-unitary (the latter adds a complex conjucation when acting on a state multiplied by a number).

\paragraph{Transformation of Operators}
Consider a symmetry operator $u$. In order for this to be a symmetry, it must also act on all operators according to $A\to uA\adj{u}$.

\paragraph{Time Evolution From Symmetry}
Consider some system with time translation symmetry - that is, any system for which time translations do not change the theory. Introduce the transformation operator
\begin{align*}
	u_{\tau}\ket{\Psi(t)} = \ket{\Psi(t + \tau)}.
\end{align*}
This transformation is a smooth map acting on a manifold - namely, Hilbert space. Hence we can use the language of Lie algebra to treat this (if you know nothing about Lie algebra, pretend that I didn't write this and carry on. If you want some reference material, please look at my summary of SI2360). We expand the transformation operator around the identity as
\begin{align*}
	u_{\tau} = 1 - i\frac{\tau}{\hbar}H
\end{align*}
for some operator $H$. The requirement that this be unitary yields $\adj{H} - H = 0$, and hence the generator $H$ is self-adjoint. By continuous application of this we obtain
\begin{align*}
	u_{\tau} = e^{-i\frac{\tau}{\hbar}H}.
\end{align*}
This reproduces the Schrödinger equation, tying it all together neatly. It also demonstrates that the Hamiltonian generates time translation in a mathematical sense.

\paragraph{Space Translation}
Consider the space operator $x^{i}$. A space translation $u$ transforms $x^{i}$ to $x^{i} + a^{i}$, meaning $ux^{i}\adj{u} = x^{i} + a^{i}$. Expanding the translation around the identity yields
\begin{align*}
	u = 1 + i\frac{a^{i}}{\hbar}p_{i}
\end{align*}
for some operator $p_{i}$. The requirement that $u$ be unitary implies that $p$ is self-adjoint. The transformation rule yields
\begin{align*}
	(1 + i\frac{a^{i}}{\hbar}p_{i})x^{i}(1 - i\frac{a^{i}}{\hbar}p_{i}) = x^{i} + i\frac{a^{i}}{\hbar}\pb{p_{i}}{x^{i}}
\end{align*}
and the requirement
\begin{align*}
	\comm{p_{i}}{x^{i}} = -i\hbar.
\end{align*}

\paragraph{Time Evolution of the Density Matrix}
The time evolution of the density matrix is given by
\begin{align*}
	\rho(t) = \sum P_{i}u_{t_{0}}(t)\ket{\Psi_{i}}\bra{\Psi_{i}}\adj{u_{t_{0}}(t)} = u_{t_{0}}(t)\rho(t_{0})\adj{u_{t_{0}}(t)}.
\end{align*}
This implies
\begin{align*}
	i\hbar\dv{t}\rho = Hu_{t_{0}}(t)\rho(t_{0})\adj{u_{t_{0}}(t)} - u_{t_{0}}(t)\rho(t_{0})\adj{u_{t_{0}}(t)}H = H\rho(t) - \rho(t)H = \comm{H}{\rho}.
\end{align*}

\paragraph{The Heisenberg Equation}
Heisenberg's outlook starts from preserving expectation values under time translations, arriving at the transformation rule
\begin{align*}
	A_{\text{H}} = \adj{u_{t_{0}}}(t)A_{\text{S}}u_{t_{0}}(t).
\end{align*}
$A_{\text{H}}$ is the operator according to Heisenberg and $A_{\text{S}}$ is the operator according to Schrödinger. We now have
\begin{align*}
	i\hbar\dv{t}\expval{A_{\text{H}}} &= -\adj{u_{t_{0}}}(t)HA_{\text{S}}u_{t_{0}}(t) + \adj{u_{t_{0}}}(t)(i\hbar\del{t}{A_{\text{S}}})u_{t_{0}}(t) + \adj{u_{t_{0}}}(t)A_{\text{S}}Hu_{t_{0}}(t) \\
	                                  &= -\adj{u_{t_{0}}}(t)Hu_{t_{0}}(t)\adj{u_{t_{0}}}(t)A_{\text{S}}u_{t_{0}}(t) + \adj{u_{t_{0}}}(t)(i\hbar\del{t}{A_{\text{S}}})u_{t_{0}}(t) + \adj{u_{t_{0}}}(t)A_{\text{S}}u_{t_{0}}(t)\adj{u_{t_{0}}}(t)Hu_{t_{0}}(t) \\
	                                  &= -H_{\text{H}}A_{\text{H}} + \adj{u_{t_{0}}}(t)(i\hbar\del{t}{A_{\text{S}}})u_{t_{0}}(t) + A_{\text{H}}H_{\text{H}} \\
	                                  &= -{H_{\text{H}}}\comm{A_{\text{H}}} + (i\hbar\del{t}{A_{\text{S}}})_{\text{H}}.
\end{align*}