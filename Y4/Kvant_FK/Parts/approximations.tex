\section{Approximation Methods}

\paragraph{Time-Independent Perturbation Theory}
Consider a Hamiltonian of the form
\begin{align*}
	H = H_{0} + \lambda V,
\end{align*}
where $\lambda$ is a dimensionless parameter, and suppose that we know the eigenstates $\ket{n_{0}}$ of $H_{0}$. We are then interested in the eigenstates
\begin{align*}
	H\ket{n(\lambda)} = E_{n}(\lambda)\ket{n(\lambda)}.
\end{align*}
To do this we expand the new eigenstates as
\begin{align*}
	\ket{n(\lambda)} = \ket{n_{0}} + \lambda\ket{n_{1}} + \lambda^{2}\ket{n_{2}} + \dots
\end{align*}
and the new eigenvalues as
\begin{align*}
	E_{n}(\lambda) = E_{n}^{(0)} + \lambda E_{n}^{(1)} + \lambda^{2}E_{n}^{(2)} + \dots
\end{align*}
and obtain
\begin{align*}
	(H_{0} + \lambda V)\left(\ket{n_{0}} + \lambda\ket{n_{1}} + \lambda^{2}\ket{n_{2}} + \dots\right) = \left(E_{n}^{(0)} + \lambda E_{n}^{(1)} + \lambda^{2}E_{n}^{(2)} + \dots\right)\left(\ket{n_{0}} + \lambda\ket{n_{1}} + \lambda^{2}\ket{n_{2}} + \dots\right).
\end{align*}
We may now identify terms, one order at a time. The zeroth-order terms are trivial. The first-order terms are
\begin{align*}
	H_{0}\ket{n_{1}} + V\ket{n_{0}} = E_{n}^{(0)}\ket{n_{1}} + E_{n}^{(1)}\ket{n_{0}}.
\end{align*}
To simplify this, we must first say something about the states. We first require $\braket{n_{0}}{n_{i}}$ to be real. Requiring orthonormality for the new eigenvectors, we obtain
\begin{align*}
	\braket{n_{0}} + 2\lambda\braket{n_{0}}{n_{1}} + \dots = 1,
\end{align*}
meaning $\braket{n_{0}}{n_{1}}$ = 0. Using this, we obtain
\begin{align*}
	E_{n}^{(1)} = \expval{V}{n_{0}}.
\end{align*}
Next, we ty to identify the expansion coefficients by projecting onto some other state $\ket{m_{0}}$, yielding
\begin{align*}
	E_{m}^{(0)}\braket{m_{0}}{n_{1}} + \mel{m_{0}}{V}{n_{0}} = E_{n}^{(0)}\braket{m_{0}}{n_{1}},
\end{align*}
and thus
\begin{align*}
	\braket{m_{0}}{n_{1}} = \frac{\mel{m_{0}}{V}{n_{0}}}{E_{n}^{(0)} - E_{m}^{(0)}}.
\end{align*}
Hence we have
\begin{align*}
	\ket{n_{1}} = \sum\limits_{m\neq n}\frac{\mel{m_{0}}{V}{n_{0}}}{E_{n}^{(0)} - E_{m}^{(0)}}\ket{m_{0}}.
\end{align*}

%TODO: Show
This causes certain issues if any eigenvalue is degenerate. In such a case, you should for each degenerate subspace choose a basis such that the perturbation is diagonal. This will produce the same formula in the end.

Next we collect the second-order terms. We have
\begin{align*}
	H_{0}\ket{n_{2}} + V\ket{n_{1}} = E_{n}^{(0)}\ket{n_{2}} + E_{n}^{(1)}\ket{n_{1}} + E_{n}^{(2)}\ket{n_{0}}.
\end{align*}
Orthonormality yields
\begin{align*}
	\braket{n_{0}} + 2\lambda\braket{n_{0}}{n_{1}} + \lambda^{2}(2\braket{n_{0}}{n_{2}} + \braket{n_{1}}{n_{1}}) = 1,
\end{align*}
and thus
\begin{align*}
	\braket{n_{0}}{n_{2}} = -\frac{1}{2}\braket{n_{1}}{n_{1}}.
\end{align*}
The second-order expression from earlier then yields
\begin{align*}
	E_{n}^{(2)} = \mel{n_{0}}{V}{n_{1}} = \sum\limits_{m\neq n}\frac{\mel{m_{0}}{V}{n_{0}}}{E_{n}^{(0)} - E_{m}^{(0)}}\mel{n_{0}}{V}{m_{0}} = \sum\limits_{m\neq n}\frac{\abs{\mel{m_{0}}{V}{n_{0}}}^{2}}{E_{n}^{(0)} - E_{m}^{(0)}}.
\end{align*}
We may now compute the second-order state in a similar manner.

\paragraph{The Variational Principle}
Any state $\ket{\Psi}$ produces a limit on the ground-state energy of a system according to
\begin{align*}
	E_{0} \leq \frac{\expval{H}{\Psi}}{\braket{\Psi}}.
\end{align*}
This is easily shown by the fact that
\begin{align*}
	\ket{\Psi} = \sum\limits_{n}c_{n}\ket{E_{n}},
\end{align*}
yielding
\begin{align*}
	\expval{H}{\Psi} = \sum\limits_{n}E_{n}\abs{c_{n}}^{2} \leq E_{0}\sum\limits_{n}\abs{c_{n}}^{2} = E_{0}\braket{\Psi}.
\end{align*}
The variational principle uses this to approximate the ground state energy. The trick is to introduce a family of states parametrized by some set of parameters $\vb*{\alpha}$ and minimize the expression
\begin{align*}
	\frac{\expval{H}{\Psi(\vb*{\alpha})}}{\braket{\Psi(\vb*{\alpha})}}
\end{align*}
with respect to these parameters.

\paragraph{Time-Dependent Perturbation Theory}
Suppose we have a Hamiltonian of the form
\begin{align*}
	H = H_{0} + V(t),
\end{align*}
where $H_{0}$ has known eigenstates $\ket{n}$ with eigenvalues $E_{n}$. We write a general time-dependent state as
\begin{align*}
	\ket{\Psi} = \sum\limits_{n}c_{n}(t)e^{-i\frac{E_{n}}{\hbar}t}\ket{n}.
\end{align*}
This is useful as all the information about the time dependence of the Hamiltonian is contained in the expansion coefficients. Inserting this into the Schrödinger equation yields
\begin{align*}
	i\hbar\sum\limits_{n}\left(\dv{c_{n}}{t}e^{-i\frac{E_{n}}{\hbar}t} -  i\frac{E_{n}}{\hbar}c_{n}(t)e^{-i\frac{E_{n}}{\hbar}t}\right)\ket{n} = \sum\limits_{n}(E_{n} + V)c_{n}(t)e^{-i\frac{E_{n}}{\hbar}t}\ket{n}, \\
	i\hbar\sum\limits_{n}\dv{c_{n}}{t}e^{-i\frac{E_{n}}{\hbar}t} \ket{n} = \sum\limits_{n}Vc_{n}(t)e^{-i\frac{E_{n}}{\hbar}t}\ket{n}.
\end{align*}
This may be projected onto any basis state, implying
\begin{align*}
	i\hbar\dv{c_{n}}{t} = \sum\limits_{m}c_{m}(t)e^{-i\frac{(E_{m} - E_{n})}{\hbar}t}\mel{n}{V}{m}.
\end{align*}
Integrating this yields
\begin{align*}
	c_{n}(t) = c_{n} - \frac{i}{\hbar}\integ{0}{t}{\tau}{\sum\limits_{m}c_{m}(t)e^{-i\frac{(E_{m} - E_{n})}{\hbar}\tau}\mel{n}{V}{m}}
\end{align*}
This only gives an implicit solution for the expansion coefficients. To obtain a solution, we use the fact that time dependence of the coefficients should be at least proportional to the magnitude of $V$. Hence we use the first-order approximation
\begin{align*}
	c_{n}^{(1)}(t) = c_{n} - \frac{i}{\hbar}\integ{0}{t}{\tau}{\sum\limits_{m}c_{m}e^{-i\frac{(E_{m} - E_{n})}{\hbar}\tau}\mel{n}{V}{m}}.
\end{align*}
We can iterate this to obtain higher-order terms. For instance,
\begin{align*}
	c_{n}^{(2)}(t) = c_{n} - \frac{i}{\hbar}\integ{0}{t}{\tau}{\sum\limits_{m}e^{-i\frac{(E_{m} - E_{n})}{\hbar}\tau}\mel{n}{V}{m}\left(c_{m} - \frac{i}{\hbar}\integ{0}{\tau}{\tau^{\prime}}{\sum\limits_{k}c_{k}e^{-i\frac{(E_{k} - E_{m})}{\hbar}\tau^{\prime}}\mel{m}{V}{k}}\right)}.
\end{align*}

\paragraph{A Typical Example}
In a typical setup a system is prepared in some state $i$ and the perturbation is turned on at $t = 0$. This yields
\begin{align*}
	c_{n}^{(1)}(t) &= \delta_{ni} - \frac{i}{\hbar}\integ{0}{t}{\tau}{\sum\limits_{m}\delta_{mi}e^{-i\frac{(E_{m} - E_{n})}{\hbar}\tau}\mel{n}{V}{m}} \\
	               &= \delta_{ni} - \frac{i}{\hbar}\integ{0}{t}{\tau}{e^{-i\frac{(E_{i} - E_{n})}{\hbar}\tau}\mel{n}{V}{i}}.
\end{align*}
This allows us to exclude certain transitions depending on the properties of the potential.

\paragraph{Harmonic Perturbations}
Let us use a harmonic perturbation
\begin{align*}
	V(t) = V\left(e^{i\omega t} + e^{-i\omega t}\right).
\end{align*}
In the typical case previously described, we obtain
\begin{align*}
	c_{n}^{(1)}(t) = \delta_{ni} - \frac{i}{\hbar}\integ{0}{t}{\tau}{e^{-i\frac{(E_{i} - E_{n})}{\hbar}\tau}\mel{n}{V}{i}}.
\end{align*}
Introducing
\begin{align*}
	\omega_{in} = \frac{(E_{i} - E_{n})}{\hbar},
\end{align*}
we obtain
\begin{align*}
	c_{n}^{(1)}(t) &= \delta_{ni} - \frac{i}{\hbar}\mel{n}{V}{i}\integ{0}{t}{\tau}{e^{-i\omega_{in}\tau}\left(e^{i\omega\tau} + e^{-i\omega\tau}\right)} \\
	               &= \delta_{ni} - \frac{i}{\hbar}\mel{n}{V}{i}\integ{0}{t}{\tau}{e^{i(\omega - \omega_{in})\tau} + e^{i(-\omega - \omega_{in})\tau}}.
\end{align*}
Consider now some $n\neq i$. For this case we obtain
\begin{align*}
	c_{n}^{(1)}(t) &= -\frac{1}{\hbar}\mel{n}{V}{i}\left(\frac{e^{i(\omega - \omega_{in})t} - 1}{(\omega - \omega_{in})} - \frac{e^{i(-\omega - \omega_{in})t} - 1}{\omega + \omega_{in}}\right).
\end{align*}
In the limit of large times, one obtains
\begin{align*}
	P(i\to n) = \frac{2\pi}{\hbar}\abs{\mel{n}{V}{i}}^{2}\delta(E_{n} - E_{i} - \hbar\omega),
\end{align*}
which is called Fermi's golden rule.

\paragraph{Adiabatic Evolution}
Consider some time-dependent Hamiltonian. Imagining that we can diagonalize the Hamiltonian at any time, energy levels obtained may or may not cross as a function of time. For two energy levels to cross, there can be no coupling in the Hamiltonian between the corresponding states. A general state may now be written as
\begin{align*}
	\ket{\Psi(t)} = \sum\limits_{n}c_{n}(t)\ket{n(t)}.
\end{align*}
Inserted into the Schrödinger equation, we obtain
\begin{align*}
	i\hbar\sum\limits_{n}\dv{c_{n}}{t}\ket{n(t)} + c_{n}(t)\dv{t}\ket{n(t)} = \sum\limits_{n}c_{n}(t)E_{n}(t)\ket{n(t)}.
\end{align*}
Projecting onto some particular state $\ket{m(t)}$ yields
\begin{align*}
	i\hbar\left(\dv{c_{m}}{t} + \sum\limits_{n}c_{n}(t)\bra{m}\dv{t}\ket{n(t)}\right) = c_{m}(t)E_{m}(t).
\end{align*}
The Schrödinger equation may not be used on the basis states due to how they are defined (I should say more about this). However, we have
\begin{align*}
	\dv{t}(H\ket{n(t)}) = \dv{t}(E_{n}(t)\ket{n(t)}), \\
	\dv{H}{t}\ket{n(t)} + H\dv{t}\ket{n(t)} = \dv{E_{n}}{t}\ket{n(t)} + E_{n}(t)\dv{t}\ket{n(t)}.
\end{align*}
Projecting onto the state $\ket{m}$ (with suppressed time dependence) yields
\begin{align*}
	\mel{m}{\dv{H}{t}}{n} + E_{m}(t)\bra{m}\dv{t}\ket{n(t)} = \dv{E_{n}}{t}\delta_{mn} + E_{n}(t)\bra{m}\dv{t}\ket{n(t)}.
\end{align*}
For $m\neq n$ we thus have
\begin{align*}
	\bra{m}\dv{t}\ket{n(t)} = -\frac{\mel{m}{\dv{H}{t}}{n}}{E_{m} - E_{n}}.
\end{align*}
Inserted into our previous expression we obtain
\begin{align*}
	i\hbar\left(\dv{c_{m}}{t} + c_{m}(t)\bra{m}\dv{t}\ket{m} - \sum\limits_{n \neq m}c_{n}(t)\frac{\mel{m}{\dv{H}{t}}{n}}{E_{m} - E_{n}}\right) = c_{m}(t)E_{m}(t)
\end{align*}

This means that we may ignore contribution from states such that the potential varies sufficiently slowly that the above matrix elements are much smaller than the energy differences. These are the states causing crossings, meaning that in the limit of very slowly varying potentials the above equation is diagonal in $c_{m}$, and if one starts in $\ket{m}$, one will stay there.

\paragraph{Berry Phase}
Dropping non-diagonal terms we obtain
\begin{align*}
	i\hbar\left(\dv{c_{m}}{t} + c_{m}(t)\bra{m}\dv{t}\ket{m}\right) = c_{m}(t)E_{m}(t),
\end{align*}
with the solution
\begin{align*}
	c_{m}(t) = c_{m}(0)e^{-\frac{i}{\hbar}\integ{0}{t}{\tau}{E(\tau)} + i\gamma(t)}.
\end{align*}
$\gamma$ is termed the Berry phase, and given by
\begin{align*}
	\gamma(t) = i\integ{0}{t}{\tau}{\bra{m}\dv{t}\ket{m}}.
\end{align*}
We find that it is imaginary, as
\begin{align*}
	\braket{m} = 1 \rightarrow 2\Re(\bra{m}\dv{t}\ket{m}) = 0.
\end{align*}

The Berry phase has a geometric interpretation. To understand it, construct a vector $\vb{R}(t)$ of parameters entering into the Hamiltonian. The eigenvalues of the instantaneous eigenstates are thus functions of these parameters. The Berry phase is given by
\begin{align*}
	\gamma &= i\integ{0}{t}{\tau}{\bra{m}\dv{t}\ket{m}} \\
	       &= i\integ{0}{t}{\tau}{\braket{m}{\grad_{\vb{R}}{m}}}\cdot\dv{\vb{R}}{t} \\
	       &= i\integ{}{}{\vb{R}}{\cdot\braket{m}{\grad_{\vb{R}}{m}}},
\end{align*}
where the integral is now converted to a curve integral in parameter space. This is the geometric interpretation.

Imagine now that we vary the parameters periodically with period $T$. The Berry phase then contains an integral over a closed curve. By introducing the Berry connection
\begin{align*}
	\vb{A}_{m} = \braket{m}{\grad_{\vb{R}}{m}},
\end{align*}
which indeed satisfies
\begin{align*}
	i\braket{m}{\grad_{\vb{R}}{m}} = \vb{A},
\end{align*}
we may also introduce the Berry curvature
\begin{align*}
	\vb*{\Omega} = \curl{\vb{A}},
\end{align*}
the flux of which may equally well determine the Berry phase.

Prior to Berry's paper, it was believed that the Berry phase could be eliminated. Namely, by modifying the phase of the basis states by
\begin{align*}
	\ket{n}\to e^{i\chi(\vb{R})}\ket{n},
\end{align*}
the Berry connection would be transformed to
\begin{align*}
	\vb{A}\to \vb{A} - \grad_{\vb{R}}{\chi}.
\end{align*}
For non-periodic paths through parameter space, this would indeed be the case. However, for a closed path, we may instead compute the Berry phase using a flux integral of the Berry curvature. As the Berry curvature is unchanged by this change of phase, the Berry phase is also left unchanged.