\section{Approximation Methods}

\paragraph{Perturbation Theory}
Consider a Hamiltonian of the form
\begin{align*}
	H = H_{0} + \lambda V,
\end{align*}
where $\lambda$ is a dimensionless parameter, and suppose that we know the eigenstates $\ket{n_{0}}$ of $H_{0}$. We are then interested in the eigenstates
\begin{align*}
	H\ket{n(\lambda)} = E_{n}(\lambda)\ket{n(\lambda)}.
\end{align*}
To do this we expand the new eigenstates as
\begin{align*}
	\ket{n(\lambda)} = \ket{n_{0}} + \lambda\ket{n_{1}} + \lambda^{2}\ket{n_{2}} + \dots
\end{align*}
and the new eigenvalues as
\begin{align*}
	E_{n}(\lambda) = E_{n}^{(0)} + \lambda E_{n}^{(1)} + \lambda^{2}E_{n}^{(2)} + \dots
\end{align*}
and obtain
\begin{align*}
	(H_{0} + \lambda V)\left(\ket{n_{0}} + \lambda\ket{n_{1}} + \lambda^{2}\ket{n_{2}} + \dots\right) = \left(E_{n}^{(0)} + \lambda E_{n}^{(1)} + \lambda^{2}E_{n}^{(2)} + \dots\right)\left(\ket{n_{0}} + \lambda\ket{n_{1}} + \lambda^{2}\ket{n_{2}} + \dots\right).
\end{align*}
We may now identify terms, one order at a time. The zeroth-order terms are trivial. The first-order terms are
\begin{align*}
	H_{0}\ket{n_{1}} + V\ket{n_{0}} = E_{n}^{(0)}\ket{n_{1}} + E_{n}^{(1)}\ket{n_{0}}.
\end{align*}
To simplify this, we must first say something about the states. We first require $\braket{n_{0}}{n_{i}}$ to be real. Requiring orthonormality for the new eigenvectors, we obtain
\begin{align*}
	\braket{n_{0}} + 2\lambda\braket{n_{0}}{n_{1}} + \dots = 1,
\end{align*}
meaning $\braket{n_{0}}{n_{1}}$ = 0. Using this, we obtain
\begin{align*}
	E_{n}^{(1)} = \expval{V}{n_{0}}.
\end{align*}
Next, we ty to identify the expansion coefficients by projecting onto some other state $\ket{m_{0}}$, yielding
\begin{align*}
	E_{m}^{(0)}\braket{m_{0}}{n_{1}} + \mel{m_{0}}{V}{n_{0}} = E_{n}^{(0)}\braket{m_{0}}{n_{1}},
\end{align*}
and thus
\begin{align*}
	\braket{m_{0}}{n_{1}} = \frac{\mel{m_{0}}{V}{n_{0}}}{E_{n}^{(0)} - E_{m}^{(0)}}.
\end{align*}
Hence we have
\begin{align*}
	\ket{n_{1}} = \sum\limits_{m\neq n}\frac{\mel{m_{0}}{V}{n_{0}}}{E_{n}^{(0)} - E_{m}^{(0)}}\ket{m_{0}}.
\end{align*}
This causes certain issues if any eigenvalue is degenerate, but in such cases you choose an orthogonal basis for that particular eigenspace so that the corresponding term becomes zero.

Next we collect the second-order terms. We have
\begin{align*}
	H_{0}\ket{n_{2}} + V\ket{n_{1}} = E_{n}^{(0)}\ket{n_{2}} + E_{n}^{(1)}\ket{n_{1}} + E_{n}^{(2)}\ket{n_{0}}.
\end{align*}
Orthonormality yields
\begin{align*}
	\braket{n_{0}} + 2\lambda\braket{n_{0}}{n_{1}} + \lambda^{2}(2\braket{n_{0}}{n_{2}} + \braket{n_{1}}{n_{1}}) = 1,
\end{align*}
and thus
\begin{align*}
	\braket{n_{0}}{n_{2}} = -\frac{1}{2}\braket{n_{1}}{n_{1}}.
\end{align*}
The second-order expression from earlier then yields
\begin{align*}
	E_{n}^{(2)} = \mel{n_{0}}{V}{n_{1}} = \sum\limits_{m\neq n}\frac{\mel{m_{0}}{V}{n_{0}}}{E_{n}^{(0)} - E_{m}^{(0)}}\mel{n_{0}}{V}{m_{0}} = \sum\limits_{m\neq n}\frac{\abs{\mel{m_{0}}{V}{n_{0}}}^{2}}{E_{n}^{(0)} - E_{m}^{(0)}}.
\end{align*}
We may now compute the second-order state in a similar manner.

\paragraph{The Variational Principle}
Any state $\ket{\Psi}$ produces a limit on the ground-state energy of a system according to
\begin{align*}
	E_{0} \leq \frac{\expval{H}{\Psi}}{\braket{\Psi}}.
\end{align*}
This is easily shown by the fact that
\begin{align*}
	\ket{\Psi} = \sum\limits_{n}c_{n}\ket{E_{n}},
\end{align*}
yielding
\begin{align*}
	\expval{H}{\Psi} = \sum\limits_{n}E_{n}\abs{c_{n}}^{2} \leq E_{0}\sum\limits_{n}\abs{c_{n}}^{2} = E_{0}\braket{\Psi}.
\end{align*}
The variational principle uses this to approximate the ground state energy. The trick is to introduce a family of states parametrized by some set of parameters $\vb*{\alpha}$ and minimize the expression
\begin{align*}
	\frac{\expval{H}{\Psi(\vb*{\alpha})}}{\braket{\Psi(\vb*{\alpha})}}
\end{align*}
with respect to these parameters.