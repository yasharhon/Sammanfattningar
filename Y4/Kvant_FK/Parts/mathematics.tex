\section{Useful Mathematics}

\paragraph{A Useful Commutation Relation}
You might happen upon commutation relations of the form $\comm{f(A)}{B}$ show up. We would like to try to simplify it for the particular case where $\comm{A}{B} = C$, where $C$ is some operator commuting with $A$. To do this, we first study $\comm{A^{n}}{B}$ in a general case. We have
\begin{align*}
	\comm{A^{n}}{B} = A\comm{A^{n - 1}}{B} + \comm{A}{B}A^{n - 1},
\end{align*}
prompting us to find this commutator by induction. For $n = 2$ we have
\begin{align*}
	\comm{A^{2}}{B} = A\comm{A}{B} + \comm{A}{B}A.
\end{align*}
For $n = 3$ we obtain
\begin{align*}
	\comm{A^{3}}{B} = A\comm{A^{2}}{B} + \comm{A}{B}A^{2} = A(A\comm{A}{B} + \comm{A}{B}A) + \comm{A}{B}A^{2} = A^{2}\comm{A}{B} + A\comm{A}{B}A + \comm{A}{B}A^{2}.
\end{align*}
A suitable induction hypothesis looks to be
\begin{align*}
	\comm{A^{n}}{B} = \sum\limits_{k = 1}^{n}A^{n - k}\comm{A}{B}A^{k - 1}.
\end{align*}
Assuming it to be true, we have
\begin{align*}
	\comm{A^{n + 1}}{B} &= A\comm{A^{n}}{B} + \comm{A}{B}A^{n} \\
	                    &= A\sum\limits_{k = 1}^{n}A^{n - k}\comm{A}{B}A^{k - 1} + \comm{A}{B}A^{n} \\
	                    &= \sum\limits_{k = 1}^{n}A^{n + 1 - k}\comm{A}{B}A^{k - 1} + \comm{A}{B}A^{n} \\
	                    &= \sum\limits_{k = 1}^{n + 1}A^{n + 1 - k}\comm{A}{B}A^{k - 1},
\end{align*}
proving the hypothesis by induction.

Now we write
\begin{align*}
	f(A) = \sum\limits_{n = 0}^{\infty}\frac{1}{n!}f_{n}A^{n}
\end{align*}
to obtain
\begin{align*}
	\comm{f(A)}{B} &= \comm{\sum\limits_{n = 0}^{\infty}\frac{1}{n!}f_{n}A^{n}}{B} \\
	               &= \sum\limits_{n = 0}^{\infty}\frac{1}{n!}f_{n}\comm{A^{n}}{B} \\
	               &= \sum\limits_{n = 0}^{\infty}\frac{1}{n!}f_{n}\sum\limits_{k = 1}^{n}A^{n - k}\comm{A}{B}A^{k - 1}.
\end{align*}
Assuming $\comm{A}{B}$ to commute with $A$, we have
\begin{align*}
	\comm{f(A)}{B} &= \sum\limits_{n = 0}^{\infty}\frac{1}{n!}f_{n}A^{n - 1}\comm{A}{B}\sum\limits_{k = 1}^{n} \\
	               &= \comm{A}{B}\sum\limits_{n = 1}^{\infty}\frac{1}{(n - 1)!}f_{n}A^{n - 1} \\
	               &= \comm{A}{B}\sum\limits_{m = 0}^{\infty}\frac{1}{m!}f_{m + 1}A^{m} \\
	               &= \comm{A}{B}\dv{f}{A}.
\end{align*}

\paragraph{Symmetries on Hilbert Space}
A symmetry on Hilbert space is a transformation that leaves all expectation values unaltered.

\paragraph{Wigner's Theorem}
Wigner's theorem states that any operator that is a symmetry is either unitary or anti-unitary (the latter meaning $\mel{\Phi}{\adj{U}U}{\Psi} = \braket{\Psi}{\Phi}$).

\paragraph{Transformation of Operators}
Consider a symmetry operator $u$. In order for this to be a symmetry, it must also act on all operators according to $A\to uA\adj{u}$.

\paragraph{Time Evolution From Symmetry}
Consider some system with time translation symmetry - that is, any system for which time translations do not change the theory. Introduce the transformation operator
\begin{align*}
	u_{\tau}\ket{\Psi(t)} = \ket{\Psi(t + \tau)}.
\end{align*}
This transformation is a smooth map acting on a manifold - namely, Hilbert space. Hence we can use the language of Lie algebra to treat this (if you know nothing about Lie algebra, pretend that I didn't write this and carry on. If you want some reference material, please look at my summary of SI2360). We expand the transformation operator around the identity as
\begin{align*}
	u_{\tau} = 1 - i\frac{\tau}{\hbar}H
\end{align*}
for some operator $H$. The requirement that this be unitary yields $\adj{H} - H = 0$, and hence the generator $H$ is self-adjoint. By continuous application of this we obtain
\begin{align*}
u_{\tau} = e^{-i\frac{\tau}{\hbar}H}.
\end{align*}
This reproduces the Schrödinger equation, tying it all together neatly. It also demonstrates that the Hamiltonian generates time translation in a mathematical sense.

\paragraph{Tensors}
A tensor of rank $n$ is a multilinear map from $n$ vectors in some vector space to a scalar.

It is clear that the set of tensors of some rank form a vector space, and so we would like to identify some basis for the space of tensors.

\paragraph{Basis for $n = 1$}
We start with rank $1$ tensors. The inner product is certainly a rank $1$ tensor according to the definition, and so we would like to use that. Now let the set of $v_{i}$ denote the set of orthonormal basis vectors for the space $V$. We then choose the basis
\begin{align*}
	e_{i}(v) = \bra{v_{i}}{v}
\end{align*}
as the basis for the set of rank $1$ tensors. This may also be denoted simply as the tensor $v_{i}$ (the confusion will disappear later).

\paragraph{The Tensor Product}
To find a basis for tensors of higher order, we first need to introduce the tensor product. We define it for rank $1$ tensors as
\begin{align*}
	s(v)\otimes t(w) = \braket{s}{v}\braket{t}{w}.
\end{align*}
The tensor product has allowed us to construct a rank $2$ tensor from two rank $1$ tensors. Repeatedly applying it allows us to construct tensors of any rank. The tensor product is also bilinear, in line with our definition.

\paragraph{The Tensor Product of Operators}
It follows naturally that
\begin{align*}
	(S\otimes T)(v\otimes w) = S(v)\otimes T(w).
\end{align*}