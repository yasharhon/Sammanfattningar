\documentclass[a4paper, 11pt]{article}
\usepackage[swedish]{babel}
\usepackage{hyperref}
\usepackage[margin=0.5in]{geometry}

\usepackage{siunitx}
\usepackage{amsmath}
\usepackage{amssymb}
\usepackage[arrowdel]{physics}

\sisetup{exponent-product = \cdot}

\newcommand{\emf}{\mathcal{E}}

\newcommand{\cc}[1]{#1^{\star}}
\newcommand{\ub}[1]{\vb{e}_{\vb{#1}}}
\newcommand{\del}[3][]{\partial_{#2}^{#1}#3}
\newcommand{\delp}[2]{\partial_{#1}^{\prime}#2}
\newcommand{\gradp}[1]{\grad^{\prime}{#1}}
\newcommand{\divp}[1]{\grad^{\prime}\cdot{#1}}
\newcommand{\curlp}[1]{\grad^{\prime}\times{#1}}
\newcommand{\dalemb}[1][]{\Box #1}
\newcommand{\gradperp}[1][]{\grad_{\perp}{#1}}
\newcommand{\laplaperp}[1][]{\laplacian_{\perp}{#1}}
\newcommand{\integ}[5][]{\int\limits_{#2}^{#3}\dd[#1]{#4}#5}
\newcommand{\vinteg}[4]{\int\limits_{#1}^{#2}\dd{\vb{#3}}\cdot #4}
\newcommand{\tinteg}[4]{\int\limits_{#1}^{#2}\dd{\vb{#3}}\times #4}
\newcommand{\fou}[1]{\mathcal{F}\left(#1\right)}
\newcommand{\fouinv}[1]{\mathcal{F}^{-1}\left(#1\right)}

\title{Sammanfattning av EI1320 Teoretisk elektroteknik}
\author{Yashar Honarmandi \\ yasharh@kth.se}
\date{\today}

\begin{document}

\maketitle

\begin{abstract}
	Denna sammanfattningen innehåller centrala definitioner och satser i SF1672 Flervariabelanalys.
\end{abstract}

\pagenumbering{roman}
\thispagestyle{empty}

\newpage

\tableofcontents

\newpage

\pagenumbering{arabic}

\section{Notation}
Om inte annat specifieras, kommer alla ekvationer använda notationen som ges i denna tabellen.

\begin{table}[!h]
	\centering
	\begin{tabular}{| l | c |}
		\hline
		\textbf{Storhet} & \multicolumn{1}{|l|}{\textbf{Symbol}}\\
		Position         & $\vect{r}$ \\
		\hline
		Tid              & $t$ \\
		\hline
		Period           & $T$ \\
		\hline
		Frekvens         & $f$ \\
		\hline
		Vinkelfrekvens   & $\omega$ \\
		\hline
		Våglängd         & $\lambda$ \\
		\hline
		Vågvektor        & $\vect{k}$ \\
		\hline
		Vågtal           & $k$ \\
		\hline
		Amplitud         & $A$ \\
		\hline
		Vågfart          & $c$ \\
		\hline
		Tryck            & $p$ \\
		\hline
	\end{tabular}
\end{table}

%Vakuumperm

\section{Lite vektoranalys och annan matte}

\paragraph{Diracs delta i högre dimensioner}
Diracs deltafunktion generaliserar utan vidare till högre dimensioner. Med andra ord är $\delta(\vb{r})$ en funktion som är noll överalt förutom origo och som uppfyller
\begin{align*}
	\integ{V}{}{V}{\delta(\vb{r})} = 1
\end{align*}
om $V$ innesluter origo.

\paragraph{Nablaoperatorn i olika koordinatsystem}
Betrakta två olika koordinatsystem $S$ och $S'$. Med hjälp av de kartesiska basvektorerna (som är lika i bägge koordinatsystemen) kan vi skriva ortsvektorn i de två som
\begin{align*}
	\vb{r} = r_{i}\ub{i},\ \vb{r}' = r_{i}'\ub{i}.
\end{align*}
Vidare kan vi skriva nablaoperatorn som
\begin{align*}
	\grad{} = \ub{i}\del{i}{},\ \grad{} = \ub{i}\delp{i}{}.
\end{align*}
Betrakta nu en funktion av $\vb{R} = \vb{r} - \vb{r}'$. Då kan vi visa att
\begin{align*}
	\del{i}{f} = -\delp{i}{f}.
\end{align*}

\paragraph{Gradienten av $R$}
Betrakta funktionen
\begin{align*}
	f(\vb{R}) = \sqrt{R_{j}R_{j}} = R.
\end{align*}
Vi har
\begin{align*}
	\del{i}{R} = \frac{1}{2}(R_{k}R_{k})^{-\frac{1}{2}}\cdot 2R_{j}\del{i}{R_{j}} = (R_{k}R_{k})^{-\frac{1}{2}}R_{j}\del{i}{(r_{j} - r_{j}'} = \frac{R_{j}}{R}\delta{ij} = \frac{R_{i}}{R}\delta{ij}.
\end{align*}
Detta ger
\begin{align*}
	\grad{R} = -\grad'{R} = \ub{\vb{R}}.
\end{align*}

\paragraph{Divergensen av $\frac{1}{R^{2}}$-fältet}
Med resultatet ovan har vi
\begin{align*}
	\div{\frac{1}{R^{2}}\ub{\vb{R}}} &= \div{\frac{1}{R^{3}}\vb{R}} \\
	                                 &= \vb{R}\cdot\grad{\frac{1}{R^{3}}} + \frac{1}{R^{3}}\div{\vb{R}} \\
	                                 &= -\frac{3}{R^{4}}\vb{R}\cdot\grad{R} + \frac{1}{R^{3}}\div{\vb{R}} \\
	                                 &= -\frac{3}{R^{4}}\vb{R}\cdot\ub{\vb{R}} + \frac{3}{R^{3}} \\
	                                 &= -\frac{3}{R^{3}} + \frac{3}{R^{3}} \\
	                                 &= 0
\end{align*}
så länge $\vb{R}\neq\vb{0}$.

Mer allmänt kan man visa att
\begin{align*}
	\div{\frac{1}{R^{2}}\ub{\vb{R}}} = 4\pi\delta(\vb{R}).
\end{align*}
Jag kan inte bevisa det, men jag kan rationalisera det kort. Utanför origo är det klart att detta stämmer. För att förstå vad som händer i origo, kan vi tillämpa den koordinatoberoende definitionen av divergens. Med den definitionen är divergensen av ett vektorfält kvoten av fältets flöde genom en litan yta kring en punkt och volymen den lilla ytan inneslutar. Med flervariabelanalys kan man visa att för fältet vi betraktar är flödet exakt $4\pi$. Om vi jämför detta med Diracs delta, ser vi att det verkar stämma.

\paragraph{$\frac{1}{R}$ och Greenfunktioner}
Med resultaten vi har får vi även
\begin{align*}
	\grad{\frac{1}{R}} = -\frac{1}{R^{2}}\grad{R} = -\frac{1}{R^{2}}\ub{\vb{R}}.
\end{align*}
Detta betyder att
\begin{align*}
	\laplacian{\frac{1}{R}} = \div{\grad{\frac{1}{R}}} = -\div{\frac{1}{R^{2}}\ub{\vb{R}}} = -4\pi\delta(\vb{R}).
\end{align*}
Detta betyder att $\frac{1}{R}$ är en Greenfunktion till Laplaceoperatorn (i tre dimensioner).

\section{Elektrostatik}

\paragraph{Coulombs lag}
Elektrostatiken utgår från Coulombs lag, som är en experimentellt framtagen lag. Den säjer att om två laddningar $Q$ och $q$ är separerade med en sträcka $\vb{R}$, är kraften mellan dem
\begin{align*}
	\vb{F} = \frac{1}{4\pi\varepsilon_{0}}\frac{qQ}{R^{3}}\vb{R}.
\end{align*}
Alternativt, i termer av enhetsvektorer,
\begin{align*}
	\vb{F} = \frac{1}{4\pi\varepsilon_{0}}\frac{qQ}{R^{2}}\ub{R}.
\end{align*}
Båda laddningarna antas ha försumbar utsträckning, och betecknas punktladdningar. $\varepsilon_{0}$ kallas vakuumpermittiviteten, och har enhet \si{\farad\per\meter}.

\paragraph{Elektriskt fält}
Det elektriska fältet som genereras av en laddning $Q$ definieras som att en liten testladdning $q$ upplever en kraft
\begin{align*}
	\vb{F} = q\vb{E}
\end{align*}
från $Q$. I våran definition skulle vi kunna lägga på ett $\lim\limits_{q\to 0}$, för att vara tydliga.

Baserad på detta får vi att en punktladdning $Q$ genererar ett elektriskt fält
\begin{align*}
	\vb{E} = \frac{1}{4\pi\varepsilon_{0}}\frac{Q}{R^{2}}\ub{R}.
\end{align*}

\paragraph{Linjäritet och allmäna uttryck för elektriska fältet}
Eftersom krafter superponeras, gör även elektriska fält det. I diskreta fall motsvarar detta att summera över punktladdningar. I kontinuerliga fall integrerar vi i stället, där varje element i integrationen behandlas som en punktladdning, och vi får
\begin{align*}
	\vb{E} = \frac{1}{4\pi\varepsilon_{0}}\integ{}{}{q}{\frac{1}{R^{2}}\ub{R}}.
\end{align*}
Laddningen kan vara spridd ut på en linje, en yta eller en volym, i vilka fall vi får $\dd{q} = \rho\dd{l}$, $\dd{q} = \sigma\dd{S}$ respektiva $\dd{q} = \lambda\dd{V}$. Förutom de olika elementerna finns en linjeladdningstäthet, ytladdningstäthet eller volymladdningstäthet. Notera att med hjälp av Diracs delta kan alla dessa fallen skrivas som volymladdningstätheter.

\paragraph{Gauss' lag}
För att härleda Gauss' lag börjag vi med att titta på flödet av fältet $\frac{1}{R^{2}}\ub{R}$ genom en godtycklig yta. Integrationselementet
\begin{align*}
	\dd{\Omega} = \frac{\ub{R}\cdot\vb{\dd{a}}}{R^{2}}
\end{align*}
är rymdvinkeln som areaelementet upptar när det ses från origo. Vi kan se på något sätt att detta motsvarar att projicera areaelementet ned på enhetssfären kring origo. Alternativt, om fältets källa är en linjeladdning, skulle man projicera ned på enhetscirkeln. Flödet vi betraktar ges då av fönsterfunktionen
\begin{align*}
	f(\vb{r}') = \integ{S}{}{S}{\frac{\ub{R}\cdot\ub{n}}{R^{2}}} = \integ{\Omega}{}{\Omega}{} =
	\begin{cases}
		4\pi, &\vb{r}'\text{ innanför }S, \\
		0,    &\vb{r}'\text{ utanför }S.
	\end{cases}
\end{align*}

Vi kommer nu ihåg hur elektriska fältet ser ut på integralform, specifikt som en volymintegral, och får då för flödet genom en godtycklig yta
\begin{align*}
	\vinteg{S}{}{a}{\vb{E}} &= \vinteg{S}{}{a}{\frac{1}{4\pi\varepsilon_{0}}\integ{}{}{\tau'}{\frac{\rho}{R^{2}}\ub{R}}} \\
	                      &= \frac{1}{4\pi\varepsilon_{0}}\integ{}{}{\tau'}{\vinteg{S}{}{S}{\rho\frac{\ub{R}}{R^{2}}}} \\
	                      &= \frac{1}{4\pi\varepsilon_{0}}\integ{}{}{\tau'}{\rho f(\vb{r}')} \\
	                      &= \frac{Q_{\text{innesluten}}}{\varepsilon_{0}}.
\end{align*}
Den sista integralen är lika med laddningen som är innesluten i $S$ eftersom fönsterfunktionen ger ett bidrag $4\pi$ om och endast om det finns laddning i den aktuella punkten.

Gauss' lag är ett bra verktyg för att beräkna elektriska fält för geometrier med mycket symmetri.

\paragraph{Gauss' lag på differentialform}
Betrakta nu en godtycklig yta $S$ som exakt inneslutar (den lite godtyckliga) volymen $V$. Gauss' lag ger då
\begin{align*}
	\vinteg{S}{}{a}{\vb{E}} = \frac{1}{\varepsilon_{0}}\integ{V}{}{\tau}{\rho}.
\end{align*}
Vi kan använda divergenssatsen för att skriva om vänstersidan som en integral över $V$. Detta ger
\begin{align*}
	\integ{V}{}{\tau}{\div{\vb{E}}} = \frac{1}{\varepsilon_{0}}\integ{V}{}{\tau}{\rho}.
\end{align*}
Eftersom $V$ är godtyckligt vald, måste det gälla att
\begin{align*}
	\div{\vb{E}} = \frac{\rho}{\varepsilon_{0}}.
\end{align*}

\paragraph{Randvillkor för elektrist fält}
Ytladdningar ger diskontinuiteter i elektriskt fält. För att studera det, betrakta en punkt på en yta. Gör en liten låda kring punkten så att fältet är ungefär konstant på sidorna som inte rör ytan, och döp varje sida $1$ och $2$. Inför även (enhets)normalvektorn $\vb{n}_{12}$. Gauss' lag ger oss, om bara tar med de nämnda sidorna, att elektriksa fältets normalkomponent relativt ytan uppfyller
\begin{align*}
	\vb{E}_{1}\cdot\vb{n}_{12}A - \vb{E}_{2}\cdot\vb{n}_{12}A = \frac{\sigma A}{\varepsilon_{0}},
\end{align*}
där $A$ är sidornas yta. Positiv riktning för fältets normalkomponent är ut från ytan. Vid att låta lådan bli oändligt tunn kommer även de andra sidorna inte att ge något bidrag, varför det här måste stämma. Vi får därmed att
\begin{align*}
	(\vb{E}_{1} - \vb{E}_{2})\cdot\vb{n}_{12} = \frac{\sigma}{\varepsilon_{0}}.
\end{align*}

Vi kan även betrakta en liten fyrkantig slinga på samma sätt, med två sidor parallella med ytan och två normala på ytan. Eftersom integralen av elektriska fältet kring en sluten kurva alltid är $0$, får vi
\begin{align*}
	E_{1}^{\parallel} - E_{2}^{\parallel} = 0,
\end{align*}
där fältets komponenter ges med avseende på någon riktning. Eftersom denna slingan kan ha vilken som helst orientering så länge man har två paralllella och två normala sidor, måste det gälla för ala tangentiella riktningar, vilket ger
\begin{align*}
	(\vb{E}_{1} - \vb{E}_{2})\times\vb{n}_{12} = \vb{0}.
\end{align*}

Dessa två resultat kan sammanfattas som
\begin{align*}
	\vb{E}_{1} - \vb{E}_{2} = \frac{\sigma}{\varepsilon_{0}}\vb{n}_{12}.
\end{align*}

\paragraph{Kraften på en ytladdning}
Kring en ytladdning är elektriska fältet diskontinuerligt, så hur beräknar man kraften på en sådan? Med hjälp av superposition kan elektriska fältet skrivas som en summa av bidrag från själva laddningen och allt annat. Termen från andra saker, som är den som ger kraften, är kontinuerlig i ytan eftersom man skulle kunna ta bort ytladdningen utan att ändra den. Vidare är fältet tillräcklig nära ytladdningen $\pm\frac{\sigma}{2\varepsilon_{0}}\vb{n}_{12}$, som försvinner om man tar medelvärdet. Alltså ges kraften av laddningen gånger medelärdet av fältet på varje sida.

\paragraph{Elektrostatisk potential}
Med vår kunnskap från vektoranalysen kan vi skriva
\begin{align*}
	\vb{E} = -\frac{1}{4\pi\varepsilon_{0}}\integ{}{}{\tau'}{\rho\grad{\frac{1}{R}}} = -\grad(\frac{1}{4\pi\varepsilon_{0}}\integ{}{}{\tau'}{\rho\frac{1}{R}}).
\end{align*}
Vi definierar därmed den elektrostatiska potentialen enligt
\begin{align*}
	\vb{E} = -\grad{V}.
\end{align*}
Från våran definition ser vi att nollnivån för potentialen kan sättas arbiträrt, då det elektriska fältet (som är det som är fysikaliskt) inte ändras om potentialen ändras med en konstant. Vi brukar lägga nollnivån i oändligheten, vilket för ändliga laddningstätheter ger
\begin{align*}
	V = \frac{1}{4\pi\varepsilon_{0}}\integ{}{}{\tau'}{\rho\frac{1}{R}}.
\end{align*}

En konsekvens av detta är
\begin{align*}
	\curl{\vb{E}} = \vb{0}.
\end{align*}

\paragraph{Potential och elektrisk spänning}
Betrakta storheten $\vb{E}\cdot\dd{\vb{r}}$. Vi har
\begin{align*}
	\vb{E}\cdot\dd{\vb{r}} = E_{i}\dd{x_{i}} = -\del{i}{V}\dd{x_{i}} = -\dd{V}.
\end{align*}
Om vi nu jämför detta med den elektriska spänningen
\begin{align*}
	U_{12} = \vinteg{\vb{r}_{1}}{\vb{r}_{2}}{l}{\vb{E}}
\end{align*}
mellan två punkter (som är den välkända spänningen vi känner från kretsvärlden), kan vi se att detta blir
\begin{align*}
	U_{12} = \vinteg{\vb{r}_{1}}{\vb{r}_{2}}{l}{\vb{E}} = -\integ{\vb{r}_{1}}{\vb{r}_{2}}{V}{} = V(\vb{r}_{1}) - V(\vb{r}_{2}),
\end{align*}
oberoende av vägen mellan punkterna. Om vi lägger potentialens referens i oändligheten, ser vi då att
\begin{align*}
	V(\vb{r}_{0}) = -\vinteg{\infty}{\vb{r}_{0}}{l}{\vb{E}},
\end{align*}
ett typ inverst påstående av $\vb{E} = -\grad{V}$. Intgrationsbanan parametriseras från $\infty$ till $\vb{r}_{0}$, ett mistag jag gör lite för många gånger. Vi ser även från detta att
\begin{align*}
	V(\vb{r}_{2}) - V(\vb{r}_{1}) = -\vinteg{\vb{r}_{1}}{\vb{r}_{2}}{l}{\vb{E}}.
\end{align*}

\paragraph{Potential och arbete}
Antag att du vill förflytta en laddning $q$ i ett elektriskt fält. Den minsta kraften du måste verka med på laddningen för att göra detta är $\vb{F} = -q\vb{E}$, eftersom du arbetar mot det elektriska fältet. Arbetet du gör då är
\begin{align*}
	W = \vinteg{\vb{r}_{1}}{\vb{r}_{2}}{l}{\vb{F}} = -q\vinteg{\vb{r}_{1}}{\vb{r}_{2}}{l}{\vb{E}} = q(V(\vb{r}_{2}) - V(\vb{r}_{1})).
\end{align*}
Med andra ord är potentialskillnaden mellan två punkter lika med arbetet som måste göras för att förflytta en laddning från ena punkten till den andra per laddning.

\paragraph{Entydighetssats för potentialen}
Betrakta en region $H$ med känd laddningstäthet. Inuti (men inte en del av) $H$ finns tre regioner avgränsade av ytorna $S_{D}$, $S_{N}$ och $S_{Q}$ (med normalvektorerna pekande in mot de avgränsade regionerna). På $S_{D}$ är $V = V_{\text{S}}$ känd. På $S_{N}$ är $\grad_{\vb{n}}{V} = -\vb{E}_{\vb{n}}$ känd. $S_{Q}$ är en perfekt ledare med känd total ytladdning $Q$. Vi vill försöka visa att elektriska fältet är entydigt i $H$.

% TODO: Varför sista?
För att visa detta, antag att vi har två lösningar $V_{1}$ och $V_{2}$ och bilda $V_{0} = V_{1} - V_{2}$. Då vet vi att $\laplacian{V_{0}} = 0$ i $H$, $V_{0} = 0$ på $S_{D}$ och $S_{Q}$, $\grad_{\vb{n}}{V_{0}} = 0$ på $S_{N}$ och $\integ{S_{Q}}{}{S}{\grad_{\vb{n}}{V_{0}}} = \vb{0}$. Detta stämmer eftersom
\begin{align*}
	\integ{S_{Q}}{}{a}{\grad_{\vb{n}}{V_{0}}} = \integ{S_{Q}}{}{a}{\sigma}
\end{align*}
enligt randvillkoret fär elektriska fältet. Vi får därmed
\begin{align*}
	\integ{S_{D} + S_{Q} + S_{N}}{}{a}{V_{0}\grad_{\vb{n}}{V_{0}}} = \integ{H}{}{\tau}{V_{0}\laplacian{V_{0}} + \abs{\grad{V_{0}}}^{2}} = \integ{H}{}{\tau}{\abs{\grad{V_{0}}}^{2}}.
\end{align*}
På $S_{D}$ och $S_{N}$ är en av faktorerna i integranden på vänstersidan lika med $0$, så vi behöver endast betrakta $S_{Q}$. Här har vi att $V_{0}$ är konstant på ytan, vilket ger
\begin{align*}
	\integ{S_{Q}}{}{a}{V_{0}\grad_{\vb{n}}{V_{0}}} = V_{0}\integ{S_{Q}}{}{a}{\grad_{\vb{n}}{V_{0}}}.
\end{align*}
Gauss' lag ger att detta är $0$ (även fast ytan har omvänd orientering, då Gauss' lag ändå bara innehåller laddningen i ytans inre), vilket implicerar
\begin{align*}
	\grad{V} = \vb{E}_{2} - \vb{E}_{1} = \vb{0},
\end{align*}
och beviset är klart.

\paragraph{Randvillkor för potentialen}
För att betrakta randvillkor för potentialen vid en ytladdning, kan man integrera elektriska fältet längs med en rak linje normalt på ytaddningen över ytan och låta linjen bli godtyckligt kort. Då fås randvillkoret för elektriska fältet i termer av potentialen som
\begin{align*}
	\grad_{\vb{n}_{12}}V_{1} - \grad_{\vb{n}_{12}}V_{2} = -\frac{\sigma}{\varepsilon_{0}},
\end{align*}
där $\grad_{\vb{n}_{12}}$ är riktningsderivatan i normalriktningen från område $2$ mot område $1$.

\paragraph{Poissons ekvation}
Gauss' lag på differentialform kombinerad med definitionen av potentialen ger Poissons ekvation
\begin{align*}
	\laplacian{V} = -\frac{\rho}{\varepsilon_{0}}.
\end{align*}

\paragraph{Laplace' ekvation i sfäriska koordinater}
Genom att ställa upp Poissons ekvation i sfäriska koordinater på ett laddningsfritt domän som avgränsas av två sfäriska skal får man Laplace' ekvation. Den har allmän lösning på formen
\begin{align*}
	V = \sum\limits_{l = 0}^{\infty}\sum\limits_{m = -l}^{l}\left(A_{lm}r^{l} + \frac{B_{lm}}{r^{l + 1}}\right)Y_{lm}(\theta, \phi),
\end{align*}
där $Y_{lm}$ är klotytefunktionerna. $A_{lm}$ bestäms av laddningarna utanför det yttre skalet och $B_{lm}$ av laddningarna innanför det inre skalet.

Vi har allmänt att
\begin{align*}
	Y_{lm}(\theta, \phi)\propto P_{l}^{m}(\cos{\theta})e^{im\phi},
\end{align*}
där $P_{l}^{m}$ är Legendrepolynomen. I fall som är rotationssymmetriska med avseende på $xy$-planet kan lösningen därför förenklas till
\begin{align*}
	V = \sum\limits_{l = 0}^{\infty}\left(A_{l}r^{l} + \frac{B_{l}}{r^{l + 1}}\right)P_{l}^{0}(\cos{\theta}).
\end{align*}

\paragraph{Anpassningsmetod}
Betrakta ett fall likt fallet ovan där vi även känner $V$ på $z$-axeln. Eftersom $P_{l}(1) = 1$ och $P_{l}(-1) = (-1)^{l}$ får vi längs $z$-axeln
\begin{align*}
	V = \sum\limits_{l = 0}^{\infty}\left(A_{l}\abs{z}^{l} + \frac{B_{l}}{\abs{z}^{l + 1}}\right)
	\begin{cases}
		1,        &z > 0, \\
		(-1)^{l}, &z < 0.
	\end{cases}
\end{align*}

\paragraph{Elektrostatisk energi}
Vi är nu intresserade av energin som krävs för att skapa en viss laddningsfördelning. Vi kommer därför beräkna energin som krävs för att transportera all laddningen från oändligheten och placera den på rätt sätt.

Vi börjar med att betrakta en samling laddningar som ska ligga på avstånd $r_{ij}$ från varandra. Att placera första laddningen på rätt plats kräver inget arbete. Att sen placera ut andra laddningen kommer kräva att man arbetar mot elektriska fältet från första. Man måste alltså göra ett arbete
\begin{align*}
	W_{2} = \frac{1}{4\pi\varepsilon_{0}}q_{2}\frac{q_{1}}{r_{12}}.
\end{align*}
På samma sätt måste laddning $3$ motarbeta elektriska fältet från både $1$ och $2$. Det totala arbetet är därmed
\begin{align*}
	W = \frac{1}{4\pi\varepsilon_{0}}\sum\limits_{i = 1}^{n}\sum\limits_{j < i}\frac{q_{i}q_{j}}{r_{ij}}.
\end{align*}
Eftersom $r_{ij} = r_{ji}$ kan vi nu skriva om den inre summan genom att i stället summera över alla andra partiklar än $i$, och lägga till en faktor $\frac{1}{2}$. Vi får då
\begin{align*}
	W = \frac{1}{8\pi\varepsilon_{0}}\sum\limits_{i = 1}^{n}\sum\limits_{j \neq i}\frac{q_{i}q_{j}}{r_{ij}}.
\end{align*}
Vid att ordna om faktorerna får vi
\begin{align*}
	W = \frac{1}{2}\sum\limits_{i = 1}^{n}q_{i}\sum\limits_{j \neq i}\frac{q_{j}}{4\pi\varepsilon_{0}r_{ij}} = \frac{1}{2}\sum\limits_{i = 1}^{n}q_{i}V_{i}(\vb{r}_{i}),
\end{align*}
där $V_{i}$ är potentialen som laddning $i$ känner av på grund av alla de andra laddingarna. Ett uttryck man hade kunnat gissa sig fram till från början.

Vi generaliserar vidare vår definition till kontinuerliga laddningfördelningar som
\begin{align*}
	W = \frac{1}{2}\integ{V}{}{\tau}{\rho V},
\end{align*}
där $V$ är en region som innehåller all laddning. Vi kan med hjälp av resultaten från tidigare skriva detta som
\begin{align*}
	W = \frac{1}{2}\integ{V}{}{\tau}{V\varepsilon_{0}\div{\vb{E}}} = \frac{1}{2}\varepsilon_{0}\integ{V}{}{\tau}{\div(V\vb{E}) - \vb{E}\cdot\grad{V}} = \frac{1}{2}\varepsilon_{0}\left(\vinteg{S}{}{a}{V\vb{E}} + \integ{V}{}{\tau}{E^{2}}\right),
\end{align*}
där $S$ är ytan som omsluter $V$.

Om man tittar på den ursprungliga integralen, ger den inget bidrag där det inte finns laddning. Därför kan vi börja med att låta $V$ vara exakt området där det finns laddning. Om vi gör $V$ större, kommer den ursprungliga integralen att vara oändrad. Däremot kommer integralen av elektriska fältets belopp öka, så ytintegralen måste minska motsvarande. Att ytintegralen avtar kan man se eftersom ytan ökar som $r^{2}$, medan fältet avtar som $\frac{1}{R^{2}}$ och potentialen som $\frac{1}{R}$. Vi kan nu repetera processen tills vi integrerar över hela rummet. Då försvinner ytintegralen, och kvar står
\begin{align*}
	W = \frac{1}{2}\varepsilon_{0}\integ{}{}{\tau}{E^{2}}.
\end{align*}

Det visar sig att om man använder resultatet för en laddningsfördelning på en diskret fördelning, får man inte samma svar. Detta är för att uttrycket för en diskret fördelning inte tar hänsyn till energin som krävs för att skapa punktladdningar till att börja med, vilket uttrycket för kontinuerliga fördelningar inkluderar. Denna finessen kom in i beräkningarna i övergången till kontinuerliga fördelningar, eftersom vi för diskreta fördelningar endast använde potentialen varje laddning känner på grund av alla andra. För kontinuerliga fördelningar är detta inte ett problem eftersom varje element har försvinnande liten laddning och därmed bidrar med försvinnande lite potential. För diskreta fördelningar gäller detta ej, dock.

\paragraph{Perfekta ledare}
En perfekt ledare har obegränsat med fria laddningar som kan röra sig i materialet. Från detta följer att
\begin{itemize}
	\item $\vb{E} = \vb{0}$ överallt inuti ledaren. Annars skulle någon av de fria laddningarna påverkas av fältet och röra sig sån att de kansellerade det.
	\item $\rho = \varepsilon_{0}\div{\vb{E}} = 0$ inuti ledaren. Därmed är alla fria laddningar på ytan.
	\item $V$ är konstant inuti ledaren.
	\item $\vb{E} = \frac{\sigma}{\varepsilon_{0}}\ub{n}$ precis utanför ledaren på grund av elektriska fältets randvillkor, alternativt eftersom tangentiella komponenter skulle transportera laddningar på ytan som skulle kansellera fältet.
\end{itemize}

\paragraph{Kraften på en ledare}
Betrakta en perfekt ledare. Randvillkoret för elektriska fältet ger
\begin{align*}
	\vb{E} = \frac{\sigma}{2\varepsilon_{0}}\vb{n}
\end{align*}
precis utanför ytan. Då verkar en krafttäthet
\begin{align*}
	\vb{f}  = \frac{\sigma^{2}}{2\varepsilon_{0}}\ub{n}
\end{align*}
på ytan. Detta motsvarar ett tryck utåt på laddaren - oberoende av vilken sorts ytladdning man har! I termer av elektriska fältet kan trycket skrivas som
\begin{align*}
	p = \frac{1}{2}\varepsilon_{0}E^{2}.
\end{align*}

\paragraph{Kapacitans}
Betrakta först två ledare med olika laddningar $\pm Q$. Man kan se av integraluttrycket för elektriska fältet att det är proportionellt mot $Q$. Eftersom potentialskillnaden mellan ledarna är en integral av elektriska fältet, är även denna proportionell mot $Q$. Vi definierar därmed systemets kapacitans som proportionalitetskonstanten mellan de två, alltså
\begin{align*}
	Q = CV.
\end{align*}
Denna beror enbart av systemets geometri.

Betrakta nu ett system av olika ledare, med var sin potential och laddning. Vi har även någon potentialreferens. Vi kan börja med att sätta alla potentialer förutom en till $0$, och räkna ut alla $Q_{i}$. Vid att superponera alla dina resultat får du en mängd slutgiltiga samband på formen $Q_{i} = C_{ij}V_{j}$. Detta definierar kapacitansmatrisen. Man kan visa/argumentera för att matrisen är symmetrisk och positivt definit. Om man vill visa det, kan man göra en energibetraktning av tillföringen av en extra laddning till systemet, och jämföra energin om man först tillför laddningen och sen tillför ytladdning till ledarna för att höja potentialerna, eller ändrar ordningen, och använder elektrostatiska kraftens konservativitet.

\paragraph{Energin för ett system av ledare}
För ett system av ledare $D_{i}$ har vi
\begin{align*}
	W &= \frac{1}{2}\integ{}{}{\tau}{\rho V} \\
	  &= \frac{1}{2}\sum\integ{D_{i}}{}{a}{\sigma_{i}V_{i}} \\
	  &= \frac{1}{2}\sum V_{i}\integ{D_{i}}{}{a}{\sigma_{i}} \\
	  &= \frac{1}{2}\sum V_{i}Q_{i}.
\end{align*}
Om vi använder kapacitansmatrisen får vi
\begin{align*}
	W = \frac{1}{2}V_{i}C_{ij}V_{j}.
\end{align*}
För en enda kondensator blir detta
\begin{align*}
	W = \frac{1}{2}CV^{2} = \frac{1}{2C}Q^{2}.
\end{align*}

\paragraph{Speglingsmetoder}
Vissa problem kan lösas med speglingsmetoder. Då kan man ersätta vissa komponenter av ett problem med andra på ett sådant sätt att randvillkor som ges i problemet fortfarande är uppfylda.

\paragraph{Spegling och potentialen från två linjeladdningar}
Vi kommer behöva lite standardlösningar för att använda när vi speglar problem. Vi betraktar därför först två oändligt långa parallella linjeladdningar med laddning $\pm\lambda$ per längd separerade med ett avstånd $2h$. Problemet är tvådimensionellt, och vi inför $\vb{s}$ som vektorn från punkten mitt emellan laddningarna till en godtycklig punkt i planet. Elektriska fältet från laddningen till höger, som vi döper nummer $1$, ges av $\vb{E} = \frac{\lambda}{2\pi\varepsilon_{0}\abs{\vb{s} - \vb{s}_{1}}}\ub{s}'$, där $\vb{s}_{1} = h\ub{x}$ och $\ub{s}'$ pekar i samma riktning som $\vb{s} - \vb{s}_{1}$. Vi definierar $V = 0$ där $\vb{s} = 0$ och får
\begin{align*}
	V(\vb{s}) = -\vinteg{\vb{0}}{\vb{s}}{s'}{\frac{\lambda}{2\pi\varepsilon_{0}\abs{\vb{s}' - \vb{s}_{1}}}\ub{s}'} = -\vinteg{-\vb{s}_{1}}{\vb{s} - \vb{s}_{1}}{u'}{\frac{\lambda}{2\pi\varepsilon_{0}u'}\ub{u}'}.
\end{align*}
Vi har rotationssymmetri i planet, och kan därmed välja en radiell riktning för integrationen, vilket ger
\begin{align*}
	V(\vb{s}) = -\frac{\lambda}{2\pi\varepsilon_{0}}\ln{\frac{\abs{\vb{s} - \vb{s}_{1}}}{\abs{-\vb{s}_{1}}}} = -\frac{\lambda}{2\pi\varepsilon_{0}}\ln{\frac{\abs{\vb{s} - \vb{s}_{1}}}{h}}.
\end{align*}
Den totala potentialen är därmed
\begin{align*}
	V(\vb{s}) = \frac{\lambda}{2\pi\varepsilon_{0}}\ln{\frac{\abs{\vb{s} - \vb{s}_{2}}}{\abs{\vb{s} - \vb{s}_{1}}}}.
\end{align*}

Ekvipotentialytorna uppfyller $V = V_{0}$, vilket ger
\begin{align*}
	\frac{\abs{\vb{s} - \vb{s}_{2}}}{\abs{\vb{s} - \vb{s}_{1}}} = e^{\frac{2\pi\varepsilon_{0}V_{0}}{\lambda}}.
\end{align*}
Vi definierar $u = \frac{2\pi\varepsilon_{0}V_{0}}{\lambda}$ och skriver om avstånden på vänstersidan för att få
\begin{align*}
	\frac{(x + h)^{2} + y^{2}}{(x - h)^{2} + y^{2}} = e^{2u}.
\end{align*}
Detta ger
\begin{align*}
	x^{2} + 2xh + h^{2} + y^{2}                                                          &= e^{2u}(x^{2} - 2xh + h^{2} + y^{2}), \\
	x^{2}(1 - e^{2u}) + 2xh(1 + e^{2u}) + y^{2}(1 - e^{2u})                              &= h^{2}(e^{2u} - 1), \\
	x^{2}(e^{-u} - e^{u}) + 2xh(e^{-u} + e^{u}) + y^{2}(e^{-u} - e^{u})                  &= h^{2}(e^{u} - e^{-u}), \\
	-x^{2}\sinh{u} + 2xh\cosh{u} - y^{2}\sinh{u}                                         &= h^{2}\sinh{u}, \\
	x^{2} - 2xh\coth{u} + h^{2} + y^{2}                                                  &= 0, \\
	x^{2} - 2xh\coth{u} + h^{2}\left(\coth^{2}{u} - \frac{1}{\sinh^{2}{u}}\right) + y^{2}          &= 0, \\
	(x - h\coth{u})^{2} + y^{2}                                                          &= \frac{h^{2}}{\sinh^{2}{u}}
\end{align*}
Det är alltså en cirkel med centrum i $h\coth{u}\ub{x}$ och radie $a = \frac{h}{\abs{\sinh{u}}}$. Avstånden från cirkelns centrum till de två laddningarna är
\begin{align*}
	d_{1} = h\abs{\coth{u} - 1},\ d_{2} = h\abs{\coth{u} + 1}.
\end{align*}
Eftersom $\abs{\coth{u}} > 1$, får vi
\begin{align*}
	d_{1}d_{2} = h^{2}(\coth^{2}{u} - 1) = \frac{h^{2}}{\sinh^{2}{u}} = a^{2}.
\end{align*}

Speglingsstrategin är nu att om du har en linjeladdning $\lambda$ parallell med en ledande cirkulärcylindrisk yta med radien $a$ och laddning $-\lambda$ per längd, och avståndet från cylinderaxeln till linjeladdningen är $d$, kan cylinderytan ersättas med en linjeladdning $\lambda_{\text{s}} = -\lambda$ ett avstånd $d_{\text{s}} = \frac{a^{2}}{d}$ från cylinderaxeln.

\paragraph{Spegling och potentialen från två punktladdningar}
Betrakta två punktladdningar liggande på $z$-axeln, där den översta, döpt nummer 1, ligger i punkten $h\ub{z}$ och den andra i origo. Problemet är cylindersymmetriskt, så vi inför cylinderkoordinater. Potentialen är då
\begin{align*}
	V = \frac{1}{4\pi\varepsilon_{0}}\left(\frac{q_{1}}{\sqrt{\rho^{2} + (z - h)^{2}}} + \frac{q_{2}}{\sqrt{\rho^{2} + z^{2}}}\right).
\end{align*}
Ekvipotentialytorna för en nollskild potential är komplicerade, men ekvipotentialytan för $V = 0$ ges av
\begin{align*}
	\frac{q_{1}}{\sqrt{\rho^{2} + (z - h)^{2}}} + \frac{q_{2}}{\sqrt{\rho^{2} + z^{2}}} &= 0, \\
	k = \frac{q_{1}}{q_{2}}                                                             &= -\frac{\sqrt{\rho^{2} + (z - h)^{2}}}{\sqrt{\rho^{2} + z^{2}}}, \\
	\rho^{2}(k^{2} - 1) + k^{2}z^{2}                                                         &= (z - h)^{2}, \\
	z^{2}(1 - k^{2}) - 2zh + h^{2}                                                      &= \rho^{2}(k^{2} - 1), \\
	\rho^{2} + z^{2} - \frac{2zh}{1 - k^{2}} + \frac{h^{2}}{1 - k^{2}}                  &= 0, \\
	\rho^{2} + \left(z - \frac{h}{1 - k^{2}}\right) - \frac{h^{2}}{(1 - k^{2})^{2}} + \frac{h^{2}}{1 - k^{2}} &= 0, \\
	\rho^{2} + \left(z - \frac{h}{1 - k^{2}}\right)                                     &= \frac{h^{2}k^{2}}{(1 - k^{2})^{2}}.
\end{align*}
Detta är en sfär med centrum i $\frac{1}{1 - k^{2}}h\ub{z}$ och radius $a = h\abs{\frac{k}{1 - k^{2}}}$. Notera att det är en förutsättning att $k < 0$, alltså att laddningarna har olika tecken.

Om vi nu antar $k^{2} > 1$ ligger sfärens centrum under laddning $2$. Avstånden från sfärens centrum till de två punktladdningarna är då
\begin{align*}
	d_{2} = \frac{1}{k^{2} - 1}h,\ d_{1} = h + d_{2} = \frac{k^{2}}{k^{2} - 1}h.
\end{align*}
Detta ger
\begin{align*}
	d_{1}d_{2} = a^{2},\ k = -\sqrt{\frac{d_{1}}{d_{2}}} = -\frac{d_{1}}{a} = -\frac{a}{d_{2}}.
\end{align*}

Speglingsstrategin är nu att om du har en punktladdning $q$ ett avstånd $d$ från centrum av en jordad ledande sfärisk yta med radien $a$, kan ledaren ersättas med en punktladdning $q_{s} = -q\frac{a}{d}$ ett avstånd $d_{s} = \frac{a^{2}}{d}$ från sfärens centrum (bort från punktladdningen).

\section{Dipoler och dielektrika}

\paragraph{Elektriska dipolen}
Betrakta två punktladdningar på en linje. Linjen går genom origo, och laddningarna ligger lika långa avstånd $\frac{1}{2}d$ från origo. Potentialen i punkten $\vb{r}$, som ligger ett avstånd $R_{+}$ respektiva $R_{-}$ från de två laddningarna, ges av
\begin{align*}
	V = \frac{q}{4\pi\varepsilon_{0}R_{+}} - \frac{q}{4\pi\varepsilon_{0}R_{-}} = \frac{q}{4\pi\varepsilon_{0}}\frac{R_{-} - R_{+}}{R_{+}R_{-}}.
\end{align*}
Om $r >> d$ fås
\begin{align*}
	V \approx \frac{q}{4\pi\varepsilon_{0}}\frac{d\cos{\theta}}{r^{2}}
\end{align*}
där $\theta$ är vinkeln mellan $\vb{r}$ och linjen. Vi kan då skriva detta som
\begin{align*}
	V = \frac{q}{4\pi\varepsilon_{0}}\frac{\vb{d}\cdot\ub{r}}{r^{2}}.
\end{align*}
Vi definierar nu dipolmomentet $\vb{p} = q\vb{d}$, och får då
\begin{align*}
	V = \frac{\vb{p}\cdot\ub{r}}{4\pi\varepsilon_{0}r^{2}}.
\end{align*}

Notera att vi kan skriva dipolmomentet som $\vb{p} = \sum q_{i}\vb{r}_{i}$.

\paragraph{Ideala dipoler}
En ideal dipol fås i gränsen för en dipol när $d$ blir oändligt liten på ett sådant sätt att $\vb{p}$ hålls konstant.

\paragraph{Fältet från en dipol}
Fältet från en dipol ges av
\begin{align*}
	\vb{E} = -\grad{\frac{\vb{p}\cdot\ub{r}}{4\pi\varepsilon_{0}r^{2}}} = -\grad{\frac{\vb{p}\cdot\vb{r}}{4\pi\varepsilon_{0}r^{3}}}.
\end{align*}
I nämnaren har vi
\begin{align*}
	\vb{p}\cdot\vb{r} = p_{i}r_{i} \implies \del{i}{\vb{p}\cdot\vb{r}} = p_{i} \implies \grad{\vb{p}\cdot\vb{r}} = \vb{p}.
\end{align*}
Vi får då
\begin{align*}
	\vb{E} = -\frac{1}{4\pi\varepsilon_{0}}\left(\frac{\vb{p}}{r^{3}} - \frac{3\vb{p}\cdot\vb{r}}{r^{4}}\grad{r}\right) = \frac{1}{4\pi\varepsilon_{0}r^{3}}\left(3(\vb{p}\cdot\ub{r})\ub{r} - \vb{p}\right).
\end{align*}

\paragraph{Dipolmoment för en laddningsfördelning}
För en laddningsfördelning ges dipolmomentet av
\begin{align*}
	\vb{p} = \integ{}{}{V}{\vb{r}\rho}.
\end{align*}

\paragraph{Förflyttning av koordinatsystem och dipolmoment}
Vid förflyttning av origo en sträcka $\vb{a}$ fås
\begin{align*}
	\vb{p}' = \integ{}{}{V}{\vb{r}'\rho} = \integ{}{}{V}{(\vb{r} - \vb{a})\rho} = \vb{p} - q\vb{a}.
\end{align*}
Alltså beror dipolmomentet av origos position om det finns en netto mängd laddning i systemet.

\paragraph{Multipolutveckling}
Betrakta en punkt $\vb{r}$ och en annan punkt $\vb{r}'$. Vid att definiera $\vb{R} = \vb{r} - \vb{r}'$ får vi att om de två punkterna inte är samma, är $\laplacian{\frac{1}{R}} = 0$ överallt förutom där $\vb{R} = \vb{0}$. Vid att lägga vårat koordinatsystem så att $\vb{r}'$ är parallell med $z$-axeln och definiera vinkeln mellan $\vb{r}$ och $\vb{r}'$ som $\gamma$ fås
\begin{align*}
	\frac{1}{R} =
	\begin{cases}
		\sum\limits_{l = 0}^{\infty}A_{l}r^{l}P_{l}(\cos{\gamma}),          &r < r', \\
		\sum\limits_{l = 0}^{\infty}\frac{B_{l}}{r^{l}}P_{l}(\cos{\gamma}), &r > r'.
	\end{cases}
\end{align*}
Speciellt, på $z$-axeln är $\gamma = 0$ och
\begin{align*}
	\frac{1}{R} = \frac{1}{r_{>} - r_{<}} = \frac{1}{r_{>}}\left(1 - \frac{r_{<}}{r_{>}}\right)^{-1} = \frac{1}{r_{>}}\sum\limits_{l = 0}^{\infty}\left(\frac{r_{<}}{r_{>}}\right)^{l} = \sum\limits_{l = 0}^{\infty}\frac{r_{<}^{l}}{r_{>}^{l + 1}},
\end{align*}
där $r_{<} = \min(r', r)$ och $r_{>} = \max(r', r)$. Vi utvidgar därmed lösningen till
\begin{align*}
	\frac{1}{R} = \sum\limits_{l = 0}^{\infty}\frac{r_{<}^{l}}{r_{>}^{l + 1}}P_{l}(\cos{\gamma}).
\end{align*}

Vi söker nu potentialen utanför en sfär som omsluter en rumladdning. Vid att låta $\vb{r}$ peka utanför sfären och $\vb{r}'$ inuti fås $r_{<} = r',\ r_{>} = r$ och
\begin{align*}
	V = \frac{1}{4\pi\varepsilon_{0}}\integ{}{}{V'}{\frac{\rho}{R}} = \frac{1}{4\pi\varepsilon_{0}}\sum\limits_{l = 0}^{\infty}\frac{1}{r^{l + 1}}\integ{}{}{V'}{(r')^{l}\rho P_{l}(\cos{\gamma})} = \sum\limits_{l = 0}^{\infty}V_{l}.
\end{align*}
De olika $V_{l}$ kommer ge oss termer som ser ut som olika multipoler, och vi vill nu studera dem. Vi noterar först att om $e_{i}$ respektiva $e_{i}'$ är komponenterna av $\ub{r}$ respektiva $\ub{r}'$, kan vi skriva
\begin{align*}
	\cos(\gamma) &= e_{i}e_{i}',\ \cos[2](\gamma) = e_{i}e_{i}'e_{i}e_{j}',\ 1 = e_{i}e_{i} = e_{i}e_{j}\delta_{ij}.
\end{align*}

För $l = 0$ får vi
\begin{align*}
	V_{0} = \frac{1}{4\pi\varepsilon_{0}}\frac{1}{r}\integ{}{}{V'}{\rho},
\end{align*}
alltså ett bidrag motsvarande en punktladdning med samma totala laddning i origo. För $l = 1$ fås
\begin{align*}
	V_{1} &= \frac{1}{4\pi\varepsilon_{0}}\frac{1}{r^{2}}\integ{}{}{V'}{r'\rho P_{1}(\cos{\gamma})} \\
	      &= \frac{1}{4\pi\varepsilon_{0}r^{2}}\integ{}{}{V'}{r'\cos{\gamma}\rho} \\
	      &= \frac{1}{4\pi\varepsilon_{0}r^{2}}\integ{}{}{V'}{r'e_{i}e_{i}'\rho} \\
	      &= \frac{e_{i}}{4\pi\varepsilon_{0}r^{2}}\integ{}{}{V'}{r_{i}'\rho} \\
	      &= \frac{1}{4\pi\varepsilon_{0}r^{2}}\ub{r}\cdot\integ{}{}{V'}{r_{i}'\rho}
\end{align*}
alltså ett bidrag motsvarande en dipol med samma totala dipolmoment i origo. För $l = 2$ fås
\begin{align*}
	V_{2} &= \frac{1}{4\pi\varepsilon_{0}}\frac{1}{r^{3}}\integ{}{}{V'}{r'^{2}\rho P_{2}(\cos{\gamma})} \\
	      &= \frac{1}{8\pi\varepsilon_{0}r^{3}}\integ{}{}{V'}{r_{i}'r_{i}'\rho(3\cos[2](\gamma) - 1)} \\
	V_{2} &= \frac{1}{8\pi\varepsilon_{0}r^{3}}\integ{}{}{V'}{r_{k}'r_{k}'\rho(3\cos[2](\gamma) - 1)}  \\
	      &= \frac{1}{8\pi\varepsilon_{0}r^{3}}\integ{}{}{V'}{r_{k}'r_{k}'\rho(3e_{i}e_{i}'e_{i}e_{j}' - e_{i}e_{j}\delta_{ij})} \\
	      &= \frac{1}{8\pi\varepsilon_{0}r^{3}}e_{i}e_{j}\integ{}{}{V'}{r_{k}'r_{k}'\rho(3e_{i}'e_{j}' - \delta_{ij})} \\
	      &= \frac{1}{8\pi\varepsilon_{0}r^{3}}e_{i}e_{j}\integ{}{}{V'}{\rho(3r_{i}'r_{j}' - (r')^{2}\delta_{ij})}.
\end{align*}
Vi definierar nu kvadrupolmomentstensorn
\begin{align*}
	Q_{ij} = \frac{1}{2}\integ{}{}{V'}{\rho(3r_{i}'r_{j}' - (r')^{2}\delta_{ij})},
\end{align*}
vilket ger
\begin{align*}
	V = \frac{e_{i}Q_{ij}e_{j}}{4\pi\varepsilon_{0}r^{3}}.
\end{align*}
Detta är kvadrupolbidraget.

Allmänt blir det $l$:te bidraget på formen
\begin{align*}
	V_{l} = \frac{Q_{i_{1}\dots i_{l}}e_{i_{1}}\dots e_{i_{l}}}{4\pi\varepsilon_{0}r^{l + 1}}.
\end{align*}

\paragraph{Additionssatsen}
Hellre än att hantera komponenterna av kvadrupolmomentstensorn, använder vi en sats som säjer
\begin{align*}
	P_{l}(\cos{\gamma}) = \frac{4\pi}{2l + 1}\sum\limits_{m = -l}^{l}Y_{lm}(\theta, \phi)Y_{lm}^{*}(\theta', \phi').
\end{align*}
Då kan vi skriva
\begin{align*}
	V = \sum\limits_{l = 0}^{\infty}V_{l},\ V_{l} = \frac{1}{(2l + 1)\varepsilon_{0}r^{l + 1}}\sum\limits_{m = -l}^{l}q_{lm}Y_{lm}(\theta, \phi),
\end{align*}
där $q_{lm}$ är det sfäriska multipolmomentet
\begin{align*}
	q_{lm} = \integ{}{}{V*}{\rho (r')^{l}})Y_{lm}^{*}(\theta', \phi').
\end{align*}
Varje $V_{l}$ har alltså $2l + 1$ oberoende komponenter, vilket på grund av multipolmomenttensorernas symmetri och spårlöshet är lika med antalet oberoende komponenter i multipolmomenttensorn.

\paragraph{Kraft på en laddningsfördelning}
Betrakta en laddningsfördelning i ett yttre måttligt varierande elektriskt fält. Kraften på laddningsfördelningen ges då av
\begin{align*}
	\vb{F} = \integ{}{}{V}{\rho\vb{E}}.
\end{align*}
Vi vill nu approximera elektriska fältet i laddningsfördelningen vid att utveckla den kring en referenspunkt $\vb{r}_{0}$. Komponentvis har vi
\begin{align*}
	E_{i} \approx E_{i, 0} + (r_{j} - r_{j, 0})\del{j}{E_{i}},
\end{align*}
varför
\begin{align*}
	\vb{E} \approx \vb{E}_{0} + ((\vb{r} - \vb{r}_{0})\cdot\grad)\vb{E}
\end{align*}
och
\begin{align*}
	\vb{F} \approx \integ{}{}{V}{\rho\left(\vb{E}_{0} + ((\vb{r} - \vb{r}_{0})\cdot\grad)\vb{E}\right)}.
\end{align*}
Vidare, under antagandet att fältet varierar måttligt kan alla derivator antas vara konstanta, vilket ger
\begin{align*}
	\vb{F} \approx \integ{}{}{V}{\rho\vb{E}_{0}} + \integ{}{}{V}{\rho((\vb{r} - \vb{r}_{0})\cdot\grad)\vb{E}} = Q\vb{E}_{0} + (\vb{p}\cdot\grad)\vb{E},
\end{align*}
där dipolmomentet mäts relativ $\vb{r}_{0}$. Att dra ut elektriska fältet i andra termen från integralen är helt okej - man kan tänka sig att integrationen skapar en operator som sedan får verka på fältet.

\paragraph{Vridmoment på en laddningsfördelning}
På en punktladdning har vi
\begin{align*}
	\vb{M} = \vb{r}\times\vb{F} = q\vb{r}\times\vb{E}.
\end{align*}
För en laddningsfördelning fås då, relativt en referenspunkt $O$, vridmomentet
\begin{align*}
	\vb{M} &= \integ{}{}{V}{\rho\vb{r}\times\vb{E}} \\
	       &\approx \integ{}{}{V}{\rho((\vb{r} - \vb{r}_{0}) + \vb{r}_{0})\times(\vb{E}_{0} + ((\vb{r} - \vb{r}_{0})\cdot\grad)\vb{E})}.
\end{align*}
Vi försummar nu termer av andra ordningen i $\vb{r} - \vb{r}_{0}$ och får
\begin{align*}
	\vb{M} &= \integ{}{}{V}{\rho((\vb{r} - \vb{r}_{0})\times\vb{E}_{0} + \vb{r}_{0}\times(\vb{E}_{0} + ((\vb{r} - \vb{r}_{0})\cdot\grad)\vb{E}))} \\
	       &= \left(\integ{}{}{V}{\rho((\vb{r} - \vb{r}_{0})}\right)\times\vb{E}_{0} + \vb{r}_{0}\times\integ{}{}{V}{\vb{E}_{0} + ((\vb{r} - \vb{r}_{0})\cdot\grad)\vb{E})} \\
	       &= \vb{p}\times\vb{E}_{0} + \vb{r}_{0}\times\vb{F}.
\end{align*}

\paragraph{Polarisation}
Polarisationen $\vb{P}$ uppfyller
\begin{align*}
	\vb{p} = \integ{}{}{V}{\vb{P}}.
\end{align*}
Detta fältet uppfyller inga speciella randvillkor, och är allmänt ej rotationsfritt.

\paragraph{Polariserbarhet}
Polariserbarheten $\alpha$ uppfyller
\begin{align*}
	\vb{p} = \alpha\vb{E}.
\end{align*}

\paragraph{Fält och potential från polarisation}
Från ett litet rymdelement fås
\begin{align*}
	\dd{\vb{p}} = \vb{P}\dd{V},\ \dd{V} = \frac{1}{4\pi\varepsilon_{0}}\frac{\dd{\vb{p}}\cdot\ub{R}}{R^{2}},\ V = \frac{1}{4\pi\varepsilon_{0}}\integ{}{}{V}{\frac{\vb{P}\cdot\ub{R}}{R^{2}}}.
\end{align*}
Elektriska fältet kommer ha en singularitet som beter sig som $\frac{1}{R^{3}}$. Denna kommer inte upphävas av volymelementet, och kräver specialbehandling.

Vi kommer dock lösa detta genom att använda att
\begin{align*}
	\grad'{\frac{1}{R}} = \grad'{\frac{1}{\abs{\vb{r} - \vb{r}'}}} = \frac{1}{R^{2}}\ub{R},
\end{align*}
vilket ger
\begin{align*}
	V &= \frac{1}{4\pi\varepsilon_{0}}\integ{}{}{V}{\vb{P}\cdot\grad'{\frac{1}{R}}} \\
	  &= \frac{1}{4\pi\varepsilon_{0}}\integ{}{}{V}{\div'{\frac{\vb{P}}{R^{2}}} - \frac{1}{R}\div'{\vb{P}}} \\
	  &= \frac{1}{4\pi\varepsilon_{0}}\vinteg{}{}{S}{\frac{\vb{P}}{R}} - \frac{1}{4\pi\varepsilon_{0}}\integ{}{}{V}{\frac{1}{R}\div'{\vb{P}}},
\end{align*}
vilket motsvara coulombpotentialen från två ekvivalenta laddningsfördelningar
\begin{align*}
	\rho_{\text{b}} = -\div'{\vb{P}},\ \sigma_{\text{b}} = \vb{P}\cdot\ub{n}.
\end{align*}
Vi kalla dessa för bundna laddningsfördelningar. Nu kan vi beräkna elektriska fältet som
\begin{align*}
	\vb{E} = \frac{1}{4\pi\varepsilon_{0}}\integ{}{}{S}{\frac{\sigma_{\text{b}}}{R^{2}}\ub{R}} + \frac{1}{4\pi\varepsilon_{0}}\integ{}{}{V}{\rho_{\text{b}}\frac{1}{R^{2}}\ub{R}}.
\end{align*}

\paragraph{Introduktion av D-fältet}
Gauss' lag på integralform ger oss nu
\begin{align*}
	\div{\vb{E}} = \frac{\rho}{\varepsilon_{0}} = \frac{-\div'{\vb{P}} + \rho_{\text{f}}}{\varepsilon_{0}},
\end{align*}
där $\rho_{\text{f}}$ är den fria laddningstätheten. Detta implicerar
\begin{align*}
	\div(\varepsilon_{0}\vb{E} + \vb{P}) = \rho_{\text{f}},
\end{align*}
vilket uppmanar oss att definiera
\begin{align*}
	\vb{D} = \varepsilon_{0}\vb{E} + \vb{P}.
\end{align*}

\paragraph{Randvillkor för D-fältet}
Vi kan visa att
\begin{align*}
	D_{1}^{\perp} - D_{2}^{\perp} = \sigma_{\text{b}},\ \vb{D}_{1}^{\parallel} - \vb{D}_{2}^{\parallel} = \vb{P}_{1}^{\parallel} - \vb{P}_{2}^{\parallel}.
\end{align*}

\paragraph{Linjära dielektrika}
Linjära dielektrika uppfyller
\begin{align*}
	\vb{P} = \varepsilon_{0}\chi_{\text{e}}\vb{E}
\end{align*}
för måttliga fältstyrkor. $\chi_{\text{e}}$ är dielektrikumets elektriska susceptibilitet.

\paragraph{D-fält i linjära dielektrika}
I ett linjärt dielektrikum fås
\begin{align*}
	\vb{D} = \varepsilon_{0}(1 + \chi_{\text{e}})\vb{E} = \varepsilon_{0}\varepsilon_{\text{r}}\vb{E} = \varepsilon\vb{E}.
\end{align*}
$\varepsilon_{\text{r}}$ är dielektrikumets relativa permittivitet, och $\varepsilon$ är dets permittivitet.

\paragraph{Energi för linjära dielektrika}
I linjära dielektrika tillkommer även arbete för att polarisera dielektrikumet. Om man betraktar en atom som en punktladdning $q$ och en laddning $-q$ jämnt fördelad över ett klot med radie $a$, skapar det negativt laddade molnet ett elektriskt fält
\begin{align*}
	\vb{E}_{-q} = -\frac{q}{4\pi\varepsilon_{0}a^{3}}\vb{r},\ r < a.
\end{align*}
Kraften på laddningen i mitten blir då
\begin{align*}
	\vb{F} = -\frac{q^{2}}{4\pi\varepsilon_{0}a^{3}}\vb{r} = -\frac{q^{2}}{4\pi\varepsilon_{0}a^{3}}\grad{\frac{1}{2}r^{2}}.
\end{align*}
Arbetet som krävs för att förflytta laddningen i mitten från centrum blir då
\begin{align*}
	W = \frac{1}{2}\frac{q^{2}r^{2}}{4\pi\varepsilon_{0}a^{3}} = -\frac{1}{2}\vb{p}\cdot\vb{E}_{-q}.
\end{align*}

Om nu atomen är i ett yttre elektriskt fält $\vb{E}$, kommer det elektriska fältet att förflytta laddningen i centrum. Jämvikt fås när $\vb{E} = -\vb{E}_{-q}$, och arbetet som görs för att förflytta laddningen i mitten är då
\begin{align*}
	W = \frac{1}{2}\vb{p}\cdot\vb{E}.
\end{align*}

Med detta argumentet i bakgrunden ställer vi upp energin i ett dielektrikum som
\begin{align*}
	W = \frac{1}{2}\integ{}{}{V}{\vb{P}\cdot\vb{E}}.
\end{align*}
I tillägg har vi det övriga bidraget för att skapa det elektriska fältet, och totala energin ges av
\begin{align*}
	W = \frac{1}{2}\integ{}{}{V}{(\varepsilon_{0}\vb{E} + \vb{P})\cdot\vb{E}} = \frac{1}{2}\integ{}{}{V}{\vb{D}\cdot\vb{E}}.
\end{align*}

% TODO: Griffiths argument
% TODO: Which is greater, why

\paragraph{Krafter på ett dielektrikum}
Betrakta ett dielektrikum någonstans i närheten av två perfekta ledare med laddning $\pm Q$ på de respektiva. Ledarna kan antingen vara isärkopplade eller kopplade ihop med ett batteri som upprätthåller en konstant spänningsskillnad $U$ mellan dem. Dielektrikumets position beskrivs av dets geometri och en referensvektor $\vb{r}$ (till exempel till dets geometriska centrum). Att beräkna elektriska fältet är allmänt svårt, men vi ska försöka undveka detta med ett trick.

Betrakta första fallet först. Om dielektrikumet förflyttas en sträcka $\dd{\vb{r}}$, gör elektriska fältet från ledarna ett arbete på det, som nödvändigtvis måste balanseras av ett mekaniskt arbete. Arbetet som görs på systemet är
\begin{align*}
	\dd{W} = \grad{W}\cdot\dd{\vb{r}} = -\dd{W_{\text{e}}} = -\vb{F}_{\text{e}}\cdot\dd{\vb{r}}.
\end{align*}
Samtidigt kan vi skriva systemets energi $W$ i termer av parameterna i systemets beskrivning. Därmed är kraften på dielektrikumet
\begin{align*}
	\vb{F}_{\text{e}} = -\grad{W} = -\grad{\frac{1}{2}\frac{Q^{2}}{C}} = \frac{1}{2}\frac{Q^{2}}{C^{2}}\grad{C} = \frac{1}{2}V^{2}\grad{C}.
\end{align*}

Betrakta nu det andra fallet. Om spänningen mellan ledarna hålls konstant, kommer translation av dielektrikumet behöva balansera arbetet både från translation i elektriska fältet och arbetet som krävs för att transportera laddning mellan ledarna för att hålla spänningsskillnaden mellan dem konstant. Vi får 
\begin{align*}
	\dd{W} = U\dd{Q} - \dd{W_{\text{e}}} = U\dd{Q} - \vb{F}_{\text{e}}\cdot\dd{\vb{r}}.
\end{align*}
Samtidigt ges vänstersidan av $\dd{W} = \frac{1}{2}U\dd{Q}$, så $U\dd{Q} = 2\dd{W} = 2\grad{W}\cdot\dd{\vb{r}}$, vilket ger
\begin{align*}
	\vb{F}_{\text{e}} = \grad{W} = \frac{1}{2}U^{2}\grad{C}.
\end{align*}

\paragraph{Moment på ett dielektrikum}
Vid att göra en liknande analys som den ovan fås
\begin{align*}
	\vb{N}_{\text{e}} = \pm\del{\phi}{W_{\text{e}}}\ub{z},
\end{align*}
där $+$ och $-$ är fallen där $Q$ respektiva $U$ är konstant.

\section{Magnetostatik}

\paragraph{Ström}
Ström definieras som $I = \dv{Q}{t}$.

\paragraph{Kraft på strömslingor}
Betrakta två slingor $C$ och $C'$. Genom varje slinga går en ström $I$ respektiva $I'$ i samma riktning som kurvans orientering. Experiment har visat att kraften på $C$ från $C'$ ges av
\begin{align*}
	\vb{F} = \vinteg{C}{}{\vb{r}}{I\times\frac{\mu_{0}I'}{4\pi}\vinteg{C'}{}{\vb{r}'}{\times\frac{1}{R^{2}}\ub{R}}}
\end{align*}
där $\vb{R} = \vb{r} - \vb{r}'$.

\paragraph{Magnetiska fältet}
Genom att definiera det magnetiska fältet från $C'$ i punkten $\vb{r}$ som
\begin{align*}
	\vb{B} = \frac{\mu_{0}I'}{4\pi}\vinteg{C'}{}{\vb{r}'}{\times\frac{1}{R^{2}}\ub{R}}
\end{align*}
fås
\begin{align*}
	\vb{F} = \vinteg{C}{}{\vb{r}}{I\times\vb{B}}.
\end{align*}

\paragraph{Ytströmtäthet och rymdströmtäthet}
För strömmar fördelade i rummet eller på en yta kan vi utvidga definitionen av magnetiska fältet till att bli
\begin{align*}
	\vb{B} = \frac{\mu_{0}}{4\pi}\vinteg{}{}{S}{\vb{J}\times\frac{1}{R^{2}}\ub{R}}\ \text{eller}\ \vb{B} = \frac{\mu_{0}}{4\pi}\vinteg{}{}{V}{\vb{J}\times\frac{1}{R^{2}}\ub{R}},
\end{align*}
där $\vb{J}$ är den lokala strömtätheten.

\section{Magnetiska dipoler och magneter}

\paragraph{Magnetiskt dipolmoment för en slinga}
Betrakta en strömslinga $C$ som för en ström $I$. Vi söker vektorpotentialen på stort avstånd från slingan. Det exakta uttrycket ges av
\begin{align*}
	\vb{A} = \frac{\mu_{0}}{4\pi}\integ{C}{}{\vb{l}'}{\frac{I}{R}}.
\end{align*}
Om $C$ omkransar en yta $S$, fås
\begin{align*}
	\vb{A} = \frac{\mu_{0}I}{4\pi}\tinteg{S}{}{a}{\gradp{\frac{1}{R}}} = \frac{\mu_{0}I}{4\pi}\tinteg{S}{}{a}{\frac{1}{R^{2}}\ub{R}}.
\end{align*}
Vid stora avstånd fås
\begin{align*}
	\vb{A} = \frac{\mu_{0}}{4\pi}\left(I\integ{S}{}{\vb{a}}{}\right)\times\frac{1}{r^{2}}\ub{r} = \frac{\mu_{0}}{4\pi r^{2}}\vb{m}\times\ub{r},
\end{align*}
där vi har definierat det magnetiska dipolmomentet
\begin{align*}
	\vb{m} = I\integ{S}{}{\vb{a}}{} = I\vb{S}.
\end{align*}
Igen har vi definierat slingans vektorarea $\vb{S}$.

Hur ska man välja vektorarean? Tänk dig att $C$ är randen till två olika ytor $S_{1}$ och $S_{2}$, som till sammans innesluter området $V$. Detta ger
\begin{align*}
	\vb{S}_{1} - \vb{S}_{2} = \integ{S_{1}}{}{\vb{a}}{} - \integ{S_{2}}{}{\vb{a}}{} = \integ{S_{1} + S_{2}}{}{\vb{a}}{} = \integ{V}{}{\tau}{\grad{1}} = \vb{0},
\end{align*}
och därmed spelar inte valet roll.

\paragraph{Magnetiskt dipolmoment för en allmän strömtäthet}
För att studera en allmän strömtäthet, vill vi dela den upp i slingor. Betrakta då konen med spets i origo, rand $C$ och $\vb{r}$, som ligger på $C$, som generatris. För denna är ytelementet $\dd{\vb{a}} = \frac{1}{2}\vb{r}\times\dd{\vb{r}}$, vilket ger
\begin{align*}
	\vb{m} = I\vb{S} = I\integ{S}{}{\vb{a}}{} = \frac{1}{2}\int\limits_{C}{\vb{r}'\times I\dd{\vb{l}'}}.
\end{align*}
Vi generaliserar detta genom att ersätta $I\dd{\vb{l}'}$ med $\vb{J}\dd{\tau'}$ och integrera över resultatet för varje litet slingelement för att få
\begin{align*}
	\vb{m} = \frac{1}{2}\int{\dd{\tau'}\vb{r}'\times \vb{J}}.
\end{align*}

\paragraph{Dipolmomentets beroende av origo}
Om vi förflyttar vårat koordinatsystem fås
\begin{align*}
	\vb{m}_{O} = \frac{1}{2}\integ{}{}{\tau'}{(\vb{r}' - \vb{r}_{O})\times\vb{J}} = \vb{m} - \frac{1}{2}\vb{r}_{O}\times\integ{}{}{\tau'}{\vb{J}}.
\end{align*}
Vi har
\begin{align*}
	\integ{}{}{\tau'}{J_{i}} = \integ{}{}{\tau'}{\vb{J}\cdot\gradp{r_{i}'}} = \integ{}{}{\tau}{\divp(r_{i}'\vb{J}) - r_{i}'\divp{\vb{J}}}.
\end{align*}
Vi använder nu att vi arbetar med magnetostatik för att ta bort sista termen, vilket ger
\begin{align*}
	\integ{}{}{\tau'}{J_{i}} = \integ{}{}{\tau'}{\divp(r_{i}'\vb{J})} = \vinteg{}{}{a}{r_{i}'\vb{J}}.
\end{align*}
Slingan förutsätts vara ändlig, och då behöver vi bara integrera över en yta som inneslutar den. På den ytan är $\vb{J} = \vb{0}$, vilket ger att integralen av komponenten blir $0$, och slutligen
\begin{align*}
	\vb{m}_{O} = \vb{m}.
\end{align*}

\paragraph{Magnetiska fältet från en dipol}
Vi får
\begin{align*}
	\vb{B} &= \curl{\vb{A}} \\
	       &= \frac{\mu_{0}}{4\pi}\curl(\frac{1}{r^{2}}\vb{m}\times\ub{r}) \\
	       &= \frac{\mu_{0}}{4\pi}\left(\grad{\frac{1}{r^{3}}}\times(\vb{m}\times\vb{r}) + \frac{1}{r^{3}}\curl(\vb{m}\times\vb{r})\right) \\
	       &= \frac{\mu_{0}}{4\pi}\left(-\frac{3}{r^{4}}\grad{r}\times(\vb{m}\times\vb{r}) + \frac{1}{r^{3}}\curl(\vb{m}\times\vb{r})\right) \\
	       &= \frac{\mu_{0}}{4\pi}\left(-\frac{3}{r^{4}}\ub{r}\times(\vb{m}\times\vb{r}) + \frac{1}{r^{3}}(\vb{m}\div{\vb{r}}  - (\vb{m}\cdot\grad)\vb{r})\right) \\
	       &= \frac{1}{r^{3}}\frac{\mu_{0}}{4\pi}\left(-3\ub{r}\times(\vb{m}\times\ub{r}) + \frac{1}{r^{3}}(3\vb{m}  - \vb{m})\right) \\
	       &= \frac{1}{r^{3}}\frac{\mu_{0}}{4\pi}\left(-3(\ub{r}\cdot\ub{r})\vb{m} + 3\times(\vb{m}\cdot\ub{r})\ub{r} + 2\vb{m}\right) \\
	       &= \frac{1}{r^{3}}\frac{\mu_{0}}{4\pi}\left(3(\vb{m}\cdot\ub{r})\ub{r} - \vb{m}\right).
\end{align*}

\paragraph{Kraft på en magnetisk dipol}
Kraften på en strömslinga $C$ ges av
\begin{align*}
	\vb{F} &= \int\limits_{C}{I\dd{\vb{l}}\times\vb{B}} \\
	       &= I\int\limits_{S}{(\dd{\vb{a}}\times\grad')\times\vb{B}} \\
	       &= I\integ{S}{}{a}{\grad'(\vb{B}\cdot\ub{n}) - \ub{n}\div{\vb{B}})}.
\end{align*}
Andra termen är $0$. Första termen ger komponentvis
\begin{align*}
	F_{i} = I\integ{S}{}{a}{\delp{i}{(B_{j}n_{j})}}.
\end{align*}
Vi antar att magnetfältet varierar måttligt över slingan, och approximerar derivatornas värde i mittpunkten, varför den faktorn kan tas utanför integralen. Detta ger
\begin{align*}
	F_{i} = I\del{i}{B_{j}}\integ{S}{}{a_{j}}{} = \del{i}{(B_{j}Ia_{j})},
\end{align*}
och slutligen
\begin{align*}
	\vb{F} = \grad(\vb{m}\cdot\vb{B}).
\end{align*}

Det här var oklart, så vi testar ett annat sätt: Skriv
\begin{align*}
	\vb{B} = \vb{B}_{0} + ((\vb{r} - \vb{r}_{0})\cdot\grad)\vb{B},
\end{align*}
där derivatan av $\vb{B}$ antas vara konstant. Detta ger
\begin{align*}
	\vb{B} &= I\tinteg{C}{}{l}{\vb{B}_{0} + ((\vb{r} - \vb{r}_{0})\cdot\grad)\vb{B}} \\
	       &= I\left(\integ{C}{}{\vb{l}}{}\right)\times\vb{B}_{0} + I\left(\tinteg{C}{}{l}{(\vb{r}\cdot\grad)\vb{B}}\right) - I\left(\integ{C}{}{\vb{l}}{}\right)\times(\vb{r}_{0}\cdot\grad)\vb{B} \\
	       &= I\left(\tinteg{C}{}{l}{(\vb{r}\cdot\grad)\vb{B}}\right).
\end{align*}
På indexform fås
\begin{align*}
	F_{i} &= I\integ{C}{}{l_{j}}{\varepsilon_{ijk}[(\vb{r}\cdot\grad)\vb{B}]_{k}} \\
	      &= I\integ{C}{}{l_{j}}{\varepsilon_{ijk}r_{m}\del{m}{B_{k}}} \\
	      &= \varepsilon_{ijk}I\del{m}{B_{k}}\integ{V}{}{l_{j}}{r_{m}} \\
	      &= \varepsilon_{ijk}I\del{m}{B_{k}}\integ{S}{}{a_{b}}{\varepsilon_{jbc}\del{c}{r_{m}}} \\
	      &= \varepsilon_{ijk}\varepsilon_{jbc}\delta_{cm}I\del{m}{B_{k}}\integ{S}{}{a_{b}}{} \\
	      &= \varepsilon_{ijk}\varepsilon_{cjb}m_{b}\del{c}{B_{k}} \\
	      &= (\delta_{ic}\delta_{kb} - \delta_{ib}\delta_{kc})m_{b}\del{c}{B_{k}} \\
	      &= m_{k}\del{i}{B_{k}} - m_{i}\del{k}{B_{k}} \\
	      &= \del{i}{m_{k}B_{k}} - m_{i}\div{\vb{B}},
\end{align*}
vilket slutligen ger
\begin{align*}
	\vb{F} = \grad(\vb{m}\cdot\vb{B}).
\end{align*}

\paragraph{Vridmoment på en slinga}
Vridmomentet ges av
\begin{align*}
	\vb{M} &= \int\limits_{C}{\vb{r}\times\dd{\vb{F}}} \\
	       &= I\int\limits_{C}{\vb{r}\times(\dd{\vb{l}}\times\vb{B})} \\
	       &= I\int\limits_{C}{(\vb{B}\cdot\vb{r})\dd{\vb{l}} - (\vb{r}\cdot\dd{\vb{l}})\vb{B}}.
\end{align*}
Vi approximerar fältet till att vara konstant och lika med fältet i mitten, vilket ger
\begin{align*}
	\vb{B}\int\limits_{C}{(\vb{r}\cdot\dd{\vb{l}})} = \vb{B}\vinteg{S}{}{\vb{a}}{\curl{\vb{r}}} = \vb{0}.
\end{align*}
Detta ger
\begin{align*}
	\vb{M} &= I\integ{C}{}{\vb{l}}{\vb{B}\cdot\vb{r}} \\
	       &= I\tinteg{S}{}{a}{\grad(\vb{B}\cdot\vb{r})} \\
	       &= I\tinteg{S}{}{a}{\vb{B}} \\
	       &= \vb{m}\times\vb{B}.
\end{align*}

\paragraph{Magnetisering}
Magnetiseringen $\vb{M}$ uppfyller
\begin{align*}
	\vb{m} = \integ{}{}{\tau'}{\vb{M}}.
\end{align*}

% TODO: Relevant?
I en atom har elektronerna omloppstid
\begin{align*}
	T = \frac{2\pi r}{v},
\end{align*}
vilket ger strömmen
\begin{align*}
	\vb{I} = -\frac{e}{T}\ub{\phi} = -\frac{ev}{2\pi r}\ub{\phi}.
\end{align*}
Dessa ger då magnetiska momentet
\begin{align*}
	\vb{m} = \frac{1}{2}\int{\vb{r}\times I\dd{\vb{r}}} = -\frac{1}{2}erv\ub{z}.
\end{align*}

\paragraph{Vektorpotential från magnetisering}
Vi har
\begin{align*}
	\vb{A} = \frac{\mu_{0}}{4\pi}\integ{}{}{\tau'}{\frac{1}{R^{2}}\vb{M}\times\ub{R}}.
\end{align*}
Detta kan skrivas som
\begin{align*}
	\vb{A} &= \frac{\mu_{0}}{4\pi}\integ{}{}{\tau'}{\vb{M}\times\gradp{\frac{1}{R}}} \\
	       &= \frac{\mu_{0}}{4\pi}\integ{}{}{\tau'}{\frac{1}{R}\curlp{\vb{M}} - \curlp{\left(\frac{1}{R}\vb{M}\right)}} \\
	       &= \frac{\mu_{0}}{4\pi}\integ{}{}{\tau'}{\frac{1}{r}\curlp{\vb{M}}} - \frac{\mu_{0}}{4\pi}\tinteg{}{}{a'}{\frac{1}{R}\vb{M}}.
\end{align*}
Detta motsvarar magnetiska fältet från en volymström
\begin{align*}
	\vb{J}_{\text{bv}} = \curlp{\vb{M}}
\end{align*}
och en ytström
\begin{align*}
	\vb{J}_{\text{bs}} = \vb{M}\times\ub{n}.
\end{align*}

\paragraph{Multipolutveckling av vektorpotentialen}
Vi har
\begin{align*}
	\frac{1}{R} = \sum\limits_{l = 0}^{\infty}\frac{r'^{l}}{r^{l + 1}}P_{l}(\cos{\theta'}),
\end{align*}
vilket för en strömslinga ger
\begin{align*}
	\vb{A} &= \frac{\mu_{0}}{4\pi}\integ{}{}{\vb{l}æ}{\frac{I}{R}} \\
	       &= \frac{\mu_{0}I}{4\pi}\sum\limits_{l = 0}^{\infty}\integ{}{}{\vb{l}'}{\frac{r'^{l}}{r^{l + 1}}P_{l}(\cos{\theta'})} \\
	       &= \frac{\mu_{0}I}{4\pi}\sum\limits_{l = 0}^{\infty}\frac{1}{r^{l + 1}}\integ{}{}{\vb{l}'}{r'^{l}P_{l}(\cos{\theta'})}.
\end{align*}
Den första termen ges av
\begin{align*}
	\vb{A}_{0} = \frac{\mu_{0}I}{4\pi}\frac{1}{r^{1}}\integ{}{}{\vb{l}'}{P_{0}(\cos{\theta'})} = \frac{\mu_{0}I}{4\pi}\frac{1}{r^{1}}\integ{}{}{\vb{l}'}{} = \vb{0}
\end{align*}
för en sluten strömslinga. Inte oförväntad, då vi inte känner till magnetiska monopoler. Den andra termen ges av
\begin{align*}
	\vb{A}_{1} &= \frac{\mu_{0}I}{4\pi}\frac{1}{r^{2}}\integ{}{}{\vb{l}'}{r'^{1}P_{1}(\cos{\theta'})} \\
	           &= \frac{\mu_{0}I}{4\pi r^{2}}\integ{}{}{\vb{l}'}{r'\cos{\theta'}} \\
	           &= \frac{\mu_{0}I}{4\pi r^{2}}\integ{}{}{\vb{l}'}{\vb{r}'\cdot\ub{r}} \\
	           &= -\frac{\mu_{0}}{4\pi r^{2}}\ub{r}\times I\integ{}{}{\vb{a}'}{},
\end{align*}
vilket motsvarar en dipolterm. Vidare skulle man kunna skriva upp kvadrupoltermen också.

\paragraph{$H$-fältet}
Ampères lag ger
\begin{align*}
	\curl{\vb{B}} = \mu_{0}\vb{J}.
\end{align*}
Med resultatet ovan fås
\begin{align*}
	\curl{\vb{B}}                           &= \mu_{0}\vb{J}_{\text{fri}} + \mu_{0}\curl{\vb{M}}, \\
	\curl(\frac{1}{\mu_{0}}\vb{B} - \vb{M}) &= \vb{J}_{\text{fri}}.
\end{align*}
Vi definierar då
\begin{align*}
	\vb{H} = \frac{1}{\mu_{0}}\vb{B} - \vb{M},
\end{align*}
vilket ger
\begin{align*}
	\curl{\vb{H}} = \vb{J}_{\text{fri}}.
\end{align*}

\paragraph{Amperes lag för $H$-fältet}
Vi får
\begin{align*}
	I_{\text{fri}} = \vinteg{}{}{l}{\vb{H}}.
\end{align*}

\paragraph{Linjära magnetiserbara material}
Betrakta ett material som uppfyller
\begin{align*}
	\vb{M} = \frac{1}{\mu{0}}\chi_{\text{m}}\vb{B}.
\end{align*}
Dessa uppfyller
\begin{align*}
	\vb{B} &= \mu_{0}(\vb{H} + \vb{M}), \\
	\vb{M} &= \chi_{\text{m}}(\vb{H} + \vb{M}), \\
	\vb{M} &= \frac{\chi_{\text{m}}}{1 - \chi_{\text{m}}}\vb{H} = \chi_{\text{m}}^{H}\vb{H}.
\end{align*}
Detta ger slutligen
\begin{align*}
	\vb{B} = \mu_{0}(1 + \chi_{\text{m}}^{H})\vb{H} = \mu_{0}\mu_{\text{r}}\vb{H} = \mu\vb{H},
\end{align*}
där $\mu_{\text{r}} = 1 + \chi_{\text{m}}^{H}$ kallas den relativa permeabiliteten och $\mu$ kallas permeabiliteten.

\paragraph{Klassificering av magnetiska material}
Det finns olika sorters magnetism i material. Bland dessa är:
\begin{itemize}
	\item diamagnetiska material, med $\chi_{\text{m}} < 0$, typiskt kring \num{-1e-5}.
	\item paramagnetiska material, med $\chi_{\text{m}} > 0$, typiskt kring \num{1e-4}.
	\item ferromagnetiska material, med $\chi_{\text{m}} >> 1$. Dessa är dock ofta icke-linjära.
\end{itemize}

\paragraph{Randvillkor för $H$-fältet}
I en gränsyta fås
\begin{align*}
	(\vb{H}_{1} - \vb{H}_{2})\cdot\vb{n}_{12} = (\vb{M}_{2} - \vb{M}_{1})\cdot\vb{n}_{12},\ \vb{n}_{12}\times(\vb{H}_{1} - \vb{H}_{2}) = \mu_{0}\vb{J}_{\text{f}}.
\end{align*}

\paragraph{Ömsesidig induktans}
Betrakta två strömslingor $C_{1}$ och $C_{2}$. En ström $I_{1}$ ger ett magnetiskt flöde $\Phi_{21}$ genom $C_{2}$. Vi har att
\begin{align*}
	\Phi_{12} = \vinteg{C_{2}}{}{l_{2}}{\vb{A}} = \vinteg{C_{2}}{}{l_{2}}{\frac{\mu_{0}I_{1}}{4\pi}\integ{C_{1}}{}{\vb{l}_{1}}{\frac{1}{R}}} = M_{21}I_{1},
\end{align*}
där $\vb{R} = \vb{r}_{2} - \vb{r}_{1}$ och $M_{21}$ är $C_{2}$:s ömsesidiga induktans från $C_{1}$. Med andra ord:
\begin{align*}
	M_{21} = \frac{\mu_{0}}{4\pi}\vinteg{C_{2}}{}{l_{2}}{\integ{C_{1}}{}{\vb{l}_{1}}{\frac{1}{R}}}.
\end{align*}
Det gäller att $M_{12} = M_{21}$.

\paragraph{Egeninduktans}
Om det går en ström i en slinga induceras även ett magnetiskt flöde från slingas egna fält. Vi döper denna $M_{11} = L_{1}$, och har $\Phi_{11} = L_{1}I_{1}$.

\paragraph{Induktansmatris}
Om man har ett problem med $n$ strömslingor, får man
\begin{align*}
	\Phi_{i} = \sum\limits_{j}M_{ij}I_{j},
\end{align*}
som kan skrivas som ett matrisproblem som involverar induktansmatrisen $M$.

\paragraph{Krafter på magneter}
% TODO: Beräkna

\section{Lite om elektronik}

\paragraph{Ledningsförmåga}
Fria laddningar kan transporteras av krafter per laddning. Om det i ett ämne finns en krafttäthet $\vb{f}$ ger denna alltså upphov till en inducerad strömtäthet. Vi antar att denna är linjär ock skriver
\begin{align*}
	J_{i} = \sigma_{ij}f_{j},
\end{align*}
där $\sigma$ är ledningsförmågan. Den har enhet \si{\siemens\per\meter} eller \si{\per\ohm\per\meter}. En perfekt ledare har oändlig (och diagonal) sådan.

\paragraph{Resistans och Ohms lag}
Betrakta ett batteri (en ideal spänningskälla) som är kopplad til två ideala ledare. De ideala ledarna har potential $V_{+}$ respektiva $V_{-}$, och är förbundna av ett material med ledningsförmåga $\sigma$. I det ledande området är $\vb{J} = \sigma\vb{E}$. Om kontaktytan mellan det ledande området och ledarna är $S_{+}$ respektiva $S_{-}$ fås
\begin{align*}
	U = V_{+} - V_{-} = -\vinteg{S_{-}}{S_{+}}{\vb{r}}{\vb{E}},\ I = \vinteg{S_{-}}{}{\vb{a}}{\vb{J}} = -\vinteg{S_{+}}{}{\vb{a}}{\vb{J}}.
\end{align*}
Vägintegralen för spänningen kan tas för att vara mellan två godtyckliga punkter på ytorna. Vi definierar (av någon anledning) nu det ledande områdets resistans som
\begin{align*}
	R = \frac{U}{I}.
\end{align*}

\paragraph{Randvillkor mellan ledare}
För att bevara $\div{\vb{J}} = 0$ fås kravet
\begin{align*}
	\vb{n}_{12}\cdot(\vb{J}_{1} - \vb{J}_{2}) = 0.
\end{align*}
Randvillkoret för elektriska fältet ger även
\begin{align*}
	\vb{n}_{12}\times\left(\sigma_{1}^{-1}\vb{J}_{1} - \sigma_{2}^{-1}\vb{J}_{2}\right) = \vb{0}.
\end{align*}

\paragraph{Effektutveckling och Joules lag}
Betrakta en punktladdning $q$. På denna uträtter det elektriska och magnetiska fältet arbetet
\begin{align*}
	\dd{W} = \vb{F}\cdot\dd{\vb{r}} = q(\vb{E} + \vb{v}\times\vb{B})\cdot\dd{\vb{r}} = q\vb{E}\cdot\dd{\vb{r}}.
\end{align*}
Då tillförs $q$ effekten
\begin{align*}
	P = q\vb{v}\cdot\vb{E}.
\end{align*}

Betrakta nu en volymström $\vb{J} = \rho\vb{v}$. Ett litet element i detta tillförs effekten
\begin{align*}
	\dd{P} = \rho\dd{\tau}\vb{v}\cdot\vb{E} = \dd{\tau}\vb{J}\cdot\vb{E}.
\end{align*}
Vi kan nu införa effekttätheten
\begin{align*}
	p = \vb{J}\cdot\vb{E}.
\end{align*}
Denna integreras över ledarens volym för att ge
\begin{align*}
	P &= \integ{}{}{\tau}{p} \\
	  &= \integ{}{}{\tau}{\vb{J}\cdot\vb{E}} \\
	  &= -\integ{}{}{\tau}{\vb{J}\cdot\grad{V}} \\
	  &= -\integ{}{}{\tau}{\div(V\vb{J}) - V\div{\vb{J}}} \\
	  &= -\vinteg{}{}{a}{V\vb{J}}.
\end{align*}
Om det ledande området är omringad av en isolator har $\vb{J}$ ingen normalkomponent där, och
\begin{align*}
	P &= -\vinteg{S_{+}}{}{a}{V\vb{J}} - \vinteg{S_{-}}{}{a}{V\vb{J}} \\
	  &= (V_{+} - V_{-})I,
\end{align*}
alltså
\begin{align*}
	P = UI.
\end{align*}

\paragraph{Elektromotorisk kraft}
Betrakta en smal resistiv slinga med tvärsnitt $A$ som beskrivs av en naturlig rymdkoordinat $l$. Om slingan för strömmen $I$ fås
\begin{align*}
	\vb{J} = \frac{I}{A}\ub{l} = \sigma\vb{f}.
\end{align*}
Detta ger
\begin{align*}
	\vb{f}\cdot\dd{\vb{l}} = \frac{I}{A}\sigma^{-1}\dd{l}.
\end{align*}
Om vi jämför detta med resultatet för ett cylindriskt motstånd fås
\begin{align*}
	\vb{f}\cdot\dd{\vb{l}} = I\dd{R}.
\end{align*}
Resistansen hos ett kort segment i slingan är alltså
\begin{align*}
	\dd{R} \frac{\dd{l}}{A\sigma}.
\end{align*}
Vi får nu
\begin{align*}
	\vinteg{}{}{l}{\vb{f}} = I\integ{}{}{R}{} = IR.
\end{align*}
För att få en ström krävs det alltså att
\begin{align*}
	\emf = \vinteg{}{}{l}{\vb{f}}
\end{align*}
är nollskild. Vi definierar $\emf$ som den elektromotoriska kraften.

\section{Grunderna i elektrodynamik}

\paragraph{Lite om kvasistatiska fält}
Vi kommer här betrakta kvasistatiska fall, alltså fall där $\rho$ och $\vb{J}$ ändras långsamt. I såna fall kommer vi försöka bilda en dynamisk teori genom att extrapolera Biot-Savarts och Coulombs lagar trivialt, fast nu med integration av tidsberoende källor. Notera att det totala elektriska fältet även får en extra term från ändringen av magnetiska fältet, som vi kommer se.

\paragraph{Elektromagnetisk induktion}
Betrakta en strömslinga $C$ som rör sig godtyckligt och ändrar form under en liten tid $\dd{t}$. Vi är nu intresserade av ändringen i magnetiska flödet
\begin{align*}
	\dv{\Phi}{t} = \dv{t}\vinteg{S(t)}{}{a}{\vb{B}} = \lim\limits_{\dd{t}\to 0}\frac{1}{\dd{t}}\left(\vinteg{S(t + \dd{t})}{}{a}{\vb{B}(t + \dd{t})} - \vinteg{S(t)}{}{a}{\vb{B}(t)}\right).
\end{align*}
Vi serieutvecklar magnetfältet med avseende på tid (den följande variationen av koordinater tas med i det faktum att vi integrerar över olika ytor) för att få
\begin{align*}
	\dv{\Phi}{t} \approx \lim\limits_{\dd{t}\to 0}\frac{1}{\dd{t}}\left(\vinteg{S(t + \dd{t})}{}{a}{\vb{B}(t)} - \vinteg{S(t)}{}{a}{\vb{B}(t)} + \dd{t}\vinteg{S(t + \dd{t})}{}{a}{\del{t}{\vb{B}}(t)}\right).
\end{align*}
Låt nu kurvan vid $t$ och $t + \dd{t}$ förbindas av ytan $S_{\text{o}}$. Då fås
\begin{align*}
	\vinteg{S(t + \dd{t})}{}{a}{\vb{B}(t)} - \vinteg{S(t)}{}{a}{\vb{B}(t)} + \vinteg{S_{\text{o}}}{}{a}{\vb{B}(t)} = \integ{V}{}{\tau}{\div{\vb{B}}} = 0,
\end{align*}
där vi har lagt på ett minustecken för att ändra orienteringen på ena ytan. Om varje punkt på $C$ rör sig med en hastighet $\vb{v}$ fås
\begin{align*}
	\vinteg{S(t + \dd{t})}{}{a}{\vb{B}(t)} - \vinteg{S(t)}{}{a}{\vb{B}(t)} = -\integ{C(t)}{}{}{\vb{B}\cdot(\dd{\vb{l}}\times\vb{v}\dd{t})}.
\end{align*}
Vi har
\begin{align*}
	\vb{B}\cdot(\dd{\vb{l}}\times\vb{v}) = B_{i}\varepsilon_{ijk}\dd{l}_{j}v_{k} = \dd{l}_{j}\varepsilon_{jki}v_{k}B_{i} = \dd{\vb{l}}\cdot(\vb{v}\times\vb{B}).
\end{align*}
Detta ger slutligen
\begin{align*}
	\dv{\Phi}{t} = \vinteg{S(t + \dd{t})}{}{a}{\del{t}{\vb{B}}(t)} - \int\limits_{C(t)}{\dd{\vb{l}}\cdot(\vb{v}\times\vb{B})}.
\end{align*}

\paragraph{Rörlig slinga i statiskt fält}
Betrakta fallet då en slinga rör sig i ett statiskt fält. Vi får då
\begin{align*}
	\int\limits_{C(t)}{\dd{\vb{l}}\cdot(\vb{v}\times\vb{B})} = -\dv{\Phi}{t}.
\end{align*}
Vi känner igen vänstra termen som elektromotoriska kraften, alltså linjeintegralen av kraft per laddning över slingan, och får
\begin{align*}
	\emf = -\dv{\Phi}{t}.
\end{align*}

\paragraph{Varierande magnetiskt fält och fältlagar}
Betrakta en statisk strömslinga i ett varierande magnetiskt fält. Michael Faraday upptäckte att även en sån upplever en kraft. Hans hypotes var att detta berodde på att det inducerades en elektromotorisk spänning på grund av ett elektriskt fält. Mer specifikt,
\begin{align*}
	\vinteg{}{}{l}{\vb{E}} = -\vinteg{S}{}{a}{\del{t}{\vb{B}}(t)}.
\end{align*}

Maxwell generaliserade detta genom att ta bort den fysikaliska slingan och i stället betrakta en integrationsbana i rummet. Om Faradays hypotes stämde, skulle Stokes' sats ge
\begin{align*}
	\vinteg{S}{}{a}{\left(\curl{\vb{E}} + \del{t}{\vb{B}}(t)\right)} = 0.
\end{align*}
Om integrationsbanan är godtycklig, ger det
\begin{align*}
	\curl{\vb{E}} + \del{t}{\vb{B}}(t) = \vb{0}.
\end{align*}
Detta är en dynamisk fältlag för det elektriska och magnetiska fältet.

\paragraph{Elektriska fältets nya term}
Vi ser att elektriska fältet har nollskild rotation i dynamiken, så vi delar upp det i en rotationsfri del och en divergensfri del. Om vi nu tittar på den divergensfria delen, fås
\begin{align*}
	\vb{E} = -\frac{1}{4\pi}\integ{}{}{V}{\del{t}{\vb{B}}\times\frac{1}{R^{2}}\ub{R}}
\end{align*}
i analogi med Biot-Savarts lag. Den rotationsfria har redan beskrivits.

\paragraph{Potentialer för dynamiska fält}
Vi har för den rotationsfria terman att
\begin{align*}
	\curl{\vb{E}} + \del{t}{\vb{B}}(t) = \curl(\vb{E} + \del{t}{\vb{A}}) = \vb{0}.
\end{align*}
Detta ger
\begin{align*}
	\div(\vb{E} + \del{t}{\vb{A}}) = 0,\ \curl(\vb{E} + \del{t}{\vb{A}}) = \vb{0},
\end{align*}
och därmed måste
\begin{align*}
	\vb{E} = -\del{t}{\vb{A}}.
\end{align*}
Med generaliseringen av $\vb{A}$ fås
\begin{align*}
	\vb{E} = -\frac{\mu_{0}}{4\pi}\integ{}{}{\tau'}{\frac{1}{R}\del{t}{\vb{J}}}.
\end{align*}
Vi ser då att i dynamiken ges sambandet mellan fält och potentialer av
\begin{align*}
	\vb{B} = \curl{\vb{A}},\ \vb{E} = -\grad{V} - \del{t}{\vb{A}}.
\end{align*}

\paragraph{EMK från induktans}
Vi får i ett system med $n$ statiska slingor
\begin{align*}
	\emf_{i} = -\sum\limits_{j}M_{ij}\dv{I_{j}}{t} = -L_{i}\dv{I_{i}}{t} + \emf_{\text{övriga}}.
\end{align*}
Vi kan ställa upp detta med hjälp av egeninduktansen som
\begin{align*}
	\dv{I_{i}}{t} + \frac{R_{i}I_{i}}{L_{i}} = \frac{\emf_{\text{övriga}}}{L_{i}}.
\end{align*}

\paragraph{Magnetisk energi}
Den magnetiska energin $W_{\text{m}}$ är arbetet som krävs för att starta en strömkälla $\vb{J}$. Magnetiska fältet gör inget arbete i statiska fall, och därmed kan vi endast betrakta magnetisk energi i dynamiken.

Under igångsättningen finns ett inducerat elektriskt fält $\vb{E} = -\del{t}{\vb{A}}$. För att upprätthålla $\vb{J}$ mot det elektriska fältet tillförs effekten
\begin{align*}
	P = -\integ{}{}{\tau}{\vb{E}\cdot\vb{J}} = \integ{}{}{\tau}{\vb{J}\cdot\del{t}{\vb{A}}}.
\end{align*}
Med Ampères lag skriver vi detta som
\begin{align*}
	P &= \frac{1}{\mu_{0}}\integ{}{}{\tau}{(\curl{\vb{B}})\cdot\del{t}{\vb{A}}} \\
	  &= \frac{1}{\mu_{0}}\integ{}{}{\tau}{\div(\vb{B}\times\del{t}{\vb{A}}) + \vb{B}\cdot\curl(\del{t}{\vb{A}})} \\
	  &= \frac{1}{\mu_{0}}\vinteg{}{}{a}{\vb{B}\times\del{t}{\vb{A}}} + \frac{1}{\mu_{0}}\integ{}{}{\tau}{\vb{B}\cdot\del{t}{\vb{B}}}.
\end{align*}
När vi utvidgar integrationsvolymen mot oändligheten försvinner ytintegralen, och kvar står
\begin{align*}
	P = \frac{1}{2\mu_{0}}\dv{t}\integ{}{}{\tau}{B^{2}}.
\end{align*}
Därmed fås
\begin{align*}
	W_{\text{m}} = \frac{1}{2\mu_{0}}\integ{}{}{\tau}{B^{2}},
\end{align*}
som alternativt kan skrivas som
\begin{align*}
	W_{\text{m}} = \frac{1}{2}\integ{}{}{\tau}{\vb{J}\cdot\vb{A}}.
\end{align*}

\paragraph{Magnetisk energi i material}

\paragraph{Magnetisk energi för en slinga med tjocklek}
Vi delar strömslingan i små delar $V_{i}$ med tvärsnitt $S_{i}$. Magnetiska energin ges av
\begin{align*}
	W_{\text{m}} &= \frac{1}{2}\sum\limits_{i}\integ{V_{i}}{}{\tau}{\vb{J}_{i}\cdot\vb{A}}.
\end{align*}
Varje element för en ström $\vb{J}_{i}$, och bidrar då med fält $\vb{A}_{i}$ respektiva $\vb{B}_{i}$. Detta ger
\begin{align*}
	\frac{1}{2}\sum\limits_{i}\sum\limits_{j}\integ{V_{i}}{}{\tau}{\vb{J}_{i}\cdot\vb{A}_{j}} = \sum\limits_{i}\sum\limits_{j}W_{\text{m}, ij}.
\end{align*}
Egenenergierna $W_{\text{m}, ii}$ motsvarar uttrycken vi har härlett innan, alltså integralen av kvadratet av något bidrag till magnetiska fältet. Detta sättet är att föredra framför att räkna på flödet i slingor (vilket också skulle kunna funka om man delar upp strömtätheten i skivor).

Om vi vidare antar att $\vb{A}$ är ungefär konstant över varje tvärsnitt fås
\begin{align*}
	\vb{J}_{i}\dd{\tau} = \vb{J}_{i}S_{i}\cdot\dd{vb{l}} = I_{i}\dd{\vb{l}}.
\end{align*}
Därmed kan vi skriva
\begin{align*}
	W_{\text{m}, ij} = \frac{1}{2}I_{i}\vinteg{C_{i}}{}{l}{\vb{A}_{j}} = \frac{1}{2}I_{i}\Phi_{ij} = \frac{1}{2}I_{i}M_{ij}I_{j}.
\end{align*}
För systemet fås då
\begin{align*}
	W_{\text{m}} = \frac{1}{2}\sum\limits_{i}\sum\limits_{j}I_{i}M_{ij}I_{j}.
\end{align*}
Vi vet att detta är strikt positivt, så induktansmatrisen måste vara positivt definit.

\paragraph{Induktanser från energitermer}
Med detta kan vi skriva
\begin{align*}
	M_{ij} = \frac{2W_{\text{m}, ij}}{I_{i}I_{j}},\ L_{i} = \frac{2W_{\text{m}, ii}}{I_{i}^{2}}.
\end{align*}

\paragraph{Krafter på magneter}
Betrakta en magnet någonstans i närheten av en strömtäthet $\vb{J}$. Magnetens position beskrivs av någon referensvektor $\vb{r}$.

Betrakta magneten först. Om det förflyttas en sträcka $\dd{\vb{r}}$, görs ett arbete $\dd{W}$ av kraften $\vb{F}$. Kraftjämvikten ger
\begin{align*}
	\vb{F} + \vb{F}_{\text{m}} = \vb{0},
\end{align*}
och därmed
\begin{align*}
	\dd{W} = \vb{F}\cdot\dd{\vb{r}} = -\vb{F}_{\text{m}}\cdot\dd{\vb{r}},\ \vb{F}_{\text{m}} = -\grad{W}.
\end{align*}

Om systemet är isolerad får man att upplagrad magnetisk energi är enda energikällan som finns, vilket ger $\dd{W} = \dd{W}_{\text{m}}$ och
\begin{align*}
	\vb{F}_{\text{m}} = -\grad{W_{\text{m}}}.
\end{align*}
Jag tror att detta motsvarar att alla magnetiska flöden bevaras. Om detta är sant får man
\begin{align*}
	\vb{F}_{\text{m}} = \frac{1}{2}\frac{\Phi^{2}}{L^{2}}\grad{L} = \frac{1}{2}I^{2}\grad{L}.
\end{align*}

Om systemet i stället är anslutet till strömkällor så att alla strömmar hålls konstanta fås
\begin{align*}
	\grad{W_{\text{m}}} = \frac{1}{2}I^{2}\grad{L}.
\end{align*}
Samtidigt gör källan ett arbete
\begin{align*}
	\dd{W} = -I\dd{\Phi} + \dd{W_{\text{m}}}
\end{align*}
för att upprätthålla konstanta strömmar. Eftersom $\Phi = IL$ drar vi slutsatsen att $W = -W_{\text{m}}$, vilket ger
\begin{align*}
	\vb{F}_{\text{m}} = \grad{W_{\text{m}}} = \frac{1}{2}I^{2}\grad{L}.
\end{align*}

\paragraph{Generalisering av Ampères lag}
I statiska situationer har vi sett att Ampères lag är konsekvent med att strömmarna är divergensfria. I dynamiska situationer är inte strömmarna nödvändigtvis divergensfria, så vi kommer försöka göra en ny ad hoc dynamisk Ampères lag.

Genom att utgå från vår generalisering av Biot-Savarts lag fås
\begin{align*}
	\curl{\vb{B}} &= \frac{\mu_{0}}{4\pi}\integ{}{}{\tau'}{\curl(\vb{J}\times\frac{1}{R^{2}}\ub{R})} \\
	              &= \frac{\mu_{0}}{4\pi}\integ{}{}{\tau'}{\div(\frac{1}{R^{2}}\ub{R})\vb{J} - (\vb{J}\cdot\grad)\frac{1}{R^{2}}\ub{R}} \\
	              &= \mu_{0}\vb{J} + \frac{\mu_{0}}{4\pi}\integ{}{}{\tau'}{(\vb{J}\cdot\grad')\frac{1}{R^{2}}\ub{R}}.
\end{align*}
Vi betecknar den andra termen som $\vb{I}$. Komponentvis har vi
\begin{align*}
	I_{i} &= \frac{\mu_{0}}{4\pi}\integ{}{}{\tau'}{J_{j}\del{j}{\frac{1}{R^{3}}R_{i}}} \\
	      &= \frac{\mu_{0}}{4\pi}\integ{}{}{\tau'}{\del{j}{\left(\frac{1}{R^{3}}J_{j}R_{i}\right)} - \frac{1}{R^{3}}R_{i}\del{j}{J_{j}}} \\
	      &= \frac{\mu_{0}}{4\pi}\integ{}{}{a_{i}'}{\left(\frac{1}{R^{3}}J_{j}R_{i}\right)} - \frac{\mu_{0}}{4\pi}\integ{}{}{\tau'}{\frac{1}{R^{3}}R_{i}\div{\vb{J}}}.
\end{align*}
Genom att utvidga integrationsområdet mot oändligheten försvinner första termen, vilket ger
\begin{align*}
	\curl{\vb{B}} &= \mu_{0}\vb{J} - \frac{\mu_{0}}{4\pi}\integ{}{}{\tau'}{\div{\vb{J}}\frac{1}{R^{2}}\ub{R}}.
\end{align*}
Med kontinuitetsekvationen fås
\begin{align*}
	\curl{\vb{B}} &= \mu_{0}\vb{J} + \frac{\mu_{0}}{4\pi}\integ{}{}{\tau'}{\del{t}{\rho}\frac{1}{R^{2}}\ub{R}}.
\end{align*}
Vi känner igen andra termen som proportionell mot en tidsderivata av elektriska fältet, och en möjlig kandidat till en ny Ampères lag är
\begin{align*}
	\curl{\vb{B}} - \mu_{0}\varepsilon_{0}\del{t}{\vb{E}} = \mu_{0}\vb{J}.
\end{align*}

Observera att detta inte är en härledning, då vi har utgått från dessa generaliserade varianterna av Coulombs och Biot-Savarts lagar. Dessa uppfyller till exempel inte $\curl{\vb{E}} = -\del{t}{\vb{B}}$. För att uppfylla detta måste induktiva korrektionstermen läggas till, men då kommer inte den nya Ampères lag att gälla, så man får en jobbig iterativ process med korrektioner.

\paragraph{Maxwells ekvationer}
All vår kunnskap om elektrostatik- och dynamik slås nu ihop för att skriva ned de ekvationerna som beskriver det vi vet:
\begin{align*}
	\curl{\vb{E}} + \del{t}{\vb{B}} = \vb{0}, \\
	\div{\vb{B}} = 0, \\
	\curl{\vb{B}} - \mu_{0}\varepsilon_{0}\del{t}{\vb{E}} = \mu_{0}\vb{J}, \\
	\div{\vb{E}} = \frac{\rho}{\varepsilon_{0}}.
\end{align*}
Dessa kallas för Maxwells ekvationer.

\paragraph{Poyntings sats}
Betrakta en volym $V$ som omkransas av en yta $S$. Joules lag ger att $V$ tillförs effekten
\begin{align*}
	P_{\text{mek}} = \integ{V}{}{}{\vb{J}\cdot\vb{E}}.
\end{align*}
Vi kan tolka detta som
\begin{align*}
	\vb{J}\cdot\vb{E} = \del{t}{w_{\text{mek}}},
\end{align*}
där $w_{\text{mek}}$ är den mekaniska energitätheten. Med hjälp av Maxwells ekvationer skriver vi
\begin{align*}
	\del{t}{w_{\text{mek}}} &= \frac{1}{\mu_{0}}\vb{E}\cdot\curl{\vb{B}} - \varepsilon_{0}\vb{E}\cdot\del{t}{\vb{E}} \\
	                        &= \frac{1}{\mu_{0}}\left(\vb{B}\cdot\curl{\vb{B}} - \div(\vb{E}\times\vb{B})\right) - \frac{1}{2}\varepsilon_{0}\cdot\del{t}{E^{2}} \\
	                        &= -\frac{1}{\mu_{0}}\div(\vb{E}\times\vb{B}) - \del{t}{\left(\frac{1}{2}\varepsilon_{0}\cdot\del{t}{E^{2}} + \frac{1}{2\mu_{0}}B^{2}\right)}.
\end{align*}
Vi har en fältenergitäthet från två av termerna, som vi förstår väl, men det finns även en extra term här. Den är en flödestäthet av energi. Vi kallar den Poyntings vektor
\begin{align*}
	\vb{S} = \frac{1}{\mu_{0}}\vb{E}\times\vb{B}.
\end{align*}
Den uppfyller konserveringslagen
\begin{align*}
	\div{\vb{S}} + \del{t}{(w_{\text{mek}} + w_{\text{em}})} = 0.
\end{align*}

\paragraph{Rörelsemängd i elektromagnetism}
Betrakta ett föremål med laddningstäthet $\rho$ och strömtäthet $\vb{J}$. Detta har rörelsemängd som kommer från laddningarnas rörelse. Om vi inför rörelsemängdstätheten $\vb{g}_{\text{mek}}$ ger Newtons andra lag
\begin{align*}
	\del{t}{\vb{g}_{\text{mek}}} = \rho\vb{E} + \vb{J}\times\vb{B}.
\end{align*}
Maxwells ekvationer ger
\begin{align*}
	\del{t}{\vb{g}_{\text{mek}}} = \varepsilon_{0}\vb{E}(\div{\vb{E}}) + \frac{1}{\mu_{0}}(\curl{\vb{B}})\times\vb{B} - \varepsilon_{0}\del{t}{\vb{E}}\times\vb{B}.
\end{align*}
Vi skriver med hjälp av Maxwells ekvationer
\begin{align*}
	(\curl{\vb{B}})\times\vb{B} &= -\frac{1}{2}\grad{B^{2}} + (\vb{B}\cdot\grad)\vb{B}, \\
	\del{t}{\vb{E}}\times\vb{B} &= \del{t}{(\vb{E}\times\vb{B})} - \vb{E}\times\del{t}{\vb{B}} \\
	                            &= \del{t}{(\vb{E}\times\vb{B})} + \vb{E}\times(\curl{\vb{E}}) \\
	                            &= \del{t}{(\vb{E}\times\vb{B})} + \frac{1}{2}\grad{E^{2}} - (\vb{E}\cdot\grad)\vb{E},
\end{align*}
vilket, till sammans med Maxwells ekvationer, ger
\begin{align*}
	\del{t}{\vb{g}_{\text{mek}}} &= \varepsilon_{0}\vb{E}(\div{\vb{E}}) + \frac{1}{\mu_{0}}\left(-\frac{1}{2}\grad{B^{2}} + (\vb{B}\cdot\grad)\vb{B}\right) - \varepsilon_{0}\left(\del{t}{(\vb{E}\times\vb{B})} + \frac{1}{2}\grad{E^{2}} - (\vb{E}\cdot\grad)\vb{E}\right) \\
	                             &= \varepsilon_{0}\vb{E}(\div{\vb{E}}) + \frac{1}{\mu_{0}}(\vb{B}\cdot\grad)\vb{B} - \varepsilon_{0}\left(\del{t}{(\vb{E}\times\vb{B})} - (\vb{E}\cdot\grad)\vb{E}\right) - \frac{1}{2}\grad(\varepsilon_{0}E^{2} + \frac{1}{\mu_{0}}B^{2}) \\
	                             &= \varepsilon_{0}(\vb{E}(\div{\vb{E}}) + (\vb{E}\cdot\grad)\vb{E}) + \frac{1}{\mu_{0}}(\vb{B}\div{\vb{B}} + (\vb{B}\cdot\grad)\vb{B}) - \grad(\frac{1}{2}\varepsilon_{0}E^{2} + \frac{1}{2\mu_{0}}B^{2}) - \varepsilon_{0}\mu_{0}\del{t}{\left(\frac{1}{\mu_{0}}\vb{E}\times\vb{B}\right)} \\
	                             &= \varepsilon_{0}(\vb{E}(\div{\vb{E}}) + (\vb{E}\cdot\grad)\vb{E}) + \frac{1}{\mu_{0}}(\vb{B}\div{\vb{B}} + (\vb{B}\cdot\grad)\vb{B}) - \grad(w_{\text{e}} + w_{\text{m}}) - \varepsilon_{0}\mu_{0}\del{t}{\vb{S}}.
\end{align*}
Notera att vi adderade en term som är $\vb{0}$.

\paragraph{Elektromagnetisk rörelsemängd och Maxwells spänningstensor}
Vi betraktar nu den mekaniska rörelsemängden komponentvis:
\begin{align*}
	\del{t}{g_{\text{mek}, i}} &= \varepsilon_{0}(E_{i}\del{j}{E_{j}} + E_{j}\del{j}{E_{i}}) + \frac{1}{\mu_{0}}(B_{i}\del{j}{B_{j}} + B_{j}\del{j}{B_{i}}) - \del{i}{w_{\text{em}}} - \varepsilon_{0}\mu_{0}\del{t}{S_{i}} \\
	                                &= \del{j}{\left(\varepsilon_{0}E_{i}E_{j} + \frac{1}{\mu_{0}}B_{i}B_{j} - w_{\text{em}}\delta_{ij}\right)} - \varepsilon_{0}\mu_{0}\del{t}{S_{i}}.
\end{align*}
Vi definierar då elektromagnetiska rörelsemängdstätheten
\begin{align*}
	\vb{g}_{\text{em}} = \mu_{0}\varepsilon_{0}\vb{S}
\end{align*}
och Maxwells spänningstensor
\begin{align*}
	T_{ij} = \varepsilon_{0}E_{i}E_{j} + \frac{1}{\mu_{0}}B_{i}B_{j} - \frac{1}{2}\delta_{ij}\left(\varepsilon_{0}E^{2} + \frac{1}{\mu_{0}}B^{2}\right).
\end{align*}
Då kan vi skriva
\begin{align*}
	\del{t}{(g_{\text{mek}, i} + g_{\text{em}, i})} + \del{j}{(-T_{ij})} = 0,
\end{align*}
alltså en kontinuitetsekvation.

Om man vill kan man skriva
\begin{align*}
	\del{j}{T_{ij}} = \delta_{jk}\del{k}{T_{ij}} = \ub{j}\cdot\ub{k}\del{k}{T_{ij}} = (\ub{k}\del{k}){}\cdot({T_{ij}\ub{j}}) = \div{{T_{ij}\ub{j}}},
\end{align*}
och vektorfältet som transporterar komponent $i$ av rörelsemängden är $-\div{{T_{ij}\ub{j}}}$. Men att skriva så är lite fult, och ekvationen vi började med är en fullgod kontinuitetsekvation.

\paragraph{Flöde av rörelsemängd och ytspänningsvektorer}
I ett område har vi
\begin{align*}
	\dv{t}(p_{\text{mek}, i} + p_{\text{em}, i}) &= \integ{}{}{\tau}{\del{t}{g_{\text{mek}, i} + g_{\text{em}, i}}} \\
	                                             &= \integ{}{}{\tau}{\del{j}{T_{ij}}} \\
	                                             &= \integ{}{}{a_{j}}{T_{ij}}.
\end{align*}
Vi skriver integranden som
\begin{align*}
	T_{ij}n_{j} &= \varepsilon_{0}E_{i}E_{j}n_{j} + \frac{1}{\mu_{0}}B_{i}B_{j}n_{j} - \frac{1}{2}\delta_{ij}\left(\varepsilon_{0}E^{2} + \frac{1}{\mu_{0}}B^{2}\right)n_{j} \\
	            &= \varepsilon_{0}\vb{n}\cdot\vb{E}E_{i} + \frac{1}{\mu_{0}}\vb{n}\cdot\vb{B}B_{i} - \frac{1}{2}\left(\varepsilon_{0}E^{2} + \frac{1}{\mu_{0}}B^{2}\right)n_{i}.
\end{align*}
Vi definierar den elektriska och magnetiska ytspänningsvektorn som
\begin{align*}
	\vb{T}_{\text{e}} = \varepsilon_{0}(\vb{n}\cdot\vb{E})\vb{E} - \frac{1}{2}\varepsilon_{0}E^{2}\vb{n},\ \vb{T}_{\text{m}} = \frac{1}{\mu_{0}}(\vb{n}\cdot\vb{B})\vb{B} - \frac{1}{2\mu_{0}}B^{2}\vb{n}.
\end{align*}
Vi ser då att flödet av dessa ut genom området ger ändringen av total rörelsemängd. I många praktiska fall är tidsderivatan av elektromagnetisk rörelsemängd försumbar, och kvar står en flödesintegral på ena sidan och summan av alla krafter på andra.

\paragraph{Geometrisk konstruktion av ytspänningsvektorerna}
Det gäller att
\begin{align*}
	T_{\text{e}} = w_{\text{e}}
\end{align*}
och att om vinkeln mellan $\vb{E}$ och $\vb{n}$ är $\alpha$, bildar ytspänningsvektorn vinkeln $\alpha$ med $\vb{E}$ och $2\alpha$ med $\vb{n}$. Den ligger även i samma plan som $\vb{E}$ och $\vb{n}$. Det samma gäller för den magnetiska ytspänningsvektorn.

Vi visar detta så rakt fram som vi kan för den elektriska ytspänningsvektorn. Det sista påståendets sannhet är uppenbar eftersom $\vb{T}_{\text{e}}$ är en linjärkombination av $\vb{E}$ och $\vb{n}$. Vi har vidare
\begin{align*}
	T_{\text{e}}^{2} &= \varepsilon_{0}^{2}\left((\vb{n}\cdot\vb{E})\vb{E} - \frac{1}{2}E^{2}\vb{n}\right)^{2} \\
	                 &= \varepsilon_{0}^{2}\left((\vb{n}\cdot\vb{E})^{2}E^{2} + \frac{1}{4}E^{4}\vb{n}^{2} - (\vb{n}\cdot\vb{E})\vb{E}\cdot E^{2}\vb{n}\right) \\
	                 &= \varepsilon_{0}^{2}\left((\vb{n}\cdot\vb{E})^{2}E^{2} + \frac{1}{4}E^{4} - (\vb{n}\cdot\vb{E})^{2}E^{2}\right) \\
	                 &= \frac{1}{4}\varepsilon_{0}^{2}E^{4} \\
	                 &= w_{\text{e}}^{2},
\end{align*}
vilket visar det första påståendet. Vi har vidare
\begin{align*}
	\vb{T}_{\text{e}}\cdot\vb{E} = \varepsilon_{0}(\vb{n}\cdot\vb{E})E^{2} - \frac{1}{2}\varepsilon_{0}E^{2}(\vb{n}\cdot\vb{E}) = \frac{1}{2}\varepsilon_{0}E^{2}(\vb{n}\cdot\vb{E}) = (\vb{n}\cdot\vb{E})w_{\text{e}},
\end{align*}
varför
\begin{align*}
	\cos{\alpha} = \frac{\vb{n}\cdot\vb{E}}{E}.
\end{align*}
Jämför detta med
\begin{align*}
	\vb{T}_{\text{e}}\cdot\vb{n} = \varepsilon_{0}(\vb{n}\cdot\vb{E})^{2} - \frac{1}{2}\varepsilon_{0}E^{2}n^{2} = \frac{1}{2}\varepsilon_{0}E^{2}(2\cos[2](\alpha) - 1) = T_{\text{e}}\cos{2\alpha}.
\end{align*}
Beviset är helt analogt för den magnetiska ytspänningsvektorn.

\paragraph{Elektromagnetiskt rörelsemängdsmoment}
Vi har
\begin{align*}
	\vb{N}_{\text{mek}} = \integ{}{}{\tau}{\vb{r}\times\del{t}{\vb{g}_{\text{mek}}}} = \integ{}{}{\tau}{\del{t}{\vb{l}_{\text{mek}}}},
\end{align*}
där vi har infört tätheten av rörelsemängdsmoment
\begin{align*}
	\vb{l}_{\text{em}} = \vb{r}\times\vb{g}_{\text{em}}.
\end{align*}
Vi vill gärna ha en konserveringslag för denna, och använder konserveringslagen för rörelsemängd för att skriva
\begin{align*}
	\del{t}{(l_{\text{mek}, i} + l_{\text{em}, i})} &= \varepsilon_{ijk}r_{j}\del{t}{(g_{\text{mek}, k} + g_{\text{em}, k})} \\
	                                                &= \varepsilon_{ijk}r_{j}\del{p}{T_{kp}} \\
	                                                &= \del{p}{(\varepsilon_{ijk}r_{j}T_{kp})} - \varepsilon_{ijk}T_{kp}\del{p}{r_{j}} \\
	                                                &= \del{p}{(\varepsilon_{ijk}r_{j}T_{kp})} - \varepsilon_{ijk}T_{kj}.
\end{align*}
Eftersom Maxwells spänningstensor är symmetrisk, försvinner andra termen. Vi skriver först om
\begin{align*}
	\del{p}{(\varepsilon_{ijk}r_{j}T_{kp})} = \del{j}{(\varepsilon_{ipk}r_{p}T_{kj})} = -\del{j}{(\varepsilon_{ikp}r_{p}T_{kj})}
\end{align*}
och definierar
\begin{align*}
	M_{ij} = \varepsilon_{ikp}r_{p}T_{kj}.
\end{align*}
Detta ger
\begin{align*}
	\del{t}{(l_{\text{mek}, i} + l_{\text{em}, i})} + \del{j}{M_{ij}} = 0.
\end{align*}

\paragraph{Flöde av elektromagnetiskt rörelsemängdsmoment}
Vi har
\begin{align*}
	\dv{t}(L_{\text{mek}, i} + L_{\text{em}, i}) &= \integ{}{}{\tau}{\del{t}{(l_{\text{mek}, i} + l_{\text{em}, i})}} \\
	                                             &= -\integ{}{}{\tau}{\del{j}{M_{ij}}} \\
	                                             &= -\integ{}{}{a_{j}}{M_{ij}} \\
	                                             &= -\integ{}{}{a_{j}}{\varepsilon_{ikp}r_{p}T_{kj}},
\end{align*}
vilket ger
\begin{align*}
	\dv{t}\vb{L} = -\integ{}{}{a}{(\vb{T}_{\text{e}} + \vb{T}_{\text{m}})\times\vb{r}}.
\end{align*}

\section{Elektromagnetiska vågor}

\paragraph{Elektromagnetiska vågekvationer}
Det gäller allmänt att
\begin{align*}
	\curl(\curl(\vb{F})) = \grad(\div{\vb{F}}) - \laplacian{\vb{F}}.
\end{align*}
Maxwells ekvationer ger
\begin{align*}
	\curl(\curl{\vb{E}}) &= -\curl(\del{t}{\vb{B}}) = -\del{t}{\curl{\vb{B}}} = -\mu_{0}\del{t}{\vb{J}} - \mu_{0}\varepsilon_{0}\del[2]{t}{\vb{E}}, \\
	\curl(\curl{\vb{B}}) &= \curl(\mu_{0}\vb{J} + \mu_{0}\varepsilon_{0}\del{t}{\vb{E}}) = \mu_{0}\curl{\vb{J}} - \mu_{0}\varepsilon_{0}\del[2]{t}{\vb{B}},
\end{align*}
vilket ger vågekvationerna
\begin{align*}
	(\laplacian - \mu_{0}\varepsilon_{0}\del[2]{t}{})\vb{E} &= \frac{1}{\varepsilon_{0}}\grad{\rho} + \mu_{0}\del{t}{\vb{J}}, \\
	(\laplacian - \mu_{0}\varepsilon_{0}\del[2]{t}{})\vb{B} &= -\mu_{0}\curl{\vb{J}}.
\end{align*}

\paragraph{Plana vågors våghastighet}
Man kan visa att en plan våg i källfritt rum propagerar med farten
\begin{align*}
	c = \frac{1}{\sqrt{\mu_{0}\varepsilon_{0}}}.
\end{align*}

\paragraph{Fouriertransformering av fälten}
Fouriertransformering ger
\begin{align*}
	\vb{E}_{\omega} = \integ{-\infty}{\infty}{t}{\vb{E}e^{i\omega t}},
\end{align*}
med inverstransform
\begin{align*}
	\vb{E} = \frac{1}{2\pi}\integ{-\infty}{\infty}{\omega}{\vb{E}_{\omega}e^{-i\omega t}}.
\end{align*}
Det motsvarande gäller även för magnetfältet. Vi använder beteckningen
\begin{align*}
	\vb{E}_{\omega} = \fou{\vb{E}},\ \vb{E} = \fouinv{\vb{E}_{\omega}}.
\end{align*}
Eftersom fälten är reella, måste det gälla att
\begin{align*}
	\vb{E} = \frac{1}{\pi}\Re\left(\integ{0}{\infty}{t}{\vb{E}_{\omega}e^{-i\omega t}}\right).
\end{align*}

Vi kan visa följande egenskaper i källfria rum:
\begin{align*}
	\fou{\vb{E}} = -i\omega\vb{E}_{\omega},\ \curl{\vb{E}_{\omega}} = i\omega\vb{B}_{\omega},\ \curl{\vb{B}_{\omega}} = -i\frac{\omega}{c^{2}}\vb{E}_{\omega},\ \div{\vb{E}} = \div{\vb{B}} = 0.
\end{align*}

\paragraph{Strikt tidsharmonisering}
En strikt tidsharmonisering är en idealisering av en given fältkomponent $F$ på formen
\begin{align*}
	F = F_{0}\cos{\omega_{0}t + \alpha} = \Re{F_{0}e^{-\omega_{0}t - i\alpha}}.
\end{align*}
Genom att baka ihop allt førutom det harmoniska tidsberoende får man en komplex amplitud för vågen. Denna har belop $F_{0}$ och argument $-\alpha$.

\paragraph{Plana vågor}
En plan våg ges av
\begin{align*}
	\vb{E}_{\omega} = \Re{\vb{E}_{0}e^{i(\vb{k}\cdot\vb{r} - \omega t)}},
\end{align*}
där amplituden är en konstant. För en sån ger Maxwells ekvationer
\begin{align*}
	\vb{k}\times\vb{E}_{\omega} = \omega\vb{B},\ \vb{k}\times\vb{B} = -\omega c^{2}\vb{E}_{\omega}.
\end{align*}

\paragraph{Tidsmedelvärde av harmoniska storheter}
Låt $a$ och $b$ vara harmoniska storheter med periodtid $T$. Då kan man visa att
\begin{align*}
	\expval{ab} = \frac{1}{2}\Re{\cc{A}B}
\end{align*}
där $A$ och $B$ är storheternas amplituder.

\paragraph{Maxwells ekvationer i linjära material}
I linjära materialer är Maxwells ekvationer
\begin{align*}
	\curl{\vb{E}} = -\del{t}{\vb{B}},\ \div{\vb{B}} = 0,\ \curl{\vb{H}} = \vb{J}_{\text{f}} + \del{t}{\vb{D}},\ div{\vb{D}} = \rho_{\text{f}}.
\end{align*}

\paragraph{Laddningstätheter i homogena ohmska material}
I ett homogent ohmsk material fås
\begin{align*}
	\div{\curl{\vb{H}}} &= \div{\vb{J}_{\text{f}}} + \del{t}{\div{\vb{D}}} \\
	                    &= \frac{\sigma}{\varepsilon}\div{\vb{D}} + \del{t}{\rho_{\text{f}}} \\
	                    &= \frac{\sigma}{\varepsilon}\div{\vb{D}}\rho_{\text{f}} + \del{t}{\rho_{\text{f}}},
\end{align*}
och därmed avtar fria laddningstätheten exponentiellt mot $0$. Därmed kommer vi bortse från denna.

\paragraph{Komplex permittivitet och permeabilitet}
Maxwells ekvationer för en monokromatisk plan våg i ett material ger
\begin{align*}
	\vb{k}\times\vb{B} = -\omega\mu\left(\varepsilon + \frac{\sigma}{\omega}i\right)\vb{E}.
\end{align*}
Detta får oss att införa den komplexa permittiviteten $\varepsilon_{\text{c}} = \varepsilon + \frac{\sigma}{\omega}i$. De två termerna kommer betecknas $\varepsilon'$ och $\varepsilon''$. Den imaginära termen kommer från lednings- och polarisationsförluster i materialet. På liknande sätt kan vi införa en komplex permeabilitet $\mu_{\text{c}}$, där den imaginära termen kommer från magnetiseringsförluster.

\paragraph{Dispersion i material och komplexa vågvektorn}
Vi har
\begin{align*}
	\vb{k}\times(\vb{k}\times\vb{E}) = (\vb{k}\cdot\vb{E})\vb{k} - (\vb{k}\cdot\vb{k})\vb{E} = \omega\vb{k}\times\vb{B} = -\omega\mu\varepsilon_{\text{c}}\vb{E},
\end{align*}
vilket ger dispersionsrelationen
\begin{align*}
	\vb{k}\cdot\vb{k} = \omega\mu\varepsilon_{\text{c}}.
\end{align*}
Vi ser då att vi även måste definiera en komplex vågvektor. Den imaginära och komplexa termen är parallella för en plan våg.

\paragraph{Inträngningsdjup}
Vågvektorns imaginära term kommer ge att fälten avtar exponentiellt med en hastighet proportionell mot komplexa termens belopp. Vi definierar då inträngningsdjupet
\begin{align*}
	\delta = \frac{1}{k''}.
\end{align*}

\paragraph{Vågimpedans}
För en monokromatisk plan våg fås
\begin{align*}
	\vb{B} = \frac{1}{\omega}\vb{k}\times\vb{E} = \sqrt{\mu\varepsilon_{\text{c}}}\ub{k}\times\vb{E},
\end{align*}
och därmed
\begin{align*}
	\vb{H} = \sqrt{\frac{\varepsilon_{\text{c}}}{\mu}}\ub{k}\times\vb{E}.
\end{align*}
Vi definierar då vågimpedansen
\begin{align*}
	\eta = \sqrt{\frac{\mu}{\varepsilon_{\text{c}}}}.
\end{align*}

\paragraph{Normalt infall på en gränsyta}
Betrakta en gränsyta i $z = 0$ mellan två linjära material där det inte finns fria laddningar eller strömtätheter på gränsytan. Låt det infallande elektriska och magnetiska fältet vara
\begin{align*}
	\vb{E}_{\text{I}} = E_{\text{I}}e^{i(kz - \omega t)}\ub{x}.
\end{align*}
Detta ger upphov till reflekterade och transmitterade fält
\begin{align*}
	\vb{E}_{\text{R}} = E_{\text{R}}e^{i(-kz - \omega t)}\ub{x},\ \vb{E}_{\text{T}} = E_{\text{T}}e^{i(kz - \omega t)}\ub{x}.
\end{align*}
I tangentiell riktning är $\vb{E}$ och $\vb{H}$ kontinuerliga. Detta ger
\begin{align*}
	E_{\text{I}} + E_{\text{R}} = E_{\text{T}},\ \frac{1}{\eta_{1}}(E_{\text{I}} - E_{\text{R}}) = \frac{1}{\eta_{2}}E_{\text{T}}.
\end{align*}
Minustecknet på det reflekterade magnetfältet följer av att vågvektorn byter tecken. Genom att definiera
\begin{align*}
	\beta = \frac{\eta_{1}}{\eta_{2}}
\end{align*}
fås
\begin{align*}
	E_{\text{R}} = \frac{1 - \beta}{1 + \beta}E_{\text{I}},\ E_{\text{T}} = \frac{2}{1 + \beta}E_{\text{I}}.
\end{align*}

\paragraph{Brytningsindex}
Brytningsindexen för ett linjärt medium definieras som
\begin{align*}
	n = \frac{c}{v},
\end{align*}
där $v$ ges av
\begin{align*}
	v = \frac{1}{\sqrt{\varepsilon\mu}}.
\end{align*}

\paragraph{Intensitet}
Intensitet för en elektromagnetisk våg ges av infallande effekt per area, med andra ord flödet av Poyntingvektorn per area. För en harmonisk våg i ett linjärt medium ges (tidsmedelvärdet av) intensiteten av
\begin{align*}
	I = \frac{1}{2}\varepsilon vE^{2}.
\end{align*}

\paragraph{Reflektions- och transmissionskoefficienter}
Reflektionskoefficienten definieras av
\begin{align*}
	R = \frac{I_{\text{R}}}{I_{\text{I}}}.
\end{align*}
För normalt infall på gränsytan mellan linjära medier ges den av
\begin{align*}
	R = \left(\frac{1 - \beta}{1 + \beta}\right)^{2}.
\end{align*}
På samma sätt definieras transmissionskoefficienten av
\begin{align*}
	T = \frac{I_{\text{T}}}{I_{\text{I}}},
\end{align*}
och ges i fallet som beskrivs ovan av
\begin{align*}
	T = \frac{4\varepsilon_{2}v_{2}}{\varepsilon_{1}v_{1}(1 + \beta)^{2}} = \sqrt{\frac{\varepsilon_{2}\mu_{1}}{\varepsilon_{1}\mu_{2}}}\frac{4}{(1 + \beta)^{2}}.
\end{align*}

Om de relativa permeabiliteterna är nära $1$ kan vi skriva
\begin{align*}
	R = \left(\frac{n_{2} - n_{1}}{n_{2} + n_{1}}\right)^{2},\ T = \frac{4n_{1}n_{2}}{(n_{1} + n_{2})^{2}}.
\end{align*}

\paragraph{Snett infall}
Betrakta en liknande gränsyta som för normalt infall, men nu med snett infallande elektriska och magnetiska fält
\begin{align*}
	\vb{E}_{\text{I}} = \vb{E}_{0\text{I}}e^{i(\vb{k}_{\text{I}}\cdot\vb{r} - \omega t)},\ \vb{B}_{\text{I}} = (\vb{k}_{\text{I}}\times\vb{E}_{0\text{I}})e^{i(\vb{k}_{\text{I}}\cdot\vb{r} - \omega t)}.
\end{align*}
Dessa ger upphov till reflekterade fält
\begin{align*}
	\vb{E}_{\text{R}} = \vb{E}_{0\text{R}}e^{i(\vb{k}_{\text{R}}\cdot\vb{r} - \omega t)},\ \vb{B}_{\text{R}} = (\vb{k}_{\text{R}}\times\vb{E}_{0\text{R}})e^{i(\vb{k}_{\text{R}}\cdot\vb{r} - \omega t)},\ \vb{E}_{\text{T}} = \vb{E}_{0\text{T}}e^{i(\vb{k}_{\text{T}}\cdot\vb{r} - \omega t)},\ \vb{B}_{\text{T}} = (\vb{k}_{\text{T}}\times\vb{E}_{0\text{T}})e^{i(\vb{k}_{\text{T}}\cdot\vb{r} - \omega t)}.
\end{align*}
Frekvensen anges av källan, ooch då uppfyller vågvektorerna
\begin{align*}
	k_{\text{I}}v_{1} = k_{\text{R}}v_{1} = k_{\text{T}}v_{2}.
\end{align*}
Randvillkoren för fälten gäller fortfarande, och kommer vara på formen
\begin{align*}
	F_{\text{I}}e^{i(\vb{k}_{\text{I}}\cdot\vb{r} - \omega t)} + F_{\text{R}}e^{i(\vb{k}_{\text{R}}\cdot\vb{r} - \omega t)} = F_{\text{T}}e^{i(\vb{k}_{\text{T}}\cdot\vb{r} - \omega t)}.
\end{align*}
Exponenterna innehåller all information om rum och tid. Om dessa vore olika, skulle en liten ändring i position eller tid bryta likheten. Detta bekreftar att frekvensen ej ändras. Vi får vidare att
\begin{align*}
	\vb{k}_{\text{I}}\cdot\vb{r} = \vb{k}_{\text{R}}\cdot\vb{r} = \vb{k}_{\text{T}}\cdot\vb{r}
\end{align*}
på gränsytan, vilket implicerar att endast $z$-komponenten av vågvektorn kan ändras under reflektion och transmission. Vi tar det därmed som en lag att vågvektorerna bildar infallsplanet, och vi kan inskränka oss till fallet där vågvektorerna ligger i $xz$-planet.

Låt vågvektorerna peka i vinklar $\theta_{\text{I}}, \theta_{\text{R}}$ och $\theta_{\text{T}}$ från normalriktningen pekande in i medierna. Om tangentialkomponenterna av vågvektorn ej ändras, ger detta reflektionslagen
\begin{align*}
	\sin{\theta_{\text{I}}} = \sin{\theta_{\text{R}}},\ \theta_{\text{I}} = \theta_{\text{R}}
\end{align*}
och
\begin{align*}
	k_{\text{I}}\sin{\theta_{\text{I}}} = k_{\text{T}}\sin{\theta_{\text{T}}}.
\end{align*}
Detta ger Snells lag
\begin{align*}
	\frac{1}{v_{1}}\sin{\theta_{\text{I}}} = \frac{1}{v_{2}}\sin{\theta_{\text{T}}}
\end{align*}
alternativt
\begin{align*}
	\frac{\sin{\theta_{\text{T}}}}{\sin{\theta_{\text{I}}}} = \frac{n_{1}}{n_{2}}.
\end{align*}

Randvillkoren ger vidare
\begin{align*}
	\varepsilon_{1}(\vb{E}_{0\text{I}}^{\perp} + \vb{E}_{0\text{R}}^{\perp})           &= \varepsilon_{2}\vb{E}_{0\text{T}}^{\perp}, \\
	\vb{B}_{0\text{I}}^{\perp} + \vb{B}_{0\text{R}}^{\perp}                            &= \vb{B}_{0\text{T}}^{\perp}, \\
	\vb{E}_{0\text{I}}^{\parallel} + \vb{E}_{0\text{R}}^{\parallel}                    &= \vb{E}_{0\text{T}}^{\parallel}, \\
	\frac{1}{\mu_{1}}(\vb{B}_{0\text{I}}^{\parallel} + \vb{B}_{0\text{R}}^{\parallel}) &= \frac{1}{\mu_{1}}\vb{B}_{0\text{T}}^{\perp},
\end{align*}
där $\vb{B} = \frac{1}{v}\vb{k}\times\vb{E}$ i alla ekvationer.

\paragraph{Plan polarisering}
Betrakta snett infall där elektriska fältet oscillerar i infallsplanet, även kallad TM-fallet. Då ger randvillkoren
\begin{align*}
	\varepsilon_{1}(-E_{0\text{I}}\sin{\theta_{\text{I}}} + E_{0\text{R}}\sin{\theta_{\text{R}}}) = -\varepsilon_{2}E_{0\text{T}}\sin{\theta_{\text{T}}}, \\
	E_{0\text{I}}\cos{\theta_{\text{I}}} + E_{0\text{R}}\cos{\theta_{\text{R}}} = E_{0\text{T}}\cos{\theta_{\text{T}}}, \\
	\frac{1}{\eta_{1}}(E_{0\text{I}} - E_{0\text{R}}) = \frac{1}{\eta_{2}}E_{0\text{T}}.
\end{align*}
Första och sista randvillkoret är linjärt beroende på grund av reflektionslagen och Snells lag, och reducerar båda till
\begin{align*}
	E_{0\text{I}} - E_{0\text{R}} = \beta E_{0\text{T}}.
\end{align*}
Om vi definierar
\begin{align*}
	\alpha = \frac{\cos{\theta_{\text{T}}}}{\cos{\theta_{\text{I}}}}
\end{align*}
blir lösningen
\begin{align*}
	E_{0\text{R}} = \frac{\beta - \alpha}{\alpha + \beta}E_{0\text{I}},\ E_{0\text{T}} = \frac{2}{\alpha + \beta}E_{0\text{I}}.
\end{align*}
Detta är Fresnels ekvationer för plan polarisering. Det finns motsvarande ekvationer för polarisering normalt på infalls planet.

\paragraph{Brewstervinkeln}
För $\alpha = \beta$ försvinner den reflekterade vågen. Vi skriver först
\begin{align*}
	\alpha = \frac{\sqrt{1 - \left(\frac{n_{1}}{n_{2}}\right)^{2}\sin[2](\theta_{\text{I}})}}{\cos{\theta_{\text{I}}}}.
\end{align*}
Låt Brewstervinkeln $\theta_{\text{B}}$ vara infallsvinkeln så att den reflekterade vågen försvinner. Denna uppfyller
\begin{align*}
	\sin[2](\theta_{\text{B}}) = \frac{1 - \beta^{2}}{\left(\frac{n_{1}}{n_{2}}\right)^{2} - \beta^{2}}.
\end{align*}
Om permeabiliteterna är lika blir detta
\begin{align*}
	\tan{\theta_{\text{B}}} = \frac{n_{2}}{n_{1}}.
\end{align*}

\paragraph{Normalt infall på ledande yta}
Vid normalt infall av en elektromagnetisk våg på en ledande yta fås på liknande sätt som innan en reflekterad våg. Det transmitterade fältet blir dock
\begin{align*}
	\vb{E}_{\text{T}} = E_{\text{T}}e^{i(k_{2}z - \omega t)}\ub{x},\ \vb{B}_{\text{T}} = \frac{k_{2}}{\omega}E_{\text{T}}e^{i(k_{2}z - \omega t)}\ub{y}
\end{align*}
där $k_{2} = k' + i\kappa$ är den nya vågvektorn. Notera att detta implicerar att fälten avtar exponentiellt. Randvillkoren blir även något annorlunda, eftersom det kan finnas källor på gränsytan. Det finns däremot ingen laddningstäthet vid normalt infall eftersom det inte finns någon normalkomponent av elektriska fältet. För en ohmsk ledare kan det heller inte finnas någon strömtäthet på gränsytan eftersom det skulle kräva ett oändligt elektriskt fält.

Randvillkoret för elektriska fältet ger
\begin{align*}
	E_{\text{I}} + E_{\text{R}} = E_{\text{T}},
\end{align*}
och randvillkoret för $H$-fältet ger
\begin{align*}
	\frac{1}{\mu_{1}v_{1}}(E_{\text{I}} - E_{\text{R}}) = \frac{k_{2}}{\mu_{2}\omega}E_{\text{T}}.
\end{align*}
Genom att definiera
\begin{align*}
	\beta' = \frac{k_{2}\mu_{1}v_{1}}{\mu_{2}\omega}
\end{align*}
fås
\begin{align*}
	E_{\text{R}} = \frac{1 - \beta'}{1 + \beta'}E_{\text{I}},\ E_{\text{T}} = \frac{2}{1 + \beta'}E_{\text{I}}.
\end{align*}
För en perfekt ledare har $k_{2}$ stort belopp, vilket ger att vågen reflekteras fullständigt med fasvridning $\pi$.

\paragraph{Metalliska vågledare}
Betrakta en källfri rymd som omges av en metallisk cylinder. Cylinderns symmetriaxel är $z$-axeln. Vi löser detta med separationsansatsen
\begin{align*}
	\vb{E} = \vb{E}_{0}(\vb{s})e^{i(k_{z}z - \omega t)},\ \vb{B} = \vb{B}_{0}(\vb{s})e^{i(k_{z}z - \omega t)}.
\end{align*}
I rymden blir vågekvationen
\begin{align*}
	\laplacian{\vb{E}} - \frac{1}{c^{2}}\del{t}{\vb{E}} = \dalemb{\vb{E}} = \vb{0}.
\end{align*}
Med den givna ansatsen fås
\begin{align*}
	\dalemb = \laplaperp - k_{z}^{2} + \frac{\omega^{2}}{c^{2}}
\end{align*}
där $\laplaperp$ gör derivationer i planet normalt på $z$-axeln. Genom att definiera det normala vågtalet $k_{\perp}^{2} = \frac{\omega^{2}}{c^{2}} - k_{z}^{2}$ fås att fältamplituderna är egenfunktioner till $\laplaperp$ med egenvärden $k_{\perp}^{2}$.

Vi delar nu upp fälten i komponenter tangentiellt och normalt på $z$-axeln. Maxwells ekvationer i tangentialplanet och $z$-riktning ger då
\begin{alignat*}{3}
	(\gradperp{E_{z}} - ik_{z}\vb{E}_{\perp})\times\ub{z} &= i\omega\vb{B}_{\perp},\      &\gradperp\times\vb{E}_{\perp}            &= i\omega B_{z}\ub{z}, \\
	(\gradperp{B_{z}} - ik_{z}\vb{B}_{\perp})\times\ub{z} &= -i\frac{\omega}{c^{2}}\vb{E}_{\perp},\ &\gradperp\times\vb{B}_{\perp} &= -i\frac{\omega}{c^{2}}E_{z}\ub{z}.
\end{alignat*}
Om man kryssar ekvationerna till vänster med $\ub{z}$ från vänster får man på något sätt
\begin{align*}
	\vb{E}_{\perp} = \frac{i}{k_{\perp}^{2}}(k_{z}\gradperp{E_{z}} - \omega\ub{z}\times\gradperp{B_{z}}),\ \vb{B}_{\perp} = \frac{i}{k_{\perp}^{2}}(k_{z}\gradperp{B_{z}} + \frac{\omega}{c^{2}}\ub{z}\times\gradperp{E_{z}}).
\end{align*}

Ekvationen behöver även randvillkor. Vi har antagit att det är metall på randen. Randvillkoret för elektriska fältet ger
\begin{align*}
	\vb{0} = \vb{n}\times\vb{E}_{0} = \vb{n}\times\vb{E}_{\perp} + E_{z}\vb{n}\times\ub{z}.
\end{align*}
Detta implicerar direkt att $E_{z} = 0$ och $\vb{n}\times\vb{E}_{\perp} = \vb{0}$. Detta gäller över hela randen, så $\vb{n}\times\gradperp{E_{z}} = \vb{0}$. Detta implicerar på något sätt $\vb{n}\cdot\gradperp{B_{z}} = 0$.

\paragraph{Gränsfrekvens för vågledare}
Fälten i en vågledare ges av ett Sturm-Liouville-problem, så spektrumet av $k_{\perp}$ är oändligt (detaljerna bestäms av randens geometri). Det finns då för varje vågtal i spektrumet ett $k$ så att $k < k_{\perp}$. För detta $k$ fås $k_{z} = i\sqrt{k_{\perp}^{2} - k^{2}}$, och fältet avtar exponentiellt i ledaren. Gränsfrekvensen för ett givet vågtal definieras som $\omega_{\text{g}} = ck_{\perp}$, och är frekvensen som ligger på gränsen mellan propagerande och evanescenta moder. Vågtalet med lägst gränsfrekvens kallas för grundmoden.

Notera att vi nu kan skriva $k_{z} = \frac{1}{c}\sqrt{\omega^{2} - \omega_{\text{g}}^{2}}$.

\end{document}
