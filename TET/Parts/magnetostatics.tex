\section{Magnetostatik}

\paragraph{Ström}
Ström definieras som $I = \dv{Q}{t}$.

\paragraph{Kraft på strömslingor}
Betrakta två slingor $C$ och $C'$. Genom varje slinga går en ström $I$ respektiva $I'$ i samma riktning som kurvans orientering. Experiment har visat att kraften på $C$ från $C'$ ges av
\begin{align*}
	\vb{F} = \vinteg{C}{}{\vb{r}}{I\times\frac{\mu_{0}I'}{4\pi}\vinteg{C'}{}{\vb{r}'}{\times\frac{1}{R^{2}}\ub{R}}}
\end{align*}
där $\vb{R} = \vb{r} - \vb{r}'$.

\paragraph{Magnetiska fältet}
Genom att definiera det magnetiska fältet från $C'$ i punkten $\vb{r}$ som
\begin{align*}
	\vb{B} = \frac{\mu_{0}I'}{4\pi}\vinteg{C'}{}{\vb{r}'}{\times\frac{1}{R^{2}}\ub{R}}
\end{align*}
fås
\begin{align*}
	\vb{F} = \vinteg{C}{}{\vb{r}}{I\times\vb{B}}.
\end{align*}

\paragraph{Ytströmtäthet och rymdströmtäthet}
För strömmar fördelade i rummet eller på en yta kan vi utvidga definitionen av magnetiska fältet till att bli
\begin{align*}
	\vb{B} = \frac{\mu_{0}}{4\pi}\vinteg{}{}{S}{\vb{J}\times\frac{1}{R^{2}}\ub{R}}\ \text{eller}\ \vb{B} = \frac{\mu_{0}}{4\pi}\vinteg{}{}{V}{\vb{J}\times\frac{1}{R^{2}}\ub{R}},
\end{align*}
där $\vb{J}$ är den lokala strömtätheten.