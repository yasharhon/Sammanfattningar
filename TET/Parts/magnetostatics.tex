\section{Magnetostatik}

\paragraph{Ström}
Ström definieras som $I = \dv{Q}{t}$.

\paragraph{Strömtäthet}
I fall där strömmen inte flödar längs med linjer utan längs ytor eller fritt i rummet definierar vi strömtätheten $\vb{J}$ som vektorfältet som beskriver flödet av laddningar. Strömtäthetens belopp ges av $J = \rho v$, där $\vb{v}$ är hastighetsfältet för laddningarna.

\paragraph{Kontinuitetsekvation för strömtätheten}
Strömtätheten uppfyller
\begin{align*}
	\del{t}{\rho} + \div{\vb{J}} = 0.
\end{align*}

\paragraph{Kraft på strömslingor}
Betrakta två slingor $C$ och $C'$. Genom varje slinga går en ström $I$ respektiva $I'$ i samma riktning som kurvans orientering. Experiment har visat att kraften på $C$ från $C'$ ges av
\begin{align*}
	\vb{F} = \integ{C}{}{\vb{r}}{I\times\left(\frac{\mu_{0}I'}{4\pi}\integ{C'}{}{\vb{r}'}{\times\frac{1}{R^{2}}\ub{R}}\right)}
\end{align*}
där $\vb{R} = \vb{r} - \vb{r}'$.

\paragraph{Magnetiska fältet}
Genom att definiera det magnetiska fältet från $C'$ i punkten $\vb{r}$ som
\begin{align*}
	\vb{B} = \frac{\mu_{0}I'}{4\pi}\integ{C'}{}{\vb{r}'}{\times\frac{1}{R^{2}}\ub{R}}
\end{align*}
fås
\begin{align*}
	\vb{F} = \integ{C}{}{\vb{r}}{I\times\vb{B}}.
\end{align*}

\paragraph{Magnetfält från strömtätheter}
För strömmar fördelade i rummet eller på en yta kan vi utvidga definitionen av magnetiska fältet till att bli
\begin{align*}
	\vb{B} = \frac{\mu_{0}}{4\pi}\integ{}{}{S}{\vb{J}\times\frac{1}{R^{2}}\ub{R}}\ \text{eller}\ \vb{B} = \frac{\mu_{0}}{4\pi}\integ{}{}{V}{\vb{J}\times\frac{1}{R^{2}}\ub{R}}.
\end{align*}

\paragraph{Potential för magnetfältet}
Vi har
\begin{align*}
	\vb{B} &= \frac{\mu_{0}}{4\pi}\integ{}{}{V}{\vb{J}\times\frac{1}{R^{2}}\ub{R}} \\
	       &= -\frac{\mu_{0}}{4\pi}\integ{}{}{V}{\vb{J}\times\grad{\frac{1}{R}}} \\
	       &= \curl{\frac{\mu_{0}}{4\pi}\integ{}{}{V}{\frac{1}{R}\vb{J}}}.
\end{align*}
Rotationsoperatorn kan tas utanför integrationen då den verkar på koordinater som det ej integreras över. Vi definierar då vektorpotentialen
\begin{align*}
	\vb{A} = \frac{\mu_{0}}{4\pi}\integ{}{}{V}{\frac{1}{R}\vb{J}},
\end{align*}
och har då
\begin{align*}
	\vb{B} = \curl{\vb{A}}.
\end{align*}

\paragraph{Entydighet för vektorpotentialen}
Vektorpotentialen kan även definieras som
\begin{align*}
	\vb{A} = \frac{\mu_{0}}{4\pi}\integ{}{}{V}{\frac{1}{R}\vb{J}} + \grad{\Lambda},
\end{align*}
där $\Lambda$ är en godtycklig funktion. Eftersom gradienter ej har rotation, kommer denna vektorpotentialen att ge samma magnetfält. Vi kommer oftast sätta $\Lambda = 0$.

\paragraph{Magnetfältets divergens}
Eftersom $\vb{B}$ är rotationen av en vektorpotential, gäller det att
\begin{align*}
	\div{\vb{B}} = 0.
\end{align*}

\paragraph{Vektorpotentialens divergens}
Vi har
\begin{align*}
	\div{\vb{A}} &= \div{\frac{\mu_{0}}{4\pi}\integ{}{}{V}{\frac{1}{R}\vb{J}}} \\
	             &= \frac{\mu_{0}}{4\pi}\integ{}{}{V}{\vb{J}\cdot\div{\frac{1}{R}}} \\
	             &= -\frac{\mu_{0}}{4\pi}\integ{}{}{V}{\vb{J}\cdot\div'{\frac{1}{R}}} \\
	             &= \frac{\mu_{0}}{4\pi}\integ{}{}{V}{\frac{1}{R}\div'{\vb{J}} - \div'{\frac{1}{R}\vb{J}}} \\
	             &= \frac{\mu_{0}}{4\pi}\integ{}{}{V}{\frac{1}{R}\div'{\vb{J}}} -  \frac{\mu_{0}}{4\pi}\vinteg{}{}{S}{\frac{1}{R}\vb{J}}.
\end{align*}
I elektrostatiska fall är alla laddningar statiska, och detta ger
\begin{align*}
	\div{\vb{A}} = 0.
\end{align*}

\paragraph{Magnetiskt flöde}
Det magnetiska flödet definieras som
\begin{align*}
	\Phi = \vinteg{}{}{S}{\vb{B}}.
\end{align*}
Med hjälp av Stokes' sats fås
\begin{align*}
	\Phi = \vinteg{}{}{r}{A}.
\end{align*}
Detta blir alltid $0$ genom en sluten yta.

\paragraph{Vektorpotentialens laplacian}
På samma sätt som för elektriska potentialen fås
\begin{align*}
	\laplacian{\vb{A}} = -\mu_{0}\vb{J}.
\end{align*}

\paragraph{Magnetfältets rotation}
Vi har
\begin{align*}
	\curl{\vb{B}} = \curl{\curl{\vb{A}}} = \grad{\div{\vb{A}}} - \laplacian{\vb{A}} = \mu_{0}\vb{J}.
\end{align*}

\paragraph{Ampères cirkulationslag}
Strömmen genom en yta ges av
\begin{align*}
	I = \vinteg{}{}{S}{\vb{J}} = \frac{1}{\mu_{0}}\vinteg{}{}{S}{\curl{\vb{B}}} = \frac{1}{\mu_{0}}\vinteg{}{}{r}{\vb{B}}.
\end{align*}

\paragraph{Randvillkor för magnetfältet}
Som med elektriska fältet rör vi oss nära en ytströmtäthet. Kring denna lägger vi en yta och beräknar flödet genom den när ytans tjocklek blir liten. Detta ger
\begin{align*}
	B_{1}^{\perp} - B_{2}^{\perp} = 0.
\end{align*}
Vidare kan vi skriva
\begin{align*}
	\integ{}{}{V}{\mu_{0}\vb{J}} = \integ{}{}{V}{\curl{\vb{B}}}.
\end{align*}
Med indexräkning fås
\begin{align*}
	\left[\integ{}{}{V}{\curl{\vb{B}}}\right]_{i} = \varepsilon_{ijk}\integ{}{}{V}{\del{j}{B_{k}}} = \varepsilon_{ijk}\integ{}{}{S_{j}}{B_{k}} = \left[\integ{}{}{\vb{S}}{\times\vb{B}}\right]_{i}.
\end{align*}
Detta ger
\begin{align*}
	\vb{n}_{12}\times(\vb{B}_{1} - \vb{B}_{2}) = \mu_{0}\vb{J}.
\end{align*}
Ett motsvarande bevis kan även göras med Ampères lag för en liten strömslinga.

\paragraph{Magnetiskt dipolmoment för en slinga}
Betrakta en strömslinga $C$ som bär en ström $I$. Vi söker vektorpotentialen på stort avstånd från slingan. Det exakta uttrycket ges av
\begin{align*}
	\vb{A} = \frac{\mu_{0}}{4\pi}\integ{C}{}{\vb{r}}{\frac{I}{R}}.
\end{align*}
Om $C$ omkransar en yta $S$, fås
\begin{align*}
	\vb{A} = \frac{\mu_{0}I}{4\pi}\integ{S}{}{\vb{S}}{\times\grad'{\frac{1}{R}}} = \frac{\mu_{0}I}{4\pi}\integ{S}{}{\vb{S}}{\times\frac{1}{R^{2}}\ub{R}}.
\end{align*}
Vid stora avstånd fås
\begin{align*}
	\vb{A} = \frac{\mu_{0}}{4\pi}\left(I\integ{S}{}{\vb{S}}{}\right)\times\frac{1}{r^{2}}\ub{r} = \frac{\mu_{0}}{4\pi r^{2}}\vb{m}\times\ub{r},
\end{align*}
där vi har definierat det magnetiska dipolmomentet
\begin{align*}
	\vb{m} = I\integ{S}{}{\vb{S}}{} = I\vb{S}.
\end{align*}
Igen har vi definierat slingans vektorarea $\vb{S}$.

Hur ska man välja vektorarean? Tänk dig att $C$ är randen till två olika ytor $S_{1}$ och $S_{2}$. Detta ger
\begin{align*}
	\vb{S}_{1} - \vb{S}_{2} = \integ{S_{1}}{}{\vb{r}}{} - \integ{S_{2}}{}{\vb{r}}{} = \integ{S_{1} + S_{2}}{}{\vb{S}}{} = \integ{V}{}{V}{\grad{1}} = \vb{0},
\end{align*}
och därmed spelar inte valet roll.

\paragraph{Magnetiskt dipolmoment för en allmän strömtäthet}
För att studera en allmän strömtäthet, vill vi dela den upp i slingor. Betrakta då konen med $\vb{r}$, som ligger på $C$, som generatris. För denna är ytelementet $\vb{S} = \frac{1}{2}\vb{r}\times\dd{\vb{r}}$ fås
\begin{align*}
	\vb{m} = I\vb{S} = I\integ{S}{}{\vb{S}}{} = \frac{1}{2}\int\limits_{C}{\vb{r}\times I\dd{\vb{r}}}.
\end{align*}
Vi generaliserar detta genom att låta $I\dd{\vb{r}}$ gå mot $\vb{J}\dd{V}$ och integrera över dessa resultat för att få
\begin{align*}
	\vb{m} = \frac{1}{2}\int{\dd{V}\vb{r}\times \vb{J}}.
\end{align*}

\paragraph{Dipolmomentets beroende av origo}
Om vi förflyttar vårat koordinatsystem fås
\begin{align*}
	\vb{m}_{O} = \frac{1}{2}\int{\dd{V}(\vb{r} - \vb{r}_{O})\times \vb{J}} = \vb{m} - \frac{1}{2}\vb{r}_{O}\times\int{\dd{V}\vb{J}}.
\end{align*}
Vi har
\begin{align*}
	\integ{}{}{V}{J_{i}} = \integ{}{}{V}{\vb{J}\cdot\grad{r_{i}} + r_{i}\div{\vb{J}}} = \integ{}{}{V}{\div{r_{i}\vb{J}} + r_{i}\div{\vb{J}}}.
\end{align*}
Vi använder nu att vi arbetar med magnetostatik för att ta bort sista termen, vilket ger
\begin{align*}
	\integ{}{}{V}{J_{i}} = \integ{}{}{V}{\div{r_{i}\vb{J}}} = \integ{}{}{\vb{S}}{\cdot r_{i}\vb{J}}.
\end{align*}
Slingan förutsätts vara ändlig, och då behöver vi bara integrera över en yta som inneslutar den. På den ytan är $\vb{J} = \vb{0}$, vilket ger att integralen av komponenten blir $0$, och slutligen
\begin{align*}
	\vb{m}_{O} = \vb{m}.
\end{align*}

\paragraph{Magnetiska fältet från en dipol}
Vi får
\begin{align*}
	\vb{B} &= \curl{\vb{A}} \\
	       &= \frac{\mu_{0}}{4\pi}\curl(\frac{1}{r^{2}}\vb{m}\times\ub{r}) \\
	       &= \frac{\mu_{0}}{4\pi}\left(\grad{\frac{1}{r^{3}}}\times(\vb{m}\times\vb{r}) + \frac{1}{r^{3}}\curl(\vb{m}\times\vb{r})\right) \\
	       &= \frac{\mu_{0}}{4\pi}\left(-\frac{3}{r^{4}}\grad{r}\times(\vb{m}\times\vb{r}) + \frac{1}{r^{3}}\curl(\vb{m}\times\vb{r})\right) \\
	       &= \frac{\mu_{0}}{4\pi}\left(-\frac{3}{r^{4}}\ub{r}\times(\vb{m}\times\vb{r}) + \frac{1}{r^{3}}(\vb{m}\div{\vb{r}}  - (\vb{m}\cdot\grad)\vb{r})\right) \\
	       &= \frac{1}{r^{3}}\frac{\mu_{0}}{4\pi}\left(-3\ub{r}\times(\vb{m}\times\ub{r}) + \frac{1}{r^{3}}(3\vb{m}  - \vb{m})\right) \\
	       &= \frac{1}{r^{3}}\frac{\mu_{0}}{4\pi}\left(-3(\ub{r}\cdot\ub{r})\vb{m} + 3\times(\vb{m}\cdot\ub{r})\ub{r} + 2\vb{m}\right) \\
	       &= \frac{1}{r^{3}}\frac{\mu_{0}}{4\pi}\left(3(\vb{m}\cdot\ub{r})\ub{r} - \vb{m}\right).
\end{align*}

\paragraph{Kraft på en magnetisk dipol}
Kraften på en strömslinga $C$ ges av
\begin{align*}
	\vb{F} &= \int\limits_{C}{I\dd{\vb{r}}\times\vb{B}} \\
	       &= I\int\limits_{S}{(\dd{\vb{S}}\times\grad')\times\vb{B}} \\
	       &= I\integ{S}{}{S}{\grad'(\vb{B}\cdot\ub{n}) - \ub{n}\div{\vb{B}})}.
\end{align*}
Komponentvis fås
\begin{align*}
	F_{i} = I\integ{S}{}{S}{\delp{i}{B_{j}}n_{j}}.
\end{align*}
Vi antar att magnetfältet varierar måttligt över slingan, och approximerar derivatornas värde i mittpunkten, varför den faktorn kan tas utanför integralen. Detta ger
\begin{align*}
	F_{i} = I\del{i}{B_{j}}\integ{S}{}{S_{j}}{} = \del{i}{B_{j}S_{j}},
\end{align*}
och slutligen
\begin{align*}
	\vb{F} = \grad(\vb{m}\cdot\vb{B}).
\end{align*}

\paragraph{Vridmoment på en slinga}
Vridmomentet ges av
\begin{align*}
	\vb{M} &= \int\limits_{C}{\vb{r}\times\dd{\vb{F}}} \\
	       &= I\int\limits_{C}{\vb{r}\times(\dd{\vb{r}}\times\vb{B})} \\
	       &= I\int\limits_{C}{(\vb{B}\cdot\vb{r})\dd{\vb{r}} - (\vb{r}\cdot\dd{\vb{r}})\vb{B}}.
\end{align*}
Vi approximerar fältet till att vara konstant och lika med fältet i mitten, vilket ger
\begin{align*}
	\vb{B}\int\limits_{C}{(\vb{r}\cdot\dd{\vb{r}})} = \vb{B}\vinteg{S}{}{\vb{S}}{\curl{\vb{r}}} = \vb{0}.
\end{align*}
Detta ger
\begin{align*}
	\vb{M} &= I\integ{C}{}{\vb{r}}{\vb{B}\cdot\vb{r}} \\
	       &= I\integ{S}{}{\vb{S}}{\times\grad{\vb{B}\cdot\vb{r}}} \\
	       &= I\integ{S}{}{\vb{S}}{\times\vb{B}} \\
	       &= \vb{m}\times\vb{B}.
\end{align*}

\paragraph{Magnetisering}
Magnetiseringen $\vb{M}$ uppfyller
\begin{align*}
	\vb{m} = \integ{}{}{V}{\vb{M}}.
\end{align*}

I en atom har elektronerna omloppstid
\begin{align*}
	T = \frac{2\pi r}{v},
\end{align*}
vilket ger strömmen
\begin{align*}
	\vb{I} = -\frac{e}{T}\ub{\phi} = -\frac{ev}{2\pi r}\ub{\phi}.
\end{align*}
Dessa ger då magnetiska momentet
\begin{align*}
	\vb{m} = \frac{1}{2}\int{\vb{r}\times I\dd{\vb{r}}} = -\frac{1}{2}erv\ub{z}.
\end{align*}

\paragraph{Vektorpotential från magnetisering}
Vi har
\begin{align*}
	\vb{A} = \frac{\mu_{0}}{4\pi}\integ{}{}{V}{\frac{1}{r^{2}}\vb{M}\times\ub{r}}.
\end{align*}
Detta kan skrivas som
\begin{align*}
	\vb{A} &= \frac{\mu_{0}}{4\pi}\integ{}{}{V}{\vb{M}\times\grad'{\frac{1}{r}}} \\
	       &= \frac{\mu_{0}}{4\pi}\integ{}{}{V}{\vb{M}\times\grad'{\frac{1}{r}}} \\
	       &= \frac{\mu_{0}}{4\pi}\integ{}{}{V}{\frac{1}{r}\curl'{\vb{M}} - \curl'{\frac{1}{r}\vb{M}}} \\
	       &= \frac{\mu_{0}}{4\pi}\integ{}{}{V}{\frac{1}{r}\curl'{\vb{M}}} - \frac{\mu_{0}}{4\pi}\integ{}{}{\vb{S}}{\times\frac{1}{r}\vb{M}}.
\end{align*}
Detta motsvarar magnetiska fältet från en volymström
\begin{align*}
	\vb{J}_{\text{bv}} = \curl{\vb{M}}
\end{align*}
och en ytström
\begin{align*}
	\vb{J}_{\text{bs}} = \vb{M}\times\ub{n}.
\end{align*}

\paragraph{Multipolutveckling av vektorpotentialen}
Vi har
\begin{align*}
	\frac{1}{R} = \sum\limits_{l = 0}^{\infty}\frac{r'^{l}}{r^{l + 1}}P_{l}(\cos{\theta'}),
\end{align*}
vilket för en strömslinga ger
\begin{align*}
	\vb{A} &= \frac{\mu_{0}}{4\pi}\integ{}{}{\vb{r}}{\frac{I}{R}} \\
	       &= \frac{\mu_{0}I}{4\pi}\sum\limits_{l = 0}^{\infty}\integ{}{}{\vb{r}'}{\frac{r'^{l}}{r^{l + 1}}P_{l}(\cos{\theta'})} \\
	       &= \frac{\mu_{0}I}{4\pi}\sum\limits_{l = 0}^{\infty}\frac{1}{r^{l + 1}}\integ{}{}{\vb{r}'}{r'^{l}P_{l}(\cos{\theta'})}.
\end{align*}
Den första termen ges av
\begin{align*}
	\vb{A}_{0} = \frac{\mu_{0}I}{4\pi}\frac{1}{r^{1}}\integ{}{}{\vb{r}'}{P_{0}(\cos{\theta'})} = \frac{\mu_{0}I}{4\pi}\frac{1}{r^{1}}\integ{}{}{\vb{r}'}{} = \vb{0}
\end{align*}
för en sluten strömslinga. Inte oförväntad, då vi inte känner till magnetiska monopoler. Den andra termen ges av
\begin{align*}
	\vb{A}_{1} &= \frac{\mu_{0}I}{4\pi}\frac{1}{r^{2}}\integ{}{}{\vb{r}'}{r'^{1}P_{1}(\cos{\theta'})} \\
	           &= \frac{\mu_{0}I}{4\pi r^{2}}\integ{}{}{\vb{r}'}{r'\cos{\theta'}} \\
	           &= \frac{\mu_{0}I}{4\pi r^{2}}\integ{}{}{\vb{r}'}{r'\cdot\ub{r}} \\
	           &= -\frac{\mu_{0}I}{4\pi r^{2}}\ub{r}\times\integ{}{}{\vb{S}}{},
\end{align*}
vilket motsvarar en dipolterm. Vidare skulle man kunna skriva upp kvadrupoltermen också.

\paragraph{$H$-fältet}
Ampères lag ger
\begin{align*}
	\curl{\vb{B}} = \mu_{0}\vb{J}.
\end{align*}
Med resultatet ovan fås
\begin{align*}
	\curl{\vb{B}}                           &= \mu_{0}\vb{J}_{\text{fri}} + \mu_{0}\curl{\vb{M}}, \\
	\curl(\frac{1}{\mu_{0}}\vb{B} - \vb{M}) &= \mu_{0}\vb{J}_{\text{fri}}.
\end{align*}
Vi definierar då
\begin{align*}
	\vb{H} = \frac{1}{\mu_{0}}\vb{B} - \vb{M},
\end{align*}
vilket ger
\begin{align*}
	\curl{\vb{H}} = \mu_{0}\vb{J}_{\text{fri}}.
\end{align*}

\paragraph{Amperes lag för $H$-fältet}
Vi får
\begin{align*}
	I_{\text{fri}} = \vinteg{}{}{r}{\vb{H}}.
\end{align*}

\paragraph{Linjära magnetiserbara material}
Betrakta ett material som uppfyller
\begin{align*}
	\vb{M} = \frac{1}{\mu{0}}\chi_{\text{m}}\vb{B}.
\end{align*}
Dessa uppfyller
\begin{align*}
	\vb{B} &= \mu_{0}(\vb{H} + \vb{M}), \\
	\vb{M} &= \chi_{\text{m}}(\vb{H} + \vb{M}), \\
	\vb{M} &= \frac{\chi_{\text{m}}}{1 - \chi_{\text{m}}}\vb{H} = \chi_{\text{m}}^{H}\vb{H}.
\end{align*}
Detta ger slutligen
\begin{align*}
	\vb{B} = \mu_{0}(1 + \chi_{\text{m}}^{H})\vb{H} = \mu_{0}\mu_{\text{r}}\vb{H} = \mu\vb{H},
\end{align*}
där $\mu_{\text{r}} = 1 + \chi_{\text{m}}^{H}$ kallas den relativa permeabiliteten och $\mu$ kallas permeabiliteten.

\paragraph{Klassificering av magnetiska material}
Det finns olika sorters magnetism i material. BLand dessa är:
\begin{itemize}
	\item diamagnetiska material, med $\chi_{\text{m}} < 0$, typiskt kring \num{-1e-5}.
	\item paramagnetiska material, med $\chi_{\text{m}} > 0$, typiskt kring \num{1e-4}.
	\item ferromagnetiska material, med $\chi_{\text{m}} >> 1$. Dessa är dock ofta icke-linjära.
\end{itemize}

\paragraph{Randvillkor för $H$-fältet}
I en gränsyta fås
\begin{align*}
	H_{1}^{\perp} - H_{2}^{\perp} = M_{2}^{\perp} - M_{1}^{\perp},\ \vb{n}_{12}\times(\vb{H}_{1} - \vb{H}_{2}) = \mu_{0}\vb{J}_{\text{f}}.
\end{align*}

\paragraph{Ömsesidig induktans}
Betrakta två strömslingor $C_{1}$ och $C_{2}$. En ström $I_{1}$ ger ett magnetiskt flöde $\Phi_{21}$ genom $C_{2}$. Vi har att
\begin{align*}
	\Phi_{12} = \vinteg{C_{2}}{}{l_{2}}{\vb{A}} = \vinteg{C_{2}}{}{l_{2}}{\frac{\mu_{0}I_{1}}{4\pi}\integ{C_{1}}{}{\vb{l}_{1}}{\frac{1}{R}}} = M_{21}I_{1},
\end{align*}
där $M_{21}$ är $C_{2}$:s ömsesidiga induktans från $C_{1}$. Det gäller att $M_{12} = M_{21}$.

\paragraph{Egeninduktans}
Om det går en ström i en slinga induceras även ett magnetiskt flöde från slingas egna fält. Vi döper denna $M_{11} = L_{1}$, och har $\Phi_{11} = L_{1}I_{1}$.

\paragraph{Induktansmatris}
Om man har ett problem med $n$ strömslingor, får man
\begin{align*}
	\Phi_{i} = \sum\limits_{j}M_{ij}I_{j},
\end{align*}
som kan skrivas som ett matrisproblem som involverar induktansmatrisen $M$.