\section{Magnetostatik}

\paragraph{Ström}
Ström definieras som $I = \dv{Q}{t}$.

\paragraph{Strömtäthet}
I fall där strömmen inte flödar längs med linjer utan längs ytor eller fritt i rummet definierar vi strömtätheten $\vb{J}$ som vektorfältet som beskriver flödet av laddningar. Strömtäthetens belopp ges av $J = \rho v$, där $\vb{v}$ är hastighetsfältet för laddningarna.

\paragraph{Kontinuitetsekvation för strömtätheten}
Strömtätheten uppfyller
\begin{align*}
	\del{t}{\rho} + \div{\vb{J}} = 0.
\end{align*}

\paragraph{Kraft på strömslingor}
Betrakta två slingor $C$ och $C'$. Genom varje slinga går en ström $I$ respektiva $I'$ i samma riktning som kurvans orientering. Experiment har visat att kraften på $C$ från $C'$ ges av
\begin{align*}
	\vb{F} = \integ{C}{}{\vb{r}}{I\times\left(\frac{\mu_{0}I'}{4\pi}\integ{C'}{}{\vb{r}'}{\times\frac{1}{R^{2}}\ub{R}}\right)}
\end{align*}
där $\vb{R} = \vb{r} - \vb{r}'$.

\paragraph{Magnetiska fältet}
Genom att definiera det magnetiska fältet från $C'$ i punkten $\vb{r}$ som
\begin{align*}
	\vb{B} = \frac{\mu_{0}I'}{4\pi}\integ{C'}{}{\vb{r}'}{\times\frac{1}{R^{2}}\ub{R}}
\end{align*}
fås
\begin{align*}
	\vb{F} = \integ{C}{}{\vb{r}}{I\times\vb{B}}.
\end{align*}

\paragraph{Magnetfält från strömtätheter}
För strömmar fördelade i rummet eller på en yta kan vi utvidga definitionen av magnetiska fältet till att bli
\begin{align*}
	\vb{B} = \frac{\mu_{0}}{4\pi}\integ{}{}{S}{\vb{J}\times\frac{1}{R^{2}}\ub{R}}\ \text{eller}\ \vb{B} = \frac{\mu_{0}}{4\pi}\integ{}{}{V}{\vb{J}\times\frac{1}{R^{2}}\ub{R}}.
\end{align*}

\paragraph{Potential för magnetfältet}
Vi har
\begin{align*}
	\vb{B} &= \frac{\mu_{0}}{4\pi}\integ{}{}{V}{\vb{J}\times\frac{1}{R^{2}}\ub{R}} \\
	       &= -\frac{\mu_{0}}{4\pi}\integ{}{}{V}{\vb{J}\times\grad{\frac{1}{R}}} \\
	       &= \curl{\frac{\mu_{0}}{4\pi}\integ{}{}{V}{\frac{1}{R}\vb{J}}}.
\end{align*}
Rotationsoperatorn kan tas utanför integrationen då den verkar på koordinater som det ej integreras över. Vi definierar då vektorpotentialen
\begin{align*}
	\vb{A} = \frac{\mu_{0}}{4\pi}\integ{}{}{V}{\frac{1}{R}\vb{J}},
\end{align*}
och har då
\begin{align*}
	\vb{B} = \curl{\vb{A}}.
\end{align*}

\paragraph{Entydighet för vektorpotentialen}
Vektorpotentialen kan även definieras som
\begin{align*}
	\vb{A} = \frac{\mu_{0}}{4\pi}\integ{}{}{V}{\frac{1}{R}\vb{J}} + \grad{\Lambda},
\end{align*}
där $\Lambda$ är en godtycklig funktion. Eftersom gradienter ej har rotation, kommer denna vektorpotentialen att ge samma magnetfält. Vi kommer oftast sätta $\Lambda = 0$.

\paragraph{Magnetfältets divergens}
Eftersom $\vb{B}$ är rotationen av en vektorpotential, gäller det att
\begin{align*}
	\div{\vb{B}} = 0.
\end{align*}

\paragraph{Vektorpotentialens divergens}
Vi har
\begin{align*}
	\div{\vb{A}} &= \div{\frac{\mu_{0}}{4\pi}\integ{}{}{V}{\frac{1}{R}\vb{J}}} \\
	             &= \frac{\mu_{0}}{4\pi}\integ{}{}{V}{\vb{J}\cdot\div{\frac{1}{R}}} \\
	             &= -\frac{\mu_{0}}{4\pi}\integ{}{}{V}{\vb{J}\cdot\div'{\frac{1}{R}}} \\
	             &= \frac{\mu_{0}}{4\pi}\integ{}{}{V}{\frac{1}{R}\div'{\vb{J}} - \div'{\frac{1}{R}\vb{J}}} \\
	             &= \frac{\mu_{0}}{4\pi}\integ{}{}{V}{\frac{1}{R}\div'{\vb{J}}} -  \frac{\mu_{0}}{4\pi}\vinteg{}{}{S}{\frac{1}{R}\vb{J}}.
\end{align*}
I elektrostatiska fall är alla laddningar statiska, och detta ger
\begin{align*}
	\div{\vb{A}} = 0.
\end{align*}

\paragraph{Magnetiskt flöde}
Det magnetiska flödet definieras som
\begin{align*}
	\Phi = \vinteg{}{}{S}{\vb{B}}.
\end{align*}
Med hjälp av Stokes' sats fås
\begin{align*}
	\Phi = \vinteg{}{}{r}{A}.
\end{align*}
Detta blir alltid $0$ genom en sluten yta.

\paragraph{Vektorpotentialens laplacian}
På samma sätt som för elektriska potentialen fås
\begin{align*}
	\laplacian{\vb{A}} = -\mu_{0}\vb{J}.
\end{align*}

\paragraph{Magnetfältets rotation}
Vi har
\begin{align*}
	\curl{\vb{B}} = \curl{\curl{\vb{A}}} = \grad{\div{\vb{A}}} - \laplacian{\vb{A}} = -\mu_{0}\vb{J}.
\end{align*}

\paragraph{Ampères cirkulationslag}
Strömmen genom en yta ges av
\begin{align*}
	I = \vinteg{}{}{S}{\vb{J}} = \frac{1}{\mu_{0}}\vinteg{}{}{S}{\curl{\vb{B}}} = \frac{1}{\mu_{0}}\vinteg{}{}{r}{\vb{B}}.
\end{align*}

\paragraph{Randvillkor för magnetfältet}
Som med elektriska fältet rör vi oss nära en ytströmtäthet. Kring denna lägger vi en yta och beräknar flödet genom den när ytans tjocklek blir liten. Detta ger
\begin{align*}
	B_{1}^{\perp} - B_{2}^{\perp} = 0.
\end{align*}
Vidare kan vi skriva
\begin{align*}
	\integ{}{}{V}{\mu_{0}\vb{J}} = \integ{}{}{V}{\curl{\vb{B}}}.
\end{align*}
Med indexräkning fås
\begin{align*}
	\left[\integ{}{}{V}{\curl{\vb{B}}}\right]_{i} = \varepsilon_{ijk}\integ{}{}{V}{\del{j}{B_{k}}} = \varepsilon_{ijk}\integ{}{}{S_{j}}{B_{k}} = \left[\integ{}{}{\vb{S}}{\times\vb{B}}\right]_{i}.
\end{align*}
Detta ger
\begin{align*}
	\vb{n}_{12}\times(\vb{B}_{1} - \vb{B}_{2}) = \mu_{0}\vb{J}.
\end{align*}
Ett motsvarande bevis kan även göras med Ampères lag för en liten strömslinga.