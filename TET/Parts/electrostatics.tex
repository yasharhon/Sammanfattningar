\section{Elektrostatik}

\paragraph{Coulombs lag}
Elektrostatiken utgår från Coulombs lag, som är en experimentellt framtagen lag. Den säjer att om två laddningar $Q$ och $q$ är separerade med en sträcka $\vb{R}$, är kraften mellan dem
\begin{align*}
	\vb{F} = \frac{1}{4\pi\varepsilon_{0}}\frac{qQ}{R^{3}}\vb{R}.
\end{align*}
Alternativt, i termer av enhetsvektorer,
\begin{align*}
	\vb{F} = \frac{1}{4\pi\varepsilon_{0}}\frac{qQ}{R^{2}}\ub{R}.
\end{align*}
Båda laddningarna antas ha försumbar utsträckning, och betecknas punktladdningar. $\varepsilon_{0}$ kallas vakuumpermittiviteten, och har enhet \si{\farad\per\meter}.

\paragraph{Elektriskt fält}
Det elektriska fältet som genereras av en laddning $Q$ definieras som att en liten testladdning $q$ upplever en kraft
\begin{align*}
	\vb{F} = q\vb{E}
\end{align*}
från $Q$. I våran definition skulle vi kunna lägga på ett $\lim\limits_{q\to 0}$.

Baserad på detta får vi att en punktladdning $Q$ genererar ett elektriskt fält
\begin{align*}
	\vb{E} = \frac{1}{4\pi\varepsilon_{0}}\frac{Q}{R^{2}}\ub{R}.
\end{align*}

Eftersom krafter superponeras, gör även elektriska fält det. I diskreta fall motsvarar detta att summera över punktladdningar. I kontinuerliga fall integrerar vi i stället, där varje element i integrationen behandlas som en punktladdning, och vi får
\begin{align*}
	\vb{E} = \frac{1}{4\pi\varepsilon_{0}}\integ{}{}{q}{\frac{1}{R^{2}}\ub{R}}.
\end{align*}
Laddningen kan vara spridd ut på en linje, en yta eller en volym, i vilka fall vi får $\dd{q} = \rho\dd{l}$, $\dd{q} = \sigma\dd{S}$ respektiva $\dd{q} = \lambda\dd{V}$. Förutom de olika elementerna finns en linjeladdningstäthet, ytladdningstäthet eller volymladdningstäthet.

\paragraph{Gauss' lag}
För att härleda Gauss' lag börjag vi med att titta på flödet av fältet $\frac{1}{R^{2}}\ub{R}$ genom en godtycklig yta, där $\vb{R}$ pekar från en utgångspunkt $\vb{r}'$ till ett givet ytelement. Integrationselementet
\begin{align*}
	\dd{\Omega} = \frac{\ub{R}\cdot\vb{\dd{S}}}{R^{2}}
\end{align*}
är rymdvinkeln som areaelementet upptar när det ses från origo. Vi kan se på något sätt att detta motsvarar att projicera areaelementet ned på enhetssfären kring origo. Alternativt, om kurvor är involverade, skulle man projicera ned på enhetscirkeln. Flödet vi betraktar ges då av fönsterfunktionen
\begin{align*}
	f(\vb{r}') = \integ{S}{}{S}{\frac{\ub{R}\cdot\ub{n}}{R^{2}}} = \integ{\Omega}{}{\Omega}{} =
	\begin{cases}
		4\pi, &\vb{r}'\text{ innanför }S, \\
		0,    &\vb{r}'\text{ utanför }S.
	\end{cases}
\end{align*}

Vi kommer nu ihåg hur elektriska fältet ser ut på integralform, specifikt som en volymintegral, och får då för flödet genom en godtycklig yta
\begin{align*}
	\vinteg{S}{}{S}{\vb{E}} &= \vinteg{S}{}{S}{\frac{1}{4\pi\varepsilon_{0}}\integ{}{}{V}{\frac{\rho}{R^{2}}\ub{R}}} \\
	                      &= \frac{1}{4\pi\varepsilon_{0}}\integ{}{}{V}{\integ{S}{}{S}{\rho\frac{\ub{R}\cdot\ub{n}}{R^{2}}}} \\
	                      &= \frac{1}{4\pi\varepsilon_{0}}\integ{}{}{V}{\rho f(\vb{r})} \\
	                      &= \frac{Q_{\text{innesluten}}}{\varepsilon_{0}}.
\end{align*}
Vektorn $\vb{R}$ är nu specifierad för varje punkt på ytan och i hela rummet. Den sista integralen är lika med laddningen som är innesluten i $S$ eftersom fönsterfunktionen ger ett bidrag $4\pi$ om och endast om det finns laddning i den aktuella punkten.

Gauss' lag är ett bra verktyg för att beräkna elektriska fält för geometrier med mycket symmetri.

\paragraph{Gauss' lag på integralform}
Betrakta nu en godtycklig yta $S$ som exakt inneslutar volymen $V$. Gauss' lag ger då
\begin{align*}
	\vinteg{S}{}{S}{\vb{E}} = \frac{1}{\varepsilon_{0}}\integ{V}{}{V}{\rho}.
\end{align*}
Vi kan använda divergenssatsen för att skriva om vänstersidan som en integral över $V$. Därmed kan vi dra slutsatsen
\begin{align*}
	\div{\vb{E}} = \frac{\rho}{\varepsilon_{0}}.
\end{align*}

\paragraph{Elektrostatisk potential}
Med vår kunnskap från vektoranalysen kan vi skriva
\begin{align*}
	\vb{E} = -\frac{1}{4\pi\varepsilon_{0}}\integ{}{}{V}{\rho\grad{\frac{1}{R}}} = -\grad{\left(\frac{1}{4\pi\varepsilon_{0}}\integ{}{}{V}{\rho\frac{1}{R}}\right)}.
\end{align*}
Vi definierar därmed den elektrostatiska potentialen enligt
\begin{align*}
	\vb{E} = -\grad{V}.
\end{align*}
Från våran definition ser vi att nollnivån för potentialen kan sättas arbiträrt, då det elektriska fältet (som är det som är fysikaliskt) inte ändras om potentialen ändras med en konstant. Vi brukar lägga nollnivån i oändligheten.

Vi kan även från detta visa att
\begin{align*}
	\curl{\vb{E}} = \vb{0}.
\end{align*}

\paragraph{Potential och elektrisk spänning}
Betrakta storheten $\vb{E}\cdot\dd{\vb{r}}$. Vi har
\begin{align*}
	\vb{E}\cdot\dd{\vb{r}} = E_{i}\dd{x_{i}} = -\del{i}{V}\dd{x_{i}} = -\dd{V}.
\end{align*}
Om vi nu jämför detta med den elektriska spänningen
\begin{align*}
	U_{12} = \vinteg{\vb{r}_{1}}{\vb{r}_{2}}{\vb{r}}{\vb{E}}
\end{align*}
mellan två punkter, kan vi se att detta blir
\begin{align*}
	U_{12} = \vinteg{\vb{r}_{1}}{\vb{r}_{2}}{\vb{r}}{\vb{E}} = -\integ{\vb{r}_{1}}{\vb{r}_{2}}{V}{} = V(\vb{r}_{1}) - V(\vb{r}_{2}),
\end{align*}
oberoende av vägen mellan punkterna. Om vi lägger potentialens referens i oändligheten, ser vi då att
\begin{align*}
	V(\vb{r}) = -\vinteg{\infty}{\vb{r}}{\vb{r}}{\vb{E}},
\end{align*}
ett typ inverst påstående av $\vb{E} = -\grad{V}$.