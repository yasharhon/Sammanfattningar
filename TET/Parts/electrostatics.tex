\section{Elektrostatik}

\paragraph{Coulombs lag}
Elektrostatiken utgår från Coulombs lag, som är en experimentellt framtagen lag. Den säjer att om två laddningar $Q$ och $q$ är separerade med en sträcka $\vb{R}$, är kraften mellan dem
\begin{align*}
	\vb{F} = \frac{1}{4\pi\varepsilon_{0}}\frac{qQ}{R^{3}}\vb{R}.
\end{align*}
Alternativt, i termer av enhetsvektorer,
\begin{align*}
	\vb{F} = \frac{1}{4\pi\varepsilon_{0}}\frac{qQ}{R^{2}}\ub{R}.
\end{align*}
Båda laddningarna antas ha försumbar utsträckning, och betecknas punktladdningar. $\varepsilon_{0}$ kallas vakuumpermittiviteten, och har enhet \si{\farad\per\meter}.

\paragraph{Elektriskt fält}
Det elektriska fältet som genereras av en laddning $Q$ definieras som att en liten testladdning $q$ upplever en kraft
\begin{align*}
	\vb{F} = q\vb{E}
\end{align*}
från $Q$. I våran definition skulle vi kunna lägga på ett $\lim\limits_{q\to 0}$.

Baserad på detta får vi att en punktladdning $Q$ genererar ett elektriskt fält
\begin{align*}
	\vb{E} = \frac{1}{4\pi\varepsilon_{0}}\frac{Q}{R^{2}}\ub{R}.
\end{align*}

Eftersom krafter superponeras, gör även elektriska fält det. I diskreta fall motsvarar detta att summera över punktladdningar. I kontinuerliga fall integrerar vi i stället, där varje element i integrationen behandlas som en punktladdning, och vi får
\begin{align*}
	\vb{E} = \frac{1}{4\pi\varepsilon_{0}}\integ{}{}{q}{\frac{1}{R^{2}}\ub{R}}.
\end{align*}
Laddningen kan vara spridd ut på en linje, en yta eller en volym, i vilka fall vi får $\dd{q} = \rho\dd{l}$, $\dd{q} = \sigma\dd{S}$ respektiva $\dd{q} = \lambda\dd{V}$. Förutom de olika elementerna finns en linjeladdningstäthet, ytladdningstäthet eller volymladdningstäthet. Notera att med hjälp av Diracs delta kan alla dessa fallen skrivas som volymladdningstätheter.

\paragraph{Gauss' lag}
För att härleda Gauss' lag börjag vi med att titta på flödet av fältet $\frac{1}{R^{2}}\ub{R}$ genom en godtycklig yta, där $\vb{R}$ pekar från en utgångspunkt $\vb{r}'$ till ett givet ytelement. Integrationselementet
\begin{align*}
	\dd{\Omega} = \frac{\ub{R}\cdot\vb{\dd{S}}}{R^{2}}
\end{align*}
är rymdvinkeln som areaelementet upptar när det ses från origo. Vi kan se på något sätt att detta motsvarar att projicera areaelementet ned på enhetssfären kring origo. Alternativt, om kurvor är involverade, skulle man projicera ned på enhetscirkeln. Flödet vi betraktar ges då av fönsterfunktionen
\begin{align*}
	f(\vb{r}') = \integ{S}{}{S}{\frac{\ub{R}\cdot\ub{n}}{R^{2}}} = \integ{\Omega}{}{\Omega}{} =
	\begin{cases}
		4\pi, &\vb{r}'\text{ innanför }S, \\
		0,    &\vb{r}'\text{ utanför }S.
	\end{cases}
\end{align*}

Vi kommer nu ihåg hur elektriska fältet ser ut på integralform, specifikt som en volymintegral, och får då för flödet genom en godtycklig yta
\begin{align*}
	\vinteg{S}{}{S}{\vb{E}} &= \vinteg{S}{}{S}{\frac{1}{4\pi\varepsilon_{0}}\integ{}{}{V}{\frac{\rho}{R^{2}}\ub{R}}} \\
	                      &= \frac{1}{4\pi\varepsilon_{0}}\integ{}{}{V}{\integ{S}{}{S}{\rho\frac{\ub{R}\cdot\ub{n}}{R^{2}}}} \\
	                      &= \frac{1}{4\pi\varepsilon_{0}}\integ{}{}{V}{\rho f(\vb{r})} \\
	                      &= \frac{Q_{\text{innesluten}}}{\varepsilon_{0}}.
\end{align*}
Vektorn $\vb{R}$ är nu specifierad för varje punkt på ytan och i hela rummet. Den sista integralen är lika med laddningen som är innesluten i $S$ eftersom fönsterfunktionen ger ett bidrag $4\pi$ om och endast om det finns laddning i den aktuella punkten.

Gauss' lag är ett bra verktyg för att beräkna elektriska fält för geometrier med mycket symmetri.

\paragraph{Gauss' lag på differentialform}
Betrakta nu en godtycklig yta $S$ som exakt inneslutar volymen $V$. Gauss' lag ger då
\begin{align*}
	\vinteg{S}{}{S}{\vb{E}} = \frac{1}{\varepsilon_{0}}\integ{V}{}{V}{\rho}.
\end{align*}
Vi kan använda divergenssatsen för att skriva om vänstersidan som en integral över $V$. Därmed kan vi dra slutsatsen
\begin{align*}
	\div{\vb{E}} = \frac{\rho}{\varepsilon_{0}}.
\end{align*}

\paragraph{Randvillkor för elektrist fält}
Ytladdningar ger diskontinuiteter i elektriskt fält. För att studera det, betrakta en punkt på ytan. Gör en liten låda kring punkten så att fältet är ungefär konstant på sidorna som inte rör ytan. Gauss' lag ger oss, om bara tar med de nämnda sidorna, att elektriksa fältets normalkomponent relativt ytan uppfyller
\begin{align*}
	E_{\text{ovan}}^{\perp}A - E_{\text{under}}^{\perp}A = \frac{\sigma A}{\varepsilon_{0}},
\end{align*}
där $A$ är sidornas yta. Positiv riktning för fältets normalkomponent är ut från ytan. Vid att låta lådan bli oändligt tunn kommer även de andra sidorna inte att ge något bidrag, varför det här måste stämma. Vi får därmed att
\begin{align*}
	E_{\text{ovan}}^{\perp} - E_{\text{under}}^{\perp} = \frac{\sigma}{\varepsilon_{0}}
\end{align*}

Vi kan även betrakta en liten fyrkantig slinga på samma sätt, med två sidor parallella med ytan och två normala på ytan. Eftersom integralen av elektriska fältet kring en sluten kurva alltid är $0$, får vi
\begin{align*}
	E_{\text{ovan}}^{\parallel} - E_{\text{under}}^{\parallel} = 0.
\end{align*}
Eftersom denna slingan kan ha vilken som helst orientering så länge man har två paralllella och två normala sidor, gäller det även på vektorform att
\begin{align*}
	\vb{E}_{\text{ovan}}^{\parallel} - \vb{E}_{\text{under}}^{\parallel} = \vb{0}.
\end{align*}

Dessa två resultat kan sammanfattas som
\begin{align*}
	\vb{E}_{\text{ovan}}^{\parallel} - \vb{E}_{\text{under}}^{\parallel} = \frac{\sigma}{\varepsilon_{0}}\vb{n}.
\end{align*}

\paragraph{Elektrostatisk potential}
Med vår kunnskap från vektoranalysen kan vi skriva
\begin{align*}
	\vb{E} = -\frac{1}{4\pi\varepsilon_{0}}\integ{}{}{V}{\rho\grad{\frac{1}{R}}} = -\grad{\left(\frac{1}{4\pi\varepsilon_{0}}\integ{}{}{V}{\rho\frac{1}{R}}\right)}.
\end{align*}
Vi definierar därmed den elektrostatiska potentialen enligt
\begin{align*}
	\vb{E} = -\grad{V}.
\end{align*}
Från våran definition ser vi att nollnivån för potentialen kan sättas arbiträrt, då det elektriska fältet (som är det som är fysikaliskt) inte ändras om potentialen ändras med en konstant. Vi brukar lägga nollnivån i oändligheten.

Vi kan även från detta visa att
\begin{align*}
	\curl{\vb{E}} = \vb{0}.
\end{align*}

\paragraph{Potential och elektrisk spänning}
Betrakta storheten $\vb{E}\cdot\dd{\vb{r}}$. Vi har
\begin{align*}
	\vb{E}\cdot\dd{\vb{r}} = E_{i}\dd{x_{i}} = -\del{i}{V}\dd{x_{i}} = -\dd{V}.
\end{align*}
Om vi nu jämför detta med den elektriska spänningen
\begin{align*}
	U_{12} = \vinteg{\vb{r}_{1}}{\vb{r}_{2}}{\vb{r}}{\vb{E}}
\end{align*}
mellan två punkter (som är den välkända spänningen vi känner från kretsvärlden), kan vi se att detta blir
\begin{align*}
	U_{12} = \vinteg{\vb{r}_{1}}{\vb{r}_{2}}{\vb{r}}{\vb{E}} = -\integ{\vb{r}_{1}}{\vb{r}_{2}}{V}{} = V(\vb{r}_{1}) - V(\vb{r}_{2}),
\end{align*}
oberoende av vägen mellan punkterna. Om vi lägger potentialens referens i oändligheten, ser vi då att
\begin{align*}
	V(\vb{r}) = -\vinteg{\infty}{\vb{r}}{\vb{r}}{\vb{E}},
\end{align*}
ett typ inverst påstående av $\vb{E} = -\grad{V}$. Vi ser även från detta att
\begin{align*}
	V(\vb{r}_{2}) - V(\vb{r}_{1}) = -\vinteg{\vb{r}_{1}}{\vb{r}_{2}}{\vb{r}}{\vb{E}}.
\end{align*}

\paragraph{Potential och arbete}
Antag att du vill förflytta en laddning $q$ i ett elektriskt fält. Den minsta kraften du måste verka med på laddningen för att göra detta är $\vb{F} = -q\vb{E}$, eftersom du arbetar mot det elektriska fältet. Arbetet du gör då är
\begin{align*}
	W = \vinteg{\vb{r}_{1}}{\vb{r}_{2}}{\vb{r}}{\vb{F}} = -q\vinteg{\vb{r}_{1}}{\vb{r}_{2}}{\vb{r}}{\vb{E}} = q(V(\vb{r}_{2}) - V(\vb{r}_{1})).
\end{align*}
Med andra ord är potentialskillnaden mellan två punkter lika med arbetet som måste göras för att förflytta en laddning från ena punkten till den andra per laddning.

\paragraph{Randvillkor för potentialen}
För att betrakta randvillkor för potentialen vid en ytladdning, kan man integrera elektriska fältet längs med en rak linje normalt på ytaddningen över ytan och låta linjen bli godtyckligt kort. Då försvinner integralen, och vi får att potentialen är kontinuerlig. Randvillkoret för elektriska fältet kan skrivas i termer av potentialen som
\begin{align*}
	\grad_{\vb{n}}V_{\text{över}} - \grad_{\vb{n}}V_{\text{under}} = -\frac{\sigma}{\varepsilon_{0}},
\end{align*}
där $\grad_{\vb{n}}$ är riktningsderivatan i normalriktningen.

\paragraph{Elektrostatisk energi}
Vi är nu intresserade av energin som krävs för att skapa en viss laddningsfördelning. Vi kommer därför beräkna energin som krävs för att transportera all laddningen från oändligheten och placera den på rätt sätt.

Vi börjar med att betrakta en samling laddningar som ska ligga på avstånd $r_{ij}$ från varandra. Att placera första laddningen på rätt plats kräver inget arbete. Att sen placera ut andra laddningen kommer kräva att man arbetar mot elektriska fältet från första. Man måste alltså göra ett arbete
\begin{align*}
	W_{2} = \frac{1}{4\pi\varepsilon_{0}}q_{2}\frac{q_{1}}{r_{12}}.
\end{align*}
På samma sätt måste laddning $3$ motarbeta elektriska fältet från både $1$ och $2$. Det totala arbetet är därmed
\begin{align*}
	W = \frac{1}{4\pi\varepsilon_{0}}\sum\limits_{i = 1}^{n}\sum\limits_{j < i}\frac{q_{i}q_{j}}{r_{ij}}.
\end{align*}
Eftersom $r_{ij} = r_{ji}$ kan vi nu skriva om den inre summan genom att i stället summera över alla andra partiklar än $i$, och lägga till en faktor $\frac{1}{2}$. Vi får då
\begin{align*}
	W = \frac{1}{8\pi\varepsilon_{0}}\sum\limits_{i = 1}^{n}\sum\limits_{j \neq i}\frac{q_{i}q_{j}}{r_{ij}}.
\end{align*}
Vid att ordna om faktorerna får vi
\begin{align*}
	W = \frac{1}{2}\sum\limits_{i = 1}^{n}q_{i}\sum\limits_{j \neq i}\frac{q_{j}}{4\pi\varepsilon_{0}r_{ij}} = \frac{1}{2}\sum\limits_{i = 1}^{n}q_{i}V_{i}(\vb{r}_{i}),
\end{align*}
där $V_{i}$ är potentialen som laddning $i$ känner av på grund av alla de andra laddingarna. Ett uttryck man hade kunnat gissa sig fram till från början.

Vi generaliserar vidare vår definition till kontinuerliga laddningfördelningar som
\begin{align*}
	W = \frac{1}{2}\integ{}{}{V}{\rho V}.
\end{align*}
Vi kan med hjälp av resultaten från tidigare skriva detta som
\begin{align*}
	W = \frac{1}{2}\integ{}{}{V}{V\varepsilon_{0}\div{\vb{E}}} = \frac{1}{2}\varepsilon_{0}\integ{}{}{V}{\div{V\vb{E}} - \vb{E}\cdot\grad{V}} = \frac{1}{2}\varepsilon_{0}\left(\vinteg{}{}{S}{V\vb{E}} + \integ{}{}{V}{E^{2}}\right).
\end{align*}

Nu kan vi fundera lite över integrationsdomänder. Om man tittar på den ursprungliga integralen, ger den inget bidrag där det inte finns laddning. Därför kan vi börja med att integrera exakt över området där det finns laddning. Om vi gör området större, kommer den ursprungliga integralen att vara oändrad. Däremot kommer integralen av elektriska fältets belopp öka, så ytintegralen måste minska motsvarande. Vi kan nu repetera processen tills vi integrerar över hela rummet. Då försvinner ytintegralen, vilket man kan argumentera lite bättre för, och kvar står
\begin{align*}
	W = \frac{1}{2}\varepsilon_{0}\integ{}{}{V}{E^{2}}.
\end{align*}

Det visar sig att om man använder resultatet för en laddningsfördelning på en diskret fördelning, får man inte samma svar. Detta är för att uttrycket för en diskret fördelning inte tar hänsyn till energin som krävs för att skapa punktladdningar till att börja med, vilket uttrycket för kontinuerliga fördelningar inkluderar. Denna finessen kom in i beräkningarna i övergången till kontinuerliga fördelningar, eftersom vi för diskreta fördelningar endast använde potentialen varje laddning känner på grund av alla andra. För kontinuerliga fördelningar är detta inte ett problem eftersom varje element har försvinnande liten laddning och därmed bidrar med försvinnande lite potential. För diskreta fördelningar gäller detta ej, dock.

\paragraph{Perfekta ledare}
En perfekt ledare har obegränsat med fria laddningar som kan röra sig i materialet. Från detta följer att
\begin{itemize}
	\item $\vb{E} = \vb{0}$ överallt inuti ledaren. Annars skulle någon av de fria laddningarna påverkas av fältet och röra sig sån att de kansellerade det.
	\item $\rho = \varepsilon_{0}\div{\vb{E}} = 0$ inuti ledaren. Därmed är alla fria laddningar på ytan.
	\item $V$ är konstant inuti ledaren.
	\item $\vb{E} = \frac{\sigma}{\varepsilon_{0}}\ub{n}$ precis utanför ledaren på grund av elektriska fältets randvillkor, alternativt eftersom tangentiella komponenter skulle transportera laddningar på ytan som skulle kansellera fältet.
\end{itemize}

\paragraph{Kraften på en ytladdning}
Kring en ytladdning är elektriska fältet diskontinuerligt, så hur beräknar man kraften på en sådan? Med hjälp av superposition kan elektriska fältet skrivas som en summa av bidrag från själva laddningen och allt annat. Denna termen är kontinuerlig i ytan eftersom man skulle kunna ta bort ytladdningen utan att ändra den. Vidare ger randvillkoren att fältet på varje sida skiljer sig med en term $\pm\frac{\sigma}{2\varepsilon_{0}}\ub{n}$, som försvinner om man tar medelvärdet. Alltså ges kraften av laddningen gånger medelärdet av fältet på varje sida.

\paragraph{Kraften på en ledare}
Betrakta en ledare som specialfall. Här får vi en krafttäthet
\begin{align*}
	\vb{f}  = \frac{\sigma^{2}}{2\varepsilon_{0}}\ub{n},
\end{align*}
som motsvarar ett tryck utåt på laddaren - oberoende av vilken sorts ytladdning man har! I termer av elektriska fältet kan trycket skrivas som
\begin{align*}
	p = \frac{1}{2}\varepsilon_{0}E^{2}.
\end{align*}

\paragraph{Kapacitans}
Betrakta först två ledare med olika laddningar $\pm q$. Man kan se av integraluttrycket för elektriska fältet att det är proportionellt mot $q$. Eftersom potentialskillnaden mellan ledarna är en integral av elektriska fältet, är även denna proportionell mot $q$. Vi definierar därmed systemets kapacitans som proportionalitetskonstanten mellan de två, alltså
\begin{align*}
	Q = CV.
\end{align*}

Betrakta nu ett system av olika ledare, med var sin potential och laddning. Vi har även någon potentialreferens. Vi kan börja med att sätta alla potentialer förutom en till $0$, och räkna ut alla $Q_{i}$. Vid att superponera alla dina resultat får du en mängd slutgiltiga samband på formen $Q_{i} = C_{ij}V_{j}$. Detta definierar kapacitansmatrisen. Man kan visa/argumentera för att matrisen är symmetrisk och positivt definit.

\paragraph{Energin för ett system av ledare}
För ett system av ledare har vi
\begin{align*}
	W &= \frac{1}{2}\integ{}{}{V}{\rho V} \\
	  &= \frac{1}{2}\integ{\text{ledare }i}{}{S}{\sigma_{i}V_{i}} \\
	  &= \frac{1}{2}V_{i}\integ{\text{ledare }i}{}{S}{\sigma_{i}} \\
	  &= \frac{1}{2}V_{i}Q_{i}.
\end{align*}
Om vi använder kapacitansmatrisen får vi
\begin{align*}
	W = \frac{1}{2}V_{i}C_{ij}V_{j}.
\end{align*}
För en enda kondensator blir detta
\begin{align*}
	W = \frac{1}{2}CV^{2} = \frac{1}{2C}Q^{2}.
\end{align*}

\paragraph{Poissons ekvation}
Om vi tittar på våra resultat, får vi nu Poissons ekvation
\begin{align*}
	\laplacian{V} = -\frac{\rho}{\varepsilon_{0}}.
\end{align*}

\paragraph{Entydighetssats för potentialen}
Betrakta en region $H$ med känd laddningstäthet som innehåller tre regioner avgränsade av ytorna $S_{D}$, $S_{N}$ och $S_{Q}$ (med normalvektorerna pekande in mot de avgränsade regionerna). På $S_{D}$ är $V = V_{\text{S}}$ känd. På $S_{N}$ är $\grad_{\vb{n}}{V} = -E_{\vb{n}}$ känd. $S_{Q}$ är en perfekt ledare med känd total ytladdning $Q$. Vi vill försöka visa att elektriska fältet är entydigt i $H$.

% TODO: Varför sista?
För att visa detta, antag att vi har två lösningar $V_{1}$ och $V_{2}$ och bilda $V_{0} = V_{1} - V_{2}$. Då vet vi att $\laplacian{V_{0}} = 0$ i $H$, $V_{0} = 0$ på $S_{D}$ och $S_{Q}$, att $\grad_{\vb{n}}{V_{0}} = 0$ på $S_{N}$ och $\integ{S_{Q}}{}{S}{\grad_{\vb{n}}{V_{0}}} = 0$. Detta stämmer eftersom
\begin{align*}
	\integ{S_{Q}}{}{S}{\sigma} = \integ{S_{Q}}{}{S}{\sigma}
\end{align*}
Vi får därmed
\begin{align*}
	\integ{S_{D} + S_{Q} + S_{N}}{}{S}{V_{0}\grad_{\vb{n}}{V_{0}}} = \integ{H}{}{V}{V_{0}\laplacian{V_{0}} + \abs{\grad{V_{0}}}^{2}} = \integ{H}{}{V}{\abs{\grad{V_{0}}}^{2}}.
\end{align*}
På $S_{D}$ och $S_{N}$ är en av faktorerna i integranden lika med $0$, så vi behöver endast betrakta $S_{Q}$. Här har vi att $V_{0}$ måste vara konstant på ytan, vilket ger
\begin{align*}
	\integ{S_{Q}}{}{S}{V_{0}\grad_{\vb{n}}{V_{0}}} = V_{0}\integ{S_{Q}}{}{S}{\grad_{\vb{n}}{V_{0}}} = 0,
\end{align*}
vilket implicerar
\begin{align*}
	\grad{V} = \vb{E}_{2} - \vb{E}_{1} = \vb{0},
\end{align*}
och beviset är klart.

\paragraph{Speglingsmetoder}
Vissa problem kan lösas med speglingsmetoder. Då kan man ersätta vissa komponenter av ett problem med andra på ett sådant sätt att randvillkor som ges i problemet fortfarande är uppfylda.

\paragraph{Spegling och potentialen från två linjeladdningar}
Vi kommer behöva lite standardlösningar för att använda när vi speglar problem. Vi betraktar därför först två oändligt långa parallella linjeladdningar med laddning $\pm\lambda$ per längd separerade med ett avstånd $2h$. Problemet är tvådimensionellt, och vi inför $\vb{s}$ som vektorn från punkten mitt emellan laddningarna till en godtycklig punkt i planet. Elektriska fältet från laddningen till höger, som vi döper nummer $1$, ges av $\vb{E} = \frac{\lambda}{2\pi\varepsilon_{0}\abs{\vb{s} - \vb{s}_{1}}}\ub{s}'$, där $\vb{s}_{1} = h\ub{x}$ och $\ub{s}'$ pekar i samma riktning som $\vb{s} - \vb{s}_{1}$. Vi definierar $V = 0$ där $\vb{s} = 0$ och får
\begin{align*}
	V(\vb{s}) = -\vinteg{\vb{0}}{\vb{s}}{s'}{\frac{\lambda}{2\pi\varepsilon_{0}\abs{\vb{s}' - \vb{s}_{1}}}\ub{s}'} = -\vinteg{-\vb{s}_{1}}{\vb{s} - \vb{s}_{1}}{u'}{\frac{\lambda}{2\pi\varepsilon_{0}u'}\ub{u}'}.
\end{align*}
Vi har rotationssymmetri i planet, och kan därmed välja en radiell riktning för integrationen, vilket ger
\begin{align*}
	V(\vb{s}) = -\frac{\lambda}{2\pi\varepsilon_{0}}\ln{\frac{\abs{\vb{s} - \vb{s}_{1}}}{\abs{-\vb{s}_{1}}}} = -\frac{\lambda}{2\pi\varepsilon_{0}}\ln{\frac{\abs{\vb{s} - \vb{s}_{1}}}{h}}.
\end{align*}
Den totala potentialen är därmed
\begin{align*}
	V(\vb{s}) = \frac{\lambda}{2\pi\varepsilon_{0}}\ln{\frac{\abs{\vb{s} - \vb{s}_{2}}}{\abs{\vb{s} - \vb{s}_{1}}}}.
\end{align*}

Ekvipotentialytorna uppfyller $V = V_{0}$, vilket ger
\begin{align*}
	\frac{\abs{\vb{s} - \vb{s}_{2}}}{\abs{\vb{s} - \vb{s}_{1}}} = e^{\frac{2\pi\varepsilon_{0}V_{0}}{\lambda}}.
\end{align*}
Vi definierar $u = \frac{2\pi\varepsilon_{0}V_{0}}{\lambda}$ och skriver om avstånden på vänstersidan för att få
\begin{align*}
	\frac{(x + h)^{2} + y^{2}}{(x - h)^{2} + y^{2}} = e^{2u}.
\end{align*}
Detta ger
\begin{align*}
	x^{2} + 2xh + h^{2} + y^{2}                                                          &= e^{2u}(x^{2} - 2xh + h^{2} + y^{2}), \\
	x^{2}(1 - e^{2u}) + 2xh(1 + e^{2u}) + y^{2}(1 - e^{2u})                              &= h^{2}(e^{2u} - 1), \\
	x^{2}(e^{-u} - e^{u}) + 2xh(e^{-u} + e^{u}) + y^{2}(e^{-u} - e^{u})                  &= h^{2}(e^{u} - e^{-u}), \\
	-x^{2}\sinh{u} + 2xh\cosh{u} - y^{2}\sinh{u}                                         &= h^{2}\sinh{u}, \\
	x^{2} - 2xh\coth{u} + h^{2} + y^{2}                                                  &= 0, \\
	x^{2} - 2xh\coth{u} + h^{2}\left(\coth^{2}{u} - \frac{1}{\sinh^{2}{u}}\right) + y^{2}          &= 0, \\
	(x - h\coth{u})^{2} + y^{2}                                                          &= \frac{h^{2}}{\sinh^{2}{u}}
\end{align*}
Det är alltså en cirkel med centrum i $h\coth{u}\ub{x}$ och radie $a = \frac{h}{\abs{\sinh{u}}}$. Avstånden från cirkelns centrum till de två laddningarna är
\begin{align*}
	d_{1} = h\abs{\coth{u} - 1},\ d_{2} = h\abs{\coth{u} + 1}.
\end{align*}
Eftersom $\abs{\coth{u}} > 1$, får vi
\begin{align*}
	d_{1}d_{2} = h^{2}(\coth^{2}{u} - 1) = \frac{h^{2}}{\sinh^{2}{u}} = a^{2}.
\end{align*}

Speglingsstrategin är nu att om du har en linjeladdning $\lambda$ parallell med en ledande cirkulärcylindrisk yta med radien $a$ och laddning $-\lambda$ per längd, och avståndet från cylinderaxeln till linjeladdningen är $d$, kan cylinderytan ersättas med en linjeladdning $\lambda_{\text{s}} = -\lambda$ ett avstånd $d_{\text{s}} = \frac{a^{2}}{d}$ från cylinderaxeln.

\paragraph{Spegling och potentialen från två punktladdningar}
Betrakta två punktladdningar liggande på $z$-axeln, där den översta, döpt nummer 1, ligger i punkten $h\ub{z}$ och den andra i origo. Problemet är cylindersymmetriskt, så vi inför cylinderkoordinater. Potentialen är då
\begin{align*}
	V = \frac{1}{4\pi\varepsilon_{0}}\left(\frac{q_{1}}{\sqrt{\rho^{2} + (z - h)^{2}}} + \frac{q_{2}}{\sqrt{\rho^{2} + z^{2}}}\right).
\end{align*}
Ekvipotentialytorna för en nollskild potential är komplicerade, men ekvipotentialytan för $V = 0$ ges av
\begin{align*}
	\frac{q_{1}}{\sqrt{\rho^{2} + (z - h)^{2}}} + \frac{q_{2}}{\sqrt{\rho^{2} + z^{2}}} &= 0, \\
	k = \frac{q_{1}}{q_{2}}                                                             &= -\frac{\sqrt{\rho^{2} + (z - h)^{2}}}{\sqrt{\rho^{2} + z^{2}}}, \\
	\rho^{2}(k^{2} - 1) + k^{2}z^{2}                                                         &= (z - h)^{2}, \\
	z^{2}(1 - k^{2}) - 2zh + h^{2}                                                      &= \rho^{2}(k^{2} - 1), \\
	\rho^{2} + z^{2} - \frac{2zh}{1 - k^{2}} + \frac{h^{2}}{1 - k^{2}}                  &= 0, \\
	\rho^{2} + \left(z - \frac{h}{1 - k^{2}}\right) - \frac{h^{2}}{(1 - k^{2})^{2}} + \frac{h^{2}}{1 - k^{2}} &= 0, \\
	\rho^{2} + \left(z - \frac{h}{1 - k^{2}}\right)                                     &= \frac{h^{2}k^{2}}{(1 - k^{2})^{2}}.
\end{align*}
Detta är en sfär med centrum i $\frac{1}{1 - k^{2}}h\ub{z}$ och radius $a = h\abs{\frac{k}{1 - k^{2}}}$. Notera att det är en förutsättning att $k < 0$, alltså att laddningarna har olika tecken.

Om vi nu antar $k^{2} > 1$ ligger sfärens centrum under laddning $2$. Avstånden från sfärens centrum till de två punktladdningarna är då
\begin{align*}
	d_{2} = \frac{1}{k^{2} - 1}h,\ d_{1} = h + d_{2} = \frac{k^{2}}{k^{2} - 1}h.
\end{align*}
Detta ger
\begin{align*}
	d_{1}d_{2} = a^{2},\ k = -\sqrt{\frac{d_{1}}{d_{2}}} = -\frac{d_{1}}{a} = -\frac{a}{d_{2}}.
\end{align*}

Speglingsstrategin är nu att om du har en punktladdning $q$ ett avstånd $d$ från centrum av en jordad ledande sfärisk yta med radien $a$, kan ledaren ersättas med en punktladdning $q_{s} = -q\frac{a}{d}$ ett avstånd $d_{s} = \frac{a^{2}}{d}$ från sfärens centrum (bort från punktladdningen).

\paragraph{Laplace' ekvation i sfäriska koordinater}
Vid att ställa upp Poissons ekvation i sfäriska koordinater på ett laddningsfritt domän som avgränsas av två sfäriska skal får man Laplace' ekvation. Den har allmän lösning på formen
\begin{align*}
	V = \sum\limits_{l = 0}^{\infty}\sum\limits_{m = -l}^{l}\left(A_{lm}r^{l} + \frac{B_{lm}}{r^{l + 1}}\right)Y_{lm}(\theta, \phi),
\end{align*}
där $Y_{lm}$ är klotytefunktionerna. $A_{lm}$ bestäms av laddningarna utanför det yttre skalet och $B_{lm}$ av laddningarna innanför det inre skalet.

Vi har allmänt att
\begin{align*}
	Y_{lm}(\theta, \phi)\propto P_{l}^{m}(\cos{\theta})e^{im\phi},
\end{align*}
där $P_{l}^{m}$ är Legendrepolynomen. I fall som är rotationssymmetriska med avseende på $xy$-planet kan lösningen därför förenklas till
\begin{align*}
	V = \sum\limits_{l = 0}^{\infty}\left(A_{l}r^{l} + \frac{B_{l}}{r^{l + 1}}\right)P_{l}^{0}(\cos{\theta}).
\end{align*}

\paragraph{Anpassningsmetod}
Betrakta ett fall likt fallet ovan där vi även känner $V$ på $z$-axeln. Eftersom $P_{l}(1) = 1$ och $P_{l}(-1) = (-1)^{l}$ får vi längs $z$-axeln
\begin{align*}
	V = \sum\limits_{l = 0}^{\infty}\left(A_{l}\abs{z}^{l} + \frac{B_{l}}{\abs{z}^{l + 1}}\right)
	\begin{cases}
		1,        &z > 0, \\
		(-1)^{l}, &z < 0.
	\end{cases}
\end{align*}

\paragraph{Elektriska dipolen}
Betrakta två punktladdningar på en linje. Linjen går genom origo, och laddningarna ligger lika långa avstånd $\frac{1}{2}d$ från origo. Potentialen i punkten $\vb{r}$, som ligger ett avstånd $R_{+}$ respektiva $R_{-}$ från de två laddningarna, ges av
\begin{align*}
	V = \frac{q}{4\pi\varepsilon_{0}R_{+}} - \frac{q}{4\pi\varepsilon_{0}R_{-}} = \frac{q}{4\pi\varepsilon_{0}}\frac{R_{-} - R_{+}}{R_{+}R_{-}}.
\end{align*}
Om $r >> d$ fås
\begin{align*}
	V \approx \frac{q}{4\pi\varepsilon_{0}}\frac{d\cos{\theta}}{r^{2}}
\end{align*}
där $\theta$ är vinkeln mellan $\vb{r}$ och linjen. Vi kan då skriva detta som
\begin{align*}
	V = \frac{q}{4\pi\varepsilon_{0}}\frac{\vb{d}\cdot\ub{r}}{r^{2}}.
\end{align*}
Vi definierar nu dipolmomentet $\vb{p} = q\vb{d}$, och får då
\begin{align*}
	V = \frac{\vb{p}\cdot\ub{r}}{4\pi\varepsilon_{0}r^{2}}.
\end{align*}

Notera att vi kan skriva dipolmomentet som $\vb{p} = \sum q_{i}\vb{r}_{i}$.

\paragraph{Ideala dipoler}
En ideal dipol fås i gränsen för en dipol när $d$ blir oändligt liten på ett sådant sätt att $\vb{p}$ hålls konstant.

\paragraph{Fältet från en dipol}
Fältet från en dipol ges av
\begin{align*}
	\vb{E} = -\grad{\frac{\vb{p}\cdot\ub{r}}{4\pi\varepsilon_{0}r^{2}}} = -\grad{\frac{\vb{p}\cdot\vb{r}}{4\pi\varepsilon_{0}r^{3}}}.
\end{align*}
I nämnaren har vi
\begin{align*}
	\vb{p}\cdot\vb{r} = p_{i}r_{i} \implies \del{i}{\vb{p}\cdot\vb{r}} = p_{i} \implies \grad{\vb{p}\cdot\vb{r}} = \vb{p}.
\end{align*}
Vi får då
\begin{align*}
	\vb{E} = -\frac{1}{4\pi\varepsilon_{0}}\left(\frac{\vb{p}}{r^{3}} - \frac{3\vb{p}\cdot\vb{r}}{r^{4}}\grad{r}\right) = \frac{1}{4\pi\varepsilon_{0}r^{3}}\left(3(\vb{p}\cdot\ub{r})\ub{r} - \vb{p}\right).
\end{align*}

\paragraph{Dipolmoment för en laddningsfördelning}
För en laddningsfördelning ges dipolmomentet av
\begin{align*}
	\vb{p} = \integ{}{}{V}{\vb{r}\rho}.
\end{align*}

\paragraph{Förflyttning av koordinatsystem och dipolmoment}
Vid förflyttning av origo en sträcka $\vb{a}$ fås
\begin{align*}
	\vb{p}' = \integ{}{}{V}{\vb{r}'\rho} = \integ{}{}{V}{(\vb{r} - \vb{a})\rho} = \vb{p} - q\vb{a}.
\end{align*}
Alltså beror dipolmomentet av origos position om det finns en netto mängd laddning i systemet.

\paragraph{Multipolutveckling}
Betrakta en punkt $\vb{r}$ och en annan punkt $\vb{r}'$. Vid att definiera $\vb{R} = \vb{r} - \vb{r}'$ får vi att om de två punkterna inte är samma, är $\laplacian{\frac{1}{R}} = 0$ överallt förutom där $\vb{R} = \vb{0}$. Vid att lägga vårat koordinatsystem så att $\vb{r}'$ är parallell med $z$-axeln och definiera vinkeln mellan $\vb{r}$ och $\vb{r}'$ som $\gamma$ fås
\begin{align*}
	\frac{1}{R} =
	\begin{cases}
		\sum\limits_{l = 0}^{\infty}A_{l}r^{l}P_{l}(\cos{\gamma}),          &r < r', \\
		\sum\limits_{l = 0}^{\infty}\frac{B_{l}}{r^{l}}P_{l}(\cos{\gamma}), &r > r'.
	\end{cases}
\end{align*}
Speciellt, på $z$-axeln är $\gamma = 0$ och
\begin{align*}
	\frac{1}{R} = \frac{1}{r_{>} - r_{<}} = \frac{1}{r_{>}}\left(1 - \frac{r_{<}}{r_{>}}\right)^{-1} = \frac{1}{r_{>}}\sum\limits_{l = 0}^{\infty}\left(\frac{r_{<}}{r_{>}}\right)^{l} = \sum\limits_{l = 0}^{\infty}\frac{r_{<}^{l}}{r_{>}^{l + 1}},
\end{align*}
där $r_{<} = \min(r', r)$ och $r_{>} = \max(r', r)$. Vi utvidgar därmed lösningen till
\begin{align*}
	\frac{1}{R} = \sum\limits_{l = 0}^{\infty}\frac{r_{<}^{l}}{r_{>}^{l + 1}}P_{l}(\cos{\gamma}).
\end{align*}

Vi söker nu potentialen utanför en sfär som omsluter en rumladdning. Vid att låta $\vb{r}$ peka utanför sfären och $\vb{r}'$ inuti fås $r_{<} = r',\ r_{>} = r$ och
\begin{align*}
	V = \frac{1}{4\pi\varepsilon_{0}}\integ{}{}{V'}{\frac{\rho}{R}} = \frac{1}{4\pi\varepsilon_{0}}\sum\limits_{l = 0}^{\infty}\frac{1}{r^{l + 1}}\integ{}{}{V'}{(r')^{l}\rho P_{l}(\cos{\gamma})} = \sum\limits_{l = 0}^{\infty}V_{l}.
\end{align*}
De olika $V_{l}$ kommer ge oss termer som ser ut som olika multipoler, och vi vill nu studera dem. Vi noterar först att om $e_{i}$ respektiva $e_{i}'$ är komponenterna av $\ub{r}$ respektiva $\ub{r}'$, kan vi skriva
\begin{align*}
	\cos(\gamma) &= e_{i}e_{i}',\ \cos[2](\gamma) = e_{i}e_{i}'e_{i}e_{j}',\ 1 = e_{i}e_{i} = e_{i}e_{j}\delta_{ij}.
\end{align*}

För $l = 0$ får vi
\begin{align*}
	V_{0} = \frac{1}{4\pi\varepsilon_{0}}\frac{1}{r}\integ{}{}{V'}{\rho},
\end{align*}
alltså ett bidrag motsvarande en punktladdning med samma totala laddning i origo. För $l = 1$ fås
\begin{align*}
	V_{1} &= \frac{1}{4\pi\varepsilon_{0}}\frac{1}{r^{2}}\integ{}{}{V'}{r'\rho P_{1}(\cos{\gamma})} \\
	      &= \frac{1}{4\pi\varepsilon_{0}r^{2}}\integ{}{}{V'}{r'\cos{\gamma}\rho} \\
	      &= \frac{1}{4\pi\varepsilon_{0}r^{2}}\integ{}{}{V'}{r'e_{i}e_{i}'\rho} \\
	      &= \frac{e_{i}}{4\pi\varepsilon_{0}r^{2}}\integ{}{}{V'}{r_{i}'\rho} \\
	      &= \frac{1}{4\pi\varepsilon_{0}r^{2}}\ub{r}\cdot\integ{}{}{V'}{r_{i}'\rho}
\end{align*}
alltså ett bidrag motsvarande en dipol med samma totala dipolmoment i origo. För $l = 2$ fås
\begin{align*}
	V_{2} &= \frac{1}{4\pi\varepsilon_{0}}\frac{1}{r^{3}}\integ{}{}{V'}{r'^{2}\rho P_{2}(\cos{\gamma})} \\
	      &= \frac{1}{8\pi\varepsilon_{0}r^{3}}\integ{}{}{V'}{r_{i}'r_{i}'\rho(3\cos[2](\gamma) - 1)} \\
	V_{2} &= \frac{1}{8\pi\varepsilon_{0}r^{3}}\integ{}{}{V'}{r_{k}'r_{k}'\rho(3\cos[2](\gamma) - 1)}  \\
	      &= \frac{1}{8\pi\varepsilon_{0}r^{3}}\integ{}{}{V'}{r_{k}'r_{k}'\rho(3e_{i}e_{i}'e_{i}e_{j}' - e_{i}e_{j}\delta_{ij})} \\
	      &= \frac{1}{8\pi\varepsilon_{0}r^{3}}e_{i}e_{j}\integ{}{}{V'}{r_{k}'r_{k}'\rho(3e_{i}'e_{j}' - \delta_{ij})} \\
	      &= \frac{1}{8\pi\varepsilon_{0}r^{3}}e_{i}e_{j}\integ{}{}{V'}{\rho(3r_{i}'r_{j}' - (r')^{2}\delta_{ij})}.
\end{align*}
Vi definierar nu kvadrupolmomentstensorn
\begin{align*}
	Q_{ij} = \frac{1}{2}\integ{}{}{V'}{\rho(3r_{i}'r_{j}' - (r')^{2}\delta_{ij})},
\end{align*}
vilket ger
\begin{align*}
	V = \frac{e_{i}Q_{ij}e_{j}}{4\pi\varepsilon_{0}r^{3}}.
\end{align*}
Detta är kvadrupolbidraget.

Allmänt blir det $l$:te bidraget på formen
\begin{align*}
	V_{l} = \frac{Q_{i_{1}\dots i_{l}}e_{i_{1}}\dots e_{i_{l}}}{4\pi\varepsilon_{0}r^{l + 1}}.
\end{align*}

\paragraph{Additionssatsen}
Hellre än att hantera komponenterna av kvadrupolmomentstensorn, använder vi en sats som säjer
\begin{align*}
	P_{l}(\cos{\gamma}) = \frac{4\pi}{2l + 1}\sum\limits_{m = -l}^{l}Y_{lm}(\theta, \phi)Y_{lm}^{*}(\theta', \phi').
\end{align*}
Då kan vi skriva
\begin{align*}
	\sum\limits_{l = 0}^{\infty}V_{l},\ V_{l} = \frac{1}{(2l + 1)\varepsilon_{0}r^{l + 1}}\sum\limits_{m = -l}^{l}q_{lm}Y_{lm}(\theta, \phi),
\end{align*}
där $q_{lm}$ är det sfäriska multipolmomentet
\begin{align*}
	q_{lm} = \integ{}{}{V*}{\rho (r')^{l}})Y_{lm}^{*}(\theta', \phi').
\end{align*}
Varje $V_{l}$ har alltså $2l + 1$ oberoende komponenter, vilket på grund av multipolmomenttensorernas symmetri och spårlöshet är lika med antalet oberoende komponenter i multipolmomenttensorn.

\paragraph{Kraft på en laddningsfördelning}
Betrakta en laddningsfördelning i ett yttre måttligt varierande elektriskt fält. Kraften på laddningsfördelningen ges då av
\begin{align*}
	\vb{F} = \integ{}{}{V}{\rho\vb{E}}.
\end{align*}
Vi vill nu approximera elektriska fältet i laddningsfördelningen vid att utveckla den kring en referenspunkt $\vb{r}_{0}$. Komponentvis har vi
\begin{align*}
	E_{i} \approx E_{i, 0} + (r_{j} - r_{j, 0})\del{j}{E_{i}},
\end{align*}
varför
\begin{align*}
	\vb{E} \approx \vb{E}_{0} + ((\vb{r} - \vb{r}_{0})\cdot\grad)\vb{E}
\end{align*}
och
\begin{align*}
	\vb{F} \approx \integ{}{}{V}{\rho\left(\vb{E}_{0} + ((\vb{r} - \vb{r}_{0})\cdot\grad)\vb{E}\right)}.
\end{align*}
Vidare, under antagandet att fältet varierar måttligt kan alla derivator antas vara konstanta, vilket ger
\begin{align*}
	\vb{F} \approx \integ{}{}{V}{\rho\vb{E}_{0}} + \integ{}{}{V}{\rho((\vb{r} - \vb{r}_{0})\cdot\grad)\vb{E}} = Q\vb{E}_{0} + (\vb{p}\cdot\grad)\vb{E},
\end{align*}
där dipolmomentet mäts relativ $\vb{r}_{0}$. Att dra ut elektriska fältet i andra termen från integralen är helt okej - man kan tänka sig att integrationen skapar en operator som sedan får verka på fältet.

\paragraph{Vridmoment på en laddningsfördelning}
På en punktladdning har vi
\begin{align*}
	\vb{M} = \vb{r}\times\vb{F} = q\vb{r}\times\vb{E}.
\end{align*}
För en laddningsfördelning fås då, relativt en referenspunkt $O$, vridmomentet
\begin{align*}
	\vb{M} &= \integ{}{}{V}{\rho\vb{r}\times\vb{E}} \\
	       &\approx \integ{}{}{V}{\rho((\vb{r} - \vb{r}_{0}) + \vb{r}_{0})\times(\vb{E}_{0} + ((\vb{r} - \vb{r}_{0})\cdot\grad)\vb{E})}.
\end{align*}
Vi försummar nu termer av andra ordningen i $\vb{r} - \vb{r}_{0}$ och får
\begin{align*}
	\vb{M} &= \integ{}{}{V}{\rho((\vb{r} - \vb{r}_{0})\times\vb{E}_{0} + \vb{r}_{0}\times(\vb{E}_{0} + ((\vb{r} - \vb{r}_{0})\cdot\grad)\vb{E}))} \\
	       &= \left(\integ{}{}{V}{\rho((\vb{r} - \vb{r}_{0})}\right)\times\vb{E}_{0} + \vb{r}_{0}\times\integ{}{}{V}{\vb{E}_{0} + ((\vb{r} - \vb{r}_{0})\cdot\grad)\vb{E})} \\
	       &= \vb{p}\times\vb{E}_{0} + \vb{r}_{0}\times\vb{F}.
\end{align*}

\paragraph{Polarisation}
Polarisationen $\vb{P}$ uppfyller
\begin{align*}
	\vb{p} = \integ{}{}{V}{\vb{P}}.
\end{align*}

\paragraph{Polariserbarhet}
Polariserbarheten $\alpha$ uppfyller
\begin{align*}
	\vb{p} = \alpha\vb{E}.
\end{align*}

\paragraph{Fält och potential från polarisation}
Från ett litet rymdelement fås
\begin{align*}
	\dd{\vb{p}} = \vb{P}\dd{V},\ \dd{V} = \frac{1}{4\pi\varepsilon_{0}}\frac{\dd{\vb{p}}\cdot\ub{R}}{R^{2}},\ V = \frac{1}{4\pi\varepsilon_{0}}\integ{}{}{V}{\frac{\vb{P}\cdot\ub{R}}{R^{2}}}.
\end{align*}
Elektriska fältet kommer ha en singularitet som beter sig som $\frac{1}{R^{3}}$. Denna kommer inte upphävas av volymelementet, och kräver specialbehandling.

Vi kommer dock lösa detta genom att använda att
\begin{align*}
	\grad'{\frac{1}{R}} = \grad'{\frac{1}{\abs{\vb{r} - \vb{r}'}}} = \frac{1}{R^{2}}\ub{R},
\end{align*}
vilket ger
\begin{align*}
	V &= \frac{1}{4\pi\varepsilon_{0}}\integ{}{}{V}{\vb{P}\cdot\grad'{\frac{1}{R}}} \\
	  &= \frac{1}{4\pi\varepsilon_{0}}\integ{}{}{V}{\div'{\frac{\vb{P}}{R^{2}}} - \frac{1}{R}\div'{\vb{P}}} \\
	  &= \frac{1}{4\pi\varepsilon_{0}}\vinteg{}{}{S}{\frac{\vb{P}}{R}} - \frac{1}{4\pi\varepsilon_{0}}\integ{}{}{V}{\frac{1}{R}\div'{\vb{P}}},
\end{align*}
vilket motsvara coulombpotentialen från två ekvivalenta laddningsfördelningar
\begin{align*}
	\rho = -\div'{\vb{P}},\ \sigma = \vb{P}\cdot\ub{n}.
\end{align*}
Nu kan vi beräkna elektriska fältet som
\begin{align*}
	\vb{E} = \frac{1}{4\pi\varepsilon_{0}}\integ{}{}{S}{\frac{\sigma}{R^{2}}\ub{R}} + \frac{1}{4\pi\varepsilon_{0}}\integ{}{}{V}{\rho\frac{1}{R^{2}}\ub{R}}.
\end{align*}