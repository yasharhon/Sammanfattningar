\section{Lite vektoranalys och annan matte}

\paragraph{Diracs delta i högre dimensioner}
Diracs deltafunktion generaliserar utan vidare till högre dimensioner. Med andra ord är $\delta(\vb{r})$ en funktion som är noll överalt förutom origo och som uppfyller
\begin{align*}
	\integ{V}{}{V}{\delta(\vb{r})} = 1
\end{align*}
om $V$ innesluter origo.

\paragraph{Nablaoperatorn i olika koordinatsystem}
Betrakta två olika koordinatsystem $S$ och $S'$. Med hjälp av de kartesiska basvektorerna (som är lika i bägge koordinatsystemen) kan vi skriva ortsvektorn i de två som
\begin{align*}
	\vb{r} = r_{i}\ub{i},\ \vb{r}' = r_{i}'\ub{i}.
\end{align*}
Vidare kan vi skriva nablaoperatorn som
\begin{align*}
	\grad{} = \ub{i}\del{i}{},\ \grad{} = \ub{i}\delp{i}{}.
\end{align*}
Betrakta nu en funktion av $\vb{R} = \vb{r} - \vb{r}'$. Då kan vi visa att
\begin{align*}
	\del{i}{f} = -\delp{i}{f}.
\end{align*}

\paragraph{Gradienten av $R$}
Betrakta funktionen
\begin{align*}
	f(\vb{R}) = \sqrt{R_{j}R_{j}} = R.
\end{align*}
Vi har
\begin{align*}
	\del{i}{R} = \frac{1}{2}(R_{k}R_{k})^{-\frac{1}{2}}\cdot 2R_{j}\del{i}{R_{j}} = (R_{k}R_{k})^{-\frac{1}{2}}R_{j}\del{i}{(r_{j} - r_{j}'} = \frac{R_{j}}{R}\delta{ij} = \frac{R_{i}}{R}\delta{ij}.
\end{align*}
Detta ger
\begin{align*}
	\grad{R} = -\grad'{R} = \ub{\vb{R}}.
\end{align*}

\paragraph{Divergensen av $\frac{1}{R^{2}}$-fältet}
Med resultatet ovan har vi
\begin{align*}
	\div{\frac{1}{R^{2}}\ub{\vb{R}}} &= \div{\frac{1}{R^{3}}\vb{R}} \\
	                                 &= \vb{R}\cdot\grad{\frac{1}{R^{3}}} + \frac{1}{R^{3}}\div{\vb{R}} \\
	                                 &= -\frac{3}{R^{4}}\vb{R}\cdot\grad{R} + \frac{1}{R^{3}}\div{\vb{R}} \\
	                                 &= -\frac{3}{R^{4}}\vb{R}\cdot\ub{\vb{R}} + \frac{3}{R^{3}} \\
	                                 &= -\frac{3}{R^{3}} + \frac{3}{R^{3}} \\
	                                 &= 0
\end{align*}
så länge $\vb{R}\neq\vb{0}$.

Mer allmänt kan man visa att
\begin{align*}
	\div{\frac{1}{R^{2}}\ub{\vb{R}}} = 4\pi\delta(\vb{R}).
\end{align*}
Jag kan inte bevisa det, men jag kan rationalisera det kort. Utanför origo är det klart att detta stämmer. För att förstå vad som händer i origo, kan vi tillämpa den koordinatoberoende definitionen av divergens. Med den definitionen är divergensen av ett vektorfält kvoten av fältets flöde genom en litan yta kring en punkt och volymen den lilla ytan inneslutar. Med flervariabelanalys kan man visa att för fältet vi betraktar är flödet exakt $4\pi$. Om vi jämför detta med Diracs delta, ser vi att det verkar stämma.

\paragraph{$\frac{1}{R}$ och Greenfunktioner}
Med resultaten vi har får vi även
\begin{align*}
	\grad{\frac{1}{R}} = -\frac{1}{R^{2}}\grad{R} = -\frac{1}{R^{2}}\ub{\vb{R}}.
\end{align*}
Detta betyder att
\begin{align*}
	\laplacian{\frac{1}{R}} = \div{\grad{\frac{1}{R}}} = -\div{\frac{1}{R^{2}}\ub{\vb{R}}} = -4\pi\delta(\vb{R}).
\end{align*}
Detta betyder att $\frac{1}{R}$ är en Greenfunktion till Laplaceoperatorn (i tre dimensioner).