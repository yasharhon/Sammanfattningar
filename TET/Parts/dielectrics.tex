\section{Dipoler och dielektrika}

\paragraph{Elektriska dipolen}
Betrakta två punktladdningar på en linje. Linjen går genom origo, och laddningarna ligger lika långa avstånd $\frac{1}{2}d$ från origo. Potentialen i punkten $\vb{r}$, som ligger ett avstånd $R_{+}$ respektiva $R_{-}$ från de två laddningarna, ges av
\begin{align*}
	V = \frac{q}{4\pi\varepsilon_{0}R_{+}} - \frac{q}{4\pi\varepsilon_{0}R_{-}} = \frac{q}{4\pi\varepsilon_{0}}\frac{R_{-} - R_{+}}{R_{+}R_{-}}.
\end{align*}
Om $r >> d$ fås
\begin{align*}
	V \approx \frac{q}{4\pi\varepsilon_{0}}\frac{d\cos{\theta}}{r^{2}}
\end{align*}
där $\theta$ är vinkeln mellan $\vb{r}$ och linjen. Vi kan då skriva detta som
\begin{align*}
	V = \frac{q}{4\pi\varepsilon_{0}}\frac{\vb{d}\cdot\ub{r}}{r^{2}}.
\end{align*}
Vi definierar nu dipolmomentet $\vb{p} = q\vb{d}$, och får då
\begin{align*}
	V = \frac{\vb{p}\cdot\ub{r}}{4\pi\varepsilon_{0}r^{2}}.
\end{align*}

Notera att vi kan skriva dipolmomentet som $\vb{p} = \sum q_{i}\vb{r}_{i}$.

\paragraph{Ideala dipoler}
En ideal dipol fås i gränsen för en dipol när $d$ blir oändligt liten på ett sådant sätt att $\vb{p}$ hålls konstant.

\paragraph{Fältet från en dipol}
Fältet från en dipol ges av
\begin{align*}
	\vb{E} = -\grad{\frac{\vb{p}\cdot\ub{r}}{4\pi\varepsilon_{0}r^{2}}} = -\grad{\frac{\vb{p}\cdot\vb{r}}{4\pi\varepsilon_{0}r^{3}}}.
\end{align*}
I nämnaren har vi
\begin{align*}
	\vb{p}\cdot\vb{r} = p_{i}r_{i} \implies \del{i}{\vb{p}\cdot\vb{r}} = p_{i} \implies \grad{\vb{p}\cdot\vb{r}} = \vb{p}.
\end{align*}
Vi får då
\begin{align*}
	\vb{E} = -\frac{1}{4\pi\varepsilon_{0}}\left(\frac{\vb{p}}{r^{3}} - \frac{3\vb{p}\cdot\vb{r}}{r^{4}}\grad{r}\right) = \frac{1}{4\pi\varepsilon_{0}r^{3}}\left(3(\vb{p}\cdot\ub{r})\ub{r} - \vb{p}\right).
\end{align*}

\paragraph{Dipolmoment för en laddningsfördelning}
För en laddningsfördelning ges dipolmomentet av
\begin{align*}
	\vb{p} = \integ{}{}{V}{\vb{r}\rho}.
\end{align*}

\paragraph{Förflyttning av koordinatsystem och dipolmoment}
Vid förflyttning av origo en sträcka $\vb{a}$ fås
\begin{align*}
	\vb{p}' = \integ{}{}{V}{\vb{r}'\rho} = \integ{}{}{V}{(\vb{r} - \vb{a})\rho} = \vb{p} - q\vb{a}.
\end{align*}
Alltså beror dipolmomentet av origos position om det finns en netto mängd laddning i systemet.

\paragraph{Multipolutveckling}
Betrakta en punkt $\vb{r}$ och en annan punkt $\vb{r}'$. Vid att definiera $\vb{R} = \vb{r} - \vb{r}'$ får vi att om de två punkterna inte är samma, är $\laplacian{\frac{1}{R}} = 0$ överallt förutom där $\vb{R} = \vb{0}$. Vid att lägga vårat koordinatsystem så att $\vb{r}'$ är parallell med $z$-axeln och definiera vinkeln mellan $\vb{r}$ och $\vb{r}'$ som $\gamma$ fås
\begin{align*}
	\frac{1}{R} =
	\begin{cases}
		\sum\limits_{l = 0}^{\infty}A_{l}r^{l}P_{l}(\cos{\gamma}),          &r < r', \\
		\sum\limits_{l = 0}^{\infty}\frac{B_{l}}{r^{l}}P_{l}(\cos{\gamma}), &r > r'.
	\end{cases}
\end{align*}
Speciellt, på $z$-axeln är $\gamma = 0$ och
\begin{align*}
	\frac{1}{R} = \frac{1}{r_{>} - r_{<}} = \frac{1}{r_{>}}\left(1 - \frac{r_{<}}{r_{>}}\right)^{-1} = \frac{1}{r_{>}}\sum\limits_{l = 0}^{\infty}\left(\frac{r_{<}}{r_{>}}\right)^{l} = \sum\limits_{l = 0}^{\infty}\frac{r_{<}^{l}}{r_{>}^{l + 1}},
\end{align*}
där $r_{<} = \min(r', r)$ och $r_{>} = \max(r', r)$. Vi utvidgar därmed lösningen till
\begin{align*}
	\frac{1}{R} = \sum\limits_{l = 0}^{\infty}\frac{r_{<}^{l}}{r_{>}^{l + 1}}P_{l}(\cos{\gamma}).
\end{align*}

Vi söker nu potentialen utanför en sfär som omsluter en rumladdning. Vid att låta $\vb{r}$ peka utanför sfären och $\vb{r}'$ inuti fås $r_{<} = r',\ r_{>} = r$ och
\begin{align*}
	V = \frac{1}{4\pi\varepsilon_{0}}\integ{}{}{V'}{\frac{\rho}{R}} = \frac{1}{4\pi\varepsilon_{0}}\sum\limits_{l = 0}^{\infty}\frac{1}{r^{l + 1}}\integ{}{}{V'}{(r')^{l}\rho P_{l}(\cos{\gamma})} = \sum\limits_{l = 0}^{\infty}V_{l}.
\end{align*}
De olika $V_{l}$ kommer ge oss termer som ser ut som olika multipoler, och vi vill nu studera dem. Vi noterar först att om $e_{i}$ respektiva $e_{i}'$ är komponenterna av $\ub{r}$ respektiva $\ub{r}'$, kan vi skriva
\begin{align*}
	\cos(\gamma) &= e_{i}e_{i}',\ \cos[2](\gamma) = e_{i}e_{i}'e_{i}e_{j}',\ 1 = e_{i}e_{i} = e_{i}e_{j}\delta_{ij}.
\end{align*}

För $l = 0$ får vi
\begin{align*}
	V_{0} = \frac{1}{4\pi\varepsilon_{0}}\frac{1}{r}\integ{}{}{V'}{\rho},
\end{align*}
alltså ett bidrag motsvarande en punktladdning med samma totala laddning i origo. För $l = 1$ fås
\begin{align*}
	V_{1} &= \frac{1}{4\pi\varepsilon_{0}}\frac{1}{r^{2}}\integ{}{}{V'}{r'\rho P_{1}(\cos{\gamma})} \\
	      &= \frac{1}{4\pi\varepsilon_{0}r^{2}}\integ{}{}{V'}{r'\cos{\gamma}\rho} \\
	      &= \frac{1}{4\pi\varepsilon_{0}r^{2}}\integ{}{}{V'}{r'e_{i}e_{i}'\rho} \\
	      &= \frac{e_{i}}{4\pi\varepsilon_{0}r^{2}}\integ{}{}{V'}{r_{i}'\rho} \\
	      &= \frac{1}{4\pi\varepsilon_{0}r^{2}}\ub{r}\cdot\integ{}{}{V'}{r_{i}'\rho}
\end{align*}
alltså ett bidrag motsvarande en dipol med samma totala dipolmoment i origo. För $l = 2$ fås
\begin{align*}
	V_{2} &= \frac{1}{4\pi\varepsilon_{0}}\frac{1}{r^{3}}\integ{}{}{V'}{r'^{2}\rho P_{2}(\cos{\gamma})} \\
	      &= \frac{1}{8\pi\varepsilon_{0}r^{3}}\integ{}{}{V'}{r_{i}'r_{i}'\rho(3\cos[2](\gamma) - 1)} \\
	V_{2} &= \frac{1}{8\pi\varepsilon_{0}r^{3}}\integ{}{}{V'}{r_{k}'r_{k}'\rho(3\cos[2](\gamma) - 1)}  \\
	      &= \frac{1}{8\pi\varepsilon_{0}r^{3}}\integ{}{}{V'}{r_{k}'r_{k}'\rho(3e_{i}e_{i}'e_{i}e_{j}' - e_{i}e_{j}\delta_{ij})} \\
	      &= \frac{1}{8\pi\varepsilon_{0}r^{3}}e_{i}e_{j}\integ{}{}{V'}{r_{k}'r_{k}'\rho(3e_{i}'e_{j}' - \delta_{ij})} \\
	      &= \frac{1}{8\pi\varepsilon_{0}r^{3}}e_{i}e_{j}\integ{}{}{V'}{\rho(3r_{i}'r_{j}' - (r')^{2}\delta_{ij})}.
\end{align*}
Vi definierar nu kvadrupolmomentstensorn
\begin{align*}
	Q_{ij} = \frac{1}{2}\integ{}{}{V'}{\rho(3r_{i}'r_{j}' - (r')^{2}\delta_{ij})},
\end{align*}
vilket ger
\begin{align*}
	V = \frac{e_{i}Q_{ij}e_{j}}{4\pi\varepsilon_{0}r^{3}}.
\end{align*}
Detta är kvadrupolbidraget.

Allmänt blir det $l$:te bidraget på formen
\begin{align*}
	V_{l} = \frac{Q_{i_{1}\dots i_{l}}e_{i_{1}}\dots e_{i_{l}}}{4\pi\varepsilon_{0}r^{l + 1}}.
\end{align*}

\paragraph{Additionssatsen}
Hellre än att hantera komponenterna av kvadrupolmomentstensorn, använder vi en sats som säjer
\begin{align*}
	P_{l}(\cos{\gamma}) = \frac{4\pi}{2l + 1}\sum\limits_{m = -l}^{l}Y_{lm}(\theta, \phi)Y_{lm}^{*}(\theta', \phi').
\end{align*}
Då kan vi skriva
\begin{align*}
	V = \sum\limits_{l = 0}^{\infty}V_{l},\ V_{l} = \frac{1}{(2l + 1)\varepsilon_{0}r^{l + 1}}\sum\limits_{m = -l}^{l}q_{lm}Y_{lm}(\theta, \phi),
\end{align*}
där $q_{lm}$ är det sfäriska multipolmomentet
\begin{align*}
	q_{lm} = \integ{}{}{V*}{\rho (r')^{l}})Y_{lm}^{*}(\theta', \phi').
\end{align*}
Varje $V_{l}$ har alltså $2l + 1$ oberoende komponenter, vilket på grund av multipolmomenttensorernas symmetri och spårlöshet är lika med antalet oberoende komponenter i multipolmomenttensorn.

\paragraph{Kraft på en laddningsfördelning}
Betrakta en laddningsfördelning i ett yttre måttligt varierande elektriskt fält. Kraften på laddningsfördelningen ges då av
\begin{align*}
	\vb{F} = \integ{}{}{V}{\rho\vb{E}}.
\end{align*}
Vi vill nu approximera elektriska fältet i laddningsfördelningen vid att utveckla den kring en referenspunkt $\vb{r}_{0}$. Komponentvis har vi
\begin{align*}
	E_{i} \approx E_{i, 0} + (r_{j} - r_{j, 0})\del{j}{E_{i}},
\end{align*}
varför
\begin{align*}
	\vb{E} \approx \vb{E}_{0} + ((\vb{r} - \vb{r}_{0})\cdot\grad)\vb{E}
\end{align*}
och
\begin{align*}
	\vb{F} \approx \integ{}{}{V}{\rho\left(\vb{E}_{0} + ((\vb{r} - \vb{r}_{0})\cdot\grad)\vb{E}\right)}.
\end{align*}
Vidare, under antagandet att fältet varierar måttligt kan alla derivator antas vara konstanta, vilket ger
\begin{align*}
	\vb{F} \approx \integ{}{}{V}{\rho\vb{E}_{0}} + \integ{}{}{V}{\rho((\vb{r} - \vb{r}_{0})\cdot\grad)\vb{E}} = Q\vb{E}_{0} + (\vb{p}\cdot\grad)\vb{E},
\end{align*}
där dipolmomentet mäts relativ $\vb{r}_{0}$. Att dra ut elektriska fältet i andra termen från integralen är helt okej - man kan tänka sig att integrationen skapar en operator som sedan får verka på fältet.

\paragraph{Vridmoment på en laddningsfördelning}
På en punktladdning har vi
\begin{align*}
	\vb{M} = \vb{r}\times\vb{F} = q\vb{r}\times\vb{E}.
\end{align*}
För en laddningsfördelning fås då, relativt en referenspunkt $O$, vridmomentet
\begin{align*}
	\vb{M} &= \integ{}{}{V}{\rho\vb{r}\times\vb{E}} \\
	       &\approx \integ{}{}{V}{\rho((\vb{r} - \vb{r}_{0}) + \vb{r}_{0})\times(\vb{E}_{0} + ((\vb{r} - \vb{r}_{0})\cdot\grad)\vb{E})}.
\end{align*}
Vi försummar nu termer av andra ordningen i $\vb{r} - \vb{r}_{0}$ och får
\begin{align*}
	\vb{M} &= \integ{}{}{V}{\rho((\vb{r} - \vb{r}_{0})\times\vb{E}_{0} + \vb{r}_{0}\times(\vb{E}_{0} + ((\vb{r} - \vb{r}_{0})\cdot\grad)\vb{E}))} \\
	       &= \left(\integ{}{}{V}{\rho((\vb{r} - \vb{r}_{0})}\right)\times\vb{E}_{0} + \vb{r}_{0}\times\integ{}{}{V}{\vb{E}_{0} + ((\vb{r} - \vb{r}_{0})\cdot\grad)\vb{E})} \\
	       &= \vb{p}\times\vb{E}_{0} + \vb{r}_{0}\times\vb{F}.
\end{align*}

\paragraph{Polarisation}
Polarisationen $\vb{P}$ uppfyller
\begin{align*}
	\vb{p} = \integ{}{}{V}{\vb{P}}.
\end{align*}
Detta fältet uppfyller inga speciella randvillkor, och är allmänt ej rotationsfritt.

\paragraph{Polariserbarhet}
Polariserbarheten $\alpha$ uppfyller
\begin{align*}
	\vb{p} = \alpha\vb{E}.
\end{align*}

\paragraph{Fält och potential från polarisation}
Från ett litet rymdelement fås
\begin{align*}
	\dd{\vb{p}} = \vb{P}\dd{V},\ \dd{V} = \frac{1}{4\pi\varepsilon_{0}}\frac{\dd{\vb{p}}\cdot\ub{R}}{R^{2}},\ V = \frac{1}{4\pi\varepsilon_{0}}\integ{}{}{V}{\frac{\vb{P}\cdot\ub{R}}{R^{2}}}.
\end{align*}
Elektriska fältet kommer ha en singularitet som beter sig som $\frac{1}{R^{3}}$. Denna kommer inte upphävas av volymelementet, och kräver specialbehandling.

Vi kommer dock lösa detta genom att använda att
\begin{align*}
	\grad'{\frac{1}{R}} = \grad'{\frac{1}{\abs{\vb{r} - \vb{r}'}}} = \frac{1}{R^{2}}\ub{R},
\end{align*}
vilket ger
\begin{align*}
	V &= \frac{1}{4\pi\varepsilon_{0}}\integ{}{}{V}{\vb{P}\cdot\grad'{\frac{1}{R}}} \\
	  &= \frac{1}{4\pi\varepsilon_{0}}\integ{}{}{V}{\div'{\frac{\vb{P}}{R^{2}}} - \frac{1}{R}\div'{\vb{P}}} \\
	  &= \frac{1}{4\pi\varepsilon_{0}}\vinteg{}{}{S}{\frac{\vb{P}}{R}} - \frac{1}{4\pi\varepsilon_{0}}\integ{}{}{V}{\frac{1}{R}\div'{\vb{P}}},
\end{align*}
vilket motsvara coulombpotentialen från två ekvivalenta laddningsfördelningar
\begin{align*}
	\rho_{\text{b}} = -\div'{\vb{P}},\ \sigma_{\text{b}} = \vb{P}\cdot\ub{n}.
\end{align*}
Vi kalla dessa för bundna laddningsfördelningar. Nu kan vi beräkna elektriska fältet som
\begin{align*}
	\vb{E} = \frac{1}{4\pi\varepsilon_{0}}\integ{}{}{S}{\frac{\sigma_{\text{b}}}{R^{2}}\ub{R}} + \frac{1}{4\pi\varepsilon_{0}}\integ{}{}{V}{\rho_{\text{b}}\frac{1}{R^{2}}\ub{R}}.
\end{align*}

\paragraph{Introduktion av D-fältet}
Gauss' lag på integralform ger oss nu
\begin{align*}
	\div{\vb{E}} = \frac{\rho}{\varepsilon_{0}} = \frac{-\div'{\vb{P}} + \rho_{\text{f}}}{\varepsilon_{0}},
\end{align*}
där $\rho_{\text{f}}$ är den fria laddningstätheten. Detta implicerar
\begin{align*}
	\div(\varepsilon_{0}\vb{E} + \vb{P}) = \rho_{\text{f}},
\end{align*}
vilket uppmanar oss att definiera
\begin{align*}
	\vb{D} = \varepsilon_{0}\vb{E} + \vb{P}.
\end{align*}

\paragraph{Randvillkor för D-fältet}
Vi kan visa att
\begin{align*}
	D_{1}^{\perp} - D_{2}^{\perp} = \sigma_{\text{b}},\ \vb{D}_{1}^{\parallel} - \vb{D}_{2}^{\parallel} = \vb{P}_{1}^{\parallel} - \vb{P}_{2}^{\parallel}.
\end{align*}

\paragraph{Linjära dielektrika}
Linjära dielektrika uppfyller
\begin{align*}
	\vb{P} = \varepsilon_{0}\chi_{\text{e}}\vb{E}
\end{align*}
för måttliga fältstyrkor. $\chi_{\text{e}}$ är dielektrikumets elektriska susceptibilitet.

\paragraph{D-fält i linjära dielektrika}
I ett linjärt dielektrikum fås
\begin{align*}
	\vb{D} = \varepsilon_{0}(1 + \chi_{\text{e}})\vb{E} = \varepsilon_{0}\varepsilon_{\text{r}}\vb{E} = \varepsilon\vb{E}.
\end{align*}
$\varepsilon_{\text{r}}$ är dielektrikumets relativa permittivitet, och $\varepsilon$ är dets permittivitet.

\paragraph{Energi för linjära dielektrika}
I linjära dielektrika tillkommer även arbete för att polarisera dielektrikumet. Om man betraktar en atom som en punktladdning $q$ och en laddning $-q$ jämnt fördelad över ett klot med radie $a$, skapar det negativt laddade molnet ett elektriskt fält
\begin{align*}
	\vb{E}_{-q} = -\frac{q}{4\pi\varepsilon_{0}a^{3}}\vb{r},\ r < a.
\end{align*}
Kraften på laddningen i mitten blir då
\begin{align*}
	\vb{F} = -\frac{q^{2}}{4\pi\varepsilon_{0}a^{3}}\vb{r} = -\frac{q^{2}}{4\pi\varepsilon_{0}a^{3}}\grad{\frac{1}{2}r^{2}}.
\end{align*}
Arbetet som krävs för att förflytta laddningen i mitten från centrum blir då
\begin{align*}
	W = \frac{1}{2}\frac{q^{2}r^{2}}{4\pi\varepsilon_{0}a^{3}} = -\frac{1}{2}\vb{p}\cdot\vb{E}_{-q}.
\end{align*}

Om nu atomen är i ett yttre elektriskt fält $\vb{E}$, kommer det elektriska fältet att förflytta laddningen i centrum. Jämvikt fås när $\vb{E} = -\vb{E}_{-q}$, och arbetet som görs för att förflytta laddningen i mitten är då
\begin{align*}
	W = \frac{1}{2}\vb{p}\cdot\vb{E}.
\end{align*}

Med detta argumentet i bakgrunden ställer vi upp energin i ett dielektrikum som
\begin{align*}
	W = \frac{1}{2}\integ{}{}{V}{\vb{P}\cdot\vb{E}}.
\end{align*}
I tillägg har vi det övriga bidraget för att skapa det elektriska fältet, och totala energin ges av
\begin{align*}
	W = \frac{1}{2}\integ{}{}{V}{(\varepsilon_{0}\vb{E} + \vb{P})\cdot\vb{E}} = \frac{1}{2}\integ{}{}{V}{\vb{D}\cdot\vb{E}}.
\end{align*}

% TODO: Griffiths argument
% TODO: Which is greater, why

\paragraph{Krafter på ett dielektrikum}
Betrakta ett dielektrikum någonstans i närheten av två perfekta ledare med laddning $\pm Q$ på de respektiva. Ledarna kan antingen vara isärkopplade eller kopplade ihop med ett batteri som upprätthåller en konstant spänningsskillnad $U$ mellan dem. Dielektrikumets position beskrivs av dets geometri och en referensvektor $\vb{r}$ (till exempel till dets geometriska centrum). Att beräkna elektriska fältet är allmänt svårt, men vi ska försöka undveka detta med ett trick.

Betrakta första fallet först. Om dielektrikumet förflyttas en sträcka $\dd{\vb{r}}$, gör elektriska fältet från ledarna ett arbete på det, som nödvändigtvis måste balanseras av ett mekaniskt arbete. Arbetet som görs på systemet är
\begin{align*}
	\dd{W} = \grad{W}\cdot\dd{\vb{r}} = -\dd{W_{\text{e}}} = -\vb{F}_{\text{e}}\cdot\dd{\vb{r}}.
\end{align*}
Samtidigt kan vi skriva systemets energi $W$ i termer av parameterna i systemets beskrivning. Därmed är kraften på dielektrikumet
\begin{align*}
	\vb{F}_{\text{e}} = -\grad{W} = -\grad{\frac{1}{2}\frac{Q^{2}}{C}} = \frac{1}{2}\frac{Q^{2}}{C^{2}}\grad{C} = \frac{1}{2}V^{2}\grad{C}.
\end{align*}

Betrakta nu det andra fallet. Om spänningen mellan ledarna hålls konstant, kommer translation av dielektrikumet behöva balansera arbetet både från translation i elektriska fältet och arbetet som krävs för att transportera laddning mellan ledarna för att hålla spänningsskillnaden mellan dem konstant. Vi får 
\begin{align*}
	\dd{W} = U\dd{Q} - \dd{W_{\text{e}}} = U\dd{Q} - \vb{F}_{\text{e}}\cdot\dd{\vb{r}}.
\end{align*}
Samtidigt ges vänstersidan av $\dd{W} = \frac{1}{2}U\dd{Q}$, så $U\dd{Q} = 2\dd{W} = 2\grad{W}\cdot\dd{\vb{r}}$, vilket ger
\begin{align*}
	\vb{F}_{\text{e}} = \grad{W} = \frac{1}{2}U^{2}\grad{C}.
\end{align*}

\paragraph{Moment på ett dielektrikum}
Vid att göra en liknande analys som den ovan fås
\begin{align*}
	\vb{N}_{\text{e}} = \pm\del{\phi}{W_{\text{e}}}\ub{z},
\end{align*}
där $+$ och $-$ är fallen där $Q$ respektiva $U$ är konstant.