\section{Lite om elektronik}

\paragraph{Ledningsförmåga}
Fria laddningar kan transporteras av krafter per laddning. Om det i ett ämne finns en krafttäthet $\vb{f}$ ger denna alltså upphov till en inducerad strömtäthet. Vi antar att denna är linjär ock skriver
\begin{align*}
	J_{i} = \sigma_{ij}f_{j},
\end{align*}
där $\sigma$ är ledningsförmågan. Den har enhet \si{\siemens\per\meter} eller \si{\per\ohm\per\meter}. En perfekt ledare har oändlig (och diagonal) sådan.

\paragraph{Resistans och Ohms lag}
Betrakta ett batteri (en ideal spänningskälla) som är kopplad til två ideala ledare. De ideala ledarna har potential $V_{+}$ respektiva $V_{-}$, och är förbundna av ett material med ledningsförmåga $\sigma$. I det ledande området är $\vb{J} = \sigma\vb{E}$. Om kontaktytan mellan det ledande området och ledarna är $S_{+}$ respektiva $S_{-}$ fås
\begin{align*}
	U = V_{+} - V_{-} = -\vinteg{S_{-}}{S_{+}}{\vb{r}}{\vb{E}},\ I = \vinteg{S_{-}}{}{\vb{a}}{\vb{J}} = -\vinteg{S_{+}}{}{\vb{a}}{\vb{J}}.
\end{align*}
Vägintegralen för spänningen kan tas för att vara mellan två godtyckliga punkter på ytorna. Vi definierar (av någon anledning) nu det ledande områdets resistans som
\begin{align*}
	R = \frac{U}{I}.
\end{align*}

\paragraph{Randvillkor mellan ledare}
För att bevara $\div{\vb{J}} = 0$ fås kravet
\begin{align*}
	\vb{n}_{12}\cdot(\vb{J}_{1} - \vb{J}_{2}) = 0.
\end{align*}
Randvillkoret för elektriska fältet ger även
\begin{align*}
	\vb{n}_{12}\times\left(\sigma_{1}^{-1}\vb{J}_{1} - \sigma_{2}^{-1}\vb{J}_{2}\right) = \vb{0}.
\end{align*}

\paragraph{Effektutveckling och Joules lag}
Betrakta en punktladdning $q$. På denna uträtter det elektriska och magnetiska fältet arbetet
\begin{align*}
	\dd{W} = \vb{F}\cdot\dd{\vb{r}} = q(\vb{E} + \vb{v}\times\vb{B})\cdot\dd{\vb{r}} = q\vb{E}\cdot\dd{\vb{r}}.
\end{align*}
Då tillförs $q$ effekten
\begin{align*}
	P = q\vb{v}\cdot\vb{E}.
\end{align*}

Betrakta nu en volymström $\vb{J} = \rho\vb{v}$. Ett litet element i detta tillförs effekten
\begin{align*}
	\dd{P} = \rho\dd{\tau}\vb{v}\cdot\vb{E} = \dd{\tau}\vb{J}\cdot\vb{E}.
\end{align*}
Vi kan nu införa effekttätheten
\begin{align*}
	p = \vb{J}\cdot\vb{E}.
\end{align*}
Denna integreras över ledarens volym för att ge
\begin{align*}
	P &= \integ{}{}{\tau}{p} \\
	  &= \integ{}{}{\tau}{\vb{J}\cdot\vb{E}} \\
	  &= -\integ{}{}{\tau}{\vb{J}\cdot\grad{V}} \\
	  &= -\integ{}{}{\tau}{\div(V\vb{J}) - V\div{\vb{J}}} \\
	  &= -\vinteg{}{}{a}{V\vb{J}}.
\end{align*}
Om det ledande området är omringad av en isolator har $\vb{J}$ ingen normalkomponent där, och
\begin{align*}
	P &= -\vinteg{S_{+}}{}{a}{V\vb{J}} - \vinteg{S_{-}}{}{a}{V\vb{J}} \\
	  &= (V_{+} - V_{-})I,
\end{align*}
alltså
\begin{align*}
	P = UI.
\end{align*}

\paragraph{Elektromotorisk kraft}
Betrakta en smal resistiv slinga med tvärsnitt $A$ som beskrivs av en naturlig rymdkoordinat $l$. Om slingan för strömmen $I$ fås
\begin{align*}
	\vb{J} = \frac{I}{A}\ub{l} = \sigma\vb{f}.
\end{align*}
Detta ger
\begin{align*}
	\vb{f}\cdot\dd{\vb{l}} = \frac{I}{A}\sigma^{-1}\dd{l}.
\end{align*}
Om vi jämför detta med resultatet för ett cylindriskt motstånd fås
\begin{align*}
	\vb{f}\cdot\dd{\vb{l}} = I\dd{R}.
\end{align*}
Resistansen hos ett kort segment i slingan är alltså
\begin{align*}
	\dd{R} \frac{\dd{l}}{A\sigma}.
\end{align*}
Vi får nu
\begin{align*}
	\vinteg{}{}{l}{\vb{f}} = I\integ{}{}{R}{} = IR.
\end{align*}
För att få en ström krävs det alltså att
\begin{align*}
	\emf = \vinteg{}{}{l}{\vb{f}}
\end{align*}
är nollskild. Vi definierar $\emf$ som den elektromotoriska kraften.