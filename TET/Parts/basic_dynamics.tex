\section{Grundläggande dynamik}

\paragraph{Lite om kvasistatiska fält}
Vi kommer här betrakta kvasistatiska fall, alltså fall där $\rho$ och $\vb{J}$ ändras långsamt. I såna fall extrapolerar vi Biot-Savarts lag trivialt. Coulombs lag extrapoleras även trivialt, men totala elektriska fältet får även en extra term från ändringen av magnetiska fältet, som vi kommer se.

\paragraph{Elektromagnetisk induktion}
Betrakta en strömslinga $C$ som rör sig godtyckligt och ändrar form under en liten tid $\dd{t}$. Vi är nu intresserade av ändringen i magnetiska flödet
\begin{align*}
	\dv{\Phi}{t} = \dv{t}\vinteg{S(t)}{}{a}{\vb{B}} = \lim\limits_{\dd{t}\to 0}\frac{1}{\dd{t}}\left(\vinteg{S(t + \dd{t})}{}{a}{\vb{B}(t + \dd{t})} - \vinteg{S(t)}{}{a}{\vb{B}(t)}\right).
\end{align*}
Vi serieutvecklar magnetfältet med avseende på tid (den följande variationen av koordinater tas med i det faktum att vi integrerar över olika ytor) för att få
\begin{align*}
	\dv{\Phi}{t} \approx \lim\limits_{\dd{t}\to 0}\frac{1}{\dd{t}}\left(\vinteg{S(t + \dd{t})}{}{a}{\vb{B}(t)} - \vinteg{S(t)}{}{a}{\vb{B}(t)} + \dd{t}\vinteg{S(t + \dd{t})}{}{a}{\del{t}{\vb{B}}(t)}\right).
\end{align*}
Låt nu kurvan vid $t$ och $t + \dd{t}$ förbindas av ytan $S_{\text{o}}$. Då fås
\begin{align*}
	\vinteg{S(t + \dd{t})}{}{a}{\vb{B}(t)} - \vinteg{S(t)}{}{a}{\vb{B}(t)} + \vinteg{S_{\text{o}}}{}{a}{\vb{B}(t)} = \integ{V}{}{\tau}{\div{\vb{B}}} = 0,
\end{align*}
där vi har lagt på ett minustecken för att ändra orienteringen på ena ytan. Om varje punkt på $C$ rör sig med en hastighet $\vb{v}$ fås
\begin{align*}
	\vinteg{S(t + \dd{t})}{}{a}{\vb{B}(t)} - \vinteg{S(t)}{}{a}{\vb{B}(t)} = -\integ{C(t)}{}{}{\vb{B}\cdot(\dd{\vb{l}}\times\vb{v}\dd{t})}.
\end{align*}
Vi har
\begin{align*}
	\vb{B}\cdot(\dd{\vb{l}}\times\vb{v}) = B_{i}\varepsilon_{ijk}\dd{l}_{j}v_{k} = \dd{l}_{j}\varepsilon_{jki}v_{k}B_{i} = \dd{\vb{l}}\cdot(\vb{v}\times\vb{B}).
\end{align*}
Detta ger slutligen
\begin{align*}
	\dv{\Phi}{t} = \vinteg{S(t + \dd{t})}{}{a}{\del{t}{\vb{B}}(t)} - \int\limits_{C(t)}{\dd{\vb{l}}\cdot(\vb{v}\times\vb{B})}.
\end{align*}

\paragraph{Rörlig slinga i statiskt fält}
Betrakta fallet då en slinga rör sig i ett statiskt fält. Vi får då
\begin{align*}
	\int\limits_{C(t)}{\dd{\vb{l}}\cdot(\vb{v}\times\vb{B})} = -\dv{\Phi}{t}.
\end{align*}
Vi känner igen vänstra termen som elektromotoriska spänningen, alltså integralen av kraft per laddning över slingan, och får
\begin{align*}
	\emf = -\dv{\Phi}{t}.
\end{align*}

\paragraph{Varierande magnetiskt fält och fältlagar}
Betrakta en statisk strömslinga i ett varierande magnetiskt fält. Michael Faraday upptäckte att även en sån upplever en kraft. Hans hypotes var att detta berodde på att det inducerades en elektromotorisk spänning på grund av ett elektriskt fält. Mer specifikt,
\begin{align*}
	\vinteg{}{}{l}{\vb{E}} = -\vinteg{S}{}{a}{\del{t}{\vb{B}}(t)}.
\end{align*}

Maxwell generaliserade detta genom att ta bort den fysikaliska slingan och i stället betrakta en integrationsbana i rummet. Om Faradays hypotes stämde, skulle Stokes' sats ge
\begin{align*}
	\vinteg{S}{}{a}{\left(\curl{\vb{E}} + \del{t}{\vb{B}}(t)\right)} = 0.
\end{align*}
Om integrationsbanan är godtycklig, ger det
\begin{align*}
	\curl{\vb{E}} + \del{t}{\vb{B}}(t) = \vb{0}.
\end{align*}
Detta är en dynamisk fältlag för det elektriska och magnetiska fältet.

\paragraph{Potentialer för dynamiska fält}
För långsamt varierande $\rho$ och $\vb{J}$ ser uttrycken för magnetiska fältet och elektriska fältets rotationsfria del lika ut som de gjorde i statiken, fast med tidsberoende källor. De motsvarande potentialerna kommer utvidgas på samma sätt. Om vi nu tittar på elektriska fältets icke-rotationsfria del, fås
\begin{align*}
	\div{\vb{E}} = \div(-\del{t}{\vb{B}}) = -\del{t}{\div{\vb{B}}} = 0.
\end{align*}'
I analogi med Biot-Savarts lag fås då
\begin{align*}
	\vb{E} = -\frac{1}{4\pi}\integ{}{}{V}{\del{t}{\vb{B}}\times\frac{1}{R^{2}}\ub{R}}.
\end{align*}

Vi kan även skriva
\begin{align*}
	\curl{\vb{E}} + \del{t}{\vb{B}}(t) = \curl(\vb{E} + \del{t}{\vb{A}}) = \vb{0}.
\end{align*}
Detta ger
\begin{align*}
	\div(\vb{E} + \del{t}{\vb{A}}) = 0,\ \curl(\vb{E} + \del{t}{\vb{A}}) = \vb{0},
\end{align*}
och därmed måste
\begin{align*}
	\vb{E} = -\del{t}{\vb{A}}).
\end{align*}
Med generaliseringen av $\vb{A}$ fås
\begin{align*}
	\vb{E} = -\frac{\mu_{0}}{4\pi}\integ{}{}{\tau'}{\frac{1}{R}\del{t}{\vb{J}}}.
\end{align*}

I allmänhet arbetar vi med potentialerna
\begin{align*}
	\vb{B} = \curl{\vb{A}},\ \vb{E} = -\grad{V} - \del{t}{\vb{A}}.
\end{align*}

\paragraph{EMK från induktans}
Vi får i ett system med $n$ slingor
\begin{align*}
	\emf_{i} = -\sum\limits_{j}M_{ij}\dv{I_{j}}{t} = -L_{i}\dv{I_{i}}{t} + \emf_{\text{övriga}}.
\end{align*}
Vi kan ställa upp detta med hjälp av egeninduktansen som
\begin{align*}
	\dv{I_{i}}{t} + \frac{R_{i}}{L_{i}} = \frac{\emf_{\text{övriga}}}{L_{i}}.
\end{align*}

\paragraph{Magnetisk energi}
Den magnetiska energin $W_{\text{m}}$ är arbetet som krävs för att starta en strömkälla $\vb{J}$. Magnetiska fältet gör inget arbete i statiska fall, och därmed kan vi endast betrakta den i tidsberoende fall.

Under igångsättningen finns ett inducerat elektriskt fält $\vb{E} = -\del{t}{\vb{A}}$. För att upprätthålla $\vb{J}$ mot det elektriska fältet tillförs effekten
\begin{align*}
	P = -\integ{}{}{\tau}{\vb{E}\cdot\vb{J}} = \integ{}{}{\tau}{\vb{J}\cdot\del{t}{\vb{A}}}.
\end{align*}
Med Ampères lag skriver vi detta som
\begin{align*}
	P &= \frac{1}{\mu_{0}}\integ{}{}{\tau}{(\curl{\vb{B}})\cdot\del{t}{\vb{A}}} \\
	  &= \frac{1}{\mu_{0}}\integ{}{}{\tau}{\div(\vb{B}\times\del{t}{\vb{A}}) + \vb{B}\cdot\curl(\del{t}{\vb{A}})} \\
	  &= \frac{1}{\mu_{0}}\vinteg{}{}{S}{\vb{B}\times\del{t}{\vb{A}}} + \frac{1}{\mu_{0}}\integ{}{}{\tau}{\vb{B}\cdot\del{t}{\vb{B}}}.
\end{align*}
När vi utvidgar integrationsvolymen mot oändligheten försvinner ytintegralen, och kvar står
\begin{align*}
	P = \frac{1}{2\mu_{0}}\dv{t}\integ{}{}{\tau}{B^{2}}.
\end{align*}
Därmed fås
\begin{align*}
	W_{\text{m}} = \frac{1}{2\mu_{0}}\integ{}{}{\tau}{B^{2}},
\end{align*}
som alternativt kan skrivas som
\begin{align*}
	W_{\text{m}} = \frac{1}{2}\integ{}{}{\tau}{\vb{J}\cdot\vb{A}}.
\end{align*}

\paragraph{Magnetisk energi för en slinga med tjocklek}
Vi delar strömslingan i små delar $V_{i}$ med tvärsnitt $S_{i}$. Magnetiska energin ges av
\begin{align*}
	W_{\text{m}} &= \frac{1}{2}\sum\limits_{i}\integ{V_{i}}{}{\tau}{\vb{J}_{i}\cdot\vb{A}}.
\end{align*}
Varje element för en ström $\vb{J}_{i}$, och bidrar då med fält $\vb{A}_{i}$ respektiva $\vb{B}_{i}$. Detta ger
\begin{align*}
	\frac{1}{2}\sum\limits_{i}\sum\limits_{j}\integ{V_{i}}{}{\tau}{\vb{J}_{i}\cdot\vb{A}_{j}} = \sum\limits_{i}\sum\limits_{j}W_{\text{m}, ij}.
\end{align*}
Egenenergierna $W_{\text{m}, ii}$ motsvarar uttrycken vi har härlett innan. Detta sättet är att föredra framför att räkna på flödet i slingor (vilket också skulle kunna funka om man delar upp strömtätheten i skivor). Om vi vidare antar att $\vb{A}$ är ungefär konstant över varje tvärsnitt fås
\begin{align*}
	\vb{J}_{i}\dd{\tau} = I_{i}\dd{\vb{l}}.
\end{align*}
Därmed kan vi skriva
\begin{align*}
	W_{\text{m}, ij} = \frac{1}{2}I_{i}\vinteg{C_{i}}{}{l}{\vb{A}_{j}} = \frac{1}{2}I_{i}\Phi_{ij} = \frac{1}{2}I_{i}M_{ij}I_{j}.
\end{align*}
För systemet fås då
\begin{align*}
	W_{\text{m}} = \frac{1}{2}\sum\limits_{i}\sum\limits_{j}I_{i}M_{ij}I_{j}.
\end{align*}
Vi vet att detta är strikt positivt, så induktansmatrisen måste vara positivt definit.

Vi kan även notera att egeninduktansen ges av
\begin{align*}
	L_{i} = \frac{2W_{\text{m}, ii}}{I_{i}^{2}},
\end{align*}
men jag vet inte om det har någon relevans för något.

\paragraph{Generalisering av Ampères lag}
I statiska situationer har vi sett att Ampères lag är konsekvent med att strömmarna är divergensfria. I dynamiska situationer är inte strömmarna nödvändigtvis divergensfria, så vi kommer försöka göra en ny ad hoc dynamisk Ampères lag.

Vi har
\begin{align*}
	\curl{\vb{B}} &= \frac{\mu_{0}}{4\pi}\integ{}{}{\tau'}{\curl(\vb{J}\times\frac{1}{R^{2}}\ub{R})} \\
	              &= \frac{\mu_{0}}{4\pi}\integ{}{}{\tau'}{\div(\frac{1}{R^{2}}\ub{R})\vb{J} - (\vb{J}\cdot\grad)\frac{1}{R^{2}}\ub{R}} \\
	              &= \mu_{0}\vb{J} + \frac{\mu_{0}}{4\pi}\integ{}{}{\tau'}{(\vb{J}\cdot\grad')\frac{1}{R^{2}}\ub{R}}.
\end{align*}
Vi betecknar den andra termen som $\vb{I}$. Komponentvis har vi
\begin{align*}
	I_{i} &= \frac{\mu_{0}}{4\pi}\integ{}{}{\tau'}{J_{j}\del{j}{\frac{1}{R^{3}}R_{i}}} \\
	      &= \frac{\mu_{0}}{4\pi}\integ{}{}{\tau'}{\del{j}{\left(\frac{1}{R^{3}}J_{j}R_{i}\right)} - \frac{1}{R^{3}}R_{i}\del{j}{J_{j}}} \\
	      &= \frac{\mu_{0}}{4\pi}\integ{}{}{S_{i}}{\left(\frac{1}{R^{3}}J_{j}R_{i}\right)} - \frac{\mu_{0}}{4\pi}\integ{}{}{\tau'}{\frac{1}{R^{3}}R_{i}\div{\vb{J}}}.
\end{align*}
Genom att skicka integrationsområdet mot oändligheten försvinner första termen, vilket ger
\begin{align*}
	\curl{\vb{B}} &= \mu_{0}\vb{J} - \frac{\mu_{0}}{4\pi}\integ{}{}{\tau'}{\div{\vb{J}}\frac{1}{R^{3}}\ub{R}}.
\end{align*}
Med kontinuitetsekvationen fås
\begin{align*}
	\curl{\vb{B}} &= \mu_{0}\vb{J} + \frac{\mu_{0}}{4\pi}\integ{}{}{\tau'}{\del{t}{\rho}\frac{1}{R^{3}}\ub{R}}.
\end{align*}
Vi känner igen andra termen som proportionell mot en tidsderivata av elektriska fältet, och en möjlig kandidat till en ny Ampères lag är
\begin{align*}
	\curl{\vb{B}} - \mu_{0}\varepsilon_{0}\del{t}{\vb{E}} = \mu_{0}\vb{J}.
\end{align*}

Observera att detta inte är en härledning, då vi har utgått från dessa generaliserade varianterna av Coulombs och Biot-Savarts lagar. Dessa uppfyller till exempel inte $\curl{\vb{E}} = -\del{t}{\vb{B}}$. För att uppfylla detta måste induktiva korrektionstermen läggas till, men då kommer inte den nya Ampères lag att gälla, så man får en jobbig iterativ process med korrektioner.

\paragraph{Maxwells ekvationer}
All vår kunnskap om elektrostatik- och dynamik för att skriva ned de ekvationerna som beskriver det vi kan:
\begin{align*}
	\curl{\vb{E}} + \del{t}{\vb{B}} = \vb{0}, \\
	\div{\vb{B}} = 0, \\
	\curl{\vb{B}} - \mu_{0}\varepsilon_{0}\del{t}{\vb{E}} = \mu_{0}\vb{J}, \\
	\div{\vb{E}} = \frac{\rho}{\varepsilon_{0}}.
\end{align*}
Dessa kallas för Maxwells ekvationer.

\paragraph{Poyntings sats}
Betrakta en volym $V$ som omkransas av en yta $S$. Joules lag ger att $V$ tillförs effekten
\begin{align*}
	P_{\text{mek}} = \integ{V}{}{}{\vb{J}\cdot\vb{E}}.
\end{align*}
Vi kan tolka detta som
\begin{align*}
	\vb{J}\cdot\vb{E} = \del{t}{w_{\text{mek}}},
\end{align*}
där $w_{\text{mek}}$ är energitätheten. Med hjälp av Maxwells ekvationer skriver vi
\begin{align*}
	\del{t}{w_{\text{mek}}} &= \frac{1}{\mu_{0}}\vb{E}\cdot\curl{B} - \varepsilon_{0}\vb{E}\cdot\del{t}{\vb{E}} \\
	                        &= \frac{1}{\mu_{0}}\left(\vb{B}\cdot\curl{E} - \div(\vb{E}\times\vb{B})\right) - \frac{1}{2}\varepsilon_{0}\cdot\del{t}{E^{2}} \\
	                        &= -\frac{1}{\mu_{0}}\div(\vb{E}\times\vb{B}) - \del{t}{\left(\frac{1}{2}\varepsilon_{0}\cdot\del{t}{E^{2}} + \frac{1}{2\mu_{0}}B^{2}\right)}.
\end{align*}
Vi har en fältenergitäthet från de första två termerna, som vi förstår väl, men det finns även en extra term här. Den är en flödestäthet av energi. Vi kallar den då Poyntings vektor
\begin{align*}
	\vb{S} = \frac{1}{\mu_{0}}\vb{E}\times\vb{B}.
\end{align*}
Den uppfyller konserveringslagen
\begin{align*}
	\div{\vb{S}} + \del{t}{(w_{\text{mek}} + w_{\text{em}})} = 0.
\end{align*}