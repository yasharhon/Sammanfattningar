\section{Elektromagnetiska vågor}

\paragraph{Elektromagnetiska vågekvationer}
Det gäller allmänt att
\begin{align*}
	\curl(\curl(\vb{F})) = \grad(\div{\vb{F}}) - \laplacian{\vb{F}}.
\end{align*}
Maxwells ekvationer ger
\begin{align*}
	\curl(\curl{\vb{E}}) &= -\curl(\del{t}{\vb{B}}) = -\del{t}{\curl{\vb{B}}} = -\mu_{0}\del{t}{\vb{J}} - \mu_{0}\varepsilon_{0}\del[2]{t}{\vb{E}}, \\
	\curl(\curl{\vb{B}}) &= \curl(\mu_{0}\vb{J} + \mu_{0}\varepsilon_{0}\del{t}{\vb{E}}) = \mu_{0}\curl{\vb{J}} - \mu_{0}\varepsilon_{0}\del[2]{t}{\vb{B}},
\end{align*}
vilket ger vågekvationerna
\begin{align*}
	(\laplacian - \mu_{0}\varepsilon_{0}\del[2]{t}{})\vb{E} &= \frac{1}{\varepsilon_{0}}\grad{\rho} + \mu_{0}\del{t}{\vb{J}}, \\
	(\laplacian - \mu_{0}\varepsilon_{0}\del[2]{t}{})\vb{B} &= -\mu_{0}\curl{\vb{J}}.
\end{align*}

\paragraph{Plana vågors våghastighet}
Man kan visa att en plan våg i källfritt rum propagerar med farten
\begin{align*}
	c = \frac{1}{\sqrt{\mu_{0}\varepsilon_{0}}}.
\end{align*}
Vi inför d'Alembertoperatorn $\dalemb{} = \laplacian{} - \frac{1}{c^{2}}\del[2]{t}{}$ och skriver då vågekvationerna som
\begin{align*}
	\dalemb{\vb{E}} &= \frac{1}{\varepsilon_{0}}\grad{\rho} + \mu_{0}\del{t}{\vb{J}}, \\
	\dalemb{\vb{B}} &= -\mu_{0}\curl{\vb{J}}.
\end{align*}

\paragraph{Fouriertransformering av fälten}
Fouriertransformering ger
\begin{align*}
	\vb{E}_{\omega} = \integ{-\infty}{\infty}{t}{\vb{E}e^{i\omega t}},
\end{align*}
med inverstransform
\begin{align*}
	\vb{E} = \frac{1}{2\pi}\integ{-\infty}{\infty}{\omega}{\vb{E}_{\omega}e^{-i\omega t}}.
\end{align*}
Det motsvarande gäller även för magnetfältet. Vi använder beteckningen
\begin{align*}
	\vb{E}_{\omega} = \fou{\vb{E}},\ \vb{E} = \fouinv{\vb{E}_{\omega}}.
\end{align*}
Eftersom fälten är reella, måste det gälla att
\begin{align*}
	\vb{E} = \frac{1}{\pi}\Re\left(\integ{0}{\infty}{t}{\vb{E}_{\omega}e^{-i\omega t}}\right).
\end{align*}

Vi kan visa följande egenskaper i källfria rum:
\begin{align*}
	\fou{\vb{E}} = -i\omega\vb{E}_{\omega},\ \curl{\vb{E}_{\omega}} = i\omega\vb{B}_{\omega},\ \curl{\vb{B}_{\omega}} = -i\frac{\omega}{c^{2}}\vb{E}_{\omega},\ \div{\vb{E}_{\omega}} = \div{\vb{B}_{\omega}} = 0.
\end{align*}

\paragraph{Strikt tidsharmonisering}
En strikt tidsharmonisering är en idealisering av en given fältkomponent $F$ på formen
\begin{align*}
	F = F_{0}\cos(\omega_{0}t + \alpha) = \Re{F_{0}e^{-\omega_{0}t - i\alpha}}.
\end{align*}
Genom att baka ihop allt førutom det harmoniska tidsberoende får man en komplex amplitud för vågen. Denna har belop $F_{0}$ och argument $-\alpha$.

\paragraph{Plana vågor}
En plan våg ges av
\begin{align*}
	\vb{E}_{\omega} = \Re{\vb{E}_{0}e^{i(\vb{k}\cdot\vb{r} - \omega t)}},
\end{align*}
där amplituden är en konstant. För en sån ger Maxwells ekvationer
\begin{align*}
	\vb{k}\times\vb{E}_{\omega} = \omega\vb{B},\ \vb{k}\times\vb{B} = -\omega c^{2}\vb{E}_{\omega}.
\end{align*}

\paragraph{Tidsmedelvärde av harmoniska storheter}
Låt $a$ och $b$ vara harmoniska storheter med periodtid $T$. Då kan man visa att
\begin{align*}
	\expval{ab} = \frac{1}{2}\Re{\cc{A}B}
\end{align*}
där $A$ och $B$ är storheternas amplituder.

\paragraph{Maxwells ekvationer i linjära material}
I linjära materialer är Maxwells ekvationer
\begin{align*}
	\curl{\vb{E}} = -\del{t}{\vb{B}},\ \div{\vb{B}} = 0,\ \curl{\vb{H}} = \vb{J}_{\text{f}} + \del{t}{\vb{D}},\ \div{\vb{D}} = \rho_{\text{f}}.
\end{align*}

\paragraph{Laddningstätheter i homogena ohmska material}
I ett homogent ohmsk material fås
\begin{align*}
	0 = \div(\curl{\vb{H}}) &= \div{\vb{J}_{\text{f}}} + \del{t}{\div{\vb{D}}} \\
	                        &= \frac{\sigma}{\varepsilon}\div{\vb{D}} + \del{t}{\rho_{\text{f}}} \\
	                        &= \frac{\sigma}{\varepsilon}\rho_{\text{f}} + \del{t}{\rho_{\text{f}}},
\end{align*}
och därmed avtar fria laddningstätheten exponentiellt mot $0$. Därmed kommer vi bortse från denna.

\paragraph{Komplex permittivitet och permeabilitet}
Maxwells ekvationer för en monokromatisk plan våg i ett linjärt, ohmskt material ger
\begin{align*}
	\vb{k}\times\vb{B} = -\omega\mu\left(\varepsilon + \frac{\sigma}{\omega}i\right)\vb{E}.
\end{align*}
Detta får oss att införa den komplexa permittiviteten $\varepsilon_{\text{c}} = \varepsilon + \frac{\sigma}{\omega}i$. De två termerna kommer betecknas $\varepsilon'$ och $\varepsilon''$. Den imaginära termen kommer från lednings- och polarisationsförluster i materialet. På liknande sätt kan vi införa en komplex permeabilitet $\mu_{\text{c}}$, där den imaginära termen kommer från magnetiseringsförluster. I mer allmänna material kan både reella och imaginära termen i en införd komplex permittivitet eller permeabilitet bero av vågfrekvensen.

\paragraph{Dispersion i material och komplexa vågvektorn}
Vi har
\begin{align*}
	\vb{k}\times(\vb{k}\times\vb{E}) = (\vb{k}\cdot\vb{E})\vb{k} - (\vb{k}\cdot\vb{k})\vb{E} = \omega\vb{k}\times\vb{B} = -\omega^{2}\mu\varepsilon_{\text{c}}\vb{E},
\end{align*}
vilket ger dispersionsrelationen
\begin{align*}
	\vb{k}\cdot\vb{k} = \omega^{2}\mu\varepsilon_{\text{c}}.
\end{align*}
Vi ser då att vi även måste definiera en komplex vågvektor. Den imaginära och komplexa termen är parallella för en plan våg.

\paragraph{Komplexa vågvektorn för en plan våg}
För en plan våg ger dispersionsrelationen
\begin{align*}
	\vb{k}\cdot\vb{k} = (k' + ik'')^{2} = \omega^{2}\mu\varepsilon_{\text{c}} = \omega^{2}\mu\varepsilon\left(1 + \frac{\sigma}{\omega\varepsilon}i\right) = \omega^{2}\mu\varepsilon\sqrt{1 + \left(\frac{\sigma}{\omega\varepsilon}\right)^{2}}e^{i\arctan{\frac{\sigma}{\omega\varepsilon}}},
\end{align*}
och därmed
\begin{align*}
	k' + ik'' = \omega\sqrt{\mu\varepsilon}\sqrt[4]{1 + \left(\frac{\sigma}{\omega\varepsilon}\right)^{2}}e^{i\frac{1}{2}\arctan{\frac{\sigma}{\omega\varepsilon}}}.
\end{align*}
Man kan säkert visa att
\begin{align*}
	k' = \omega\sqrt{\frac{\mu\varepsilon}{2}}\sqrt{\sqrt{1 + \left(\frac{\sigma}{\omega\varepsilon}\right)^{2}} + 1},\ k'' = \omega\sqrt{\frac{\mu\varepsilon}{2}}\sqrt{\sqrt{1 + \left(\frac{\sigma}{\omega\varepsilon}\right)^{2}} - 1}.
\end{align*}

\paragraph{Inträngningsdjup}
Vågvektorns imaginära term kommer ge att fälten avtar exponentiellt med en hastighet proportionell mot komplexa termens belopp. Vi definierar då inträngningsdjupet
\begin{align*}
	\delta = \frac{1}{k''}.
\end{align*}

\paragraph{Vågimpedans}
För en monokromatisk plan våg fås
\begin{align*}
	\vb{B} = \frac{1}{\omega}\vb{k}\times\vb{E} = \sqrt{\mu\varepsilon_{\text{c}}}\ub{k}\times\vb{E},
\end{align*}
och därmed
\begin{align*}
	\vb{H} = \sqrt{\frac{\varepsilon_{\text{c}}}{\mu}}\ub{k}\times\vb{E}.
\end{align*}
Vi definierar då vågimpedansen
\begin{align*}
	\eta = \sqrt{\frac{\mu}{\varepsilon_{\text{c}}}}.
\end{align*}

\paragraph{Normalt infall på en gränsyta}
Betrakta en gränsyta i $z = 0$ mellan två linjära material där det inte finns fria laddningar eller strömtätheter på gränsytan. Låt det infallande elektriska och magnetiska fältet vara
\begin{align*}
	\vb{E}_{\text{I}} = E_{\text{I}}e^{i(kz - \omega t)}\ub{x}.
\end{align*}
Detta ger upphov till reflekterade och transmitterade fält
\begin{align*}
	\vb{E}_{\text{R}} = E_{\text{R}}e^{i(-kz - \omega t)}\ub{x},\ \vb{E}_{\text{T}} = E_{\text{T}}e^{i(kz - \omega t)}\ub{x}.
\end{align*}
I tangentiell riktning är $\vb{E}$ och $\vb{H}$ kontinuerliga. Detta ger
\begin{align*}
	E_{\text{I}} + E_{\text{R}} = E_{\text{T}},\ \frac{1}{\eta_{1}}(E_{\text{I}} - E_{\text{R}}) = \frac{1}{\eta_{2}}E_{\text{T}}.
\end{align*}
Minustecknet på det reflekterade magnetfältet följer av att vågvektorn byter tecken. Genom att definiera
\begin{align*}
	\beta = \frac{\eta_{1}}{\eta_{2}}
\end{align*}
fås
\begin{align*}
	E_{\text{R}} = \frac{1 - \beta}{1 + \beta}E_{\text{I}},\ E_{\text{T}} = \frac{2}{1 + \beta}E_{\text{I}}.
\end{align*}

\paragraph{Brytningsindex}
Brytningsindexen för ett linjärt medium definieras som
\begin{align*}
	n = \frac{c}{v},
\end{align*}
där $v$ ges av
\begin{align*}
	v = \frac{1}{\sqrt{\varepsilon\mu}}.
\end{align*}

\paragraph{Intensitet}
Intensitet för en elektromagnetisk våg ges av infallande effekt per area, med andra ord flödet av Poyntingvektorn per area. För en harmonisk våg i ett linjärt medium ges (tidsmedelvärdet av) intensiteten av
\begin{align*}
	I = \frac{1}{2}\varepsilon vE^{2}.
\end{align*}

\paragraph{Reflektions- och transmissionskoefficienter}
Reflektionskoefficienten definieras av
\begin{align*}
	R = \frac{I_{\text{R}}}{I_{\text{I}}}.
\end{align*}
För normalt infall på gränsytan mellan linjära medier ges den av
\begin{align*}
	R = \left(\frac{1 - \beta}{1 + \beta}\right)^{2}.
\end{align*}
På samma sätt definieras transmissionskoefficienten av
\begin{align*}
	T = \frac{I_{\text{T}}}{I_{\text{I}}},
\end{align*}
och ges i fallet som beskrivs ovan av
\begin{align*}
	T = \frac{4\varepsilon_{2}v_{2}}{\varepsilon_{1}v_{1}(1 + \beta)^{2}} = \sqrt{\frac{\varepsilon_{2}\mu_{1}}{\varepsilon_{1}\mu_{2}}}\frac{4}{(1 + \beta)^{2}}.
\end{align*}

Om de relativa permeabiliteterna är nära $1$ kan vi skriva
\begin{align*}
	R = \left(\frac{n_{2} - n_{1}}{n_{2} + n_{1}}\right)^{2},\ T = \frac{4n_{1}n_{2}}{(n_{1} + n_{2})^{2}}.
\end{align*}

\paragraph{Snett infall}
Betrakta en liknande gränsyta som för normalt infall, men nu med snett infallande elektriska och magnetiska fält
\begin{align*}
	\vb{E}_{\text{I}} = \vb{E}_{0\text{I}}e^{i(\vb{k}_{\text{I}}\cdot\vb{r} - \omega t)},\ \vb{B}_{\text{I}} = (\vb{k}_{\text{I}}\times\vb{E}_{0\text{I}})e^{i(\vb{k}_{\text{I}}\cdot\vb{r} - \omega t)}.
\end{align*}
Dessa ger upphov till reflekterade fält
\begin{align*}
	\vb{E}_{\text{R}} = \vb{E}_{0\text{R}}e^{i(\vb{k}_{\text{R}}\cdot\vb{r} - \omega t)},\ \vb{B}_{\text{R}} = (\vb{k}_{\text{R}}\times\vb{E}_{0\text{R}})e^{i(\vb{k}_{\text{R}}\cdot\vb{r} - \omega t)},\ \vb{E}_{\text{T}} = \vb{E}_{0\text{T}}e^{i(\vb{k}_{\text{T}}\cdot\vb{r} - \omega t)},\ \vb{B}_{\text{T}} = (\vb{k}_{\text{T}}\times\vb{E}_{0\text{T}})e^{i(\vb{k}_{\text{T}}\cdot\vb{r} - \omega t)}.
\end{align*}
Frekvensen anges av källan, ooch då uppfyller vågvektorerna
\begin{align*}
	k_{\text{I}}v_{1} = k_{\text{R}}v_{1} = k_{\text{T}}v_{2}.
\end{align*}
Randvillkoren för fälten gäller fortfarande, och kommer vara på formen
\begin{align*}
	F_{\text{I}}e^{i(\vb{k}_{\text{I}}\cdot\vb{r} - \omega t)} + F_{\text{R}}e^{i(\vb{k}_{\text{R}}\cdot\vb{r} - \omega t)} = F_{\text{T}}e^{i(\vb{k}_{\text{T}}\cdot\vb{r} - \omega t)}.
\end{align*}
Exponenterna innehåller all information om rum och tid. Om dessa vore olika, skulle en liten ändring i position eller tid bryta likheten. Detta bekreftar att frekvensen ej ändras. Vi får vidare att
\begin{align*}
	\vb{k}_{\text{I}}\cdot\vb{r} = \vb{k}_{\text{R}}\cdot\vb{r} = \vb{k}_{\text{T}}\cdot\vb{r}
\end{align*}
på gränsytan, vilket implicerar att endast $z$-komponenten av vågvektorn kan ändras under reflektion och transmission. Vi tar det därmed som en lag att vågvektorerna bildar infallsplanet, och vi kan inskränka oss till fallet där vågvektorerna ligger i $xz$-planet.

Låt vågvektorerna peka i vinklar $\theta_{\text{I}}, \theta_{\text{R}}$ och $\theta_{\text{T}}$ från normalriktningen pekande in i medierna. Om tangentialkomponenterna av vågvektorn ej ändras, ger detta reflektionslagen
\begin{align*}
	\sin{\theta_{\text{I}}} = \sin{\theta_{\text{R}}},\ \theta_{\text{I}} = \theta_{\text{R}}
\end{align*}
och
\begin{align*}
	k_{\text{I}}\sin{\theta_{\text{I}}} = k_{\text{T}}\sin{\theta_{\text{T}}}.
\end{align*}
Detta ger Snells lag
\begin{align*}
	\frac{1}{v_{1}}\sin{\theta_{\text{I}}} = \frac{1}{v_{2}}\sin{\theta_{\text{T}}}
\end{align*}
alternativt
\begin{align*}
	\frac{\sin{\theta_{\text{T}}}}{\sin{\theta_{\text{I}}}} = \frac{n_{1}}{n_{2}}.
\end{align*}

Randvillkoren ger vidare
\begin{align*}
	\varepsilon_{1}(\vb{E}_{0\text{I}}^{\perp} + \vb{E}_{0\text{R}}^{\perp})           &= \varepsilon_{2}\vb{E}_{0\text{T}}^{\perp}, \\
	\vb{B}_{0\text{I}}^{\perp} + \vb{B}_{0\text{R}}^{\perp}                            &= \vb{B}_{0\text{T}}^{\perp}, \\
	\vb{E}_{0\text{I}}^{\parallel} + \vb{E}_{0\text{R}}^{\parallel}                    &= \vb{E}_{0\text{T}}^{\parallel}, \\
	\frac{1}{\mu_{1}}(\vb{B}_{0\text{I}}^{\parallel} + \vb{B}_{0\text{R}}^{\parallel}) &= \frac{1}{\mu_{1}}\vb{B}_{0\text{T}}^{\perp},
\end{align*}
där $\vb{B} = \frac{1}{v}\vb{k}\times\vb{E}$ i alla ekvationer.

\paragraph{Plan polarisering}
Betrakta snett infall där elektriska fältet oscillerar i infallsplanet, även kallad TM-fallet. Då ger randvillkoren
\begin{align*}
	\varepsilon_{1}(-E_{0\text{I}}\sin{\theta_{\text{I}}} + E_{0\text{R}}\sin{\theta_{\text{R}}}) = -\varepsilon_{2}E_{0\text{T}}\sin{\theta_{\text{T}}}, \\
	E_{0\text{I}}\cos{\theta_{\text{I}}} + E_{0\text{R}}\cos{\theta_{\text{R}}} = E_{0\text{T}}\cos{\theta_{\text{T}}}, \\
	\frac{1}{\eta_{1}}(E_{0\text{I}} - E_{0\text{R}}) = \frac{1}{\eta_{2}}E_{0\text{T}}.
\end{align*}
Första och sista randvillkoret är linjärt beroende på grund av reflektionslagen och Snells lag, och reducerar båda till
\begin{align*}
	E_{0\text{I}} - E_{0\text{R}} = \beta E_{0\text{T}}.
\end{align*}
Om vi definierar
\begin{align*}
	\alpha = \frac{\cos{\theta_{\text{T}}}}{\cos{\theta_{\text{I}}}}
\end{align*}
blir lösningen
\begin{align*}
	E_{0\text{R}} = \frac{\beta - \alpha}{\alpha + \beta}E_{0\text{I}},\ E_{0\text{T}} = \frac{2}{\alpha + \beta}E_{0\text{I}}.
\end{align*}
Detta är Fresnels ekvationer för plan polarisering. Det finns motsvarande ekvationer för polarisering normalt på infalls planet.

\paragraph{Brewstervinkeln}
För $\alpha = \beta$ försvinner den reflekterade vågen. Vi skriver först
\begin{align*}
	\alpha = \frac{\sqrt{1 - \left(\frac{n_{1}}{n_{2}}\right)^{2}\sin[2](\theta_{\text{I}})}}{\cos{\theta_{\text{I}}}}.
\end{align*}
Låt Brewstervinkeln $\theta_{\text{B}}$ vara infallsvinkeln så att den reflekterade vågen försvinner. Denna uppfyller
\begin{align*}
	\sin[2](\theta_{\text{B}}) = \frac{1 - \beta^{2}}{\left(\frac{n_{1}}{n_{2}}\right)^{2} - \beta^{2}}.
\end{align*}
Om permeabiliteterna är lika blir detta
\begin{align*}
	\tan{\theta_{\text{B}}} = \frac{n_{2}}{n_{1}}.
\end{align*}

\paragraph{Normalt infall på ledande yta}
Vid normalt infall av en elektromagnetisk våg på en ledande yta fås en reflekterad våg, helt analogt med vad vi har sett tidigare. Det transmitterade fältet blir dock
\begin{align*}
	\vb{E}_{\text{T}} = E_{\text{T}}e^{i(k_{2}z - \omega t)}\ub{x},\ \vb{B}_{\text{T}} = \frac{k_{2}}{\omega}E_{\text{T}}e^{i(k_{2}z - \omega t)}\ub{y}
\end{align*}
där $k_{2} = k' + i\kappa$ är den nya vågvektorn. Notera att detta implicerar att fälten avtar exponentiellt. Randvillkoren blir även något annorlunda, eftersom det kan finnas källor på gränsytan. Det finns däremot ingen laddningstäthet vid normalt infall eftersom det inte finns någon normalkomponent av elektriska fältet. För en ohmsk ledare kan det heller inte finnas någon strömtäthet på gränsytan eftersom det skulle kräva ett oändligt elektriskt fält.

Randvillkoret för elektriska fältet ger
\begin{align*}
	E_{\text{I}} + E_{\text{R}} = E_{\text{T}},
\end{align*}
och randvillkoret för $H$-fältet ger
\begin{align*}
	\frac{1}{\mu_{1}v_{1}}(E_{\text{I}} - E_{\text{R}}) = \frac{k_{2}}{\mu_{2}\omega}E_{\text{T}}.
\end{align*}
Genom att definiera
\begin{align*}
	\beta' = \frac{k_{2}\mu_{1}v_{1}}{\mu_{2}\omega}
\end{align*}
fås
\begin{align*}
	E_{\text{R}} = \frac{1 - \beta'}{1 + \beta'}E_{\text{I}},\ E_{\text{T}} = \frac{2}{1 + \beta'}E_{\text{I}}.
\end{align*}
För en perfekt ledare har $k_{2}$ stort belopp, vilket ger att vågen reflekteras fullständigt med fasvridning $\pi$.

\paragraph{Metalliska vågledare}
Betrakta en källfri rymd som omges av en metallisk cylinder. Cylinderns symmetriaxel är $z$-axeln. Vi löser detta med separationsansatsen
\begin{align*}
	\vb{E} = \vb{E}_{0}(\vb{s})e^{i(k_{z}z - \omega t)},\ \vb{B} = \vb{B}_{0}(\vb{s})e^{i(k_{z}z - \omega t)}.
\end{align*}
I rymden blir vågekvationen
\begin{align*}
	\laplacian{\vb{E}} - \frac{1}{c^{2}}\del{t}{\vb{E}} = \dalemb{\vb{E}} = \vb{0}.
\end{align*}
Med den givna ansatsen fås
\begin{align*}
	\dalemb = \laplaperp - k_{z}^{2} + \frac{\omega^{2}}{c^{2}}
\end{align*}
där $\laplaperp$ gör derivationer i planet normalt på $z$-axeln. Genom att definiera det normala vågtalet $k_{\perp}^{2} = \frac{\omega^{2}}{c^{2}} - k_{z}^{2}$ fås att fältamplituderna är egenfunktioner till $\laplaperp$ med egenvärden $k_{\perp}^{2}$.

Vi delar nu upp fälten i komponenter tangentiellt och normalt på $z$-axeln. Maxwells ekvationer i tangentialplanet och $z$-riktning ger då
\begin{alignat*}{3}
	(\gradperp{E_{z}} - ik_{z}\vb{E}_{\perp})\times\ub{z} &= i\omega\vb{B}_{\perp},\      &\gradperp\times\vb{E}_{\perp}            &= i\omega B_{z}\ub{z}, \\
	(\gradperp{B_{z}} - ik_{z}\vb{B}_{\perp})\times\ub{z} &= -i\frac{\omega}{c^{2}}\vb{E}_{\perp},\ &\gradperp\times\vb{B}_{\perp} &= -i\frac{\omega}{c^{2}}E_{z}\ub{z}.
\end{alignat*}
Om man kryssar ekvationerna till vänster med $\ub{z}$ från vänster får man på något sätt
\begin{align*}
	\vb{E}_{\perp} = \frac{i}{k_{\perp}^{2}}(k_{z}\gradperp{E_{z}} - \omega\ub{z}\times\gradperp{B_{z}}),\ \vb{B}_{\perp} = \frac{i}{k_{\perp}^{2}}(k_{z}\gradperp{B_{z}} + \frac{\omega}{c^{2}}\ub{z}\times\gradperp{E_{z}}).
\end{align*}

Ekvationen behöver även randvillkor. Vi har antagit att det är metall på randen. Randvillkoret för elektriska fältet ger
\begin{align*}
	\vb{0} = \vb{n}\times\vb{E}_{0} = \vb{n}\times\vb{E}_{\perp} + E_{z}\vb{n}\times\ub{z}.
\end{align*}
Detta implicerar direkt att $E_{z} = 0$ och $\vb{n}\times\vb{E}_{\perp} = \vb{0}$. Detta gäller över hela randen, så $\vb{n}\times\gradperp{E_{z}} = \vb{0}$. Detta implicerar på något sätt $\vb{n}\cdot\gradperp{B_{z}} = 0$.

\paragraph{Gränsfrekvens för vågledare}
Fälten i en vågledare ges av ett Sturm-Liouville-problem, så spektrumet av $k_{\perp}$ är oändligt (detaljerna bestäms av randens geometri). Det finns då för varje vågtal $k_{\perp}$ i spektrumet ett $\abs{\vb{k}} = k$ så att $k < k_{\perp}$. För detta $\abs{\vb{k}}$ fås $k_{z} = i\sqrt{k_{\perp}^{2} - k^{2}}$, och fältet avtar exponentiellt i ledaren. Gränsfrekvensen för ett givet vågtal definieras som $\omega_{\text{g}} = ck_{\perp}$, och är frekvensen som ligger på gränsen mellan propagerande och evanescenta moder. Vågtalet med lägst gränsfrekvens kallas för grundmoden.

Lägre frekvenser än $\omega_{\text{g}}$ kommer alltså ge evanescenta moder och högre frekvenser ger propagerande moder. Detta visas tydligast i omskrivningen $k_{z} = \frac{1}{c}\sqrt{\omega^{2} - \omega_{\text{g}}^{2}}$.