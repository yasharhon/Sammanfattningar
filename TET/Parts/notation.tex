\section{Lite om notation}

I denna kursen kommer beteckningarna avvika något från de jag brukar använda. Differentialerna för längd, yta respektiva volym kommer betecknas $\dd{l}, \dd{a}, \dd{\tau}$ för att inte skapa förvirring när man jämför med fysikaliska storheter som kommer dyka upp.

Vi kommer vara intresserade av interaktioner mellan kroppar ute i rymden och kroppar som ger upphov till elektromagnetiska fält. Vi inför därför källpunktsvektorn $\vb{r}'$ till någon punkt på källan och fältpunktsvektorn $\vb{r}$ till någon kropp som känner av det elektromagnetiska fältet. Vi inför även vektorn $\vb{R} = \vb{r} - \vb{r}'$ från källpunkt till fältpunkt. Notera att ett $'$ även kommer användas i integration för att skilja mellan integraler över fältpunkter och källpunkter. Icke-primade differentialer kan även betyda integrationer över områden med både källpunkter och fältpunkter (typisk innehåller integranden källan, så att rena fältpunkter ej bidrar).

Apropå, om jag någon gång inte specifierar integrationsområde, betyder det typiskt att man integrerar över hela rummet.