\section{Lite om notation}

I denna kursen kommer beteckningarna avvika något från de jag brukar använda. Differentialerna för längd, yta respektiva volym kommer betecknas $\dd{l}, \dd{a}, \dd{\tau}$ för att inte skapa förvirring när man jämför med fysikaliska storheter som kommer dyka upp.

Vi kommer vara intresserade av interaktioner mellan kroppar ute i rymden och kroppar som ger upphov till elektromagnetiska fält. Vi inför därför källpunktsvektorn $\vb{r}'$ till någon punkt på köllan och fältpunktsvektorn $\vb{r}$ till någon kropp som känner av det elektromagnetiska fältet. Vektorn $\vb{R} = \vb{r} - \vb{r}'$ kommer ofta vara av intresse.