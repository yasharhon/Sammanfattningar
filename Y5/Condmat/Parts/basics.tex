\section{Basic Concepts}

\paragraph{The Classical Hall Effect}
Consider a slab of some conducting material. In the simple kinetic theory of electrons in a conductor the equation of motion is
\begin{align*}
	\expval{\dv{\vb{v}}{t}} + \frac{1}{\tau}\expval{\vb{v}} = \frac{1}{m}\vb{F}.
\end{align*}
Suppose now that the slab were to be immersed in an electric field $E\vb{e}_{x}$ and a magnetic field $B\vb{e}_{z}$. In the steady state we have
\begin{align*}
	\frac{1}{\tau}\expval{\vb{v}} = -\frac{e}{m}\left(\left(E + v_{y}B\right)\vb{e}_{x} + \left(E_{y} - v_{x}B\right)\vb{e}_{y}\right).
\end{align*}
We impose boundary conditions such that there is no net current flow in the $y$ direction. In the case of $B = 0$ we then find the conductivity $\sigma = \frac{ne^{2}\tau}{m}$. This will naturally also be the case for non-zero $B$ as $v_{y} = 0$ in the steady state. This also means that $v_{x} = -\frac{e\tau}{m}E$, and we find
\begin{align*}
	E_{y} = -\frac{e\tau}{m}EB.
\end{align*}
The steady-state current is now
\begin{align*}
	\vb{J} = -ne\expval{\vb{v}} = \frac{ne^{2}\tau}{m}\vb{E}.
\end{align*}
On tensor form we have
\begin{align*}
	E_{i} = \tensor{\rho}{_{i}^{j}}J_{j}.
\end{align*}
Evidently the resistivity tensor has diagonal components $\frac{1}{\sigma} = \frac{m}{ne^{2}\tau}$, but the addition of the magnetic field also provides an off-diagonal component
\begin{align*}
	\tensor{\rho}{_{y}^{x}} = -\frac{B}{ne}.
\end{align*}
This phenomenon is termed Hall resistance or the Hall effect. From this we define the Hall coefficient
\begin{align*}
	R_{\text{H}} = \frac{\tensor{\rho}{_{y}^{x}}}{B} = -\frac{1}{ne}.
\end{align*}

\paragraph{Failures of the Hall Effect}
The predictions based on these calculations turned out to give correct predictions for many materials, but some materials exhibited a reverse Hall effect in having a positive Hall coefficient, a property unexplainable by these purely classical arguments. The key point here is the assumption that the charge carriers have negative charge, an assumption which turns out to not be true in all materials.

\paragraph{The Bloch Theorem}
The full Hamiltonian for a set of electrons in a crystal lattice is impossible to solve. Bloch's theorem is the statement that by simplifying the theory such that the electrons are moving in some effective potential, the states are given by
\begin{align*}
	\braket{\vb{r}}{\vb{k}} = u(\vb{r})e^{i\vb{k}\cdot\vb{r}},
\end{align*}
where $u$ has the same periodicity as the potential.

\paragraph{The Tight Binding Approximation}
The tight binding approximation is an approximative solution of the Schrödinger equation in  a periodic potential. It states that, given some solution of one element of the periodic potential, the full state is a superposition of solutions localized at each element in the periodic array. More specifically, it is the approximation
\begin{align*}
	\ket{\psi_{\vb{q}}} = \frac{1}{\sqrt{N}}\sum\limits_{j}e^{i\vb{q}\cdot\vb{R}_{j}}\ket{j},
\end{align*}
with the corresponding Bloch function
\begin{align*}
	u_{\vb{q}}(\vb{r}) = \frac{1}{\sqrt{N}}\sum\limits_{j}e^{-i\vb{q}\cdot(\vb{r} - \vb{R}_{j})}\phi_{0}(\vb{r} - \vb{R}_{j}),
\end{align*}
where
\begin{align*}
	\phi_{0}(\vb{r} - \vb{R}_{j}) = \braket{\vb{r}}{j}.
\end{align*}

To compute the energy of such states we will need inner products of these states. First we have
\begin{align*}
	\braket{\psi_{\vb{q}}} = \frac{1}{N}\sum\limits_{j, l}e^{i\vb{q}\cdot(\vb{R}_{j} - \vb{R}_{l})}\braket{l}{j} = \sum\limits_{j}e^{i\vb{q}\cdot\vb{R}_{j}}\braket{0}{j},
\end{align*}
which we dub $\eta(\vb{q})$. Now, because these states are eigenstates of the translation operator one can show that
\begin{align*}
	\braket{\psi_{\vb{q}}}{\psi_{\vb{k}}} = \eta(\vb{q})\delta^{3}(\vb{q} - \vb{k})_{\vb{G}},
\end{align*}
where the Dirac delta is up to a reciprocal lattice vector. Next, let us consider matrix elements of the Hamiltonian. Writing
\begin{align*}
	\ham = \frac{p^{2}}{2m} + \sum\limits_{j}V(\vb{r} - \vb{R}_{j})
\end{align*}
we have
\begin{align*}
	\ham\ket{i} = E_{0}\ket{i} + \left(\sum\limits_{j\neq i}V(\vb{r} - \vb{R}_{j})\right)\ket{i}.
\end{align*}
Denoting the operator in the last term as $\Delta V_{i}$ we note that it does not have $\ket{i}$ as an eigenvector. For a Bloch state we then find
\begin{align*}
	\mel{\psi_{\vb{q}}}{\ham}{\psi_{\vb{q}}} = E_{0}\eta(\vb{q}) + \frac{1}{N}\sum\limits_{j, k}e^{i\vb{q}\cdot(\vb{R}_{k} - \vb{R}_{j})}\mel{j}{\Delta V_{k}}{k}.
\end{align*}
Defining the right term as $\Lambda(\vb{q})$ we find
\begin{align*}
	E_{\vb{q}} = E_{0} + \frac{\Lambda(\vb{q})}{\eta(\vb{q})}.
\end{align*}

Approximating the atoms to be far apart we can neglect most contributions to $\eta$ by setting it to $1$. In this approximation we introduce a first non-trivial correction by first setting
\begin{align*}
	\braket{j}{k} = \zeta
\end{align*}
and
\begin{align*}
	\expval{\Delta V_{k}}{k} = \Delta E,\ \mel{j}{\Delta V_{k}}{k} = t_{0},
\end{align*}
where the off-diagonal expressions only hold for nearest neighbors. $t_{0}$ is called the transfer integral. We then have
\begin{align*}
	\Lambda(\vb{q}) = \Delta E + t_{0}\sum\limits_{\vb*{\delta}}e^{-i\vb{q}\cdot\vb*{\delta}},
\end{align*}
where the sum is over nearest-neighbor lattice vectors. A similar result holds for the overlap. We often normalize the term by extracting a factor $z$ from the sum, which is the coordination number, and write what is left as $\gamma(\vb{q})$. We then finally have
\begin{align*}
	E_{\vb{q}} = E_{0} + \frac{\Delta E + t_{0}z\gamma(\vb{q})}{1 + z\zeta\gamma(\vb{q})}.
\end{align*}
Evidently, then, the band width is $2t_{0}z$. Furthermore, we can use this to determine the curvature and thus an effective mass by expanding
\begin{align*}
	E_{\vb{q}} \approx E_{0} + \frac{\hbar^{2}}{2m}\vb{q}^{2}
\end{align*}
close to the minimum.

\paragraph{Wannier Functions}
Wannier functions are constructions of states that are orthogonal. For band $n$, the Wannier function centered around atom $j$ is
\begin{align*}
	\ket{\chi_{n, j}} = \frac{1}{\sqrt{N}}\sum\limits_{\vb{q}}e^{-i\vb{q}\cdot\vb{R}_{j}}\ket{\psi_{n, \vb{q}}}.
\end{align*}
The $\ket{\psi_{n, \vb{q}}}$ are the normalized Bloch states from the tight binding approximation. These distinguish themselves in being useful even if the tight binding approximation breaks down.

\paragraph{Graphene}
Graphene is a phase of carbon where it forms a single atomic layer with the atoms arranged in a honeycomb lattice. This can be represented as a triangular lattice with two atoms in the basis. We choose the lattice vectors $a\left(\frac{3}{2}, \pm\frac{\sqrt{3}}{2}\right)$ and the basis displacement vector $a(1, 0)$. Employing the tight-binding approximation with $E_{0} = 0$ we write the Hamiltonian as
\begin{align*}
	\ham = -t\sum\limits_{\expval{i, j}}\op{i}{j} + \op{j}{i},
\end{align*}
which is a sum over nearest neighbors. The fact that there are two atoms in the basis divides the lattice into two sublattices $A$ and $B$. The Bloch states are then
\begin{align*}
	\ket{\vb{q}, A} = \frac{1}{\sqrt{N}}\sum\limits_{j\in A}e^{i\vb{q}\cdot\vb{R}_{j}}\ket{j},
\end{align*}
with a similar state for the other sublattice. Neglecting overlap between neighboring wavefunctions we find
\begin{align*}
	\ham\ket{\vb{q}, A} =& -\frac{t}{\sqrt{N}}\left(\sum\limits_{\expval{i, j}}\op{i}{j} + \op{j}{i}\right)\sum\limits_{k\in A}e^{i\vb{q}\cdot\vb{R}_{k}}\ket{k} \\
	                    =& -\frac{t}{\sqrt{N}}\sum\limits_{k\in A}\sum\limits_{j = \text{nn}(k)}e^{i\vb{q}\cdot\vb{R}_{k}}\ket{j} \\
	                    =& -\frac{t}{\sqrt{N}}\sum\limits_{k\in A}\sum\limits_{j = \text{nn}(k)}e^{i\vb{q}\cdot(\vb{R}_{j} - \vb*{\delta}_{k\to j})}\ket{j} \\
	                    =& -t\sum\limits_{\vb*{\delta}}e^{-i\vb{q}\cdot\vb*{\delta}}\ket{\vb{q}, B},
\end{align*}
and similarly
\begin{align*}
	\ham\ket{\vb{q}, B} =& -t\sum\limits_{\vb*{\delta}}e^{i\vb{q}\cdot\vb*{\delta}}\ket{\vb{q}, A}.
\end{align*}
We denote the prefactor as $-tf(\vb{q})$, and find that the Hamiltonian has eigenvalues $\pm t\abs{f(\vb{q})}$.

Of note about the Brillouin zone of the triangular lattice, which is a hexagon, is that it only has one unique corner (and its opposite), as the others are related to these by a reciprocal lattice vector. Choosing, say, $\vb{K} = \frac{4\pi}{3\sqrt{3}a}\left(\frac{\sqrt{3}}{2}, \frac{1}{2}\right)$ and its opposite we have
\begin{align*}
	f(\vb{K}) = e^{i\frac{4\pi}{3\sqrt{3}}\frac{\sqrt{3}}{2}} + e^{i\frac{4\pi}{3\sqrt{3}}\left(-\frac{\sqrt{3}}{4} - \frac{\sqrt{3}}{4}\right)} + e^{i\frac{4\pi}{3\sqrt{3}}\left(-\frac{\sqrt{3}}{4} + \frac{\sqrt{3}}{4}\right)} = 2\cos(\frac{2\pi}{3}) + 1 = 0.
\end{align*}
Thus there is no band gap. Next we have
\begin{align*}
	\grad_{\vb{q}}f(\vb{q}) = i\sum\limits_{\vb*{\delta}}\vb*{\delta}e^{i\vb{q}\cdot\vb*{\delta}},
\end{align*}
and close to the corner we have
\begin{align*}
	f(\vb{q}) \approx& (\vb{q} - \vb{K})\cdot ia\left((1, 0)e^{i\frac{2\pi}{3}} + \left(-\frac{1}{2}, -\frac{\sqrt{3}}{2}\right)e^{-i\frac{2\pi}{3}} + \left(-\frac{1}{2}, \frac{\sqrt{3}}{2}\right)\right) \\
	                =& ia\left(k_{x}e^{i\frac{2\pi}{3}} + \left(-\frac{1}{2}k_{x} - \frac{\sqrt{3}}{2}k_{y}\right)e^{-i\frac{2\pi}{3}} - \frac{1}{2}k_{x} + \frac{\sqrt{3}}{2}k_{y}\right) \\
	                =& ia\left(k_{x}\left(e^{i\frac{2\pi}{3}} - \frac{1}{2}e^{-i\frac{2\pi}{3}} - \frac{1}{2}\right) + k_{y}\left(-\frac{\sqrt{3}}{2}e^{-i\frac{2\pi}{3}} + \frac{\sqrt{3}}{2}\right)\right).
\end{align*}
This means that the dispersion is linear, a characteristic of solutions of the Dirac equation. This relation must be investigated further.

To do this we must remedy the fact that $f$ does not share the periodicity of the reciprocal lattice due to the $\vb*{\delta}$ not being lattice vectors. However, by introducing
\begin{align*}
	\ket{\vb{q}, \tilde{B}} = ie^{-i\vb{q}\cdot\vb*{\delta}_{1}}\ket{\vb{q}, B},
\end{align*}
the same treatment results in the Hamiltonian containing a new function $\tilde{f}(\vb{q}) = ie^{-i\vb{q}\cdot\vb*{\delta}_{1}}f(\vb{q})$. This function has the periodicity of the reciprocal lattice. Writing
\begin{align*}
	\tilde{f}(\vb{q}) = i\left(1 + e^{i\vb{q}\cdot\sqrt{3}a\left(-\frac{\sqrt{3}}{2}, \frac{1}{2}\right)} + e^{i\vb{q}\cdot\sqrt{3}a\left(-\frac{\sqrt{3}}{2}, -\frac{1}{2}\right)}\right),
\end{align*}
we have
\begin{align*}
	\grad_{\vb{q}}\tilde{f} =& -\sqrt{3}a\left(\left(-\frac{\sqrt{3}}{2}, \frac{1}{2}\right)e^{i\vb{q}\cdot\sqrt{3}a\left(-\frac{\sqrt{3}}{2}, \frac{1}{2}\right)} + \left(-\frac{\sqrt{3}}{2}, -\frac{1}{2}\right)e^{i\vb{q}\cdot\sqrt{3}a\left(-\frac{\sqrt{3}}{2}, -\frac{1}{2}\right)}\right).
\end{align*}
As $\tilde{f}(\vb{K}) = 0$ we have
\begin{align*}
	\tilde{f}(\vb{K} + \vb{k}) \approx& -\sqrt{3}a\left(\left(-\frac{\sqrt{3}}{2}, \frac{1}{2}\right)e^{i\frac{4\pi}{3}\left(-\frac{3}{4} + \frac{1}{4}\right)} + \left(-\frac{\sqrt{3}}{2}, -\frac{1}{2}\right)e^{i\frac{4\pi}{3}\left(-\frac{3}{4} - \frac{1}{4}\right)}\right)\cdot\vb{k} \\
	=& -\sqrt{3}a\left(\left(-\frac{\sqrt{3}}{2}, \frac{1}{2}\right)e^{-i\frac{2\pi}{3}} + \left(-\frac{\sqrt{3}}{2}, -\frac{1}{2}\right)e^{-i\frac{4\pi}{3}}\right)\cdot\vb{k} \\
	=& -\sqrt{3}a\left(-\frac{\sqrt{3}}{2}\left(e^{-i\frac{2\pi}{3}} + e^{-i\frac{4\pi}{3}}\right), \frac{1}{2}\left(e^{-i\frac{2\pi}{3}} - e^{-i\frac{4\pi}{3}}\right)\right)\cdot\vb{k} \\
	=& -\sqrt{3}a\left(-\sqrt{3}\cos(-\frac{2\pi}{3}), i\sin(-\frac{2\pi}{3})\right)\cdot\vb{k} \\
	=& -\frac{3}{2}a\left(1, -i\right)\cdot\vb{k}.
\end{align*}
In the matrix form we can represent the Hamiltonian in this basis as
\begin{align*}
	\tilde{\ham} = \frac{3}{2}at\mqty[
		0               & k_{x} - ik_{y} \\
		k_{x} + ik_{y}  & 0 \\
	] = v_{\text{F}}\vb*{\sigma}\cdot\vb{k}.
\end{align*}
This is an effective Hamiltonian for states close to the Brillouin zone boundary which is exactly the Dirac Hamiltonian in two dimensions. Doing the same around $-\vb{K}$ the Hamiltonian is $\tilde{\ham} = v_{\text{F}}(-\sigma_{x}k_{x} + \sigma_{y}k_{y})$. We can write this in a unified way by introducing a so-called valley degree-of-freedom index $\tau_{z} = \pm 1$.