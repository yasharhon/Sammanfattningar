\section{Basic Concepts}

\paragraph{The Classical Hall Effect}
Consider a slab of some conducting material. In the simple kinetic theory of electrons in a conductor the equation of motion is
\begin{align*}
	\expval{\dv{\vb{v}}{t}} + \frac{1}{\tau}\expval{\vb{v}} = \frac{1}{m}\vb{F}.
\end{align*}
Suppose now that the slab were to be immersed in an electric field $E\vb{e}_{x}$ and a magnetic field $B\vb{e}_{z}$. In the steady state we have
\begin{align*}
	\frac{1}{\tau}\expval{\vb{v}} = -\frac{e}{m}\left(\left(E + v_{y}B\right)\vb{e}_{x} + \left(E_{y} - v_{x}B\right)\vb{e}_{y}\right).
\end{align*}
We impose boundary conditions such that there is no net current flow in the $y$ direction. In the case of $B = 0$ we then find the conductivity $\sigma = \frac{ne^{2}\tau}{m}$. This will naturally also be the case for non-zero $B$ as $v_{y} = 0$ in the steady state. This also means that $v_{x} = -\frac{e\tau}{m}E$, and we find
\begin{align*}
	E_{y} = -\frac{e\tau}{m}EB.
\end{align*}
The steady-state current is now
\begin{align*}
	\vb{J} = -ne\expval{\vb{v}} = \frac{ne^{2}\tau}{m}\vb{E}.
\end{align*}
On tensor form we have
\begin{align*}
	E_{i} = \tensor{\rho}{_{i}^{j}}J_{j}.
\end{align*}
Evidently the resistivity tensor has diagonal components $\frac{1}{\sigma} = \frac{m}{ne^{2}\tau}$, but the addition of the magnetic field also provides an off-diagonal component
\begin{align*}
	\tensor{\rho}{_{y}^{x}} = -\frac{B}{ne}.
\end{align*}
This phenomenon is termed Hall resistance or the Hall effect. From this we define the Hall coefficient
\begin{align*}
	R_{\text{H}} = \frac{\tensor{\rho}{_{y}^{x}}}{B} = -\frac{1}{ne}.
\end{align*}

\paragraph{Failures of the Hall Effect}
The predictions based on these calculations turned out to give correct predictions for many materials, but some materials exhibited a reverse Hall effect in having a positive Hall coefficient, a property unexplainable by these purely classical arguments. The key point here is the assumption that the charge carriers have negative charge, an assumption which turns out to not be true in all materials.

\paragraph{The Bloch Theorem}
The full Hamiltonian for a set of electrons in a crystal lattice is impossible to solve. Bloch's theorem is the statement that by simplifying the theory such that the electrons are moving in some effective potential, the states are given by
\begin{align*}
	\psi_{n, \vb{k}} = \braket{\vb{r}}{n, \vb{k}} = u_{n, \vb{k}}(\vb{r})e^{i\vb{k}\cdot\vb{r}},
\end{align*}
where $u$ has the same periodicity as the potential. The wavevector $\vb{k}$ is confined to the first Brillouin zone and the index $n$ labels which band the state is in. One useful property of these states is that
\begin{align*}
	\mel{\vb{r}}{\ham}{n, \vb{k}} =& \left(\frac{1}{2m}\vb{p}^{2} + V(\vb{r})\right)u_{n, \vb{k}}(\vb{r})e^{i\vb{k}\cdot\vb{r}} \\
	=& e^{i\vb{k}\cdot\vb{r}}\left(\frac{1}{2m}(\vb{p} + \hbar^{2}\vb{k})^{2} + V(\vb{r})\right)u_{n, \vb{k}}(\vb{r}),
\end{align*}
hence we may change the wavefunction basis to only include the Bloch functions and let the parameter $\vb{k}$ enter as a parameter in the Hamiltonian. The Feynman-Hellman theorem then states that
\begin{align*}
	\expval{\vb{v}}{n, \vb{k}} = \frac{1}{m}\expval{\vb{p} + \hbar\vb{k}}{u_{n, \vb{k}}} = \frac{1}{\hbar}\grad_{\vb{k}}E_{\vb{k}},
\end{align*}
where $E_{\vb{k}}$ is the energy eigenvalue associated with a given Bloch state.

\paragraph{The Tight Binding Approximation}
The tight binding approximation is an approximative solution of the Schrödinger equation in  a periodic potential. It states that, given some solution of one element of the periodic potential, the full state is a superposition of solutions localized at each element in the periodic array. More specifically, it is the approximation
\begin{align*}
	\ket{\psi_{\vb{q}}} = \frac{1}{\sqrt{N}}\sum\limits_{j}e^{i\vb{q}\cdot\vb{R}_{j}}\ket{j},
\end{align*}
with the corresponding Bloch function
\begin{align*}
	u_{\vb{q}}(\vb{r}) = \frac{1}{\sqrt{N}}\sum\limits_{j}e^{-i\vb{q}\cdot(\vb{r} - \vb{R}_{j})}\phi_{0}(\vb{r} - \vb{R}_{j}),
\end{align*}
where
\begin{align*}
	\phi_{0}(\vb{r} - \vb{R}_{j}) = \braket{\vb{r}}{j}.
\end{align*}

To compute the energy of such states we will need inner products of these states. First we have
\begin{align*}
	\braket{\psi_{\vb{q}}} = \frac{1}{N}\sum\limits_{j, l}e^{i\vb{q}\cdot(\vb{R}_{j} - \vb{R}_{l})}\braket{l}{j} = \sum\limits_{j}e^{i\vb{q}\cdot\vb{R}_{j}}\braket{0}{j},
\end{align*}
which we dub $\eta(\vb{q})$. Now, because these states are eigenstates of the translation operator one can show that
\begin{align*}
	\braket{\psi_{\vb{q}}}{\psi_{\vb{k}}} = \eta(\vb{q})\delta^{3}(\vb{q} - \vb{k})_{\vb{G}},
\end{align*}
where the Dirac delta is up to a reciprocal lattice vector. Next, let us consider matrix elements of the Hamiltonian. Writing
\begin{align*}
	\ham = \frac{p^{2}}{2m} + \sum\limits_{j}V(\vb{r} - \vb{R}_{j})
\end{align*}
we have
\begin{align*}
	\ham\ket{i} = E_{0}\ket{i} + \left(\sum\limits_{j\neq i}V(\vb{r} - \vb{R}_{j})\right)\ket{i}.
\end{align*}
Denoting the operator in the last term as $\Delta V_{i}$ we note that it does not have $\ket{i}$ as an eigenvector. For a Bloch state we then find
\begin{align*}
	\mel{\psi_{\vb{q}}}{\ham}{\psi_{\vb{q}}} = E_{0}\eta(\vb{q}) + \frac{1}{N}\sum\limits_{j, k}e^{i\vb{q}\cdot(\vb{R}_{k} - \vb{R}_{j})}\mel{j}{\Delta V_{k}}{k}.
\end{align*}
Defining the right term as $\Lambda(\vb{q})$ we find
\begin{align*}
	E_{\vb{q}} = E_{0} + \frac{\Lambda(\vb{q})}{\eta(\vb{q})}.
\end{align*}

Approximating the atoms to be far apart we can neglect most contributions to $\eta$ by setting it to $1$. In this approximation we introduce a first non-trivial correction by first setting
\begin{align*}
	\braket{j}{k} = \zeta
\end{align*}
and
\begin{align*}
	\expval{\Delta V_{k}}{k} = \Delta E,\ \mel{j}{\Delta V_{k}}{k} = t_{0},
\end{align*}
where the off-diagonal expressions only hold for nearest neighbors. $t_{0}$ is called the transfer integral. We then have
\begin{align*}
	\Lambda(\vb{q}) = \Delta E + t_{0}\sum\limits_{\vb*{\delta}}e^{-i\vb{q}\cdot\vb*{\delta}},
\end{align*}
where the sum is over nearest-neighbor lattice vectors. A similar result holds for the overlap. We often normalize the term by extracting a factor $z$ from the sum, which is the coordination number, and write what is left as $\gamma(\vb{q})$. We then finally have
\begin{align*}
	E_{\vb{q}} = E_{0} + \frac{\Delta E + t_{0}z\gamma(\vb{q})}{1 + z\zeta\gamma(\vb{q})}.
\end{align*}
Evidently, then, the band width is $2t_{0}z$. Furthermore, we can use this to determine the curvature and thus an effective mass by expanding
\begin{align*}
	E_{\vb{q}} \approx E_{0} + \frac{\hbar^{2}}{2m}\vb{q}^{2}
\end{align*}
close to the minimum.

\paragraph{Wannier Functions}
Wannier functions are constructions of states that are orthogonal. For band $n$, the Wannier function centered around atom $j$ is
\begin{align*}
	\ket{\chi_{n, j}} = \frac{1}{\sqrt{N}}\sum\limits_{\vb{q}}e^{-i\vb{q}\cdot\vb{R}_{j}}\ket{\psi_{n, \vb{q}}}.
\end{align*}
The $\ket{\psi_{n, \vb{q}}}$ are the normalized Bloch states from the tight binding approximation. These distinguish themselves in being useful even if the tight binding approximation breaks down.

\paragraph{Graphene}
Graphene is a phase of carbon where it forms a single atomic layer with the atoms arranged in a honeycomb lattice. This can be represented as a triangular lattice with two atoms in the basis. We choose the lattice vectors $a\left(\frac{3}{2}, \pm\frac{\sqrt{3}}{2}\right)$ and the basis displacement vector $a(1, 0)$. Employing the tight-binding approximation with $E_{0} = 0$ we write the Hamiltonian as
\begin{align*}
	\ham = -t\sum\limits_{\expval{i, j}}\op{i}{j} + \op{j}{i},
\end{align*}
which is a sum over nearest neighbors. The fact that there are two atoms in the basis divides the lattice into two sublattices $A$ and $B$. The Bloch states are then
\begin{align*}
	\ket{\vb{q}, A} = \frac{1}{\sqrt{N}}\sum\limits_{j\in A}e^{i\vb{q}\cdot\vb{R}_{j}}\ket{j},
\end{align*}
with a similar state for the other sublattice. Neglecting overlap between neighboring wavefunctions we find
\begin{align*}
	\ham\ket{\vb{q}, A} =& -\frac{t}{\sqrt{N}}\left(\sum\limits_{\expval{i, j}}\op{i}{j} + \op{j}{i}\right)\sum\limits_{k\in A}e^{i\vb{q}\cdot\vb{R}_{k}}\ket{k} \\
	                    =& -\frac{t}{\sqrt{N}}\sum\limits_{k\in A}\sum\limits_{j = \text{nn}(k)}e^{i\vb{q}\cdot\vb{R}_{k}}\ket{j} \\
	                    =& -\frac{t}{\sqrt{N}}\sum\limits_{k\in A}\sum\limits_{j = \text{nn}(k)}e^{i\vb{q}\cdot(\vb{R}_{j} - \vb*{\delta}_{k\to j})}\ket{j} \\
	                    =& -t\sum\limits_{\vb*{\delta}}e^{-i\vb{q}\cdot\vb*{\delta}}\ket{\vb{q}, B},
\end{align*}
and similarly
\begin{align*}
	\ham\ket{\vb{q}, B} =& -t\sum\limits_{\vb*{\delta}}e^{i\vb{q}\cdot\vb*{\delta}}\ket{\vb{q}, A}.
\end{align*}
We denote the prefactor as $-tf(\vb{q})$, and find that the Hamiltonian has eigenvalues $\pm t\abs{f(\vb{q})}$.

Of note about the Brillouin zone of the triangular lattice, which is a hexagon, is that it only has one unique corner (and its opposite), as the others are related to these by a reciprocal lattice vector. Choosing, say, $\vb{K} = \frac{4\pi}{3\sqrt{3}a}\left(\frac{\sqrt{3}}{2}, \frac{1}{2}\right)$ and its opposite we have
\begin{align*}
	f(\vb{K}) = e^{i\frac{4\pi}{3\sqrt{3}}\frac{\sqrt{3}}{2}} + e^{i\frac{4\pi}{3\sqrt{3}}\left(-\frac{\sqrt{3}}{4} - \frac{\sqrt{3}}{4}\right)} + e^{i\frac{4\pi}{3\sqrt{3}}\left(-\frac{\sqrt{3}}{4} + \frac{\sqrt{3}}{4}\right)} = 2\cos(\frac{2\pi}{3}) + 1 = 0.
\end{align*}
Thus there is no band gap. Next we have
\begin{align*}
	\grad_{\vb{q}}f(\vb{q}) = i\sum\limits_{\vb*{\delta}}\vb*{\delta}e^{i\vb{q}\cdot\vb*{\delta}},
\end{align*}
and close to the corner we have
\begin{align*}
	f(\vb{q}) \approx& (\vb{q} - \vb{K})\cdot ia\left((1, 0)e^{i\frac{2\pi}{3}} + \left(-\frac{1}{2}, -\frac{\sqrt{3}}{2}\right)e^{-i\frac{2\pi}{3}} + \left(-\frac{1}{2}, \frac{\sqrt{3}}{2}\right)\right) \\
	                =& ia\left(k_{x}e^{i\frac{2\pi}{3}} + \left(-\frac{1}{2}k_{x} - \frac{\sqrt{3}}{2}k_{y}\right)e^{-i\frac{2\pi}{3}} - \frac{1}{2}k_{x} + \frac{\sqrt{3}}{2}k_{y}\right) \\
	                =& ia\left(k_{x}\left(e^{i\frac{2\pi}{3}} - \frac{1}{2}e^{-i\frac{2\pi}{3}} - \frac{1}{2}\right) + k_{y}\left(-\frac{\sqrt{3}}{2}e^{-i\frac{2\pi}{3}} + \frac{\sqrt{3}}{2}\right)\right).
\end{align*}
This means that the dispersion is linear, a characteristic of solutions of the Dirac equation. This relation must be investigated further.

To do this we must remedy the fact that $f$ does not share the periodicity of the reciprocal lattice due to the $\vb*{\delta}$ not being lattice vectors. However, by introducing
\begin{align*}
	\ket{\vb{q}, \tilde{B}} = ie^{-i\vb{q}\cdot\vb*{\delta}_{1}}\ket{\vb{q}, B},
\end{align*}
the same treatment results in the Hamiltonian containing a new function $\tilde{f}(\vb{q}) = ie^{-i\vb{q}\cdot\vb*{\delta}_{1}}f(\vb{q})$. This function has the periodicity of the reciprocal lattice. Writing
\begin{align*}
	\tilde{f}(\vb{q}) = i\left(1 + e^{i\vb{q}\cdot\sqrt{3}a\left(-\frac{\sqrt{3}}{2}, \frac{1}{2}\right)} + e^{i\vb{q}\cdot\sqrt{3}a\left(-\frac{\sqrt{3}}{2}, -\frac{1}{2}\right)}\right),
\end{align*}
we have
\begin{align*}
	\grad_{\vb{q}}\tilde{f} =& -\sqrt{3}a\left(\left(-\frac{\sqrt{3}}{2}, \frac{1}{2}\right)e^{i\vb{q}\cdot\sqrt{3}a\left(-\frac{\sqrt{3}}{2}, \frac{1}{2}\right)} + \left(-\frac{\sqrt{3}}{2}, -\frac{1}{2}\right)e^{i\vb{q}\cdot\sqrt{3}a\left(-\frac{\sqrt{3}}{2}, -\frac{1}{2}\right)}\right).
\end{align*}
As $\tilde{f}(\vb{K}) = 0$ we have
\begin{align*}
	\tilde{f}(\vb{K} + \vb{k}) \approx& -\sqrt{3}a\left(\left(-\frac{\sqrt{3}}{2}, \frac{1}{2}\right)e^{i\frac{4\pi}{3}\left(-\frac{3}{4} + \frac{1}{4}\right)} + \left(-\frac{\sqrt{3}}{2}, -\frac{1}{2}\right)e^{i\frac{4\pi}{3}\left(-\frac{3}{4} - \frac{1}{4}\right)}\right)\cdot\vb{k} \\
	=& -\sqrt{3}a\left(\left(-\frac{\sqrt{3}}{2}, \frac{1}{2}\right)e^{-i\frac{2\pi}{3}} + \left(-\frac{\sqrt{3}}{2}, -\frac{1}{2}\right)e^{-i\frac{4\pi}{3}}\right)\cdot\vb{k} \\
	=& -\sqrt{3}a\left(-\frac{\sqrt{3}}{2}\left(e^{-i\frac{2\pi}{3}} + e^{-i\frac{4\pi}{3}}\right), \frac{1}{2}\left(e^{-i\frac{2\pi}{3}} - e^{-i\frac{4\pi}{3}}\right)\right)\cdot\vb{k} \\
	=& -\sqrt{3}a\left(-\sqrt{3}\cos(-\frac{2\pi}{3}), i\sin(-\frac{2\pi}{3})\right)\cdot\vb{k} \\
	=& -\frac{3}{2}a\left(1, -i\right)\cdot\vb{k}.
\end{align*}
In the matrix form we can represent the Hamiltonian in this basis as
\begin{align*}
	\tilde{\ham} = \frac{3}{2}at\mqty[
		0               & k_{x} - ik_{y} \\
		k_{x} + ik_{y}  & 0 \\
	] = v_{\text{F}}\vb*{\sigma}\cdot\vb{k}.
\end{align*}
This is an effective Hamiltonian for states close to the Brillouin zone boundary which is exactly the Dirac Hamiltonian in two dimensions. Doing the same around $-\vb{K}$ the Hamiltonian is $\tilde{\ham} = v_{\text{F}}(-\sigma_{x}k_{x} + \sigma_{y}k_{y})$. We can write this in a unified way by introducing a so-called valley degree-of-freedom index $\tau_{z} = \pm 1$.

\paragraph{Polyacetylene}
Consider a one-dimensional chain with two atoms in the basis separated by a distance $a$ we have
\begin{align*}
	\gamma(q) = 2\cos(qa).
\end{align*}
By approximating $\zeta = 0$ and setting $E_{0} = 0$ we then find
\begin{align*}
	E_{q} = 2t\cos(qa).
\end{align*}
In the localized basis for the two atoms we have
\begin{align*}
	\ham = \mqty[
		0          & -2t\cos(qa) \\
		-2t\cos(qa) & 0
	].
\end{align*}
There exists one unique Dirac point at $k_{0} = \frac{\pi}{2a}$, and close to this point we have $E_{q} \approx 2taq$, hence the Hamiltonian in terms of a small displacement $k$ from the Dirac point becomes $\ham \approx 2ta\sigma_{x}k$. This is the Dirac Hamiltonian in one dimension.

Because only one Pauli matrix has been used, more possible non-trivial terms can be included by incorporating the Pauli matrices. The $\sigma_{z}$ term might represent some site difference. To obtain the other kind of term we suppose the hopping elements to have some directionality - that is, $t = t_{0} \pm \delta t$. The Hamiltonian then becomes
\begin{align*}
	\ham = \mqty[
		0                                & -2t\cos(qa) - 2i\delta t\sin(qa) \\
		-2t\cos(qa) + 2i\delta t\sin(qa) & 0
	],
\end{align*}
and close to the Dirac point we have
\begin{align*}
	\ham \approx 2ta\sigma_{x}k + 2\delta t\sigma_{y}.
\end{align*}
This has the consequence of reducing the energy of the valence band - more specificallye we have
\begin{align*}
	\Delta E =& \frac{L}{2\pi}\inte{-\frac{\pi}{2a}}{\frac{\pi}{2a}}\dd{k}v_{\text{F}}\abs{k} - \sqrt{v_{\text{F}}^{2}k^{2} + m_{y}^{2} + m_{z}^{2}} \\
	=& \frac{L}{\pi}\inte{0}{\frac{\pi}{2a}}\dd{k}v_{\text{F}}k - \sqrt{v_{\text{F}}^{2}k^{2} + m_{y}^{2} + m_{z}^{2}} \\
	=& \frac{L}{2v_{\text{F}}\pi}\left(\left(\frac{v_{\text{F}}\pi}{2a}\right)^{2} - \frac{v_{\text{F}}\pi}{2a}\sqrt{\left(\frac{v_{\text{F}}\pi}{2a}\right)^{2} + m_{y}^{2} + m_{z}^{2}} - (m_{y}^{2} + m_{z}^{2})\ln(\frac{\sqrt{m_{y}^{2} + m_{z}^{2} + \left(\frac{v_{\text{F}}\pi}{2a}\right)^{2}} + \frac{v_{\text{F}}\pi}{2a}}{\sqrt{m_{y}^{2} + m_{z}^{2}}})\right),
\end{align*}
where we have introduced the Fermi velocity $v_{\text{F}} = 2ta$. Defining $m^{2} = m_{y}^{2} + m_{z}^{2}$ we find in the limit that $m$ becomes small that
\begin{align*}
	\Delta E \approx -\frac{Lm^{2}}{2v_{\text{F}}\pi}\ln(\frac{\sqrt{m_{y}^{2} + m_{z}^{2} + \left(\frac{v_{\text{F}}\pi}{2a}\right)^{2}} + \frac{v_{\text{F}}\pi}{2a}}{\sqrt{m_{y}^{2} + m_{z}^{2}}}) \approx -\frac{Lm^{2}}{2v_{\text{F}}\pi}\ln(\frac{v_{\text{F}}\pi}{\abs{m}a})
\end{align*}
, meaning there is a spontaneous tendency for the system to arrange itself such that this happens. This has the consequence of deforming the lattice, however, and a proper attempt will need to account for both of these effects at the same time.

The Su-Schrieffer-Heeger model is an attempt to solve this. Its Hamiltonian is
\begin{align*}
	\ham = \sum\limits_{j}-t(u_{j + 1} - u_{j})(\op{j}{j + 1} + \op{j + 1}{j}) + \text{h.c.} + \frac{\lambda}{2}(u_{j + 1} - u_{j})^{2},
\end{align*}
where the degrees of freedom are the $u_{j}$ for the lattice points and the electronic states represented by kets. The hopping element $t$ now depends on the lattice structure. We can approximate
\begin{align*}
	t(u_{j + 1} - u_{j}) \approx t - \frac{\alpha}{2}(u_{j + 1} - u_{j})
\end{align*}
for small displacements. For the dimeric case we may choose coordinates such that $u_{j} = (-1)^{j}u$, hence $\delta t = \alpha u$. The total energy change for dimerization is then
\begin{align*}
	\Delta E\approx -\frac{Lm^{2}}{2v_{\text{F}}\pi}\ln(\frac{v_{\text{F}}\pi}{a\alpha \abs{u}}) + 2\lambda Nu^{2}.
\end{align*}
At small displacements the logarithmic term dominates, meaning dimerization still occurs spontaneously. This is known as Peierls instability.

There was an implicit degeneracy in the above argument with respect to the directionality of the hopping term (or, equivalently, the sign of $\delta t$). This corresponds to the dimerization being possible in one of two ways. In excited states, these dimerized variants can be found in different domains of the polymer. Between two such regions is a so-called domain wall, called a soliton. To describe it we will need an energy functional describing the process of dimerization. By introducing the staggered displacement
\begin{align*}
	\phi_{j} = (-1)^{j}u_{j},
\end{align*}
which is just $\pm u$ in the ground states, we make the anzats
\begin{align*}
	E(\phi) = \inte{}{}\dd{x}\frac{1}{2}A\left(\dv{\phi}{x}\right)^{2} - B\phi^{2} + C\phi^{4}.
\end{align*}
We make the anzats such that there is a non-zero displacement in the ground state and such there is a degeneracy with respect to the sign of $\phi$. There must be $\phi$-dependent terms because the case of $\phi$ being constant does not correspond to a constant displacement of the chain. The Euler-Lagrange equation for this theory is
\begin{align*}
	-2B\phi + 4C\phi^{3} - A\dv[2]{\phi}{x} = 0.
\end{align*}
Notable is the existence of the ground state
\begin{align*}
	\phi = \pm\sqrt{\frac{B}{2C}}.
\end{align*}
To understand the behavior of such a system we can consider $x$ as a time parameter and $\phi$ as a coordinate for a particle in a potential $V(q) = Bq^{2} - Cq^{4}$. This potential has two maxima in the ground-state values of $\phi$ and there exists a solution which moves from one to the other asymptotically with respect to $x$. This solution is
\begin{align*}
	\phi = \sqrt{\frac{B}{2C}}\tanh(\sqrt{\frac{B}{A}}x).
\end{align*}

Looking back to the Dirac equation which emerged before dimerization, we see that we will have to extend it according to
\begin{align*}
	\ham = -i\sigma_{x}\dv{x} + m(x)\sigma_{y},
\end{align*}
where $m(x) = \alpha\phi(x)$ according to the Hamiltonian, assuming the ground-state energy to be zero. These then decouple, and the solutions in position space are
\begin{align*}
	\psi_{1} = c_{1}e^{\integ{0}{x}{y}{m(x)\phi(x)}},\ \psi_{2} = c_{2}e^{-\integ{0}{x}{y}{m(x)\phi(x)}}.
\end{align*}
For a soliton we must enforce $c_{1} = 0$, and similarly for an antisolition we enforce $c_{2} = 0$. It also turns out that there are no normalizable ground states unless $m$ changes sign somewhere. We define the topological charge
\begin{align*}
	Q = \frac{1}{2}\left(\text{sgn}(\infty) - \text{sgn}(-\infty)\right).
\end{align*}
It turns out that
\begin{align*}
	Q = N_{2} - N_{1},
\end{align*}
where $N_{1, 2}$ is the number of zero modes of each kind.

Note that the Hamiltonian anticommutes with $\sigma_{z}$. This leads to $\sigma_{z}$ generating a so-called chiral symmetry. Its eigenstates thus come in pairs with opposite eigenvalues, except in the case where the eigenvalue is zero. This case thus applies to the unpaired bound state.

\paragraph{Charged Particles in Magnetic Fields}
For studying the quantum Hall effect we will need to study the physics of classical particles in magnetic fields. We recall that the Lagrangian is
\begin{align*}
	\lag = \frac{1}{2}m\dot{x}^{i}x_{i} - q\left(\phi(x) - \dot{x}^{i}A_{i}\right).
\end{align*}
The canonical momenta are
\begin{align*}
	p_{i} = \pdv{\lag}{\dot{x}^{i}} = m\dot{x}_{i} + qA_{i},
\end{align*}
hence the Hamiltonian is
\begin{align*}
	\ham =& p_{i}\dot{x}^{i} - \lag \\
	     =& \frac{1}{m}p_{i}(p^{i} - qA^{i}) - \frac{1}{2m}(p^{i} - qA^{i})(p_{i} - qA_{i}) + q\left(\phi(x) - \frac{1}{m}(p^{i} - qA^{i})A_{i}\right) \\
	     =& \frac{1}{2m}(p^{i} - qA^{i})(p_{i} - qA_{i}) + q\phi(x).
\end{align*}
Note that the canonical momenta are not the physical momenta.

Consider now a two-dimensional system with $\vb{B} = B\vb{e}_{z}$. A choice of $\vb{A}$ is then $\vb{A} = -xB\vb{e}_{y}$. For this case we find that $p_{x}$ is a component of the physical momentum and
\begin{align*}
	\ham = \frac{1}{2m}\left(p_{x}^{2} + (p_{y} + qBx)^{2}\right) + q\phi(x).
\end{align*}
For $\phi = 0$ this problem is separable and we can construct eigenstates of the Hamiltonian by starting with eigenstates of $p_{y}$. In $x$ we may then introduce
\begin{align*}
	\frac{1}{2}m\omega_{\text{c}}^{2} = \frac{1}{2m}\cdot q^{2}B^{2},\ x\p = x + \frac{p_{y}}{qB} = x \pm k_{y}\ell^{2},
\end{align*}
where we in the last step introduced the magnetic length $\ell = \sqrt{\frac{\hbar}{\abs{q}B}}$. The choice of sign depends on the charge of $q$, which we will take to be positive. The resulting Hamiltonian is then that of a shifted harmonic oscillator. Its energy levels are
\begin{align*}
	E = \left(n + \frac{1}{2}\right)\hbar\omega_{\text{c}}.
\end{align*}

We can now try to impose periodic boundary conditions. While this cannot be done in all directions, we can still argue that because the solutions in the $x$ directions are Gaussian, the solutions can be considered to be confined. The set of solutions that are localized within the sample are then those with $0 < k_{y} < \frac{L_{x}}{\ell^{2}}$. The degeneracy is then
\begin{align*}
	N \approx \frac{L_{y}}{2\pi}\inte{0}{\frac{L_{x}}{\ell^{2}}} = \frac{A}{2\pi\ell^{2}},
\end{align*}
which is the number of quanta of flux $\frac{2\pi\ell^{2}h}{q}$ contained in the sample. That is, every Landau level is paired with one flux quantum. What this means is that when the flux becomes an integer multiple of $2\pi\ell^{2}B$ one set of Landau will be filled and the one above will be empty.

Evidently there is no current in the $x$-direction, but what about the $y$ direction? The solutions contain a factor $e^{ik_{y}y}$, so there might be some. We compute this by only considering the lowest Landau level, equivalent to assuming the energy separation to be large. The physical momentum being $P_{i} = p_{i} - qA_{i}$ we have
\begin{align*}
	I_{y} =& \inte{}{}\dd[2]{\vb{r}}\expval{J_{y}} \\
	      =& \frac{qL_{y}}{m}\inte{-\infty}{\infty}\dd{x}\frac{1}{L_{y}}\sqrt{\frac{m\omega_{\text{c}}}{\pi\hbar}}e^{-ik_{y}y}e^{-\frac{m\omega_{\text{c}}}{2\hbar}(x + k_{y}\ell^{2})^{2}}(-i\hbar\del{}{y} + qBx)e^{ik_{y}y}e^{-\frac{m\omega_{\text{c}}}{2\hbar}(x + k_{y}\ell^{2})^{2}} \\
	      =& q\sqrt{\frac{\omega_{\text{c}}}{\pi m\hbar}}\inte{-\infty}{\infty}\dd{x}(\hbar k_{y} + qBx)e^{-\frac{m\omega_{\text{c}}}{\hbar}(x + k_{y}\ell^{2})^{2}},
\end{align*}
which vanishes identically. This is not surprising as $k_{y}$ only determines the center of the harmonic oscillator.

Let us now add an electric field using $\phi = -Ex$. The choice of $\vb{A}$ is still appropriate, and the Hamiltonian is
\begin{align*}
	\ham &= \frac{1}{2m}\left(p_{x}^{2} + (p_{y} + qBx)^{2}\right) - qEx \\
	     &= \frac{1}{2m}\left(p_{x}^{2} + p_{y}^{2} + 2qBxp_{y} + q^{2}B^{2}x^{2} - 2mqEx\right) \\
	     =& \frac{1}{2m}\left(p_{x}^{2} + \left(qBx + p_{y} - \frac{mE}{B}\right)^{2} - \left(p_{y} - \frac{mE}{B}\right)^{2} + p_{y}^{2}\right) \\
	     =& \frac{1}{2m}p_{x}^{2} + \frac{1}{2}m\frac{q^{2}B^{2}}{m^{2}}\left(x + \frac{p_{y}}{qB} - \frac{mE}{qB^{2}}\right)^{2} - \frac{1}{2m}\left(\left(\frac{mE}{B}\right)^{2} - \frac{2p_{y}mE}{B}\right) \\
	     =& \frac{1}{2m}p_{x}^{2} + \frac{1}{2}m\omega_{\text{c}}^{2}\left(x + k_{y}\ell^{2} - \frac{mE}{qB^{2}}\right)^{2} - \frac{1}{2}m\left(\frac{E}{B}\right)^{2} +  qE\left(k_{y}\ell^{2} - \frac{mE}{qB^{2}}\right).
\end{align*}
This has an extra term in the peak position which, in the case of weak fields, can be taken to be evaluated at the center of the Landau level. This system also has a net current due to the recentering of the states. Note that the electric field lifts the degeneracy of the individual levels but introduces degeneracy between levels. States cannot decay between levels unless there exists disorder or phonons on which the states can scatter.

\paragraph{The Integer Quantum Hall Effect}
%TODO: Please fix this
Suppose you were to make the sample finite by introducing a potential that screens off a region of space. We can surmise that
\begin{align*}
	\psi = \frac{1}{\sqrt{L_{y}}}e^{ik_{y}y}f_{k_{y}}(x)
\end{align*}
for some set of functions $f_{k_{y}}$. The group velocity
\begin{align*}
	v_{y} = \frac{1}{\hbar}\pdv{E}{k_{y}} \approx \frac{\ell^{2}}{\hbar}\dv{V}{x}.
\end{align*}
By the confining nature of the potential the group velocity thus has different directions in either end of the sample. The net current is approximately
\begin{align*}
	I = \frac{q}{L_{y}}\inte{-\infty}{\infty}\frac{L_{y}}{2\pi\hbar}\dd{k_{y}}\pdv{E}{k_{y}}n_{k_{y}},
\end{align*}
where $n_{k_{y}}$ is the probability of a given state being occupied. In the limit of zero temperature we find
\begin{align*}
	I = \frac{q}{\hbar}\Delta\mu,
\end{align*}
where $\Delta\mu$ is the change in chemical potential between either edge. Defining the Hall voltage according to $V_{\text{H}} = \frac{\Delta\mu}{q}$ we find $I = \frac{\nu q^{2}}{\hbar}V_{\text{H}}$, where we introduce the number $\nu$ of occupied Landau levels in the bulk.

\paragraph{Berry Phase}
Consider a Hamiltonian described by some set of parameters $R$, which may be time dependent. For each value of $R$ we have a set of eigenstates
\begin{align*}
	\ham(R)\ket{n(R)} = E_{n}(R)\ket{n(R)}.
\end{align*}
If the spectrum is discrete and non-degenerate for all $R$, the adiabatic theorem tells us that if $R$ is varied such that the Hamiltonian changes sufficiently slowly, a state which is initialized to an eigenstate at $t = 0$ will evolve to a corresponding eigenstate at a later time. In the general case we have
\begin{align*}
	\ket{\psi_{n}(t)} = e^{i\gamma_{n}(t)}e^{-\frac{i}{\hbar}\inte{0}{t}\dd{\tau} E_{n}(\tau)}\ket{n(R(t))}.
\end{align*}
If the Hamiltonian has explicit time dependence, the former factor is non-zero. Its exponent is the so-called Berry phase. Inserting this into the Schrödinger equation we find
\begin{align*}
	\ham\ket{\psi_{n}(t)} =& E_{n}\ket{\psi_{n}(t)} \\
	=& i\hbar e^{i\gamma_{n}(t)}e^{-\frac{i}{\hbar}\inte{0}{t}\dd{\tau} E_{n}(\tau)}\left(i\dv{\gamma_{n}}{t}\ket{n(R(t))} - \frac{i}{\hbar}E_{n} \ket{n(R(t))} + \pdv{t}\ket{n(R(t))}\right),
\end{align*}
and taking the inner product with $\ket{\psi_{n}(t)}$ we have
\begin{align*}
	\dv{\gamma_{n}}{t}\ket{n(R(t))} = i\expval{\pdv{t}}{n(R(t))}.
\end{align*}
The solution is
\begin{align*}
	\gamma_{n} = i\inte{0}{t}\tau\expval{\pdv{\tau}}{n(R(\tau))}.
\end{align*}
Noting that
\begin{align*}
	\pdv{\tau}\ket{n(R(\tau))} =& \dv{R}{\tau}\cdot\grad_{R}\ket{n(R)}
\end{align*}
we can define the Berry connection
\begin{align*}
	A_{n} = i\expval{\grad_{R}}{n(R)}
\end{align*}
and find
\begin{align*}
	\gamma_{n} = i\inte{C}{}\dd{R}\cdot A_{n}.
\end{align*}
$C$ is now the orbit in parameter space traversed during the time evolution.

Is the Berry phase really of interest? It might not seem so. Making a phase change
\begin{align*}
	\ket{n(R)}\to e^{i\zeta(R)}\ket{n(R)}
\end{align*}
we have
\begin{align*}
	A_{n}\to A_{n} - \grad_{R}\zeta.
\end{align*}
Thus we can apparently remove the Berry phase entirely. For closed orbits in parameter space, however, the Berry phase is independent of this transformation. Thus it might be an observable.

Due to Stokes' theorem the line integral of the Berry connection about some closed path is related to the surface integral of the differential of some antisymmetric tensor. That is, we can define the Berry curvature
\begin{align*}
	\omega_{n, \mu\nu} = \del{}{\mu}A_{n, \nu} - \del{}{\nu}A_{n, \mu},
\end{align*}
which satisfies
\begin{align*}
	\inte{\bound{S}}{}\dd{R}\cdot A_{n} = \frac{1}{2}\inte{S}{}\dd{R_{\mu}}\wedge\dd{R_{\nu}}\omega_{n, \mu\nu}.
\end{align*}
In the particular case of three dimensions the Berry curvature can be expressed in terms of a pseudomagnetic field
\begin{align*}
	\vb{b}_{n} = i\left(\grad_{\vb{R}}\ket{n(\vb{r})}\right)\adj\times\grad_{\vb{R}}\ket{n(\vb{r})}.
\end{align*}

\paragraph{The Aharanov-Bohm Effect}
Consider two wires passing a region with a very localized magnetic field, one wire being wrapped around this region. This will turn out to be a case where the Berry phase has a physical consequence, as one can measure a quantum phase difference between electrons passing through either wire.

To study this, we will consider a charged particle confined to some box with one corner at $\vb{R}_{0}$ by a potential $V(\vb{r} - \vb{R}_{0})$. We will study this by adiabatically moving the box around in space. If the box never enters the region with magnetic field, we may take
\begin{align*}
	\vb{A} = \frac{\Phi_{0}}{2\pi}\grad{\chi}.
\end{align*}
Supposing the ground state of the electron in the box is $\xi(\vb{r} - \vb{R}_{0})$ in the absence of the magnetic field, we can find a normalized eigenfunction $e^{i\phi}\xi(\vb{r} - \vb{R}_{0})$ by the gauge transformation anzats
\begin{align*}
	(\vb{p} - q\vb{A})e^{i\phi}\xi(\vb{r} - \vb{R}_{0}) = e^{i\phi}\vb{p}\xi(\vb{r} - \vb{R}_{0}).
\end{align*}
We find
\begin{align*}
	(\vb{p} - q\vb{A})e^{i\phi}\xi(\vb{r} - \vb{R}_{0}) = e^{i\phi}\vb{p}\xi(\vb{r} - \vb{R}_{0}) - \left(i\hbar\grad{\phi} + q\vb{A}\right)\xi(\vb{r} - \vb{R}_{0}),
\end{align*}
hence
\begin{align*}
	\phi = -\frac{q}{\hbar}\inte{\vb{R}_{0}}{\vb{r}}\dd{\vb{r}\p}\cdot\vb{A} = -\frac{2\pi}{\Phi_{0}}\inte{\vb{R}_{0}}{\vb{r}}\dd{\vb{r}\p}\cdot\vb{A} = \chi(\vb{R}_{0}) - \chi(\vb{r}).
\end{align*}

There is still an ambiguity in the choice of \chi, as well as the possibility to add an $\vb{R}_{0}$-dependent phase $\theta$ to the wavefunction. We now fix them by imposing that
\begin{align*}
	\psi(\vb{r}) = e^{i(\theta + \chi(\vb{R}_{0}) - \chi(\vb{r}))}\xi(\vb{r} - \vb{R}_{0})
\end{align*}
be real at some point $\vb*{\Delta}$ as measured within the box. Assuming $\xi$ to be real and positive at $\vb*{\Delta}$, we require
\begin{align*}
	\theta = \chi(\vb{R}_{0} + \vb*{\Delta}) - \chi(\vb{R}_{0}),
\end{align*}
hence
\begin{align*}
	\psi(\vb{r}) = e^{i(\chi(\vb{R}_{0} + \vb*{\Delta}) - \chi(\vb{r}))}\xi(\vb{r} - \vb{R}_{0}).
\end{align*}

How does this tie in to the Berry phase? $\vb{R}_{0}$ now plays the role of the parameters in the Hamiltonian. For the ground state we then have
\begin{align*}
	\vb{A} =& i\integ[3]{}{}{\vb{r}}{\psi\cc\grad_{\vb{R}_{0}}\psi} \\
	       =& i\integ[3]{}{}{\vb{r}}{e^{-i\chi(\vb{R}_{0} + \vb*{\Delta})}\xi(\vb{r} - \vb{R}_{0})\left(ie^{i\chi(\vb{R}_{0} + \vb*{\Delta})}\xi(\vb{r} - \vb{R}_{0})\grad_{\vb{R}_{0}}\chi(\vb{R}_{0} + \vb*{\Delta}) + e^{i\chi(\vb{R}_{0} + \vb*{\Delta})}\grad_{\vb{R}_{0}}\xi(\vb{r} - \vb{R}_{0})\right)} \\
	       =& i\integ[3]{}{}{\vb{r}}{i\xi^{2}(\vb{r} - \vb{R}_{0})\grad_{\vb{R}_{0}}\chi(\vb{R}_{0} + \vb*{\Delta}) + \xi(\vb{r} - \vb{R}_{0})\grad_{\vb{R}_{0}}\xi(\vb{r} - \vb{R}_{0})} \\
	       =& -\frac{2\pi}{\Phi_{0}}\vb{A}(\vb{R}_{0} + \vb*{\Delta}) + \frac{1}{2}i\integ[3]{}{}{\vb{r}}{\grad_{\vb{R}_{0}}\xi^{2}(\vb{r} - \vb{R}_{0})},
\end{align*}
having used the fact that the state is normalized. The latter term is the integral of a total derivative, which yields no contribution as the wavefunction vanishes at infinity. Thus, by slowly moving the electron we have
\begin{align*}
	\gamma = -\inte{}{}\dd{\vb{R}_{0}}\cdot\frac{2\pi}{\Phi_{0}}\vb{A}(\vb{R}_{0} + \vb*{\Delta}).
\end{align*}
Taking the displacement $\vb*{\Delta}$ to be small, the phase difference between two wires is thus
\begin{align*}
	\delta\phi = -\frac{2\pi\Phi}{\Phi_{0}},
\end{align*}
where $\Phi$ is the total enclosed flux. Note that at no point along the path of the electron did it have to interact with the magnetic field, and yet this phase difference arises. This effect is thus a pure quantum effect, and is termed the Aharanov-Bohm effect.

\paragraph{Spin-$\frac{1}{2}$ Berry Phase}

\paragraph{Berry Phase and Bloch Bands}
A slight generalization of Bloch states can be made for systems in a uniform magnetic field. In a periodic potential the Hamiltonian is invariant under translations by lattice vectors. Generally, in a magnetic field, we have
\begin{align*}
	T_{\vb{a}}\ham T_{\vb{a}}\adj =& \frac{1}{2m}\left(\vb{p} - q\vb{A}(\vb{r} + \vb{a})\right)^{2} + V(\vb{r}).
\end{align*}
In a uniform field we have
\begin{align*}
	\vb{A}(\vb{r} + \vb{a}) - \vb{A}(\vb{r}) = \grad{f_{\vb{a}}}(\vb{r}).
\end{align*}
An extra gauge transformation $e^{i\phi_{\vb{a}}(\vb{r})}$ will certainly leave the potential invariant. We find
\begin{align*}
	\left(\vb{p} - q\vb{A}(\vb{r} + \vb{a})\right)e^{-i\phi_{\vb{a}}(\vb{r})} = e^{-i\phi_{\vb{a}}(\vb{r})}\left(\vb{p} - \grad{\phi_{\vb{a}}(\vb{r})} - q\vb{A}(\vb{r} + \vb{a})\right),
\end{align*}
so we require
\begin{align*}
	\grad{\phi_{\vb{a}}(\vb{r})} + q\vb{A}(\vb{r} + \vb{a}) = q\vb{A}(\vb{r}),
\end{align*}
with the solution
\begin{align*}
	\phi_{\vb{a}} = -qf_{\vb{a}}.
\end{align*}
The operator
\begin{align*}
	T_{\vb{a}} = e^{-iqf_{\vb{a}}(\vb{r})}e^{\frac{i}{\hbar}\vb{p}\cdot\vb{a}}
\end{align*}
thus leaves the Hamiltonian invariant. The Bloch basis is still applicable due to the translation symmetry of the system. Having established this we can now take $\vb{k}$ to be a parameter of the Hamiltonian in the Bloch basis.

Consider now such a system in a weak uniform magnetic field. This gives rise to a perturbation term $-q\vb{E}\cdot\vb{r}$ in the Hamiltonian. The first-order change in the state is then
\begin{align*}
	\delta\ket{n, \vb{k}} = -q\vb{E}\cdot\sum\limits_{m, \vb{q}}\ket{m, \vb{q}}\frac{\mel{m, \vb{q}}{\vb{r}}{n, \vb{k}}}{E_{n, \vb{k}} - E_{m, \vb{q}}}.
\end{align*}
As the involved states are eigenstates of the Hamiltonian we can write this as matrix elements of $\comm{\vb{r}}{\ham}$. We have
\begin{align*}
	\comm{\vb{r}}{\ham} = \frac{1}{2m}\comm{\vb{r}}{\vb{p}^{2}} = \frac{1}{2m}\left(\vb{p}\cdot\comm{\vb{r}}{\vb{p}} + \comm{\vb{r}}{\vb{p}}\cdot\vb{p}\right) = \frac{i\hbar}{m}\vb{p}.
\end{align*}
We then find
\begin{align*}
	\delta\ket{n, \vb{k}} = -i\hbar q\vb{E}\cdot\sum\limits_{m, \vb{q}}\ket{m, \vb{q}}\frac{\mel{m, \vb{q}}{\vb{v}}{n, \vb{k}}}{(E_{n, \vb{k}} - E_{m, \vb{q}})^{2}}.
\end{align*}
When computing the matrix element, the contributions from different unit cells differ by a factor $e^{i(\vb{k} - \vb{q})\cdot\vb{R}}$. Thus they vanish unless $\vb{q} = \vb{k}$. We then have
\begin{align*}
	\delta\ket{u_{n, \vb{k}}} = -iq\vb{E}\cdot\sum\limits_{m\neq n}\ket{u_{m, \vb{k}}}\frac{\mel{u_{m, \vb{k}}}{\grad_{\vb{k}}\ham}{u_{n, \vb{k}}}}{(E_{n, \vb{k}} - E_{m, \vb{k}})^{2}}.
\end{align*}
We then have to first order
\begin{align*}
	\expval{\vb{v}} =& \frac{1}{\hbar}\left(\expval{\grad_{\vb{k}}\ham}{u_{n, \vb{k}}} + \frac{2}{\hbar}\Re(\mel{\delta u_{n, \vb{k}}}{\grad_{\vb{k}}\ham}{u_{n, \vb{k}}})\right) \\
	=& \frac{1}{\hbar}\grad_{\vb{k}}\epsilon_{n, \vb{k}} + \vb{v}_{\text{a}}(n, \vb{k}).
\end{align*}
The new term is called the anomalous velocity. We have
\begin{align*}
	\mel{u_{n, \vb{k}}}{\grad_{\vb{k}}\ham}{\delta u_{n, \vb{k}}} =&  -iq\sum\limits_{m\neq n}\mel{u_{n, \vb{k}}}{\grad_{\vb{k}}\ham}{u_{m, \vb{k}}}\vb{E}\cdot\frac{\mel{u_{m, \vb{k}}}{\grad_{\vb{k}}\ham}{u_{n, \vb{k}}}}{(E_{n, \vb{k}} - E_{m, \vb{k}})^{2}}.
\end{align*}
Its complex conjugate is
\begin{align*}
	\mel{\delta u_{n, \vb{k}}}{\grad_{\vb{k}}\ham}{u_{n, \vb{k}}} =&  iq\sum\limits_{m\neq n}\mel{u_{m, \vb{k}}}{\grad_{\vb{k}}\ham}{u_{n, \vb{k}}}\vb{E}\cdot\frac{\mel{u_{n, \vb{k}}}{\grad_{\vb{k}}\ham}{u_{m, \vb{k}}}}{(E_{n, \vb{k}} - E_{m, \vb{k}})^{2}}.
\end{align*}
Thu
\begin{align*}
	\vb{v}_{\text{a}}(n, \vb{k}) =& \frac{iq}{\hbar}\vb{E}\cdot\sum\limits_{m\neq n}\frac{\mel{u_{n, \vb{k}}}{\grad_{\vb{k}}\ham}{u_{m, \vb{k}}}\mel{u_{m, \vb{k}}}{\grad_{\vb{k}}\ham}{u_{n, \vb{k}}} - \mel{u_{m, \vb{k}}}{\grad_{\vb{k}}\ham}{u_{n, \vb{k}}}\mel{u_{n, \vb{k}}}{\grad_{\vb{k}}\ham}{u_{m, \vb{k}}}}{(E_{n, \vb{k}} - E_{m, \vb{k}})^{2}}.
\end{align*}
It can be shown that the sum is equal to the Berry curvature, hence
\begin{align*}
	v_{\text{a}, i}(n, \vb{k}) =& \frac{iq}{\hbar}E^{j}\omega_{n, ji} = \frac{iq}{\hbar}\leci{}{ijk}E^{j}b_{n}^{k}.
\end{align*}
Noting that perturbation theory creates semiclassical motion, we can write
\begin{align*}
	\vb{v}_{\text{a}}(n, \vb{k}) = -\dv{\vb{k}}{t}\times\vb{b}_{n}.
\end{align*}
For a wave packet we can then show that
\begin{align*}
	\hbar\dv{\vb{R}}{t} = \grad_{\vb{k}}\epsilon_{n, \vb{k}} - \hbar\dv{\vb{k}}{t}\times\vb{b}_{n},\ \hbar\dv{\vb{k}}{t} = -\grad_{\vb{R}}V + q\dv{\vb{R}}{t}\times\vb{B}.
\end{align*}
Using this, we can consider the effect of time reversal. Evidently it reverses $\vb{v}$ and $\vb{k}$, which must imply $\vb{b}_{n}(\vb{k}) = -\vb{b}_{n}(-\vb{k})$. Under parity we instead have that $\vb{v}$, $\vb{k}$ and $\vb{E}$ are reversed, implying $\vb{b}_{n}(\vb{k}) = \vb{b}_{n}(-\vb{k})$. For a Hamiltonian with both symmetries, the Berry curvature must therefore be zero, and so must the anomalous velocity be.

\paragraph{Quantization of Hall Conductance}
In the presence of a magnetic field, time reversal symmetry in a semiconductor is broken. We have for a filled band
\begin{align*}
	\vb{J}_{n} = -\frac{e}{A}\sum\limits_{\vb{k}}\vb{v}_{\text{a}}(n, \vb{k}) = \sigma_{xy}^{n}\vb{e}_{z}\times\vb{E},
\end{align*}
allowing us to identify
\begin{align*}
	\sigma_{xy}^{n} = \frac{e^{2}}{\hbar A}\sum\limits_{\vb{k}}\vb{b}_{n}(\vb{k}) \approx \frac{e^{2}}{h}\inte{}{}\dd[2]{\vb{k}}b_{n}(\vb{k}),
\end{align*}
as the Berry curvature in this case always points in the $z$-direction. This is proportional to the total Berry curvature.

To discuss the implications of this, we first introduce the result
\begin{align*}
	\frac{1}{2\pi}\inte{M}{}\dd{A}K = 2 - 2g_{M}
\end{align*}
for a closed two-dimensional surface. $K$ is the curvature of the surface and $g_{M}$ is its genus. In the case of the Berry curvature, the point $\vb{k} = \vb{0}$ may not be included in the parameter domain as the Hamiltonian is not gapped there, thus there are only two topologically distinct classes of parameter surfaces allowed. We have
\begin{align*}
	\frac{1}{2\pi}\inte{}{}\dd[2]{\vb{k}}\omega_{n}(\vb{k}) = \frac{1}{2\pi}\inte{}{}\dd{\vb{k}}\cdot\vb{A}_{n}(\vb{k}) = \frac{i}{2\pi}\inte{}{}\dd{\vb{k}}\cdot\inte{}{}\dd[2]{\vb{r}}u_{n, \vb{k}}\cc\grad_{\vb{k}}u_{n, \vb{k}}.
\end{align*}
Note that in two dimensions we can use the Berry curvature directly. The $\vb{k}$ integral is about the boundary of the first Brillouin zone. Now, because we consider a general system, it may be the case that opposite edges of the Brillouin zone differ by a phase factor. The Bloch function is generally modified according to
\begin{align*}
	u_{n, \vb{k}} = e^{-i(\theta_{n}(\vb{k}) + \vb{G}\cdot\vb{r})}u_{n, \vb{k} + \vb{G}}.
\end{align*}
If $\theta_{n}$ is independent of $\vb{k}$, the gradients at opposite edges are the same, meaning their contributions cancel. Assuming the Bloch functions to be normalized we find in the general case that
\begin{align*}
	\frac{1}{2\pi}\inte{}{}\dd[2]{\vb{k}}\omega_{n}(\vb{k}) = \frac{1}{2\pi}\inte{}{}\dd{\vb{k}}\cdot\grad_{\vb{k}}\theta_{n}.
\end{align*}
Because we are integrating about a loop, this must be a multiple of $2\pi$ - in other words,
\begin{align*}
	\frac{1}{2\pi}\inte{}{}\dd[2]{\vb{k}}\omega_{n}(\vb{k})
\end{align*}
is an integer known as the first Chern number. This leads to the conductivity being quantized.

\paragraph{The Haldane Model}
The Haldane model is a simple model of two bands with non-zero Chern number. This is obtained by starting with a two-level model for the honeycomb lattice and opening a band gap and breaking time reversal symmetry. The Hamiltonian we use is
\begin{align*}
	\ham = \mqty[
		m              & -tf(\vb{k}) \\
		-tf\cc(\vb{k}) & -m
	].
\end{align*}
The eigenvalues are $\pm\sqrt{m^{2} + t^{2}\abs{f(\vb{k})}^{2}}$, using the same $f$ as for the honeycomb lattice. This opens up a band gap of size $2m$ at the corners of the Brillouin zone.

\paragraph{Superconductors}