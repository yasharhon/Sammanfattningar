\section{Basic Concepts}

\paragraph{All About Grassman Variables}
Grassman variables $\theta_{i}$ are defined such that multiplication of two different variables anticommutes. This means that all functions of a set of Grassman variables can be written as
\begin{align*}
	F(\theta) = \sum\limits_{i = 0}^{n}\frac{1}{i!}A^{(i), j_{1}\dots j_{i}}\theta_{j_{1}}\dots\theta_{j_{i}},
\end{align*}
with summation over the indices $j$. The coefficients $A$ are completely antisymmetric.

Functions of Grassman numbers are defined as even $A^{k} = 0$ for all even $k$ and similarly for odd functions. This implies that $FG = GF$ if either is even and $FG = -GF$ otherwise.

We define derivatives with respect to Grassman variables as
\begin{align*}
	\pdv{\theta_{i}}(\theta_{j}) = \delta_{ij}.
\end{align*}
These derivatives anticommute. They satisfy the product rule
\begin{align*}
	\pdv{\theta_{i}}(FG) = \pdv{F}{\theta_{i}}{G} + (-1)^{\abs{F}}F\pdv{G}{\theta_{i}},
\end{align*}
where $\abs{F}$ is $1$ if $F$ is even and $0$ if $F$ is odd. For a composite function
\begin{align*}
	f(G(\theta)) &= \sum\limits_{n}f_{n}G^{n}(\theta),
\end{align*}
the chain rule
\begin{align*}
	\pdv{\theta_{i}} f(G(\theta)) &= \pdv{G}{\theta_{i}}\dv{f}{G}
\end{align*}
also applies.

Integrals over Grassman variables are defined according to
\begin{align*}
	\inte{}{}\dd{\theta_{i}} = 0,\ \inte{}{}\dd{\theta_{i}}\theta_{j} = \delta_{ij}.
\end{align*}
This has the peculiar consequence
\begin{align*}
	\inte{}{}\dd{\theta_{i}}F(\theta) = \pdv{F}{\theta_{i}}.
\end{align*}
We know that all functions of Grassman variables can be factorized as $F(\theta) = A(\theta) + \theta_{i}B(\theta)$, where neither $A$ nor $B$ depend on $\theta_{i}$. This has the consequence that shifts in any one variable does not affect the integral. Scaling of variables does not work that way, however, as we saw that differentiation and integration has the same effect. This produces the general result
\begin{align*}
	\theta_{j} = A_{ij}\phi_{j}\implies \dd{\theta}_{1}\dots\dd{\theta}_{n} = \frac{1}{\det{A}}\dd{\phi}_{1}\dots\dd{\phi}_{n}.
\end{align*}

\paragraph{The Path Integral Approach}
We will now switch from canonical quantization to path integrals as our basis for quantum field theory. This has a few advantages:
\begin{itemize}
	\item It switches from the Hamiltonian to the Lagrangian, which can more easily be manifestly Lorentz invariant.
	\item The switch to the Lagrangian allows for more aesthetic handling of Lagrangians with field derivatives.
	\item It allows for better handling of non-Abelian gauge theories.
\end{itemize}

\paragraph{Path Integrals in Quantum Mechanics}
Consider a system with Lagrangian
\begin{align*}
	\ham = \frac{p^{2}}{2m} + V(q).
\end{align*}
In the Heisenberg pictures, the $q$ operator is given by
\begin{align*}
	q(t) = e^{i\ham t}q(0)e^{-i\ham t},
\end{align*}
We study states of the form $\ket{q, t}$, which are eigenstates of $q(t)$ with eigenvalue $q$. We note that
\begin{align*}
	q(t)e^{i\ham t}\ket{q} = e^{i\ham t}qe^{-i\ham t}e^{i\ham t}\ket{q} = qe^{i\ham t}\ket{q},
\end{align*}
hence this is the form of the states we are considering. Eigenstates at reference time $t\p$ will be implicitly labelled when convenient.

We might consider the amplitude of the system evolving from $q\p$ to $q^{\prime\prime}$ over some time. This is given by
\begin{align*}
	\braket{q^{\prime\prime}, t^{\prime\prime}}{q\p, t\p} = \mel{q^{\prime\prime}}{e^{-i\ham(t^{\prime\prime} - t\p)}}{q\p},
\end{align*}
which coincides with the expression in the Schrödinger picture. To introduce the path integral we will discretize the time interval into $N$ slices of width $\Delta$ and write
\begin{align*}
	\mel{q^{\prime\prime}}{e^{-i\ham(t^{\prime\prime} - t\p)}}{q\p} = \mel{q^{\prime\prime}}{e^{-i\ham\Delta}\dots e^{-i\ham\Delta}}{q\p}.
\end{align*}
Between each pair of time evolution operators we now place a completeness relation, reducing the problem to computing
\begin{align*}
	\mel{q_{i + 1}}{e^{-i\ham\Delta}}{q_{i}}.
\end{align*}
The final value of the amplitude will be the product of all such factors from $0$ to $N$. We can now split the exponentials according to
\begin{align*}
	\mel{q_{i + 1}}{e^{-i\ham\Delta}}{q_{i}} = \mel{q_{i + 1}}{e^{-i\Delta\frac{p^{2}}{2m}}e^{-i\Delta V(q)}e^{-\frac{1}{2}\Delta^{2}\comm{\frac{p^{2}}{2m}}{V(q)}}\dots}{q_{i}}.
\end{align*}
Due to our discretization we may now ignore the higher-order factors, leaving
\begin{align*}
	\mel{q_{i + 1}}{e^{-i\ham\Delta}}{q_{i}} = \mel{q_{i + 1}}{e^{-i\Delta\frac{p^{2}}{2m}}e^{-i\Delta V(q)}}{q_{i}}.
\end{align*}
The first factor is easily dealt with. To handle the second, we introduce a new completeness relation such that
\begin{align*}
	\mel{q_{i + 1}}{e^{-i\ham\Delta}}{q_{i}} &= e^{-i\Delta V(q_{i})}\integ{}{}{p}\mel{q_{i + 1}}{e^{-i\Delta\frac{p^{2}}{2m}}}{p}\braket{p}{q_{i}} \\
	                                         &= e^{-i\Delta V(q_{i})}\integ{}{}{p}e^{-i\Delta\frac{p^{2}}{2m}}\braket{q_{i + 1}}{p}\braket{p}{q_{i}} \\
	                                         &= e^{-i\Delta V(q_{i})}\integ{}{}{p}e^{-i\Delta\frac{p^{2}}{2m} + ip(q_{i + 1} - q_{i})} \\
	                                         &= e^{-i\Delta V(q_{i})}\integ{}{}{p}e^{-i\frac{\Delta}{2m}\left(p - \frac{m}{\Delta}(q_{i + 1} - q_{i})\right)^{2}}e^{i\frac{m\Delta}{2}\left(\frac{q_{i + 1} - q_{i}}{\Delta}\right)^{2}} \\
	                                         &= e^{i\Delta\left(\frac{1}{2}m\left(\frac{q_{i + 1} - q_{i}}{\Delta}\right)^{2} - V(q_{i})\right)}\integ{}{}{p}e^{-i\frac{\Delta}{2m}\left(p - \frac{m}{\Delta}(q_{i + 1} - q_{i})\right)^{2}}
\end{align*}
We will leave the integral as is, as it merely provides some normalization factor in the end. Dubbing it $A$ we find
\begin{align*}
	\braket{q^{\prime\prime}, t^{\prime\prime}}{q\p, t\p} &= \inte{}{}\dd{q}_{1}\dots\dd{q}_{N}A^{N}e^{i\Delta\left(\frac{1}{2}m\left(\frac{q_{1} - q\p}{\Delta}\right)^{2} - V(q\p)\right)}\dots e^{i\Delta\left(\frac{1}{2}m\left(\frac{q^{\prime\prime} - q_{N}}{\Delta}\right)^{2} - V(q_{N})\right)} \\
	                                                      &= \inte{}{}\dd{q}_{1}\dots\dd{q}_{N}A^{N}e^{i\sum\limits_{i = 0}^{N}\Delta\left(\frac{1}{2}m\left(\frac{q_{i + 1} - q_{i}}{\Delta}\right)^{2} - V(q_{i})\right)}.
\end{align*}
Now comes the kicker: We let $N$ go to infinity and $\Delta$ to zero such that the sum in the exponent goes to an integral. In this limit we see that what is left is in fact the integral of the Lagrangian, or the action. The action of what, you ask? It's the action corresponding to a particular choice of all the intermediate $q$.

The integration over an infinite set of such intermediate values corresponds to an integral over function space - over all possible functional forms of $q(t)$. This is what is termed the path integral. It is generally denoted as
\begin{align*}
	\pinte{}{x}.
\end{align*}
Renaming the normalization constant we therefore have
\begin{align*}
	\braket{q^{\prime\prime}, t^{\prime\prime}}{q\p, t\p} &= C\pinte{}{q}e^{iS}.
\end{align*}

One might ask whether the limit that defines the path integral really exists. Yes, one might very well ask that.

\paragraph{Observables from Path Integrals}
Let us now consider
\begin{align*}
	\mel{q^{\prime\prime}, t^{\prime\prime}}{\torp q(t_{n})\dots q(t_{1})}{q\p, t\p}.
\end{align*}
We may consider the times to be correctly ordered for convenience, allowing us to remove the time ordering. Discretizing time and taking $t_{i} - t\p = k_{i}\Delta$ we can write
\begin{align*}
	q(t_{i + 1})q(t_{i}) &= e^{i\ham k_{i + 1}\Delta}qe^{-i\ham k_{i + 1}\Delta}e^{i\ham k_{i}\Delta}qe^{-i\ham k_{i}\Delta} = e^{i\ham k_{i + 1}\Delta}qe^{-i\ham (k_{i + 1} - k_{i})\Delta}qe^{-i\ham k_{i}\Delta}.
\end{align*}
Using completeness we have
\begin{align*}
	e^{-i\ham (k_{i + 1} - k_{i})\Delta} = \inte{}{}\dd{q_{k_{i } + 1}}\dots\dd{q_{k_{i + 1}}}\op{q_{k_{i + 1}}}e^{-i\ham \Delta}\dots\op{q_{k_{i} + 1}}e^{-i\ham \Delta},
\end{align*}
hence
\begin{align*}
	q(t_{i + 1})q(t_{i}) &= \inte{}{}\dd{q_{k_{i } + 1}}\dots\dd{q_{k_{i + 1}}}e^{i\ham k_{i + 1}\Delta}q\op{q_{k_{i + 1}}}e^{-i\ham \Delta}\dots\op{q_{k_{i} + 1}}e^{-i\ham \Delta}qe^{-i\ham k_{i}\Delta} \\
	                     &= \inte{}{}\dd{q_{k_{i } + 1}}\dots\dd{q_{k_{i + 1}}}q_{k_{i + 1}}e^{i\ham k_{i + 1}\Delta}\op{q_{k_{i + 1}}}e^{-i\ham \Delta}\dots\op{q_{k_{i} + 1}}e^{-i\ham \Delta}qe^{-i\ham k_{i}\Delta}.
\end{align*}
On the left end we pair up $e^{-i\ham N\Delta}$ with $e^{i\ham k_{n}\Delta}$ and on the right end no modifications are needed, hence
\begin{align*}
	\mel{q^{\prime\prime}, t^{\prime\prime}}{\torp q(t_{n})\dots q(t_{1})}{q\p, t\p} &= \inte{}{}\dd{q_{1}}\dots\dd{q_{N}}q_{k_{1}}\dots q_{k_{n}}\mel{q^{\prime\prime}}{e^{-i\ham \Delta}}{q_{N}}\dots \mel{q_{1}}{e^{-i\ham \Delta}}{q\p}.
\end{align*}
We can now recognize results from previously, and in the limit of infinitely fine discretization we obtain
\begin{align*}
	\mel{q^{\prime\prime}, t^{\prime\prime}}{\torp q(t_{n})\dots q(t_{1})}{q\p, t\p} = C\pinte{}{q}q(t_{n})\dots q(t_{1})e^{iS}.
\end{align*}
Note that the ordering in the integration is no longer important.

%TODO: Show
As a side note, an ad hoc argument for why the path integral converges can be found by going back to the discrete variant, performing a Wick rotation of the time step $\Delta$ and thus show that the amplitude of the contributions diverges exponentially as one strays from the path of least action.

\paragraph{Green's Functions from Path Integrals}
We can use the above to obtain the Green's function from path integrals. We recall that the Green's function is defined as
\begin{align*}
	G(t_{1}, \dots, t_{n}) = \frac{\expval{\torp q(t_{1})\dots q(t_{n})}{\Omega}}{\braket{\Omega}}.
\end{align*}
Here we  will use $t = 0$ as our reference. To get an expression in terms of the ground state we use completion to write
\begin{align*}
	\mel{q^{\prime\prime}, t^{\prime\prime}}{\torp q(t_{n})\dots q(t_{1})}{q\p, t\p} &= \sum\limits_{n, m}\mel{q^{\prime\prime}}{e^{-i\ham t^{\prime\prime}}}{n}\mel{n}{\torp q(t_{n})\dots q(t_{1})}{m}\mel{m}{e^{i\ham t\p}}{q\p} \\
	&= \sum\limits_{n, m}e^{i(E_{m}t\p - E_{n}t^{\prime\prime})}\braket{q^{\prime\prime}}{n}\mel{n}{\torp q(t_{n})\dots q(t_{1})}{m}\braket{m}{q\p}.
\end{align*}
We will now consider the limit of large times (which is what is relevant for QFT contexts anyway). We will also extend time to be complex. In summary we write
\begin{align*}
	t\p = -\tau(1 - i\varepsilon),\ t^{\prime\prime} = \tau(1 - i\varepsilon),
\end{align*}
where we take $\tau$ to be large and let $\varepsilon\to 0$ at the end of our calculations. Thus we have
\begin{align*}
	\mel{q^{\prime\prime}, t^{\prime\prime}}{\torp q(t_{n})\dots q(t_{1})}{q\p, t\p} &= \sum\limits_{n, m}e^{-i\tau(1 - i\varepsilon)(E_{m} + E_{n})}\braket{q^{\prime\prime}}{n}\mel{n}{\torp q(t_{n})\dots q(t_{1})}{m}\braket{m}{q\p} \\
	&= \sum\limits_{n, m}e^{-i\tau(E_{m} + E_{n})}e^{-\tau\varepsilon(E_{m} + E_{n})}\braket{q^{\prime\prime}}{n}\mel{n}{\torp q(t_{n})\dots q(t_{1})}{m}\braket{m}{q\p}.
\end{align*}
In the limit of large times the only contribution thus comes from the ground state, and we have
\begin{align*}
	\mel{q^{\prime\prime}, t^{\prime\prime}}{\torp q(t_{n})\dots q(t_{1})}{q\p, t\p} &= \braket{q^{\prime\prime}}{\Omega}\braket{\Omega}{q\p}\mel{\Omega}{\torp q(t_{n})\dots q(t_{1})}{\Omega}.
\end{align*}
The left-hand side is now a path integral and the right-hand side is proportional to the Green's function. We thus have
\begin{align*}
	G(t_{1}, \dots, t_{n}) = \frac{\pinte{}{q}q(t_{n})\dots q(t_{1})e^{iS}}{\pinte{}{q}e^{iS}}.
\end{align*}

\paragraph{Path Integrals in Field Theories}
In moving to field theory there is really nothing new that is introduced. To see this you could simply perform your field theory calculation on a lattice and reuse what we have done above. The eigenstates of $q$ are now replaced by eigenstates of the fields and times are replaced by points in spacetime. We thus arrive at
\begin{align*}
	G(x_{1}, \dots, x_{n}) = \frac{\pinte{}{\phi}\phi(t_{n})\dots \phi(x_{1})e^{iS}}{\pinte{}{\phi}e^{iS}}.
\end{align*}
The path integral is now over all possible forms of the field $\phi$ throughout spacetime.

\paragraph{The Generating Functional}
The generating functional is defined as
\begin{align*}
	Z = \pinte{}{\phi}e^{i\left(S + \inte{}{}\dd[4]{x}\phi J\right)}.
\end{align*}
It is a functional of the extra field $J$. It evidently satisfies
\begin{align*}
	Z(0) = \pinte{}{\phi}e^{iS}.
\end{align*}
Furthermore it satisfies
\begin{align*}
	-i\var{Z} &= \pinte{}{\phi}\phi e^{i\left(S + \inte{}{}\dd[4]{x}\phi J\right)}\cdot\dd{\alpha}\dv{\alpha}\inte{}{}\dd[4]{x}\phi J \\
	          &= \inte{}{}\dd[4]{x}\pinte{}{\phi}\phi e^{i\left(S + \inte{}{}\dd[4]{x}\phi J\right)}\cdot\dv{\alpha}\phi\var{J},
\end{align*}
hence
\begin{align*}
	-i\fdv{Z}{J} &= \pinte{}{\phi}\phi e^{i\left(S + \inte{}{}\dd[4]{x}\phi J\right)}\phi.
\end{align*}
Similarly we can apply more functional derivatives and consider specific points, setting $J = 0$ at the end, to compute correlation functions.

\paragraph{The Generating Functional for a Free Scalar Field}
Consider the non-interacting scalar field theory with Lagrangian
\begin{align*}
	\lag = -\frac{1}{2}\del{}{\mu}\phi\del{\mu}{}\phi - \frac{1}{2}m^{2}\phi^{2},
\end{align*}
taken with the field to approach zero at infinity. Before writing the generating functional, we rewrite the action as
\begin{align*}
	S &= \integ[4]{}{}{x}{-\frac{1}{2}\del{}{\mu}\phi\del{\mu}{}\phi - \frac{1}{2}m^{2}\phi^{2}} \\
	  &= \integ[4]{}{}{x}{-\frac{1}{2}\left(\del{}{\mu}\left(\phi\del{\mu}{}\phi\right) - \phi\del{}{\mu}\del{\mu}{}\phi\right) - \frac{1}{2}m^{2}\phi^{2}} \\
	  &= \integ[4]{}{}{x}{-\frac{1}{2}\phi\dalem\phi - \frac{1}{2}m^{2}\phi^{2}} \\
	  &= \integ[4]{}{}{x}{\frac{1}{2}\phi\left(-\dalem - m^{2}\right)\phi},
\end{align*}
where we have introduced the d'Alembert operator $-\del{}{\mu}\del{\mu}{}$. Next we are going to use a clever trick by introducing the Klein-Gordon propagator $i\tilde{G} = \Delta$, which satisfies
\begin{align*}
	\left(-\dalem - m^{2}\right)\tilde{G} = \delta^{4}(x - y).
\end{align*}
We then have
\begin{align*}
	\integ[4]{}{}{x}{-\frac{1}{2}\del{}{\mu}\phi\del{\mu}{}\phi - \frac{1}{2}m^{2}\phi^{2} + \phi J} &= \integ[4]{}{}{x}{\frac{1}{2}\phi\left(-\dalem - m^{2}\right)\phi + \phi\integ[4]{}{}{y}{\delta^{4}(x - y)J(y)}} \\
	&= \integ[4]{}{}{x}{\frac{1}{2}\phi\left(-\dalem_{x} - m^{2}\right)\phi + \phi\integ[4]{}{}{y}{J(y)\left(-\dalem_{x} - m^{2}\right)\tilde{G}(x, y)}}.
\end{align*}
We could do some very neat completing of squares here, but that would require
\begin{align*}
	\integ[4]{}{}{y}{J(y)\tilde{G}(x, y)\left(-\dalem_{x} - m^{2}\right)\phi} = \phi(x)J(x).
\end{align*}
To prove that we will perform some clever integration by parts. We have
\begin{align*}
	-\integ[4]{}{}{y}{J(y)\tilde{G}(x, y)\dalem_{x}\phi} &= \integ[4]{}{}{y}{\del{}{\mu}\left(J(y)\tilde{G}(x, y)\del{\mu}{}\phi\right) - J(y)\del{}{\mu}\tilde{G}(x, y)\del{\mu}{}\phi} \\
	&= -\integ[4]{}{}{y}{\del{\mu}{}(J(y)\del{}{\mu}\tilde{G}(x, y)\phi) + J(y)\phi\dalem_{x}\tilde{G}(x, y)} \\
	&= -\integ[4]{}{}{y}{J(y)\phi\dalem_{x}\tilde{G}(x, y)}.
\end{align*}
This completes the proof, and we thus have
\begin{align*}
	\integ[4]{}{}{x}{-\frac{1}{2}\del{}{\mu}\phi\del{\mu}{}\phi - \frac{1}{2}m^{2}\phi^{2} + \phi J} =& \integ[4]{}{}{x}{\frac{1}{2}\phi\left(-\dalem_{x} - m^{2}\right)\phi \\
	&+ \frac{1}{2}\left(\integ[4]{}{}{y}{\phi(x)J(y)\left(-\dalem_{x} - m^{2}\right)\tilde{G}(x, y) + J(y)\tilde{G}(x, y)}\left(-\dalem_{x} - m^{2}\right)\phi(x)\right)} \\
	=& \frac{1}{2}\integ[4]{}{}{x}{\left(\phi + \integ[4]{}{}{y}{J(y)G(x, y)}\right)\left(-\dalem_{x} - m^{2}\right)\left(\phi + \integ[4]{}{}{z}{J(z)G(x, z)}\right) \\
	 &- \integ[4]{}{}{y}{\integ[4]{}{}{z}{J(y)G(x, y)\left(-\dalem_{x} - m^{2}\right)J(z)G(x, z)}}}.
\end{align*}
The final integral can be simplified as
\begin{align*}
	\integ[4]{}{}{x}{\integ[4]{}{}{y}{\integ[4]{}{}{z}{J(y)G(x, y)\left(-\dalem_{x} - m^{2}\right)J(z)G(x, z)}}} &= \integ[4]{}{}{x}{\integ[4]{}{}{y}{\integ[4]{}{}{z}{J(y)G(x, y)J(z)\delta^{4}(x - z)}}} \\
	                           &= \integ[4]{}{}{x}{\integ[4]{}{}{y}{J(y)G(x, y)J(x)}}.
\end{align*}
Now define a new field
\begin{align*}
	\chi(x) = \phi(x) + \integ[4]{}{}{y}{J(y)G(x, y)}.
\end{align*}
This shift has no effect on the integration measure in the path integral, hence we have
\begin{align*}
	Z &= \pinte{}{\chi}e^{\frac{i}{2}\integ[4]{}{}{x}{\chi(x)\left(-\dalem_{x} - m^{2}\right)\chi(y) - \integ[4]{}{}{y}{J(y)G(x, y)J(x)}}} \\
	  &= Ne^{-\frac{1}{2}\integ[4]{}{}{x}{\integ[4]{}{}{y}{J(y)\Delta(x, y)J(x)}}}.
\end{align*}

Let's do a sanity check now, to be sure. We have
\begin{align*}
	&G(x_{1}, x_{2}) = \frac{1}{Z(0)}\left(-i\fdv{J(x_{1})}\right)\left(-i\fdv{J(x_{2})}\right)N_{c}e^{-\frac{1}{2}\integ[4]{}{}{x}{\integ[4]{}{}{y}{J(y)\Delta(x, y)J(x)}}} \\
	                =& \fdv{J(x_{1})}\left(\frac{1}{2}\integ[4]{}{}{x}{\integ[4]{}{}{y}{J(y)\Delta(x, y)\delta^{4}(x - x_{2})}} + \frac{1}{2}\integ[4]{}{}{x}{\integ[4]{}{}{y}{\delta^{4}(y - x_{2})\Delta(x, y)J(x)}}\right)e^{-\frac{1}{2}\integ[4]{}{}{x}{\integ[4]{}{}{y}{J(y)\Delta(x, y)J(x)}}} \\
	                =& \fdv{J(x_{1})}\left(\frac{1}{2}\integ[4]{}{}{y}{J(y)\Delta(x_{2}, y)} + \frac{1}{2}\integ[4]{}{}{x}{\Delta(x, x_{2})J(x)}\right)e^{-\frac{1}{2}\integ[4]{}{}{x}{\integ[4]{}{}{y}{J(y)\Delta(x, y)J(x)}}}.
\end{align*}
We would in principle need to apply the product rule, but it is now obvious that only the term which applies the functional derivative to the non-exponential part will give a non-zero contribution, hence we find
\begin{align*}
	G(x_{1}, x_{2}) = \frac{1}{2}(\Delta(x_{1}, x_{2}) + \Delta(x_{2}, x_{1})),
\end{align*}
and using the fact that the propagator is symmetric under switching the two arguments we have
\begin{align*}
	G(x_{1}, x_{2}) = \Delta(x_{1}, x_{2}),
\end{align*}
as expected.

\paragraph{The Generating Functional for an Interacting Scalar Field}
Consider the interacting scalar field theory described by
\begin{align*}
	\lag = -\frac{1}{2}\del{}{\mu}\phi\del{\mu}{}\phi - \frac{1}{2}m^{2}\phi^{2} - \frac{1}{4!}\lambda\phi^{4}.
\end{align*}
The generating functional is given by
\begin{align*}
	Z = \pinte{}{\phi}e^{i\left(S_{\text{free}} + \integ[4]{}{}{x}{\phi J - \frac{1}{4!}\lambda\phi^{4}}\right)}.
\end{align*}
We will be working in the framework of perturbation theory and will therefore at some point take the parameter $\lambda$ to be small. We prepare for that by performing a series expansion in $\lambda$ according to
\begin{align*}
	Z = \sum\limits_{m = 0}^{\infty}\frac{1}{m!}\left(-\frac{i\lambda}{4!}\right)^{m}\pinte{}{\phi}\left(\integ[4]{}{}{x}{\phi^{4}}\right)^{m}e^{i\left(S_{\text{free}} + \integ[4]{}{}{x}{\phi J}\right)}.
\end{align*}
Here comes the kicker: The fields can be brought down from the exponent by functional differentiation. Now, that differentiation has no dependence on $\phi$, hence we may extract it from the path integral to find
\begin{align*}
	Z &= \sum\limits_{m = 0}^{\infty}\frac{1}{m!}\left(-\frac{i\lambda}{4!}\right)^{m}\left(\integ[4]{}{}{x}{\fdv[m]{J}\right)^{m}}\pinte{}{\phi}e^{i\left(S_{\text{free}} + \integ[4]{}{}{x}{\phi J}\right)} \\
	  &= \sum\limits_{m = 0}^{\infty}\frac{1}{m!}\left(-\frac{i\lambda}{4!}\right)^{m}\left(\integ[4]{}{}{x}{\fdv[m]{J}\right)^{m}}Z_{\text{free}}.
\end{align*}
This can be related to Feynman diagrams in a very straightforward way.
%TODO: Complete

\paragraph{The Generating Functional in Momentum Space}
Introducing the Fourier transform
\begin{align*}
	\phi(x) = \inte{}{}\frac{\dd[4]{k}}{(2\pi)^{4}}\tilde{\phi}(k)e^{ikx},
\end{align*}
we can write the $n$-point correlation function as
\begin{align*}
	G(x_{1}, \dots, x_{n}) = \frac{1}{Z(0)}\pinte{}{\phi}e^{iS}\inte{}{}\frac{\dd[4]{k_{1}}}{(2\pi)^{4}}\dots \frac{\dd[4]{k_{n}}}{(2\pi)^{4}}e^{i\sum\limits_{i}k_{i}x_{i}}\tilde{\phi}(k_{1})\dots \tilde{\phi}(k_{n}).
\end{align*}
Defining
\begin{align*}
	G(x_{1}, \dots, x_{n}) = \inte{}{}\frac{\dd[4]{k_{1}}}{(2\pi)^{4}}\dots \frac{\dd[4]{k_{n}}}{(2\pi)^{4}}e^{i\sum\limits_{i}k_{i}x_{i}}\tilde{G}(k_{1}, \dots, k_{n})
\end{align*}
we then see
\begin{align*}
	\tilde{G}(k_{1}, \dots, k_{n}) = \frac{1}{Z(0)}\pinte{}{\phi}e^{iS}\tilde{\phi}(k_{1})\dots \tilde{\phi}(k_{n}).
\end{align*}
To write this even more neatly we note that we could in principle switch the path integral to be over $\tilde{\phi}$. This will at most produce a dimensional factor which is equal in the numerator and denominator, provided that we can appropriately define the generating functional. Let us therefore do this.

Consider the integral
\begin{align*}
	\inte{}{}\dd[4]{x}\phi(x)J(x) &= \inte{}{}\dd[4]{x}\inte{}{}\frac{\dd[4]{k_{1}}}{(2\pi)^{4}}\frac{\dd[4]{k_{2}}}{(2\pi)^{4}}e^{i(k_{1}x + k_{2}x)}\tilde{\phi}(k_{1})\tilde{J}(k_{2}).
\end{align*}
The two integrals arise because we are considering two different functions with their own unique Fourier transforms. Performing the $x$ integration we find
\begin{align*}
	\inte{}{}\dd[4]{x}\inte{}{}\frac{\dd[4]{k_{1}}}{(2\pi)^{4}}\frac{\dd[4]{k_{2}}}{(2\pi)^{4}}e^{i(k_{1}x + k_{2}x)}\tilde{\phi}(k_{1})\tilde{J}(k_{2}) &= \inte{}{}\frac{\dd[4]{k_{1}}\dd[4]{k_{2}}}{(2\pi)^{4}}\delta^{4}(k_{1} + k_{2})\tilde{\phi}(k_{1})\tilde{J}(k_{2}) \\
	&= \inte{}{}\frac{\dd[4]{k}}{(2\pi)^{4}}\tilde{\phi}(k)\tilde{J}(-k).
\end{align*}
Note the structure of this result, as it will be reused regularly. Pending a rewrite of the action in terms of $\tilde{\phi}$ we have
\begin{align*}
	Z[J] &= \pinte{}{\phi}e^{i\left(S[\phi] + \inte{}{}\dd[4]{x}\phi J\right)} \\
	     &= \pinte{}{\phi}e^{i\left(S[\tilde{\phi}] + \inte{}{}\frac{\dd[4]{k}}{(2\pi)^{4}}\tilde{\phi}(k)\tilde{J}(-k)\right)}.
\end{align*}
The change of integration measure involves a dimensional factor, as previously stated, but because it will always be common for factors in a numerator and denominator, we may define
\begin{align*}
	Z[\tilde{J}] &= \pinte{}{\tilde{\phi}}e^{i\left(S[\tilde{\phi}] + \inte{}{}\frac{\dd[4]{k}}{(2\pi)^{4}}\tilde{\phi}(k)\tilde{J}(-k)\right)}.
\end{align*}
This functional evidently satisfies
\begin{align*}
	\tilde{G}(k_{1}, \dots, k_{n}) = \frac{1}{Z[0]}\eval{\left(-i(2\pi)^{4}\fdv{\tilde{J}(k_{1})}\right)\dots\left(-i(2\pi)^{4}\fdv{\tilde{J}(k_{n})}\right)Z[\tilde{J}]}_{\tilde{J} = 0}.
\end{align*}

\paragraph{Momentum Space Generating Functional for a Free Scalar}
We once again consider a system with Lagrangian
\begin{align*}
	\lag = -\frac{1}{2}\del{}{\mu}\phi\del{\mu}{}\phi - \frac{1}{2}m^{2}\phi^{2}.
\end{align*}
The action can then be written as
\begin{align*}
	S &= \integ[4]{}{}{x}{-\frac{1}{2}\del{}{\mu}\phi\del{\mu}{}\phi - \frac{1}{2}m^{2}\phi^{2}} \\
	  &= \integ[4]{}{}{x}{\frac{1}{2}\phi\left(-\dalem - m^{2}\right)\phi} \\
	  &= \frac{1}{2}\integ[4]{}{}{x}{\inte{}{}\frac{\dd[4]{k_{1}}}{(2\pi)^{4}}\frac{\dd[4]{k_{2}}}{(2\pi)^{4}}e^{ik_{1}x}\tilde{\phi}(k_{1})\left(-\dalem - m^{2}\right)\left(e^{ik_{2}x}\tilde{\phi}(k_{2})\right)} \\
	  &= \frac{1}{2}\integ[4]{}{}{x}{\inte{}{}\frac{\dd[4]{k_{1}}}{(2\pi)^{4}}\frac{\dd[4]{k_{2}}}{(2\pi)^{4}}e^{ik_{1}x}\tilde{\phi}(k_{1})\left(-k_{2}^{2} - m^{2}\right)e^{ik_{2}x}\tilde{\phi}(k_{2})} \\
	  &= -\frac{1}{2}\inte{}{}\frac{\dd[4]{k}}{(2\pi)^{4}}\tilde{\phi}(-k)\left(k^{2} + m^{2}\right)\tilde{\phi}(k).
\end{align*}
The integral in the exponent can once again, this time with a very straight-forward ompleting of the square, be written as
\begin{align*}
	 &\inte{}{}\frac{\dd[4]{k}}{(2\pi)^{4}}-\frac{1}{2}\tilde{\phi}(-k)\left(k^{2} + m^{2}\right)\tilde{\phi}(k) + \tilde{\phi}(k)\tilde{J}(-k) \\
	=& \inte{}{}\frac{\dd[4]{k}}{(2\pi)^{4}}-\frac{1}{2}\tilde{\phi}(-k)\left(k^{2} + m^{2}\right)\tilde{\phi}(k) + \frac{1}{2}(\tilde{\phi}(k)\tilde{J}(-k) + \tilde{\phi}(-k)\tilde{J}(k)) \\
	=& -\frac{1}{2}\inte{}{}\frac{\dd[4]{k}}{(2\pi)^{4}}\left(\tilde{\phi}(-k) - \frac{1}{k^{2} + m^{2}}\tilde{J}(-k)\right)\left(k^{2} + m^{2}\right)\left(\tilde{\phi}(k) - \frac{1}{k^{2} + m^{2}}\tilde{J}(k)\right) - \frac{1}{k^{2} + m^{2}}\tilde{J}(k)\tilde{J}(-k).
\end{align*}
Defining
\begin{align*}
	\tilde{\chi} = \tilde{\phi}(k) - \frac{1}{k^{2} + m^{2}}\tilde{J}(k)
\end{align*}
and switching the integration measure we find
\begin{align*}
	\gf{\tilde{J}} &= Ne^{\frac{i}{2}\inte{}{}\frac{\dd[4]{k}}{(2\pi)^{4}}\frac{1}{k^{2} + m^{2}}\tilde{J}(k)\tilde{J}(-k)}.
\end{align*}

Let us once again perform a sanity check. We have
\begin{align*}
	\tilde{G}(k_{1}, k_{2}) &= \eval{-(2\pi)^{8}\fdv{\tilde{J}(k_{1})}\frac{i}{2(2\pi)^{4}}\left(\frac{1}{k_{2}^{2} + m^{2}}\tilde{J}(-k_{2}) + \frac{1}{k_{2}^{2} + m^{2}}\tilde{J}(-k_{2})\right)e^{\frac{i}{2}\inte{}{}\frac{\dd[4]{k}}{(2\pi)^{4}}\frac{1}{k^{2} + m^{2}}\tilde{J}(k)\tilde{J}(-k)}}_{\tilde{J} = 0} \\
	&= \eval{-i(2\pi)^{4}\fdv{\tilde{J}(k_{1})}\frac{1}{k_{2}^{2} + m^{2}}\tilde{J}(-k_{2})e^{\frac{i}{2}\inte{}{}\frac{\dd[4]{k}}{(2\pi)^{4}}\frac{1}{k^{2} + m^{2}}\tilde{J}(k)\tilde{J}(-k)}}_{\tilde{J} = 0}.
\end{align*}
Once again the contribution from the second differentiation of the exponential can be ignored and we find
\begin{align*}
	\tilde{G}(k_{1}, k_{2}) &= -i(2\pi)^{4}\frac{1}{k_{2}^{2} + m^{2}}\delta^{4}(k_{1} + k_{2}).
\end{align*}
This is very similar to the momentum-space propagator, except that it only implicitly contains information about $G$ only being a function of the displacement between two points.

\paragraph{Momentum Space Generating Functional for an Interacting Scalar}
We now consider an interacting scalar field with Lagrangian
\begin{align*}
	\lag = -\frac{1}{2}\del{}{\mu}\phi\del{\mu}{}\phi - \frac{1}{2}m^{2}\phi^{2} - \frac{1}{4!}\lambda\phi^{4}.
\end{align*}
The interacting part of the action can be rewritten as
\begin{align*}
	S &= -\frac{1}{4!}\lambda\integ[4]{}{}{x}{\inte{}{}\frac{\dd[4]{k_{1}}}{(2\pi)^{4}}\frac{\dd[4]{k_{2}}}{(2\pi)^{4}}\frac{\dd[4]{k_{3}}}{(2\pi)^{4}}\frac{\dd[4]{k_{4}}}{(2\pi)^{4}}\tilde{\phi}(k_{1})\tilde{\phi}(k_{2})\tilde{\phi}(k_{3})\tilde{\phi}(k_{4})e^{ix\sum\limits_{i}k_{i}}} \\
	  &= -\frac{1}{4!}\lambda\inte{}{}\frac{\dd[4]{k_{1}}}{(2\pi)^{4}}\frac{\dd[4]{k_{2}}}{(2\pi)^{4}}\frac{\dd[4]{k_{3}}}{(2\pi)^{4}}\frac{\dd[4]{k_{4}}}{(2\pi)^{4}}\delta^{4}\left(\sum\limits_{i}k_{i}\right)\tilde{\phi}(k_{1})\tilde{\phi}(k_{2})\tilde{\phi}(k_{3})\tilde{\phi}(k_{4}).
\end{align*}
Similar to the position space case we then have
\begin{align*}
	 &Z[\tilde{J}] \\
	=& \sum\limits_{m = 0}^{\infty}\frac{1}{m!}\left(-\frac{i\lambda}{4!}\right)^{m}\pinte{}{\tilde{\phi}}\left(\inte{}{}\frac{\dd[4]{k_{1}}}{(2\pi)^{4}}\frac{\dd[4]{k_{2}}}{(2\pi)^{4}}\frac{\dd[4]{k_{3}}}{(2\pi)^{4}}\frac{\dd[4]{k_{4}}}{(2\pi)^{4}}\delta^{4}\left(\sum\limits_{i}k_{i}\right)\tilde{\phi}(k_{1})\tilde{\phi}(k_{2})\tilde{\phi}(k_{3})\tilde{\phi}(k_{4})\right)^{m}e^{i\left(S_{\text{free}} +  + \inte{}{}\frac{\dd[4]{k}}{(2\pi)^{4}}\tilde{\phi}(k)\tilde{J}(-k)\right)} \\
	=& \sum\limits_{m = 0}^{\infty}\frac{1}{m!}\left(-\frac{i\lambda}{4!}\right)^{m}\fdv{\tilde{J}(k_{1})}\fdv{\tilde{J}(k_{2})}\fdv{\tilde{J}(k_{3})}\fdv{\tilde{J}(k_{4})}Z_{\text{free}}.
\end{align*}
This can also be used to derive Feynman rules.
%TODO: Complete

\paragraph{Multiple Scalar Fields}
We now consider a system of multiple scalar fields $\phi^{(a)}$. We take them to be real scalar fields, hence
\begin{align*}
	\left(\tilde{\phi}^{(a)}\right)\cc(k) = \tilde{\phi}^{(a)}(-k).
\end{align*}
The most general action for such a field theory is
\begin{align*}
	S &= \frac{1}{2}\integ[4]{}{}{x}{\phi^{(a)}(x)D_{ab}\phi^{(b)}(x)} = \frac{1}{2}\inte{}{}\frac{\dd[4]{k}}{(2\pi)^{4}}\tilde{\phi}^{(a)}(-k)\tilde{M}_{ab}(k)\tilde{\phi}^{(b)}(k).
\end{align*}
$D$ is a matrix of differential operators and $\tilde{M}$ a matrix of functions. The latter can be symmetrized to
\begin{align*}
	S &= \frac{1}{2}\inte{}{}\frac{\dd[4]{k}}{(2\pi)^{4}}\tilde{\phi}^{(a)}(-k)M_{ab}(k)\tilde{\phi}^{(b)}(k),\ M_{ab}(k) = \frac{1}{2}(\tilde{M}_{ab}(k) + \tilde{M}_{ba}(-k)).
\end{align*}

We can now go about finding a generating functional for this kind of system. We have
\begin{align*}
	Z &= \pinte{}{\phi^{a}}e^{iS + i\integ[4]{}{}{x}{\phi^{(a)}J_{a}}}
\end{align*}
in position space and
\begin{align*}
	Z &= \pinte{}{\tilde{\phi}^{a}}e^{iS + i\inte{}{}\frac{\dd[4]{k}}{(2\pi)^{4}}\tilde{\phi}^{(a)}(k)\tilde{J}_{a}}(-k)
\end{align*}
in momentum space. The latter exponent can be handled the most easily, so we will do that one. We have
\begin{align*}
	S + \inte{}{}\frac{\dd[4]{k}}{(2\pi)^{4}}\tilde{\phi}^{(a)}(k)\tilde{J}_{a}(-k) =& \inte{}{}\frac{\dd[4]{k}}{(2\pi)^{4}}\frac{1}{2}\tilde{\phi}^{(a)}(-k)M_{ab}(k)\tilde{\phi}^{(b)}(k) + \tilde{\phi}^{(a)}(k)\tilde{J}_{a}(-k) \\
	=& \inte{}{}\frac{\dd[4]{k}}{(2\pi)^{4}}\frac{1}{2}\left(\tilde{\phi}^{(a)}(-k) + M_{ac}^{-1}(-k)\tilde{J}_{c}(-k)\right)M_{ab}(k)\left(\tilde{\phi}^{(b)}(k) + M_{bd}^{-1}(k)\tilde{J}_{d}(k)\right) \\
	 &- \inte{}{}\frac{\dd[4]{k}}{(2\pi)^{4}}\frac{1}{2}M_{ac}^{-1}(-k)\tilde{J}_{c}(k)M_{ab}(k)M_{bd}^{-1}(k)\tilde{J}_{d}(k) \\
	=& \inte{}{}\frac{\dd[4]{k}}{(2\pi)^{4}}\frac{1}{2}\tilde{\chi}^{(a)}(-k)M_{ab}(k)\tilde{\chi}^{(b)}(k) - \inte{}{}\frac{\dd[4]{k}}{(2\pi)^{4}}\frac{1}{2}M_{ac}^{-1}(-k)\tilde{J}_{c}(-k)\delta_{ad}\tilde{J}_{d}(k) \\
	=& \frac{1}{2}\inte{}{}\frac{\dd[4]{k}}{(2\pi)^{4}}\tilde{\chi}^{(a)}(-k)M_{ab}(k)\tilde{\chi}^{(b)}(k) - \frac{1}{2}\inte{}{}\frac{\dd[4]{k}}{(2\pi)^{4}}\tilde{J}_{a}(-k)M_{ac}^{-1}(-k)\tilde{J}_{c}(k).
\end{align*}
To do this we have defined the new fields
\begin{align*}
	\tilde{\chi}^{(a)}(k) = \tilde{\phi}^{(a)}(k) + M_{ac}^{-1}(-k)\tilde{J}_{c}(k)
\end{align*}
and used the fact that
\begin{align*}
	M^{-1}(k) = (M^{T}(k))^{-1} = (M^{-1}(k))^{T}.
\end{align*}
This means
\begin{align*}
	Z &= Ne^{-\frac{i}{2}\inte{}{}\frac{\dd[4]{k}}{(2\pi)^{4}}\tilde{J}_{a}(-k)M_{ac}^{-1}(-k)\tilde{J}_{c}(k)}.
\end{align*}
%TODO: Compute propagator

We now see directly that the two-point correlation function is
\begin{align*}
	G_{ab}(k_{1}, k_{2}) = \frac{1}{2}(2\pi)^{4}\left(M_{ba}^{-1}(-k_{1}) + M_{ab}^{-1}(k_{1})\right)\delta(k_{1} + k_{2}).
\end{align*}
Note that it depends very directly on the existence of $M^{-1}$, which can turn out to cause issues, as will be seen.

\paragraph{Failing to Quantize the Electromagnetic Field}
The momentum space Lagrangian is
\begin{align*}
	\lag &= -\frac{1}{4}\left(k_{\mu}A_{\nu} - k_{\nu}A_{\mu}\right)\left(k^{\mu}A^{\nu} - k^{\nu}A^{\mu}\right) \\
	     &= -\frac{1}{4}\left(A_{\nu}\left(k^{2}A^{\nu} - k_{\mu}k^{\nu}A^{\mu}\right) - A_{\nu}\left(k_{\mu}k^{\nu}A^{\mu} - k^{2}A^{\nu}\right)\right) \\
	     &= \frac{1}{2}A_{\nu}\left(k^{\mu}k^{\nu}A_{\mu} - k^{2}A^{\nu}\right) \\
	     &= \frac{1}{2}A_{\nu}\left(k^{\mu}k^{\nu} - k^{2}g^{\mu\nu}\right)A_{\mu}.
\end{align*}
This is a bit of a disaster, as $M$ always has at least one eigenvalue of zero. Namely, we have
\begin{align*}
	M^{\mu\nu}k_{\nu} = k^{\mu}k^{\nu}k_{\nu} - k^{2}g^{\mu\nu}k_{\nu} = 0.
\end{align*}
This means that $M^{-1}$ does not exist for any $k$. That scenario is a lot worse than for the Klein-Gordon equation, where $M^{-1}$ only didn't exist at the set of $k$ such that $k^{2} = -m^{2}$.