\section{Useful Mathematics}

\paragraph{All About Grassman Variables}
Grassman variables $\theta_{i}$ are defined such that multiplication of two different variables anticommutes. This means that all functions of a set of Grassman variables can be written as
\begin{align*}
	F(\theta) = \sum\limits_{i = 0}^{n}\frac{1}{i!}A^{(i), j_{1}\dots j_{i}}\theta_{j_{1}}\dots\theta_{j_{i}},
\end{align*}
with summation over the indices $j$. The coefficients $A$ are by definition completely antisymmetric.

Functions of Grassman numbers are defined as even if $A^{(k)} = 0$ for all even $k$ and similarly for odd functions. This implies that $FG = GF$ if either is even and $FG = -GF$ otherwise.

We define derivatives with respect to Grassman variables as
\begin{align*}
\pdv{\theta_{i}}(\theta_{j}) = \delta_{ij}.
\end{align*}
These derivatives anticommute. They satisfy the product rule
\begin{align*}
\pdv{\theta_{i}}(FG) = \pdv{F}{\theta_{i}}{G} + (-1)^{\abs{F}}F\pdv{G}{\theta_{i}},
\end{align*}
where $\abs{F}$ is $1$ if $F$ is even and $0$ if $F$ is odd. Note that this implies the sign convention
\begin{align*}
\pdv{\theta_{i}}\theta_{i}\theta_{j} = -\pdv{\theta_{i}}\theta_{j}\theta_{i}.
\end{align*}
For a composite function
\begin{align*}
f(G(\theta)) &= \sum\limits_{n}f_{n}G^{n}(\theta),
\end{align*}
the chain rule
\begin{align*}
\pdv{\theta_{i}} f(G(\theta)) &= \pdv{G}{\theta_{i}}\dv{f}{G}
\end{align*}
also applies.

Integrals over Grassman variables are defined according to
\begin{align*}
\inte{}{}\dd{\theta_{i}} = 0,\ \inte{}{}\dd{\theta_{i}}\theta_{j} = \delta_{ij}.
\end{align*}
This has the peculiar consequence
\begin{align*}
\inte{}{}\dd{\theta_{i}}F(\theta) = \pdv{F}{\theta_{i}}.
\end{align*}
We know that all functions of Grassman variables can be factorized as $F(\theta) = A(\theta) + \theta_{i}B(\theta)$, where neither $A$ nor $B$ depend on $\theta_{i}$. This has the consequence that shifts in any one variable does not affect the integral. Scaling of variables does not work that way, however, as we saw that differentiation and integration has the same effect. This produces the general result
\begin{align*}
\theta_{j} = A_{ij}\phi_{j}\implies \dd{\theta}_{1}\dots\dd{\theta}_{n} = \frac{1}{\det(A)}\dd{\phi}_{1}\dots\dd{\phi}_{n}.
\end{align*}

The Dirac delta function is defined in this context such that it obeys the relation
\begin{align*}
\inte{}{}\dd{\theta}_{i}F(\theta)\delta(\theta_{i}) = \eval{F}_{\theta_{i} = 0}.
\end{align*}
From what we have above we see that $\delta(\theta_{i}) = \theta_{i}$ satisfies this. This again has consequences of scaling the Dirac delta being different - more specifically
\begin{align*}
\prod\limits_{i}\delta(A_{ij}\delta_{j}) = \det(A)\prod\limits_{i}\delta(\delta_{i}).
\end{align*}

Complex Grassman variables can also be defined according to
\begin{align*}
	\theta_{k} = \frac{1}{\sqrt{2}}(\phi(k) + i\psi(k)).
\end{align*}
Using the relations above we find
\begin{align*}
	\dd{\theta}_{k}\dd{\theta}_{k}\cc = i\dd{\phi}_{k}\dd{\psi}_{k}.
\end{align*}

Let us now get to what we really need: A Gaussian integral. We therefore consider
\begin{align*}
	\inte{}{}\dd{\theta}_{k}\dd{\theta}_{k}\cc e^{\theta\adj A\theta}.
\end{align*}
This integral is easily solved by diagonalizing $A$, transforming the variables with a corresponding unitary transform and switching to integrals over the real and complex part. We then find that this integral is proportional to $\det(A)$.