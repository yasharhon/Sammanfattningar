\section{Quantum Pumps}

\paragraph{Berry Phase, Connection and Curvature}
Consider a system with a Hamiltonian and eigenstates parametrized by some set of parameters $\chi$ - that is, we have for each value of $\chi$ a set of eigenstates
\begin{align*}
	\ham(\chi)\ket{n(\chi)} = E_{n}(\chi)\ket{n(\chi)}.
\end{align*}
The adiabatic theorem tells us that if $R$ is varied such that the Hamiltonian changes sufficiently slowly, a state which is initialized to an eigenstate at $t = 0$ will evolve to a corresponding eigenstate at a later time. In the general case we have
\begin{align*}
	\ket{\psi_{n}(t)} = e^{i\gamma_{n}(t)}e^{-\frac{i}{\hbar}\inte{0}{t}\dd{\tau} E_{n}(\tau)}\ket{n(\chi(t))}.
\end{align*}
The former factor is the complex exponential of the so-called Berry phase. Inserting this into the Schrödinger equation we find
\begin{align*}
	\gamma_{n} = i\inte{0}{t}\dd{\tau}\expval{\pdv{\tau}}{n(\chi(\tau))}.
\end{align*}
Noting that
\begin{align*}
	\pdv{\tau}\ket{n(\chi(\tau))} =& \dv{R}{\tau}\cdot\grad_{R}\ket{\chi(R)},
\end{align*}
we can define the Berry connection
\begin{align*}
	A_{n} = i\expval{\grad_{\chi}}{n(\chi)}
\end{align*}
and find
\begin{align*}
	\gamma_{n} = i\inte{C}{}\dd{\chi}\cdot A_{n}.
\end{align*}
$C$ is the orbit in parameter space traversed during the time evolution.

In a slightly more sophisticated manner, the Berry connection may be taken to be a 1-form
\begin{align*}
	A_{n} = i\bra{n(R)}\del{}{\mu}{\ket{n(R)}}\df{\chi^{\mu}}.
\end{align*}
Due to Stokes' theorem, the line integral of the Berry connection about some closed path is related to the surface integral of its exterior derivative, termed the Berry curvature. Its components are
\begin{align*}
	\Omega^{(2)}_{n, \mu\nu} = \del{}{\mu}A_{n, \nu} - \del{}{\nu}A_{n, \mu},
\end{align*}
and we have
\begin{align*}
	\inte{\bound{S}}{}A_{n} = \frac{1}{2}\inte{S}{}\dd{\chi^{\mu}}\wedge\dd{\chi^{\nu}}\Omega^{(2)}_{n, \mu\nu}.
\end{align*}

\paragraph{A More Sophisticated Definition}
From this point on we switch to the more compact notation
\begin{align*}
	\del{}{\mu}\ket{n} = \ket{\del{}{\mu}n}
\end{align*}
and suppress the parameter dependence. The Berry curvature is given by
\begin{align*}
	\Omega^{(2)} = \df{A_{n}} = \frac{1}{2}(\del{}{\mu}A_{n, \nu} - \del{}{\nu}A_{n, \mu})\df{\chi^{\mu}}\wedge\df{\chi^{\nu}},
\end{align*}
and we find
\begin{align*}
	\Omega^{(2)}_{\mu\nu} = i\left(\braket{\del{}{\mu}n}{\del{}{\nu}n} + \braket{n}{\del{}{\mu}\del{}{\nu}n} - \braket{\del{}{\nu}n}{\del{}{\mu}n} - \braket{n}{\del{}{\nu}\del{}{\mu}n}\right) = i\left(\braket{\del{}{\mu}n}{\del{}{\nu}n} - \braket{\del{}{\nu}n}{\del{}{\mu}n}\right).
\end{align*}
This can be expressed without derivatives of the state. To do that we differentiate the eigenvalue expression to yield
\begin{align*}
	\del{}{\mu}\ham\ket{n} + \ham\ket{\del{}{\mu}n} = \del{}{\mu}E_{n}\ket{n} + E_{n}\ket{\del{}{\mu}n}.
\end{align*}
Using the orthogonality of the eigenstates, we have for some $n\neq m$ that
\begin{align*}
	\mel{m}{\del{}{\mu}\ham}{n} = (E_{n} - E_{m})\braket{m}{\del{}{\mu}n}.
\end{align*}
We can now solve for the inner product on the left-hand side and its complex conjugate, as well as sum over $m$, to find
\begin{align*}
	\Omega^{(2)}_{\mu\nu} = i\sum\limits_{m\neq n}\frac{\mel{n}{\del{}{\mu}\ham}{m}\mel{m}{\del{}{\nu}\ham}{n} - \text{c.c}}{(E_{n} - E_{m})^{2}}.
\end{align*}

Finally we introduce a third definition
\begin{align*}
	\Omega^{(2)} = \frac{i}{2}\oint\frac{\dd{z}}{2\pi i}\tr(G\df{\ham}G^{2}\df{\ham}).
\end{align*}
$G$ is given by $(z - \ham)^{-1}$ and the integral is a counter-clockwise contour integral around the energy of the state in consideration. There is also the appearance of the exterior derivative of the Hamiltonian.

Does this correspond to our previous notion of the Berry curvature? To investigate, let us rewrite the above operators as
\begin{align*}
	\ham = \sum\limits_{n}E_{n}\op{n},\ G = \sum\limits_{n}\frac{1}{z - E_{n}}\op{n}.
\end{align*}
Next we note that
\begin{align*}
	G\df{G^{-1}} = -\df{G}G^{-1} = -G\df{\ham},
\end{align*}
hence
\begin{align*}
	G\df{G^{-1}}G\df{G^{-1}}G = G\df{\ham}G\df{\ham}G,
\end{align*}
and by cyclic permutation we have
\begin{align*}
	\tr(G\df{G^{-1}}G\df{G^{-1}}G) =& \tr(G\df{\ham}G\df{\ham}G) \\
	=& \tr(G\del{}{\mu}\ham G\del{}{\nu}\ham G)\df{\chi^{\mu}}\df{\chi^{\nu}} \\
	=& \tr(G\del{}{\nu}\ham G^{2}\del{}{\mu}\ham)\df{\chi^{\mu}}\df{\chi^{\nu}} \\
	=& -\tr(G\df{\ham}G^{2}\df{\ham}).
\end{align*}

As a warmup to the final computation, consider a case where the spectrum is parameter-independent.  In the eigenbasis of the Hamiltonian we generally have
\begin{align*}
	\df{G^{-1}} =& \sum\limits_{n}\left(-\del{}{\mu}E_{n}\op{n} + (z - E_{n})\left(\op{\del{}{\mu}n}{n} + \op{n}{\del{}{\mu}n}\right)\right)\df{\chi^{\mu}}, \\
	\df{G}      =& \sum\limits_{n}\left(\frac{\del{}{\mu}E_{n}}{(z - E_{n})^{2}}\op{n} + \frac{1}{z - E_{n}}\left(\op{\del{}{\mu}n}{n} + \op{n}{\del{}{\mu}n}\right)\right)\df{\chi^{\mu}},
\end{align*}
and thus in this case
\begin{align*}
	G\df{G^{-1}}G\df{G^{-1}}G =& \sum\frac{(z - E_{2})(z - E_{4})}{(z - E_{1})(z - E_{3})(z - E_{5})}\op{1}\left(\op{\del{}{\mu}2}{2} + \op{2}{\del{}{\mu}2}\right)\op{3}\left(\op{\del{}{\nu}4}{4} + \op{4}{\del{}{\nu}4}\right)\op{5}e^{\mu\nu},
\end{align*}
where the natural numbers are summed over and we abbreviate the differential form basis vector. Multiplying this out we have
\begin{align*}
	G\df{G^{-1}}G\df{G^{-1}}G = \sum&\frac{(z - E_{2})(z - E_{4})}{(z - E_{1})(z - E_{3})(z - E_{5})}\ket{1}\left(\braket{1}{\del{}{\mu}2}\kdelta{}{23} + \kdelta{}{12}\braket{\del{}{\mu}2}{3}\right)\left(\braket{3}{\del{}{\nu}4}\kdelta{}{45} + \kdelta{}{34}\braket{\del{}{\nu}4}{5}\right)\bra{5}e^{\mu\nu} \\
	= \sum&\ket{1}\left(\frac{(z - E_{2})(z - E_{4})}{(z - E_{1})(z - E_{3})(z - E_{5})}\braket{1}{\del{}{\mu}2}\kdelta{}{23}\braket{3}{\del{}{\nu}4}\kdelta{}{45} \right. \\
	&+ \left. \frac{(z - E_{2})(z - E_{4})}{(z - E_{1})(z - E_{3})(z - E_{5})}\braket{1}{\del{}{\mu}2}\kdelta{}{23}\kdelta{}{34}\braket{\del{}{\nu}4}{5} \right. \\
	&+ \left. \frac{(z - E_{2})(z - E_{4})}{(z - E_{1})(z - E_{3})(z - E_{5})}\kdelta{}{12}\braket{\del{}{\mu}2}{3}\braket{3}{\del{}{\nu}4}\kdelta{}{45} \right. \\
	&+ \left. \frac{(z - E_{2})(z - E_{4})}{(z - E_{1})(z - E_{3})(z - E_{5})}\kdelta{}{12}\braket{\del{}{\mu}2}{3}\kdelta{}{34}\braket{\del{}{\nu}4}{5}\right)\bra{5}e^{\mu\nu} \\
	= \sum&\ket{1}\left(\frac{1}{z - E_{1}}\braket{1}{\del{}{\mu}2}\braket{2}{\del{}{\nu}4}\bra{4} + \frac{z - E_{2}}{(z - E_{1})(z - E_{5})}\braket{1}{\del{}{\mu}2}\braket{\del{}{\nu}2}{5}\bra{5} \right. \\
	&+ \left. \frac{1}{z - E_{3}}\braket{\del{}{\mu}1}{3}\braket{3}{\del{}{\nu}4}\bra{4} + \frac{1}{z - E_{5}}\braket{\del{}{\mu}1}{3}\braket{\del{}{\nu}3}{5}\bra{5}\right)e^{\mu\nu},
\end{align*}
and
\begin{align*}
	\tr(G\df{G^{-1}}G\df{G^{-1}}G) =& \sum\left(\frac{1}{z - E_{1}}\braket{1}{\del{}{\mu}2}\braket{2}{\del{}{\nu}4}\braket{4}{1} + \frac{z - E_{2}}{(z - E_{1})(z - E_{5})}\braket{1}{\del{}{\mu}2}\braket{\del{}{\nu}2}{5}\braket{5}{1} \right. \\
	&+ \left. \frac{1}{z - E_{3}}\braket{\del{}{\mu}1}{3}\braket{3}{\del{}{\nu}4}\braket{4}{1} + \frac{1}{z - E_{5}}\braket{\del{}{\mu}1}{3}\braket{\del{}{\nu}3}{5}\braket{5}{1}\right)e^{\mu\nu} \\
	1=& \sum\left(\frac{1}{z - E_{1}}\braket{1}{\del{}{\mu}2}\braket{2}{\del{}{\nu}4}\kdelta{}{41} + \frac{z - E_{2}}{(z - E_{1})(z - E_{5})}\braket{1}{\del{}{\mu}2}\braket{\del{}{\nu}2}{5}\kdelta{}{51} \right. \\
	&+ \left. \frac{1}{z - E_{3}}\braket{\del{}{\mu}1}{3}\braket{3}{\del{}{\nu}4}\kdelta{}{41} + \frac{1}{z - E_{5}}\braket{\del{}{\mu}1}{3}\braket{\del{}{\nu}3}{5}\kdelta{}{51}\right)e^{\mu\nu} \\
	=& \sum\left(\frac{1}{z - E_{1}}\braket{1}{\del{}{\mu}2}\braket{2}{\del{}{\nu}1} + \frac{z - E_{2}}{(z - E_{1})^{2}}\braket{1}{\del{}{\mu}2}\braket{\del{}{\nu}2}{1} \right. \\
	&+ \left. \frac{1}{z - E_{3}}\braket{\del{}{\mu}1}{3}\braket{3}{\del{}{\nu}1} + \frac{1}{z - E_{1}}\braket{\del{}{\mu}1}{3}\braket{\del{}{\nu}3}{1}\right)e^{\mu\nu}.
\end{align*}
Let us now perform the contour integral about a particular energy $E_{n}$. All of them are equal to 1 if and only if $n$ is equal to the index that appears in the denominator, hence
\begin{align*}
	\Omega^{(2)} =& -\frac{i}{2}\sum\left(\braket{n}{\del{}{\mu}1}\braket{1}{\del{}{\nu}n} + \braket{n}{\del{}{\mu}1}\braket{\del{}{\nu}1}{n} + \braket{\del{}{\mu}1}{n}\braket{n}{\del{}{\nu}1} + \braket{\del{}{\mu}n}{1}\braket{\del{}{\nu}1}{n}\right)e^{\mu\nu} \\
	=& -\frac{i}{2}\sum\left(-\braket{\del{}{\mu}n}{1}\braket{1}{\del{}{\nu}n} + \braket{\del{}{\mu}n}{1}\braket{1}{\del{}{\nu}n} + \braket{1}{\del{}{\mu}n}\braket{\del{}{\nu}n}{1} - \braket{\del{}{\mu}n}{1}\braket{1}{\del{}{\nu}n}\right)e^{\mu\nu} \\
	=& \frac{i}{2}\left(\braket{\del{}{\mu}n}{\del{}{\nu}n} - \braket{\del{}{\nu}n}{\del{}{\mu}n}\right)e^{\mu\nu},
\end{align*}
and thus
\begin{align*}
	\Omega^{(2)}_{\mu\nu} = i\left(\braket{\del{}{\mu}n}{\del{}{\nu}n} - \braket{\del{}{\nu}n}{\del{}{\mu}n}\right).
\end{align*}

Let us now go to the general case. It will contain an operator product
\begin{align*}
	 &\op{1}\left(-\del{}{\mu}E_{2}\op{2} + (z - E_{2})\left(\op{\del{}{\mu}2}{2} + \op{2}{\del{}{\mu}2}\right)\right)\op{3}\left(-\del{}{\nu}E_{4}\op{4} + (z - E_{4})\left(\op{\del{}{\nu}4}{4} + \op{4}{\del{}{\nu}4}\right)\right)\op{5} \\
	=& \ket{1}\left(-\del{}{\mu}E_{2}\kdelta{}{12}\kdelta{}{23} + (z - E_{2})\left(\braket{1}{\del{}{\mu}2}\kdelta{}{23} + \kdelta{}{12}\braket{\del{}{\mu}2}{3}\right)\right)\left(-\del{}{\nu}E_{4}\kdelta{}{34}\kdelta{}{45} + (z - E_{4})\left(\braket{3}{\del{}{\nu}4}\kdelta{}{45} + \kdelta{}{34}\braket{\del{}{\nu}4}{5}\right)\right)\bra{5},
\end{align*}
and the trace will turn this to
\begin{align*}
	\left(-\del{}{\mu}E_{2}\kdelta{}{12}\kdelta{}{23} + (z - E_{2})\left(\braket{1}{\del{}{\mu}2}\kdelta{}{23} + \kdelta{}{12}\braket{\del{}{\mu}2}{3}\right)\right)\left(-\del{}{\nu}E_{4}\kdelta{}{34}\kdelta{}{45} + (z - E_{4})\left(\braket{3}{\del{}{\nu}4}\kdelta{}{45} + \kdelta{}{34}\braket{\del{}{\nu}4}{5}\right)\right)\kdelta{}{15}.
\end{align*}
Each bracket has three terms, so let us denote their products (after adding the extra factors) as $a_{ij}$, with $i$ and $j$ denoting which terms from each of the brackets are multiplied. We know that when tracing $a_{22} + a_{23} + a_{32} + a_{33}$, we get the result. We will thus have completed the proof if we can show that the others yield no net contribution. First we have
\begin{align*}
	\sum a_{11} =& \sum\frac{1}{(z - E_{1})(z - E_{3})(z - E_{5})}\left(-\del{}{\mu}E_{2}\kdelta{}{12}\kdelta{}{23}\right)\left(-\del{}{\nu}E_{4}\kdelta{}{34}\kdelta{}{45}\right)\kdelta{}{15}e^{\mu\nu} \\
	            =& \sum\frac{\del{}{\mu}(E_{2})\del{}{\nu}(E_{4})}{(z - E_{1})(z - E_{3})(z - E_{5})}\kdelta{}{12}\kdelta{}{23}\kdelta{}{34}\kdelta{}{45}\kdelta{}{15}e^{\mu\nu} \\
	            =& \sum\frac{\del{}{\mu}(E_{1})\del{}{\nu}(E_{1})}{(z - E_{1})^{3}}e^{\mu\nu}.
\end{align*}
This is identically zero as it contains a contraction of symmetric components with the antisymmetric differential form basis. As for the others we have
\begin{align*}
	\sum a_{12} =& -\sum\frac{\del{}{\mu}E_{2}(z - E_{4})}{(z - E_{1})(z - E_{3})(z - E_{5})}\kdelta{}{12}\kdelta{}{23}\braket{3}{\del{}{\nu}4}\kdelta{}{45}\kdelta{}{15}e^{\mu\nu} = -\sum\frac{\del{}{\mu}E_{1}}{(z - E_{1})^{2}}\braket{1}{\del{}{\nu}1}e^{\mu\nu}, \\
	\sum a_{13} =& -\sum\frac{\del{}{\mu}E_{2}(z - E_{4})}{(z - E_{1})(z - E_{3})(z - E_{5})}\kdelta{}{12}\kdelta{}{23}\kdelta{}{34}\braket{\del{}{\nu}4}{5}\kdelta{}{15}e^{\mu\nu} = -\sum\frac{\del{}{\mu}E_{1}}{(z - E_{1})^{2}}\braket{\del{}{\nu}1}{1}e^{\mu\nu}, \\
	\sum a_{21} =& -\sum\frac{\del{}{\nu}E_{4}(z - E_{2})}{(z - E_{1})(z - E_{3})(z - E_{5})}\braket{1}{\del{}{\mu}2}\kdelta{}{23}\kdelta{}{34}\kdelta{}{45}\kdelta{}{15}e^{\mu\nu} = -\sum\frac{\del{}{\nu}E_{1}}{(z - E_{1})^{2}}\braket{1}{\del{}{\mu}1}e^{\mu\nu}, \\
	\sum a_{31} =& -\sum\frac{\del{}{\nu}E_{4}(z - E_{2})}{(z - E_{1})(z - E_{3})(z - E_{5})}\kdelta{}{12}\braket{\del{}{\mu}2}{3}\kdelta{}{34}\kdelta{}{45}\kdelta{}{15}e^{\mu\nu} = -\sum\frac{\del{}{\nu}E_{1}}{(z - E_{1})^{2}}\braket{\del{}{\mu}1}{1}e^{\mu\nu},
\end{align*}
and these all cancel each other exactly, completing the proof.

\paragraph{Properties of Parametrized States}
We will now derive some useful properties of derivatives of states of parametrized systems. Because orthogonality is preserved we have
\begin{align*}
	\del{}{\mu}\braket{m}{n} = \braket{\del{}{\mu}m}{n} + \braket{m}{\del{}{\mu}n} = 0.
\end{align*}
Because the identity is also preserved we have
\begin{align*}
	\sum\op{\del{}{\mu}n}{n} + \op{n}{\del{}{\mu}n} = 0.
\end{align*}

\paragraph{The Single Spin}
Consider a single spin-$\frac{1}{2}$ in an external field of length $1$. The Hamiltonian is
\begin{align*}
	\ham = h_{x}\sigma_{x} + h_{y}\sigma_{y} + h_{z}\sigma_{z},
\end{align*}
with the external field being restricted in length. With respect to the $\sigma_{z}$ eigenstates at $\theta = \phi = 0$, which are of course angle-independent, we have
\begin{align}
	\ket{\downarrow}_{\theta, \phi} = \mqty[
		-\sin(\frac{\theta}{2})e^{-i\phi} \\
		\cos(\frac{\theta}{2})
	],\ \ket{\uparrow}_{\theta, \phi} = \mqty[
		\cos(\frac{\theta}{2})e^{-i\phi} \\
		\sin(\frac{\theta}{2})
	],
	\label{eq:spin_angle_states}
\end{align}
and thus
\begin{align*}
	A_{-, \theta} = 0,\ A_{-, \phi} = \sin[2](\frac{\theta}{2}),\ A_{+, \theta} = 0,\ A_{+, \phi} = \cos[2](\frac{\theta}{2}).
\end{align*}
The Berry curvature is then
\begin{align*}
	\Omega^{(2)}_{\pm, \theta\phi} = \mp\frac{1}{2}\sin(\theta).
\end{align*}
This implies that the Berry phase induced after an adiabatic cycle is equal to half the subtended solid angle.

\paragraph{Higher Berry Curvature and the KS Invariant}
For an infinite $1d$ system, $\Omega^{(2)}$ might diverge. A convergent quantity might instead be found by splitting the Hamiltonian into a sum of local terms working at a finite range, i.e.
\begin{align*}
	\ham = \sum\limits_{p\in\mathbb{Z}}\ham_{p}.
\end{align*}
The quantity
\begin{align*}
	F^{(2)}_{pq} = \frac{i}{2}\oint\frac{\dd{z}}{2\pi i}\tr(G\df{\ham_{p}}G^{2}\df{\ham_{q}})
\end{align*}
then decays exponentially with respect to $\abs{p - q}$ if the Hamiltonian is gapped, and is thus well-defined. Next we can construct the two-form
\begin{align*}
	F^{(2)}_{q} = \sum\limits_{p\in\mathbb{Z}}F^{(2)}_{pq} = \frac{i}{2}\oint\frac{\dd{z}}{2\pi i}\tr(G\df{\ham}G^{2}\df{\ham_{q}}).
\end{align*}
Its exterior derivative is given by
\begin{align*}
	\df{F^{(2)}_{q}} = \sum\limits_{p\in\mathbb{Z}}F^{(3)}_{pq}.
\end{align*}
We have
\begin{align*}
	 &\del{}{\rho}(G\del{}{\mu}\ham_{p}G^{2}\del{}{\nu}\ham_{q})e^{\rho\mu\nu} \\
	=& \left(\del{}{\rho}G\del{}{\mu}\ham_{p}G^{2}\del{}{\nu}\ham_{q} + G\del{}{\rho}\del{}{\mu}\ham_{p}G^{2}\del{}{\nu}\ham_{q} + G\del{}{\mu}\ham_{p}\del{}{\rho}G^{2}\del{}{\nu}\ham_{q} + G\del{}{\mu}\ham_{p}G^{2}\del{}{\rho}\del{}{\nu}\ham_{q}\right)e^{\rho\mu\nu} \\
	=& \left(\del{}{\rho}G\del{}{\mu}\ham_{p}G^{2}\del{}{\nu}\ham_{q} + G\del{}{\mu}\ham_{p}(\del{}{\rho}GG + G\del{}{\rho}G)\del{}{\nu}\ham_{q}\right)e^{\rho\mu\nu} \\
	=& \left(\del{}{\rho}\ham G\del{}{\mu}\ham_{p}G^{2}\del{}{\nu}\ham_{q} + G\del{}{\mu}\ham_{p}(G\del{}{\rho}\ham G^{2} + G^{2}\del{}{\rho}\ham G)\del{}{\nu}\ham_{q}\right)e^{\rho\mu\nu} \\
	=& \left(\del{}{\rho}\ham G\del{}{\mu}\ham_{p}G^{2}\del{}{\nu}\ham_{q} + G\del{}{\mu}\ham_{p}(G\del{}{\rho}\ham G^{2} + G^{2}\del{}{\rho}\ham G)\del{}{\nu}\ham_{q}\right)e^{\rho\mu\nu} \\
	=& \left(\del{}{\rho}\ham G\del{}{\mu}\ham_{p}G^{2}\del{}{\nu}\ham_{q} + G\del{}{\mu}\ham_{p}G\del{}{\rho}\ham G^{2}\del{}{\nu}\ham_{q} + G\del{}{\mu}\ham_{p}G^{2}\del{}{\rho}\ham G\del{}{\nu}\ham_{q}\right)e^{\rho\mu\nu},
\end{align*}
and thus
\begin{align*}
	 &\tr(\del{}{\rho}(G\del{}{\mu}\ham_{p}G^{2}\del{}{\nu}\ham_{q}))e^{\rho\mu\nu} \\
	=& \tr(\del{}{\rho}\ham G\del{}{\mu}\ham_{p}G^{2}\del{}{\nu}\ham_{q} + G\del{}{\mu}\ham_{p}G\del{}{\rho}\ham G^{2}\del{}{\nu}\ham_{q} + G\del{}{\mu}\ham_{p}G^{2}\del{}{\rho}\ham G\del{}{\nu}\ham_{q})e^{\rho\mu\nu} \\
	=& \tr(\del{}{\rho}\ham G\del{}{\mu}\ham_{p}G^{2}\del{}{\nu}\ham_{q} - G\del{}{\rho}\ham G^{2}\del{}{\mu}\ham_{q}G\del{}{\nu}\ham_{p} + G\del{}{\mu}\ham_{p}G^{2}\del{}{\rho}\ham G\del{}{\nu}\ham_{q})e^{\rho\mu\nu}.
\end{align*}
Somehow we are to find
\begin{align*}
	F^{(3)}_{pq} = \frac{i}{6}\oint\frac{\dd{z}}{2\pi i}\tr(G^{2}\df{\ham}G\df{\ham_{p}}G\df{\ham_{q}} - G\df{\ham}G^{2}\df{\ham_{p}}G\df{\ham_{q}}) - (p \leftrightarrow q).
\end{align*}

To compute this we expand in eigenstates of the Hamiltonian according to
\begin{align*}
	G^{2}\df{\ham}G\df{\ham_{p}}G\df{\ham_{q}} =& \sum\frac{\op{1}\op{2}\df{\ham}\op{3}\df{\ham_{p}}\op{4}\df{\ham_{q}}}{(z - E_{1})(z - E_{2})(z - E_{3})(z - E_{4})} \\
	=& \sum\frac{\op{1}\df{\ham}\op{2}\df{\ham_{p}}\op{3}\df{\ham_{q}}}{(z - E_{1})^{2}(z - E_{2})(z - E_{3})}.
\end{align*}
Let us now compute the contour integral around the ground state. The contributions from where only one number is zero is
\begin{align*}
	\sum &-\op{0}\df{\ham}\op{2}\df{\ham_{p}}\op{3}\df{\ham_{q}}\left(\frac{1}{(E_{0} - E_{2})^{2}(E_{0} - E_{3})} + \frac{1}{(E_{0} - E_{2})(E_{0} - E_{3})^{2}}\right) \\
	&+ \frac{\op{1}\df{\ham}\op{0}\df{\ham_{p}}\op{3}\df{\ham_{q}}}{(E_{0} - E_{1})^{2}(E_{0} - E_{3})} + \frac{\op{1}\df{\ham}\op{2}\df{\ham_{p}}\op{0}\df{\ham_{q}}}{(E_{0} - E_{1})^{2}(E_{0} - E_{2})}.
\end{align*}
Introducing
\begin{align*}
	G_{0} = \sum\limits_{n\neq 0}\frac{1}{E_{0} - E_{n}}\op{n},
\end{align*}
this can be written as
\begin{align*}
	&-\op{0}\left(\df{\ham}G_{0}^{2}\df{\ham_{p}}G_{0}\df{\ham_{q}} + \df{\ham}G_{0}\df{\ham_{p}}G_{0}^{2}\df{\ham_{q}}\right) + G_{0}^{2}\df{\ham}\op{0}\df{\ham_{p}}G_{0}\df{\ham_{q}} + G_{0}^{2}\df{\ham}G_{0}\df{\ham_{p}}\op{0}\df{\ham_{q}}.
\end{align*}
Similarly, when two of the numbers are zero we get the contribution
\begin{align*}
	 &\sum\frac{1}{2}\left(2\frac{\op{0}\df{\ham}\op{0}\df{\ham_{p}}\op{3}\df{\ham_{q}}}{(E_{0} - E_{3})^{3}} + 2\frac{\op{0}\df{\ham}\op{2}\df{\ham_{p}}\op{0}\df{\ham_{q}}}{(E_{0} - E_{2})^{3}}\right) - 2\frac{\op{1}\df{\ham}\op{0}\df{\ham_{p}}\op{0}\df{\ham_{q}}}{(E_{0} - E_{1})^{3}} \\
	=& \op{0}\df{\ham}\op{0}\df{\ham_{p}}G_{0}^{3}\df{\ham_{q}} + \op{0}\df{\ham}G_{0}^{3}\df{\ham_{p}}\op{0}\df{\ham_{q}} - 2G_{0}^{3}\df{\ham}\op{0}\df{\ham_{p}}\op{0}\df{\ham_{q}}.
\end{align*}
Finally, if none or all of them are zero there is no contribution. Next, we have
\begin{align*}
	G\df{\ham}G^{2}\df{\ham_{p}}G\df{\ham_{q}} =& \sum\frac{\op{1}\df{\ham}\op{2}\df{\ham_{p}}\op{3}\df{\ham_{q}}}{(z - E_{1})(z - E_{2})^{2}(z - E_{3})}.
\end{align*}
The contributions after computing the contour integral are
\begin{align*}
	\sum &-\op{1}\df{\ham}\op{0}\df{\ham_{p}}\op{3}\df{\ham_{q}}\left(\frac{1}{(E_{0} - E_{1})^{2}(E_{0} - E_{3})} + \frac{1}{(E_{0} - E_{1})(E_{0} - E_{3})^{2}}\right) \\
	&+ \frac{\op{0}\df{\ham}\op{2}\df{\ham_{p}}\op{3}\df{\ham_{q}}}{(E_{0} - E_{2})^{2}(E_{0} - E_{3})} + \frac{\op{1}\df{\ham}\op{2}\df{\ham_{p}}\op{0}\df{\ham_{q}}}{(E_{0} - E_{1})(E_{0} - E_{2})^{2}} \\
	=& -G_{0}^{2}\df{\ham}\op{0}\df{\ham_{p}}G_{0}\df{\ham_{q}} - G_{0}\df{\ham}\op{0}\df{\ham_{p}}G_{0}^{2}\df{\ham_{q}} + \op{0}\df{\ham}G_{0}^{2}\df{\ham_{p}}G_{0}\df{\ham_{q}} + G_{0}\df{\ham}G_{0}^{2}\df{\ham_{p}}\op{0}\df{\ham_{q}}
\end{align*}
when one number is zero and
\begin{align*}
	 &\sum\frac{1}{2}\left(2\frac{\op{0}\df{\ham}\op{0}\df{\ham_{p}}\op{3}\df{\ham_{q}}}{(E_{0} - E_{3})^{3}} + 2\frac{\op{1}\df{\ham}\op{0}\df{\ham_{p}}\op{0}\df{\ham_{q}}}{(E_{0} - E_{1})^{3}}\right) - 2\frac{\op{0}\df{\ham}\op{2}\df{\ham_{p}}\op{0}\df{\ham_{q}}}{(E_{0} - E_{2})^{2}} \\
	=& \op{0}\df{\ham}\op{0}\df{\ham_{p}}G_{0}^{3}\df{\ham_{q}} + G_{0}^{3}\df{\ham}\op{0}\df{\ham_{p}}\op{0}\df{\ham_{q}} - 2\op{0}\df{\ham}G_{0}^{3}\df{\ham_{p}}\op{0}\df{\ham_{q}}
\end{align*}
when two are. The final result is thus
\begin{align*}
	F^{(3)}_{pq} = \frac{i}{6}&\left(-\expval{\df{\ham}G_{0}^{2}\df{\ham_{p}}G_{0}\df{\ham_{q}}} - \expval{\df{\ham}G_{0}\df{\ham_{p}}G_{0}^{2}\df{\ham_{q}}} + \expval{\df{\ham_{p}}G_{0}\df{\ham_{q}}G_{0}^{2}\df{\ham}} \right. \\
	&+ \left. \expval{\df{\ham_{q}}G_{0}^{2}\df{\ham}G_{0}\df{\ham_{p}}} + \expval{\df{\ham}}\expval{\df{\ham_{p}}G_{0}^{3}\df{\ham_{q}}} + \expval{\df{\ham}G_{0}^{3}\df{\ham_{p}}}\expval{\df{\ham_{q}}} \right. \\
	&- \left. 2\expval{\df{\ham_{q}}G_{0}^{3}\df{\ham}}\expval{\df{\ham_{p}}} + \expval{\df{\ham_{p}}G_{0}\df{\ham_{q}}G_{0}^{2}\df{\ham}} + \expval{\df{\ham_{p}}G_{0}^{2}\df{\ham_{q}}G_{0}\df{\ham}} \right. \\
	&- \left. \expval{\df{\ham}G_{0}^{2}\df{\ham_{p}}G_{0}\df{\ham_{q}}} - \expval{\df{\ham_{q}}G_{0}\df{\ham}G_{0}^{2}\df{\ham_{p}}} - \expval{\df{\ham}}\expval{\df{\ham_{p}}G_{0}^{3}\df{\ham_{q}}} \right. \\
	&- \left. \expval{\df{\ham_{p}}}\expval{\df{\ham_{q}}G_{0}^{3}\df{\ham}} + 2\expval{\df{\ham}G_{0}^{3}\df{\ham_{p}}}\expval{\df{\ham_{q}}}\right) - (p \leftrightarrow q) \\
	= \frac{i}{6}&\left(-2\expval{\df{\ham}G_{0}^{2}\df{\ham_{p}}G_{0}\df{\ham_{q}}} - \expval{\df{\ham}G_{0}\df{\ham_{p}}G_{0}^{2}\df{\ham_{q}}} + 2\expval{\df{\ham_{p}}G_{0}\df{\ham_{q}}G_{0}^{2}\df{\ham}} \right. \\
	&+ \left. \expval{\df{\ham_{p}}G_{0}^{2}\df{\ham_{q}}G_{0}\df{\ham}} + \expval{\df{\ham_{q}}G_{0}^{2}\df{\ham}G_{0}\df{\ham_{p}}} - \expval{\df{\ham_{q}}G_{0}\df{\ham}G_{0}^{2}\df{\ham_{p}}} \right. \\
	&+ \left. 3\expval{\df{\ham}G_{0}^{3}\df{\ham_{p}}}\expval{\df{\ham_{q}}} - 3\expval{\df{\ham_{q}}G_{0}^{3}\df{\ham}}\expval{\df{\ham_{p}}}\right) - (p \leftrightarrow q),
\end{align*}
where all the expectation values are computed in the ground state.

This quantity is somewhat difficult to manage, but one can reduce it somewhat. First, states excited outside of the support of $\ham_{p}$ and $\ham_{q}$ do not contribute, as they are orthogonal to the ground state and can pass through $\ham_{p}$ and $\ham_{q}$, as well as their exterior derivatives. By a similar token, $F^{(3)}_{pq}$ is non-zero only if $\ham_{p}$ and $\ham_{q}$ have overlapping support. This also implies that the only terms in the Hamiltonian that contribute are the ones with support overlapping with both $\ham_{p}$ and $\ham_{q}$.

Using these quantities we can construct a 3-form Berry curvature
\begin{align*}
	\Omega^{(3)}(f) = \frac{1}{2}\sum\limits_{p, q\in\mathbb{Z}}F^{(3)}_{pq}(f(q) - f(p)).
\end{align*}
$f$ is some sigmoid function, its particular shape turning out to be unimportant. A simple choice is $f(p) = \Theta(p - a)$ for some $a\in\mathbb{Z} + \frac{1}{2}$. For this particular choice we have
\begin{align*}
	\Omega^{(3)}(f) =& \frac{1}{2}\sum\limits_{p, q\in\mathbb{Z}}F^{(3)}_{pq}(\Theta(q - a) - \Theta(p - a)) \\
	                =& \frac{1}{2}\sum\limits_{p\in\mathbb{Z},\ q > a}F^{(3)}_{pq}(1 - \Theta(p - a)) - \frac{1}{2}\sum\limits_{p\in\mathbb{Z},\ q < a}F^{(3)}_{pq}\Theta(p - a) \\
	                =& \frac{1}{2}\sum\limits_{p < a,\ q > a}F^{(3)}_{pq} - \frac{1}{2}\sum\limits_{p > a,\ q < a}F^{(3)}_{pq} \\
	                =& \sum\limits_{p < a,\ q > a}F^{(3)}_{pq},
\end{align*}
using the antisymmetry of $F^{(3)}_{pq}$.

Finally we can define the KS invariant
\begin{align*}
	Q_{\text{KS}} = \inte{}{}\Omega^{(3)}(f),
\end{align*}
which is performed over the full parameter space of the Hamiltonian. This is a topological invariant.

\paragraph{The Dimerized Spin Chain}
Consider an infinite spin chain with Hamiltonian
\begin{align*}
	\ham_{1d} = \sum\limits_{p\in\mathbb{Z}}\ham^{1}_{p}(w) + \sum\limits_{p\in 2\mathbb{Z} + 1}\ham^{2, +}_{p, p + 1}(w) + \sum\limits_{p\in 2\mathbb{Z}}\ham^{2, -}_{p, p + 1}(w).
\end{align*}
The parameter takes values on $S^{3}$. There are three kinds of terms here. The first is
\begin{align*}
	\ham^{1}_{p}(w) = (-1)^{p}(w_{1}\sigma_{p}^{1} + w_{2}\sigma_{p}^{2}+ w_{3}\sigma_{p}^{3}),
\end{align*}
which is some fluctuating on-site term. The two others are
\begin{align*}
	\ham^{2, \pm}_{p, p + 1}(w) = g^{\pm}(w)\sum\limits_{\mu = 1, 2, 3}\sigma_{p}^{\mu}\sigma_{p + 1}^{\mu},
\end{align*}
with two functions
\begin{align*}
	g^{+}(w) = \begin{cases}
		w_{4},\ 0\leq w_{4} \leq 1, \\
		0,\ \text{otherwise},
	\end{cases}
	g^{-}(w) = \begin{cases}
		-w_{4},\ -1\leq w_{4} \leq 0, \\
		0,\ \text{otherwise}.
	\end{cases}
\end{align*}
This type of interaction defines five distinct regimes:
\begin{itemize}
	\item $w_{4} = 1$, where there is only odd-even bonding.
	\item $0 < w_{4} < 1$, where there is odd-even bonding and on-site interactions.
	\item $w_{4} = 0$, where there is only on-site interaction.
	\item $-1 < w_{4} < 0$, where there is even-odd bonding and on-site interactions.
	\item $w_{4} = 1$, where there is only even-odd bonding.
\end{itemize}

To compute the 3-form Berry curvature and KS invariant, we rewrite the Hamiltonian as a sum of local terms. These are
\begin{align*}
	\ham_{p}(w) = \ham^{1}_{p}(w) + x\ham^{2, \pm}_{p, p + 1}(w) + (1 - x)\ham^{2, \mp}_{p - 1, p}(w).
\end{align*}
The top sign is for odd $p$. The new parameter $x$ is an extra control parameter, taken to be fixed. Its introduction is an explicit representation of the ambiguity of the choice of local terms.

For the sigmoid function $f$ we choose a Heaviside function, this time leaving us with two variants - $f$ with $a\in 2\mathbb{Z} - \frac{1}{2}$ and $f\p$ with $a\in 2\mathbb{Z} + \frac{1}{2}$. To see how they differ, consider the regime $w_{4} > 0$. In this case $f$ splits the dimer in two and $f\p$ switches on between two dimers.

Because the local terms in the Hamiltonian only interact at range $1$ in either direction, the eigenstates of the system for any parameter choice are product states over each dimer. This means
\begin{align*}
	\Omega^{(3)}(f) = \Omega^{(3)}(f\p) = F^{(3)}_{a - \frac{1}{2}, a + \frac{1}{2}},
\end{align*}
with the particular choice of $a$ distinguishing the two cases. $\Omega^{(3)}(f)$ is only non-trivial if the sites $a \pm \frac{1}{2}$ belong to the same dimer, hence $\Omega^{(3)}(f) = 0$ unless $w_{4} > 0$ and $\Omega^{(3)}(f\p) = 0$ unless $w_{4} < 0$.

We will need to diagonalize the dimer, so we first transform the basis from an angle-independent one into one parallel with the Zeeman field using a unitary operator $U$. This transforms states according to $\ket{\psi}\to \ket{\psi}_{\theta, \phi} = U\ket{\psi}$ and any operator according to $A\to a = UAU\adj$, the explicit angle dependence having been removed from the left-hand side of both equalities. This angle dependence is instead baked into the basis. The small-letter notation will be useful for clarification when a matrix representation is invoked. Having applied this transformation we choose simultaneous eigenstates of $S_{z, p}\p + S_{z, p + 1}\p$ and $(S_{p}\p + S_{p + 1}\p)^{2}$, which are also eigenstates of $(S_{p}\p)^{2}$ and $(S_{p + 1}\p)^{2}$. The vector appearing in the Zeeman term has length $\sqrt{1 - w_{4}^{2}}$, meaning
\begin{align*}
	h_{p} = -2\sqrt{1 - w_{4}^{2}}S_{z, p}\p + 4xw_{4}S_{p}\p\cdot S_{p + 1}\p,\ h_{p + 1} = 2\sqrt{1 - w_{4}^{2}}S_{z, p + 1} + 4(1 - x)w_{4}S_{p}\p\cdot S_{p + 1}\p
\end{align*}
for $p = a - \frac{1}{2}$. Furthermore, as
\begin{align*}
	S_{p}\p\cdot S_{p + 1}\p = \frac{1}{2}((S_{p}\p + S_{p + 1}\p)^{2} - (S_{p}\p)^{2} - (S_{p + 1}\p)^{2}),
\end{align*}
we have
\begin{align*}
	h_{p} &= -\sqrt{1 - w_{4}^{2}}\mqty[
		1 & 0 & 0  & 0 \\
		0 & 0 & 0  & 1 \\
		0 & 0 & -1 & 0 \\
		0 & 1 & 0  & 0 
	] + 2xw_{4}\mqty[
		\frac{1}{2} & 0           & 0 & 0 \\
		0           & \frac{1}{2} & 0 & 0 \\
		0           & 0           & \frac{1}{2} & 0 \\
		0           & 0           & 0           & -\frac{3}{2} 
	], \\
	h_{p + 1} &= \sqrt{1 - w_{4}^{2}}\mqty[
		1 & 0  & 0  & 0 \\
		0 & 0  & 0  & -1 \\
		0 & 0  & -1 & 0 \\
		0 & -1 & 0  & 0 
	] + (1 - x)w_{4}\mqty[
		1 & 0 & 0 & 0 \\
		0 & 1 & 0 & 0 \\
		0 & 0 & 1 & 0 \\
		0 & 0 & 0 & -3
	]
\end{align*}
in the eigenbasis of total spin, and the total dimer Hamiltonian is
\begin{align*}
	h = w_{4}\mqty[
		1 & 0 & 0 & 0 \\
		0 & 1 & 0 & 0 \\
		0 & 0 & 1 & 0 \\
		0 & 0 & 0 & -3
	] - 2\sqrt{1 - w_{4}^{2}}\mqty[
		0 & 0 & 0 & 0 \\
		0 & 0 & 0 & 1 \\
		0 & 0 & 0 & 0 \\
		0 & 1 & 0 & 0 
	].
\end{align*}
The eigenstates $\ket{1, 1}$ and $\ket{1, -1}$ are still eigenstates of the total Hamiltonian, with energy $w_{4}$. In addition there are two eigenstates found by diagonalizing
\begin{align*}
	\mqty[
		w_{4}                  & -2\sqrt{1 - w_{4}^{2}} \\
		-2\sqrt{1 - w_{4}^{2}} & -3w_{4}
	].
\end{align*}
The energies are $\pm 2 - w_{4}$, with eigenstates
\begin{align*}
	\frac{1}{\sqrt{2}}\mqty[
		-\sqrt{1 + w_{4}} & \sqrt{1 - w_{4}} \\
		\sqrt{1 - w_{4}}  & \sqrt{1 + w_{4}}
	].
\end{align*}

We proceed by introducing hyperspherical coordinates
\begin{align*}
	w_{1} = \sin(\alpha)\cos(\theta),\ w_{2} = \sin(\alpha)\sin(\theta)\cos(\phi),\ w_{3} = \sin(\alpha)\sin(\theta)\sin(\phi),\ w_{4} = \cos(\alpha),
\end{align*}
for which we have
\begin{align*}
	h_{p} &= -\sin(\alpha)\mqty[
		1 & 0 & 0  & 0 \\
		0 & 0 & 0  & 1 \\
		0 & 0 & -1 & 0 \\
		0 & 1 & 0  & 0 
	] + x\cos(\alpha)\mqty[
		1 & 0 & 0 & 0 \\
		0 & 1 & 0 & 0 \\
		0 & 0 & 1 & 0 \\
		0 & 0 & 0 & -3
	], \\
	h_{p + 1} &= \sin(\alpha)\mqty[
		1 & 0  & 0  & 0 \\
		0 & 0  & 0  & -1 \\
		0 & 0  & -1 & 0 \\
		0 & -1 & 0  & 0 
	] + (1 - x)\cos(\alpha)\mqty[
		1 & 0 & 0 & 0 \\
		0 & 1 & 0 & 0 \\
		0 & 0 & 1 & 0 \\
		0 & 0 & 0 & -3
	], \\
	h_{p} + h_{p + 1} =& h = \cos(\alpha)\mqty[
		1 & 0 & 0 & 0 \\
		0 & 1 & 0 & 0 \\
		0 & 0 & 1 & 0 \\
		0 & 0 & 0 & -3
	] - 2\sin(\alpha)\mqty[
		0 & 0 & 0 & 0 \\
		0 & 0 & 0 & 1 \\
		0 & 0 & 0 & 0 \\
		0 & 1 & 0 & 0 
	].
\end{align*}
The eigenstates of individual spin are given in equation \ref{eq:spin_angle_states}, and we then have
\begin{align*}
	\ket{1, 1}_{\theta, \phi} &= \mqty[
		\cos[2](\frac{\theta}{2})e^{-2i\phi} \\
		\frac{1}{\sqrt{2}}\sin(\theta)e^{-i\phi} \\
		\sin[2](\frac{\theta}{2}) \\
		0
	],\ \ket{1, 0}_{\theta, \phi} = \mqty[
		-\frac{1}{\sqrt{2}}\sin(\theta)e^{-2i\phi} \\
		\cos(\theta)e^{-i\phi} \\
		\frac{1}{\sqrt{2}}\sin(\theta) \\
		0
	],\ \ket{1, -1}_{\theta, \phi} &= \mqty[
		\sin[2](\frac{\theta}{2})e^{-2i\phi} \\
		-\frac{1}{\sqrt{2}}\sin(\theta)e^{-i\phi} \\
		\cos[2](\frac{\theta}{2}) \\
		0
	],\ \ket{0, 0}_{\theta, \phi} = \mqty[
		0 \\
		0 \\
		0 \\
		e^{-i\phi}
	]
\end{align*}
with respect to the total spin basis for $\theta = \phi = 0$. We can then explicitly write
\begin{align*}
	U = \mqty[
		\cos[2](\frac{\theta}{2})e^{-2i\phi} & -\frac{1}{\sqrt{2}}\sin(\theta)e^{-2i\phi} & \sin[2](\frac{\theta}{2})e^{-2i\phi} & 0 \\
		\frac{1}{\sqrt{2}}\sin(\theta)e^{-i\phi} & \cos(\theta)e^{-i\phi} & -\frac{1}{\sqrt{2}}\sin(\theta)e^{-i\phi} & 0 \\
		\sin[2](\frac{\theta}{2}) & \frac{1}{\sqrt{2}}\sin(\theta) & \cos[2](\frac{\theta}{2}) & 0 \\
		0 & 0 & 0 & e^{-i\phi}
	].
\end{align*}

Let us also derive an expression for $g_{0}$. The eigenstates of the Hamiltonian in the angle-dependent basis are
\begin{align*}
	v_{-2 - \cos(\alpha)} = \frac{1}{\sqrt{2}}\mqty[
		0 \\
		\sqrt{1 - \cos(\alpha)} \\
		0 \\
		\sqrt{1 + \cos(\alpha)}
	],\ v_{\cos(\alpha), 1} = \mqty[
		1 \\
		0 \\
		0 \\
		0
	],\ v_{\cos(\alpha), 2} = \mqty[
		0 \\
		0 \\
		1 \\
		0
	],\ v_{2 - \cos(\alpha)} = \frac{1}{\sqrt{2}}\mqty[
		0 \\
		-\sqrt{1 + \cos(\alpha)} \\
		0 \\
		\sqrt{1 - \cos(\alpha)}
	].
\end{align*}
Forming these into a matrix $V$ and computing $VDV^{-1}$ for
\begin{align*}
	D =& \mqty[
		-\frac{1}{2(1 + \cos(\alpha))} & 0 & 0 & 0 \\
		0 & 0 & 0 & 0 \\
		0 & 0 & -\frac{1}{2(1 + \cos(\alpha))} & 0 \\
		0 & 0 & 0 & -\frac{1}{4}
	]
\end{align*}
nets us
\begin{align*}
	g_{0} =  \mqty[
		-\frac{1}{2(1 + \cos(\alpha))} & 0 & 0 & 0 \\
		0 & -\frac{1}{8}(1 + \cos(\alpha)) & 0 & \frac{1}{8}\sin(\alpha) \\
		0 & 0 & -\frac{1}{2(1 + \cos(\alpha))} & 0 \\
		0 & \frac{1}{8}\sin(\alpha) & 0 & -\frac{1}{8}(1 - \cos(\alpha))
	].
\end{align*}

The three angles are now neatly separated, as $\phi$ and $\theta$ only enter in $U$ and $\alpha$ only enters in the combination of eigenstates after $U$ has been applied. Using the explicit formula we then have
\begin{align*}
	F^{(3)}_{p, p + 1} =& \frac{i}{6}\left(-2\expval{\df{\ham}G_{0}^{2}\df{\ham_{p}}G_{0}\df{\ham_{p + 1}}} - \expval{\df{\ham}G_{0}\df{\ham_{p}}G_{0}^{2}\df{\ham_{p + 1}}} + 2\expval{\df{\ham_{p}}G_{0}\df{\ham_{p + 1}}G_{0}^{2}\df{\ham}} \right. \\
	&+ \left. \expval{\df{\ham_{p}}G_{0}^{2}\df{\ham_{p + 1}}G_{0}\df{\ham}} + \expval{\df{\ham_{p + 1}}G_{0}^{2}\df{\ham}G_{0}\df{\ham_{p}}} - \expval{\df{\ham_{p + 1}}G_{0}\df{\ham}G_{0}^{2}\df{\ham_{p}}} \right. \\
	&+ \left. 3\expval{\df{\ham}G_{0}^{3}\df{\ham_{p}}}\expval{\df{\ham_{p + 1}}} - 3\expval{\df{\ham_{p + 1}}G_{0}^{3}\df{\ham}}\expval{\df{\ham_{p}}}\right) - (p \leftrightarrow p + 1).
\end{align*}
In order to get non-trivial results, we must compute these expectation values in an angle-independent basis. To that end, we note that all the operators involved only depend on $\alpha$ in the angle-dependent basis. We then write $A = U\adj aU$ and consider its expectation value in some angle-dependent state $\ket{\psi} = U\adj\ket{\psi}_{\theta, \phi}$. We then have
\begin{align*}
	\expval{A}{\psi} = \expval{UU\adj aUU\adj}{\psi}_{\theta, \phi} = \expval{a}{\psi}_{\theta, \phi}.
\end{align*}
This means, for instance, that
\begin{align*}
	\expval{\df{\ham}G_{0}^{2}\df{\ham_{p}}G_{0}\df{\ham_{p + 1}}} =& \expval{U(\df{\ham}G_{0}^{2}\df{\ham_{p}}G_{0}\df{\ham_{p + 1}})_{\theta, \phi}U\adj}_{\theta, \phi} \\
	=& \expval{U\df{(U\adj hU)}U\adj g_{0}^{2}U\df{(U\adj h_{p}U)}U\adj g_{0}U\df{(U\adj h_{p + 1}U)}U\adj}_{\theta, \phi}.
\end{align*}

This should generally be a polynomial of order up to 2 in $x$. To simplify the calculation we could start by eliminating terms. For example, the order-2 terms come from the contributions where $h_{p}$ and $h_{p + 1}$ look identical. By antisymmetry with respect to the lattice sites these terms vanish, and the remaining contributions are of order 1 at most.

By antisymmetry we may compute any component of this form, so we choose $F^{(3)}_{p, p + 1, \alpha\theta\phi}$, for which the two latter terms vanish.

This is going to take some time, so let's start. We have
\begin{align*}
	 &\expval{\df{\ham}G_{0}^{2}\df{\ham_{p}}G_{0}\df{\ham_{p + 1}}} \\
	=& -\frac{i}{64}\cos(\theta)\left(\cos(\alpha)-1\right) \left(2 \sin(\alpha)\cos[2](\alpha) - 2\cos[3](\alpha) - 5 \sin(\alpha)\cos(\alpha) + 3\cos(\alpha)^{2} + \sin(\alpha) + 4\cos(\alpha) + 3\right)
\end{align*}

Using the explicit formula above we somehow find
\begin{align*}
	\Omega^{(3)}_{\alpha\theta\phi} = \frac{1}{2}(2 + \cos(\alpha))\tan[2](\frac{\alpha}{2})\sin(\theta)
\end{align*}
for $0 < \alpha < \frac{\pi}{2}$. We then have
\begin{align*}
	Q_{\text{KS}} =& \inte{S^{3}}{}\Omega^{(3)}_{\alpha\theta\phi} = \frac{1}{2}\inte{0}{\frac{\pi}{2}}\inte{0}{\pi}\inte{0}{2\pi}\dd{\alpha}\dd{\theta}\dd{\phi}(2 + \cos(\alpha))\tan[2](\frac{\alpha}{2})\sin(\theta) \\
	              =& 2\pi\inte{0}{\frac{\pi}{2}}\dd{\alpha}(2 + \cos(\alpha))\tan[2](\frac{\alpha}{2}) = 2\pi.
\end{align*}

What if we were to use a different sigmoid? Swapping to $f\p$, which is zero for $w_{4} > 0$. Elsewhere we find
\begin{align*}
	\Omega^{(3)}_{\alpha\theta\phi} = \frac{1}{2}(2 - \cos(\alpha))\cot[2](\frac{\alpha}{2})\sin(\theta),
\end{align*}
and thus
\begin{align*}
	Q_{\text{KS}} =& 2\pi\inte{\frac{\pi}{2}}{\pi}\dd{\alpha}(2 - \cos(\alpha))\cot[2](\frac{\alpha}{2}) \\
	              =& -2\pi\inte{\frac{\pi}{2}}{0}\dd{\beta}(2 - \cos(\pi - \beta))\cot[2](\frac{\pi - \beta}{2}) \\
	              =& 2\pi\inte{0}{\frac{\pi}{2}}\dd{\beta}(2 + \cos(\beta))\tan[2](\frac{\beta}{2}) \\
	              =& 2\pi,
\end{align*}
and indeed the choice of sigmoid was irrelevant.

\paragraph{Effective Actions}
To define the effective action, we first introduce
\begin{align*}
	E[J] = i\ln(Z),
\end{align*}
where $Z$ is the generating functional of some quantum filed theory. $E$ is essentially a measure of the vacuum energy as a function of the source $J$. There is also a strong analogy to statistical mechanics at play, with $Z$ playing the role of the partition function and $E$ the role of the Helmholtz free energy. Its functional derivatives are given by
\begin{align*}
	\fdv{E}{J(x)} = \frac{i}{Z}\fdv{Z}{J(x)} = -\frac{\pinte{}{\phi}\phi(x)e^{i\left(S + \inte{}{}\dd[d]{y}J(y)\phi(y)\right)}}{\pinte{}{\phi}e^{i\left(S + \inte{}{}\dd[d]{y}J(y)\phi(y)\right)}}.
\end{align*}
In analogy with statistical mechanics, this can be considered a classical vacuum expectation value in the presence of a source, hence we term it $\phi_{\text{cl}}(x)$. Its evaluation at $J = 0$ nets us the familiar correlation function. In analogy with statistical mechanics we can now perform a Legendre transform according to
\begin{align*}
	\Gamma[\phi_{\text{cl}}] = -E[J] - \inte{}{}\dd[d]{x}\phi_{\text{cl}}(x)J(x).
\end{align*}
The quantity $\Gamma$ is the effective action. Note that $E$ and $\Gamma$ coincide for $J = 0$.

The whole point of introducing these quantities in this manner is to connect it to something familiar: Feynman diagrams. As we have noted, the first functional derivative produces $-\expval{\phi}$. The second can be shown to be given by
\begin{align*}
	\fdv{J(x)}\fdv{J(y)}E = -i\left(\expval{\phi(x)\phi(y)} - \expval{\phi(x)}\expval{\phi(y)}\right).
\end{align*}
Thus the unconnected Feynman diagrams are explicitly removed in this case, where they contribute to the generating functional. The general result is
\begin{align*}
	\left(\prod\limits_{i = 1}^{n}\fdv{J(x_{i})}\right)E = i^{n + 1}\expval{\prod\limits_{i = 1}^{n}\phi(x_{i})}_{\text{conn}},
\end{align*}
which will be useful for computing terms in the effective action.

One can also introduce a partial effective action, where only some fields are integrated out. This is the case that is important to us. It also turns out that the systematics of computing traces in this case are equivalent to connected Feynman diagrams. We will see more of this when studying concrete examples.

\paragraph{Theories With Topological Response}
We will now consider some field theories with background fields. The effective actions of these theories contain topological terms, which are metric-independent. 

The first is the $1 + 1$-dimensional theory
\begin{align*}
	\lag = -i\overline{\psi}\left(\fsl{\del{}{}} + M_{1} + iM_{2}\gamma^{01}\right)\psi.
\end{align*}
The two masses are taken to be slowly varying background fields, which are also taken to be on a circle. Imposing this we note that
\begin{align*}
	(\gamma^{01})^{2} = 1 \implies e^{i\phi\gamma^{01}} = \cos(\phi) + i\sin(\phi)\gamma^{01},
\end{align*}
and thus
\begin{align*}
	M_{1} + iM_{2}\gamma^{01} = M\left(\cos(\alpha) + i\sin(\alpha)\gamma^{01}\right) = Me^{i\alpha\gamma^{01}},
\end{align*}
with $M$ and $\alpha$ being the magnitude and argument of the complex number $M_{1} + iM_{2}$ (note that hermiticity implies that both parameters be real). Finally, coupling the fermion field to a gauge field, the full Lagrangian for this theory is
\begin{align*}
	\lag = -i\overline{\psi}\left(\fsl{D} + Me^{i\alpha\gamma^{01}}\right)\psi - \frac{1}{4}F_{\mu\nu}F^{\mu\nu}.
\end{align*}
To compute the effective action, we integrate out the fermion and perform a perturbation expansion treating $\fsl{D} + Me^{i\alpha\gamma^{01}}$ as a perturbed version of $\fsl{\del{}{}} + M$. The effective action for the gauge field is
\begin{align*}
	\Gamma =& -i\ln(\det(-i\left(\fsl{D} + Me^{i\alpha\gamma^{01}}\right))) + \inte{}{}\dd[2]{x}-\frac{1}{4}F_{\mu\nu}F^{\mu\nu} \\
	       =& -i\tr(\ln(-i\left(\fsl{D} + Me^{i\alpha\gamma^{01}}\right))) + \inte{}{}\dd[2]{x}-\frac{1}{4}F_{\mu\nu}F^{\mu\nu}.
\end{align*}
The trace can be computed by summing over some basis with respect to both the field and Dirac structures. In particular, the new term is
\begin{align*}
	-i\tr(\ln(-i\left(\fsl{D} + Me^{i\alpha\gamma^{01}}\right))) \approx& 	-i\tr(\ln(-i\left(\fsl{\del{}{}} + M\right)\left(1 + \frac{\fsl{A} + iM\alpha\gamma^{01}}{\fsl{\del{}{}} + M}\right))) \\
	=& C_{0} - i\tr(\ln(1 + \frac{-ie\fsl{A} + iM\alpha\gamma^{01}}{\fsl{\del{}{}} + M})) \\
	\approx& C_{0} - i\tr(\frac{-ie\fsl{A} + iM\alpha\gamma^{01}}{\fsl{\del{}{}} + M} - \frac{1}{2}\frac{-ie\fsl{A} + iM\alpha\gamma^{01}}{\fsl{\del{}{}} + M}\frac{-ie\fsl{A} + iM\alpha\gamma^{01}}{\fsl{\del{}{}} + M}).
\end{align*}
Let us first compute the inverse of the denominator. We have
\begin{align*}
	(\fsl{\del{}{}} + M)(\fsl{\del{}{}} - M) = \fsl{\del{}{}}^{2} - M^{2} = \del{2}{} - M^{2} \implies \frac{1}{\fsl{\del{}{}} + M} = \frac{\fsl{\del{}{}} - M}{\del{2}{} - M^{2}}.
\end{align*}
Computing the trace in momentum space, applying the correspondence principle $p = -i\del{}{}$ and using the systematics of Feynman diagrams we find
\begin{align*}
	 &-i\tr(-ie\frac{\fsl{A} + iM\alpha\gamma^{01}}{\fsl{\del{}{}} + M} - \frac{1}{2}\frac{-ie\fsl{A} + iM\alpha\gamma^{01}}{\fsl{\del{}{}} + M}\frac{-ie\fsl{A} + iM\alpha\gamma^{01}}{\fsl{\del{}{}} + M}) \\
	=& -i\inte{}{}\frac{\dd[2]{p_{1}}}{(2\pi)^{2}}\frac{\dd[2]{p_{2}}}{(2\pi)^{2}}\tr(\frac{-ie\fsl{A} + iM\alpha\gamma^{01}}{\fsl{\del{}{}} + M}) + \frac{i}{2}\inte{}{}\frac{\dd[2]{p_{1}}}{(2\pi)^{2}}\frac{\dd[2]{p_{2}}}{(2\pi)^{2}}\frac{\dd[2]{p_{3}}}{(2\pi)^{2}}\frac{\dd[2]{p_{4}}}{(2\pi)^{2}}\tr(\frac{-ie\fsl{A} + iM\alpha\gamma^{01}}{\fsl{\del{}{}} + M}\frac{-ie\fsl{A} + iM\alpha\gamma^{01}}{\fsl{\del{}{}} + M}) \\
	=& -i\inte{}{}\frac{\dd[2]{p_{1}}}{(2\pi)^{2}}\frac{\dd[2]{p_{2}}}{(2\pi)^{2}}\tr(\frac{i\fsl{p_{1}} - M}{-p_{1}^{2} - M^{2}}(-ie\fsl{A}(p_{2}) + iM\alpha(p_{2})\gamma^{01}))\delta_{p_{1} + p_{2}} \\
	 &+ \frac{i}{2}\inte{}{}\frac{\dd[2]{p_{1}}}{(2\pi)^{2}}\frac{\dd[2]{p_{2}}}{(2\pi)^{2}}\frac{\dd[2]{p_{3}}}{(2\pi)^{2}}\frac{\dd[2]{p_{4}}}{(2\pi)^{2}}\tr(\frac{i\fsl{p_{1}} - M}{-p_{1}^{2} - M^{2}}(-ie\fsl{A}(p_{2}) + iM\alpha(p_{2})\gamma^{01})\frac{i\fsl{p_{3}} - M}{-p_{3}^{2} - M^{2}}(-ie\fsl{A}(p_{4}) + iM\alpha(p_{4})\gamma^{01})) \\
	 &\cdot\delta_{p_{2} + p_{1} + p_{3}}\delta_{-p_{1} - p_{3} + p_{4}} \\
	=& -\frac{i}{(2\pi)^{2}}\inte{}{}\frac{\dd[2]{p_{1}}}{(2\pi)^{2}}\tr(\frac{i\fsl{p_{1}} - M}{-p_{1}^{2} - M^{2}}(-ie\fsl{A}(-p_{1}) + iM\alpha(-p_{1})\gamma^{01})) \\
	 &+ \frac{i}{2(2\pi)^{2}}\inte{}{}\frac{\dd[2]{p_{1}}}{(2\pi)^{2}}\frac{\dd[2]{p_{2}}}{(2\pi)^{2}}\frac{\dd[2]{p_{3}}}{(2\pi)^{2}}\tr(\frac{i\fsl{p_{1}} - M}{-p_{1}^{2} - M^{2}}(-ie\fsl{A}(p_{2}) + iM\alpha(p_{2})\gamma^{01})\frac{i\fsl{p_{3}} - M}{-p_{3}^{2} - M^{2}}(-ie\fsl{A}(p_{1} + p_{3}) + iM\alpha(p_{1} + p_{3})\gamma^{01})) \\
	 &\cdot\delta_{p_{2} + p_{1} + p_{3}} \\
	=& -\frac{i}{(2\pi)^{2}}\inte{}{}\frac{\dd[2]{p_{1}}}{(2\pi)^{2}}\tr(\frac{i\fsl{p_{1}} - M}{-p_{1}^{2} - M^{2}}(-ie\fsl{A}(-p_{1}) + iM\alpha(-p_{1})\gamma^{01})) \\
	 &+ \frac{i}{2(2\pi)^{4}}\inte{}{}\frac{\dd[2]{p_{1}}}{(2\pi)^{2}}\frac{\dd[2]{p_{2}}}{(2\pi)^{2}}\tr(\frac{i\fsl{p_{1}} - M}{-p_{1}^{2} - M^{2}}(-ie\fsl{A}(p_{2}) + iM\alpha(p_{2})\gamma^{01})\frac{i(-\fsl{p_{2}} - \fsl{p_{1}}) - M}{-(-p_{2} - p_{1})^{2} - M^{2}}(-ie\fsl{A}(-p_{2}) + iM\alpha(-p_{2})\gamma^{01})).
\end{align*}
This can alternatively be illustrated using the Feynman diagram in figure \ref{fig:second_order_fd}.

\begin{figure}[!ht]
	\centering
	\begin{tikzpicture}
		\begin{feynman}
			\vertex (a) at (0, 0);
			\vertex (b) at (2, 0);
			\vertex (c) at (4, 0);
			\vertex (d) at (6, 0);
			\diagram*{
				(a) --[scalar, momentum = $k$] (b) --[fermion, half left, rmomentum = $p$] (c) --[scalar, rmomentum = $-k$] (d),
				(c) --[fermion, half left, momentum = $-p - k$] (b),
			};
		\end{feynman}
	\end{tikzpicture}
	\caption{Feynman diagram for the second-order term in the effective action.}
	\label{fig:second_order_fd}
\end{figure}

The second line produces the lowest-order topological terms. There we need only consider the case where the number of Dirac matrices is even, as the odd-numbered cases vanish when tracing the matrices. These terms are
\begin{align*}
	&\frac{i}{2(2\pi)^{4}}\inte{}{}\frac{\dd[2]{p}}{(2\pi)^{2}}\frac{\dd[2]{k}}{(2\pi)^{2}}\tr(\frac{i\fsl{p} - M}{-p^{2} - M^{2}}(-ie\fsl{A}(k) + iM\alpha(k)\gamma^{01})\frac{i(-\fsl{p} - \fsl{k}) - M}{-(p - k)^{2} - M^{2}}(-ie\fsl{A}(-k) + iM\alpha(-k)\gamma^{01})) \\
	\supset& \frac{e}{2(2\pi)^{4}}\inte{}{}\frac{\dd[2]{p}}{(2\pi)^{2}}\frac{\dd[2]{k}}{(2\pi)^{2}}\tr(\frac{i\fsl{p} - M}{-p^{2} - M^{2}}\fsl{A}(k)\frac{i(-\fsl{p} - \fsl{k}) - M}{-(-p - k)^{2} - M^{2}}iM\alpha(-k)\gamma^{01}) \\
	&+ \frac{e}{2(2\pi)^{4}}\inte{}{}\frac{\dd[2]{p}}{(2\pi)^{2}}\frac{\dd[2]{k}}{(2\pi)^{2}}\tr(\frac{i\fsl{p} - M}{-p^{2} - M^{2}}iM\alpha(k)\gamma^{01}\frac{i(-\fsl{p} - \fsl{k}) - M}{-(-p - k)^{2} - M^{2}}\fsl{A}(-k)) \\
	=& \frac{ieM}{2(2\pi)^{4}}\left(\inte{}{}\frac{\dd[2]{p}}{(2\pi)^{2}}\frac{\dd[2]{k}}{(2\pi)^{2}}\alpha(-k)A_{\mu}(k)\tr(\frac{i\fsl{p} - M}{-p^{2} - M^{2}}\gamma^{\mu}\frac{i(-\fsl{p} - \fsl{k}) - M}{-(-p - k)^{2} - M^{2}}\gamma^{01}) \right. \\
	 &+ \left. \inte{}{}\frac{\dd[2]{p}}{(2\pi)^{2}}\frac{\dd[2]{k}}{(2\pi)^{2}}\alpha(k)A_{\mu}(-k)\tr(\frac{i\fsl{p} - M}{-p^{2} - M^{2}}\gamma^{01}\frac{i(-\fsl{p} - \fsl{k}) - M}{-(-p - k)^{2} - M^{2}}\gamma^{\mu})\right) \\
	\supset& -\frac{ieM}{2(2\pi)^{4}}\left(\inte{}{}\frac{\dd[2]{p}}{(2\pi)^{2}}\frac{\dd[2]{k}}{(2\pi)^{2}}\frac{\alpha(-k)A_{\mu}(k)}{(-p^{2} - M^{2})(-(-p - k)^{2} - M^{2})}\tr(\left(i\fsl{p}\gamma^{\mu}M + M\gamma^{\mu}i(-\fsl{p} - \fsl{k})\right)\gamma^{01}) \right. \\
	 &+ \left. \frac{\alpha(k)A_{\mu}(-k)}{(-p^{2} - M^{2})(-(-p - k)^{2} - M^{2})}\tr(\left(i\fsl{p}\gamma^{01}M + M\gamma^{01}i(-\fsl{p} - \fsl{k})\right)\gamma^{\mu})\right) \\
	=& \frac{eM^{2}}{2(2\pi)^{4}}\left(\inte{}{}\frac{\dd[2]{p}}{(2\pi)^{2}}\frac{\dd[2]{k}}{(2\pi)^{2}}\frac{\alpha(-k)A_{\mu}(k)}{(-p^{2} - M^{2})(-(-p - k)^{2} - M^{2})}\tr(\fsl{p}\gamma^{\mu}\gamma^{01} + \gamma^{\mu}(-\fsl{p} - \fsl{k})\gamma^{01}) \right. \\
	 &+ \left. \frac{\alpha(k)A_{\mu}(-k)}{(-p_{1}^{2} - M^{2})(-(-p - k_{1})^{2} - M^{2})}\tr(\fsl{p}\gamma^{01}\gamma^{\mu} + \gamma^{01}(-\fsl{p} - \fsl{k})\gamma^{\mu})\right) \\
	=& \frac{eM^{2}}{2(2\pi)^{4}}\left(\inte{}{}\frac{\dd[2]{p}}{(2\pi)^{2}}\frac{\dd[2]{k}}{(2\pi)^{2}}\frac{\alpha(-k)A_{\mu}(k)}{(-p^{2} - M^{2})(-(-p - k)^{2} - M^{2})}\tr(p_{\nu}\gamma^{\nu}\gamma^{\mu}\gamma^{01} + (-p - k)_{\nu}\gamma^{\mu}\gamma^{\nu}\gamma^{01}) \right. \\
	 &+ \left. \frac{\alpha(k)A_{\mu}(-k)}{(-p^{2} - M^{2})(-(-p - k)^{2} - M^{2})}\tr(p_{\nu}\gamma^{\nu}\gamma^{01}\gamma^{\mu} + (-p - k)_{\nu}\gamma^{01}\gamma^{\nu}\gamma^{\mu})\right).
\end{align*}
Because we take the mass fields to be slowly varying, we can remove all contributions over order 1 in $\frac{k}{p}$. Inverting the $k$-integral in the second term leaves us with
\begin{align*}
	 &\frac{eM^{2}}{2(2\pi)^{4}}\left(\inte{}{}\frac{\dd[2]{p}}{(2\pi)^{2}}\frac{\dd[2]{k}}{(2\pi)^{2}}\frac{\alpha(-k)A_{\mu}(k)}{(-p^{2} - M^{2})^{2}}\tr(p_{\nu}\gamma^{\nu}\gamma^{\mu}\gamma^{01} - (p + k)_{\nu}\gamma^{\mu}\gamma^{\nu}\gamma^{01} + p_{\nu}\gamma^{\nu}\gamma^{01}\gamma^{\mu} + (-p + k)_{\nu}\gamma^{01}\gamma^{\nu}\gamma^{\mu})\right) \\
	=& -\frac{eM^{2}}{(2\pi)^{4}}\inte{}{}\frac{\dd[2]{p}}{(2\pi)^{2}}\frac{\dd[2]{k}}{(2\pi)^{2}}\frac{\alpha(-k)k_{\nu}A_{\mu}(k)}{(-p^{2} - M^{2})^{2}}\tr(\gamma^{\mu}\gamma^{\nu}\gamma^{01}) \\
	=& -\frac{2eM^{2}}{(2\pi)^{4}}\inte{}{}\frac{\dd[2]{k}}{(2\pi)^{2}}\varepsilon^{\mu\nu}\alpha(-k)k_{\nu}A_{\mu}(k)\inte{}{}\frac{\dd[2]{p}}{(2\pi)^{2}}\frac{1}{(p^{2} + M^{2})^{2}}.
\end{align*}
Let us consider the innermost integral. Performing a Wick rotation and a substitution we have
\begin{align*}
	\inte{}{}\frac{\dd[2]{p}}{(2\pi)^{2}}\frac{1}{(p^{2} + M^{2})^{2}} =& \frac{1}{M^{2}}\inte{}{}\frac{\dd[2]{q}}{(2\pi)^{2}}\frac{1}{(-q_{0}^{2} + q_{1}^{2} + 1)^{2}} \\
	=& \frac{i}{M^{2}}\inte{}{}\frac{\dd[2]{\ell}}{(2\pi)^{2}}\frac{1}{(\ell_{0}^{2} + \ell_{1}^{2} + 1)^{2}} \\
	=& \frac{i}{2\pi M^{2}}\inte{0}{\infty}\dd{r}\frac{r}{(r^{2} + 1)^{2}} \\
	=& -\frac{i}{2\pi M^{2}}\eval{\frac{1}{2(r^{2} + 1)}}_{0}^{\infty} \\
	=& \frac{i}{4\pi M^{2}}.
\end{align*}
Let us also note the general result
\begin{align*}
	\inte{}{}\frac{\dd[d]{p}}{(2\pi)^{d}}\frac{(p^{2})^{a}}{(p^{2} + \Delta)^{b}} = \frac{\Gamma\left(b - a - \frac{d}{2}\right)\Gamma\left(a + \frac{d}{2}\right)}{(4\pi)^{\frac{d}{2}}\Gamma\left(b\right)\Gamma\left(\frac{d}{2}\right)}\Delta^{a + \frac{d}{2} - b}.
\end{align*}
The final expression for the momentum space effective action is then
\begin{align*}
	\Gamma = -\frac{ie}{2\pi(2\pi)^{4}}\inte{}{}\frac{\dd[2]{k}}{(2\pi)^{2}}\varepsilon^{\mu\nu}\alpha(-k)k_{\nu}A_{\mu}(k),
\end{align*}
and switching to real space we have
\begin{align*}
	\Gamma = \frac{e}{2\pi(2\pi)^{4}}\inte{}{}\dd[2]{x}\varepsilon^{\mu\nu}\alpha\del{}{\mu}A_{\nu} = \frac{e}{2\pi(2\pi)^{4}}\inte{}{}\alpha F.
\end{align*}
The normalization in this step is likely off - namely, the vertices should probably provide a factor of $(2\pi)^{2}$ each. Absorbing the coupling constant into the gauge field (which modifies the free term) as well, the effective action should be
\begin{align*}
	\Gamma = \frac{1}{2\pi}\inte{}{}\alpha F.
\end{align*}

Another model one could study is
\begin{align*}
	\lag = -i\overline{\Psi}(\fsl{\del{}{}} + M_{0} + i\gamma^{01}M_{i}\sigma^{i})\Psi,
\end{align*}
where the Pauli matrices acts in flavor space and the other Dirac matrices act on the Dirac structure. The $M$ take values on $S^{3}$ at any point in spacetime. The effective action is then
\begin{align*}
	\Gamma =& -i\ln(\det(-i(\fsl{\del{}{}} + M_{0} + i\gamma^{01}M_{i}\sigma^{i}))) \\
	       =& -i\tr(\ln(-i(\fsl{\del{}{}} + M_{0} + i\gamma^{01}M_{i}\sigma^{i}))) \\
	       =& -i\tr(\ln(-i(\fsl{\del{}{}} + M_{0})\left(1 + \frac{i\gamma^{01}M_{i}\sigma^{i}}{\fsl{\del{}{}} + M_{0}}\right))) \\
	       =& C_{0} - i\tr(\ln(1 + \frac{i\gamma^{01}M_{i}\sigma^{i}}{\fsl{\del{}{}} + M_{0}})).
\end{align*}
The relevant parts are
\begin{align*}
	\Gamma =& -i\tr(\frac{i\gamma^{01}M_{i}\sigma^{i}}{\fsl{\del{}{}} + M_{0}} - \frac{1}{2}\frac{i\gamma^{01}M_{i}\sigma^{i}}{\fsl{\del{}{}} + M_{0}}\frac{i\gamma^{01}M_{i}\sigma^{i}}{\fsl{\del{}{}} + M_{0}}) \\
	       =& -i\tr(\frac{\fsl{\del{}{}} - M_{0}}{\del{2}{} - M_{0}^{2}}i\gamma^{01}M_{i}\sigma^{i} - \frac{1}{2}\frac{\fsl{\del{}{}} - M_{0}}{\del{2}{} - M_{0}^{2}}i\gamma^{01}M_{i}\sigma^{i}\frac{\fsl{\del{}{}} - M_{0}}{\del{2}{} - M_{0}^{2}}i\gamma^{01}M_{j}\sigma^{j}) \\
	       =& -\frac{i}{(2\pi)^{2}}\inte{}{}\frac{\dd[2]{p}}{(2\pi)^{2}}\tr(\frac{i\fsl{p} - M_{0}}{-p^{2} - M_{0}^{2}}i\gamma^{01}M_{i}(p)\sigma^{i}) \\
	        &- \frac{i}{2(2\pi)^{4}}\inte{}{}\frac{\dd[2]{p}}{(2\pi)^{2}}\frac{\dd[2]{k}}{(2\pi)^{2}}\tr(\frac{-i\fsl{k} - M_{0}}{-k^{2} - M_{0}^{2}}\gamma^{01}M_{i}(p)\sigma^{i}\frac{-i(\fsl{k} - \fsl{p}) - M_{0}}{-(k - p)^{2} - M_{0}^{2}}\gamma^{01}M_{j}(-p)\sigma^{j}).
\end{align*}
The first-order term should be trivial. The topological term should thus arise from
\begin{align*}
	\Gamma =& \frac{i}{2(2\pi)^{4}}\inte{}{}\frac{\dd[2]{p}}{(2\pi)^{2}}\frac{\dd[2]{k}}{(2\pi)^{2}}\frac{M_{i}(p)M_{j}(-p)}{(-k^{2} - M_{0}^{2})(-(k - p)^{2} - M_{0}^{2})}\tr(\fsl{k}\gamma^{01}\sigma^{i}(\fsl{k} - \fsl{p})\gamma^{01}\sigma^{j}) \\
	=&
\end{align*}
The new term in the action contains
\begin{align*}
	\inte{X}{}\pub{\phi}{\omega},
\end{align*}
where $X$ is spacetime and $\omega$ satisfies $\df{\omega} = 2\pi\text{Vol}(S^{3})$, the latter being the volume form on $S^{3}$ normalized to volume 1. This volume form is what is of interest.

A final model to study in one dimension is
\begin{align*}
	\lag = -i\overline{\Psi}\left(\fsl{\del{}{}} + M_{0}(x) + \sum\limits_{a = 2, 3, 5}iM_{a}(x)\gamma^{a}\right)\Psi
\end{align*}
for background mass fields. Confining the mass fields to $S^{3}$ we may write the above as
\begin{align*}
	\lag = -i\overline{\Psi}\left(\fsl{\del{}{}} + M\left(\cos(\alpha) + i\sin(\alpha)\sum\limits_{a = 2, 3, 5}m_{a}\gamma^{a}\right)\right)\Psi,\ \sum\limits_{a = 2, 3, 5}m_{a}^{2} = 1.
\end{align*}
We once again employ the perturbation approach to write the effective action as
\begin{align*}
	\Gamma =& -i\ln(\det(-i\left(\fsl{\del{}{}} + M\left(\cos(\alpha) + i\sin(\alpha)\sum\limits_{a = 2, 3, 5}m_{a}\gamma^{a}\right)\right))) \\
	       =& -i\tr(\ln(-i\left(\fsl{\del{}{}} + M\left(\cos(\alpha) + i\sin(\alpha)\sum\limits_{a = 2, 3, 5}m_{a}\gamma^{a}\right)\right))) \\
	 \approx& -i\tr(\ln(-i(\fsl{\del{}{}} + M)\left(1 + \frac{i\alpha\sum\limits_{a = 2, 3, 5}m_{a}\gamma^{a}}{\fsl{\del{}{}} + M}\right))) \\
	       =& C_{0} - i\tr(\ln(1 + \frac{i\alpha\sum\limits_{a = 2, 3, 5}m_{a}\gamma^{a}}{\fsl{\del{}{}} + M})).
z\end{align*}
Because we are working with all Dirac matrices, we note that the only shot at obtaining a topological term is to consider a third-order term in the expansion. The Feynman diagram is shown in figure \ref{fig:third_order_fd}.

\begin{figure}[!ht]
	\centering
	\begin{tikzpicture}
		\begin{feynman}
			\vertex (a) at (0, 0);
			\vertex (b) at (2, 0);
			\vertex (c) at (4.6, 1);
			\vertex (d) at (5.9, 1.5);
			\vertex (e) at (4.6, -1);
			\vertex (f) at (5.9, -1.5);
			\diagram*{
				(a) --[scalar, momentum = $p$] (b) --[fermion, out = 90, in = 135, rmomentum = $k_{1}$] (c) --[scalar, rmomentum = $k_{2}$] (d),
				(f) --[scalar, momentum = $-p - k_{2}$] (e) --[fermion, out = -135, in = -90, momentum = $-k_{1} - p$] (b),
				(c) --[fermion, out = -45, in = 45, momentum = $k_{2} - k_{1}$] (e),
			};
		\end{feynman}
	\end{tikzpicture}
	\caption{Feynman diagram for the third-order term in the effective action.}
	\label{fig:third_order_fd}
\end{figure}

This is given by
\begin{align*}
	\Gamma =& -\frac{i}{3}\inte{}{}\frac{\dd[2]{p}}{(2\pi)^{2}}\frac{\dd[2]{k_{1}}}{(2\pi)^{2}}\frac{\dd[2]{k_{2}}}{(2\pi)^{2}}\tr(\frac{i\alpha\sum\limits_{a = 2, 3, 5}m_{a}\gamma^{a}}{\fsl{\del{}{}} + M}\frac{i\alpha\sum\limits_{b = 2, 3, 5}m_{b}\gamma^{b}}{\fsl{\del{}{}} + M}\frac{i\alpha\sum\limits_{c = 2, 3, 5}m_{c}\gamma^{c}}{\fsl{\del{}{}} + M}) \\
	       =& -\frac{1}{3}\inte{}{}\frac{\dd[2]{p}}{(2\pi)^{2}}\frac{\dd[2]{k_{1}}}{(2\pi)^{2}}\frac{\dd[2]{k_{2}}}{(2\pi)^{2}} \\
	        &\cdot\sum\limits_{a, b, c = 2, 3, 5}\alpha m_{a}(p)\alpha m_{b}(k_{2})\alpha m_{c}(-p - k_{2})\tr(\frac{i\fsl{k_{1}} - M}{-k_{1}^{2} - M^{2}}\gamma^{a}\frac{i(-\fsl{k_{1}} - \fsl{p}) - M}{-(-k_{1} - p)^{2} - M^{2}}\gamma^{b}\frac{i(\fsl{k_{2}} - \fsl{k_{1}}) - M}{-(k_{2} - k_{1})^{2} - M^{2}}\gamma^{c}) \\
	       =& -\frac{1}{3}\inte{}{}\frac{\dd[2]{p}}{(2\pi)^{2}}\frac{\dd[2]{k_{1}}}{(2\pi)^{2}}\frac{\dd[2]{k_{2}}}{(2\pi)^{2}}\sum\limits_{a, b, c = 2, 3, 5}\frac{M_{a}(p)M_{b}(k_{2})M_{c}(-p - k_{2})}{(-k_{1}^{2} - M^{2})(-(-k_{1} - p)^{2} - M^{2})(-(k_{2} - k_{1})^{2} - M^{2})} \\
	        &\cdot\tr((i\fsl{k_{1}} - M)\gamma^{a}(i(-\fsl{k_{1}} - \fsl{p}) - M)\gamma^{b}(i(\fsl{k_{2}} - \fsl{k_{1}}) - M)\gamma^{c}).
\end{align*}
The topological term corresponds to exactly one of $a, b, c$ being 5 and one of the numerators in the fraction being a mass. For example, setting $a = 5$ nets you
\begin{align*}
	 &-M\left(\tr(\gamma^{5}(-\fsl{k_{1}} - \fsl{p})\gamma^{b}(\fsl{k_{2}} - \fsl{k_{1}})\gamma^{c} + \fsl{k_{1}}\gamma^{5}\gamma^{b}(\fsl{k_{2}} - \fsl{k_{1}})\gamma^{c} + \fsl{k_{1}}\gamma^{5}(-\fsl{k_{1}} - \fsl{p})\gamma^{b}\gamma^{c})\right) \\
	=& -M\left(\tr((-k_{1} - p)_{\mu}(k_{2} - k_{1})_{\nu}\gamma^{5}\gamma^{\mu}\gamma^{b}\gamma^{\nu}\gamma^{c} + k_{1, \mu}(k_{2} - k_{1})_{\nu}\gamma^{\mu}\gamma^{5}\gamma^{b}\gamma^{\nu}\gamma^{c} + k_{1, \mu}(-k_{1} - p)_{\nu}\gamma^{\mu}\gamma^{5}\gamma^{\nu}\gamma^{b}\gamma^{c})\right) \\
	=& -M\left(\tr(-(-k_{1} - p)_{\mu}(k_{2} - k_{1})_{\nu}\gamma^{\mu}\gamma^{\nu}\gamma^{b}\gamma^{c}\gamma^{5} - k_{1, \mu}(k_{2} - k_{1})_{\nu}\gamma^{\nu}\gamma^{\mu}\gamma^{b}\gamma^{c}\gamma^{5} + k_{1, \mu}(-k_{1} - p)_{\nu}\gamma^{\nu}\gamma^{\mu}\gamma^{b}\gamma^{c}\gamma^{5})\right) \\
	=& -M\left(-(-k_{1} - p)_{\mu}(k_{2} - k_{1})_{\nu}\varepsilon^{\mu\nu bc} - k_{1, \mu}(k_{2} - k_{1})_{\nu}\varepsilon^{\nu\mu bc} + k_{1, \mu}(-k_{1} - p)_{\nu}\varepsilon^{\nu\mu bc}\right) \\
	=& -M\varepsilon^{\mu\nu bc}\left(-(-k_{1} - p)_{\mu}(k_{2} - k_{1})_{\nu} + k_{1, \mu}(k_{2} - k_{1})_{\nu} - k_{1, \mu}(-k_{1} - p)_{\nu}\right) \\
	=&  -M\varepsilon^{\mu\nu bc}\left(-(-k_{1} - p)_{\mu}(k_{2} - k_{1})_{\nu} - k_{1, \mu}(-k_{2} - p)_{\nu}\right).
\end{align*}
Approximating $k_{1}$ to be much larger than the other momenta, the second term produces no contribution. Left is
\begin{align*}
	\Gamma \supset& -\frac{M}{3}\sum\limits_{b, c = 2, 3}\varepsilon^{\mu\nu bc}\inte{}{}\frac{\dd[2]{p}}{(2\pi)^{2}}\frac{\dd[2]{k_{1}}}{(2\pi)^{2}}\frac{\dd[2]{k_{2}}}{(2\pi)^{2}}M_{5}(p)\frac{M_{b}(k_{2})M_{c}(-p - k_{2})(-k_{1} - p)_{\mu}(k_{2} - k_{1})_{\nu}}{(-k_{1}^{2} - M^{2})(-(-k_{1} - p)^{2} - M^{2})(-(k_{2} - k_{1})^{2} - M^{2})}.
\end{align*}
One term is symmetric in $\mu$ and $\nu$ and two are linear in $k_{1}$, meaning the only surviving term should be
\begin{align*}
	\Gamma \supset& \frac{M}{3}\sum\limits_{b, c = 2, 3}\varepsilon^{\mu\nu bc}\inte{}{}\frac{\dd[2]{p}}{(2\pi)^{2}}\frac{\dd[2]{k_{1}}}{(2\pi)^{2}}\frac{\dd[2]{k_{2}}}{(2\pi)^{2}}M_{5}(p)\frac{M_{b}(k_{2})M_{c}(-p - k_{2})p_{\mu}k_{2, \nu}}{(-k_{1}^{2} - M^{2})(-(-k_{1} - p)^{2} - M^{2})(-(k_{2} - k_{1})^{2} - M^{2})} \\
	       \approx& -\sum\limits_{b, c = 2, 3}\frac{M}{3}\varepsilon^{\mu\nu bc}\inte{}{}\frac{\dd[2]{p}}{(2\pi)^{2}}\frac{\dd[2]{k_{2}}}{(2\pi)^{2}}p_{\mu}M_{5}(p)k_{2, \nu}M_{b}(k_{2})M_{c}(-p - k_{2})\inte{}{}\frac{\dd[2]{k_{1}}}{(2\pi)^{2}}\frac{1}{(k_{1}^{2} + M^{2})^{3}} \\
	             =& -\frac{1}{3M^{3}}\frac{\Gamma\left(\frac{5}{2}\right)\Gamma\left(\frac{1}{2}\right)}{2\sqrt{\pi}\Gamma\left(3\right)\Gamma\left(\frac{1}{2}\right)}\sum\limits_{b, c = 2, 3}\varepsilon^{\mu\nu bc}\inte{}{}\frac{\dd[2]{p}}{(2\pi)^{2}}\frac{\dd[2]{k_{2}}}{(2\pi)^{2}}p_{\mu}M_{5}(p)k_{2, \nu}M_{b}(k_{2})M_{c}(-p - k_{2}) \\
	             =& -\frac{1}{3M^{3}}\frac{3}{16}\sum\limits_{b, c = 2, 3}\varepsilon^{\mu\nu bc}\inte{}{}\frac{\dd[2]{p}}{(2\pi)^{2}}\frac{\dd[2]{k_{2}}}{(2\pi)^{2}}p_{\mu}M_{5}(p)k_{2, \nu}M_{b}(k_{2})M_{c}(-p - k_{2}).
\end{align*}
Using the facts that the values of $b, c$ are disjoint from those of $\mu, \nu$, allo we have that the real space term is
\begin{align*}
	\Gamma \supset \frac{1}{16M^{3}}\sum\limits_{b, c = 2, 3}\varepsilon^{\mu\nu}\varepsilon^{bc}\inte{}{}\dd[2]{x}M_{c}\del{}{\mu}M_{5}\del{}{\nu}M_{b}.
\end{align*}

To understand the systematics of the above calculations, let us adopt aliases for the momenta and write the integrand as
\begin{align*}
	 &\sum\limits_{a, b, c = 2, 3, 5}\frac{M_{a}(p)M_{b}(k_{2})M_{c}(-p - k_{2})}{(-k_{1}^{2} - M^{2})(-(-k_{1} - p)^{2} - M^{2})(-(k_{2} - k_{1})^{2} - M^{2})} \\
	 &\cdot\tr((i\fsl{k_{1}} - M)\gamma^{a}(i(-\fsl{k_{1}} - \fsl{p}) - M)\gamma^{b}(i(\fsl{k_{2}} - \fsl{k_{1}}) - M)\gamma^{c}) \\
	=& \sum\limits_{a, b, c = 2, 3, 5}\frac{M_{a}(u_{1})M_{b}(u_{2})M_{c}(u_{3})}{(-v_{1}^{2} - M^{2})(-v_{2}^{2} - M^{2})(-v_{3}^{2} - M^{2})}\tr((i\fsl{v_{1}} - M)\gamma^{a}(i\fsl{v_{2}} - M)\gamma^{b}(i\fsl{v_{3}} - M)\gamma^{c}).
\end{align*}
Because $v_{1}$ is explicitly integrated over, we can deduce some more restrictions on the necessary considerations to identify topological terms. In addition to fixing exactly one latin index to $5$ we may also take the first factor to be the one contributing the factor $M$. Thus this factor amounts to
\begin{align*}
	 &Mv_{2, \mu}v_{3, \nu}\sum\limits_{a, b, c = 2, 3, 5}\frac{M_{a}(u_{1})M_{b}(u_{2})M_{c}(u_{3})}{(-v_{1}^{2} - M^{2})(-v_{2}^{2} - M^{2})(-v_{3}^{2} - M^{2})}\tr(\gamma^{a}\gamma^{\mu}\gamma^{b}\gamma^{\nu}\gamma^{c}) \\
	=& Mv_{2, \mu}v_{3, \nu}\sum\limits_{b, c = 2, 3}\frac{M_{5}(u_{1})M_{b}(u_{2})M_{c}(u_{3})\varepsilon^{\mu b\nu c} + M_{a}(u_{1})M_{5}(u_{2})M_{c}(u_{3})\varepsilon^{\nu c\mu b} + M_{a}(u_{1})M_{b}(u_{2})M_{5}(u_{3})\varepsilon^{\mu b\nu c}}{(-v_{1}^{2} - M^{2})(-v_{2}^{2} - M^{2})(-v_{3}^{2} - M^{2})} \\
	=& -Mv_{2, \mu}v_{3, \nu}\sum\limits_{b, c = 2, 3}\varepsilon^{\mu\nu bc}\frac{M_{5}(u_{1})M_{b}(u_{2})M_{c}(u_{3}) - M_{b}(u_{1})M_{5}(u_{2})M_{c}(u_{3}) + M_{b}(u_{1})M_{c}(u_{2})M_{5}(u_{3})}{(-v_{1}^{2} - M^{2})(-v_{2}^{2} - M^{2})(-v_{3}^{2} - M^{2})}.
\end{align*}
To proceed we use the relations
\begin{align*}
	v_{2} = -v_{1} - u_{1},\ v_{3} = u_{2} - v_{1},
\end{align*}
which follow from the definition, to write the above as
\begin{align*}
	 &M\varepsilon^{\mu\nu}(v_{1} + u_{1})_{\mu}(u_{2} - v_{1})_{\nu}\sum\limits_{b, c = 2, 3}\varepsilon^{bc}\frac{M_{5}(u_{1})M_{b}(u_{2})M_{c}(u_{3}) - M_{b}(u_{1})M_{5}(u_{2})M_{c}(u_{3}) + M_{b}(u_{1})M_{c}(u_{2})M_{5}(u_{3})}{(-v_{1}^{2} - M^{2})(-v_{2}^{2} - M^{2})(-v_{3}^{2} - M^{2})}.
\end{align*}
Once again we are left with only the contribution
\begin{align*}
	M\varepsilon^{\mu\nu}u_{1, \mu}u_{2, \nu}\sum\limits_{b, c = 2, 3}\varepsilon^{bc}\frac{M_{5}(u_{1})M_{b}(u_{2})M_{c}(u_{3}) - M_{b}(u_{1})M_{5}(u_{2})M_{c}(u_{3}) + M_{b}(u_{1})M_{c}(u_{2})M_{5}(u_{3})}{(-v_{1}^{2} - M^{2})(-v_{2}^{2} - M^{2})(-v_{3}^{2} - M^{2})},
\end{align*}
which in real space is proportional to
\begin{align*}
	\varepsilon^{\mu\nu}\sum\limits_{b, c = 2, 3}\varepsilon^{bc}\left(M_{c}\del{}{\mu}M_{5}\del{}{\nu}M_{b} - M_{c}\del{}{\mu}M_{b}\del{}{\nu}M_{5} + M_{5}\del{}{\mu}M_{b}\del{}{\nu}M_{c}\right).
\end{align*}
Constants of proportionality are the same, hence the full topological term is
\begin{align*}
	\Gamma =& \frac{1}{16M^{3}}\sum\limits_{b, c = 2, 3}\inte{}{}\dd[2]{x}\varepsilon^{\mu\nu}\varepsilon^{bc}\left(M_{c}\del{}{\mu}M_{5}\del{}{\nu}M_{b} - M_{c}\del{}{\mu}M_{b}\del{}{\nu}M_{5} + M_{5}\del{}{\mu}M_{b}\del{}{\nu}M_{c}\right) \\
	       =& \frac{1}{16M^{3}}\sum\limits_{a, b, c = 2, 3, 5}\inte{}{}\dd[2]{x}\varepsilon^{\mu\nu}\varepsilon^{abc}M_{a}\del{}{\mu}M_{b}\del{}{\nu}M_{c}.
\end{align*}
We can take this one step further by taking the mass fields to be a map from spacetime to $S^{3}$ and introducing the 2-form $\omega$ with components $\omega_{ab} = \varepsilon_{cab}M^{a}$ to write the above as
\begin{align*}
	\Gamma = \frac{1}{16M^{3}}\inte{}{}\pub{M}{\omega},
\end{align*}
with the integral still being over spacetime.

Let us study $\omega$ more closely. Its exterior derivative is
\begin{align*}
	\df{\omega} =& \frac{1}{2}\del{}{d}\varepsilon_{cab}M^{c}e^{dab} = \frac{1}{2}\varepsilon_{dab}e^{dab},
\end{align*}
which is almost the volume form on $S^{3}$.

\paragraph{A New Model}
The models studied above have connected $d$-dimensional systems to response terms described by integrals of $d + 1$-forms. Let us consider spatial dimension $1$ and introduce a mass matrix
\begin{align*}
	M = M_{0} + i\gamma^{01}\sum\limits_{l = 1}^{5}M_{l}\Gamma^{l}.
\end{align*}
The matrices $\Gamma$ represent a Clifford algebra and act in flavor space, and we take the mass fields to lie on $S^{5}$. The Lagrangian of the model is
\begin{align*}
	\lag = -i\overline{\Psi}\left(\fsl{\del{}{}} + M\right)\Psi = -i\overline{\Psi}\left(\fsl{\del{}{}} + M_{0} + i\gamma^{01}\sum\limits_{l = 1}^{5}M_{l}\Gamma^{l}\right)\Psi.
\end{align*}
The effective action is
\begin{align*}
	\Gamma =& -i\ln(\det(-i\left(\fsl{\del{}{}} + M_{0} + i\gamma^{01}\sum\limits_{l = 1}^{5}M_{l}\Gamma^{l}\right))) \\
	       =& -i\tr(\ln(-i\left(\fsl{\del{}{}} + M_{0} + i\gamma^{01}\sum\limits_{l = 1}^{5}M_{l}\Gamma^{l}\right))) \\
	       =& C_{0} - i\tr(\ln(1 + \frac{m + i\gamma^{01}\sum\limits_{l = 1}^{5}M_{l}\Gamma^{l}}{\fsl{\del{}{}} + M})) \\
	       =& C_{0} - i\tr(\ln(1 + \frac{(\fsl{\del{}{}} - M)\left(m + i\gamma^{01}\sum\limits_{l = 1}^{5}M_{l}\Gamma^{l}\right)}{\del{2}{} - M^{2}})),
\end{align*}
where we have introduced the mass perturbation $m = M_{0} - M$, with $M$ being a fixed mass scale parameter. The topological term comes from the fifth-order expansion, with the Feynman diagram shown in figure

\begin{figure}[!ht]
	\centering
	\begin{tikzpicture}
	\begin{feynman}
	\vertex (a) at (-6, 0);
	\vertex (a1) at (-2, 0);
	\vertex (b) at (-1.86, 5.7);
	\vertex (b1) at (-0.62, 1.9);
	\vertex (c) at (4.8, 3.6);
	\vertex (c1) at (1.62, 1.18);
	\vertex (d) at (4.8, -3.6);
	\vertex (d1) at (1.62, -1.18);
	\vertex (e) at (-1.86, -5.7);
	\vertex (e1) at (-0.62, -1.9);
	\diagram*{
		(a) --[scalar, momentum = $p_{1}$] (a1) --[fermion, momentum = $k$] (b1) --[scalar, rmomentum = $p_{2}$] (b),
		(c) --[scalar, momentum = $p_{3}$] (c1) --[fermion, momentum = $p_{2} + p_{3} + k$] (d1) --[scalar, rmomentum = $p_{4}$] (d),
		(b1) --[fermion, momentum = $p_{2} + k$] (c1),
		(d1) --[fermion, momentum = $p_{2} + p_{3} + p_{3} + k$] (e1) --[scalar, rmomentum' = $-p_{1} - p_{2} - p_{3} - p_{4}$] (e),
		(e1) --[fermion, momentum = $-p_{1} + k$] (a1),
	};
	\end{feynman}
	\end{tikzpicture}
	\caption{Feynman diagram for the third-order term in the effective action.}
	\label{fig:fifth_order_fd}
\end{figure}

This translates to
\begin{align*}
	\Gamma =&  -\frac{i}{5}\inte{}{}\frac{\dd[2]{p_{1}}}{(2\pi)^{2}}\dots\frac{\dd[2]{k}}{(2\pi)^{2}}\tr(\left(\frac{(\fsl{\del{}{}} - M)\left(m + i\gamma^{01}\sum\limits_{l = 1}^{5}M_{l}\Gamma^{l}\right)}{\del{2}{} - M^{2}}\right)^{5}) \\
	\supset&  \frac{1}{5}\sum\limits_{l_{i} = 1}^{5}\inte{}{}\frac{\dd[2]{p_{1}}}{(2\pi)^{2}}\dots\frac{\dd[2]{k}}{(2\pi)^{2}}\text{tr}\left(\frac{(i\fsl{k} - M)\gamma^{01}M_{l_{1}}\Gamma^{l_{1}}}{-k^{2} - M^{2}}\frac{(i(\fsl{p}_{2} + \fsl{k}) - M)\gamma^{01}M_{l_{2}}\Gamma^{l_{2}}}{-(p_{2} + k)^{2} - M^{2}}\frac{(i(\fsl{p}_{2} + \fsl{p}_{3} + \fsl{k}) - M)\gamma^{01}M_{l_{3}}\Gamma^{l_{3}}}{-(p_{2} + p_{3} + k)^{2} - M^{2}} \right. \\
	&\cdot \left.\frac{(i(\fsl{p}_{2} + \fsl{p}_{3} + \fsl{p}_{4} + \fsl{k}) - M)\gamma^{01}M_{l_{4}}\Gamma^{l_{4}}}{-(p_{2} + p_{3} + p_{4} + k)^{2} - M^{2}}\frac{(i(-\fsl{p}_{1} + \fsl{k}) - M)\gamma^{01}M_{l_{5}}\Gamma^{l_{5}}}{(-p_{1} + k)^{2} - M^{2}}\right).
\end{align*}
Let us now consider the contents of the trace. We are looking for topological terms, which appear in the presence of all $\Gamma$ and all $\gamma$ appearing exactly once. At this point we can therefore note that the topological term can be related to a $2$-form. This is an intrinsic limitation of the theory.

The matrices will produce a Levi-Civita tensor, meaning any contributions of orders $1$ or $2$ in $k$ will vanish. As such the topological term is given by
\begin{align*}
	\Gamma \supset& -\frac{M}{5}\sum\limits_{l_{i} = 1}^{5}\inte{}{}\frac{\dd[2]{p_{1}}}{(2\pi)^{2}}\dots\frac{\dd[2]{k}}{(2\pi)^{2}}\text{tr}\left(\frac{M_{l_{1}}\Gamma^{l_{1}}}{-k^{2} - M^{2}}\frac{(i(\fsl{p}_{2} + \fsl{k}) - M)\gamma^{01}M_{l_{2}}\Gamma^{l_{2}}}{-(p_{2} + k)^{2} - M^{2}}\frac{(i(\fsl{p}_{2} + \fsl{p}_{3} + \fsl{k}) - M)\gamma^{01}M_{l_{3}}\Gamma^{l_{3}}}{-(p_{2} + p_{3} + k)^{2} - M^{2}} \right. \\
	&\cdot \left.\frac{(i(\fsl{p}_{2} + \fsl{p}_{3} + \fsl{p}_{4} + \fsl{k}) - M)\gamma^{01}M_{l_{4}}\Gamma^{l_{4}}}{-(p_{2} + p_{3} + p_{4} + k)^{2} - M^{2}}\frac{(i(-\fsl{p}_{1} + \fsl{k}) - M)M_{l_{5}}\Gamma^{l_{5}}}{(-p_{1} + k)^{2} - M^{2}}\right).
\end{align*}
The contents of the trace are
\begin{align*}
	&\tr(\Gamma^{l_{1}}(i(\fsl{p}_{2} + \fsl{k}) - M)\gamma^{01}\Gamma^{l_{2}}(i(\fsl{p}_{2} + \fsl{p}_{3} + \fsl{k}) - M)\gamma^{01}\Gamma^{l_{3}}(i(\fsl{p}_{2} + \fsl{p}_{3} + \fsl{p}_{4} + \fsl{k}) - M)\gamma^{01}\Gamma^{l_{4}}(i(-\fsl{p}_{1} + \fsl{k}) - M)\Gamma^{l_{5}}) \\
	=& \tr(\Gamma^{l_{1}}\Gamma^{l_{2}}\Gamma^{l_{3}}\Gamma^{l_{4}}\Gamma^{l_{5}})\tr((i(\fsl{p}_{2} + \fsl{k}) - M)\gamma^{01}(i(\fsl{p}_{2} + \fsl{p}_{3} + \fsl{k}) - M)\gamma^{01}(i(\fsl{p}_{2} + \fsl{p}_{3} + \fsl{p}_{4} + \fsl{k}) - M)\gamma^{01}(i(-\fsl{p}_{1} + \fsl{k}) - M)),
\end{align*}
exploiting the product structure of the operators. The remaining trace gives a contribution to the topological term according to
\begin{align*}
	-M^{3}\tr((i(\fsl{p}_{2} + \fsl{k}) - M)\gamma^{01}(i(\fsl{p}_{2} + \fsl{p}_{3} + \fsl{k}) - M)\gamma^{01}(i(\fsl{p}_{2} + \fsl{p}_{3} + \fsl{p}_{4} + \fsl{k}) - M)\gamma^{01}(i(-\fsl{p}_{1} + \fsl{k}) - M))
\end{align*}

What if we were to add coupling to a gauge field too? The effective action would be
\begin{align*}
	\Gamma =& -i\tr(\ln(1 + \frac{ie\fsl{A} + i\sum\limits_{a = 2, 3, 5}M_{a}\gamma^{a}}{\fsl{\del{}{}} + M})).
\end{align*}
The structure of the Feynman diagram is identical, hence the trace part of the effective action is
\begin{align*}
	-i\tr((i\fsl{k_{1}} - M)(e\fsl{A} + M_{a}\gamma^{a})(i(-\fsl{k_{1}} - \fsl{p}) - M)(e\fsl{A} + M_{b}\gamma^{a})(i(\fsl{k_{2}} - \fsl{k_{1}}) - M)(e\fsl{A} + M_{c}\gamma^{c})).
\end{align*}
Because all of the latin-indiced Dirac matrices are needed, however, the gauge field does not appear in any topological terms.

\paragraph{Dimensional Reduction in Topological Insulators}
Topological insulators are generally described by some fermion model. Two classes of such models are A and AIII. The two are defined by not respecting time reversal and charge conjugation as symmetries. In addition, AIII has some unitary operator that anticommutes with the Hamiltonian, also known as having a chiral symmetry. The existence of the chiral symmetry implies symmetry of the energy spectrum about the Fermi level.

It turns out that every model of class A in even spatial dimension can be related to a model of class AIII in one lower dimension. The process of relating the two is termed dimensional reduction. The scheme is as follows: Start with a Dirac model in $d = 2n + 2$ dimensions given by
\begin{align*}
	\ham = m\Gamma_{(2n + 3)}^{2n + 3} + \sum\limits_{a = 1}^{2n + 2}k_{a}\Gamma_{(2n + 3)}^{a},
\end{align*}
with $\Gamma_{(2n + 3)}^{a}$ being generators of the $2n + 3$-dimensional Clifford algebra. This is the class A model. The dimensional reduction is performed by setting $k_{2n + 2} = 0$, yielding a model in $d = 2n + 2$ in class AIII, as the Hamiltonian now anticommutes with $\Gamma_{(2n + 3)}^{2n + 2}$.

Generally for these models we introduce Bloch states $\ket{u_{a}^{\pm}(k)}$, with $k$ being confined to the first Brillouin zone, $a$ being a band index and the sign dictating whether the state is occupied. From this we can define a non-Abelian Berry curvature for the occupied state according to
\begin{align*}
	A_{ab, \mu} = \mel{u_{a}^{-}(k)}{\del{}{\mu}}{u_{b}^{-}(k)}.
\end{align*}
The Berry curvature is then $F = \df{A} + A^{2}$. Models in class A in $d = 2n + 2$ can be characterized by the $n + 1$th Chern character
\begin{align*}
	\text{ch}_{n + 1}(F) = \frac{1}{(n + 1)!}\tr(\left(\frac{iF}{2\pi}\right)^{n + 1}),
\end{align*}
as well as its integral, called the Chern number. To introduce a similar concept for class AIII, we first introduce the projection matrix $P(k)$ onto the occupied bands and $Q = 1 - 2P$. The chiral symmetry somehow implies that the $Q$ matrix can be written as
\begin{align*}
	Q = \mqty[
		0     & q \\
		q\adj & 0
	],
\end{align*}
with $q$ being a unitary matrix. We can now introduce the winding number for a model in AIII and $d = 2n + 1$ as
\begin{align*}
	\nu_{2n + 1} = \inte{}{}\omega_{2n + 1}(q),\ \omega_{2n + 1}(q) = \frac{(-1)^{n}n!}{(2n + 1)!}\left(\frac{i}{2\pi}\right)^{n + 1}\tr((q^{-1}\df{q})^{2n + 1}).
\end{align*}

Let us move on to a more specific example. The simplest would be starting in $d = 2$ with the Hamiltonian
\begin{align*}
	\ham = k_{x}\sigma_{x} + k_{y}\sigma_{y} + m\sigma_{z}.
\end{align*}
Introducing $\lambda = \sqrt{k^{2} + m^{2}}$, the eigenstates are given by
\begin{align*}
	\ket{u^{-}(k)} = \frac{1}{\sqrt{2\lambda(\lambda + m)}}\mqty[
		-k_{x} + ik_{y} \\
		\lambda + m
	],\ \ket{u^{+}(k)} = \frac{1}{\sqrt{2\lambda(\lambda - m)}}\mqty[
		k_{x} - ik_{y} \\
		\lambda - m
	].
\end{align*}
Writing
\begin{align*}
	\del{}{\mu}f(\sqrt{k^{2} + m^{2}}) = f\p\frac{k_{\mu}}{\sqrt{k^{2} + m^{2}}}
\end{align*}
we find
\begin{align*}
	A_{x} =& \frac{1}{\sqrt{2\lambda(\lambda + m)}}\mqty[
	-k_{x} - ik_{y} & \lambda + m
	]\mqty[
		-\frac{1}{\sqrt{2\lambda(\lambda + m)}} - \frac{(-k_{x} + ik_{y})(4\lambda + 2m)}{2(2\lambda(\lambda + m))^{\frac{3}{2}}}\frac{k_{x}}{\lambda} \\
		-\frac{\frac{m}{2\lambda^{2}}}{2\sqrt{\frac{1}{2} + \frac{m}{2\lambda}}}\frac{k_{x}}{\lambda}
	] \\
	=& \frac{1}{2\lambda(\lambda + m)}\mqty[
	-k_{x} - ik_{y} & \lambda + m
	]\mqty[
	-1 - \frac{(-k_{x} + ik_{y})(2\lambda + m)}{2\lambda(\lambda + m)}\frac{k_{x}}{\lambda} \\
	-\frac{mk_{x}}{2\lambda^{2}}
	] \\
	=& \frac{1}{2\lambda(\lambda + m)}\left(k_{x} + ik_{y} - \frac{k^{2}(2\lambda + m)k_{x}}{2\lambda^{2}(\lambda + m)} - \frac{m(\lambda + m)k_{x}}{2\lambda^{2}}\right) \\
	=& \frac{ik_{y}}{2\lambda(\lambda + m)} + \frac{k_{x}}{4\lambda^{3}(\lambda + m)^{2}}\left(2\lambda^{2}(\lambda + m) - k^{2}(2\lambda + m) - m(\lambda + m)^{2}\right) \\
	=& \frac{ik_{y}}{2\lambda(\lambda + m)} + \frac{k_{x}}{4\lambda^{3}(\lambda + m)^{2}}\left(2\lambda^{3} + 2\lambda^{2}m - 2k^{2}\lambda - k^{2}m - m(\lambda^{2} + m^{2} + 2\lambda m)\right) \\
	=& \frac{ik_{y}}{2\lambda(\lambda + m)}.
\end{align*}
By symmetry we have
\begin{align*}
	A_{y} =& \frac{1}{2\lambda(\lambda + m)}\mqty[
		-k_{x} - ik_{y} & \lambda + m
	]\mqty[
		i - \frac{(-k_{x} + ik_{y})(2\lambda + m)}{2\lambda(\lambda + m)}\frac{k_{y}}{\lambda} \\
		-\frac{mk_{y}}{2\lambda^{2}}
	] \\
	=& -\frac{i}{2\lambda(\lambda + m)}\mqty[
		-k_{x} - ik_{y} & \lambda + m
	]\mqty[
		-1 - \frac{(-k_{x} + ik_{y})(2\lambda + m)}{2\lambda(\lambda + m)}\frac{ik_{y}}{\lambda} \\
		-\frac{mik_{y}}{2\lambda^{2}}
	] \\
	=& -\frac{i}{2\lambda(\lambda + m)}\left(k_{x} + ik_{y} - \frac{k^{2}(2\lambda + m)ik_{y}}{2\lambda^{2}(\lambda + m)} - \frac{m(\lambda + m)ik_{y}}{2\lambda^{2}}\right) \\
	=& -\frac{ik_{x}}{2\lambda(\lambda + m)}.
\end{align*}
The Berry curvature is then
\begin{align*}
	F_{xy} =& -\frac{i}{\lambda(\lambda + m)} + \frac{i(2\lambda + m)}{2\lambda^{3}(\lambda + m)^{2}}(k_{x}^{2} + k_{y}^{2}) \\
	       =& -\frac{i}{\lambda(\lambda + m)}\left(1 - \frac{(2\lambda + m)}{2\lambda^{2}(\lambda + m)}(\lambda^{2} - m^{2})\right) \\
	       =& -\frac{i}{2\lambda^{3}(\lambda + m)}\left(2\lambda^{2} - (\lambda - m)(2\lambda + m)\right) \\
	       =& -\frac{im}{2\lambda^{3}}.
\end{align*}
Approximating the integral over the first Brillouin zone to an integral over all momenta we have
\begin{align*}
	\text{Ch}_{1} =& \frac{i}{2\pi}\inte{}{}F = \frac{m}{4\pi}\inte{}{}\dd[2]{k}\frac{1}{(k^{2} + m^{2})^{\frac{3}{2}}} \\
	              =& \frac{m}{4\pi\abs{m}}\frac{4\pi^{2}\Gamma\left(\frac{1}{2}\right)\Gamma\left(1\right)}{4\pi\Gamma\left(\frac{3}{2}\right)\Gamma\left(1\right)} \\
	              =& \frac{m}{2\abs{m}}\frac{\Gamma\left(\frac{1}{2}\right)}{\Gamma\left(\frac{3}{2}\right)} \\
	              =& \frac{m}{2\abs{m}}.
\end{align*}

\paragraph{Summary of Articles}
The \href{https://journals-aps-org.focus.lib.kth.se/prb/pdf/10.1103/PhysRevB.102.245113}{first article} is about the Berry phase in gapped systems, studied through effective field theory. Promoting parameters of the theory to background fields nets new terms to the effective action, so-called Weiss-Zumino-Witten terms. Considering such terms yields constraints on the phase diagram of the theory, and in particular implies the existence of gapless points which are stable under (some kinds of) deformations. Such points are called diabolical points. For theories with finite degrees of freedom, the existence of such points is identified using the Berry curvature. For field theories one generalization of the Berry phase is the WZW terms appearing in an effective action.

The WZW terms can be related to the ordinary Berry phase by considering interfaces in the model and performing dimensional reduction along these. The result is that the integrals of the Berry curvature are the same for the full model and the effective model along the interface. By that argument these terms also distinguish themselves from the higher Berry curvature, which is mathematically distinct from the usual Berry curvature.

The \href{https://iopscience.iop.org/article/10.1088/1367-2630/12/6/065010/pdf}{second article} is about dimensional reduction of topological insulators. It demonstrates how Hamiltonians that differ with respect to dimensionality and symmetry properties can be related via dimensional reduction.