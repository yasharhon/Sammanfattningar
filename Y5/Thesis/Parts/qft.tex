\section{Quantum Field Theory}

\paragraph{Effective Actions}
To define the effective action, we first introduce
\begin{align*}
	E[J] = -i\ln(Z[J]),
\end{align*}
where $Z$ is the generating functional of some quantum filed theory. $E$ is essentially a measure of the vacuum energy as a function of the source $J$. There is also a strong analogy to statistical mechanics at play, with $Z$ playing the role of the partition function and $E$ the role of the Helmholtz free energy. Its functional derivatives are given by
\begin{align*}
	\fdv{E}{J_{a}(x)} = -\frac{i}{Z}\fdv{Z}{J^{a}(x)} = \frac{\pinte{}{\phi}\phi^{a}(x)e^{i\left(S + \inte{}{}\dd[d]{y}J(y)\phi(y)\right)}}{\pinte{}{\phi}e^{i\left(S + \inte{}{}\dd[d]{y}J(y)\phi(y)\right)}}.
\end{align*}
In analogy with statistical mechanics, this can be considered a classical vacuum expectation value in the presence of a source, hence we term it $\phi_{J}(x)$. Its evaluation at $J = 0$ nets us the familiar correlation function. As for two-point correlators we have
\begin{align*}
	\fdv{J_{a}(x)}\fdv{J_{b}(y)}E =& -i\left(i^{2}\expval{\phi^{a}(x)\phi^{b}(y)}_{J} - i^{2}\expval{\phi^{a}(x)}_{J}\expval{\phi^{b}(y)}_{J}\right).
\end{align*}
This demonstrates the explicit removal of disconnected Feynman diagrams in $E$. The general result is
\begin{align*}
	\left(\prod\limits_{i = 1}^{n}\fdv{J_{a_{i}}(x_{i})}\right)E = -i^{n + 1}\expval{\prod\limits_{i = 1}^{n}\phi^{a_{i}}(x_{i})}_{\text{conn}},
\end{align*}
which will be useful for computing terms in the effective action. This implies that $E$ is a restricted sum of Feynman diagrams, containing only diagrams that are connected.

$E$ is a functional of the currents, but a functional of the fields (or, rather, their expectation values, as that is the closest we're gonna get) might be more of interest. To do this, we first introduce the notation
\begin{align*}
	\eta^{a}(x) = \fdv{E}{J_{a}(x)},
\end{align*}
so as to distinguish between the expectation value in the presence of the source and the actual quantum field. In analogy with statistical mechanics we achieve this with the Legendre transform
\begin{align*}
	\Gamma[\eta] = E[J_{\eta}] - \inte{}{}\dd[d]{x}\eta^{a}(x)J_{a, \eta}(x).
\end{align*}
As we are used to, the above defines $J_{a, \eta}$ as a functional satisfying
\begin{align*}
	\fdv{\Gamma}{\eta^{a}(x)} = -J_{a}(x).
\end{align*}
The quantity $\Gamma$ is the effective action. Note that $E$ and $\Gamma$ coincide for $J = 0$.

We will now try to obtain a method for computing the effective action. To do this we write
\begin{align*}
	e^{i\Gamma[\eta]} = \pinte{}{\phi}e^{i\left(S[\phi] + \inte{}{}\dd[d]{x}\phi^{a}J_{a} - \eta^{a}J_{a}\right)} = \pinte{}{\phi}e^{i\left(S[\phi] + \inte{}{}\dd[d]{x}(\phi^{a} - \eta^{a})J_{a}\right)} = \pinte{}{\chi}e^{i\left(S[\chi + \eta] + \inte{}{}\dd[d]{x}\chi^{a}J_{a}\right)}.
\end{align*}
Next we perform the expansion
\begin{align*}
	S[\eta + \chi] = S[\eta] + S_{2}[\eta, \chi] + S_{\text{int}}[\eta, \chi] + S_{\text{lin}}[\eta, \chi],
\end{align*}
with
\begin{align*}
	S_{2} = \frac{1}{2}\inte{}{}\dd[d]{x}\chi^{a}\Delta_{ab}(\eta)\chi^{b},\ S_{\text{lin}}[\eta, \chi] = \inte{}{}\dd[d]{x}\left(\inte{}{}\dd[d]{y}\eval{\fdv{\lag(x)}{\phi(y)}}_{\phi = \eta} + J_{a}\right)\chi^{a}.
\end{align*}
This yields
\begin{align*}
	e^{i\Gamma[\eta]} = e^{iS[\eta]}\pinte{}{\chi}e^{i\left(S_{\text{lin}} + S_{\text{int}} + \frac{1}{2}\inte{}{}\dd[d]{x}\chi^{a}\Delta_{ab}(\eta)\chi^{b}\right)} = e^{iS[\eta]}\pinte{}{\chi}e^{\frac{i}{2}\inte{}{}\dd[d]{x}\chi^{a}\Delta_{ab}(\eta)\chi^{b}}\sum\limits_{n = 0}^{\infty}\frac{i^{n}}{n!}\left(S_{\text{lin}} + S_{\text{int}}\right)^{n}.
\end{align*}
In other words, the effective action has one term given by the usual action evaluated at the configuration in question, one from the logarithm of the determinant of $\Delta$ and a set of terms obtained from bubble diagrams with $\Delta^{-1}$ as the propagator and $S_{\text{lin}}$ and $S_{\text{int}}$ containing interactions. Now, the above seems to imply the existence of tadpoles in the effective action, but due to the condition on the current these cancel exactly. This means that all Feynman diagrams contributing to the effective action are 1-particle irreducible - that is, cannot be split into two by cutting a single line.

\paragraph{Notes on Convention}
We will be considering fermionic models in arbitrary dimensions, and for this we will need Dirac matrices. These satisfy
\begin{align*}
	\tr(\gamma^{\mu_{1}}\dots\gamma^{\mu_{d + 1}}\gamma_{5}) = -i^{\frac{d - 1}{2}}\left(-2\right)^{\frac{d + 1}{2}}\varepsilon^{\mu_{1}\dots\mu_{d + 1}}
\end{align*}
in odd-dimensional space and
\begin{align*}
	\tr(\gamma^{\mu_{1}}\dots\gamma^{\mu_{d + 1}}) =  -i^{\frac{d}{2} - 1}\left(-2\right)^{\frac{d}{2}}\varepsilon^{\mu_{1}\dots\mu_{d + 1}}
\end{align*}
in even-dimensional space. We will also need hermitian variants of these, found by adding a factor of $-i$, and we then find
\begin{align*}
	\tr(\Gamma_{a_{1}}^{(l)}\dots\Gamma_{a_{l}}^{(l)}) =  i^{\frac{l - 1}{2}}\left(-2\right)^{\frac{l - 1}{2}}\varepsilon_{a_{1}\dots a_{l}}.
\end{align*}

We will also need to perform loop integrals. By performing a Wick rotation we find
\begin{align*}
	\inte{}{}\frac{\dd[d + 1]{k}}{(2\pi)^{d}}\frac{(k^{2})^{a}}{(k^{2} + \Delta)^{b}} = i\frac{\Gamma\left(b - a - \frac{d + 1}{2}\right)\Gamma\left(a + \frac{d + 1}{2}\right)}{(4\pi)^{\frac{d + 1}{2}}\Gamma\left(b\right)\Gamma\left(\frac{d + 1}{2}\right)}\Delta^{a + \frac{d + 1}{2} - b}.
\end{align*}

\paragraph{The Quantum Hall Effect}
We will obtain the quantum Hall effect as an effective theory of a fermionic model given in $2 + 1$ dimensions by
\begin{align*}
	\lag = \overline{\psi}(\fsl{D} + M)\psi,
\end{align*}
with a slowly-varying background gauge field. The effective action is given by
\begin{align*}
	\Gamma = -i\ln(\det(\fsl{D} + M)) = -i\tr(\ln(\fsl{D} + m)) = C_{0} - i\tr(\ln(1 + \frac{i\fsl{A}}{\fsl{\del{}{}} + M})),
\end{align*}
having absorbed the coupling constant into the gauge field. The term in the expansion we are looking for is of order 2, with momentum prescription given in figure \ref{fig:qhe_fd}. Note that the photon lines in this figure are grounded to the background fields.

\begin{figure}[!ht]
	\centering
	\begin{tikzpicture}
		\begin{feynman}
			\vertex (a) at (0, 0);
			\vertex (b) at (2, 0);
			\vertex (c) at (4, 0);
			\vertex (d) at (6, 0);
			\diagram*{
				(a) --[photon, momentum = $p$] (b) --[fermion, half left, momentum = $k$] (c) --[photon, rmomentum = $-p$] (d),
				(c) --[fermion, half left, momentum = $-p + k$] (b),
			};
		\end{feynman}
	\end{tikzpicture}
	\caption{Feynman diagram for the second-order term in the effective action.}
	\label{fig:qhe_fd}
\end{figure}

Thus we have
\begin{align*}
	\Gamma \supset& \frac{i}{2}\inte{}{}\frac{\dd[3]{p}}{(2\pi)^{3}}\frac{\dd[3]{k}}{(2\pi)^{3}}\tr(\frac{1}{i\gamma^{\mu}k_{\mu} + M}\cdot -i\gamma^{\nu}A_{\nu}(-p)\frac{1}{i\gamma^{\rho}(-p + k)_{\rho} + M}\cdot -i\gamma^{\sigma}A_{\sigma}(p)) \\
	             =& -\frac{i}{2}\inte{}{}\frac{\dd[3]{p}}{(2\pi)^{3}}\frac{\dd[3]{k}}{(2\pi)^{3}}\frac{A_{\nu}(-p)A_{\sigma}(p)}{(-k^{2} - M^{2})(-(-p + k)^{2} - M^{2})}\tr((i\gamma^{\mu}k_{\mu} - M)\gamma^{\nu}(i\gamma^{\rho}(-p + k)_{\rho} - M)\gamma^{\sigma}).
\end{align*}
In the long-wavelength limit we truncate all terms of higher order than 1 in $\frac{p}{k}$. This makes the integrand even in $k$, yielding
\begin{align*}
	\Gamma =& -\frac{i}{2}\inte{}{}\frac{\dd[3]{p}}{(2\pi)^{3}}\frac{\dd[3]{k}}{(2\pi)^{3}}\frac{A_{\nu}(-p)A_{\sigma}(p)}{(k^{2} + M^{2})^{2}}\tr((i\gamma^{\mu}k_{\mu} - M)\gamma^{\nu}(i\gamma^{\rho}(-p + k)_{\rho} - M)\gamma^{\sigma}).
\end{align*}
This produces three terms depending on how many Dirac matrices are included in the trace. Including two matrices is trivial, as the loop integral converges in this case. The case of four matrices is of interest, as it produces a divergent term. This case is given by
\begin{align*}
	\Gamma \supset& \frac{i}{2}\inte{}{}\frac{\dd[3]{p}}{(2\pi)^{3}}\frac{\dd[3]{k}}{(2\pi)^{3}}\frac{k_{\mu}A_{\nu}(-p)(-p + k)_{\rho}A_{\sigma}(p)}{(k^{2} + M^{2})^{2}}\tr(\gamma^{\mu}\gamma^{\nu}\gamma^{\rho}\gamma^{\sigma}) \\
	             =& 2i\inte{}{}\frac{\dd[3]{p}}{(2\pi)^{3}}\frac{\dd[3]{k}}{(2\pi)^{3}}\frac{k_{\mu}A_{\nu}(-p)(-p + k)_{\rho}A_{\sigma}(p)}{(k^{2} + M^{2})^{2}}\left(g^{\mu\nu}g^{\rho\sigma} - g^{\mu\rho}g^{\nu\sigma} + g^{\mu\sigma}g^{\nu\rho}\right),
\end{align*}
and the middle term in particular is given by
\begin{align*}
	\Gamma \supset& -2i\inte{}{}\frac{\dd[3]{p}}{(2\pi)^{3}}A^{\mu}(-p)A_{\mu}(p)\inte{}{}\frac{\dd[3]{k}}{(2\pi)^{3}}\frac{k^{2}}{(k^{2} + M^{2})^{2}}.
\end{align*}
This loop integral is divergent, and can be regularized using a Pauli-Villars regularizer. This is done by introducing a new term where the fermion propagator is replaced according to
\begin{align*}
	\frac{1}{i\gamma^{\mu}k_{\mu} + M} \to \frac{1}{i\gamma^{\mu}k_{\mu} + M} - \frac{1}{i\gamma^{\mu}k_{\mu} + \Lambda},
\end{align*}
with $\Lambda$ being some regularizer to be increased to infinity at the end of the calculation. The final term in the effective action, which is the one of interest, is given by
\begin{align*}
	\Gamma \supset& \frac{M}{2}\inte{}{}\frac{\dd[3]{p}}{(2\pi)^{3}}\frac{\dd[3]{k}}{(2\pi)^{3}}\frac{A_{\nu}(-p)p_{\rho}A_{\sigma}(p)}{(k^{2} + M^{2})^{2}}\tr(\gamma^{\nu}\gamma^{\rho}\gamma^{\sigma}) \\
	             =& \frac{iM\Gamma\left(\frac{1}{2}\right)}{8\pi^{\frac{3}{2}}\abs{M}}\inte{}{}\frac{\dd[3]{p}}{(2\pi)^{3}}\varepsilon^{\mu\nu\rho}A_{\mu}(-p)p_{\nu}A_{\rho}(p) \\
	             =& \frac{M}{8\pi\abs{M}}\inte{}{}\dd[3]{x}\varepsilon^{\mu\nu\rho}A_{\mu}\del{}{\nu}A_{\rho}.
\end{align*}
Adding the regularizer yields
\begin{align*}
	\Gamma = \frac{1}{8\pi}\left(\frac{M}{\abs{M}} + \frac{\Lambda}{\abs{\Lambda}}\right)\inte{}{}\dd[3]{x}\varepsilon^{\mu\nu\rho}A_{\mu}\del{}{\nu}A_{\rho}.
\end{align*}
The corresponding current is given by
\begin{align*}
	J^{\mu} = \fdv{\Gamma}{A_{\mu}} = \frac{1}{8\pi}\left(\frac{M}{\abs{M}} - \frac{\Lambda}{\abs{\Lambda}}\right)\left(\varepsilon^{\mu\nu\rho}\del{}{\nu}A_{\rho} - \del{}{\nu}\left(-\varepsilon^{\mu\nu\rho}A_{\rho}\right)\right) = \frac{1}{4\pi}\left(\frac{M}{\abs{M}} - \frac{\Lambda}{\abs{\Lambda}}\right)\varepsilon^{\mu\nu\rho}\del{}{\nu}A_{\rho},
\end{align*}
and in particular, in the limit of $\Lambda \to -\infty$,
\begin{align*}
	J^{i} = \frac{1}{2\pi}\varepsilon^{ij}\left(-\del{}{0}A_{j} + \del{}{j}A_{0}\right) = -\frac{1}{2\pi}\varepsilon^{ij}E_{j},
\end{align*}
which also reproduces the Hall effect.

\paragraph{Theories With Topological Response}
We will now consider some field theories with auxillary fields. The effective actions of these theories contain topological terms, which are metric-independent. The significance of this metric-independence is that correlation functions, and therefore the theory, is stable under deformations of spacetime or generalized coordinate transformations. It also implies that all correlations lengths are zero.

The scheme for constructing such theories was laid out by Abanov and Wiegmann. The idea is to introduce a tuple of slowly varying fields $V$. We also introduce the matrices $\Gamma_{i}^{(2k + 1)}$, which are Hermitian Dirac matrices representing the Clifford algebra with $2k + 1$ generators. From these we construct the operators
\begin{align*}
	m^{(l)} = \begin{cases}
		\sum\limits_{i = 1}^{l}m^{i}\Gamma^{(l)}_{i},\ &l = 2n + 1, \\
		m_{l} + i\gamma_{5}\sum\limits_{i = 1}^{l - 1}m^{i}\Gamma^{(l - 1)}_{i},\ &l = 2n.
	\end{cases}
\end{align*}
$\gamma_{5}$ is $i^{\frac{d - 1}{2}}$ times the product of all Dirac matrices working on Dirac structure, distinguished from the $\Gamma_{i}$, which work in flavor space. Models with mass terms on $S^{d}$ in $d + 1$ dimensions are now given by
\begin{align*}
	\lag = -\overline{\psi}(\fsl{\del{}{}} + Mm^{(d + 1)})\psi,
\end{align*}
and models with mass terms on $S^{d + 1}$ are given by
\begin{align*}
	\lag = -\overline{\psi}(\fsl{\del{}{}} + Mm^{(d + 3)})\psi.
\end{align*}
For this latter case we set $m_{l} = \cos(\nu)$ for some constant $\nu$. Abanov and Wiegmann confine the $m$ to some unit sphere, which we will refrain from doing, and so we can simply set $m_{l} = 1$. These two classes will be termed A and B. We have also introduced an overall mass scale $M$. Note that the confinement to a manifold is performed by the restriction of $m$. The models we will study are particular examples of this construction, as well as some modified versions.

For even $d$ we make the factorization
\begin{align*}
	\lag = -\overline{\psi}\Gamma_{d + 3}^{(d + 3)}(\fsl{\del{}{}}\Gamma_{d + 3}^{(d + 3)} + \Gamma_{d + 3}^{(d + 3)}Mm^{(d + 1)})\psi,
\end{align*}
and redefine the conjugate field as $\overline{\psi}\to\overline{\psi}\Gamma_{d + 3}^{(d + 3)}$, such that
\begin{align*}
	-\overline{\psi}\Gamma_{d + 3}^{(d + 3)}(\fsl{\del{}{}}\Gamma_{d + 3}^{(d + 3)} + \Gamma_{d + 3}^{(d + 3)}Mm^{(d + 1)})\psi.
\end{align*}
The point of this redefinition will be made apparent.

The Dirac equation for these cases is
\begin{align*}
	(\fsl{\del{}{}} + Mm^{(l)})\psi = 0.
\end{align*}
The corresponding Hamiltonian is
\begin{align*}
	\ham = i\alpha^{i}\del{}{i} + i\gamma^{0}Mm^{(l)} = \alpha^{i}p_{i} - \beta Mm^{(l)},
\end{align*}
where the $\alpha^{i}$ are hermitian and satisfy
\begin{align*}
	\acomm{\alpha^{i}}{\alpha^{j}} = 2\kdelta{ij}{},\ \acomm{\alpha^{i}}{\beta} = 0.
\end{align*}
This implies that this Hamiltonian belongs to class AIII.

Let us consider the other metric convention too, with
\begin{align*}
	\lag = \overline{\psi}(i\fsl{\del{}{}} - Mm^{(l)})\psi.
\end{align*}
The Dirac equation is
\begin{align*}
	(i\fsl{\del{}{}} - Mm^{(l)})\psi = 0.
\end{align*}
The corresponding Hamiltonian is given by
\begin{align*}
	\ham = i\alpha^{i}\del{}{i} + \beta Mm^{(l)} = \beta(\alpha^{i}p_{i} + Mm^{(l)}).
\end{align*}

\paragraph{Some Examples}
The first is the $1 + 1$-dimensional theory
\begin{align*}
	\lag = -i\overline{\psi}\left(\fsl{\del{}{}} + M_{1} + iM_{2}\gamma_{5}\right)\psi.
\end{align*}
This is the class-A model in $d = 1$. Noting that
\begin{align*}
	(\gamma_{5})^{2} = 1 \implies e^{i\phi\gamma_{5}} = \cos(\phi) + i\sin(\phi)\gamma_{5},
\end{align*}
we can write
\begin{align*}
	M_{1} + iM_{2}\gamma_{5} = M\left(\cos(\alpha) + i\sin(\alpha)\gamma_{5}\right) = Me^{i\alpha\gamma_{5}},
\end{align*}
with $M$ and $\alpha$ being the magnitude and argument of the complex number $M_{1} + iM_{2}$ (note that hermiticity implies that both parameters be real). Adding the minor modification of coupling the fermion field to a gauge field, the full Lagrangian for this theory is
\begin{align*}
	\lag = -i\overline{\psi}\left(\fsl{D} + Me^{i\alpha\gamma_{5}}\right)\psi - \frac{1}{4}F_{\mu\nu}F^{\mu\nu}.
\end{align*}
To compute the effective action, we integrate out the fermion and perform a perturbation expansion treating $\fsl{D} + Me^{i\alpha\gamma_{5}}$ as a perturbed version of $\fsl{\del{}{}} + M$. The effective action for the gauge field is
\begin{align*}
	\Gamma =& -i\ln(\det(-i\left(\fsl{D} + Me^{i\alpha\gamma_{5}}\right))) + \inte{}{}\dd[2]{x}-\frac{1}{4}F_{\mu\nu}F^{\mu\nu} \\
	       =& -i\tr(\ln(-i\left(\fsl{D} + Me^{i\alpha\gamma_{5}}\right))) + \inte{}{}\dd[2]{x}-\frac{1}{4}F_{\mu\nu}F^{\mu\nu}.
\end{align*}
The trace can be computed by summing over some basis with respect to both the field and Dirac structures. In particular, the new term is
\begin{align*}
	-i\tr(\ln(-i\left(\fsl{D} + Me^{i\alpha\gamma_{5}}\right))) \approx& 	-i\tr(\ln(-i\left(\fsl{\del{}{}} + M\right)\left(1 + \frac{-ie\fsl{A} + iM\alpha\gamma_{5}}{\fsl{\del{}{}} + M}\right))) \\
	=& C_{0} - i\tr(\ln(1 + \frac{-ie\fsl{A} + iM\alpha\gamma_{5}}{\fsl{\del{}{}} + M})) \\
	\approx& C_{0} - i\tr(\frac{-ie\fsl{A} + iM\alpha\gamma_{5}}{\fsl{\del{}{}} + M} - \frac{1}{2}\frac{-ie\fsl{A} + iM\alpha\gamma_{5}}{\fsl{\del{}{}} + M}\frac{-ie\fsl{A} + iM\alpha\gamma_{5}}{\fsl{\del{}{}} + M}).
\end{align*}
Let us first compute the inverse of the denominator. We have
\begin{align*}
	(\fsl{\del{}{}} + M)(\fsl{\del{}{}} - M) = \fsl{\del{}{}}^{2} - M^{2} = \del{2}{} - M^{2} \implies \frac{1}{\fsl{\del{}{}} + M} = \frac{\fsl{\del{}{}} - M}{\del{2}{} - M^{2}}.
\end{align*}
Computing the trace in momentum space, applying the correspondence principle $p = -i\del{}{}$ and using the systematics of Feynman diagrams we find
\begin{align*}
	 &-i\tr(-ie\frac{\fsl{A} + iM\alpha\gamma_{5}}{\fsl{\del{}{}} + M} - \frac{1}{2}\frac{-ie\fsl{A} + iM\alpha\gamma_{5}}{\fsl{\del{}{}} + M}\frac{-ie\fsl{A} + iM\alpha\gamma_{5}}{\fsl{\del{}{}} + M}) \\
	=& -i\inte{}{}\frac{\dd[2]{p_{1}}}{(2\pi)^{2}}\frac{\dd[2]{p_{2}}}{(2\pi)^{2}}\tr(\frac{-ie\fsl{A} + iM\alpha\gamma_{5}}{\fsl{\del{}{}} + M}) + \frac{i}{2}\inte{}{}\frac{\dd[2]{p_{1}}}{(2\pi)^{2}}\frac{\dd[2]{p_{2}}}{(2\pi)^{2}}\frac{\dd[2]{p_{3}}}{(2\pi)^{2}}\frac{\dd[2]{p_{4}}}{(2\pi)^{2}}\tr(\frac{-ie\fsl{A} + iM\alpha\gamma_{5}}{\fsl{\del{}{}} + M}\frac{-ie\fsl{A} + iM\alpha\gamma_{5}}{\fsl{\del{}{}} + M}) \\
	=& -i\inte{}{}\frac{\dd[2]{p_{1}}}{(2\pi)^{2}}\frac{\dd[2]{p_{2}}}{(2\pi)^{2}}\tr(\frac{i\fsl{p_{1}} - M}{-p_{1}^{2} - M^{2}}(-ie\fsl{A}(p_{2}) + iM\alpha(p_{2})\gamma_{5}))(2\pi)^{2}\delta_{p_{1} + p_{2}} \\
	 &+ \frac{i}{2}\inte{}{}\frac{\dd[2]{p_{1}}}{(2\pi)^{2}}\frac{\dd[2]{p_{2}}}{(2\pi)^{2}}\frac{\dd[2]{p_{3}}}{(2\pi)^{2}}\frac{\dd[2]{p_{4}}}{(2\pi)^{2}}\tr(\frac{i\fsl{p_{1}} - M}{-p_{1}^{2} - M^{2}}(-ie\fsl{A}(p_{2}) + iM\alpha(p_{2})\gamma_{5})\frac{i\fsl{p_{3}} - M}{-p_{3}^{2} - M^{2}}(-ie\fsl{A}(p_{4}) + iM\alpha(p_{4})\gamma_{5})) \\
	 &\cdot(2\pi)^{4}\delta_{p_{1} + p_{2} - p_{3}}\delta_{-p_{1} + p_{3} + p_{4}} \\
	=& -i\inte{}{}\frac{\dd[2]{p_{2}}}{(2\pi)^{2}}\tr(\frac{-i\fsl{p_{2}} - M}{-p_{2}^{2} - M^{2}}(-ie\fsl{A}(p_{2}) + iM\alpha(p_{2})\gamma_{5})) \\
	 &+ \frac{i(2\pi)^{2}}{2}\inte{}{}\frac{\dd[2]{p_{1}}}{(2\pi)^{2}}\frac{\dd[2]{p_{2}}}{(2\pi)^{2}}\frac{\dd[2]{p_{4}}}{(2\pi)^{2}} \\
	 &\cdot\tr(\frac{i\fsl{p_{1}} - M}{-p_{1}^{2} - M^{2}}(-ie\fsl{A}(p_{2}) + iM\alpha(p_{2})\gamma_{5})\frac{i(\fsl{p}_{1} + \fsl{p}_{2}) - M}{-(p_{1} + p_{2})^{2} - M^{2}}(-ie\fsl{A}(p_{4}) + iM\alpha(p_{4})\gamma_{5}))\delta_{p_{2} + p_{4}} \\
	=&  -i\inte{}{}\frac{\dd[2]{p_{2}}}{(2\pi)^{2}}\tr(\frac{-i\fsl{p_{2}} - M}{-p_{2}^{2} - M^{2}}(-ie\fsl{A}(p_{2}) + iM\alpha(p_{2})\gamma_{5})) \\
	&+ \frac{i}{2}\inte{}{}\frac{\dd[2]{p_{1}}}{(2\pi)^{2}}\frac{\dd[2]{p_{4}}}{(2\pi)^{2}}\cdot\tr((-ie\fsl{A}(p_{4}) + iM\alpha(p_{4})\gamma_{5})\frac{i\fsl{p_{1}} - M}{-p_{1}^{2} - M^{2}}(-ie\fsl{A}(-p_{4}) + iM\alpha(-p_{4})\gamma_{5})\frac{i(\fsl{p}_{1} - \fsl{p}_{4}) - M}{-(p_{1} - p_{4})^{2} - M^{2}}).
\end{align*}
The corresponding Feynman diagram is shown in figure \ref{fig:second_order_fd}.

\begin{figure}[!ht]
	\centering
		\begin{tikzpicture}
			\begin{feynman}
				\vertex (a) at (0, 0);
				\vertex (b) at (2, 0);
				\vertex (c) at (4, 0);
				\vertex (d) at (6, 0);
				\diagram*{
					(a) --[scalar, momentum = $p$] (b) --[fermion, half left, momentum = $k$] (c) --[scalar, rmomentum = $-p$] (d),
					(c) --[fermion, half left, momentum = $-p + k$] (b),
				};
			\end{feynman}
		\end{tikzpicture}
	\caption{Feynman diagram for the second-order term in the effective action.}
	\label{fig:second_order_fd}
\end{figure}

The second line produces the lowest-order topological terms. There we need only consider the case where the number of Dirac matrices is even, as the odd-numbered cases vanish when tracing the matrices. Absorbing the coupling constant into the gauge field, these terms are
\begin{align*}
	       &\frac{i}{2}\inte{}{}\frac{\dd[2]{p}}{(2\pi)^{2}}\frac{\dd[2]{k}}{(2\pi)^{2}}\tr((-i\fsl{A}(p) + iM\alpha(k)\gamma_{5})\frac{i\fsl{k} - M}{-k^{2} - M^{2}}(-i\fsl{A}(-p) + iM\alpha(-p)\gamma_{5})\frac{i(-\fsl{p} + \fsl{k}) - M}{-(-p + k)^{2} - M^{2}}) \\
	\supset& \frac{iM}{2}\inte{}{}\frac{\dd[2]{p}}{(2\pi)^{2}}\frac{\dd[2]{k}}{(2\pi)^{2}}\tr(\fsl{A}(p)\frac{i\fsl{k} - M}{-k^{2} - M^{2}}\alpha(-p)\gamma_{5}\frac{i(-\fsl{p} + \fsl{k}) - M}{-(-p + k)^{2} - M^{2}}) \\
	       &+ \frac{iM}{2}\inte{}{}\frac{\dd[2]{p}}{(2\pi)^{2}}\frac{\dd[2]{k}}{(2\pi)^{2}}\tr(\alpha(p)\gamma_{5}\frac{i\fsl{k} - M}{-k^{2} - M^{2}}\fsl{A}(-p)\frac{i(-\fsl{p} + \fsl{k}) - M}{-(-p + k)^{2} - M^{2}}) \\
	      =& \frac{iM}{2}\left(\inte{}{}\frac{\dd[2]{p}}{(2\pi)^{2}}\frac{\dd[2]{k}}{(2\pi)^{2}}\alpha(-p)A_{\mu}(p)\tr(\frac{i\fsl{k} - M}{-k^{2} - M^{2}}\gamma^{\mu}\frac{i(-\fsl{p} + \fsl{k}) - M}{-(-p + k)^{2} - M^{2}}\gamma_{5}) \right. \\
	       &+ \left. \inte{}{}\frac{\dd[2]{p}}{(2\pi)^{2}}\frac{\dd[2]{k}}{(2\pi)^{2}}\alpha(p)A_{\mu}(-p)\tr(\frac{i\fsl{k} - M}{-k^{2} - M^{2}}\gamma_{5}\frac{i(-\fsl{p} + \fsl{k}) - M}{-(-p + k)^{2} - M^{2}}\gamma^{\mu})\right) \\
	\supset& \frac{M^{2}}{2}\left(\inte{}{}\frac{\dd[2]{p}}{(2\pi)^{2}}\frac{\dd[2]{k}}{(2\pi)^{2}}\frac{\alpha(-p)A_{\mu}(p)}{(-k^{2} - M^{2})(-(-p + k)^{2} - M^{2})}\tr(\left(\fsl{k}\gamma^{\mu} + \gamma^{\mu}(-\fsl{p} + \fsl{k})\right)\gamma_{5}) \right. \\
	       &+ \left. \frac{\alpha(p)A_{\mu}(-p)}{(-k^{2} - M^{2})(-(-p + k)^{2} - M^{2})}\tr(\left(\fsl{p}\gamma_{5} + \gamma_{5}(-\fsl{p} + \fsl{k})\right)\gamma^{\mu})\right) \\
	      =& \frac{M^{2}}{2}\left(\inte{}{}\frac{\dd[2]{p}}{(2\pi)^{2}}\frac{\dd[2]{k}}{(2\pi)^{2}}\frac{\alpha(-p)A_{\mu}(p)}{(-k^{2} - M^{2})(-(-p + k)^{2} - M^{2})}\tr(k_{\nu}\gamma^{\nu}\gamma^{\mu}\gamma_{5} + (-p + k)_{\nu}\gamma^{\mu}\gamma^{\nu}\gamma_{5}) \right. \\
	       &+ \left. \frac{\alpha(p)A_{\mu}(-p)}{(-k^{2} - M^{2})(-(-p + k)^{2} - M^{2})}\tr(k_{\nu}\gamma^{\nu}\gamma_{5}\gamma^{\mu} + (-p + k)_{\nu}\gamma_{5}\gamma^{\nu}\gamma^{\mu})\right).
\end{align*}
Because we take the mass fields to be slowly varying, we can remove all contributions over order 1 in $\frac{p}{k}$. Next, because we are integrating over $k$ explicitly and due to the Levi-Civita symbol, we may remove contributions proportional to $k_{\mu}$ to any order. Inverting the $p$-integral in the second term leaves us with
\begin{align*}
	 &\frac{M^{2}}{2}\left(\inte{}{}\frac{\dd[2]{p}}{(2\pi)^{2}}\frac{\dd[2]{k}}{(2\pi)^{2}}\frac{\alpha(-p)A_{\mu}(p)}{(-k^{2} - M^{2})^{2}}\tr(k_{\nu}\gamma^{\nu}\gamma^{\mu}\gamma_{5} + (-p + k)_{\nu}\gamma^{\mu}\gamma^{\nu}\gamma_{5} + k_{\nu}\gamma^{\nu}\gamma_{5}\gamma^{\mu} + (p + k)_{\nu}\gamma_{5}\gamma^{\nu}\gamma^{\mu})\right) \\
	=& -M^{2}\inte{}{}\frac{\dd[2]{p}}{(2\pi)^{2}}\frac{\dd[2]{k}}{(2\pi)^{2}}\frac{\alpha(-p)p_{\nu}A_{\mu}(p)}{(-p^{2} - M^{2})^{2}}\tr(\gamma^{\mu}\gamma^{\nu}\gamma_{5}) \\
	=& -2M^{2}\inte{}{}\frac{\dd[2]{p}}{(2\pi)^{2}}\varepsilon^{\mu\nu}\alpha(-p)p_{\nu}A_{\mu}(p)\inte{}{}\frac{\dd[2]{k}}{(2\pi)^{2}}\frac{1}{(k^{2} + M^{2})^{2}}.
\end{align*}
Let us consider the innermost integral. Performing a Wick rotation and a substitution we have
\begin{align*}
	\inte{}{}\frac{\dd[2]{k}}{(2\pi)^{2}}\frac{1}{(k^{2} + M^{2})^{2}} =& \frac{1}{M^{2}}\inte{}{}\frac{\dd[2]{q}}{(2\pi)^{2}}\frac{1}{(-q_{0}^{2} + q_{1}^{2} + 1)^{2}} \\
	=& \frac{i}{M^{2}}\inte{}{}\frac{\dd[2]{\ell}}{(2\pi)^{2}}\frac{1}{(\ell_{0}^{2} + \ell_{1}^{2} + 1)^{2}} \\
	=& \frac{i}{2\pi M^{2}}\inte{0}{\infty}\dd{r}\frac{r}{(r^{2} + 1)^{2}} \\
	=& -\frac{i}{2\pi M^{2}}\eval{\frac{1}{2(r^{2} + 1)}}_{0}^{\infty} \\
	=& \frac{i}{4\pi M^{2}}.
\end{align*}
The final expression for the momentum space effective action is then
\begin{align*}
	\Gamma = -\frac{i}{2\pi}\inte{}{}\frac{\dd[2]{p}}{(2\pi)^{2}}\varepsilon^{\mu\nu}\alpha(-p)p_{\nu}A_{\mu}(p),
\end{align*}
and switching to real space we have
\begin{align*}
	\Gamma = \frac{1}{2\pi}\inte{}{}\dd[2]{x}\varepsilon^{\mu\nu}\alpha\del{}{\mu}A_{\nu} = \frac{1}{2\pi}\inte{}{}\alpha\wedge F.
\end{align*}
The $\alpha$ appearing here is of course only the prefactor for $\gamma_{5}$, and so we can infer the true structure of this term to be
\begin{align*}
	\Gamma = \frac{1}{2\pi}\inte{}{}\sin(\alpha)\wedge F.
\end{align*}

At this point we can also note the existence of terms involving $1 - \cos(\alpha)$, as the full effective action is
\begin{align*}
	\Gamma =& - i\tr(\ln(1 + \frac{1 - \cos(\alpha) - ie\fsl{A} + iM\alpha\gamma_{5}}{\fsl{\del{}{}} + M})).
\end{align*}
As such, the above procedure can be extended to higher order. The result will be terms with arbitrarily high powers of $1 - \cos(\alpha)$, as well as new numerical constants.

Another model to study in one dimension is
\begin{align*}
	\lag = -i\overline{\Psi}\left(\fsl{\del{}{}} + M\left(1 + \sum\limits_{a = 2, 3, 5}im_{a}(x)\gamma^{a}\right)\right)\Psi.
\end{align*}
We once again employ the perturbation approach to write the effective action as
\begin{align*}
	\Gamma =& -i\tr(\ln(-i(\fsl{\del{}{}} + M)\left(1 + \frac{iM\sum\limits_{a = 2, 3, 5},_{a}\gamma^{a}}{\fsl{\del{}{}} + M}\right))) \\
	       =& C_{0} - i\tr(\ln(1 + \frac{i\sum\limits_{a = 2, 3, 5}m_{a}\gamma^{a}}{\fsl{\del{}{}} + M})).
\end{align*}
Because we are working with all Dirac matrices, we note that the only shot at obtaining a topological term is to consider a term of (at least) order three in the expansion. The Feynman diagram is shown in figure \ref{fig:third_order_fd}.

\begin{figure}[!ht]
	\centering
		\begin{tikzpicture}
			\begin{feynman}
				\vertex (a) at (0, 0);
				\vertex (b) at (2, 0);
				\vertex (c) at (4, 1.5);
				\vertex (d) at (6, 3);
				\vertex (e) at (4, -1.5);
				\vertex (f) at (6, -3);
				\diagram*{
					(a) --[scalar, momentum = $p_{1}$] (b) --[fermion, momentum = $k$] (c) --[scalar, rmomentum = $p_{2}$] (d),
					(f) --[scalar, momentum = $-p_{1} - p_{2}$] (e) --[fermion, momentum = $-p_{1} + k$] (b),
					(c) --[fermion, momentum = $p_{2} + k$] (e),
				};
			\end{feynman}
		\end{tikzpicture}
	\caption{Feynman diagram for the third-order term in the effective action.}
	\label{fig:third_order_fd}
\end{figure}

This term is given by
\begin{align*}
\Gamma =& -\frac{iM^{3}}{3}\inte{}{}\frac{\dd[2]{p_{1}}}{(2\pi)^{2}}\frac{\dd[2]{p_{2}}}{(2\pi)^{2}}\frac{\dd[2]{k}}{(2\pi)^{2}}\tr(\frac{i\gamma m^{a}(p_{1})\Gamma_{a}^{(3)}}{\fsl{\del{}{}} + M}\frac{i\gamma m^{b}(p_{2})\Gamma_{b}^{(3)}}{\fsl{\del{}{}} + M}\frac{i\gamma m^{c}(-p_{1} - p_{2})\Gamma_{c}^{(3)}}{\fsl{\del{}{}} + M}) \\
=& -\frac{M^{3}}{3}\inte{}{}\frac{\dd[2]{p_{1}}}{(2\pi)^{2}}\frac{\dd[2]{p_{2}}}{(2\pi)^{2}}\frac{\dd[2]{k}}{(2\pi)^{2}}\frac{m_{a}(p_{1})m_{b}(p_{2})m_{c}(-p_{1} - p_{2})}{(-k^{2} - M^{2})(-(p_{2} + k)^{2} - M^{2})(-(-p_{1} + k)^{2} - M^{2})} \\
&\cdot\tr(\gamma\Gamma_{a}^{(3)}(i\fsl{k} - M)\gamma\Gamma_{b}^{(3)}(i(\fsl{p}_{2} + \fsl{k}) - M)\gamma\Gamma_{c}^{(3)}(i(-\fsl{p}_{1} + \fsl{k}) - M)),
\end{align*}
where a summation convention has now been adopted for the greek indices.

The above calculation can be systematized by adopting aliases for the momenta. Denoting the external momenta from the background as $p_{i}$ and the fermion momentum from the line starting at $p_{i}$ as $k_{i}$, these vectors satisfy.
\begin{align*}
\sum\limits_{i = 1}^{3}p_{i} = 0,\ k_{i} = k_{1} + \sum\limits_{j > 1}^{i}p_{j}
\end{align*}
by construction. The integrand can thus be written as
\begin{align*}
&\frac{m_{a}(p_{1})m_{b}(p_{2})m_{c}(-p_{1} - p_{2})}{(-k^{2} - M^{2})(-(p_{2} + k)^{2} - M^{2})(-(-p_{1} + k)^{2} - M^{2})} \\
&\cdot\tr(\gamma\Gamma_{a}^{(3)}(i\fsl{k} - M)\gamma\Gamma_{b}^{(3)}(i(\fsl{p}_{2} + \fsl{k}) - M)\gamma\Gamma_{c}^{(3)}(i(-\fsl{p}_{1} + \fsl{k}) - M)) \\
=& \frac{m_{a}(p_{1})m_{b}(p_{2})m_{c}(p_{3})}{\prod\limits_{i = 1}^{3}(-k_{i}^{2} - M^{2})}\tr(\gamma\Gamma_{a}^{(3)}(i\fsl{k_{1}} - M)\gamma\Gamma_{b}^{(3)}(i\fsl{k_{2}} - M)\gamma\Gamma_{c}^{(3)}(i\fsl{k_{3}} - M)).
\end{align*}
The topological term corresponds to exactly one of the factors from the fermion loop contributing a factor $-M$. Because $k_{1}$ is explicitly integrated over and the rest of the fermion loop will contribute a factor symmetric in $k_{1}$, the only contribution comes from when the first bracket contributes this factor. Furthermore, anticipating the Levi-Civita symbol in the topological term, any term with higher-order products of components of $k_{1}$ - or indeed, products of components of any momentum vector - can be ignored. These facts combined allow for the replacement
\begin{align*}
k_{i, \mu}k_{j, \nu} \to \sum\limits_{\alpha > 1}^{i}\sum\limits_{\beta > 1, \beta \neq \alpha}^{j}p_{\alpha, \mu}p_{\beta, \nu}.
\end{align*}
A similar relation can be derived for higher-order product. As the isospace and Dirac structure traces may be computed separately, the integrand simplifies to
\begin{align*}
&Mk_{2, \mu}k_{3, \nu}\frac{m_{a}(p_{1})m_{b}(p_{2})m_{c}(p_{3})}{\prod\limits_{i = 1}^{3}(-k_{i}^{2} - M^{2})}\tr(\gamma\Gamma_{a}^{(3)}\gamma\Gamma_{b}^{(3)}\gamma^{\mu}\gamma\Gamma_{c}^{(3)}\gamma^{\nu}) \\
=& -M\frac{m_{a}(p_{1})p_{2, \mu}m_{b}(p_{2})p_{3, \nu}m_{c}(p_{3})}{\prod\limits_{i = 1}^{3}(-k_{i}^{2} - M^{2})}\tr(\gamma^{\mu}\gamma^{\nu}\gamma)\tr(\Gamma_{a}^{(3)}\Gamma_{b}^{(3)}\Gamma_{c}^{(3)}) \\
=& 4iMp_{2, \mu}p_{3, \nu}\varepsilon^{\mu\nu}\varepsilon^{abc}\frac{m_{a}(p_{1})p_{2, \mu}m_{b}(p_{2})p_{3, \nu}m_{c}(p_{3})}{(-k_{1}^{2} - M^{2})(-k_{2}^{2} - M^{2})(-k_{3}^{2} - M^{2})}.
\end{align*}
The effective action in the slowly-varying limit, for which $k_{i} \approx k$ for all $i$, is thus
\begin{align*}
\Gamma =& -\frac{4M^{4}}{3}\frac{1}{8\pi M^{4}}\inte{}{}\frac{\dd[2]{p_{1}}}{(2\pi)^{2}}\frac{\dd[2]{p_{2}}}{(2\pi)^{2}}\varepsilon^{\mu\nu}\varepsilon^{abc}m_{a}(p_{1})p_{2, \mu}m_{b}(p_{2})p_{3, \nu}m_{c}(p_{3}),
\end{align*}
or in real space,
\begin{align*}
	\Gamma =& \frac{1}{6\pi}\sum\limits_{a, b, c = 2, 3, 5}\inte{}{}\dd[2]{x}\varepsilon^{\mu\nu}\varepsilon^{abc}m_{a}\del{}{\mu}m_{b}\del{}{\nu}m_{c}.
\end{align*}

To realize that this is indeed a Weiss-Zumino-Witten term, note that the mass fields comprise a map from spacetime to the target space. Introducing the 2-form $\omega = \frac{1}{6\pi}\varepsilon_{abc}m^{a}\wedge\df m^{b}\wedge\df m^{c}$, the effective action can be written as
\begin{align*}
	\Gamma = \inte{}{}\dd[2]{x}\varepsilon^{\mu\nu}\omega_{ab}\del{}{\mu}m^{a}\del{}{\nu}m^{b} = \inte{}{}\pub{m}{\omega}.
\end{align*}

At this point we can make some notes about the topological terms for all models in class B. In any topological term, all mass fields appear exactly once, as do momenta with each index. The Feynman diagrams also have identical shape, as we will see later. This will produce the pullback of a $d + 1$-form, the components of which are linear in the fields, in the effective action.

Let us reproduce this topological term using the other Lagrangian. We have
\begin{align*}
	\Gamma =& -\frac{iM^{3}}{3}\inte{}{}\frac{\dd[2]{p_{1}}}{(2\pi)^{2}}\frac{\dd[2]{p_{2}}}{(2\pi)^{2}}\frac{\dd[2]{k}}{(2\pi)^{2}}\tr(\frac{i\gamma_{5}\sum\limits_{a = 2, 3, 5}m^{a}\Gamma_{a}^{(3)}}{\fsl{\del{}{}} + M}\frac{i\gamma_{5}\sum\limits_{b = 2, 3, 5}m^{b}\Gamma_{b}^{(3)}}{\fsl{\del{}{}} + M}\frac{i\gamma_{5}\sum\limits_{c = 2, 3, 5}m^{c}\Gamma_{c}^{(3)}}{\fsl{\del{}{}} + M}) \\
	       =& -\frac{M^{3}}{3}\inte{}{}\frac{\dd[2]{p_{1}}}{(2\pi)^{2}}\frac{\dd[2]{p_{2}}}{(2\pi)^{2}}\frac{\dd[2]{k}}{(2\pi)^{2}} \\
	        &\cdot\sum\limits_{a, b, c = 2, 3, 5}m_{a}(p_{1})m_{b}(p_{2})m_{c}(p_{3})\tr(\gamma_{5}\frac{i\fsl{k}_{1} - M}{-k_{1}^{2} - M^{2}}\gamma_{5}\frac{i\fsl{k}_{2} - M}{-k_{2}^{2} - M^{2}}\gamma_{5}\frac{i\fsl{k}_{3} - M}{-k_{3}^{2} - M^{2}})\tr(\Gamma_{a}^{(3)}\Gamma_{b}^{(3)}\Gamma_{c}^{(3)}) \\
	       =& \frac{M^{3}}{3}\inte{}{}\frac{\dd[2]{p_{1}}}{(2\pi)^{2}}\frac{\dd[2]{p_{2}}}{(2\pi)^{2}}\frac{\dd[2]{k}}{(2\pi)^{2}}\sum\limits_{a, b, c = 2, 3, 5}
	       \frac{m_{a}(p_{1})m_{b}(p_{2})m_{c}(p_{3})}{(k_{1}^{2} + M^{2})(k_{2}^{2} + M^{2})(k_{3}^{2} + M^{2})} \\
	        &\cdot\tr(\gamma_{5}(i\fsl{k}_{1} - M)\gamma_{5}(i\fsl{k}_{2} - M)\gamma_{5}(i\fsl{k}_{3} - M))\cdot -2i\varepsilon^{abc} \\
	 \approx& -\frac{2iM^{4}}{3}\inte{}{}\frac{\dd[2]{p_{1}}}{(2\pi)^{2}}\frac{\dd[2]{p_{2}}}{(2\pi)^{2}}\frac{\dd[2]{k}}{(2\pi)^{2}}\sum\limits_{a, b, c = 2, 3, 5}\frac{\varepsilon^{abc}m_{a}(p_{1})m_{b}(p_{2})m_{c}(p_{3})}{(k_{1}^{2} + M^{2})^{3}}\tr(\fsl{k}_{2}\gamma_{5}\fsl{k}_{3}) \\
	       =& -\frac{2M^{4}}{3}\frac{\Gamma\left(2\right)}{4\pi\Gamma\left(3\right)M^{4}}\sum\limits_{a, b, c = 2, 3, 5}\inte{}{}\frac{\dd[2]{p_{1}}}{(2\pi)^{2}}\frac{\dd[2]{p_{2}}}{(2\pi)^{2}}\varepsilon^{abc}m_{a}(p_{1})p_{2, \mu}m_{b}(p_{2})p_{3, \nu}m_{c}(p_{3})\tr(\gamma^{\mu}\gamma^{\nu}\gamma_{5}) \\
	       =& -\frac{1}{12\pi}\sum\limits_{a, b, c = 2, 3, 5}\inte{}{}\frac{\dd[2]{p_{1}}}{(2\pi)^{2}}\frac{\dd[2]{p_{2}}}{(2\pi)^{2}}\varepsilon^{abc}m_{a}(p_{1})p_{2, \mu}m_{b}(p_{2})p_{3, \nu}m_{c}(p_{3})\cdot 2\varepsilon^{\mu\nu} \\
	       =& -\frac{1}{6\pi}\sum\limits_{a, b, c = 2, 3, 5}\inte{}{}\frac{\dd[2]{p_{1}}}{(2\pi)^{2}}\frac{\dd[2]{p_{2}}}{(2\pi)^{2}}\varepsilon^{\mu\nu}\varepsilon^{abc}m_{a}(p_{1})p_{2, \mu}m_{b}(p_{2})p_{3, \nu}m_{c}(p_{3}),
\end{align*}
which in real space is equal to
\begin{align*}
	\Gamma = \frac{1}{6\pi}\sum\limits_{a, b, c = 2, 3, 5}\inte{}{}\dd[2]{x}\varepsilon^{\mu\nu}\varepsilon^{abc}m_{a}\del{}{\mu}m_{b}\del{}{\nu}m_{c}.
\end{align*}

Let us demonstrate the appearance of all mass fields in one dimension. To that end we consider the theory
\begin{align*}
	\lag = -i\overline{\psi}\left(\fsl{\del{}{}} + M_{0} + i\gamma_{5}\sum\limits_{a = 1}^{3}M^{a}\Gamma_{a}^{(3)}\right)\psi.
\end{align*}
The Feynman diagram dictating the term we are looking for is shown in figure \ref{fig:fourth_order_1d_fd}, and the effective action is
\begin{align*}
	\Gamma =& -\frac{i}{5}\inte{}{}\frac{\dd[2]{p_{1}}}{(2\pi)^{2}}\frac{\dd[2]{p_{2}}}{(2\pi)^{2}}\frac{\dd[2]{p_{3}}}{(2\pi)^{2}}\frac{\dd[2]{k}}{(2\pi)^{2}}\tr\left(\frac{M_{0}(p_{2}) - M + i\gamma_{5}\sum\limits_{a = 1}^{3}M^{a}(p_{2})\Gamma_{a}^{(3)}}{i\fsl{k} + M}\frac{M_{0}(p_{3}) - M + i\gamma_{5}\sum\limits_{b = 1}^{3}M^{b}(p_{3})\Gamma_{b}^{(3)}}{i(\fsl{p}_{2} + \fsl{k}) + M} \right. \\
	        &\cdot \left. \frac{M_{0}(p_{4}) - M + i\gamma_{5}\sum\limits_{c = 1}^{3}M^{c}(p_{4})\Gamma_{c}^{(3)}}{i(\fsl{p}_{2} + \fsl{p}_{3} + \fsl{k}) + M}\frac{M_{0}(p_{1}) - M + i\gamma_{5}\sum\limits_{d = 1}^{3}M^{d}(p_{1})\Gamma_{d}^{(3)}}{i(-\fsl{p}_{1} + \fsl{k}) + M}\right).
\end{align*}
The trace part, taking account what we know of these theories, is
\begin{align*}
	&M\tr\left(\left(M_{0}(p_{2}) - M + i\gamma_{5}\sum\limits_{a = 1}^{3}M^{a}(p_{2})\Gamma_{a}^{(3)}\right)(i\fsl{k}_{2} + M)\left(M_{0}(p_{3}) - M + i\gamma_{5}\sum\limits_{b = 1}^{3}M^{b}(p_{3})\Gamma_{b}^{(3)}\right) \right. \\
	&\cdot \left. (i\fsl{k}_{3} + M)\left(M_{0}(p_{4}) - M + i\gamma_{5}\sum\limits_{c = 1}^{3}M^{c}(p_{4})\Gamma_{c}^{(3)}\right)(i\fsl{k}_{4} + M)\left(M_{0}(p_{1}) - M + i\gamma_{5}\sum\limits_{d = 1}^{3}M^{d}(p_{1})\Gamma_{d}^{(3)}\right)\right).
\end{align*}
Depending on the placement of the usual mass field, we get four terms, each of which in turn produce three more. The first is
\begin{align*}
	 &iM^{2}(M_{0}(p_{2}) - M)\sum\limits_{b, c, d = 1}^{3}M^{b}(p_{3})M^{c}(p_{4})M^{d}(p_{1})\tr(\Gamma_{b}^{(3)}\Gamma_{c}^{(3)}\Gamma_{d}^{(3)})(\tr(\fsl{k}_{2}\gamma_{5}\fsl{k}_{3}\gamma_{5}^{2}) + \tr(\fsl{k}_{2}\gamma_{5}^{2}\fsl{k}_{4}\gamma_{5}) + \tr(\gamma_{5}\fsl{k}_{3}\gamma_{5}\fsl{k}_{4}\gamma_{5})) \\
	=& 4M^{2}(M_{0}(p_{2}) - M)\sum\limits_{b, c, d = 1}^{3}\varepsilon_{bcd}M^{b}(p_{3})M^{c}(p_{4})M^{d}(p_{1}) \\
	 &\cdot\varepsilon^{\mu\nu}(-p_{2, \mu}p_{3, \nu} + p_{2, \mu}(p_{3, \nu} + p_{4, \nu}) - p_{2, \mu}(p_{3, \nu} + p_{4, \nu}) - p_{3, \mu}(p_{2, \nu} + p_{4, \nu})) \\
	=& -4M^{2}(M_{0}(p_{2}) - M)\sum\limits_{b, c, d = 1}^{3}\varepsilon_{bcd}M^{b}(p_{3})M^{c}(p_{4})M^{d}(p_{1})\varepsilon^{\mu\nu}p_{3, \mu}p_{4, \nu}.
\end{align*}
The second is
\begin{align*}
	 &iM^{2}(M_{0}(p_{3}) - M)\sum\limits_{a, c, d = 1}^{3}M^{a}(p_{2})M^{c}(p_{4})M^{d}(p_{1})\tr(\Gamma_{a}^{(3)}\Gamma_{c}^{(3)}\Gamma_{d}^{(3)})(\tr(\gamma_{5}\fsl{k}_{2}\fsl{k}_{3}\gamma_{5}^{2}) + \tr(\gamma_{5}\fsl{k}_{2}\gamma_{5}\fsl{k}_{4}\gamma_{5}) + \tr(\gamma_{5}\fsl{k}_{3}\gamma_{5}\fsl{k}_{4}\gamma_{5})) \\
	=& 4M^{2}(M_{0}(p_{3}) - M)\sum\limits_{a, c, d = 1}^{3}\varepsilon_{bcd}M^{b}(p_{2})M^{c}(p_{4})M^{d}(p_{1}) \\
	 &\cdot\varepsilon^{\mu\nu}(p_{2, \mu}p_{3, \nu} - p_{2, \mu}(p_{3, \nu} + p_{4, \nu}) - p_{2, \mu}(p_{3, \nu} + p_{4, \nu}) - p_{3, \mu}(p_{2, \nu} + p_{4, \nu})) \\
	=& -4M^{2}(M_{0}(p_{3}) - M)\sum\limits_{a, c, d = 1}^{3}\varepsilon_{bcd}M^{b}(p_{2})M^{c}(p_{4})M^{d}(p_{1})\varepsilon^{\mu\nu}(2p_{2, \mu}p_{4, \nu} + p_{3, \mu}p_{4, \nu}) \\
	=& -4M^{2}(M_{0}(p_{3}) - M)\sum\limits_{a, c, d = 1}^{3}\varepsilon_{bcd}M^{b}(p_{2})M^{c}(p_{4})M^{d}(p_{1})\varepsilon^{\mu\nu}(2p_{2, \mu}p_{4, \nu} - (p_{1, \mu} + p_{2, \mu})p_{4, \nu}) \\
	=& -4M^{2}(M_{0}(p_{3}) - M)\sum\limits_{a, c, d = 1}^{3}\varepsilon_{bcd}M^{b}(p_{2})M^{c}(p_{4})M^{d}(p_{1})\varepsilon^{\mu\nu}(p_{2, \mu}p_{4, \nu} - p_{1, \mu}p_{4, \nu}).
\end{align*}
Note the rewrite in terms of the momenta appearing within the sum. The third is
\begin{align*}
	 &iM^{2}(M_{0}(p_{4}) - M)\sum\limits_{a, b, d = 1}^{3}M^{a}(p_{2})M^{b}(p_{3})M^{d}(p_{1})\tr(\Gamma_{a}^{(3)}\Gamma_{b}^{(3)}\Gamma_{d}^{(3)})(\tr(\gamma_{5}\fsl{k}_{2}\gamma_{5}\fsl{k}_{3}\gamma_{5}) + \tr(\gamma_{5}\fsl{k}_{2}\gamma_{5}\fsl{k}_{4}\gamma_{5}) + \tr(\gamma_{5}^{2}\fsl{k}_{3}\fsl{k}_{4}\gamma_{5})) \\
	=& 4M^{2}(M_{0}(p_{4}) - M)\sum\limits_{b, c, d = 1}^{3}\varepsilon_{bcd}M^{b}(p_{2})M^{c}(p_{3})M^{d}(p_{1}) \\
	 &\cdot\varepsilon^{\mu\nu}(-p_{2, \mu}p_{3, \nu} - p_{2, \mu}(p_{3, \nu} + p_{4, \nu}) + p_{2, \mu}(p_{3, \nu} + p_{4, \nu}) + p_{3, \mu}(p_{2, \nu} + p_{4, \nu})) \\
	=& 4M^{2}(M_{0}(p_{4}) - M)\sum\limits_{b, c, d = 1}^{3}\varepsilon_{bcd}M^{b}(p_{2})M^{c}(p_{3})M^{d}(p_{1})\varepsilon^{\mu\nu}(-2p_{2, \mu}p_{3, \nu} + p_{3, \mu}p_{4, \nu}) \\
	=& -4M^{2}(M_{0}(p_{4}) - M)\sum\limits_{b, c, d = 1}^{3}\varepsilon_{bcd}M^{b}(p_{2})M^{c}(p_{3})M^{d}(p_{1})\varepsilon^{\mu\nu}(2p_{2, \mu}p_{3, \nu} + p_{3, \mu}(p_{1, \nu} + p_{2, \nu})) \\
	=& -4M^{2}(M_{0}(p_{4}) - M)\sum\limits_{b, c, d = 1}^{3}\varepsilon_{bcd}M^{b}(p_{2})M^{c}(p_{3})M^{d}(p_{1})\varepsilon^{\mu\nu}(p_{2, \mu}p_{3, \nu} + p_{3, \mu}p_{1, \nu}).
\end{align*}
The final term is
\begin{align*}
	 &iM^{2}(M_{0}(p_{1}) - M)\sum\limits_{a, b, c = 1}^{3}M^{a}(p_{2})M^{b}(p_{3})M^{c}(p_{4})\tr(\Gamma_{a}^{(3)}\Gamma_{b}^{(3)}\Gamma_{c}^{(3)})(\tr(\gamma_{5}\fsl{k}_{2}\gamma_{5}\fsl{k}_{3}\gamma_{5}) + \tr(\gamma_{5}\fsl{k}_{2}\gamma_{5}^{2}\fsl{k}_{4}) + \tr(\gamma_{5}^{2}\fsl{k}_{3}\gamma_{5}\fsl{k}_{4})) \\
	=& 4M^{2}(M_{0}(p_{1}) - M)\sum\limits_{a, b, c = 1}^{3}\varepsilon_{abc}M^{a}(p_{2})M^{b}(p_{3})M^{c}(p_{4}) \\
	 &\cdot\varepsilon^{\mu\nu}(-p_{2, \mu}p_{3, \nu} + p_{2, \mu}(p_{3, \nu} + p_{4, \nu}) - p_{2, \mu}(p_{3, \nu} + p_{4, \nu}) - p_{3, \mu}(p_{2, \nu} + p_{4, \nu})) \\
	=& -4M^{2}(M_{0}(p_{1}) - M)\sum\limits_{b, c, d = 1}^{3}\varepsilon_{bcd}M^{b}(p_{2})M^{c}(p_{3})M^{d}(p_{4})\varepsilon^{\mu\nu}p_{3, \mu}p_{4, \nu}.
\end{align*}
Transforming back to real space, these add up to
\begin{align*}
	-&4M^{2}(M_{0} - M)\sum\limits_{b, c, d = 1}^{3}\varepsilon_{bcd}\varepsilon^{\mu\nu}\left(- \del{}{\mu}M^{b}\del{}{\nu}M^{c}M^{d} - \del{}{\mu}M^{b}\del{}{\nu}M^{c}M^{d} + M^{b}\del{}{\nu}M^{c}\del{}{\mu}M^{d} \right. \\
	 &- \left. \del{}{\mu}M^{b}\del{}{\nu}M^{c}M^{d} - M^{b}\del{}{\mu}M^{c}\del{}{\nu}M^{d} - M^{b}\del{}{\mu}M^{c}\del{}{\nu}M^{d}\right) \\
	=& 24M^{2}(M_{0} - M)\sum\limits_{b, c, d = 1}^{3}\varepsilon_{bcd}\varepsilon^{\mu\nu}M^{b}\del{}{\mu}M^{c}\del{}{\nu}M^{d}.
\end{align*}
This can of course only be done in the long-wavelength limit, for which the effective action is
\begin{align*}
	\Gamma =& -\frac{i}{5}\sum\limits_{b, c, d = 1}^{3}\inte{}{}\dd[2]{x}24M^{2}(M_{0} - M)\varepsilon_{bcd}\varepsilon^{\mu\nu}M^{b}\del{}{\mu}M^{c}\del{}{\nu}M^{d}\inte{}{}\frac{\dd[2]{k}}{(2\pi)^{2}}\frac{1}{(-k^{2} - M^{2})^{4}} \\
	       =& \frac{24M^{2}}{5}\frac{\Gamma\left(3\right)\Gamma\left(1\right)}{4\pi\Gamma\left(4\right)\Gamma\left(1\right)}M^{2(1 - 4)}\sum\limits_{b, c, d = 1}^{3}\inte{}{}\dd[2]{x}(M_{0} - M)\varepsilon_{bcd}\varepsilon^{\mu\nu}M^{b}\del{}{\mu}M^{c}\del{}{\nu}M^{d} \\
	       =& \frac{2}{5\pi M^{4}}\sum\limits_{b, c, d = 1}^{3}\inte{}{}\dd[2]{x}(M_{0} - M)\varepsilon_{bcd}\varepsilon^{\mu\nu}M^{b}\del{}{\mu}M^{c}\del{}{\nu}M^{d}.
\end{align*}
This integral is in turn given by the pullback of the 2-form
\begin{align*}
	\omega = \frac{2}{5\pi}(M_{0} - M)\varepsilon_{bcd}M^{b}\df{M^{c}}\df{M^{d}}.
\end{align*}

Let us look more closely at $\omega$ by introducing an explicit parametrization. Namely, we use
\begin{align*}
	m^{2} = \cos(\phi)\sin(\theta)\sin(\alpha),\ m^{3} = \sin(\phi)\sin(\theta)\sin(\alpha),\ m^{5} = \cos(\theta)\sin(\alpha),\ m^{0} = \cos(\alpha).
\end{align*}
The full form of $\omega$ is
\begin{align*}
	\omega = \frac{4}{5\pi}(m^{5}\df{m^{2}}\df{m^{3}} + m^{3}\df{m^{5}}\df{m^{2}} + m^{2}\df{m^{3}}\df{m^{5}}).
\end{align*}
We have
\begin{align*}
	&\df{m^{2}} = c_{\phi}s_{\theta}c_{\alpha}\df{\alpha} + c_{\phi}c_{\theta}s_{\alpha}\df{\theta} - s_{\phi}s_{\theta}s_{\alpha}\df{\phi},\ \df{m^{3}} = s_{\phi}s_{\theta}c_{\alpha}\df{\alpha} + s_{\phi}c_{\theta}s_{\alpha}\df{\theta} + c_{\phi}s_{\theta}s_{\alpha}\df{\phi}, \\
	&\df{m^{5}} = c_{\theta}c_{\alpha}\df{\alpha} - s_{\theta}s_{\alpha}\df{\theta},\ \df{m^{0}} = -s_{\alpha}\df{\alpha},
\end{align*}
adopting an abbreviated trigonometric notation, and thus
\begin{align*}
	\omega =& \frac{4}{5\pi}\left(c_{\theta}s_{\alpha}\left(c_{\phi}s_{\theta}c_{\alpha}\df{\alpha} + c_{\phi}c_{\theta}s_{\alpha}\df{\theta} - s_{\phi}s_{\theta}s_{\alpha}\df{\phi}\right)\left(s_{\phi}s_{\theta}c_{\alpha}\df{\alpha} + s_{\phi}c_{\theta}s_{\alpha}\df{\theta} + c_{\phi}s_{\theta}s_{\alpha}\df{\phi}\right) \right. \\
	        &+ \left. s_{\phi}s_{\theta}s_{\alpha}\left(c_{\theta}c_{\alpha}\df{\alpha} - s_{\theta}s_{\alpha}\df{\theta}\right)\left(c_{\phi}s_{\theta}c_{\alpha}\df{\alpha} + c_{\phi}c_{\theta}s_{\alpha}\df{\theta} - s_{\phi}s_{\theta}s_{\alpha}\df{\phi}\right) \right. \\
	        &+ \left. c_{\phi}s_{\theta}s_{\alpha}\left(s_{\phi}s_{\theta}c_{\alpha}\df{\alpha} + s_{\phi}c_{\theta}s_{\alpha}\df{\theta} + c_{\phi}s_{\theta}s_{\alpha}\df{\phi}\right)\left(c_{\theta}c_{\alpha}\df{\alpha} - s_{\theta}s_{\alpha}\df{\theta}\right)\right) \\
	       =& \frac{4}{5\pi}\left(c_{\theta}s_{\alpha}\left(s_{\theta}^{2}c_{\alpha}s_{\alpha}\df{\alpha}\df{\phi} + c_{\theta}s_{\theta}s_{\alpha}^{2}\df{\theta}\df{\phi}\right) \right. \\
	        &+ \left. s_{\phi}s_{\theta}s_{\alpha}\left(c_{\phi}c_{\alpha}s_{\alpha}\df{\alpha}\df{\theta} - s_{\phi}s_{\theta}c_{\theta}c_{\alpha}s_{\alpha}\df{\alpha}\df{\phi} + s_{\phi}s_{\theta}^{2}s_{\alpha}^{2}\df{\theta}\df{\phi}\right) \right. \\
	        &+ \left. c_{\phi}s_{\theta}s_{\alpha}\left(-s_{\phi}c_{\alpha}s_{\alpha}\df{\alpha}\df{\theta} - c_{\phi}s_{\theta}c_{\theta}s_{\alpha}c_{\alpha}\df{\alpha}\df{\phi} + c_{\phi}s_{\theta}^{2}s_{\alpha}^{2}\df{\theta}\df{\phi}\right)\right) \\
	       =& \frac{4}{5\pi}\left(c_{\theta}s_{\theta}^{2}c_{\alpha}s_{\alpha}^{2}\df{\alpha}\df{\phi} + c_{\theta}^{2}s_{\theta}s_{\alpha}^{3}\df{\theta}\df{\phi} \right. \\
	        &+ \left. s_{\phi}c_{\phi}s_{\theta}c_{\alpha}s_{\alpha}^{2}\df{\alpha}\df{\theta} - s_{\phi}^{2}s_{\theta}^{2}c_{\theta}c_{\alpha}s_{\alpha}^{2}\df{\alpha}\df{\phi} + s_{\phi}^{2}s_{\theta}^{3}s_{\alpha}^{3}\df{\theta}\df{\phi} \right. \\
	        &- \left. c_{\phi}s_{\phi}s_{\theta}c_{\alpha}s_{\alpha}^{2}\df{\alpha}\df{\theta} - c_{\phi}^{2}s_{\theta}^{2}c_{\theta}s_{\alpha}^{2}c_{\alpha}\df{\alpha}\df{\phi} + c_{\phi}^{2}s_{\theta}^{3}s_{\alpha}^{3}\df{\theta}\df{\phi}\right) \\
	       =& 2\sin(\theta)\sin[3](\alpha)\df{\theta}\df{\phi}.
\end{align*}
Adding the other mass field we instead end up with
\begin{align*}
	\omega = 2(\cos(\alpha) - 1)\sin(\theta)\sin[3](\alpha)\df{\theta}\df{\phi}.
\end{align*}

What if Hsin et al are correct in that $\df{\omega} \propto \eta$? The volume form on $S^{3}$ is given by
\begin{align*}
	\eta = \sin[2](\alpha)\sin(\theta)\df{\alpha}\df{\theta}\df{\phi}.
\end{align*}
A naive guess is then
\begin{align*}
	\omega = \sin[2](\alpha)\cos(\theta)\df{\alpha}\df{\phi},
\end{align*}
but this is singular at $\theta = 0$ and $\theta = \pi$. The trick is to remove singularities at either pole - namely, we split $S^{3}$ in two patches on which we take
\begin{align*}
	\omega_{1} = \sin[2](\alpha)(\cos(\theta) - 1)\df{\alpha}\df{\phi},\ \omega_{2} = \sin[2](\alpha)(\cos(\theta) + 1)\df{\alpha}\df{\phi}.
\end{align*}
As we have shifted $\omega$ on either patch by an exact form, $\omega$ still satisfies the same property. We then have
\begin{align*}
	\Gamma = \omega_{1} - \omega_{2} = -2\sin[2](\alpha)\df{\alpha}\df{\phi},
\end{align*}
and
\begin{align*}
	\inte{\gamma}{}\Gamma =& -2\inte{0}{2\pi}\inte{0}{\pi}\dd{\alpha}\dd{\phi}\sin[2](\alpha) \\
	                      =& -2\pi,
\end{align*}
with $\gamma$ being some cycle at fixed $\theta$.

What if we were to add coupling to a gauge field too? The effective action would be
\begin{align*}
	\Gamma =& -i\tr(\ln(1 + \frac{-i\fsl{A} + i\sum\limits_{a = 2, 3, 5}M_{a}\gamma^{a}}{\fsl{\del{}{}} + M})).
\end{align*}
To order $3$, the structure of the Feynman diagram is identical, hence the trace part of the effective action is
\begin{align*}
	-i\tr((i\fsl{k_{1}} - M)(e\fsl{A} + M_{a}\gamma^{a})(i(-\fsl{k_{1}} - \fsl{p}) - M)(e\fsl{A} + M_{b}\gamma^{a})(i(\fsl{k_{2}} - \fsl{k_{1}}) - M)(e\fsl{A} + M_{c}\gamma^{c})).
\end{align*}
Because all of the latin-indiced Dirac matrices are needed, however, the gauge field does not appear in any topological terms. To get a topological term with a gauge field, we can instead consider a fourth-order term, represented by the Feynman diagram in figure \ref{fig:fourth_order_1d_fd}.

\begin{figure}[!ht]
	\centering
	\begin{tikzpicture}
		\begin{feynman}
			\vertex (a) at (-4, 4);
			\vertex (a1) at (-2, 2);
			\vertex (b) at (4, 4);
			\vertex (b1) at (2, 2);
			\vertex (c) at (4, -4);
			\vertex (c1) at (2, -2);
			\vertex (d) at (-4, -4);
			\vertex (d1) at (-2, -2);
			\diagram*{
				(a) --[scalar, momentum = $p_{1}$] (a1) --[fermion, momentum = $k$] (b1) --[scalar, rmomentum = $p_{2}$] (b),
				(c) --[scalar, momentum = $p_{3}$] (c1) --[fermion, momentum = $p_{2} + p_{3} + k$] (d1) --[scalar, rmomentum = $-p_{1} - p_{2} - p_{3}$] (d),
				(b1) --[fermion, momentum = $p_{2} + k$] (c1),
				(d1) --[fermion, momentum = $-p_{1} + k$] (a1),
			};
		\end{feynman}
	\end{tikzpicture}
	\caption{Feynman diagram for the fourth-order term in the effective action.}
	\label{fig:fourth_order_1d_fd}
\end{figure}

At this point we can deduce what will happen. The replacement of a momentum with a gauge field in the trace (which is needed to produce a topological term) while leaving the same number of mass fields causes the integrand to be symmetric in the mass indices and makes the topological term vanish.

This argument also implies the necessary conditions for the gauge fields to appear in response terms. We expect all the fields corresponding to anticommuting mass terms to appear exactly once, as well as all the derivatives. Removing a derivative from a mass field nets you the possibility of creating a symmetric expression in mass indices. In order for the gauge fields to appear in topological terms, the spacetime dimension must therefore be equal to the number of mass fields. If it is greater, the spacetime Levi-Civita will cancel the derivative product. If it is lesser, the mass index Levi-Civita will cancel the now symmetric product of mass fields.

A counterpoint to the above is that this model does not explicitly apply transformations in isospace, forcing out the Levi-Civita symbol, but that this could be avoided by sticking more rigorously to Abanov and Wiegmann's method. The fact that the Dirac matrices square to the identity then implies that there would be some topological terms that are pullbacks of symmetric tensors on the mass manifold. Of course, these all vanish because the topological terms necessarily contain a Levi-Civita symbol on spacetime, and as such only fully antisymmetric tensors on the mass manifold produce topological terms. This legitimizes the above approach.

Let us also consider two models in two dimensions. The first is
\begin{align*}
	\lag = -i\overline{\psi}\left(\fsl{D} + \sum\limits_{a = 1}^{3}M^{a}\Gamma_{a}^{(3)}\right)\psi.
\end{align*}
The effective action is
\begin{align*}
	\Gamma = -i\tr(\ln(1 + \frac{-M - i\fsl{A} + \sum\limits_{a = 1}^{3}M^{a}\Gamma_{a}^{(3)}}{\fsl{\del{}{}} + M})).
\end{align*}
The topological term, which contains the gauge field, corresponds to figure \ref{fig:fourth_order_1d_fd} (to lowest order). We may drop the term proportional to the identity, and the effective action is given by
\begin{align*}
	\Gamma =& \frac{i}{4}\inte{}{}\frac{\dd[3]{p_{1}}}{(2\pi)^{3}}\frac{\dd[3]{p_{2}}}{(2\pi)^{3}}\frac{\dd[3]{p_{3}}}{(2\pi)^{3}}\frac{\dd[3]{k}}{(2\pi)^{3}}\tr\left(\frac{-i\fsl{A} + \sum\limits_{a = 1}^{3}M^{a}\Gamma_{a}^{(3)}}{\fsl{\del{}{}} + M}\frac{-i\fsl{A} + \sum\limits_{b = 1}^{3}M^{b}\Gamma_{b}^{(3)}}{\fsl{\del{}{}} + M}\frac{-i\fsl{A} + \sum\limits_{c = 1}^{3}M^{c}\Gamma_{c}^{(3)}}{\fsl{\del{}{}} + M}\frac{-i\fsl{A} + \sum\limits_{d = 1}^{3}M^{d}\Gamma_{d}^{(3)}}{\fsl{\del{}{}} + M}\right) \\
	=& \frac{i}{4}\inte{}{}\frac{\dd[3]{p_{1}}}{(2\pi)^{3}}\frac{\dd[3]{p_{2}}}{(2\pi)^{3}}\frac{\dd[3]{p_{3}}}{(2\pi)^{3}}\frac{\dd[3]{k}}{(2\pi)^{3}}\tr\left(\left(-i\fsl{A}(p_{1}) + \sum\limits_{a = 1}^{3}M^{a}(p_{1})\Gamma_{a}^{(3)}\right)\frac{i\fsl{k} - M}{-k^{2} - M^{2}}\right. \\
	 &\left.\cdot\left(-i\fsl{A}(p_{2}) + \sum\limits_{b = 1}^{3}M^{b}(p_{2})\Gamma_{b}^{(3)}\right)\frac{i(\fsl{p}_{2} + \fsl{k}) - M}{-(p_{2} + k)^{2} - M^{2}}\right. \\
	 &\left.\cdot\left(-i\fsl{A}(p_{3}) + \sum\limits_{b = 1}^{3}M^{b}(p_{3})\Gamma_{b}^{(3)}\right)\frac{i(\fsl{p}_{2} + \fsl{p}_{3} + \fsl{k}) - M}{-(p_{2} + p_{3} + k)^{2} - M^{2}}\right. \\
	 &\left.\cdot\left(-i\fsl{A}(-p_{1} - p_{2} - p_{3}) + \sum\limits_{c = 1}^{3}M^{d}(-p_{1} - p_{2} - p_{3})\Gamma_{d}^{(3)}\right)\frac{i(-\fsl{p}_{1} + \fsl{k}) - M}{-(-p_{1} + k)^{2} - M^{2}}\right).
\end{align*}
Let us consider the trace part first. The relevant terms have exactly one gauge field appearing and two momenta. Adopting a shorthand, the first term of the trace can be written as
\begin{align*}
	 &-i\sum\limits_{b, c, d = 1}^{3}M^{b}(p_{2})M^{c}(p_{3})M^{d}(p_{4})\tr(\fsl{A}(p_{1})(i\fsl{k}_{1} - M)\Gamma_{b}^{(3)}(i\fsl{k}_{2} - M)\Gamma_{c}^{(3)}(i\fsl{k}_{3} - M)\Gamma_{d}^{(3)}(i\fsl{k}_{4} - M)) \\
	=& -i\sum\limits_{b, c, d = 1}^{3}M^{b}(p_{2})M^{c}(p_{3})M^{d}(p_{4})\tr(A_{\mu}(p_{1})\gamma^{\mu}(i\fsl{k}_{1} - M)(i\fsl{k}_{2} - M)(i\fsl{k}_{3} - M)(i\fsl{k}_{4} - M))\tr(\Gamma_{b}^{(3)}\Gamma_{c}^{(3)}\Gamma_{d}^{(3)}) \\
	=& -2M\sum\limits_{a, b, c = 1}^{3}\varepsilon_{abc}M^{a}(p_{2})M^{b}(p_{3})M^{c}(p_{4})\tr(A_{\mu}(p_{1})\gamma^{\mu}(i\fsl{k}_{2} - M)(i\fsl{k}_{3} - M)(i\fsl{k}_{4} - M)).
\end{align*}
This is the point at which we can still generalize. At a glance it might seem like we could do it all in one go by fixing the position of the gauge field and cyclically permuting the momenta, but this is not the case due to the loop having been assigned a set of momenta. Using the previously established properties of the shorthand, we write the trace as
\begin{align*}
	\tr(\gamma^{\mu}(i\fsl{k}_{2} - M)(i\fsl{k}_{3} - M)(i\fsl{k}_{4} - M)) =& M\tr(\gamma^{\mu}\fsl{k}_{2}\fsl{k}_{3} + \gamma^{\mu}\fsl{k}_{2}\fsl{k}_{4} + \gamma^{\mu}\fsl{k}_{3}\fsl{k}_{4}) \\
	=& 2M\varepsilon^{\mu\nu\rho}\left(k_{2, \nu}k_{3, \rho} + k_{2, \nu}k_{4, \rho} + k_{3, \nu}k_{4, \rho}\right) \\
	=& 2M\varepsilon^{\mu\nu\rho}\left(p_{2, \nu}p_{3, \rho} + p_{2, \nu}(p_{3, \rho} + p_{4, \rho}) + p_{2, \nu}(p_{3, \rho} + p_{4, \rho}) + p_{3, \nu}(p_{2, \rho} + p_{4, \rho})\right) \\
	=& 2M\varepsilon^{\mu\nu\rho}\left(2p_{2, \nu}(p_{3, \rho} + p_{4, \rho}) + p_{3, \nu}p_{4, \rho}\right).
\end{align*}
The next is identical to the previous due to the $k_{1}$ not contributing. Third is
\begin{align*}
	\tr((i\fsl{k}_{2} - M)\gamma^{\mu}(i\fsl{k}_{3} - M)(i\fsl{k}_{4} - M)) =& M\tr(\fsl{k}_{2}\gamma^{\mu}\fsl{k}_{3} + \gamma^{\mu}\fsl{k}_{3}\fsl{k}_{4} + \fsl{k}_{2}\gamma^{\mu}\fsl{k}_{4}) \\
	=& 2M\varepsilon^{\mu\nu\rho}(-k_{2, \nu}k_{3, \rho} + k_{3, \nu}k_{4, \rho} - k_{2, \nu}k_{4, \rho}) \\
	=& 2M\varepsilon^{\mu\nu\rho}(-p_{2, \nu}p_{3, \rho} + p_{2, \nu}(p_{3, \rho} + p_{4, \rho}) + p_{3, \nu}(p_{2, \rho} + p_{4, \rho}) - p_{2, \nu}(p_{3, \rho} + p_{4, \rho})) \\
	=& 2M\varepsilon^{\mu\nu\rho}(p_{3, \nu}p_{4, \rho} - 2p_{2, \nu}p_{3, \rho}).
\end{align*}
Finally there is
\begin{align*}
	\tr((i\fsl{k}_{2} - M)(i\fsl{k}_{3} - M)\gamma^{\mu}(i\fsl{k}_{4} - M)) =& M\tr(\fsl{k}_{2}\fsl{k}_{3}\gamma^{\mu} + \fsl{k}_{2}\gamma^{\mu}\fsl{k}_{4} + \fsl{k}_{3}\gamma^{\mu}\fsl{k}_{4}) \\
	=& 2M\varepsilon^{\mu\nu\rho}\left(k_{2, \nu}k_{3, \rho} - k_{2, \nu}k_{4, \rho} - k_{3, \nu}k_{4, \rho}\right) \\
	=& 2M\varepsilon^{\mu\nu\rho}\left(p_{2, \nu}p_{3, \rho} - p_{2, \nu}(p_{3, \rho} + p_{4, \rho}) - p_{2, \nu}(p_{3, \rho} + p_{4, \rho}) - p_{3, \nu}(p_{2, \rho} + p_{4, \rho})\right) \\
	=& 2M\varepsilon^{\mu\nu\rho}\left(-2p_{2, \nu}p_{4, \rho} - p_{3, \nu}p_{2, \rho} - p_{3, \nu}p_{4, \rho}\right).
\end{align*}
At this point we may integrate $k$ out, and the rest of the effective action is
\begin{align*}
	\frac{i}{4}\inte{}{}\frac{\dd[3]{k}}{(2\pi)^{3}}\frac{1}{(-k^{2} - M^{2})^{4}} =& -\frac{\Gamma\left(4 - 2\right)\Gamma\left(2\right)}{64\pi^{2}\Gamma\left(4\right)\Gamma\left(2\right)}M^{2\left(2 - 4\right)} = -\frac{1}{384\pi^{2}M^{4}}.
\end{align*}
All that remains is sorting out the terms from the trace. Each one is preceded by a factor $-4M^{2}\varepsilon^{\mu\nu\rho}\varepsilon_{abc}$, and the correspondence principle translates the remainder to
\begin{align*}
	   &A_{\mu}(p_{1})M^{a}(p_{2})M^{b}(p_{3})M^{c}(p_{4})\left(2p_{2, \nu}(p_{3, \rho} + p_{4, \rho}) + p_{3, \nu}p_{4, \rho}\right) \\
	\to& -2A_{\mu}\del{}{\nu}M^{a}(M^{b}\del{}{\rho}M^{c} + \del{}{\rho}M^{b}M^{c}) - A_{\mu}M^{a}\del{}{\nu}M^{b}\del{}{\rho}M^{c}, \\
	   &M^{a}(p_{1})A_{\mu}(p_{2})M^{b}(p_{3})M^{c}(p_{4})\left(2p_{2, \nu}(p_{3, \rho} + p_{4, \rho}) + p_{3, \nu}p_{4, \rho}\right) \\
	\to& -2M^{a}\del{}{\nu}A_{\mu}(\del{}{\rho}M^{b}M^{c} + M^{b}\del{}{\rho}M^{c}) - M^{a}A_{\mu}\del{}{\nu}M^{b}\del{}{\rho}M^{c}, \\
	   &M^{a}(p_{1})M^{b}(p_{2})A_{\mu}(p_{3})M^{c}(p_{4})(p_{3, \nu}p_{4, \rho} - 2p_{2, \nu}p_{3, \rho}) \\
	\to& -M^{a}M^{b}\del{}{\nu}A_{\mu}\del{}{\rho}M^{c} + 2M^{a}\del{}{\nu}M^{b}\del{}{\rho}A_{\mu}M^{c}, \\
	   &M^{a}(p_{1})M^{b}(p_{2})M^{c}(p_{3})A_{\mu}(p_{4})\left(-2p_{2, \nu}p_{4, \rho} - p_{3, \nu}p_{2, \rho} - p_{3, \nu}p_{4, \rho}\right) \\
	\to& 2M^{a}\del{}{\nu}M^{b}M^{c}\del{}{\rho}A_{\mu} + M^{a}\del{}{\rho}M^{b}\del{}{\nu}M^{c}A_{\mu} + M^{a}M^{b}\del{}{\nu}M^{c}\del{}{\rho}A_{\mu}.
\end{align*}
The only terms surviving contraction with the mass field Levi-Civita are those with two derivatives on the mass fields. The final effective action is then
\begin{align*}
	\Gamma =& -\frac{1}{96\pi^{2}}\varepsilon^{\mu\nu\rho}\varepsilon_{abc}\inte{}{}\dd[3]{x}A_{\mu}\left(m^{a}\del{}{\rho}m^{b}\del{}{\nu}m^{c} - m^{a}\del{}{\nu}m^{b}\del{}{\rho}m^{c} - 2\del{}{\nu}m^{a}(m^{b}\del{}{\rho}m^{c} + \del{}{\rho}m^{b}m^{c}) - m^{a}\del{}{\nu}m^{b}\del{}{\rho}m^{c}\right) \\
	=& \frac{1}{32\pi}\varepsilon^{\mu\nu\rho}\varepsilon_{abc}\inte{}{}\dd[3]{x}A_{\mu}m^{a}\del{}{\rho}m^{b}\del{}{\nu}m^{c} \\
	=& \frac{1}{32\pi}\inte{}{}\dd[3]{x}A\wedge\omega,
\end{align*}
with $\omega = \varepsilon_{abc}m^{a}\df{m^{b}}\df{m^{c}}$.

The case for $d = 2$ is given by
\begin{align*}
\lag_{2 + 1} = -\overline{\Psi}\left(\fsl{\del{}{}}\Gamma_{a}^{(5)} + M\left(1 + \sum\limits_{a = 1}^{4}m^{a}\Gamma_{5}^{(5)}\Gamma_{a}^{(5)}\right)\right)\Psi.
\end{align*}
The masses are now confined to $S^{3}$.

The Feynman diagram corresponding to the topological term is given in figure \ref{fig:fourth_order_fd}.

\begin{figure}[!ht]
	\centering
	\begin{tikzpicture}
	\begin{feynman}
	\vertex (a) at (-4, 4);
	\vertex (a1) at (-2, 2);
	\vertex (b) at (4, 4);
	\vertex (b1) at (2, 2);
	\vertex (c) at (4, -4);
	\vertex (c1) at (2, -2);
	\vertex (d) at (-4, -4);
	\vertex (d1) at (-2, -2);
	\diagram*{
		(a) --[scalar, momentum = $p_{1}$] (a1) --[fermion, momentum = $k$] (b1) --[scalar, rmomentum = $p_{2}$] (b),
		(c) --[scalar, momentum = $p_{3}$] (c1) --[fermion, momentum = $p_{2} + p_{3} + k$] (d1) --[scalar, rmomentum = $-p_{1} - p_{2} - p_{3}$] (d),
		(b1) --[fermion, momentum = $p_{2} + k$] (c1),
		(d1) --[fermion, momentum = $-p_{1} + k$] (a1),
	};
	\end{feynman}
	\end{tikzpicture}
	\caption{Feynman diagram for the fourth-order term in the effective action.}
	\label{fig:fourth_order_fd}
\end{figure}

This term is given by
\begin{align*}
\Gamma \supset& \frac{iM^{4}}{4}\inte{}{}\frac{\dd[3]{p_{1}}}{(2\pi)^{3}}\frac{\dd[3]{p_{2}}}{(2\pi)^{3}}\frac{\dd[3]{p_{3}}}{(2\pi)^{3}}\frac{\dd[3]{k}}{(2\pi)^{3}}\tr\left(\frac{m^{a}\Gamma_{5}^{(5)}\Gamma_{a}^{(5)}}{\fsl{\del{}{}} + M}\frac{m^{b}\Gamma_{5}^{(5)}\Gamma_{b}^{(5)}}{\fsl{\del{}{}} + M}\frac{m^{c}\Gamma_{5}^{(5)}\Gamma_{c}^{(5)}}{\fsl{\del{}{}} + M}\frac{m^{d}\Gamma_{5}^{(5)}\Gamma_{d}^{(5)}}{\fsl{\del{}{}} + M}\right) \\
=& \frac{iM^{4}}{4}\inte{}{}\frac{\dd[3]{p}}{(2\pi)^{3}}\frac{\dd[3]{k}}{(2\pi)^{3}}\frac{m^{a}(p_{1})m^{b}(p_{2})m^{c}(p_{3})m^{d}(p_{4})}{\prod\limits_{i = 1}^{4}(-k_{i}^{2} - M^{2})}\tr\left(\Gamma_{5}^{(5)}\Gamma_{a}^{(5)}(i\gamma^{\mu}\Gamma_{5}^{(5)}k_{1, \mu} - M)\Gamma_{5}^{(5)}\Gamma_{b}^{(5)} \right. \\
&\cdot \left. (i\gamma^{\nu}\Gamma_{5}^{(5)}k_{2, \nu} - M)\Gamma_{5}^{(5)}\Gamma_{c}^{(5)}(i\gamma^{\rho}\Gamma_{5}^{(5)}k_{3, \rho} - M)\Gamma_{5}^{(5)}\Gamma_{d}^{(5)}(i\gamma^{\sigma}\Gamma_{5}^{(5)}k_{4, \sigma} - M)\right),
\end{align*}
where the previously introduced aliases have been adopted. Using similar arguments as for $d = 1$, this can be simplified to
\begin{align*}
	\Gamma \supset& -\frac{M^{5}}{4}\inte{}{}\frac{\dd[3]{p}}{(2\pi)^{3}}\frac{\dd[3]{k}}{(2\pi)^{3}}k_{2, \mu}k_{3, \nu}k_{4, \rho}\frac{m^{a}(p_{1})m^{b}(p_{2})m^{c}(p_{3})m^{d}(p_{4})}{\prod\limits_{i = 1}^{4}(-k_{i}^{2} - M^{2})} \\
&\cdot\tr(\Gamma_{5}^{(5)}\Gamma_{a}^{(5)}\Gamma_{5}^{(5)}\Gamma_{b}^{(5)}\gamma^{\mu}\Gamma_{5}^{(5)}\Gamma_{5}^{(5)}\Gamma_{c}^{(5)}\gamma^{\nu}\Gamma_{5}^{(5)}\Gamma_{5}^{(5)}\Gamma_{d}^{(5)}\gamma^{\rho}\Gamma_{5}^{(5)}) \\
=& -\frac{M^{5}}{4}\inte{}{}\frac{\dd[3]{p}}{(2\pi)^{3}}\frac{\dd[3]{k}}{(2\pi)^{3}}k_{2, \mu}k_{3, \nu}k_{4, \rho}\frac{m^{a}(p_{1})m^{b}(p_{2})m^{c}(p_{3})m^{d}(p_{4})}{\prod\limits_{i = 1}^{4}(k_{i}^{2} + M^{2})} \\
&\cdot\tr(\gamma^{\mu}\gamma^{\nu}\gamma^{\rho})\tr(\Gamma_{5}^{(5)}\Gamma_{a}^{(5)}\Gamma_{5}^{(5)}\Gamma_{b}^{(5)}\Gamma_{c}^{(5)}\Gamma_{d}^{(5)}\Gamma_{5}^{(5)}) \\
=& -\frac{i}{32\pi}\inte{}{}\frac{\dd[3]{p}}{(2\pi)^{3}}\varepsilon^{\mu\nu\rho}\varepsilon^{abcd}m_{a}(p_{1})p_{2, \mu}m_{b}(p_{2})p_{3, \nu}m_{c}(p_{3})p_{4, \rho}m_{d}(p_{4}) \\
=& \frac{1}{32\pi}\inte{}{}\dd[3]{x}\varepsilon^{\mu\nu\rho}\varepsilon^{abcd}m_{a}\del{}{\mu}m_{b}\del{}{\nu}m_{c}\del{}{\rho}m_{d}.
\end{align*}
This is also a Weiss-Zumino-Witten term corresponding to a $3$-form
\begin{align*}
	\omega = \frac{1}{32\pi}\varepsilon_{abcd}m^{a}\wedge\df m^{b}\wedge\df m^{c}\wedge\df m^{d}.
\end{align*}

\paragraph{A New Model}
The models studied above have connected $d$-dimensional systems to response terms described by integrals of up to $d + 1$-forms. To relate the study of these theories to the higher Berry curvature, we will instead need to find a response term described by the integral of a $d + 2$-form. Let us consider spatial dimension $1$ and introduce mass terms according to
\begin{align*}
	M = M_{0} + i\gamma^{01}\sum\limits_{l = 1}^{5}M_{l}\Gamma^{l}.
\end{align*}
The matrices $\Gamma$ represent a Clifford algebra and act in flavor space, and we take the mass fields to lie on $S^{5}$. The Lagrangian of the model is
\begin{align*}
	\lag = -i\overline{\Psi}\left(\fsl{\del{}{}} + M\right)\Psi = -i\overline{\Psi}\left(\fsl{\del{}{}} + M_{0} + i\gamma^{01}\sum\limits_{l = 1}^{5}M_{l}\Gamma^{l}\right)\Psi.
\end{align*}

The idea underlying this model follows \href{https://arxiv.org/pdf/hep-th/9911025.pdf}{Abanov and Wiegmann}. They construct models with mass fields confined to $S^{d}$ or $S^{d + 1}$ and show that the topological response terms are related to (the pullbacks of) $d$-forms and $d + 1$-forms respectively. A first obvious attempt in one dimension is therefore to use mass fields on $S^{3}$ (and the simple way to do this just so happens to be Abanov and Wiegmann's A-series model in $d = 3$), but we saw that this only produced a topological term given by a 2-form. This model attempts to fix this by extending the mass fields in a way such that, had you done it in $d = 3$, it would take the response term from being given by a 3-form to a 4-form. The hope is that it will achieve a similar result.

At this point it is pertinent to ask whether this attempt really stood any chance. The answer is no, and for a very simple reason. Looking at the above, the appearance of the pullback was no coincidence; it arrived precisely because of the form of the effective action. By its definition the pullback does not affect the rank of any tensor. As the effective action is given by an integral over spacetime, it follows that any form appearing in it must exist on spacetime, and the highest form in $d + 1$-dimensional spacetime is a $d + 1$-form. Note that this argument has no reliance on the structure of the mass fields. Nevertheless, we show the attempt below.

The effective action is
\begin{align*}
	\Gamma =& -i\ln(\det(-i\left(\fsl{\del{}{}} + M_{0} + i\gamma^{01}\sum\limits_{l = 1}^{5}M_{l}\Gamma^{l}\right))) \\
	       =& -i\tr(\ln(-i\left(\fsl{\del{}{}} + M_{0} + i\gamma^{01}\sum\limits_{l = 1}^{5}M_{l}\Gamma^{l}\right))) \\
	       =& C_{0} - i\tr(\ln(1 + \frac{m + i\gamma^{01}\sum\limits_{l = 1}^{5}M_{l}\Gamma^{l}}{\fsl{\del{}{}} + M})) \\
	       =& C_{0} - i\tr(\ln(1 + \frac{(\fsl{\del{}{}} - M)\left(m + i\gamma^{01}\sum\limits_{l = 1}^{5}M_{l}\Gamma^{l}\right)}{\del{2}{} - M^{2}})),
\end{align*}
where we have introduced the mass perturbation $m = M_{0} - M$, with $M$ being a fixed mass scale parameter. The topological term comes from the fifth-order expansion, with the Feynman diagram shown in figure \ref{fig:fifth_order_fd}.

\begin{figure}[!ht]
	\centering
	\begin{tikzpicture}
		\begin{feynman}
			\vertex (a) at (-6, 0);
			\vertex (a1) at (-2, 0);
			\vertex (b) at (-1.86, 5.7);
			\vertex (b1) at (-0.62, 1.9);
			\vertex (c) at (4.8, 3.6);
			\vertex (c1) at (1.62, 1.18);
			\vertex (d) at (4.8, -3.6);
			\vertex (d1) at (1.62, -1.18);
			\vertex (e) at (-1.86, -5.7);
			\vertex (e1) at (-0.62, -1.9);
			\diagram*{
				(a) --[scalar, momentum = $p_{1}$] (a1) --[fermion, momentum = $k$] (b1) --[scalar, rmomentum = $p_{2}$] (b),
				(c) --[scalar, momentum = $p_{3}$] (c1) --[fermion, momentum = $p_{2} + p_{3} + k$] (d1) --[scalar, rmomentum = $p_{4}$] (d),
				(b1) --[fermion, momentum = $p_{2} + k$] (c1),
				(d1) --[fermion, momentum = $p_{2} + p_{3} + p_{3} + k$] (e1) --[scalar, rmomentum' = $-p_{1} - p_{2} - p_{3} - p_{4}$] (e),
				(e1) --[fermion, momentum = $-p_{1} + k$] (a1),
			};
		\end{feynman}
	\end{tikzpicture}
	\caption{Feynman diagram for the third-order term in the effective action.}
	\label{fig:fifth_order_fd}
\end{figure}

This translates to
\begin{align*}
	\Gamma =&  -\frac{i}{5}\inte{}{}\frac{\dd[2]{p_{1}}}{(2\pi)^{2}}\dots\frac{\dd[2]{k}}{(2\pi)^{2}}\tr(\left(\frac{(\fsl{\del{}{}} - M)\left(m + i\gamma^{01}\sum\limits_{l = 1}^{5}M_{l}\Gamma^{l}\right)}{\del{2}{} - M^{2}}\right)^{5}) \\
	 \supset&  \frac{1}{5}\sum\limits_{l_{i} = 1}^{5}\inte{}{}\frac{\dd[2]{p_{1}}}{(2\pi)^{2}}\dots\frac{\dd[2]{k}}{(2\pi)^{2}}\text{tr}\left(\frac{(i\fsl{k} - M)\gamma^{01}M_{l_{1}}\Gamma^{l_{1}}}{-k^{2} - M^{2}}\frac{(i(\fsl{p}_{2} + \fsl{k}) - M)\gamma^{01}M_{l_{2}}\Gamma^{l_{2}}}{-(p_{2} + k)^{2} - M^{2}}\frac{(i(\fsl{p}_{2} + \fsl{p}_{3} + \fsl{k}) - M)\gamma^{01}M_{l_{3}}\Gamma^{l_{3}}}{-(p_{2} + p_{3} + k)^{2} - M^{2}} \right. \\
	        &\cdot \left.\frac{(i(\fsl{p}_{2} + \fsl{p}_{3} + \fsl{p}_{4} + \fsl{k}) - M)\gamma^{01}M_{l_{4}}\Gamma^{l_{4}}}{-(p_{2} + p_{3} + p_{4} + k)^{2} - M^{2}}\frac{(i(-\fsl{p}_{1} + \fsl{k}) - M)\gamma^{01}M_{l_{5}}\Gamma^{l_{5}}}{(-p_{1} + k)^{2} - M^{2}}\right).
\end{align*}
Let us now consider the contents of the trace. We are looking for topological terms, which appear in the presence of all $\Gamma$ and all $\gamma$ appearing exactly once. The matrices will produce a Levi-Civita tensor, meaning any contributions of orders $1$ or $2$ in $k$ will vanish. As such the topological term is given by
\begin{align*}
	\Gamma \supset& -\frac{M}{5}\sum\limits_{l_{i} = 1}^{5}\inte{}{}\frac{\dd[2]{p_{1}}}{(2\pi)^{2}}\dots\frac{\dd[2]{k}}{(2\pi)^{2}}\text{tr}\left(\frac{M_{l_{1}}\Gamma^{l_{1}}}{-k^{2} - M^{2}}\frac{(i(\fsl{p}_{2} + \fsl{k}) - M)\gamma^{01}M_{l_{2}}\Gamma^{l_{2}}}{-(p_{2} + k)^{2} - M^{2}}\frac{(i(\fsl{p}_{2} + \fsl{p}_{3} + \fsl{k}) - M)\gamma^{01}M_{l_{3}}\Gamma^{l_{3}}}{-(p_{2} + p_{3} + k)^{2} - M^{2}} \right. \\
	              &\cdot \left.\frac{(i(\fsl{p}_{2} + \fsl{p}_{3} + \fsl{p}_{4} + \fsl{k}) - M)\gamma^{01}M_{l_{4}}\Gamma^{l_{4}}}{-(p_{2} + p_{3} + p_{4} + k)^{2} - M^{2}}\frac{(i(-\fsl{p}_{1} + \fsl{k}) - M)M_{l_{5}}\Gamma^{l_{5}}}{(-p_{1} + k)^{2} - M^{2}}\right).
\end{align*}
The contents of the trace are
\begin{align*}
	 &\tr(\Gamma^{l_{1}}(i(\fsl{p}_{2} + \fsl{k}) - M)\gamma^{01}\Gamma^{l_{2}}(i(\fsl{p}_{2} + \fsl{p}_{3} + \fsl{k}) - M)\gamma^{01}\Gamma^{l_{3}}(i(\fsl{p}_{2} + \fsl{p}_{3} + \fsl{p}_{4} + \fsl{k}) - M)\gamma^{01}\Gamma^{l_{4}}(i(-\fsl{p}_{1} + \fsl{k}) - M)\Gamma^{l_{5}}) \\
	=& \tr(\Gamma^{l_{1}}\Gamma^{l_{2}}\Gamma^{l_{3}}\Gamma^{l_{4}}\Gamma^{l_{5}})\tr((i(\fsl{p}_{2} + \fsl{k}) - M)\gamma^{01}(i(\fsl{p}_{2} + \fsl{p}_{3} + \fsl{k}) - M)\gamma^{01}(i(\fsl{p}_{2} + \fsl{p}_{3} + \fsl{p}_{4} + \fsl{k}) - M)\gamma^{01}(i(-\fsl{p}_{1} + \fsl{k}) - M)),
\end{align*}
exploiting the product structure of the operators. The case where all Dirac matrices appear exactly once correspond to exactly two momenta appearing, meaning this topological term too will have a pullback of a 2-form onto spacetime.

\paragraph{Extending to Synthetic Dimensions}
Thus far we have seen that the effective actions in class B involve the pullback of a $d + 1$-form. To relate this to the higher Berry curvature, we can imagine the following: If we can write spacetime as the boundary of some other manifold $Y$ and extend the mass fields to $Y$, then Stokes' theorem allows us to write
\begin{align*}
	\Gamma = \inte{}{}\pub{m}{\omega} = \inte{Y}{}\df{(\pub{m}{\omega})} = \inte{Y}{}\pub{m}{(\df{\omega})}.
\end{align*}
The form $\df{\omega}$ is then a $d + 2$-form which plays the role of the higher Berry curvature.

A first question is whether $\df{\omega}$ exists. Certainly the containment of mass fields in class B to $S^{d + 1}$ implies that the answer is no. There are, however, two possible ways to solve that. The first, as proposed by Abanov and Wiegmann, is to simply not confine the mass fields at all. The alternative, which is what is done by Hsin et al, is to extend the usual mass term to a field. The consequence of this choice is that the response terms we have considered are lowest-order terms in the mass perturbation $M_{0}(x) - M$.

\paragraph{Extension Schemes}
We will consider some slightly different theories next. The construction is as follows: Suppose spacetime is some manifold $X$. We can then construct a higher manifold by assuming $X$ to be the boundaries of two different manifold $Y_{\pm}$. Extending spacetime with a synthetic time-like dimension to produce the manifold $Y_{L} = X\times I$, the full manifold on which the theory lives is
\begin{align*}
	Y = \overline{Y}_{-}\cup Y_{L}\cup Y_{+}.
\end{align*}
The bar indicates orientation reversal, and the $I$ dimension goes from $y_{-}$ to $y_{+}$ on the journey between the two borders. We assume that close to the intersections this separability applies too, and so the union of $Y_{L}$ with the regions close to the intersections forms what is called the collar neighborhood of $X$ in $Y$. Specializing to even $d$, the Lagrangian on $Y$ is
\begin{align*}
	\lag_{Y} = \overline{\Psi}(\fsl{D} + M_{1} + i\Gamma_{5}M_{2})\Psi.
\end{align*}
The corresponding Dirac equation is
\begin{align*}
	(\fsl{D} + M_{1} + i\Gamma_{5}M_{2})\Psi = 0.
\end{align*}
Compare this to the ``true'' Dirac equation
\begin{align*}
	(\fsl{D} + M)\Psi = 0,
\end{align*}
with $M$ being some hermitian matrix. $M_{1}$ will vary smoothly between constant values $\pm m_{0}$ in $Y_{\pm}$ as a function of $y$ only, and $M_{2}$ depends only on $x$ in the collar neighborhood and can be extended to $Y_{\pm}$ in some fashion. On $Y_{L}$ we have $M_{2} = M$. For even $d$, $M_{2}$ may be any hermitian matrix, whereas for odd $d$ we have $M_{2} = Q + i\gamma_{5}P$ for two hermitian matrices $Q$ and $P$. We also impose the restriction that $A = A_{i}(x)\df{x}^{i}$.

We now introduce the two Dirac operators
\begin{align*}
	D_{d} = \gamma^{i}D_{i} + M_{2},\ D_{d + 1} = \Gamma^{\mu}D_{\mu} + M_{1} + i\Gamma.
\end{align*}
The latter can be written as
\begin{align*}
	D_{d + 1} = -iK + M_{1},
\end{align*}
where
\begin{align*}
	K = i(\Gamma^{\mu}D_{\mu} + i\Gamma_{5}M_{2}) = i\Gamma^{y}(\del{}{y} + D_{X})
\end{align*}
is a hermitian operator.

The expression we would like to show is
\begin{align*}
	\frac{Z_{d}(X)}{Z_{d}^{(0)}(X)}\cdot\frac{1}{\abs{\frac{Z_{d}(X)}{Z_{d}^{(0)}(X)}}} = \exp(2\pi i(\eta_{K}(Y_{+}, \text{APS}) - \eta_{K^{(0)}}(Y_{+}, \text{APS}))).
\end{align*}
On the right-hand side we have the $\eta$-invariant of $K$ given the boundary condition, which is the number of positive eigenvalues minus the number of negative eigenvalues divided by two. On the left-hand side are the partition functions for the full case, as well as the version with a superscript zero, indicating the case where $M$ is just a constant. This can be interpreted as a Berry phase of the system, so let us therefore review the important aspects of the construction that ensure this.

The properties of $K$ arise when computing certain state overlaps. A boundary condition is needed to guarantee hermicity. The condition used in the original work is implemented by considering the eigenstates of $D_{X}$, which come in pairs as $\acomm{\Gamma^{y}}{D_{X}} = 0$, and is given by
\begin{align*}
	P^{D_{X}}_{< 0}\psi = 0
\end{align*}
on $\bound{Y_{+}}$, where $P^{D_{X}}_{< 0}$ is the projection operator onto the negative eigenstates of $D_{X}$. The hermiticity of $K$ then allows for expansion of states in terms of the eigenstates of $K$. These facts combine to imply the above formula.

In the case of odd $d$ things are slightly different. On $X$ the mass matrix we want to realize is of the form $M_{1} + i\gamma_{5}M_{2}$, where $M_{1}$ and $M_{2}$ are both Hermitian. On $Y$ we extend the mass matrix, producing the Lagrangian
\begin{align*}
	\lag = \overline{\Psi}\left(\fsl{D} + M_{k}\otimes\tau^{k}\right)\Psi,
\end{align*}
with the $\tau^{k}$ being Pauli matrices. The added mass matrix $M_{3}$ is a function of $y$ only, going from $-m_{0}$ to $m_{0}$. The above can be factorized as
\begin{align*}
	\lag = \Psi\adj\beta\otimes\tau^{3}\left(\fsl{D}\otimes\tau^{3} + M_{3} + i(M_{1}\otimes\tau^{2} - M_{2}\otimes\tau^{1})\right)\Psi,
\end{align*}
and by choosing $\Psi\adj\beta\otimes\tau^{3}$ as the adjoint field, the above can be written as
\begin{align*}
	\lag = \overline{\Psi}\left(\fsl{D}\otimes\tau^{3} + M_{3} + i(M_{1}\otimes\tau^{2} - M_{2}\otimes\tau^{1})\right)\Psi.
\end{align*}
The corresponding Dirac operators are
\begin{align*}
	D_{d} = \gamma^{i}D_{i} + M_{1} + i\gamma_{5}M_{2},\ D_{d + 1} = \Gamma^{\mu}D_{\mu} + M_{1}\tau^{1} + M_{2}\tau^{2} + M_{3}\tau^{3},
\end{align*}
and the relation $D_{d + 1} = \tau^{3}(-iK + M_{3})$ implies
\begin{align*}
	K = i\left(\tau^{3}\Gamma^{\mu}D_{\mu} + iM_{1}\otimes\tau^{2} - iM_{2}\otimes\tau^{1}\right).
\end{align*}
Along the collar neighborhood we have
\begin{align*}
	K = i\tau^{3}\Gamma^{y}\left(\del{}{y} + D_{X}\right),\ D_{X} = \Gamma^{y}\left(\Gamma^{i}D_{i} + M_{1}\otimes\tau^{1} + M_{2}\otimes\tau^{2}\right).
\end{align*}
By a somewhat more involved argument involving the same requirements on $K$, the formula for the Berry phase applies equally well to this case.

Let us next consider some examples. First is the Lagrangian
\begin{align*}
	\lag_{2 + 1} = -\overline{\Psi}\left(\fsl{\del{}{}} + M\left(1 + \sum\limits_{a = 3}^{6}m^{a}\Gamma_{a}^{(5)}\right)\right)\Psi.
\end{align*}
More explicitly, this is a model with isospin and Dirac structure, and the above matrices are
\begin{align*}
	\Gamma^{\mu} = \gamma^{\mu}\otimes 1_{\text{isospin}},\ \Gamma^{5} = \gamma^{\mu}\otimes 1_{\text{isospin}},\ \Gamma_{a}^{(5)} = 1_{\text{Dirac}}\otimes\gamma_{a}.
\end{align*}
The extended Lagrangian is
\begin{align*}
	\lag_{3 + 1} = -\overline{\Psi}\left(\fsl{\del{}{}} + M_{1} + iM\Gamma_{5}\left(1 + \sum\limits_{a = 3}^{6}m^{a}\Gamma_{a}^{(5)}\right)\right)\Psi.
\end{align*}
The topological term in the effective action arises from
\begin{align*}
	\Gamma \supset& -\frac{i}{5}\inte{}{}\dd[4]{p}\dd[4]{k}\tr(\left(\frac{M_{1} - M + iM\Gamma_{5} + iM\Gamma_{5}\sum\limits_{a = 3}^{7}m^{a}\Gamma_{a}}{\fsl{\del{}{}} + M}\right)^{5}) \\
	       \supset& -\frac{i}{5}\inte{}{}\dd[4]{p}\dd[4]{k}\frac{1}{\prod\limits_{i = 1}^{5}(-k_{i}^{2} - M^{2})}\tr\left((i\fsl{k}_{1} - M)iM\Gamma_{5}m^{a}(p_{1})\Gamma_{a}^{(5)}(i\fsl{k}_{2} - M)iM\Gamma_{5}m^{b}(p_{2})\Gamma_{b}^{(5)} \right. \\
	              &\cdot \left. (i\fsl{k}_{3} - M)iM\Gamma_{5}m^{c}(p_{3})\Gamma_{c}^{(5)}(i\fsl{k}_{4} - M)iM\Gamma_{5}m^{d}(p_{4})\Gamma_{d}^{(5)}(i\fsl{k}_{5} - M)iM\Gamma_{5}m^{e}(p_{5})\Gamma_{e}^{(5)}\right) \\
	\supset& -\frac{M^{6}}{5}\inte{}{}\dd[4]{p}\dd[4]{k}\frac{m^{a}(p_{1})m^{b}(p_{2})m^{c}(p_{3})m^{d}(p_{4})m^{e}(p_{5})}{\prod\limits_{i = 1}^{5}(-k_{i}^{2} - M^{2})}\tr(\Gamma_{5}\Gamma_{a}^{(5)}\fsl{k}_{2}\Gamma_{5}\Gamma_{b}^{(5)}\fsl{k}_{3}\Gamma_{5}\Gamma_{c}^{(5)}\fsl{k}_{4}\Gamma_{5}\Gamma_{d}^{(5)}\fsl{k}_{5}\Gamma_{5}\Gamma_{e}^{(5)}) \\
	\approx& \frac{M^{6}}{5}\inte{}{}\dd[4]{p}\dd[4]{k}\frac{m_{a}(p_{1})p_{2, \mu}m_{b}(p_{2})p_{3, \nu}m_{c}(p_{3})p_{4, \rho}m_{d}(p_{4})p_{5, \sigma}m_{e}(p_{5})}{(k^{2} + M^{2})^{5}}\tr(\Gamma^{\mu}\Gamma^{\nu}\Gamma^{\rho}\Gamma^{\sigma}\Gamma_{5})\tr(\Gamma_{a}^{(5)}\Gamma_{b}^{(5)}\Gamma_{c}^{(5)}\Gamma_{d}^{(5)}\Gamma_{e}^{(5)}) \\
	      =& \frac{M^{6}}{5}\cdot i\frac{\Gamma\left(3\right)}{(4\pi)^{2}\Gamma\left(5\right)M^{6}}\inte{}{}\dd[4]{p}\dd[4]{k}m_{a}(p_{1})p_{2, \mu}m_{b}(p_{2})p_{3, \nu}m_{c}(p_{3})p_{4, \rho}m_{d}(p_{4})p_{5, \sigma}m_{e}(p_{5})\cdot -4i\varepsilon^{\mu\nu\rho\sigma}\cdot -4\varepsilon_{abcde} \\
	      =& -\frac{1}{60\pi^{2}}\inte{}{}\dd[4]{p}\varepsilon^{\mu\nu\rho\sigma}\varepsilon^{abcde}m_{a}(p_{1})p_{2, \mu}m_{b}(p_{2})p_{3, \nu}m_{c}(p_{3})p_{4, \rho}m_{d}(p_{4})p_{5, \sigma}m_{e}(p_{5}) \\
	      =& -\frac{1}{60\pi^{2}}\inte{}{}\dd[3]{x}\dd{y}\varepsilon^{\mu\nu\rho\sigma}\varepsilon^{abcde}m_{a}\del{}{\mu}m_{b}\del{}{\nu}m_{c}\del{}{\rho}m_{d}\del{}{\sigma}m_{e}.
\end{align*}
This is a familiar Weiss-Zumino-Witten term with
\begin{align*}
	\omega_{abcd} = -\frac{1}{60\pi^{2}}\varepsilon_{eabcd}m^{e}.
\end{align*}

The first example with $d = 1$ is given by
\begin{align*}
\lag_{1 + 1} = -\overline{\Psi}\left(\fsl{\del{}{}} + M\left(m_{4} + i\gamma\sum\limits_{a = 1}^{3}m^{a}\Gamma_{a}^{(3)}\right)\right)\Psi,
\end{align*}
with $m_{d + 3}$ now being allowed to vary such that the target space of this model is $S^{3}$. The extended Lagrangian is
\begin{align*}
\lag_{1 + 2} = -\overline{\Psi}\left(\fsl{\del{}{}}\tau^{3} + M +  iM\left(m_{4}\tau^{2} - \sum\limits_{a = 1}^{3}m^{a}\Gamma_{a}^{(3)}\tau^{1}\right)\right)\Psi.
\end{align*}
The number of anticommuting matrices in this theory is seven, and so the previously performed computation for $d = 2$ can be repeated. The topological term in the effective action is given by
\begin{align*}
	\Gamma \supset& \frac{iM^{4}}{4}\inte{}{}\frac{\dd[3]{p_{1}}}{(2\pi)^{3}}\frac{\dd[3]{p_{2}}}{(2\pi)^{3}}\frac{\dd[3]{p_{3}}}{(2\pi)^{3}}\frac{\dd[3]{k}}{(2\pi)^{3}}\tr\left(\frac{m_{4}\tau^{2} - \sum\limits_{a = 1}^{3}m^{a}\Gamma_{a}^{(3)}\tau^{1}}{\fsl{\del{}{}}\tau^{3} + M}\frac{m_{4}\tau^{2} - \sum\limits_{b = 1}^{3}m^{b}\Gamma_{b}^{(3)}\tau^{1}}{\fsl{\del{}{}}\tau^{3} + M} \right. \\
	 &\cdot \left. \frac{m_{4}\tau^{2} - \sum\limits_{c = 1}^{3}m^{c}\Gamma_{c}^{(3)}\tau^{1}}{\fsl{\del{}{}}\tau^{3} + M}\frac{m_{4}\tau^{2} - \sum\limits_{d = 1}^{3}m^{d}\Gamma_{d}^{(3)}\tau^{1}}{\fsl{\del{}{}}\tau^{3} + M}\right) \\
	=& \frac{iM^{4}}{4}\inte{}{}\frac{\dd[3]{p}}{(2\pi)^{3}}\frac{\dd[3]{k}}{(2\pi)^{3}}\frac{1}{\prod\limits_{i = 1}^{4}(-k_{i}^{2} - M^{2})}\tr\left(\left(m_{4}\tau^{2} - \sum\limits_{a = 1}^{3}m^{a}\Gamma_{a}^{(3)}\tau^{1}\right)(i\gamma^{\mu}\tau^{3}k_{1, \mu} - M) \right. \\
	 &\cdot \left. \left(m_{4}\tau^{2} - \sum\limits_{a = 1}^{3}m^{b}\Gamma_{b}^{(3)}\tau^{1}\right)(i\gamma^{\nu}\tau^{3}k_{2, \nu} - M)\left(m_{4}\tau^{2} - \sum\limits_{a = 1}^{3}m^{c}\Gamma_{c}^{(3)}\tau^{1}\right)(i\gamma^{\rho}\tau^{3}k_{3, \rho} - M) \right. \\
	 &\cdot \left. \left(m_{4}\tau^{2} - \sum\limits_{d = 1}^{3}m^{d}\Gamma_{d}^{(3)}\tau^{1}\right)(i\gamma^{\sigma}\tau^{3}k_{4, \sigma} - M)\right) \\
	=& -\frac{M^{5}}{4}\inte{}{}\frac{\dd[3]{p}}{(2\pi)^{3}}\frac{\dd[3]{k}}{(2\pi)^{3}}\frac{k_{2, \mu}k_{3, \nu}k_{4, \rho}}{\prod\limits_{i = 1}^{4}(-k_{i}^{2} - M^{2})}\tr\left(\left(m_{4}\tau^{2} - \sum\limits_{a = 1}^{3}m^{a}\Gamma_{a}^{(3)}\tau^{1}\right) \right. \\
	 &\cdot \left. \left(m_{4}\tau^{2} - \sum\limits_{a = 1}^{3}m^{b}\Gamma_{b}^{(3)}\tau^{1}\right)\gamma^{\mu}\tau^{3}\left(m_{4}\tau^{2} - \sum\limits_{a = 1}^{3}m^{c}\Gamma_{c}^{(3)}\tau^{1}\right)\gamma^{\nu}\tau^{3} \right. \\
	 &\cdot \left. \left(m_{4}\tau^{2} - \sum\limits_{d = 1}^{3}m^{d}\Gamma_{d}^{(3)}\tau^{1}\right)\gamma^{\rho}\tau^{3}\right) \\
	=& -\frac{M^{5}}{4}\inte{}{}\frac{\dd[3]{p}}{(2\pi)^{3}}\frac{\dd[3]{k}}{(2\pi)^{3}}\frac{k_{2, \mu}k_{3, \nu}k_{4, \rho}}{\prod\limits_{i = 1}^{4}(k_{i}^{2} + M^{2})}\tr\left(\left(m_{4}\tau^{2} - \sum\limits_{a = 1}^{3}m^{a}\Gamma_{a}^{(3)}\tau^{1}\right) \right. \\
	 &\cdot \left. \left(m_{4}\tau^{2} - \sum\limits_{a = 1}^{3}m^{b}\Gamma_{b}^{(3)}\tau^{1}\right)\tau^{3}\left(m_{4}\tau^{2} - \sum\limits_{a = 1}^{3}m^{c}\Gamma_{c}^{(3)}\tau^{1}\right)\tau^{3}\left(m_{4}\tau^{2} - \sum\limits_{d = 1}^{3}m^{d}\Gamma_{d}^{(3)}\tau^{1}\right)\tau^{3}\right)\tr(\gamma^{\mu}\gamma^{\nu}\gamma^{\rho}).
\end{align*}
The trace has four terms depending on the placement of $m_{4}$. The first is
\begin{align*}
	 &-\sum\limits_{b, c, d = 1}^{3}\tr\left(m_{4}\tau^{2}m^{b}\Gamma_{b}^{(3)}\tau^{1}\tau^{3}m^{c}\Gamma_{c}^{(3)}\tau^{1}\tau^{3}m^{d}\Gamma_{d}^{(3)}\tau^{1}\tau^{3}\right) \\
	=& -\sum\limits_{a, b, c = 1}^{3}m^{4}(p_{1})m^{a}(p_{2})m^{b}(p_{3})m^{c}(p_{4})\tr\left(\Gamma_{a}^{(3)}\Gamma_{b}^{(3)}\Gamma_{c}^{(3)}\right)\tr(\tau^{2}\tau^{1}\tau^{3}\tau^{1}\tau^{3}\tau^{1}\tau^{3}) \\
	=& -\sum\limits_{a, b, c = 1}^{3}m^{4}(p_{1})m^{a}(p_{2})m^{b}(p_{3})m^{c}(p_{4})\tr\left(\Gamma_{a}^{(3)}\Gamma_{b}^{(3)}\Gamma_{c}^{(3)}\right)\tr(\tau^{1}\tau^{2}\tau^{3}).
\end{align*}
The second is
\begin{align*}
	 &-\sum\limits_{a, c, d = 1}^{3}\tr\left(m^{a}\Gamma_{a}^{(3)}\tau^{1}m^{4}\tau^{2}\tau^{3}m^{c}\Gamma_{c}^{(3)}\tau^{1}\tau^{3}m^{d}\Gamma_{d}^{(3)}\tau^{1}\tau^{3}\right) \\
	=& -\sum\limits_{a, b, c = 1}^{3}m^{a}(p_{1})m^{4}(p_{2})m^{b}(p_{3})m^{c}(p_{4})\tr\left(\Gamma_{a}^{(3)}\Gamma_{b}^{(3)}\Gamma_{c}^{(3)}\right)\tr(\tau^{1}\tau^{2}\tau^{3}\tau^{1}\tau^{3}\tau^{1}\tau^{3}) \\
	=& \sum\limits_{a, b, c = 1}^{3}m^{a}(p_{1})m^{4}(p_{2})m^{b}(p_{3})m^{c}(p_{4})\tr\left(\Gamma_{a}^{(3)}\Gamma_{b}^{(3)}\Gamma_{c}^{(3)}\right)\tr(\tau^{1}\tau^{2}\tau^{3}).
\end{align*}
The third is
\begin{align*}
	 &-\sum\limits_{a, b, d = 1}^{3}\tr\left(m^{a}\Gamma_{a}^{(3)}\tau^{1}m^{b}\Gamma_{b}^{(3)}\tau^{1}\tau^{3}m^{4}\tau^{2}\tau^{3}m^{d}\Gamma_{d}^{(3)}\tau^{1}\tau^{3}\right) \\
	=& -\sum\limits_{a, b, c = 1}^{3}m^{a}(p_{1})m^{b}(p_{2})m^{4}(p_{3})m^{c}(p_{4})\tr\left(\Gamma_{a}^{(3)}\Gamma_{b}^{(3)}\Gamma_{c}^{(3)}\right)\tr(\tau^{3}\tau^{2}\tau^{3}\tau^{1}\tau^{3}) \\
	=& -\sum\limits_{a, b, c = 1}^{3}m^{a}(p_{1})m^{b}(p_{2})m^{4}(p_{3})m^{c}(p_{4})\tr\left(\Gamma_{a}^{(3)}\Gamma_{b}^{(3)}\Gamma_{c}^{(3)}\right)\tr(\tau^{1}\tau^{2}\tau^{3}).
\end{align*}
The fourth is
\begin{align*}
	 &-\sum\limits_{a, b, c = 1}^{3}\tr\left(m^{a}\Gamma_{a}^{(3)}\tau^{1}m^{b}\Gamma_{b}^{(3)}\tau^{1}\tau^{3}m^{c}\Gamma_{c}^{(3)}\tau^{1}\tau^{3}m^{4}\tau^{2}\tau^{3}\right) \\
	=& -\sum\limits_{a, b, c = 1}^{3}m^{a}(p_{1})m^{b}(p_{2})m^{c}(p_{3})m^{4}(p_{4})\tr\left(\Gamma_{a}^{(3)}\Gamma_{b}^{(3)}\Gamma_{c}^{(3)}\right)\tr(\tau^{3}\tau^{1}\tau^{3}\tau^{2}\tau^{3}) \\
	=& \sum\limits_{a, b, c = 1}^{3}m^{a}(p_{1})m^{b}(p_{2})m^{c}(p_{3})m^{4}(p_{4})\tr\left(\Gamma_{a}^{(3)}\Gamma_{b}^{(3)}\Gamma_{c}^{(3)}\right)\tr(\tau^{1}\tau^{2}\tau^{3}).
\end{align*}
The traces evaluate to
\begin{align*}
	\tr\left(\Gamma_{a}^{(3)}\Gamma_{b}^{(3)}\Gamma_{c}^{(3)}\right) = -2i\varepsilon_{abc},\ \tr(\tau^{1}\tau^{2}\tau^{3}) = 2i,
\end{align*}
and in combination with the four terms we have
\begin{align*}
	\Gamma =& -\frac{M^{5}}{4}\inte{}{}\frac{\dd[3]{p}}{(2\pi)^{3}}\frac{\dd[3]{k}}{(2\pi)^{3}}k_{2, \mu}k_{3, \nu}k_{4, \rho}\frac{m^{a}(p_{1})m^{b}(p_{2})m^{c}(p_{3})m^{d}(p_{4})}{\prod\limits_{i = 1}^{4}(k_{i}^{2} + M^{2})}\cdot -4\varepsilon_{abcd}\cdot 2\varepsilon^{\mu\nu\rho} \\
	       =& \frac{i}{32\pi}\inte{}{}\frac{\dd[3]{p}}{(2\pi)^{3}}\varepsilon^{\mu\nu\rho}\varepsilon_{abcd}m^{a}(p_{1})p_{2, \mu}m^{b}(p_{2})p_{3, \nu}m^{c}(p_{3})p_{4, \rho}m^{d}(p_{4}) \\
	       =& -\frac{1}{32\pi}\inte{}{}\dd[3]{x}\varepsilon^{\mu\nu\rho}\varepsilon_{abcd}m^{a}\del{}{\mu}m^{b}\del{}{\nu}m^{c}\del{}{\rho}m^{d}.
\end{align*}