\section{Quantum Field Theory}

\paragraph{Effective Actions}
To define the effective action, we first introduce
\begin{align*}
	E[J] = -i\ln(Z),
\end{align*}
where $Z$ is the generating functional of some quantum filed theory. $E$ is essentially a measure of the vacuum energy as a function of the source $J$. There is also a strong analogy to statistical mechanics at play, with $Z$ playing the role of the partition function and $E$ the role of the Helmholtz free energy. Its functional derivatives are given by
\begin{align*}
	\fdv{E}{J(x)} = -\frac{i}{Z}\fdv{Z}{J(x)} = \frac{\pinte{}{\phi}\phi(x)e^{i\left(S + \inte{}{}\dd[d]{y}J(y)\phi(y)\right)}}{\pinte{}{\phi}e^{i\left(S + \inte{}{}\dd[d]{y}J(y)\phi(y)\right)}}.
\end{align*}
In analogy with statistical mechanics, this can be considered a classical vacuum expectation value in the presence of a source, hence we term it $\phi_{J}(x)$. Its evaluation at $J = 0$ nets us the familiar correlation function. In analogy with statistical mechanics we can now perform a Legendre transform according to
\begin{align*}
	\Gamma[\phi] = E[J_{\phi}] - \inte{}{}\dd[d]{x}\phi(x)J_{\phi}(x).
\end{align*}
As we are used to, the above defines $J_{\phi}$ as a functional satisfying
\begin{align*}
	\fdv{\Gamma}{\phi(x)} = -J_{\phi}(x).
\end{align*}
The quantity $\Gamma$ is the effective action. Note that $E$ and $\Gamma$ coincide for $J = 0$.

Let us first study properties of $E$. We have
\begin{align*}
	\fdv{J(x)}\fdv{J(y)}E =& -i\left(i^{2}\expval{\phi(x)\phi(y)} - i^{2}\expval{\phi(x)}\expval{\phi(y)}\right).
\end{align*}
This demonstrates the explicit removal of disconnected Feynman diagrams in $E$. The general result is
\begin{align*}
	\left(\prod\limits_{i = 1}^{n}\fdv{J(x_{i})}\right)E = -i^{n + 1}\expval{\prod\limits_{i = 1}^{n}\phi(x_{i})}_{\text{conn}},
\end{align*}
which will be useful for computing terms in the effective action.

One can also introduce a partial effective action, where only some fields are integrated out. This is the case that is important to us. It also turns out that the systematics of computing traces in this case are equivalent to connected Feynman diagrams. We will see more of this when studying concrete examples.

\paragraph{Theories With Topological Response}
We will now consider some field theories with background fields. The effective actions of these theories contain topological terms, which are metric-independent. The significance of this metric-independence is that correlation functions, and therefore the theory, is stable under deformations of spacetime or generalized coordinate transformations. It also implies that all correlations lengths are zero.

The first is the $1 + 1$-dimensional theory
\begin{align*}
	\lag = -i\overline{\psi}\left(\fsl{\del{}{}} + M_{1} + iM_{2}\gamma^{01}\right)\psi.
\end{align*}
The two masses are taken to be slowly varying background fields, which are also taken to be on a circle. Imposing this we note that
\begin{align*}
	(\gamma^{01})^{2} = 1 \implies e^{i\phi\gamma^{01}} = \cos(\phi) + i\sin(\phi)\gamma^{01},
\end{align*}
and thus
\begin{align*}
	M_{1} + iM_{2}\gamma^{01} = M\left(\cos(\alpha) + i\sin(\alpha)\gamma^{01}\right) = Me^{i\alpha\gamma^{01}},
\end{align*}
with $M$ and $\alpha$ being the magnitude and argument of the complex number $M_{1} + iM_{2}$ (note that hermiticity implies that both parameters be real). Finally, coupling the fermion field to a gauge field, the full Lagrangian for this theory is
\begin{align*}
	\lag = -i\overline{\psi}\left(\fsl{D} + Me^{i\alpha\gamma^{01}}\right)\psi - \frac{1}{4}F_{\mu\nu}F^{\mu\nu}.
\end{align*}
To compute the effective action, we integrate out the fermion and perform a perturbation expansion treating $\fsl{D} + Me^{i\alpha\gamma^{01}}$ as a perturbed version of $\fsl{\del{}{}} + M$. The effective action for the gauge field is
\begin{align*}
	\Gamma =& -i\ln(\det(-i\left(\fsl{D} + Me^{i\alpha\gamma^{01}}\right))) + \inte{}{}\dd[2]{x}-\frac{1}{4}F_{\mu\nu}F^{\mu\nu} \\
	       =& -i\tr(\ln(-i\left(\fsl{D} + Me^{i\alpha\gamma^{01}}\right))) + \inte{}{}\dd[2]{x}-\frac{1}{4}F_{\mu\nu}F^{\mu\nu}.
\end{align*}
The trace can be computed by summing over some basis with respect to both the field and Dirac structures. In particular, the new term is
\begin{align*}
	-i\tr(\ln(-i\left(\fsl{D} + Me^{i\alpha\gamma^{01}}\right))) \approx& 	-i\tr(\ln(-i\left(\fsl{\del{}{}} + M\right)\left(1 + \frac{\fsl{A} + iM\alpha\gamma^{01}}{\fsl{\del{}{}} + M}\right))) \\
	=& C_{0} - i\tr(\ln(1 + \frac{-ie\fsl{A} + iM\alpha\gamma^{01}}{\fsl{\del{}{}} + M})) \\
	\approx& C_{0} - i\tr(\frac{-ie\fsl{A} + iM\alpha\gamma^{01}}{\fsl{\del{}{}} + M} - \frac{1}{2}\frac{-ie\fsl{A} + iM\alpha\gamma^{01}}{\fsl{\del{}{}} + M}\frac{-ie\fsl{A} + iM\alpha\gamma^{01}}{\fsl{\del{}{}} + M}).
\end{align*}
Let us first compute the inverse of the denominator. We have
\begin{align*}
	(\fsl{\del{}{}} + M)(\fsl{\del{}{}} - M) = \fsl{\del{}{}}^{2} - M^{2} = \del{2}{} - M^{2} \implies \frac{1}{\fsl{\del{}{}} + M} = \frac{\fsl{\del{}{}} - M}{\del{2}{} - M^{2}}.
\end{align*}
Computing the trace in momentum space, applying the correspondence principle $p = -i\del{}{}$ and using the systematics of Feynman diagrams we find
\begin{align*}
	 &-i\tr(-ie\frac{\fsl{A} + iM\alpha\gamma^{01}}{\fsl{\del{}{}} + M} - \frac{1}{2}\frac{-ie\fsl{A} + iM\alpha\gamma^{01}}{\fsl{\del{}{}} + M}\frac{-ie\fsl{A} + iM\alpha\gamma^{01}}{\fsl{\del{}{}} + M}) \\
	=& -i\inte{}{}\frac{\dd[2]{p_{1}}}{(2\pi)^{2}}\frac{\dd[2]{p_{2}}}{(2\pi)^{2}}\tr(\frac{-ie\fsl{A} + iM\alpha\gamma^{01}}{\fsl{\del{}{}} + M}) + \frac{i}{2}\inte{}{}\frac{\dd[2]{p_{1}}}{(2\pi)^{2}}\frac{\dd[2]{p_{2}}}{(2\pi)^{2}}\frac{\dd[2]{p_{3}}}{(2\pi)^{2}}\frac{\dd[2]{p_{4}}}{(2\pi)^{2}}\tr(\frac{-ie\fsl{A} + iM\alpha\gamma^{01}}{\fsl{\del{}{}} + M}\frac{-ie\fsl{A} + iM\alpha\gamma^{01}}{\fsl{\del{}{}} + M}) \\
	=& -i\inte{}{}\frac{\dd[2]{p_{1}}}{(2\pi)^{2}}\frac{\dd[2]{p_{2}}}{(2\pi)^{2}}\tr(\frac{i\fsl{p_{1}} - M}{-p_{1}^{2} - M^{2}}(-ie\fsl{A}(p_{2}) + iM\alpha(p_{2})\gamma^{01}))\delta_{p_{1} + p_{2}} \\
	 &+ \frac{i}{2}\inte{}{}\frac{\dd[2]{p_{1}}}{(2\pi)^{2}}\frac{\dd[2]{p_{2}}}{(2\pi)^{2}}\frac{\dd[2]{p_{3}}}{(2\pi)^{2}}\frac{\dd[2]{p_{4}}}{(2\pi)^{2}}\tr(\frac{i\fsl{p_{1}} - M}{-p_{1}^{2} - M^{2}}(-ie\fsl{A}(p_{2}) + iM\alpha(p_{2})\gamma^{01})\frac{i\fsl{p_{3}} - M}{-p_{3}^{2} - M^{2}}(-ie\fsl{A}(p_{4}) + iM\alpha(p_{4})\gamma^{01})) \\
	 &\cdot\delta_{p_{2} + p_{1} + p_{3}}\delta_{-p_{1} - p_{3} + p_{4}} \\
	=& -\frac{i}{(2\pi)^{2}}\inte{}{}\frac{\dd[2]{p_{1}}}{(2\pi)^{2}}\tr(\frac{i\fsl{p_{1}} - M}{-p_{1}^{2} - M^{2}}(-ie\fsl{A}(-p_{1}) + iM\alpha(-p_{1})\gamma^{01})) \\
	 &+ \frac{i}{2(2\pi)^{2}}\inte{}{}\frac{\dd[2]{p_{1}}}{(2\pi)^{2}}\frac{\dd[2]{p_{2}}}{(2\pi)^{2}}\frac{\dd[2]{p_{3}}}{(2\pi)^{2}}\tr(\frac{i\fsl{p_{1}} - M}{-p_{1}^{2} - M^{2}}(-ie\fsl{A}(p_{2}) + iM\alpha(p_{2})\gamma^{01})\frac{i\fsl{p_{3}} - M}{-p_{3}^{2} - M^{2}}(-ie\fsl{A}(p_{1} + p_{3}) + iM\alpha(p_{1} + p_{3})\gamma^{01})) \\
	 &\cdot\delta_{p_{2} + p_{1} + p_{3}} \\
	=& -\frac{i}{(2\pi)^{2}}\inte{}{}\frac{\dd[2]{p_{1}}}{(2\pi)^{2}}\tr(\frac{i\fsl{p_{1}} - M}{-p_{1}^{2} - M^{2}}(-ie\fsl{A}(-p_{1}) + iM\alpha(-p_{1})\gamma^{01})) \\
	 &+ \frac{i}{2(2\pi)^{4}}\inte{}{}\frac{\dd[2]{p_{1}}}{(2\pi)^{2}}\frac{\dd[2]{p_{2}}}{(2\pi)^{2}}\tr(\frac{i\fsl{p_{1}} - M}{-p_{1}^{2} - M^{2}}(-ie\fsl{A}(p_{2}) + iM\alpha(p_{2})\gamma^{01})\frac{i(-\fsl{p_{2}} - \fsl{p_{1}}) - M}{-(-p_{2} - p_{1})^{2} - M^{2}}(-ie\fsl{A}(-p_{2}) + iM\alpha(-p_{2})\gamma^{01})).
\end{align*}
This can alternatively be illustrated using the Feynman diagram in figure \ref{fig:second_order_fd}.

\begin{figure}[!ht]
	\centering
		\begin{tikzpicture}
			\begin{feynman}
				\vertex (a) at (0, 0);
				\vertex (b) at (2, 0);
				\vertex (c) at (4, 0);
				\vertex (d) at (6, 0);
				\diagram*{
					(a) --[scalar, momentum = $k$] (b) --[fermion, half left, rmomentum = $p$] (c) --[scalar, rmomentum = $-k$] (d),
					(c) --[fermion, half left, momentum = $-p - k$] (b),
				};
			\end{feynman}
		\end{tikzpicture}
	\caption{Feynman diagram for the second-order term in the effective action.}
	\label{fig:second_order_fd}
\end{figure}

The second line produces the lowest-order topological terms. There we need only consider the case where the number of Dirac matrices is even, as the odd-numbered cases vanish when tracing the matrices. These terms are
\begin{align*}
	       &\frac{i}{2(2\pi)^{4}}\inte{}{}\frac{\dd[2]{p}}{(2\pi)^{2}}\frac{\dd[2]{k}}{(2\pi)^{2}}\tr(\frac{i\fsl{p} - M}{-p^{2} - M^{2}}(-ie\fsl{A}(k) + iM\alpha(k)\gamma^{01})\frac{i(-\fsl{p} - \fsl{k}) - M}{-(p - k)^{2} - M^{2}}(-ie\fsl{A}(-k) + iM\alpha(-k)\gamma^{01})) \\
	\supset& \frac{e}{2(2\pi)^{4}}\inte{}{}\frac{\dd[2]{p}}{(2\pi)^{2}}\frac{\dd[2]{k}}{(2\pi)^{2}}\tr(\frac{i\fsl{p} - M}{-p^{2} - M^{2}}\fsl{A}(k)\frac{i(-\fsl{p} - \fsl{k}) - M}{-(-p - k)^{2} - M^{2}}iM\alpha(-k)\gamma^{01}) \\
	       &+ \frac{e}{2(2\pi)^{4}}\inte{}{}\frac{\dd[2]{p}}{(2\pi)^{2}}\frac{\dd[2]{k}}{(2\pi)^{2}}\tr(\frac{i\fsl{p} - M}{-p^{2} - M^{2}}iM\alpha(k)\gamma^{01}\frac{i(-\fsl{p} - \fsl{k}) - M}{-(-p - k)^{2} - M^{2}}\fsl{A}(-k)) \\
	      =& \frac{ieM}{2(2\pi)^{4}}\left(\inte{}{}\frac{\dd[2]{p}}{(2\pi)^{2}}\frac{\dd[2]{k}}{(2\pi)^{2}}\alpha(-k)A_{\mu}(k)\tr(\frac{i\fsl{p} - M}{-p^{2} - M^{2}}\gamma^{\mu}\frac{i(-\fsl{p} - \fsl{k}) - M}{-(-p - k)^{2} - M^{2}}\gamma^{01}) \right. \\
	       &+ \left. \inte{}{}\frac{\dd[2]{p}}{(2\pi)^{2}}\frac{\dd[2]{k}}{(2\pi)^{2}}\alpha(k)A_{\mu}(-k)\tr(\frac{i\fsl{p} - M}{-p^{2} - M^{2}}\gamma^{01}\frac{i(-\fsl{p} - \fsl{k}) - M}{-(-p - k)^{2} - M^{2}}\gamma^{\mu})\right) \\
	\supset& -\frac{ieM}{2(2\pi)^{4}}\left(\inte{}{}\frac{\dd[2]{p}}{(2\pi)^{2}}\frac{\dd[2]{k}}{(2\pi)^{2}}\frac{\alpha(-k)A_{\mu}(k)}{(-p^{2} - M^{2})(-(-p - k)^{2} - M^{2})}\tr(\left(i\fsl{p}\gamma^{\mu}M + M\gamma^{\mu}i(-\fsl{p} - \fsl{k})\right)\gamma^{01}) \right. \\
	       &+ \left. \frac{\alpha(k)A_{\mu}(-k)}{(-p^{2} - M^{2})(-(-p - k)^{2} - M^{2})}\tr(\left(i\fsl{p}\gamma^{01}M + M\gamma^{01}i(-\fsl{p} - \fsl{k})\right)\gamma^{\mu})\right) \\
	      =& \frac{eM^{2}}{2(2\pi)^{4}}\left(\inte{}{}\frac{\dd[2]{p}}{(2\pi)^{2}}\frac{\dd[2]{k}}{(2\pi)^{2}}\frac{\alpha(-k)A_{\mu}(k)}{(-p^{2} - M^{2})(-(-p - k)^{2} - M^{2})}\tr(\fsl{p}\gamma^{\mu}\gamma^{01} + \gamma^{\mu}(-\fsl{p} - \fsl{k})\gamma^{01}) \right. \\
	       &+ \left. \frac{\alpha(k)A_{\mu}(-k)}{(-p_{1}^{2} - M^{2})(-(-p - k_{1})^{2} - M^{2})}\tr(\fsl{p}\gamma^{01}\gamma^{\mu} + \gamma^{01}(-\fsl{p} - \fsl{k})\gamma^{\mu})\right) \\
	      =& \frac{eM^{2}}{2(2\pi)^{4}}\left(\inte{}{}\frac{\dd[2]{p}}{(2\pi)^{2}}\frac{\dd[2]{k}}{(2\pi)^{2}}\frac{\alpha(-k)A_{\mu}(k)}{(-p^{2} - M^{2})(-(-p - k)^{2} - M^{2})}\tr(p_{\nu}\gamma^{\nu}\gamma^{\mu}\gamma^{01} + (-p - k)_{\nu}\gamma^{\mu}\gamma^{\nu}\gamma^{01}) \right. \\
	       &+ \left. \frac{\alpha(k)A_{\mu}(-k)}{(-p^{2} - M^{2})(-(-p - k)^{2} - M^{2})}\tr(p_{\nu}\gamma^{\nu}\gamma^{01}\gamma^{\mu} + (-p - k)_{\nu}\gamma^{01}\gamma^{\nu}\gamma^{\mu})\right).
\end{align*}
Because we take the mass fields to be slowly varying, we can remove all contributions over order 1 in $\frac{k}{p}$. Inverting the $k$-integral in the second term leaves us with
\begin{align*}
	 &\frac{eM^{2}}{2(2\pi)^{4}}\left(\inte{}{}\frac{\dd[2]{p}}{(2\pi)^{2}}\frac{\dd[2]{k}}{(2\pi)^{2}}\frac{\alpha(-k)A_{\mu}(k)}{(-p^{2} - M^{2})^{2}}\tr(p_{\nu}\gamma^{\nu}\gamma^{\mu}\gamma^{01} - (p + k)_{\nu}\gamma^{\mu}\gamma^{\nu}\gamma^{01} + p_{\nu}\gamma^{\nu}\gamma^{01}\gamma^{\mu} + (-p + k)_{\nu}\gamma^{01}\gamma^{\nu}\gamma^{\mu})\right) \\
	=& -\frac{eM^{2}}{(2\pi)^{4}}\inte{}{}\frac{\dd[2]{p}}{(2\pi)^{2}}\frac{\dd[2]{k}}{(2\pi)^{2}}\frac{\alpha(-k)k_{\nu}A_{\mu}(k)}{(-p^{2} - M^{2})^{2}}\tr(\gamma^{\mu}\gamma^{\nu}\gamma^{01}) \\
	=& -\frac{2eM^{2}}{(2\pi)^{4}}\inte{}{}\frac{\dd[2]{k}}{(2\pi)^{2}}\varepsilon^{\mu\nu}\alpha(-k)k_{\nu}A_{\mu}(k)\inte{}{}\frac{\dd[2]{p}}{(2\pi)^{2}}\frac{1}{(p^{2} + M^{2})^{2}}.
\end{align*}
Let us consider the innermost integral. Performing a Wick rotation and a substitution we have
\begin{align*}
	\inte{}{}\frac{\dd[2]{p}}{(2\pi)^{2}}\frac{1}{(p^{2} + M^{2})^{2}} =& \frac{1}{M^{2}}\inte{}{}\frac{\dd[2]{q}}{(2\pi)^{2}}\frac{1}{(-q_{0}^{2} + q_{1}^{2} + 1)^{2}} \\
	=& \frac{i}{M^{2}}\inte{}{}\frac{\dd[2]{\ell}}{(2\pi)^{2}}\frac{1}{(\ell_{0}^{2} + \ell_{1}^{2} + 1)^{2}} \\
	=& \frac{i}{2\pi M^{2}}\inte{0}{\infty}\dd{r}\frac{r}{(r^{2} + 1)^{2}} \\
	=& -\frac{i}{2\pi M^{2}}\eval{\frac{1}{2(r^{2} + 1)}}_{0}^{\infty} \\
	=& \frac{i}{4\pi M^{2}}.
\end{align*}
Let us also note the general result
\begin{align*}
	\inte{}{}\frac{\dd[d]{p}}{(2\pi)^{d}}\frac{(p^{2})^{a}}{(p^{2} + \Delta)^{b}} = i\frac{\Gamma\left(b - a - \frac{d}{2}\right)\Gamma\left(a + \frac{d}{2}\right)}{(4\pi)^{\frac{d}{2}}\Gamma\left(b\right)\Gamma\left(\frac{d}{2}\right)}\Delta^{a + \frac{d}{2} - b}.
\end{align*}
The final expression for the momentum space effective action is then
\begin{align*}
	\Gamma = -\frac{ie}{2\pi(2\pi)^{4}}\inte{}{}\frac{\dd[2]{k}}{(2\pi)^{2}}\varepsilon^{\mu\nu}\alpha(-k)k_{\nu}A_{\mu}(k),
\end{align*}
and switching to real space we have
\begin{align*}
	\Gamma = \frac{e}{2\pi(2\pi)^{4}}\inte{}{}\dd[2]{x}\varepsilon^{\mu\nu}\alpha\del{}{\mu}A_{\nu} = \frac{e}{2\pi(2\pi)^{4}}\inte{}{}\alpha F.
\end{align*}
The normalization in this step is likely off - namely, the vertices should probably provide a factor of $(2\pi)^{2}$ each. Absorbing the coupling constant into the gauge field (which modifies the free term) as well, the effective action should be
\begin{align*}
	\Gamma = \frac{1}{2\pi}\inte{}{}\alpha F.
\end{align*}

Another model one could study is
\begin{align*}
	\lag = -i\overline{\Psi}(\fsl{\del{}{}} + M_{0} + i\gamma^{01}M_{i}\sigma^{i})\Psi,
\end{align*}
where the Pauli matrices acts in flavor space and the other Dirac matrices act on the Dirac structure. The $M$ take values on $S^{3}$ at any point in spacetime. The effective action is then
\begin{align*}
	\Gamma =& -i\ln(\det(-i(\fsl{\del{}{}} + M_{0} + i\gamma^{01}M_{i}\sigma^{i}))) \\
	       =& -i\tr(\ln(-i(\fsl{\del{}{}} + M_{0} + i\gamma^{01}M_{i}\sigma^{i}))) \\
	       =& -i\tr(\ln(-i(\fsl{\del{}{}} + M_{0})\left(1 + \frac{i\gamma^{01}M_{i}\sigma^{i}}{\fsl{\del{}{}} + M_{0}}\right))) \\
	       =& C_{0} - i\tr(\ln(1 + \frac{i\gamma^{01}M_{i}\sigma^{i}}{\fsl{\del{}{}} + M_{0}})).
\end{align*}
The relevant parts are
\begin{align*}
	\Gamma =& -i\tr(\frac{i\gamma^{01}M_{i}\sigma^{i}}{\fsl{\del{}{}} + M_{0}} - \frac{1}{2}\frac{i\gamma^{01}M_{i}\sigma^{i}}{\fsl{\del{}{}} + M_{0}}\frac{i\gamma^{01}M_{i}\sigma^{i}}{\fsl{\del{}{}} + M_{0}}) \\
	       =& -i\tr(\frac{\fsl{\del{}{}} - M_{0}}{\del{2}{} - M_{0}^{2}}i\gamma^{01}M_{i}\sigma^{i} - \frac{1}{2}\frac{\fsl{\del{}{}} - M_{0}}{\del{2}{} - M_{0}^{2}}i\gamma^{01}M_{i}\sigma^{i}\frac{\fsl{\del{}{}} - M_{0}}{\del{2}{} - M_{0}^{2}}i\gamma^{01}M_{j}\sigma^{j}) \\
	       =& -\frac{i}{(2\pi)^{2}}\inte{}{}\frac{\dd[2]{p}}{(2\pi)^{2}}\tr(\frac{i\fsl{p} - M_{0}}{-p^{2} - M_{0}^{2}}i\gamma^{01}M_{i}(p)\sigma^{i}) \\
	        &- \frac{i}{2(2\pi)^{4}}\inte{}{}\frac{\dd[2]{p}}{(2\pi)^{2}}\frac{\dd[2]{k}}{(2\pi)^{2}}\tr(\frac{-i\fsl{k} - M_{0}}{-k^{2} - M_{0}^{2}}\gamma^{01}M_{i}(p)\sigma^{i}\frac{-i(\fsl{k} - \fsl{p}) - M_{0}}{-(k - p)^{2} - M_{0}^{2}}\gamma^{01}M_{j}(-p)\sigma^{j}).
\end{align*}
The first-order term should be trivial. The topological term should thus arise from
\begin{align*}
	\Gamma =& \frac{i}{2(2\pi)^{4}}\inte{}{}\frac{\dd[2]{p}}{(2\pi)^{2}}\frac{\dd[2]{k}}{(2\pi)^{2}}\frac{M_{i}(p)M_{j}(-p)}{(-k^{2} - M_{0}^{2})(-(k - p)^{2} - M_{0}^{2})}\tr(\fsl{k}\gamma^{01}\sigma^{i}(\fsl{k} - \fsl{p})\gamma^{01}\sigma^{j}) \\
	       =&
\end{align*}
The new term in the action contains
\begin{align*}
	\inte{X}{}\pub{\phi}{\omega},
\end{align*}
where $X$ is spacetime and $\omega$ satisfies $\df{\omega} = 2\pi\text{Vol}(S^{3})$, the latter being the volume form on $S^{3}$ normalized to volume 1. This volume form is what is of interest.

A final model to study in one dimension is
\begin{align*}
	\lag = -i\overline{\Psi}\left(\fsl{\del{}{}} + M_{0}(x) + \sum\limits_{a = 2, 3, 5}iM_{a}(x)\gamma^{a}\right)\Psi
\end{align*}
for background mass fields. Confining the mass fields to $S^{3}$ we may write the above as
\begin{align*}
	\lag = -i\overline{\Psi}\left(\fsl{\del{}{}} + M\left(\cos(\alpha) + i\sin(\alpha)\sum\limits_{a = 2, 3, 5}m_{a}\gamma^{a}\right)\right)\Psi,\ \sum\limits_{a = 2, 3, 5}m_{a}^{2} = 1.
\end{align*}
We once again employ the perturbation approach to write the effective action as
\begin{align*}
	\Gamma =& -i\ln(\det(-i\left(\fsl{\del{}{}} + M\left(\cos(\alpha) + i\sin(\alpha)\sum\limits_{a = 2, 3, 5}m_{a}\gamma^{a}\right)\right))) \\
	       =& -i\tr(\ln(-i\left(\fsl{\del{}{}} + M\left(\cos(\alpha) + i\sin(\alpha)\sum\limits_{a = 2, 3, 5}m_{a}\gamma^{a}\right)\right))) \\
	 \approx& -i\tr(\ln(-i(\fsl{\del{}{}} + M)\left(1 + \frac{i\alpha\sum\limits_{a = 2, 3, 5}m_{a}\gamma^{a}}{\fsl{\del{}{}} + M}\right))) \\
	       =& C_{0} - i\tr(\ln(1 + \frac{i\alpha\sum\limits_{a = 2, 3, 5}m_{a}\gamma^{a}}{\fsl{\del{}{}} + M})).
\end{align*}
Because we are working with all Dirac matrices, we note that the only shot at obtaining a topological term is to consider a third-order term in the expansion. The Feynman diagram is shown in figure \ref{fig:third_order_fd}.

\begin{figure}[!ht]
	\centering
		\begin{tikzpicture}
			\begin{feynman}
				\vertex (a) at (0, 0);
				\vertex (b) at (2, 0);
				\vertex (c) at (4.6, 1);
				\vertex (d) at (5.9, 1.5);
				\vertex (e) at (4.6, -1);
				\vertex (f) at (5.9, -1.5);
				\diagram*{
					(a) --[scalar, momentum = $p$] (b) --[fermion, out = 90, in = 135, rmomentum = $k_{1}$] (c) --[scalar, rmomentum = $k_{2}$] (d),
					(f) --[scalar, momentum = $-p - k_{2}$] (e) --[fermion, out = -135, in = -90, momentum = $-k_{1} - p$] (b),
					(c) --[fermion, out = -45, in = 45, momentum = $k_{2} - k_{1}$] (e),
				};
			\end{feynman}
		\end{tikzpicture}
	\caption{Feynman diagram for the third-order term in the effective action.}
	\label{fig:third_order_fd}
\end{figure}

This is given by
\begin{align*}
	\Gamma =& -\frac{i}{3}\inte{}{}\frac{\dd[2]{p}}{(2\pi)^{2}}\frac{\dd[2]{k_{1}}}{(2\pi)^{2}}\frac{\dd[2]{k_{2}}}{(2\pi)^{2}}\tr(\frac{i\alpha\sum\limits_{a = 2, 3, 5}m_{a}\gamma^{a}}{\fsl{\del{}{}} + M}\frac{i\alpha\sum\limits_{b = 2, 3, 5}m_{b}\gamma^{b}}{\fsl{\del{}{}} + M}\frac{i\alpha\sum\limits_{c = 2, 3, 5}m_{c}\gamma^{c}}{\fsl{\del{}{}} + M}) \\
	       =& -\frac{1}{3}\inte{}{}\frac{\dd[2]{p}}{(2\pi)^{2}}\frac{\dd[2]{k_{1}}}{(2\pi)^{2}}\frac{\dd[2]{k_{2}}}{(2\pi)^{2}} \\
	        &\cdot\sum\limits_{a, b, c = 2, 3, 5}\alpha m_{a}(p)\alpha m_{b}(k_{2})\alpha m_{c}(-p - k_{2})\tr(\frac{i\fsl{k_{1}} - M}{-k_{1}^{2} - M^{2}}\gamma^{a}\frac{i(-\fsl{k_{1}} - \fsl{p}) - M}{-(-k_{1} - p)^{2} - M^{2}}\gamma^{b}\frac{i(\fsl{k_{2}} - \fsl{k_{1}}) - M}{-(k_{2} - k_{1})^{2} - M^{2}}\gamma^{c}) \\
	       =& -\frac{1}{3}\inte{}{}\frac{\dd[2]{p}}{(2\pi)^{2}}\frac{\dd[2]{k_{1}}}{(2\pi)^{2}}\frac{\dd[2]{k_{2}}}{(2\pi)^{2}}\sum\limits_{a, b, c = 2, 3, 5}\frac{M_{a}(p)M_{b}(k_{2})M_{c}(-p - k_{2})}{(-k_{1}^{2} - M^{2})(-(-k_{1} - p)^{2} - M^{2})(-(k_{2} - k_{1})^{2} - M^{2})} \\
	        &\cdot\tr((i\fsl{k_{1}} - M)\gamma^{a}(i(-\fsl{k_{1}} - \fsl{p}) - M)\gamma^{b}(i(\fsl{k_{2}} - \fsl{k_{1}}) - M)\gamma^{c}).
\end{align*}
The topological term corresponds to exactly one of $a, b, c$ being 5 and one of the numerators in the fraction being a mass. For example, setting $a = 5$ nets you
\begin{align*}
	 &-M\left(\tr(\gamma^{5}(-\fsl{k_{1}} - \fsl{p})\gamma^{b}(\fsl{k_{2}} - \fsl{k_{1}})\gamma^{c} + \fsl{k_{1}}\gamma^{5}\gamma^{b}(\fsl{k_{2}} - \fsl{k_{1}})\gamma^{c} + \fsl{k_{1}}\gamma^{5}(-\fsl{k_{1}} - \fsl{p})\gamma^{b}\gamma^{c})\right) \\
	=& -M\left(\tr((-k_{1} - p)_{\mu}(k_{2} - k_{1})_{\nu}\gamma^{5}\gamma^{\mu}\gamma^{b}\gamma^{\nu}\gamma^{c} + k_{1, \mu}(k_{2} - k_{1})_{\nu}\gamma^{\mu}\gamma^{5}\gamma^{b}\gamma^{\nu}\gamma^{c} + k_{1, \mu}(-k_{1} - p)_{\nu}\gamma^{\mu}\gamma^{5}\gamma^{\nu}\gamma^{b}\gamma^{c})\right) \\
	=& -M\left(\tr(-(-k_{1} - p)_{\mu}(k_{2} - k_{1})_{\nu}\gamma^{\mu}\gamma^{\nu}\gamma^{b}\gamma^{c}\gamma^{5} - k_{1, \mu}(k_{2} - k_{1})_{\nu}\gamma^{\nu}\gamma^{\mu}\gamma^{b}\gamma^{c}\gamma^{5} + k_{1, \mu}(-k_{1} - p)_{\nu}\gamma^{\nu}\gamma^{\mu}\gamma^{b}\gamma^{c}\gamma^{5})\right) \\
	=& -M\left(-(-k_{1} - p)_{\mu}(k_{2} - k_{1})_{\nu}\varepsilon^{\mu\nu bc} - k_{1, \mu}(k_{2} - k_{1})_{\nu}\varepsilon^{\nu\mu bc} + k_{1, \mu}(-k_{1} - p)_{\nu}\varepsilon^{\nu\mu bc}\right) \\
	=& -M\varepsilon^{\mu\nu bc}\left(-(-k_{1} - p)_{\mu}(k_{2} - k_{1})_{\nu} + k_{1, \mu}(k_{2} - k_{1})_{\nu} - k_{1, \mu}(-k_{1} - p)_{\nu}\right) \\
	=&  -M\varepsilon^{\mu\nu bc}\left(-(-k_{1} - p)_{\mu}(k_{2} - k_{1})_{\nu} - k_{1, \mu}(-k_{2} - p)_{\nu}\right).
\end{align*}
Approximating $k_{1}$ to be much larger than the other momenta, the second term produces no contribution. Left is
\begin{align*}
	\Gamma \supset& -\frac{M}{3}\sum\limits_{b, c = 2, 3}\varepsilon^{\mu\nu bc}\inte{}{}\frac{\dd[2]{p}}{(2\pi)^{2}}\frac{\dd[2]{k_{1}}}{(2\pi)^{2}}\frac{\dd[2]{k_{2}}}{(2\pi)^{2}}M_{5}(p)\frac{M_{b}(k_{2})M_{c}(-p - k_{2})(-k_{1} - p)_{\mu}(k_{2} - k_{1})_{\nu}}{(-k_{1}^{2} - M^{2})(-(-k_{1} - p)^{2} - M^{2})(-(k_{2} - k_{1})^{2} - M^{2})}.
\end{align*}
One term is symmetric in $\mu$ and $\nu$ and two are linear in $k_{1}$, meaning the only surviving term should be
\begin{align*}
	\Gamma \supset& \frac{M}{3}\sum\limits_{b, c = 2, 3}\varepsilon^{\mu\nu bc}\inte{}{}\frac{\dd[2]{p}}{(2\pi)^{2}}\frac{\dd[2]{k_{1}}}{(2\pi)^{2}}\frac{\dd[2]{k_{2}}}{(2\pi)^{2}}M_{5}(p)\frac{M_{b}(k_{2})M_{c}(-p - k_{2})p_{\mu}k_{2, \nu}}{(-k_{1}^{2} - M^{2})(-(-k_{1} - p)^{2} - M^{2})(-(k_{2} - k_{1})^{2} - M^{2})} \\
	       \approx& -\sum\limits_{b, c = 2, 3}\frac{M}{3}\varepsilon^{\mu\nu bc}\inte{}{}\frac{\dd[2]{p}}{(2\pi)^{2}}\frac{\dd[2]{k_{2}}}{(2\pi)^{2}}p_{\mu}M_{5}(p)k_{2, \nu}M_{b}(k_{2})M_{c}(-p - k_{2})\inte{}{}\frac{\dd[2]{k_{1}}}{(2\pi)^{2}}\frac{1}{(k_{1}^{2} + M^{2})^{3}} \\
	             =& -\frac{1}{3M^{3}}\frac{\Gamma\left(\frac{5}{2}\right)\Gamma\left(\frac{1}{2}\right)}{2\sqrt{\pi}\Gamma\left(3\right)\Gamma\left(\frac{1}{2}\right)}\sum\limits_{b, c = 2, 3}\varepsilon^{\mu\nu bc}\inte{}{}\frac{\dd[2]{p}}{(2\pi)^{2}}\frac{\dd[2]{k_{2}}}{(2\pi)^{2}}p_{\mu}M_{5}(p)k_{2, \nu}M_{b}(k_{2})M_{c}(-p - k_{2}) \\
	             =& -\frac{1}{3M^{3}}\frac{3}{16}\sum\limits_{b, c = 2, 3}\varepsilon^{\mu\nu bc}\inte{}{}\frac{\dd[2]{p}}{(2\pi)^{2}}\frac{\dd[2]{k_{2}}}{(2\pi)^{2}}p_{\mu}M_{5}(p)k_{2, \nu}M_{b}(k_{2})M_{c}(-p - k_{2}).
\end{align*}
Using the facts that the values of $b, c$ are disjoint from those of $\mu, \nu$, allo we have that the real space term is
\begin{align*}
	\Gamma \supset \frac{1}{16M^{3}}\sum\limits_{b, c = 2, 3}\varepsilon^{\mu\nu}\varepsilon^{bc}\inte{}{}\dd[2]{x}M_{c}\del{}{\mu}M_{5}\del{}{\nu}M_{b}.
\end{align*}

To understand the systematics of the above calculations, let us adopt aliases for the momenta and write the integrand as
\begin{align*}
	 &\sum\limits_{a, b, c = 2, 3, 5}\frac{M_{a}(p)M_{b}(k_{2})M_{c}(-p - k_{2})}{(-k_{1}^{2} - M^{2})(-(-k_{1} - p)^{2} - M^{2})(-(k_{2} - k_{1})^{2} - M^{2})} \\
	 &\cdot\tr((i\fsl{k_{1}} - M)\gamma^{a}(i(-\fsl{k_{1}} - \fsl{p}) - M)\gamma^{b}(i(\fsl{k_{2}} - \fsl{k_{1}}) - M)\gamma^{c}) \\
	=& \sum\limits_{a, b, c = 2, 3, 5}\frac{M_{a}(u_{1})M_{b}(u_{2})M_{c}(u_{3})}{(-v_{1}^{2} - M^{2})(-v_{2}^{2} - M^{2})(-v_{3}^{2} - M^{2})}\tr((i\fsl{v_{1}} - M)\gamma^{a}(i\fsl{v_{2}} - M)\gamma^{b}(i\fsl{v_{3}} - M)\gamma^{c}).
\end{align*}
Because $v_{1}$ is explicitly integrated over, we can deduce some more restrictions on the necessary considerations to identify topological terms. In addition to fixing exactly one latin index to $5$ we may also take the first factor to be the one contributing the factor $M$. Thus this factor amounts to
\begin{align*}
	 &Mv_{2, \mu}v_{3, \nu}\sum\limits_{a, b, c = 2, 3, 5}\frac{M_{a}(u_{1})M_{b}(u_{2})M_{c}(u_{3})}{(-v_{1}^{2} - M^{2})(-v_{2}^{2} - M^{2})(-v_{3}^{2} - M^{2})}\tr(\gamma^{a}\gamma^{\mu}\gamma^{b}\gamma^{\nu}\gamma^{c}) \\
	=& Mv_{2, \mu}v_{3, \nu}\sum\limits_{b, c = 2, 3}\frac{M_{5}(u_{1})M_{b}(u_{2})M_{c}(u_{3})\varepsilon^{\mu b\nu c} + M_{a}(u_{1})M_{5}(u_{2})M_{c}(u_{3})\varepsilon^{\nu c\mu b} + M_{a}(u_{1})M_{b}(u_{2})M_{5}(u_{3})\varepsilon^{\mu b\nu c}}{(-v_{1}^{2} - M^{2})(-v_{2}^{2} - M^{2})(-v_{3}^{2} - M^{2})} \\
	=& -Mv_{2, \mu}v_{3, \nu}\sum\limits_{b, c = 2, 3}\varepsilon^{\mu\nu bc}\frac{M_{5}(u_{1})M_{b}(u_{2})M_{c}(u_{3}) - M_{b}(u_{1})M_{5}(u_{2})M_{c}(u_{3}) + M_{b}(u_{1})M_{c}(u_{2})M_{5}(u_{3})}{(-v_{1}^{2} - M^{2})(-v_{2}^{2} - M^{2})(-v_{3}^{2} - M^{2})}.
\end{align*}
To proceed we use the relations
\begin{align*}
	v_{2} = -v_{1} - u_{1},\ v_{3} = u_{2} - v_{1},
\end{align*}
which follow from the definition, to write the above as
\begin{align*}
	&M\varepsilon^{\mu\nu}(v_{1} + u_{1})_{\mu}(u_{2} - v_{1})_{\nu}\sum\limits_{b, c = 2, 3}\varepsilon^{bc}\frac{M_{5}(u_{1})M_{b}(u_{2})M_{c}(u_{3}) - M_{b}(u_{1})M_{5}(u_{2})M_{c}(u_{3}) + M_{b}(u_{1})M_{c}(u_{2})M_{5}(u_{3})}{(-v_{1}^{2} - M^{2})(-v_{2}^{2} - M^{2})(-v_{3}^{2} - M^{2})}.
\end{align*}
Once again we are left with only the contribution
\begin{align*}
	M\varepsilon^{\mu\nu}u_{1, \mu}u_{2, \nu}\sum\limits_{b, c = 2, 3}\varepsilon^{bc}\frac{M_{5}(u_{1})M_{b}(u_{2})M_{c}(u_{3}) - M_{b}(u_{1})M_{5}(u_{2})M_{c}(u_{3}) + M_{b}(u_{1})M_{c}(u_{2})M_{5}(u_{3})}{(-v_{1}^{2} - M^{2})(-v_{2}^{2} - M^{2})(-v_{3}^{2} - M^{2})},
\end{align*}
which in real space is proportional to
\begin{align*}
	\varepsilon^{\mu\nu}\sum\limits_{b, c = 2, 3}\varepsilon^{bc}\left(M_{c}\del{}{\mu}M_{5}\del{}{\nu}M_{b} - M_{c}\del{}{\mu}M_{b}\del{}{\nu}M_{5} + M_{5}\del{}{\mu}M_{b}\del{}{\nu}M_{c}\right).
\end{align*}
Constants of proportionality are the same, hence the full topological term is
\begin{align*}
	\Gamma =& \frac{1}{16M^{3}}\sum\limits_{b, c = 2, 3}\inte{}{}\dd[2]{x}\varepsilon^{\mu\nu}\varepsilon^{bc}\left(M_{c}\del{}{\mu}M_{5}\del{}{\nu}M_{b} - M_{c}\del{}{\mu}M_{b}\del{}{\nu}M_{5} + M_{5}\del{}{\mu}M_{b}\del{}{\nu}M_{c}\right) \\
	       =& \frac{1}{16M^{3}}\sum\limits_{a, b, c = 2, 3, 5}\inte{}{}\dd[2]{x}\varepsilon^{\mu\nu}\varepsilon^{abc}M_{a}\del{}{\mu}M_{b}\del{}{\nu}M_{c}.
\end{align*}
We can take this one step further by taking the mass fields to be a map from spacetime to $S^{3}$ and introducing the 2-form $\omega$ with components $\omega_{ab} = \varepsilon_{cab}M^{a}$ to write the above as
\begin{align*}
	\Gamma = \frac{1}{16M^{3}}\inte{}{}\pub{M}{\omega},
\end{align*}
with the integral still being over spacetime.

Let us study $\omega$ more closely. Its exterior derivative is
\begin{align*}
	\df{\omega} =& \frac{1}{2}\del{}{d}\varepsilon_{cab}M^{c}e^{dab} = \frac{1}{2}\varepsilon_{dab}e^{dab},
\end{align*}
which is almost the volume form on $S^{3}$.

What if we were to add coupling to a gauge field too? The effective action would be
\begin{align*}
	\Gamma =& -i\tr(\ln(1 + \frac{ie\fsl{A} + i\sum\limits_{a = 2, 3, 5}M_{a}\gamma^{a}}{\fsl{\del{}{}} + M})).
\end{align*}
To order $3$, the structure of the Feynman diagram is identical, hence the trace part of the effective action is
\begin{align*}
	-i\tr((i\fsl{k_{1}} - M)(e\fsl{A} + M_{a}\gamma^{a})(i(-\fsl{k_{1}} - \fsl{p}) - M)(e\fsl{A} + M_{b}\gamma^{a})(i(\fsl{k_{2}} - \fsl{k_{1}}) - M)(e\fsl{A} + M_{c}\gamma^{c})).
\end{align*}
Because all of the latin-indiced Dirac matrices are needed, however, the gauge field does not appear in any topological terms. To get a topological term, we can instead consider a fourth-order term, represented by the Feynman diagram in figure 

\begin{figure}[!ht]
	\centering
	\begin{tikzpicture}
		\begin{feynman}
			\vertex (a) at (-4, 4);
			\vertex (a1) at (-2, 2);
			\vertex (b) at (4, 4);
			\vertex (b1) at (2, 2);
			\vertex (c) at (4, -4);
			\vertex (c1) at (2, -2);
			\vertex (d) at (-4, -4);
			\vertex (d1) at (-2, -2);
			\diagram*{
				(a) --[scalar, momentum = $p_{1}$] (a1) --[fermion, momentum = $k$] (b1) --[scalar, rmomentum = $p_{2}$] (b),
				(c) --[scalar, momentum = $p_{3}$] (c1) --[fermion, momentum = $p_{2} + p_{3} + k$] (d1) --[scalar, rmomentum = $-p_{1} - p_{2} - p_{3}$] (d),
				(b1) --[fermion, momentum = $p_{2} + k$] (c1),
				(d1) --[fermion, momentum = $-p_{1} + k$] (a1),
			};
		\end{feynman}
	\end{tikzpicture}
	\caption{Feynman diagram for the fourth-order term in the effective action.}
	\label{fig:fourth_order_1d_fd}
\end{figure}

At this point we can deduce what will happen. The replacement of a momentum with a gauge field in the trace (which is needed to produce a topological term) while leaving the same number of mass fields causes the integrand to be symmetric in the mass indices and makes the topological term vanish. This argument also implies the necessary conditions for the gauge fields to appear in response terms. We expect all the fields corresponding to anticommuting mass terms to appear exactly once, as well as all the derivatives, so in order for the gauge fields to appear, the spacetime dimension must be at least equal to the number of mass fields.

\paragraph{A New Model}
The models studied above have connected $d$-dimensional systems to response terms described by integrals of up to $d + 1$-forms. To relate the study of these theories to the higher Berry curvature, we will instead need to find a response term described by the integral of a $d + 2$-form. Let us consider spatial dimension $1$ and introduce mass terms according to
\begin{align*}
	M = M_{0} + i\gamma^{01}\sum\limits_{l = 1}^{5}M_{l}\Gamma^{l}.
\end{align*}
The matrices $\Gamma$ represent a Clifford algebra and act in flavor space, and we take the mass fields to lie on $S^{5}$. The Lagrangian of the model is
\begin{align*}
	\lag = -i\overline{\Psi}\left(\fsl{\del{}{}} + M\right)\Psi = -i\overline{\Psi}\left(\fsl{\del{}{}} + M_{0} + i\gamma^{01}\sum\limits_{l = 1}^{5}M_{l}\Gamma^{l}\right)\Psi.
\end{align*}

The idea underlying this model follows \href{https://arxiv.org/pdf/hep-th/9911025.pdf}{Abanov and Wiegmann}. They construct models with mass fields confined to $S^{d}$ or $S^{d + 1}$ and show that the topological response terms are related to (the pullbacks of) $d$-forms and $d + 1$-forms respectively. A first obvious attempt in one dimension is therefore to use mass fields on $S^{3}$ (and the simple way to do this just so happens to be Abanov and Wiegmann's A-series model in $d = 3$), but we saw that this only produced a topological term given by a 2-form. This model attempts to fix this by extending the mass fields in a way such that, had you done it in $d = 3$, it would take the response term from being given by a 3-form to a 4-form. The hope is that it will achieve a similar result.

At this point it is pertinent to ask whether this attempt really stood any chance. The answer is no, and for a very simple reason. Looking at the above, the appearance of the pullback was no coincidence; it arrived precisely because of the form of the effective action. By its definition the pullback does not affect the rank of any tensor. As the effective action is given by an integral over spacetime, it follows that any form appearing in it must exist on spacetime, and the highest form in $d + 1$-dimensional spacetime is a $d + 1$-form. Note that this argument has no reliance on the structure of the mass fields. Nevertheless, we show the attempt below.

The effective action is
\begin{align*}
	\Gamma =& -i\ln(\det(-i\left(\fsl{\del{}{}} + M_{0} + i\gamma^{01}\sum\limits_{l = 1}^{5}M_{l}\Gamma^{l}\right))) \\
	       =& -i\tr(\ln(-i\left(\fsl{\del{}{}} + M_{0} + i\gamma^{01}\sum\limits_{l = 1}^{5}M_{l}\Gamma^{l}\right))) \\
	       =& C_{0} - i\tr(\ln(1 + \frac{m + i\gamma^{01}\sum\limits_{l = 1}^{5}M_{l}\Gamma^{l}}{\fsl{\del{}{}} + M})) \\
	       =& C_{0} - i\tr(\ln(1 + \frac{(\fsl{\del{}{}} - M)\left(m + i\gamma^{01}\sum\limits_{l = 1}^{5}M_{l}\Gamma^{l}\right)}{\del{2}{} - M^{2}})),
\end{align*}
where we have introduced the mass perturbation $m = M_{0} - M$, with $M$ being a fixed mass scale parameter. The topological term comes from the fifth-order expansion, with the Feynman diagram shown in figure \ref{fig:fifth_order_fd}.

\begin{figure}[!ht]
	\centering
	\begin{tikzpicture}
		\begin{feynman}
			\vertex (a) at (-6, 0);
			\vertex (a1) at (-2, 0);
			\vertex (b) at (-1.86, 5.7);
			\vertex (b1) at (-0.62, 1.9);
			\vertex (c) at (4.8, 3.6);
			\vertex (c1) at (1.62, 1.18);
			\vertex (d) at (4.8, -3.6);
			\vertex (d1) at (1.62, -1.18);
			\vertex (e) at (-1.86, -5.7);
			\vertex (e1) at (-0.62, -1.9);
			\diagram*{
				(a) --[scalar, momentum = $p_{1}$] (a1) --[fermion, momentum = $k$] (b1) --[scalar, rmomentum = $p_{2}$] (b),
				(c) --[scalar, momentum = $p_{3}$] (c1) --[fermion, momentum = $p_{2} + p_{3} + k$] (d1) --[scalar, rmomentum = $p_{4}$] (d),
				(b1) --[fermion, momentum = $p_{2} + k$] (c1),
				(d1) --[fermion, momentum = $p_{2} + p_{3} + p_{3} + k$] (e1) --[scalar, rmomentum' = $-p_{1} - p_{2} - p_{3} - p_{4}$] (e),
				(e1) --[fermion, momentum = $-p_{1} + k$] (a1),
			};
		\end{feynman}
	\end{tikzpicture}
	\caption{Feynman diagram for the third-order term in the effective action.}
	\label{fig:fifth_order_fd}
\end{figure}

This translates to
\begin{align*}
	\Gamma =&  -\frac{i}{5}\inte{}{}\frac{\dd[2]{p_{1}}}{(2\pi)^{2}}\dots\frac{\dd[2]{k}}{(2\pi)^{2}}\tr(\left(\frac{(\fsl{\del{}{}} - M)\left(m + i\gamma^{01}\sum\limits_{l = 1}^{5}M_{l}\Gamma^{l}\right)}{\del{2}{} - M^{2}}\right)^{5}) \\
	 \supset&  \frac{1}{5}\sum\limits_{l_{i} = 1}^{5}\inte{}{}\frac{\dd[2]{p_{1}}}{(2\pi)^{2}}\dots\frac{\dd[2]{k}}{(2\pi)^{2}}\text{tr}\left(\frac{(i\fsl{k} - M)\gamma^{01}M_{l_{1}}\Gamma^{l_{1}}}{-k^{2} - M^{2}}\frac{(i(\fsl{p}_{2} + \fsl{k}) - M)\gamma^{01}M_{l_{2}}\Gamma^{l_{2}}}{-(p_{2} + k)^{2} - M^{2}}\frac{(i(\fsl{p}_{2} + \fsl{p}_{3} + \fsl{k}) - M)\gamma^{01}M_{l_{3}}\Gamma^{l_{3}}}{-(p_{2} + p_{3} + k)^{2} - M^{2}} \right. \\
	        &\cdot \left.\frac{(i(\fsl{p}_{2} + \fsl{p}_{3} + \fsl{p}_{4} + \fsl{k}) - M)\gamma^{01}M_{l_{4}}\Gamma^{l_{4}}}{-(p_{2} + p_{3} + p_{4} + k)^{2} - M^{2}}\frac{(i(-\fsl{p}_{1} + \fsl{k}) - M)\gamma^{01}M_{l_{5}}\Gamma^{l_{5}}}{(-p_{1} + k)^{2} - M^{2}}\right).
\end{align*}
Let us now consider the contents of the trace. We are looking for topological terms, which appear in the presence of all $\Gamma$ and all $\gamma$ appearing exactly once. The matrices will produce a Levi-Civita tensor, meaning any contributions of orders $1$ or $2$ in $k$ will vanish. As such the topological term is given by
\begin{align*}
	\Gamma \supset& -\frac{M}{5}\sum\limits_{l_{i} = 1}^{5}\inte{}{}\frac{\dd[2]{p_{1}}}{(2\pi)^{2}}\dots\frac{\dd[2]{k}}{(2\pi)^{2}}\text{tr}\left(\frac{M_{l_{1}}\Gamma^{l_{1}}}{-k^{2} - M^{2}}\frac{(i(\fsl{p}_{2} + \fsl{k}) - M)\gamma^{01}M_{l_{2}}\Gamma^{l_{2}}}{-(p_{2} + k)^{2} - M^{2}}\frac{(i(\fsl{p}_{2} + \fsl{p}_{3} + \fsl{k}) - M)\gamma^{01}M_{l_{3}}\Gamma^{l_{3}}}{-(p_{2} + p_{3} + k)^{2} - M^{2}} \right. \\
	              &\cdot \left.\frac{(i(\fsl{p}_{2} + \fsl{p}_{3} + \fsl{p}_{4} + \fsl{k}) - M)\gamma^{01}M_{l_{4}}\Gamma^{l_{4}}}{-(p_{2} + p_{3} + p_{4} + k)^{2} - M^{2}}\frac{(i(-\fsl{p}_{1} + \fsl{k}) - M)M_{l_{5}}\Gamma^{l_{5}}}{(-p_{1} + k)^{2} - M^{2}}\right).
\end{align*}
The contents of the trace are
\begin{align*}
	 &\tr(\Gamma^{l_{1}}(i(\fsl{p}_{2} + \fsl{k}) - M)\gamma^{01}\Gamma^{l_{2}}(i(\fsl{p}_{2} + \fsl{p}_{3} + \fsl{k}) - M)\gamma^{01}\Gamma^{l_{3}}(i(\fsl{p}_{2} + \fsl{p}_{3} + \fsl{p}_{4} + \fsl{k}) - M)\gamma^{01}\Gamma^{l_{4}}(i(-\fsl{p}_{1} + \fsl{k}) - M)\Gamma^{l_{5}}) \\
	=& \tr(\Gamma^{l_{1}}\Gamma^{l_{2}}\Gamma^{l_{3}}\Gamma^{l_{4}}\Gamma^{l_{5}})\tr((i(\fsl{p}_{2} + \fsl{k}) - M)\gamma^{01}(i(\fsl{p}_{2} + \fsl{p}_{3} + \fsl{k}) - M)\gamma^{01}(i(\fsl{p}_{2} + \fsl{p}_{3} + \fsl{p}_{4} + \fsl{k}) - M)\gamma^{01}(i(-\fsl{p}_{1} + \fsl{k}) - M)),
\end{align*}
exploiting the product structure of the operators. The case where all Dirac matrices appear exactly once correspond to exactly two momenta appearing, meaning this topological term too will have a pullback of a 2-form onto spacetime.

\paragraph{Extending to Synthetic Dimensions}