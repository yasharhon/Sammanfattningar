\section{Topology}

\paragraph{Topological Spaces}
Let $X$ be a set and $T = \{U_{i} | i\in I\}$ be a collection of subsets of $X$ ($I$ is some set of indices). The pair $(X, T)$ (sometimes we only explicitly write $X$) is defined as a topological space if
\begin{itemize}
	\item $\emptyset,\ X\in T$.
	\item If $J$ is any subcollection of $I$, the family $\{U_{j} | j\in J\}$ satisfies
	\begin{align*}
		\bigcup\limits_{j\in J}U_{j}\in T.
	\end{align*}
	\item If $J$ is any finite subcollection of $I$, the family $\{U_{j} | j\in J\}$ satisfies
	\begin{align*}
		\bigcap\limits_{j\in J}U_{j}\in T.
	\end{align*}
\end{itemize}
If the two satisfy the definition, we say that $T$ gives a topology to $X$. The $U_{i}$ are called its open sets.

Two cases of little interest are $T = \{\emptyset,\ X\}$ and $T$ being the collection of all subsets of $X$. The two are called the trivial and discrete topologies respectively.

\paragraph{Metrics}
A metric is a map $d: X\times X\to \R$ that satisfies
\begin{itemize}
	\item $d(x, y) = d(y, x)$.
	\item $d(x, x) \geq 0$, with equality applying if and only if $x = y$.
	\item $d(x, y) + d(y, z) \geq d(x, )z$.
\end{itemize}

\paragraph{Metric Spaces}
Suppose $X$ is endowed with a metric. The collection of open disks
\begin{align*}
	U_{\varepsilon} = \{x\in X | d(x, x_{0}) < \varepsilon\}
\end{align*}
then gives a topology to $X$ called the metric topology. The pair forms a metric space.

\paragraph{Continuous Maps}
A map between two topological spaces $X$ and $Y$ is continuous if its inverse maps an open set in $Y$ to an open set in $X$.

\paragraph{Neighborhoods}
$N$ is a neighborhood of $x$ if it is a subset of $X$ and $x$ belongs to at least one open set contained within $N$.

\paragraph{Hausdorff Spaces}
A topological space is a Hausdorff space if, for any two points $x, y$, there exists neighborhoods $U_{x}, U_{y}$ of the two points that do not intersect. This is an important type of topological space, as examples in physics are practically always within this category.

\paragraph{Homeomorphisms}
A homeomorphism (not to be confused with a homomorphism, however hard it may be) is a continuous map between two topological spaces with a continuous inverse. Two topological spaces are homeomorphic if there exists a homeomorphism between them.

A specific case if homeomorphicity is diffeomorphicity, which applies between two manifolds between which there is a smooth map.

\paragraph{Homotopy Types}
Two topological spaces belong to the same homotopy type if there exists a continuous map from one to the other. This is a more relaxed version of homeomorphicity, as we no longer require the map to be invertible.

\paragraph{Donuts and Coffee Mugs}
Homeomorphicity defines an equivalence relation between topological spaces. This means that we can define topological spaces into categories based on homeomorphicity.

We are now in a position to introduce the poor man's notion of topology, which considers two bodies as equivalent if one can be deformed into the other without touching two parts of the surface or tearing a part of the body. These continuous deformations correspond to homeomorphisms, but we will try to keep the discussions more to the abstract.

\paragraph{Topological Invariants}
An important question pertaining to this division of topological spaces is what separates the different categories. One possible answer is so-called topological invariants, quantities which are invariant under homeomorphism. The issue with this answer is that the full set of topological invariants has not been identified, hence they can only be used to verify that two topological spaces belong to different categories.

\paragraph{Homotopy and Homotopy Classes}
Consider maps between two manifolds. Maps that can be continuously deformed into each other are said to be homotopic. This concept is related to that of homeomorphicity, and as such there exists topological invariants for such maps too. These divide maps into homotopy classes. A particular case is $\pi_{n}(M)$, which is the set of homotopy classes of maps from $S^{n}$ to $M$.

\paragraph{Chain Complices}
A chain complex is a doubly infinite sequence of mathematical objects related in a chain by a structure-preserving map $\del{}{}$. The structure is thus
\begin{align*}
	\dots C_{p - 1} \xrightarrow{\del{}{p + 1}} C_{p 1} \xrightarrow{\del{}{p - 1}} C_{p + 1} \dots
\end{align*}
The maps satisfy $\del{}{p - 1}\del{}{p} = 0$.

\paragraph{Homology}
Restricting ourselves to manifolds, consider the boundary operator as a link between manifolds of different dimension, creating a chain complex. The $n$th homology group is then defined as
\begin{align*}
	H_{n} = \ker(\del{}{n}) / \text{im}(\del{}{n + 1}).
\end{align*}
Alternatively, by defining $\ker(\del{}{n}) = Z_{n}$ and $\text{im}(\del{}{n + 1}) = B_{n}$, we can write this as
\begin{align*}
	H_{n} = Z_{n} / B_{n}.
\end{align*}
Its elements are homology classes, which are isomorphic to manifolds which themselves are boundaries of different manifolds.

\paragraph{Cohomology}
To define (de Rham) cohomology, we first introduce the set of closed $r$-forms $Z^{r}(M)$ and the set of exact $r$-forms $B^{r}(M)$. The cohomology group is then
\begin{align*}
	H^{r}(M) = \frac{Z^{r}(M)}{B^{r}(M)}.
\end{align*}
This divides up the forms in $Z^{r}(M)$ into classes which differ by an exact form. Two forms belonging to the same class are cohomologous.

\paragraph{Relating Homology and Cohomology}
What is so co about cohomology? Note that differential forms and manifolds fit very naturally into the language of chain complices - namely, the exterior derivative relates forms of different rank in a chain (forms with too many or a negative number of indices are identically zero), and the boundary operator relates manifolds of different dimension in a chain. The integral of a form over a manifold then induces the product
\begin{align*}
	\braket{\omega}{M} = \inte{M}{}\omega.
\end{align*}
Stokes' theorem then reads $\braket{\df{\omega}}{M} = \braket{\omega}{\bound{M}}$. By defining the pairing with a sum of manifolds to be a sum of the integrals over the two manifolds, the pairing is bilinear, and we then have for $\omega\in Z^{n}$ and $M\in Z_{n}$ that
\begin{align*}
	\braket{\omega + \df{\chi}}{M + \bound{N}} =& \braket{\omega + \df{\chi}}{M} + \braket{\omega + \df{\chi}}{\bound{N}} \\
	=& \braket{\omega}{M} + \braket{\df{\chi}}{M} + \braket{\df{(\omega + \df{\chi})}}{N} \\
	=& \braket{\omega}{M} + \braket{\chi}{\bound{M}} \\
	=& \braket{\omega}{M}.
\end{align*}
This pairing then splits up its result depending on what representative of the homology and cohomology classes are fed into it. This also means, in some sense, that $H^{n}$ belongs to the dual space of $H_{n}$. There are some subtleties to completely identifying the two as each other, however.

\paragraph{Retractability}
A domain $\Omega$ is retractable to $O$ if there exists a smooth map $\phi_{t}$ on $\Omega$ parametrized by $t$ such that $\phi_{1}(x) = x$ and $\phi_{0}(x) = O$.

\paragraph{Inverting the Exterior Derivative}
It holds that $\df{}^{2} = 0$, and thus one might believe that for every form $\omega$ such that $\df{\omega} = 0$ there exists a form $\chi$ such that $\omega = \df{\chi}$. This will, however, turn out to depend on the topological properties of the underlying space.

Assuming the underlying space to be retractable to $O$, we have $\pub{\phi_{1}}{\omega} = \omega$ and $\pub{\phi_{0}}{\omega} = 0$ for some closed form $\omega$. Define $\eta$ to be the vector tangential to the coordinate flow as $t$ is varied. The fact that $\df{\omega} = 0$ implies that $\df{\pub{\phi_{t}}{\omega}} = 0$. By the definition of the Lie derivative we have $\dv{t}\pub{\phi_{t}}{\omega} = \lied{\eta}{(\pub{\phi_{t}}{\omega})}$. We then have
\begin{align*}
	\lied{\eta}{(\pub{\phi_{t}}{\omega})} = (i_{\eta}\df{} + \df{}i_{\eta})(\pub{\phi_{t}}{\omega}) = \df{(i_{\eta}(\pub{\phi_{t}}{\omega}))}.
\end{align*}
Integrating with respect to $t$ we finally find
\begin{align*}
	\omega = \df{\inte{0}{1}\dd{t}i_{\eta}(\pub{\phi_{t}}{\omega})},
\end{align*}
and we have thus solved the problem. In the case of spaces which are not retractible, there are simple counterexamples to be found. One example would be $\sin(\theta)\df{\theta}\df{\phi}$ on $S^{2}$.

In the language of cohomology we can phrase this very concisely. If some form $\omega$ is the exterior derivative of another, it must follow that it is cohomologous to the zero form. This implies that if the exterior derivative is invertible, then $H^{r}(M) = \{0\}$.

\paragraph{Deligne-Beilinson Cohomology}
To define Deligne-Beilinson cohomology, we first need some other concepts. First, a fiber bundle is a structure $(E, B, \pi, F)$, with $E, B$ and $F$ being topological spaces and $\pi: E\to B$ a surjection. $B$ is called the base and $F$ the fiber. A principal bundle is a special case of a fiber bundle where $F$ is a (Lie) group.

Consider now some manifold $M$, an open cover $\{U_{\alpha}\}_{\alpha\in I}$ and a principal bundle $P(M, \text{U}(1))$. The bundle at different points is related by a transfer function, which generally is a homomorphism of the group onto itself, and in this case is given by
\begin{align*}
	g_{\alpha\beta} = e^{2\pi i\Lambda_{\alpha\beta}},
\end{align*}
with $\Lambda_{\alpha\beta}$ being some smooth function on $U_{\alpha}\cap U_{\beta}$. For consistency we require $g_{\alpha\beta}g_{\beta\gamma}g_{\gamma\alpha} = 1$, translating to
\begin{align*}
	\Lambda_{\alpha\beta} + \Lambda_{\beta\gamma} + \Lambda_{\gamma\alpha} = n_{\alpha\beta\gamma}\in\mathbb{Z}.
\end{align*}
These numbers tautologically satisfy
\begin{align*}
	n_{\beta\gamma\delta} - n_{\alpha\gamma\delta} + n_{\alpha\beta\delta} - n_{\alpha\beta\gamma} = 0.
\end{align*}
Built into this is an ambiguity found by the replacement
\begin{align*}
	g_{\alpha\beta} \to h_{\alpha}g_{\alpha\beta}h_{\beta}^{-1}\implies \Lambda_{\alpha\beta} \to \Lambda_{\alpha\beta} + \xi_{\alpha} - \xi_{\beta}.
\end{align*}

We can now introduce a U(1) connection on $M$ as a set of local 1-forms $A_{\alpha}$ such that
\begin{align*}
	A_{\beta} - A_{\alpha} = \df{\Lambda_{\alpha\beta}}.
\end{align*}
Combining the connection with the other objects to one tuple $\mathbf{A} = (A_{\alpha}, \Lambda_{\alpha\beta}, n_{\alpha\beta\gamma})$, called a DB cocycle, we can also define operators $\delta$ mapping its components to objects defined on higher-order overlaps according to
\begin{align*}
	(\delta_{0}A)_{\alpha\beta}             &= A_{\beta} - A_{\alpha} = \df[0]{\Lambda_{\alpha\beta}}, \\
	(\delta_{1}\Lambda)_{\alpha\beta\gamma} &= \Lambda_{\beta\gamma} - \Lambda_{\alpha\gamma} + \Lambda_{\alpha\beta} = \df[-1]{n_{\alpha\beta\gamma}}, \\
	(\delta_{2}n)_{\alpha\beta\gamma\delta}  &= n_{\beta\gamma\delta} - n_{\alpha\gamma\delta} + n_{\alpha\beta\delta} - n_{\alpha\beta\gamma} = 0.
\end{align*}
Note the extension of the exterior derivative to be an extension of numbers to functions. The generalization of the operator $\delta$ to quantities defined on intersections of higher degree is
\begin{align*}
	(\delta_{n}A)_{i_{0}\dots i_{n}} = A_{i_{1}\dots i_{n}} + \sum\limits_{j = 1}^{n - 1}(-1)^{j}A_{i_{0}\dots i_{j - 1}i_{j + 1}\dots i_{n}} + (-1)^{n}A_{i_{0}\dots i_{n - 1}}.
\end{align*}
It holds that $\delta_{n + 1}\delta_{n} = 0$. To prove that this is the case, consider some index sequence of length $n - 1$. In the sum appearing after $\delta_{n + 1}$, this sequence appears exactly twice, each time split by some other index. Denote the number of steps by which this split moves between the terms as $k$. If $k$ is odd, the terms appear with the same sign. To find the terms containing only the desired sequence, we now evaulate the underlying sum. Because $k$ is odd and one index has already been removed, these two contributions must therefore cancel. If $k$ is instead odd, the argument is identical upon permuting a few words.

The ambiguities in the components of $\mathbf{A}$ can be summarized as an invariance under
\begin{align*}
	A_{\alpha}            &\to A_{\alpha} + \df[0]{\xi_{\alpha}}, \\
	\Lambda_{\alpha\beta} &\to \Lambda_{\alpha\beta} + \xi_{\beta} - \xi_{\gamma} - \df{m_{\alpha\beta}} = \Lambda_{\alpha\beta} + (\delta_{0}\xi)_{\alpha\beta} - \df[-1]{m_{\alpha\beta}}, \\
	n_{\alpha\beta\gamma} &\to n_{\alpha\beta\gamma} - m_{\beta\gamma} + m_{\alpha\gamma} - m_{\alpha\beta} = n_{\alpha\beta\gamma} - (\delta_{1}m)_{\alpha\beta\gamma}.
\end{align*}
This is very similar to a gauge transformation, which is a key motivation for developing this framework. By introducing operators $\delta_{n}$ and $\df[n]{}$ acting on the appropriate components of $\mathbf{A}$, the ambiguity and connection between the components can be written as
\begin{align*}
	D_{1, 1}\mathbf{A} = 0,\ \mathbf{A} \to \mathbf{A} + D_{0, 1}\Xi,
\end{align*}
with
\begin{align*}
	D_{1, 1} = (\delta_{0} + 0) - (\delta_{1} + \df[0]{}) + (\delta_{2} + \df[-1]{}),\ D_{0, 1} = (\delta_{0} + \df[0]{}) - (\delta_{1} + \df[-1]{}),
\end{align*}
and $\Xi$ containing a number and a 0-form. $D_{0, 1}\Xi$ is called a DB coboundary. The indexation is defined as such: the second index specifies the order of the exterior derivative at which one truncates to 0 and the first index specifies the degree of the overlap on which the corresponding object lives. In our case, $D_{1, 1}$ means that it acts on objects with 1-forms defined on a degree-1 overlap and $D_{1, 1}$ means that it acts on objects with functions defined on a degree-1 overlap.

Considering the composition of these operators, we have
\begin{align*}
	D_{1, 1}D_{0, 1}\Xi =& D_{1, 1}D_{0, 1}\mqty[
		\xi_{\alpha} \\
		m_{\alpha\beta}
	] = D_{1, 1}\mqty[
		\df[0]{\xi_{\alpha}} \\
		(\delta_{0}\xi)_{\alpha\beta} - \df[-1]{m}_{\alpha\beta} \\
		-(\delta_{1}m)_{\alpha\beta\gamma}
	] \\
	=& \mqty[
		0 \\
		(\delta_{0}\df[0]{\xi})_{\alpha\beta} - \df[0]\left(\delta_{0}\xi - \df[-1]m\right)_{\alpha\beta} \\
		-(\delta_{1}(\delta_{0}\xi - \df[-1]m))_{\alpha\beta\gamma} - (\df[-1]\delta_{1}m)_{\alpha\beta\gamma} \\
		-(\delta_{2}\delta_{1}m)_{\alpha\beta\gamma}
	] \\
	=& \mqty[
		0 \\
		(\delta_{0}\df[0]{\xi})_{\alpha\beta} - \df[0]\left(\delta_{0}\xi\right)_{\alpha\beta} \\
		(\delta_{1}\df[-1]m)_{\alpha\beta\gamma} - (\df[-1]\delta_{1}m)_{\alpha\beta\gamma} \\
		0
	],
\end{align*}
As the exterior derivative commutes with $\delta$, this must identically vanish. The cohomology group, which gives a gauge slice, is then given by
\begin{align*}
	H_{D}^{1, 1} = \text{ker}(D_{1, 1}) / \text{im}(D_{0, 1}).
\end{align*}

We are interested in the integral of field strengths corresponding to this construction. To that end, decompose some $p$-cycle as $C^{p} = \sum\limits_{i_{0}}l_{i_{0}}^{p}$, where each $l_{i}^{p}\subset U_{i}$. There is a boundary operator on each of these given by
\begin{align*}
	\bound{l_{i_{0}}^{p}} = \sum\limits_{i_{1}}l_{i_{1}i_{0}}^{p - 1} - l_{i_{0}i_{1}}^{p - 1}.
\end{align*}
To this belongs the idea that the decomposition is ordered - that is, $l_{ij} = 0$ if $i > j$. The generalization of this is
\begin{align*}
	\bound{l_{i_{0}\dots i_{k}}^{p - k}} = \sum\limits_{i_{k + 1}}\left(l_{i_{k + 1}i_{0}\dots i_{k}}^{p - k - 1} + (-1)^{k + 1}l_{i_{0}\dots i_{k}i_{k + 1}}^{p - k - 1} + \sum\limits_{j = 1}^{k}(-1)^{j}l_{i_{0}\dots i_{j - 1}i_{k + 1}i_{j}\dots i_{k}}^{p - k - 1}\right).
\end{align*}
The minus sign describes the orientation. We then have
\begin{align*}
	\inte{C^{2}}{}\df A =& \sum\limits_{i_{0}}\inte{l_{i_{0}}^{2}}{}\df A_{i_{0}} = \sum\limits_{i_{0}}\inte{\bound{l_{i_{0}}^{2}}}{}A_{i_{0}} = \sum\limits_{i_{0}, i_{1}}\inte{l_{i_{1}i_{0}}^{1}}{}A_{i_{0}} - \inte{l_{i_{0}i_{1}}^{1}}{}A_{i_{0}}.
\end{align*}
To see how this works out, consider a specific pair $i_{0}, i_{1}$. They will each contribute such an integral exactly once and with opposite sign, and thus
\begin{align*}
	\inte{C^{2}}{}\df A =& \sum\limits_{i_{0}, i_{1}}\inte{l_{i_{0}i_{1}}^{1}}{}A_{i_{1}} - A_{i_{0}} = \sum\limits_{i_{0}, i_{1}}\inte{l_{i_{0}i_{1}}^{1}}{}\df\Lambda_{i_{0}i_{1}} = \sum\limits_{i_{0}, i_{1}}\inte{\bound{l_{i_{0}i_{1}}^{1}}}{}\Lambda_{i_{0}i_{1}} = \sum\limits_{i_{0}, i_{1}, i_{2}}\left(\inte{l_{i_{2}i_{0}i_{1}}^{0}}{} - \inte{l_{i_{0}i_{2}i_{1}}^{0}}{} + \inte{l_{i_{0}i_{1}i_{2}}^{0}}{}\right)\Lambda_{i_{0}i_{1}}.
\end{align*}
Handling this is slightly more involved. While all indices are freely summed over, the only contributions are found when one of the indices in the integration domains are an ordered sequence. We see that this corresponds to the exact same process of index removal with alternating signs as found in the definition of $\delta$, and thus
\begin{align*}
	\inte{C^{2}}{}\df A =& \sum\limits_{i_{0}, i_{1}, i_{2}}\inte{l_{i_{0}i_{1}i_{2}}^{0}}{}\df n_{i_{0}i_{1}i_{2}} = \sum\limits_{i_{0}, i_{1}, i_{2}}n_{i_{0}i_{1}i_{2}},
\end{align*}
and so the total field strength is quantized.

To introduce so-called higher order gauge fields, one generalizes the above procedure. While this can be done to obtain $H_{D}^{p, q}$ for any $p$ and $q$, we will only need to consider the case of $p = q$. We then introduce DB cocycles as tuples
\begin{align*}
	\omega = \left(\omega_{\alpha_{0}}^{(0, p)}, \dots, \omega_{\alpha_{0}\dots\alpha_{p}}^{(p, 0)}, n_{\alpha_{0}\dots\alpha_{p + 1}}^{p + 1, -1}\right).
\end{align*}
Each $\omega_{\alpha_{0}\dots\alpha_{k}}^{(k, p - k)}$ is a $p - k$-form defined on overlaps of degree $k + 1$ related to the next element by
\begin{align*}
	(\delta_{k}\omega^{(k, p - k)})_{\alpha_{0}\dots\alpha_{k}} = \df[p - k - 1]\omega_{\alpha_{0}\dots\alpha_{k}}^{(k + 1, p - k - 1)},
\end{align*}
and are thus annihilated by
\begin{align*}
	D_{p, p} = (\delta_{0} + 0) + \sum\limits_{k = 1}^{p + 1}(-1)^{k}(\delta_{k} + \df[p - k]).
\end{align*}
The coboundaries form the image of
\begin{align*}
	D_{p - 1, p} = \sum\limits_{k = 0}^{p}(-1)^{k}(\delta_{k} + \df[p - k - 1]).
\end{align*}
The quotient of the kernel of the first by the image of the second yields the cohomology group $H_{D}^{p}$.

A particular case to consider is 2-form gauge fields. The gauge transformation manifests as a shift by something in the image of
\begin{align*}
	D_{1, 2} = \delta_{0} + \df[1] - (\delta_{1} + \df[0]) + \delta_{2} + \df[-1].
\end{align*}
We have
\begin{align*}
	D_{1, 2}\mqty[
		\xi_{\alpha} \\
		\phi_{\alpha\beta} \\
		m_{\alpha\beta\gamma}
	] = \mqty[
		\df[1]\xi_{\alpha} \\
		(\delta_{0}\xi)_{\alpha\beta} - \df[0]\phi_{\alpha\beta} \\
		-(\delta_{1}\phi)_{\alpha\beta\gamma} + \df[-1]m_{\alpha\beta\gamma} \\
		(\delta_{2}m)_{\alpha\beta\gamma\delta}
	],
\end{align*}
and so the generalized gauge transformation for a 2-form gauge field $(\omega, A, \Lambda, n)$ is
\begin{align*}
	\omega_{\alpha} \to& \omega_{\alpha}^{(2)} + \df[1]\xi_{\alpha}, \\
	A_{\alpha\beta} \to& A_{\alpha\beta} - \xi_{\beta} + \xi_{\alpha} - \df[0]\phi_{\alpha\beta}, \\
	\Lambda_{\alpha\beta\gamma} \to& \Lambda_{\alpha\beta\gamma} - \phi_{\beta\gamma} + \phi_{\alpha\gamma} - \phi_{\alpha\beta} + \df[-1]m_{\alpha\beta\gamma}, \\
	n_{\alpha\beta\gamma\delta} \to& n_{\alpha\beta\gamma\delta} + m_{\beta\gamma\delta} - m_{\alpha\gamma\delta} + m_{\alpha\beta\delta} - m_{\alpha\beta\gamma}.
\end{align*}

Let us also consider the corresponding integral of the field strength. We have
\begin{align*}
	\inte{C^{3}}{}\df\omega =& \sum\limits_{i_{0}}\inte{l_{i_{0}}^{2}}{}\df\omega_{i_{0}} = \sum\limits_{i_{0}}\inte{\bound{l_{i_{0}}^{2}}}{}A_{i_{0}} = \sum\limits_{i_{0}, i_{1}}\inte{l_{i_{1}i_{0}}^{1}}{}A_{i_{0}} - \inte{l_{i_{0}i_{1}}^{1}}{}A_{i_{0}}.
\end{align*}
The proof above generalizes without changes, and so
\begin{align*}
	\inte{C^{3}}{}\df\omega =& \sum\limits_{i_{0}, i_{1}, i_{2}}\inte{l_{i_{0}i_{1}i_{2}}^{1}}{}\df\Lambda_{i_{0}i_{1}i_{2}} = \sum\limits_{i_{0}, i_{1}, i_{2}, i_{3}}\left(\inte{l_{i_{3}i_{0}i_{1}i_{2}}^{0}}{} - \inte{l_{i_{0}i_{3}i_{1}i_{2}}^{0}}{} + \inte{l_{i_{0}i_{1}i_{3}i_{2}}^{0}}{} - \inte{l_{i_{0}i_{1}i_{2}i_{3}}^{0}}{}\right)\Lambda_{i_{0}i_{1}i_{2}}.
\end{align*}
The argument previously used is identical, and so
\begin{align*}
	\inte{C^{3}}{}\df\omega = \sum\limits_{i_{0}, i_{1}, i_{2}, i_{3}}\inte{l_{i_{0}i_{1}i_{2}i_{3}}^{0}}{}(\delta_{2}\Lambda)_{i_{0}i_{1}i_{2}i_{3}} = \sum\limits_{i_{0}, i_{1}, i_{2}, i_{3}}\inte{l_{i_{0}i_{1}i_{2}i_{3}}^{0}}{}\df[-1]n_{i_{0}i_{1}i_{2}i_{3}} = \sum\limits_{i_{0}, i_{1}, i_{2}, i_{3}}n_{i_{0}i_{1}i_{2}i_{3}},
\end{align*}
which is valued in integers.