\section{Kurvor}

\subsection{Definitioner}

\paragraph{Kurvor i $\R^p$}
En kurva i $\R^p$ är en funktion $t\to\vect{x}(t) = (x_1(t), \dots, x_p(t))$.

\paragraph{$C^1$-kurvor}
En kurva är klass $C^1$ om alla dess komponenter är $C^1$.

\paragraph{Enkla kurvor}
En kurva är enkel om den inte skär sig själv.

\paragraph{Slutna kurvor}
En kurva är sluten om dens start- och slutpunkter sammanfaller.

\paragraph{Tangentvektor}
Låt $\vect{x}(t)$ vara en $C^1$-kurva definierad på $[\alpha, \beta]$, $\phi: [a, b]\to [\alpha, \beta]$ vara strängt växande och $\phi, \phi^{-1}$ vara $C^1$. Då definieras tangentvektorn till kurvan av
\begin{align*}
	\deval{\vect{x}}{t}{t} = \limit{h}{0}\frac{\vect{x}(t + h) - \vect{x}(t)}{h}.
\end{align*}

\paragraph{Längd}
Långden av en kurva ges av
\begin{align*}
	\int\limits_{\alpha}^{\beta}\abs{\deval{\vect{x}}{t}{t}}\dd{t}.
\end{align*}

\paragraph{Integral av skalärfunktioner längs med kurvor}
Integralen av en funktion $f$ längs med kurvan med parametrisering $\vect{r}(t)$ ges av
\begin{align*}
	\int\limits_{\alpha}^{\beta}f(\vect{r}(t))\abs{\deval{\vect{r}}{t}{t}}\dd{t}.
\end{align*}

\paragraph{Kurvintegraler}
Låt $F: D\to\R^2, \vect{F}(\vect{r}) = (P(\vect{r}), Q(\vect{r}))$ vara kontinuerlig på en öppen mängd $D\subset\R^2$ och låt $\gamma$ vara en $C^1$-kurva med parametrisering $\vect{r} = \vect{r}(t)$ för $\alpha\leq t\leq\beta$. Kurvintegralen av $\vect{F}$ längs $\gamma$ ges av
\begin{align*}
	\int_{\alpha}^{\beta}\vect{F}(\vect{r}(t))\cdot\deval{\vect{r}}{t}{t}\dd{t} = \int_{\alpha}^{\beta}P(\vect{r}(t))\deval{r_x}{t}{t} + Q(\vect{r}(t))\dd{t}\deval{r_y}{t}{t}
\end{align*}
och betecknas
\begin{align*}
	\cint{\gamma}{\vect{F}}{\vect{r}}
\end{align*}
eller, om du är en omoralsk människa,
\begin{align*}
	\int\limits_{\gamma}P(\vect{r})\dd{x} + Q(\vect{r})\dd{y}.
\end{align*}

\paragraph{Cirkulation}
Låt $\gamma$ vara enkel och sluten. Då definieras $\cint[c]{\gamma}{\vect{F}}{\vect{r}}$ som cirkulationen av $\vect{F}$ längs $\gamma$.

\subsection{Satser}