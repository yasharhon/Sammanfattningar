\section{Samband mellan integraler}

\paragraph{Greens sats}
Låt
\begin{itemize}
	\item $\Omega\subset\R^2$ vara öppen.
	\item $P, Q: \Omega\to\R$ är $C^1$.
	\item $D\subset\Omega$ är kompakt.
	\item $\bound{D}$ vara styckvis $C^1$.
\end{itemize}

Då gäller att
\begin{align*}
	\oint\limits_{\bound{D}}P\dd{x} + Q\dd{y} = \inteval{D}{\pdv{Q}{x} - \pdv{P}{y}}{{x, y}}.
\end{align*}
Märk att det första integralet kan skrivas som ett kurvintegral av ett vektorfält.

\proof
Beviset ges enbart för en rektangel. Det allmäna beviset involverar att betrakta flera små rektanglar.

\begin{align*}
	\inteval{D}{\pdv{Q}{x} - \pdv{P}{y}}{{x, y}} &= \int\limits_{c}^{d}\int\limits_{a}^{b}\pdv{Q}{x}(x, y)\dd{x}\dd{y} - \int\limits_{a}^{b}\int\limits_{c}^{d}\pdv{P}{y}(x, y)\dd{y}\dd{x} \\
	                                           &= \int\limits_{c}^{d}Q(b, y) - Q(a, y)\dd{y} - \int\limits_{a}^{b}P(x, d) - P(x, c)\dd{x}.
\end{align*}

Dela randen till rektangeln upp i $\gamma_1, \gamma_2, \gamma_3, \gamma_4$, där dessa beskrivas av $x = b, y = d, x = a, y = c$ respektiva. Då kan det sista integralet skrivas som
\begin{align*}
	  \int\limits_{\gamma_1}Q(x, y)\dd{y} + \int\limits_{\gamma_2}P(x, y)\dd{x} + \int\limits_{\gamma_3}Q(x, y)\dd{y} + \int\limits_{\gamma_4}P(x, y)\dd{x}.
\end{align*}	  
Varje integral över randen involverar enbart en variabel och en funktion. Därmed kan vi lägga till integralet över den andra variabeln av den andra funktionen och få
\begin{align*}
	\left(\int\limits_{\gamma_1} + \int\limits_{\gamma_2} + \int\limits_{\gamma_3} + \int\limits_{\gamma_4}\right)P(x, y)\dd{x} + Q(x, y)\dd{y} = \int\limits_{\bound{D}}P(x, y)\dd{x} + Q(x, y)\dd{y},
\end{align*}
och beviset är klart.

\paragraph{Divergenssatsen}
Låt $\vect{u}$ vara ett $C^1$-vektorfält definierat i en öppen mängd $\Omega\subset\R^3$, $K\subset\Omega$ vara kompakt och $\bound{K}$ bestå av en eller flera $C^1$-ytor med utadriktad normal. Då gäller att
\begin{align*}
	\fint{\bound{K}}{\vect{u}}{\vect{S}} = \inteval{K}{\div{\vect{u}}}{{x, y, z}}.
\end{align*}

\proof
Beviset ges enbart för ett rätblock. Det allmäna beviset involverar att betrakta flera rätblock, tror jag.

Vi delar integralerna upp som
\begin{align*}
	&\fint{\bound{K}}{\vect{u}}{\vect{S}} = \sum\limits_{i}\fint{\bound{K}}{u_i\vect{e}_i}{\vect{S}}, \\
	&\inteval{K}{\div{\vect{u}}}{{x, y, z}} = \sum\limits_{i}\inteval{K}{\div{u_i\vect{e}_i}}{{x, y, z}}
\end{align*}
och ser att det räcker att visa att
\begin{align*}
	\fint{\bound{K}}{u_i\vect{e}_i}{\vect{S}} = \inteval{K}{\div{u_i\vect{e}_i}}{{x, y, z}}, i = 1, 2, 3.
\end{align*}
I fallet $i = 1$ får man
\begin{align*}
	\fint{\bound{K}}{u_i\vect{e}_i}{\vect{S}} &= \int\limits_{e}^{f}\int\limits_{c}^{d}u_1(b, y, z)\dd{y}\dd{z} - \int\limits_{e}^{f}\int\limits_{c}^{d}u_1(a, y, z)\dd{y}\dd{z} \\
	                                          &= \int\limits_{e}^{f}\int\limits_{c}^{d}\int\limits_{a}^{b}\pdeval{u_1}{x}{x, y, z}\dd{x}\dd{y}\dd{z}\\
	                                          &= \int\limits_{e}^{f}\int\limits_{c}^{d}\int\limits_{a}^{b}\div{u_i\vect{e}_i}\dd{x}\dd{y}\dd{z}.
\end{align*}

Med ett motsvarande bevis för $i = 2, 3$ är beviset klart.