\section{Funktioner}

\subsection{Definitioner}

\paragraph{Grafen av en funktion}
Låt $f: D\to\R$ med $D\subset\R^2$. Grafen av $f$ är
\begin{align*}
	\{(x, y, z)\in\R^3: z = f(x, y)\}.
\end{align*}

\paragraph{Lokala gränsvärden}
Låt $f: D\to\R^p$ med $D\subset\R^n$ och $\vb{a}$ vara en inre punkt eller randpunkt till $D$. $\limit{\vb{x}}{\vb{a}}f(\vb{x}) = \vb{b}$ om det för varje $\varepsilon > 0$ finns ett $\delta > 0$ så att
\begin{align*}
	\abs{\vb{x} - \vb{a}} < \delta, \vb{x}\in D \implies\abs{f(\vb{x}) - \vb{b}} < \varepsilon.
\end{align*}

\paragraph{Gränsvärden mot oändligheten}
Låt $f: D\to\R^p$ med $D\subset\R^n$. $\limit{\abs{\vb{x}}}{\infty}f(\vb{x}) = \vb{b}$ om det för varje $\varepsilon > 0$ finns ett $\omega > 0$ så att
\begin{align*}
	\abs{\vb{x}} > \omega, \vb{x}\in D \implies\abs{f(\vb{x}) - \vb{b}} < \varepsilon.
\end{align*}

\paragraph{Kontinuitet}
Låt $f: D\to\R^p$ med $D\subset\R^n$. $f$ är kontinuerlig i $\vb{a}\in D$ om $\limit{\vb{x}}{\vb{a}}f(\vb{x})$ existerar och $\limit{\vb{x}}{\vb{a}} = f(\vb{a})$.

\paragraph{Likformig kontinuitet}
Låt $f: D\to\R^p$ med $D\subset\R^n$. $f$ är likformigt kontinuerlig på $D$ om det för varje $\varepsilon > 0$ finns ett $\delta > 0$ så att
\begin{align*}
	\abs{\vb{x} - \vb{y}} < \delta,\ \vb{x}, \vb{y}\in D \implies\abs{f(\vb{x}) - f(\vb{y})} < \varepsilon.
\end{align*}

\paragraph{Lokala extrempunkter}
Låt $f: D\to\R$ med $D\subset\R^n$. $f$ har ett lokalt maximum i $\vb{a}$ om $\exists\delta > 0$ så att $f(\vb{x})\leq f(\vb{a})$ för alla $\vb{x}\in D$ så att $\abs{\vb{x} - \vb{a}} < \delta$. Lokala minima definieras analogt. Om $f(\vb{x}) < f(\vb{a})$ har $f$ ett strängt lokalt maximum i $\vb{a}$.

\paragraph{Kvadratiska former}
Låt $A, B, C$ vara konstanter. En kvadratisk form från $\R^2$ är på formen
\begin{align*}
	Q(h, k) = Ah^2 + 2Bhk + Ck^2.
\end{align*}
För en mer allmän definition, se definitionen från sammanfattningen av SF1672 Linjär algebra.

\paragraph{Positivt och negativt definita kvadratiska former}
En kvadratisk form är
\begin{itemize}
	\item positivt definit om $Q(h, k) > 0$ för $(h, k)\neq (0,0)$.
	\item positivt semidefinit om $Q(h, k)\geq 0$ för $(h, k)\neq (0,0)$.
	\item negativt definit om $Q(h, k) < 0$ för $(h, k)\neq (0,0)$.
	\item negativt semidefinit om $Q(h, k)\leq 0$ för $(h, k)\neq (0,0)$.
	\item indefinit om $Q$ antar såväl positiva som negativa värden.
\end{itemize}

\paragraph{Trappfunktioner}
En funktion $\Phi$ definierat på en axelparallell rektangel $\Delta$ är en trappfunktion om det finns en indelning av $\Delta$ i mindre rektanglar
\begin{align*}
	\Delta_{i, j} = \{(x, y)\ |\ x_{i - 1}\leq x\leq x_i, y_{i - 1}\leq y\leq y_i\}
\end{align*}
så att $\Phi$ är konstant på varje $\Delta_i$.

\paragraph{Avskärningar}
Låt $f$ vara en kontinuerlig funktion i ett öppet område $\Omega\subset\R^2$. En begränsdad kvadrerbar delmängd $D$ av $\Omega$ är en avskärning om $f$ är begränsad på $D$.

\subsection{Satser}

\paragraph{Gränsvärden av funktioner och deras komponenter}
Låt $f: D\to\R^p$ med $D\subset\R^n$. $\limit{\vb{x}}{\vb{a}}f(\vb{x}) = \vb{b}$ är ekvivalent med att $\limit{\vb{x}}{\vb{a}}f_i(\vb{x}) = b_i$, där subskriptet $i$ indikerar den $i$-te komponenten av varje vektor.

\proof
Detta följer direkt av att
\begin{align*}
	\abs{f_i(\vb{x}) - b_i}\leq\abs{f(\vb{x}) - \vb{b}}\leq\sum\limits_{i = 1}^{p}\abs{f_i(\vb{x}) - b_i}.
\end{align*}

\paragraph{Största och minsta värde för funktioner}
Låt $f: D\to\R^p$ med $D\subset\R^n$ och låt $D$ vara kompakt. Då antar $f$ ett största och ett minsta värde på $D$.

\proof

\paragraph{Definitionsmängd och likformig kontinuitet}
Låt $f: D\to\R^p$ med $D\subset\R^n$ och låt $D$ vara kompakt. Då är $f$ likformigt kontinuerlig på $D$.

\proof

\paragraph{Satsen om mellanliggande värden}
Låt $f: D\to\R^p$ med $D\subset\R^n$ och låt $D$ vara bågvis sammanhängande. Om $f$ antar värderna $f(\vb{a}), f(\vb{b})$ i $D$, antar $f$ också alla värden mellan $f(\vb{a})$ och $f(\vb{b})$.

\proof

\paragraph{Inversa funktionssatsen}
Låt $f: D\to\R^p$, $D\subset\R^n$ vara öppen, $f$ vara $C^1$ och $\abs{\dd{f} (\vb{a})}\neq 0$. Då finns det öppna omgivningar $U, V$ till $(\vb{a}, f(\vb{a}))$ så att $f:U\to V$ är bijektiv och $f^{-1}: V\to U$ är $C^1$.

\proof

\paragraph{Implicita funktionssatsen}
Låt $F(\vb{x})$ vara $C^1$ och $\vb{a}$ vara på nivåkurvan $F(\vb{x}) = C$. Om $\pdeval{F}{x_n}{\vb{a}}\neq 0$ finns det en öppen omgivning $U$ av $\vb{a}$ så att restriktion av nivåkurvan till $U$ implicit definierar en $C^1$-funktion.

\proof

\paragraph{Derivatan av en implicit funktion}
Låt $F(\vb{x})$ vara $C^1$, $\vb{a}$ vara på nivåkurvan $F(\vb{x}) = C$ och $F(\vb{x}) = C$ definiera en implicit funktion nära $\vb{a}$. Om $\pdeval{F}{x_n}{\vb{a}}\neq 0$ har man
\begin{align*}
	\pdeval{x_n}{x_i}{\vb{a}'} = -\frac{\pdeval{F}{x_i}{\vb{a}}}{\pdeval{F}{x_n}{\vb{a}}}.
\end{align*}

\proof
Eftersom $F$ är konstant nära $\vb{a}$ använder vi kedjeregeln, vilket ger
\begin{align*}
	\pdeval{F}{x_i}{\vb{a}} + \pdeval{F}{x_n}{\vb{a}}\pdeval{x_n}{x_i}{\vb{a}'} = 0.
\end{align*}
Om $\pdeval{F}{x_n}{\vb{a}}\neq 0$ får man resultatet i satsen.