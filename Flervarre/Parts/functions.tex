\section{Funktioner}

\subsection{Definitioner}

\paragraph{Grafen av en funktion}
Låt $f: D\to\R$ med $D\subset\R^2$. Grafen av $f$ är
\begin{align*}
	\{(x, y, z)\in\R^3: z = f(x, y)\}.
\end{align*}

\paragraph{Kurvor i $\R^p$}
En kurva i $\R^p$ är en funktion $t\to\vect{x}(t) = (x_1(t), \dots, x_p(t))$.

\paragraph{Lokala gränsvärden}
Låt $f: D\to\R^p$ med $D\subset\R^n$ och $\vect{a}$ vara en inre punkt eller randpunkt till $D$. $\limit{\vect{x}}{\vect{a}}f(\vect{x}) = \vect{b}$om det för varje $\varepsilon > 0$ finns ett $\delta > 0$ så att
\begin{align*}
	\abs{\vect{x} - \vect{a}} < \delta, \vect{x}\in D \implies\abs{f(\vect{x}) - \vect{b}} < \varepsilon.
\end{align*}

\paragraph{Gränsvärden mot oändligheten}
Låt $f: D\to\R^p$ med $D\subset\R^n$. $\limit{\abs{\vect{x}}}{\infty}f(\vect{x}) = \vect{b}$ om det för varje $\varepsilon > 0$ finns ett $\omega > 0$ så att
\begin{align*}
	\abs{\vect{x}} > \omega, \vect{x}\in D \implies\abs{f(\vect{x}) - \vect{b}} < \varepsilon.
\end{align*}

\paragraph{Kontinuitet}
Låt $f: D\to\R^p$ med $D\subset\R^n$. $f$ är kontinuerlig i $\vect{a}\in D$ om $\limit{\vect{x}}{\vect{a}}$ existerar och $\limit{\vect{x}}{\vect{a}} = f(\vect{a}$.

\paragraph{Likformig kontinuitet}
Låt $f: D\to\R^p$ med $D\subset\R^n$. $f$ är likformigt kontinuerlig på $D$ om det för varje $\varepsilon > 0$ finns ett $\delta > 0$ så att
\begin{align*}
	\abs{\vect{x} - \vect{y}} < \delta, \vect{x}, \vect{y}\in D \implies\abs{f(\vect{x}) - f(\vect{y})} < \varepsilon.
\end{align*}

\paragraph{Lokala extrempunkter}
Låt $f: D\to\R$ med $D\subset\R^n$. $f$ har ett lokalt maximum i $\vect{a}$ om $\exists\delta > 0$ så att $f(\vect{x})\leq f(\vect{a})$ för alla $\vect{x}\in D$ så att $\abs{\vect{x} - \vect{a}} < \delta$. Lokala minima definieras analogt. Om $f(\vect{x}) < f(\vect{a})$ har $f$ ett strängt lokalt maximum i $\vect{a}$.

\paragraph{Kvadratiska former}
Låt $A, B, C$ vara konstanter. En kvadratisk form från $\R^2$ är på formen
\begin{align*}
	Q(h, k) = Ah^2 + 2Bhk + Ck^2.
\end{align*}
För en mer allmän definition, se definitionen från sammanfattningen av SF1672.

\paragraph{Positivt och negativt definita kvadratiska former}
En kvadratisk form är
\begin{itemize}
	\item positivt definit om $Q(h, k) > 0$ för $(h, k)\neq (0,0)$.
	\item negativt definit om $Q(h, k) < 0$ för $(h, k)\neq (0,0)$.
	\item indefinit om $Q$ antar såväl positiva som negativa värden.
\end{itemize}

\subsection{Satser}

\paragraph{Gränsvärden av funktioner och deras komponenter}
Låt $f: D\to\R^p$ med $D\subset\R^n$. $\limit{\vect{x}}{\vect{a}}f(\vect{x}) = \vect{b}$ är ekvivalent med att $\limit{\vect{x}}{\vect{a}}f_i(\vect{x}) = \vect{b}_i$, där subskriptet $i$ indikerar den $i$-te komponenten av varje vektor.

\proof
Detta följer direkt av att
\begin{align*}
	\abs{f_i(\vect{x}) - \vect{b}_i}\leq\abs{f(\vect{x}) - \vect{b}}\leq\sum\limits_{i = 1}^{p}\abs{f_i(\vect{x}) - \vect{b}_i}.
\end{align*}

\paragraph{Största och minsta värde för funktioner}
Låt $f: D\to\R^p$ med $D\subset\R^n$ och låt $D$ vara kompakt. Då antar $f$ ett största och ett minsta värde på $D$.

\proof

\paragraph{Definitionsmängd och likformig kontinuitet}
Låt $f: D\to\R^p$ med $D\subset\R^n$ och låt $D$ vara kompakt. Då är $f$ likformigt kontinuerlig på $D$.

\proof

\paragraph{Satsen om mellanliggande värden}
Låt $f: D\to\R^p$ med $D\subset\R^n$ och låt $D$ vara bågvis sammanhängande. Om $f$ antar värderna $f(\vect{a}), f(\vect{b})$ i $D$, antar $f$ också alla värden mellan $f(\vect{a})$ och $f(\vect{b})$.

\proof