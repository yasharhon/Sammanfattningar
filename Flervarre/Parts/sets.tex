\section{Mängdlära}

\subsection{Definitioner}

\paragraph{Öppna klot}
Ett öppet klot i $\R^n$ centrerad i $\vect{a}$ med radius $r$ är
\begin{align*}
	\{\vect{x}\in\R^n:\ \abs{\vect{x} - \vect{a}}<r\}.
\end{align*}

\paragraph{Omgivningar till punkter}
$U\subset\R^n$ är en omgivning till $\vect{a}\in\R^n$ om $U$ innehåller något öppet klot med centrum $\vect{a}$.

\paragraph{Inre punkter}
Låt $M\subset\R^n$. $\vect{a}$ är en inre punkt till $M$ om det finns ett öppet klot kring $\vect{a}$ i $M$.

\paragraph{Yttre punkter}
Låt $M\subset\R^n$. $\vect{a}$ är en yttre punkt till $M$ om det finns ett öppet klot kring $\vect{a}$ i $M$:s komplement, definierad som $\R^n\setminus M$.

\paragraph{Randpunkter}
Låt $M\subset\R^n$. $\vect{a}$ är en randpunkt till $M$ om varje öppet klot kring $\vect{a}$ innehåller punkter i $M$ och $M$:s komplement.

\paragraph{Rand}
Mängden av alla randpunkter till en mängd $M$ är randen till $M$. Denna betecknas $\bound{M}$.

\paragraph{Öppna och slutna mängder}
En mängd är öppen om $\bound{M}$ är i $M$:s komplement och sluten om $\bound{M}$ är i $M$.

\paragraph{Begränsade mängder}
En mängd $M$ är begränsad om $\exists c > 0$ så att $\abs{\vect{x}} < c\forall \vect{x}\in M$.

\paragraph{Kompakta mängder}
En mängd är kompakt om den är sluten och begränsad.

\subsection{Satser}