\section{Ytor}

\subsection{Definitioner}

\paragraph{Ytor}
En yta är en funktion $\vect{r}: D\to\R^3$ med $D\subset\R^2$.

\paragraph{Tangentplan}
Tangentplanet till en kurva spänns upp av vektorerna
\begin{align*}
	\vect{r}_s(s, t) = \left(\deval{r_1}{s}{s, t}, \deval{r_2}{s}{s, t}, \deval{r_3}{s}{s, t}\right), \\
	\vect{r}_t(s, t) = \left(\deval{r_1}{t}{s, t}, \deval{r_2}{t}{s, t}, \deval{r_3}{t}{s, t}\right), \\.
\end{align*}

\paragraph{Positiv sida av yta}
Den positiva sidan av en yta är den sidan nromalvektorn $\pdv{\vect{r}}{s}\times\pdv{\vect{r}}{t}$ pekar mot.

\paragraph{Enhetsnormal}
\begin{align*}
	\vect{N} = \frac{\pdv{\vect{r}}{s}\times\pdv{\vect{r}}{t}}{\abs{\pdv{\vect{r}}{s}\times\pdv{\vect{r}}{t}}} 
\end{align*}
är en ytas enhetsnormal.

\paragraph{Areaelement}
\begin{align*}
	\dd{S} = \abs{\pdv{\vect{r}}{s}\times\pdv{\vect{r}}{t}}\dd{s}\dd{t}
\end{align*}
ör en ytas arealement.

\paragraph{Vektoriellt arealement}
\begin{align*}
	\dd{\vect{S}} = \vect{N}\dd{S} = \pdv{\vect{r}}{s}\times\pdv{\vect{r}}{t}\dd{s}\dd{t}
\end{align*}
är ytans vektoriella areaelement.

\paragraph{Rand}
Randen av ytan $Y$ med parametrisering $\vect{R}(s, t), (s, t)\in D$ är bilden av $\bound{D}$ under $\vect{r}$ med tillhörande orientering.

\paragraph{Area av yta}
Låt $\vect{r}: D\to\R^3$ vara en parametrisering av en yta $Y$. Då ges arean av
\begin{align*}
	A_Y = \inteval{D}{\abs{\pdv{\vect{r}}{s}\times\pdv{\vect{r}}{t}}}{{s, t}}. 
\end{align*}
Detta skrivs även
\begin{align*}
	A_Y = \iint_{Y}\dd{S}.
\end{align*}

\paragraph{Integration av skalärfunktioner över ytor}
Låt $\vect{r}: D\to\R^3$ vara en parametrisering av en yta $Y$. Då ges integralet av funktionen $f$ över $Y$ av
\begin{align*}
	\inteval{D}{f(\vect{r}(s, t))\abs{\pdv{\vect{r}}{s}\times\pdv{\vect{r}}{t}}}{{s, t}}
\end{align*}
och betecknas
\begin{align*}
	\iint\limits_{Y}f(\vect{r})\dd{S}.
\end{align*}

\paragraph{Flöde genom yta}
\begin{align*}
	\fint{Y}{\vect{u}}{\vect{S}}
\end{align*}
ör flödet av $\vect{u}$ genom $Y$.

\subsection{Satser}