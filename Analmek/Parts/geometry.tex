\section{Geometry}

\paragraph{Covariant derivatives and Christoffel symbols}
When computing a derivative, one must account both for the change in the quantity itself and the change of basis. We have
\begin{align*}
	\del{b}{\tb{a}} = \chris{c}{b}{a}\tb{c}
\end{align*}
where the $\chris{c}{b}{a}$ are called Christoffel symbols. These satisfy
\begin{align*}
	\db{c}\cdot\del{b}{\tb{a}} = \db{c}\cdot\chris{d}{b}{a}\tb{d} = \kdelta{d}{c}\chris{d}{b}{a} = \chris{c}{b}{a}.
\end{align*}
Note that
\begin{align*}
	\del{a}{\tb{b}} = \del{a}{\del{b}{\vb{x}}} = \del{b}{\del{a}{\vb{x}}} = \del{b}{\tb{a}},
\end{align*}
which implies
\begin{align*}
	\chris{c}{b}{a} = \chris{c}{a}{b}.
\end{align*}

Using this, we can compute the partial derivate of $\vb{v} = v^{a}\tb{a}$ with respect to $\chi^{a}$ as
\begin{align*}
	\del{a}{\vb{v}} = \tb{b}\del{a}{v^{b}} + v^{b}\del{a}{\tb{b}} = \tb{b}\del{a}{v^{b}} + v^{b}\chris{c}{a}{b}\tb{c}.
\end{align*}
Renaming the summation indices yields
\begin{align*}
	\del{a}{\vb{v}} = \tb{b}(\del{a}{v^{b}} + v^{c}\chris{b}{a}{c}),
\end{align*}
which contains one term from the change in the coordinates and one term from the change in basis. We now define the covariant derivative of the contravariant components of $\vb{v}$ as
\begin{align*}
	\dcov{a}{v^{b}} = \del{a}{v^{b}} + v^{c}\chris{b}{a}{c}.
\end{align*}

We would also like to define the covariant derivative of the covariant components of a vector field. To do this, we use the fact that
\begin{align*}
	\del{a}{\tb{b}\cdot\db{c}} = \del{a}{\kdelta{b}{c}} = 0.
\end{align*}
The product rule yields
\begin{align*}
	\tb{b}\cdot\del{a}{\db{c}} + \db{c}\cdot\del{a}{\tb{b}} = \tb{b}\cdot\del{a}{\db{c}} + \db{c}\cdot\chris{d}{a}{b}\tb{d} = \tb{b}\cdot\del{a}{\db{c}} + \kdelta{d}{c}\cdot\chris{d}{a}{b} = \tb{b}\cdot\del{a}{\db{c}} + \chris{c}{a}{b},
\end{align*}
which implies
\begin{align*}
	\del{a}{\db{c}} = -\chris{c}{a}{b}\db{b}.
\end{align*}
Repeating the steps above now yields
\begin{align*}
	\dcov{a}{v_{b}} = \del{a}{v_{b}} - \chris{c}{a}{b}v_{c}.
\end{align*}

\paragraph{Curve length}
Consider some curve parametrized by $t$, and let $\dot{\vb{\gamma}}$ denote its tangent. The curve length is given by
\begin{align*}
	\dd{s}^{2} = \dd{\vb{x}}\cdot\dd{\vb{x}} = g(\dot{\vb{\gamma}}, \dot{\vb{\gamma}})\dd{t}^{2} = g_{ab}\dot{\chi^{a}}\dot{\chi^{b}}\dd{t}^{2}.
\end{align*}
The curve length is now given by
\begin{align*}
	L = \integ{}{}{t}{\sqrt{g_{ab}\dot{\chi^{a}}\dot{\chi^{b}}}}.
\end{align*}

\paragraph{Geodesics}
A geodesic is a curve that extremises the curve length between two points. From variational calculus, it is known that such curves satisfy the Euler-Lagrange equations, and we would like a differential equation that describes such a curve. By defining $\mathcal{L} = \sqrt{g_{ab}\dot{\chi}^{a}\dot{\chi}^{b}}$, the Euler-Lagrange equations for the curve length becomes
\begin{align*}
	\del{\chi^{a}}{\mathcal{L}} - \dv{t}\del{\dot{\chi}^{a}}{\mathcal{L}} = 0.
\end{align*}
The Euler-Lagrange equation thus becomes
\begin{align*}
	\frac{1}{2\mathcal{L}}\dot{\chi}^{b}\dot{\chi}^{c}\del{a}{g_{bc}} - \dv{t}\left(\frac{1}{2\mathcal{L}}g_{bc}(\dot{\chi}^{b}\kdelta{a}{c} + \dot{\chi}^{c}\kdelta{a}{b})\right) = 0.
\end{align*}
A minor simplification yields
\begin{align*}
	\frac{1}{2\mathcal{L}}\dot{\chi}^{b}\dot{\chi}^{c}\del{a}{g_{bc}} - \dv{t}\left(\frac{1}{\mathcal{L}}g_{ac}\dot{\chi}^{c}\right) = 0.
\end{align*}
Expanding the time derivative yields
\begin{align*}
	\frac{1}{2\mathcal{L}}\dot{\chi}^{b}\dot{\chi}^{c}\del{a}{g_{bc}} - \frac{1}{\mathcal{L}}\dv{t}(g_{ac}\dot{\chi}^{c}) + g_{ac}\dot{\chi}^{c}\frac{1}{\mathcal{L}^{2}}\dv{\lag}{t} = 0.
\end{align*}
This may be written as
\begin{align*}
	\frac{1}{2\mathcal{L}}\dot{\chi}^{b}\dot{\chi}^{c}\del{a}{g_{bc}} - \frac{1}{\lag}\dv{t}(g_{ac}\dot{\chi}^{c}) + \frac{1}{\lag}g_{ac}\dot{\chi}^{c}\dv{\ln{\lag}}{t} = 0.
\end{align*}
The curve may be reparametrized such that $\lag$ is equal to $1$ everywhere, yielding
\begin{align*}
	\frac{1}{2\mathcal{L}}\left(\dot{\chi}^{a}\dot{\chi}^{b}\del{c}{g_{ab}} - \dv{t}(2\dot{\chi}^{a}g_{ac})\right) = 0.
\end{align*}
We note that the expression in the paranthesis is the Euler-Lagrange equation for the integral of $\mathcal{L}^{2}$. Expanding the derivative yields
\begin{align*}
	\frac{1}{\mathcal{L}}\left(\frac{1}{2}\dot{\chi}^{a}\dot{\chi}^{b}\del{c}{g_{ab}} - g_{ac}\ddot{\chi}^{a} - \dot{\chi}^{a}\dot{\chi}^{b}\del{b}{g_{ac}}\right) = 0.
\end{align*}
Multiplying this by $-g^{cd}\mathcal{L}$ yields
\begin{align*}
	g_{ac}g^{cd}\ddot{\chi}^{a} + \frac{1}{2}\dot{\chi}^{a}\dot{\chi}^{b}g^{cd}(2\del{b}{g_{ac}} - \del{c}{g_{ab}}) = g_{ac}g^{cd}\ddot{\chi}^{a} + \frac{1}{2}\dot{\chi}^{a}\dot{\chi}^{b}g^{cd}(\del{b}{g_{ac}} + \del{b}{g_{ac}} - \del{c}{g_{ab}}) = 0.
\end{align*}
The $a$ and $b$ indices are summed over, and may thus be swapped. Combined with the symmetry of the metric tensor, this yields
\begin{align*}
	g_{ac}g^{cd}\ddot{\chi}^{a} + \frac{1}{2}\dot{\chi}^{a}\dot{\chi}^{b}g^{cd}(\del{b}{g_{ac}} + \del{a}{g_{cb}} - \del{c}{g_{ab}}) = 0.
\end{align*}
Summation of the first term over $a$ gives $g_{ac}g^{cd}\ddot{\chi}^{a} = g^{cd}\ddot{\chi}_{c}$, and summation over $c$ gives $g^{cd}\ddot{\chi}_{c} = \ddot{\chi}^{d}$. This thus yields
\begin{align*}
	\ddot{\chi}^{d} + \frac{1}{2}\dot{\chi}^{a}\dot{\chi}^{b}g^{cd}(\del{b}{g_{ac}} + \del{a}{g_{cb}} - \del{c}{g_{ab}}) = 0.
\end{align*}

\paragraph{Christoffel symbols and the geodesic equation}
Consider a straight line with a tangent vector of constant magnitude. In euclidean space, this is a geodesic. This curve satisfies
\begin{align*}
	\dv{\dot{\vb{\gamma}}}{t} = (\dot{\vb{\gamma}}\cdot\grad)\dot{\vb{\gamma}} = \dot{\chi}^{a}\del{a}{\dot{\vb{\gamma}}} = \dot{\chi}^{a}(\dcov{a}{\dot{\chi}^{d}})\tb{d} = (\dot{\chi}^{a}\del{a}{\dot{\chi}^{d}} + \dot{\chi}^{a}\dot{\chi}^{c}\chris{d}{a}{c})\tb{d}.
\end{align*}
Comparison to the chain rule yields
\begin{align*}
	\dv{\dot{\vb{\gamma}}}{t} = (\ddot{\chi}^{a} + \dot{\chi}^{a}\dot{\chi}^{c}\chris{d}{a}{c})\tb{d}.
\end{align*}
Comparing this to the geodesic equation yields
\begin{align*}
	\chris{d}{a}{b} = \frac{1}{2}g^{dc}(\del{b}{g_{ac}} + \del{a}{g_{cb}} - \del{c}{g_{ab}}).
\end{align*}
A better approach would have been to go through the derivation of the geodesic equation again, identifying the Christoffel symbols as you go, but I have no idea how to do that.

\paragraph{The geometry of curved space}
In curved space, we face the restriction that there is no position vector. All vectors in curved space are instead restricted to the tangent space. It turns out that tangent vectors at a point have coordinates $\dot{\chi}^{a}$ and that the tangent vectors consist of the tangent vectors to the coordinate lines, i.e. partial derivatives.

We can also impose a metric tensor such that $\vb{v}\cdot\vb{w} = g_{ab}v^{a}w^{b}$, where the metric tensor is symmetric and positive definite.

Dual vectors can be defined as linear maps from tangent vectors to scalars, i. e. on the form
\begin{align*}
	V(\vb{w}) = V_{a}w^{a}.
\end{align*}
In particular, the dual vector $\dd{f}$ can be defined as
\begin{align*}
	\dd{f}(\vb{v}) = v^{a}\del{a}{f} = \dv{f}{t}
\end{align*}
along a curve with $\vb{v}$ as a tangent. A basis for the space of dual vectors is $e^{a} = \dd{\chi^{a}}$. The tangent and dual spaces, if a metric exists, are related by $v_{a} = g_{ab}v^{b}$.

Curve lengths are defined and computed as before. By defining geodesics as curves that extremize path length, this gives a set of Christoffel symbols and therefore a covariant derivative and a sense of what it means for a vector to change along a curve.