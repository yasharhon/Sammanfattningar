\section{Coordinates}

\paragraph{Coordinates}
A general set of coordinates on $\R^{n}$ is $n$ numbers $\chi^{a}, a = 1, \dots, n$ that uniquely define a point in the space.

\example{Cartesian coordinates}
In cartesian coordinates we introduce an orthonormal basis $\vb{e}_{i}$. We can then write $\vb{x} = \chi^{i}\vb{e}_{i}$. This example is, however, not very illustrative.

\paragraph{Basis vectors}
There are two different choices of coordinate bases.

The first is the tangent basis of vectors
\begin{align*}
	\vb{E}_{a} = \del{\chi^{a}}{\vb{x}} = \del{a}{\vb{x}}.
\end{align*}

The second is the dual basis
\begin{align*}
	\vb{E}^{a} = \grad{\chi^{a}}.
\end{align*}

\paragraph{Vector coordinates}
Any vector can now be written as
\begin{align*}
	\vb{v} = v^{a}\vb{E}_{a} = v_{a}\vb{E}^{a}.
\end{align*}
The $v^{a}$ are called contravariant components and the $v_{a}$ are called covariant components.

We can now compute the scalar product
\begin{align*}
	\vb{E}_{a}\cdot\vb{E}^{b} = \del{a}{\vb{x}}\cdot\grad{\chi^{b}} = \kdelta{a}{b}.
\end{align*}

\paragraph{Coordinate transformations}
Suppose that a vector can be written as
\begin{align*}
	\vb{v} = v^{a}\vb{E}_{a} = v^{a'}\vb{E}_{a'}'.
\end{align*}
How do we transform between these? A single component is given by
\begin{align*}
	v^{a'} = \vb{E}_{a'}'\cdot v^{a}\vb{E}_{a} = v^{a}(\grad{x'^{a'}}\cdot\del{a}{\vb{x}}) = v^{a}\del{a}{x'^{a'}}.
\end{align*}

\paragraph{Tangents to curves}
The tangent to a curve is given by
\begin{align*}
	\dot{\vb{\gamma}} = \dv{\vb{x}}{t} = \del{a}{\vb{x}}\dv{\chi^{a}}{t} = \dot{\chi}^{a}\vb{E}_{a}.
\end{align*}

\paragraph{Gradients}
The gradient of a curve is given by
\begin{align*}
	\grad{f} = \del{a}{f}\grad{\chi^{a}} = \vb{E}^{a}\del{a}{f}.
\end{align*}

\paragraph{Rates of change along a curve}
The rate of change of a quantity along a path is given by
\begin{align*}
	\dv{f}{t} = \del{a}{f}\dv{x^{a}}{t} = \grad{f}\cdot\dot{\vb{\gamma}}.
\end{align*}