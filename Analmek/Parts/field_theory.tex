\section{Classical field theory}

Classical field theory can be considered a limit of classical dynamics when the number of particles is infinite. The system obtains new ``coordinates'' $\phi$, which are functions of poisition and time. Summations over coordinates now become integrals over space.

\paragraph{Lagrangian dynamics}
The Lagrangian in a field theory now becomes
\begin{align*}
	L = \integ[D]{}{}{\vb{r}}{\lag}
\end{align*}
where $\lag$ is the Lagrangian density. From this we can obtain the action, and extremize it to obtain the equations for the time evolution of the system.

\paragraph{Solving models}
To solve models, we usually allow for periodic boundary conditions. The field is then expanded as a Fourier series, or a Fourier transform in the limit of a large domain or small lattice constant. We will in any case find that the system is compact in Fourier space, i.e. there are only non-zero contributions within some compact region.

\paragraph{Nöether's theorem}
In this context Nöether's theorem states that symmetries of a system are associated with conservative current. In field theory, a symmetry is a transformation $\phi\to\phi_{a}$, where $a$ is some continuous transformation parameter, such that for the quantity
\begin{align*}
	\dv{\lag}{a} = \dv{V^{\mu}}{x^{\mu}}
\end{align*}
there are quantities $j^{\mu}$ such that
\begin{align*}
	\dv{j^{\mu}}{x^{\mu}} = 0.
\end{align*}