\section{Classical Field theory}

Classical field theory can be considered a limit of classical dynamics when the number of particles is infinite. The system obtains new ``coordinates'' $\phi^{a}$, which are functions of position and time. Summations over coordinates now become integrals over space.

\paragraph{Lagrangian Formulation of Field Theory}
The Lagrangian in a field theory now becomes
\begin{align*}
	L = \integ[d]{}{}{\vb{r}}{\lag}
\end{align*}
where $\lag$ is the Lagrangian density, which in general depends on all involved fields, their derivatives with respect to both space and time and space and time themselves. From this we can obtain the action, and extremize it to obtain the equations for the time evolution of the system. The equations of motion are of the form
\begin{align*}
	\pdv{\lag}{\phi^{a}} - \pdv{t}\pdv{\lag}{(\del{t}{\phi^{a}})} - \pdv{x^{i}}\pdv{\lag}{(\del{i}{\phi^{a}})} = 0.
\end{align*}
Alternatively, by defining $x^{0} = ct$ for some speed $c$ and extending the summation, we can write
\begin{align*}
	\pdv{\lag}{\phi^{a}} - \pdv{x^{\mu}}\pdv{\lag}{(\del{\mu}{\phi^{a}})} = 0.
\end{align*}

\example{A String}
A string has the Lagrangian density
\begin{align*}
	\lag = \frac{1}{2}\left(\rho(\del{t}{\phi})^{2} - \frac{1}{2}T(\grad{\phi})^{2}\right).
\end{align*}
The equation of motion is
\begin{align*}
	0 - \rho\del[2]{t}{\phi} + T\del[2]{i}{\phi} = 0,
\end{align*}
which is the wave equation.

\example{Klein-Gordon Theory}
The Klein-Gordon Lagrangian density is
\begin{align*}
	\lag = \frac{1}{2}\left(\frac{1}{c^{2}}\rho(\del{t}{\phi})^{2} - (\grad{\phi})^{2} - m^{2}\phi^{2}\right).
\end{align*}
The equation of motion is
\begin{align*}
	-m^{2}\phi - \frac{1}{c^{2}}\del[2]{t}{\phi} + \del[2]{i}{\phi} = 0.
\end{align*}

\example{The Schrödinger Equation}

\paragraph{Hamiltonian Formulation}
In the Hamiltonian formalism, we define the momentum density
\begin{align*}
	\pi_{a} = \del{\phi^{a}}{\lag}.
\end{align*}
The Hamiltonian is now given by
\begin{align*}
	H = \integ[D]{}{}{\vb{r}}{\ham},
\end{align*}
where $\ham = \pi_{a}\del{t}{\phi^{a}} - \lag$. The Hamiltonian equations of motion become
\begin{align*}
	\dot{\phi} = \fdv{H}{\pi},\ \dot{\pi} = -\fdv{H}{\phi}.
\end{align*}
While the Hamiltonian formalism carries no issues with it in classical contexts, it does not generalize well to relativity due to the fact that it treats the time derivative differently to the spatial derivatives, which is a big no-no.

\example{A String}

\paragraph{Reduction to Discrete Problems}
For a problem on a compact domain, one can Fourier expand the fields (and the momentum densities) to obtain a discrete set of Fourier coefficients, the dynamics of which can be studied. It is this approach which will be the basis for quantum mechanics, where the coefficients will be replaced by occupation operators.

For problems on non-compact domains, we instead employ the Fourier transform as a tool. However, we have not really helped ourselves in this case.

\paragraph{Symmetries of Field Theories}
Consider a field theory (on Euclidean space) described by the Lagrangian density \lag. A symmetry of the system is a transformation of all involved coordinates and fields such that:
\begin{enumerate}
	\item $\lag$ retains its functional form under the transformation - in other words, the expression for the Lagrangian density is unchanged.
	\item The transformation changes the action by a constant value
\end{enumerate}

Before proceeding, it would also be useful to clarify what kinds of transformation we are considering. Transformations in field theory concern both transformations of the coordinates according to
\begin{align*}
	(x^{\prime})^{\mu} = x^{\mu} + \var{x^{\mu}}
\end{align*}
and of the fields according to
\begin{align*}
	(\phi^{\prime})^{a}((x^{\prime})^{\mu}) = \phi^{a}(x^{\mu}) + \var{\phi^{a}}((x^{\prime})^{\mu}).
\end{align*}
We will distinguish between the transformed fields and the change in the field at a particular point, given by
\begin{align*}
	(\phi^{\prime})^{a}(x^{\mu}) = \phi^{a}(x^{\mu}) + \bvar{\phi^{a}}(x^{\mu}).
\end{align*}

Note that the second requirement in the definition implies that
\begin{align*}
	\var{\lag} = \del{\mu}{V^{\mu}}.
\end{align*}

\paragraph{Nöether's Theorem}
Field theory also carries with it a version of Nöether's theorem, which will be covered here. A version will be presented here which is somewhat more restricted than the version presented for systems with discrete degrees of freedom - if you wanted to compare the two, you could say that this version only contains symmetries.

Consider the action of a symmetry on a given system. The requirement on the action can be written as
\begin{align*}
	\integ{\Omega^{\prime}}{}{(x^{\prime})^{\mu}}{\lag^{\prime}} - \integ{\Omega}{}{x^{\mu}}{\lag} = \integ{\Omega}{}{x^{\mu}}{\del{\mu}{V^{\mu}}}.
\end{align*}
The functional form of the Lagrangian density is unchanged, which also carries the consequence that the integration variables may be renamed. This yields
\begin{align*}
	\integ{\Omega^{\prime}}{}{x^{\mu}}{\lag((\phi^{\prime})^{a}, x^{\mu})} - \integ{\Omega}{}{x^{\mu}}{\lag(\phi^{a}, x^{\mu})} = \integ{\Omega}{}{x^{\mu}}{\del{\nu}{V^{\nu}}}.
\end{align*}

%TODO: Extend to multiple dimensions
We reuse the argument from the previous proof, generalizing it (somehow) to multiple dimensions to obtain
\begin{align*}
	\integ{\Omega}{}{x^{\mu}}{\lag((\phi^{\prime})^{a}, x^{\mu}) - \lag(\phi^{a}, x^{\mu})} + \integ{S}{}{S_{\mu}}{\lag(\phi^{a}, x^{\mu})\var{x^{\mu}}} = \integ{\Omega}{}{x^{\mu}}{\lag((\phi^{\prime})^{a}, x^{\mu}) - \lag(\phi^{a}, x^{\mu}) + \pdv{x^{\nu}}(\lag(\phi^{a}, x^{\mu})\var{x^{\nu}})},
\end{align*}
and thus
\begin{align*}
	\integ{\Omega}{}{x^{\mu}}{\lag((\phi^{\prime})^{a}, x^{\mu}) - \lag(\phi^{a}, x^{\mu}) + \pdv{x^{\nu}}(\lag(\phi^{a}, x^{\mu})\var{x^{\nu}})} = \integ{\Omega}{}{x^{\mu}}{\del{\nu}{V^{\nu}}},
\end{align*}
where $S$ is the boundary of $\Omega$ and we have made (hopefully proper) use of the $n$-dimensional divergence theorem. The difference in the first two terms can be expanded to first order as
\begin{align*}
	\lag((\phi^{\prime})^{a}, x^{\mu}) - \lag(\phi^{a}, x^{\mu}) &= \pdv{\lag}{\phi^{a}}\bvar{\phi}^{a} + \pdv{\lag}{(\del{\nu}{\phi^{a}})}\bvar{(\del{\nu}{\phi^{a}})}.
\end{align*}
The use of the variation at a specific point is due to the fact that both Lagrangians are now evaluated at the same points. This is significant because while the total variation does not commute with the differentiation operators, this one does. Using the equations of motion, we additionally obtain
\begin{align*}
	\pdv{\lag}{\phi^{a}}\bvar{\phi}^{a} + \pdv{\lag}{(\del{\nu}{\phi^{a}})}\del{\nu}{\bvar{\phi^{a}}} = \bvar{\phi}^{a}\pdv{x^{\nu}}\pdv{\lag}{(\del{\nu}{\phi^{a}})} + \pdv{\lag}{(\del{\nu}{\phi^{a}})}\del{\nu}{\bvar{\phi^{a}}} = \pdv{x^{\nu}}\left(\bvar{\phi}^{a}\pdv{\lag}{(\del{\nu}{\phi^{a}})}\right).
\end{align*}
Hence we have
\begin{align*}
	\integ{\Omega}{}{x^{\mu}}{\pdv{x^{\nu}}\left(\bvar{\phi}^{a}\pdv{\lag}{(\del{\nu}{\phi^{a}})} + \lag\var{x^{\nu}} - V^{\nu}\right)} = 0,
\end{align*}
which is already in the form of a conservation law for the quantities
\begin{align*}
	\bvar{\phi}^{a}\pdv{\lag}{(\del{0}{\phi^{a}})} + \lag\var{x^{0}} - V^{0}
\end{align*}
and the corresponding currents
\begin{align*}
	\bvar{\phi}^{a}\pdv{\lag}{(\del{i}{\phi^{a}})} + \lag\var{x^{i}} - V^{i}.
\end{align*}

Next, we use the expansion
\begin{align*}
	\var{\phi}^{a}  &= \bvar{\phi}^{a} + \del{\mu}{\phi^{a}}\var{x}^{\mu}
\end{align*}
to obtain
\begin{align*}
	\integ{\Omega}{}{x^{\mu}}{\pdv{x^{\nu}}\left((\var{\phi}^{a} - \del{\mu}{\phi^{a}}\var{x}^{\mu})\pdv{\lag}{(\del{\nu}{\phi^{a}})} + \lag\var{x^{\nu}} - V^{\nu}\right)} = 0.
\end{align*}
The argument extends arbitrarily to the interior of the integration domain, hence we must have
\begin{align*}
	\del{\nu}{J^{\nu}} = 0,
\end{align*}
where
\begin{align*}
	J^{\nu} = V^{\nu} - \var{\phi}^{a}\pdv{\lag}{(\del{\nu}{\phi^{a}})} + \left(\del{\mu}{\phi^{a}}\pdv{\lag}{(\del{\nu}{\phi^{a}})} - \kdelta{\mu}{\nu}\lag\right)\var{x^{\mu}}.
\end{align*}
This is the final form of Nöether's theorem.

\example{Energy Conservation}
Consider a transformation where only time is varied (normalized to $1$) acting on a Lagrangian density with no explicit time dependence. The variation of the Lagragian is zero under this transformation as it has no time dependence. Nöether's theorem gives the currents as
\begin{align*}
	J^{\nu} = \left(\del{\mu}{\phi^{a}}\pdv{\lag}{(\del{\nu}{\phi^{a}})} - \kdelta{\mu}{\nu}\lag\right)\kdelta{0}{\mu} = \del{0}{\phi^{a}}\pdv{\lag}{(\del{\nu}{\phi^{a}})} - \kdelta{0}{\nu}\lag.
\end{align*}
Its time component is
\begin{align*}
	J^{0} = \del{i}{\phi^{a}}\pdv{\lag}{(\del{0}{\phi^{a}})} - \lag = \ham,
\end{align*}
hence this symmetry corresponds to the conservation of total energy.

\paragraph{Poisson Brackets}
Poisson brackets of two functionals on phase space are defined as
\begin{align*}
	\pob{F}{G} = \integ[D]{}{}{\vb{r}}{\del{\phi}{F}\del{\pi}{G} - \del{\pi}{F}\del{\phi}{G}}
\end{align*}
We can somehow show that
\begin{align*}
	\pob{\phi(x)}{\phi(y)} = \pob{\pi(x)}{\pi(y)} = 0,\ \pob{\phi(x)}{\pi(y)} = \delta(x - y).
\end{align*}