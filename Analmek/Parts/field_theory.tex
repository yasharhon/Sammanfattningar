\section{Classical Field theory}

Classical field theory can be considered a limit of classical dynamics when the number of particles is infinite. The system obtains new ``coordinates'' $\phi^{a}$, which are functions of position and time. Summations over coordinates now become integrals over space.

\paragraph{Lagrangian Formulation of Field Theory}
The Lagrangian in a field theory now becomes
\begin{align*}
	L = \integ[d]{}{}{\vb{r}}{\lag}
\end{align*}
where $\lag$ is the Lagrangian density, which in general depends on all involved fields, their derivatives with respect to both space and time and space and time themselves. From this we can obtain the action, and extremize it to obtain the equations for the time evolution of the system. The equations of motion are of the form
\begin{align*}
	\dv{\lag}{\phi^{a}} - \pdv{t}\pdv{\lag}{(\del{t}{\phi^{a}})} - \pdv{x^{i}}\pdv{\lag}{(\del{i}{\phi^{a}})} = 0.
\end{align*}
Alternatively, by defining $x^{0} = t$ and extending the summation, we can write
\begin{align*}
	\dv{\lag}{\phi^{a}} - \pdv{x^{b}}\pdv{\lag}{(\del{b}{\phi^{a}})} = 0.
\end{align*}

\paragraph{Hamiltonian Formulation}
In the Hamiltonian formalism, we define the momentum density
\begin{align*}
	\pi_{a} = \del{\phi^{a}}{\lag}.
\end{align*}
The Hamiltonian is now given by
\begin{align*}
	H = \integ[D]{}{}{\vb{r}}{\ham},
\end{align*}
where $\ham = \pi_{a}\del{t}{\phi^{a}} - \lag$. The Hamiltonian equations of motion become
\begin{align*}
	\dot{\phi} = \fdv{H}{\pi},\ \dot{\pi} = -\fdv{H}{\phi}.
\end{align*}
While the Hamiltonian formalism carries no issues with it in classical contexts, it does not generalize well to relativity due to the fact that it treats the time derivative differently to the spatial derivatives, which is a big no-no.

\paragraph{Reduction to Discrete Problems}
For a problem on a compact domain, one can Fourier expand the fields (and the momentum densities) to obtain a discrete set of Fourier coefficients, the dynamics of which can be studied. It is this approach which will be the basis for quantum mechanics, where the coefficients will be replaced by occupation operators.

For problems on non-compact domains, we instead employ the Fourier transform as a tool. However, we have not really helped ourselves in this case.

\paragraph{Nöether's Theorem}
In this context Nöether's theorem states that symmetries of a system are associated with conservative current. In field theory, a symmetry is a transformation $\phi\to\phi_{a}$, where $a$ is some continuous transformation parameter, such that for the quantity
\begin{align*}
	\dv{\lag}{a} = \dv{V^{\mu}}{x^{\mu}}
\end{align*}
there are quantities $j^{\mu}$ such that
\begin{align*}
	\dv{j^{\mu}}{x^{\mu}} = 0.
\end{align*}

\paragraph{Poisson brackets}
Poisson brackets of two functionals on phase space are defined as
\begin{align*}
	\pob{F}{G} = \integ[D]{}{}{\vb{r}}{\del{\phi}{F}\del{\pi}{G} - \del{\pi}{F}\del{\phi}{G}}
\end{align*}
We can somehow show that
\begin{align*}
	\pob{\phi(x)}{\phi(y)} = \pob{\pi(x)}{\pi(y)} = 0,\ \pob{\phi(x)}{\pi(y)} = \delta(x - y).
\end{align*}