\section{Group theory}

\paragraph{Definition of a group}
A grou is a set of objects $G$ with an operation $G\times G\to G,\ (a, b)\to ab$ such that
\begin{itemize}
	\item If $a, b\in G$ then $ab\in G$.
	\item $a(bc) = (ab)c$ for all $a, b, c\in G$.
	\item There exists an identity $e$ such that $ae = ea = a$ for all $a\in G$.
	\item There exists for every element $a$ an inverse $a^{-1}\in G$ such that $aa^{-1} = a^{-1}a = e$.
\end{itemize}
Groups can be
\begin{itemize}
	\item cyclic, i.e. all elements in the group are powers of a single element.
	\item finitie, i.e. groups containing a finite number of elements.
	\item infinite.
	\item discrete, i.e. all elements in the group can be labeled with some index.
	\item continuous.
	\item commutative, i.e. $ab = ba$ for all elements in the group.
\end{itemize}

\paragraph{Subgroups}
If $G = \{g_{\alpha}\}$ and the subset $H = \{h_{\alpha}\}$ is also a group, we call $H$ a subgroup of $G$ and write $H < G$.

\paragraph{Direct products}
GIven two groups $F$ and $G$, we define $F\times G$ as the set of ordered pairs of elements of the two groups. The group action of $F\times G$ is the group actions of $F$ and $G$ separately on the elements in the ordered pair.

\paragraph{Homomorphisms and isomorphisms}
A homomorphisms is a map $f: G\to H$ such that $f(g_{1})f(g_{2}) = f(g_{1}g_{2})$. If the map is bijective, $f$ is called an isomorphism.

\paragraph{Point groups}
Point groups are symmetries of, for instance, a crystal structure that leave at least one point in the structure invariant. Examples include
\begin{itemize}
	\item rotations.
	\item reflections.
	\item spatial inversions.
\end{itemize}
Combined with certain discrete translation, these are the space groups of the crystal. Space groups are the groups of all symmetries of a crystal.

\paragraph{Small and large rotations}
Consider a rotation of an infinitesimal displacement $\dd{\vb{x}}$ with a rotation $R$. The requirement for length to be preserved implies $R^{T}R = 1$.

Consider now a rotation by a small angle $\var{\theta}$. Taylor expanding it in terms of the angle yields
\begin{align*}
	R(\var{\theta}) \approx 1 + A\var{\theta}.
\end{align*}
The requirement for $R$ to be orthogonal yields $A^{T} = -A$. We choose the solution
\begin{align*}
	J =
	\mqty[
		0  & 1 \\
		-1 & 0
	].
\end{align*}
We can now write the rotation matrix as
\begin{align*}
	R(\var{\theta}) =
	\mqty[
		1             & \var{\theta} \\
		-\var{\theta} & 1
	].
\end{align*}

We would now like to construct a large rotation in terms of smaller rotations as
\begin{align*}
	R(\theta) = \lim\limits_{N\to\infty}\left(1 + \frac{\theta}{N}J\right)^{N} = e^{\theta J}.
\end{align*}
We can write this as an infinite series and use the fact that $J^{2} = -1$ to obtain
\begin{align*}
	R(\theta) = \cos{\theta} + J\sin{\theta}.
\end{align*}

\paragraph{Dihedral groups}
The dihedral group $D_{n}$ is the group of transformations that leave an $n$-sided polygon invariant.

\paragraph{Symmetries in classical mechanics}

\example{Newton's second law}
Newton's second law $m\ddot{\vb{x}} = -\grad{V}$, assuming the potential to be fixed, has certain symmetry properties:
\begin{itemize}
	\item The transformation $t\to t' = t + t_{0}$ is a symmetry, as $\dv{t} = \dv{t'}$ and $V$ is not changed under the transformation.
	\item The transformation $t\to\tau = -t$ is a symmetry as $\dv{t} = \dv{\tau}{t}\dv{\tau} = -\dv{\tau}$, which implies $\dv[2]{t} = \dv[2]{\tau}$ and $V$ is not changed under the transformation.
	\item Considering a system of particles, if the forces between these only depend on differences between the position vectors, the translation $\vb{x}_{i}\to\vb{y}_{i} = \vb{x}_{i} + \vb{x}_{0}$ is a symmetry as it does not change any differences.
\end{itemize}

\example{Constraining solutions using symmetries}
If a system is invariant under some transformation $\vb{x}\to\vb{R}(\vb{x})$, then any property $u$ dependant on those coordinates satisfies $u(\vb{x}) = u(\vb{R}(\vb{x}))$.

\paragraph{Connection to Noether's theorem}
We defined symmetries of the action as transformations that satisy $\var{\lag} = 0$. In particular, we can construct a set of transformations such that $\del{s}{t} = \var{t},\ \del{s}{q^{a}} = \var{q^{a}}$, where $s$ is the symmetry parameter. This is a one-parameter family of symmetries. By defining $T_{s}q(t, 0) = q(t, s)$, these symmetries satisfy
\begin{align*}
	T_{s_{2}}T_{s_{1}}q(t, 0) = T_{s_{1} + s_{2}}q(t, 0).
\end{align*}
We see that these symmetries define a group.

\example{A particle in a moving potential}