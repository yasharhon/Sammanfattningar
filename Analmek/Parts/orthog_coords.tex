\section{Differentiation and Integration in Orthogonal Coordinates}

To tie together what we have learned thus far with what we studied in Vector Calculus, we will study differentiation and integration in orthogonal coordinate systems. For this part of the summary we will take a break from the oh-so strict indexing rules established above.

\paragraph{Defining Relation}
Orthogonal coordinate systems are defined by the relation
\begin{align*}
	\tb{a}\cdot\tb{b} = h_{a}^{2}\delta_{ab}\nosum .
\end{align*}

\paragraph{Orthonormal Basis}
Based on the orthogonality conditions, we define the orthogonal basis vectors
\begin{align*}
	\vb{e}_{a} = \frac{1}{h_{a}}\tb{a}\nosum .
\end{align*}
The normalization thus implies
\begin{align*}
	h_{a} = \sqrt{\sum\limits_{i}(\del{a}{x^{i}})^{2}}.
\end{align*}

\paragraph{Physical Components}
The physical components of a vector is its projection onto the orthonormal basis vectors, denoted with a tilde.

\paragraph{Relation to Dual Basis}
By expanding the dual basis vectors in terms of their physical components, we obtain
\begin{align*}
	\tilde{E}_{a}^{b} = \vb{e}_{b}\cdot\db{a} = \frac{1}{h_{b}}\kdelta{b}{a}\nosum .
\end{align*}
This implies
\begin{align*}
	\db{a} = \tilde{E}_{b}^{a}\vb{e}_{b} = \frac{1}{h_{a}}\kdelta{a}{b}\vb{e}_{b} = \frac{1}{h_{a}}\vb{e}_{a}\nosum ,
\end{align*}
and thus
\begin{align*}
	\vb{e}_{a} = h_{a}\db{a}.
\end{align*}
We see that the dual basis would have been an equally good starting point for describing orthogonal systems.

\paragraph{Line Integrals}
Using our previous knowledge of rates of change along a curve, we have
\begin{align*}
	\vinteg{\Gamma}{}{r}{\vb{v}} &= \integ{\Gamma}{}{\chi^{a}}{\tb{a}\cdot\vb{v}} \\
	                             &= \integ{\Gamma}{}{\tau}{\dot{\chi}^{a}\tb{a}\cdot\vb{v}} \\
	                             &= \integ{\Gamma}{}{\tau}{\sum\limits_{a}\dot{\chi}^{a}h_{a}\tilde{v}_{a}}.
\end{align*}
Specifically, when integrating along a $\chi^{c}$ coordinate line, we can use the coordinate as a parameter, yielding
\begin{align*}
	\vinteg{\Gamma}{}{r}{\vb{v}} = \integ{\Gamma}{}{\tau}{\dv{\chi^{a}}{\chi^{c}}\tb{a}\cdot\vb{v}} = \integ{\Gamma}{}{\chi^{c}}{h_{c}v_{c}}.
\end{align*}

\paragraph{Surface Integrals}
Consider a coordinate level surface $S_{c}$. In three dimensions we have
\begin{align*}
	\dd{\vb{S}} &= \del{a}{\vb{r}}\times\del{b}{\vb{r}}\dd{\chi^{a}}\dd{\chi^{b}} \\
	            &= h_{a}h_{b}\vb{e}_{a}\vb{e}_{b}\dd{\chi^{a}}\dd{\chi^{b}} \\
	            &= \pm h_{a}h_{b}\dd{\chi^{a}}\dd{\chi^{b}}\vb{e}_{c}.
\end{align*}
We immediately identify the unit normal and area element.

The final results (hopefully) generalize to other dimensionalities, but I could not see any way of bypassing the need for the cross product in three dimensions.

\paragraph{Volume Integrals}
Consider an infinitesimal volume element separated by $2n$ coordinate surfaces corresponding to coordinate values $\chi^{a}$ and $\chi^{a} + \dd{\chi^{a}}$. Using what we did with line integrals
\begin{align*}
	\dd{V} = \prod\limits_{i}h_{i}\dd{\chi^{i}}.
\end{align*}
We identify the Jacobian as $\J = \prod\limits_{i}h_{i}$.