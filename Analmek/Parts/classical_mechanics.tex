\section{Classical mechanics}

\paragraph{Lagrangian mechanics and differential geometry}
In Lagrangian mechanics, configuration space is the space of all possible configurations of a system. We can impose coordinates $\chi^{a}$ on this space in order to use what we know from the previous parts on differential geometry. Note that the term configuration does not exclude the discussion of systems other than the purely mechanical with Lagrangian mechanics. This is a clear advantage of Lagrangian mechanics as opposed to Newtonian mechanics.

\paragraph{Hamilton's principle}
Hamilton's principle replaces Newton's laws as the fundamental law of nature that is postulated in order to start off the theory. To formulate it, we first define the action of a system over time is defined as
\begin{align*}
	S = \integ{}{}{t}{\lag},
\end{align*}
where $\lag$ is the Lagrangian (discussed below). Hamilton's principle states that for the motion of the system in configuration space, $\var{S} = 0$. This can be expressed as
\begin{align*}
	\var{S} = \integ{}{}{t}{\var{\lag}} = \integ{}{}{t}{\left(\del{\chi^{a}}{\lag} - \dv{t}\del{\dot{\chi}^{a}}{\lag}\right)\var{\chi^{a}}} = 0.
\end{align*}
The equations of motion of the system are thus of the form
\begin{align*}
	\del{\chi^{a}}{\lag} - \dv{t}\del{\dot{\chi}^{a}}{\lag} = 0.
\end{align*}

\paragraph{How to form a theory}
In Lagrangian mechanics, the recipe for formulating a theory to describe a system is simple: Introduce its Lagrangian and extremize the action to find the equations describing the system.

\paragraph{The Lagrangian}
The Lagrangian is essential in Lagrangian mechanics. For a system of particles in a conservative force field, it can be constructed as $\lag = E_{\text{k}} - V$. $V$ is the potential energy and is taken to be a function only of the coordinates. Note that this is not the only way to construct a Lagrangian - for instance, adding the total time derivative of some function adds a constant term to the action, and so has no effect on the variational analysis. In addition, Lagrangian mechanics can be used to describe entirely different systems in which terms such as kinetic energy do not make sense. In fact, this is an important feature of Lagrangian mechanics.

\example{An inductor}
Consider a lone inductor with inductance $L$ over which some (possibly time-dependent) potential $V$ is applied. According to classical electrodynamics, we have that
\begin{align*}
	V = L\dv{I}{t}.
\end{align*}
Let us see if we can reconstruct this using Lagrangian mechanics.

We first need to impose coordinates on the system. I choose the lone coordinate $q$ as the amount of charge carried across the inductor. Suppose that the system had a Lagrangian $\lag = \frac{1}{2}L\dot{q}^{2} - qV$. The equation of motion is
\begin{align*}
	-V - \dv{t}(L\dot{q}) = 0,\ L\ddot{q} = V.
\end{align*}
Recognizing that $\dot{q} = I$, we arrive at the desired result
\begin{align*}
	V = L\dv{I}{t}.
\end{align*}

At this point I feel it only reasonable to discuss how I (or rather, Goldstein) arrived at this particular Lagrangian. To the trained eye it is very clear that the given equation of motion would result from that Lagrangian, but surely the core idea cannot be to just guess Lagrangians based on what we already know? Well, yes it can. The goal of physics is to describe reality, so if someone told you that their measurements implied the given equations of motion, is it not your job as a theorist to construct a theory which describes that experiment? And how do you that, if not by constructing an appropriate Lagrangian? It might seem somewhat ad hoc to construct theories based on desired results in this fashion, but the most important check for a theory to satisfy is its compliance with the reality around us. If that is ad hoc to you, then perhaps mathematics will feel more satisfactory to you than physics.

There is also an argument from the physics we already know. Having introduced the coordinate $q$, we know that the potential energy is $qV$. In addition, the instantaneous power absorbed by the inductor is $P = IV = \dot{q}V$. Using the already known equation of motion, this can be written as $P = L\dot{q}\ddot{q}$. Integrating this over time gives that the inductor has energy $\frac{1}{2}L\dot{q}^{2}$. This does not have pure coordinate dependance, so we can use it as a kinetic term in the Lagrangian. Thus we arrive at the Lagrangian we used.

Again I feel my own skepticism, as it seems that the Lagrangian we constructed already contained some information about the system which it describes. It seems that the Lagrangian method couldn't possibly arrive at a different result, so now we are just using what we already know to rederive what we knew to begin with. And in a certain sense, this is correct. That doesn't mean that Lagrangian mechanics is useless or superfluous. The logical structure of physical theory is Babylonian, meaning that it is made to contain certain core results (namely, experimental results) and is constructed from certain starting points (namely, postulates). Beyond this it is non-directional, meaning that there is no need for certain results to build on others in a specific order. Translated and isolated to our example, there is no difference between going from the Lagrangian to the equations of motion and going from the equations of motion to the Lagrangian. Of course, physics as a whole value the Lagrangian way higher, as it is a more consistent way of doing a wide range of physics than simply starting from a wide variety of individual relations between quantities.

\paragraph{Kinetic energy}
Kinetic energy is defined by a rank $2$ tensor as
\begin{align*}
	E_{\text{k}} = \frac{1}{2}T_{ab}\dot{\chi}^{a}\dot{\chi}^{b},
\end{align*}
where the dot now really represents the time derivative.

%Can be used to show H = T + V

\paragraph{The kinetic metric}
Consider a system with no potential energy. The Lagrangian simply becomes $\lag = \frac{1}{2}T_{ab}\dot{\chi}^{a}\dot{\chi}^{b}$. The action computed from this Lagrangian is very similar to the integral of curve length (or, rather its square, the extremum of which was noted to be the same), except $g_{ab}$ has been replaced by $T_{ab}$. This inspires us to define $T_{ab}$ as the kinetic metric, with corresponding Christoffel symbols.

\paragraph{Motion of a classical system}
By defining $a^{b} = \dot{\chi}^{a}\dcov{a}{\dot{\chi}^{b}}$, the previous work leads us to a system with no potential satisfying $a^{b} = \ddot{\chi}^{b} + \chris{b}{a}{c}\dot{\chi}^{a}\dot{\chi}^{c} = 0$. In other words, a system with no potential moves along the geodesics of the kinetic metric.

For a system with a potential, only the $\del{\chi^{a}}{\lag}$ term is affected, and
\begin{align*}
	a^{b} = - T^{ba}\del{a}{V} = T^{ba}F,
\end{align*}
which is a generalization of Newton's second law.

\paragraph{Legendre transforms}
To illustrate the Legendre transform, consider a function $f(x, y)$ and $g(x, y, u) = ux - f(x, y)$. Its total derivative is given by
\begin{align*}
	\dd{g} = u\dd{x} + x\dd{u} - \del{x}{f}\dd{x} - \del{y}{f}\dd{y}.
\end{align*}
By choosing $u = \del{x}{f}$, we obtain
\begin{align*}
	\dd{g} = x\dd{u} - \del{y}{f}\dd{y},
\end{align*}
implying that $g$ is only a function of $u$ and $y$. To obtain $g$, invert the definition of $u$ to obtain $x(u, y)$.

\paragraph{Hamiltonian mechanics}
The Lagrangian equations of motion are $n$ coupled second-order differential equations. Hamiltonian mechanics starts with trying to decouple these into $2n$ first-order differential equations. To illustrate how this is done, consider an equation of motion
\begin{align*}
	\del{q^{a}}{\lag} - \dv{t}\del{\dot{q}^{a}}{\lag} = 0.
\end{align*}
The second-order term is the place to hack away at. We thus define the generalized momenta
\begin{align*}
	p_{a} = \del{\dot{q}^{a}}{\lag}.
\end{align*}
Up until now, we have mathematically treated the coordinates and velocities as variables, making no difference between them in a technical manner. The next step is now to replace the velocities with the momenta. As the Lagrangian describes the system, we do this by Legendre transforming the Lagrangian.

We now define the Hamiltonian
\begin{align*}
	\ham = p_{i}\dot{q}^{i} - \lag.
\end{align*}
From this definition we would like to obtain new equations of motion. This is done by computing the differential of the Hamiltonian. We have
\begin{align*}
	\dd{\lag} = \del{q^{i}}{\lag}\dd{q^{i}} + \del{\dot{q}^{i}}{\lag}\dd{\dot{q}^{i}} + \del{t}{\lag}\dd{t}.
\end{align*}
The definition of the general momenta and the equations of motion allows us to write this as
\begin{align*}
	\dd{\lag} = \dot{p}_{i}\dd{q^{i}} + p_{i}\dd{\dot{q}^{i}} + \del{t}{\lag}\dd{t}.
\end{align*}
The differential of the Hamiltonian is
\begin{align*}
	\dd{\ham} = p_{i}\dd{\dot{q}^{i}} + \dot{q}^{i}\dd{p_{i}} - \dot{p}_{i}\dd{q^{i}} - p_{i}\dd{\dot{q}^{i}} - \del{t}{\lag}\dd{t} = \dot{q}^{i}\dd{p_{i}} - \dot{p}_{i}\dd{q^{i}} - \del{t}{\lag}\dd{t},
\end{align*}
serving as an example of how the Legendre transform works. This implies that the equations  of motion are
\begin{align*}
	\dot{p}_{i} = -\del{q_{i}}{\ham},\ \dot{q}_{i} = \del{p_{i}}{\ham}.
\end{align*}

In Lagrangian mechanics, we considered paths in configuration space. In Hamiltonian mechanics, we instead consider paths in phase space, i.e. a space where the points are $(q, t)$. In this space, paths do not intersect as the system is deterministic. Paths in phase space are periodic for integrable systems and fill out the accessible parts of phase space for chaotic systems.

We note that
\begin{align*}
	\dv{\ham}{t} &= \del{q_{i}}{\ham}\dot{q}_{i} + \del{p_{i}}{\ham}\dot{p}_{i} + del{t}{\ham} \\
	             &= -\dot{p}_{i}\dot{q}_{i} + \dot{q}_{i}{\ham}\dot{p}_{i} + del{t}{\ham}
	             &= \del{t}{\ham},
\end{align*}
and so the Hamiltonian is conserved if it has no explicit time dependence.

\paragraph{Noether's theorem}
Noether's theorem relates symmetries - or, more specifically, quasi-symmetries, of physical systems to conservation laws.

What is a quasi-symmetry, then? Consider a one-parameter transformation $t\to\tau(t, s),\ q^{a}\to Q^{a}(q, s)$, where $s$ is the parameter with respect to which the system is transformed, such that $\tau(t, 0) = t,\ Q^{a}(q, 0) = q^{a}$ and for small $s = \varepsilon$ that $t\to t + \varepsilon\var{t},\ q^{a}\to q^{a} + \varepsilon\var{q^{a}}$. This is assumed to be normalized such that $\var{t}$ is either $0$ or $1$. How? Don't ask. A quasi-symmetry of a system with Lagrangian $\lag$ is a transformation such that
\begin{align*}
	\varepsilon\var{\lag} = \lag(Q, \dot{Q}, \tau) - \lag(q, \dot{q}, t) = \varepsilon\dv{F}{t}
\end{align*}
for some $F$. The variation of the Lagrangian can be written as
\begin{align*}
	\var{\lag} = \del{q^{a}}{\lag}\var{q^{a}} + \del{\dot{q}^{a}}{\lag}\var{\dot{q}^{a}} + \del{t}{\lag}\var{t}.
\end{align*}
The total time derivative of the Lagrangian is given by
\begin{align*}
	\dv{\lag}{t} = \del{t}{\lag} + \del{q^{a}}{\lag}\dot{q}^{a} + \del{\dot{q}^{a}}{\lag}\ddot{q}^{a},
\end{align*}
which yields
\begin{align*}
	\var{\lag} = \del{q^{a}}{\lag}(\var{q^{a}} - \dot{q}^{a}\var{t}) + \del{\dot{q}^{a}}{\lag}(\var{\dot{q}^{a}} - \ddot{q}^{a}\var{t}) + \dv{\lag}{t}\var{t}.
\end{align*}
The equations of motion are $\del{q^{a}}{\lag} = \dv{t}\del{\dot{q}^{a}}{\lag}$. For a set of coordinates that satisfy this - a so-called on-shell solution - we have
\begin{align*}
	\var{\lag} &= \dv{t}\del{\dot{q}^{a}}{\lag}(\var{q^{a}} - \dot{q}^{a}\var{t}) + \del{\dot{q}^{a}}{\lag}(\var{\dot{q}^{a}} - \ddot{q}^{a}\var{t}) + \dv{\lag}{t}\var{t} \\
	           &= \dv{t}\left(\del{\dot{q}^{a}}{\lag}(\var{q^{a}} - \dot{q}^{a}\var{t}) + \lag\var{t}\right).
\end{align*}
If the transformation is a quasi-symmetry of the system, then this is equal to a total time derivative of $F$, and the quantity
\begin{align*}
	J = F - \del{\dot{q}^{a}}{\lag}\var{q^{a}} + (\dot{q}^{a}\del{\dot{q}^{a}}{\lag} - \lag)\var{t}
\end{align*}
thus satisfies $\dv{J}{t} = 0$. We can introduce the general momenta and the Hamiltonian to rewrite this as
\begin{align*}
	J = F - p_{a}\var{q^{a}} + \ham\var{t}.
\end{align*}

We arrive at the conclusion that $J$ is a conserved quantity under a quasi-symmetry of the system. Identifying the conservation laws of a system is thus a matter of identifying the quasi-symmetries of a system and computing $J$ under that transformation.

\example{A free particle in space}
Consider a free particle in space. Its Lagrangian is given by $\lag = \frac{1}{2}m\dot{\vb{x}}^{2}$, and the variation of this is
\begin{align*}
	\var{\lag} = m\dot{\vb{x}}\cdot\var{\dot{\vb{x}}}.
\end{align*}
Its general momentum is
\begin{align*}
	\vb{p} = \del{\dot{\vb{x}}}{\lag} = m\dot{\vb{x}}.
\end{align*}
The Hamiltonian is
\begin{align*}
	\ham = \vb{p}\cdot\dot{\vb{x}} - \lag = \frac{1}{2}m\dot{\vb{x}}^{2}.
\end{align*}
We now want to identify quasi-symmetries of the system that make the variation of the Lagrangian either zero or the time derivative of some quantity. A key idea here is that we are only allowed to change the variations (or so I think).

A first attempt is keeping $\var{\vb{x}}$ constant and not varying thiime(a spatial translation), which implies $\var{\dot{\vb{x}}} = \vb{0}$ and $\var{\lag} = 0$. This implies that $F$ is constant. The conserved quantity is thus
\begin{align*}
	J = F - \vb{p}\cdot\var{\vb{x}} = F - \vb{p}\cdot\vb{c},
\end{align*}
i.e. the momentum of the system is conserved. We also note that the constant $F$ in this case is arbitrary, and we might as well have set it to $0$. This will be the case at least sometimes.

A second attempt is varying time, i.e. $\var{t} = 1$, but keeping the coordinates fixed, i.e. $\var{\vb{x}} = 0$ (a time translation). This yields $\var{\dot{\vb{x}}} = \vb{0}$ and $\var{\lag} = 0$. Once again $F$ is constant and taken to be zero, and the conserved quantity is thus $J = H$, i.e. the Hamiltonian of the system is conserved.

A third attempt is to somehow make the scalar product in the variation of the Lagrangian zero, without varying time. An option is $\var{\vb{x}} = \vb{\omega}\times\vb{x}$, where $\vb{\omega}$ is a constant vector. This yields $\var{\dot{\vb{x}}} = \vb{\omega}\times\dot{\vb{x}}$ and $\var{\lag} = 0$. The conserved quantity is thus
\begin{align*}
	J &= -\vb{p}\cdot(\vb{\omega}\times\vb{x}) \\
	  &= -\vb{\omega}\cdot(\vb{x}\times\vb{p}).
\end{align*}
Since $\vb{\omega}$ is constant, that means that $\vb{x}\times\vb{p}$, i.e. the angular momentum, is conserved.

\paragraph{Liouville's theorem}
As paths in phase space do not cross, we can think of the time evolution of a system as a flow in phase space. The volume element is $\dd{V} = \dd{q}\dd{p}$. Liouville's theorem states that flow in phase space is incompressible.

To show this, consider the state at some point in time and after some infinitesimal time $\dd{t}$. Denote the point in phase space at the start as $(q, p)$ and after $\dd{t}$ as $(q', p').$ To first order in time we have
\begin{align*}
	q_{i}' = q_{i} + \dot{q}_{i}\dd{t} = q_{i} + \del{p_{i}}{\ham}\dd{t},\ p_{i}' = p_{i} + \dot{p}_{i}\dd{t} = p_{i} - \del{q_{i}}{\ham}\dd{t}.
\end{align*}
The volume element is given by
\begin{align*}
	\dd{V}' &= \left(\dd{q} +  \left(\del{q}{\del{p}{\ham}}\dd{q} + \del[2]{p}{\ham}\dd{p}\right)\dd{t}\right)\left(\dd{p} -  \left(\del[2]{q}{\ham}\dd{q} + \del{p}{\del{q}{\ham}}\dd{p}\right)\dd{t}\right) \\
	        &= \dd{q}\dd{p} + \left(-\dd{q}\left(\del[2]{q}{\ham}\dd{q} + \del{p}{\del{q}{\ham}}\dd{p}\right) + \dd{p\left(\del{q}{\del{p}{\ham}}\dd{q} + \del[2]{p}{\ham}\dd{p}\right)}\right)\dd{t} \\
	        &= \dd{q}\dd{p} + \left(-\del[2]{q}{\ham}(\dd{q})^{2} + (\del{q}{\del{p}{\ham}} - \del{p}{\del{q}{\ham}})\dd{q}\dd{p} + \del[2]{p}{\ham}(\dd{p})^{2}\right)\dd{t}.
\end{align*}
The equations of motion imply that the terms containing two consecutive derivatives with respect to the same variable are equal to zero. Assuming the Hamiltonian to be sufficiently smooth, the cross-derivatives are equal. This implies
\begin{align*}
	\dd{V}' = \dd{V}.
\end{align*}

\paragraph{Poisson brackets}
Consider a function $f(q, p, t)$. Its time derivative is given by
\begin{align*}
	\dv{f}{t} &= \del{q_{i}}{f}\dot{q}_{i} + \del{p_{i}}{f}\dot{p}_{i} + \del{t}{f} \\
	          &= \del{q_{i}}{f}\del{p_{i}}{\ham} - \del{p_{i}}{f}\del{q_{i}}{\ham} + \del{t}{f} \\
	          &= \pob{f}{\ham} + \del{t}{f},
\end{align*}
where we now have defined the Poisson bracket. It is bilinear and satisfies
\begin{align*}
	\pob{f}{g}          &= -\pob{g}{f}, \\
	\pob{fg}{h}         &= f\pob{g}{h} + \pob{f}{h}g, \\
	\pob{f}{\pob{g}{h}} &+ \pob{g}{\pob{h}{f}} + \pob{h}{\pob{f}{g}} = 0.
\end{align*}
The expression above implies that if $\pob{f}{\ham} = 0$ and $f$ does not depend explicitly on time, then it is a constant of motion.

\paragraph{Restatement of Liouville's theorem}
We define $\rho(q, p, t)$ as the probability that a particle is close to $(q, p)$. Alternatively, for a large number of particles, we can define it as the number of particles close to $(q, p)$.

We have
\begin{align*}
	\dv{\rho}{t} = 0,
\end{align*}
implying
\begin{align*}
	\del{t}{\rho} = -\pob{\rho}{\ham}.
\end{align*}
This is an equivalent statement of Liouville's theorem.

\paragraph{Canonical transformation}
A canonical transformation is a transformation $(\vb{q}, \vb{p})\to (\vb{Q}, \vb{P})$ such that the Hamiltonian $H$ expressed in these new coordinates also satisfies Hamilton's equations in the new coordinates, i.e.
\begin{align*}
	\dot{Q}_{i} = \del{P_{i}}{H},\ \dot{P}_{i} = -\del{Q_{i}}{\ham}.
\end{align*}

\paragraph{Canonical transformations and Poisson brackets}
It turns out that a transformation in phase space is canonical if and only if they preserve the following equations:
\begin{align*}
	\pob{q_{i}}{q_{j}} = \pob{p_{i}}{p_{j}} = 0,\ \pob{q_{i}}{p_{j}} = \delta_{ij}.
\end{align*}

To show this, we apply the symplectic approach. Consider some point $\vb{x}$ in phase space. Under the canonical transformation, it transforms to $y$. Suppose now that the relation
\begin{align*}
	\dot{x}_{i} = J_{ij}\del{x_{j}}{\ham}
\end{align*}
applies. According to the equations of motion, this would imply
\begin{align*}
	J_{ij} = 
	\begin{cases}
		 1, &j = i + n, i = 1, 2, \dots, n, \\
		-1, &j = i - n, i = n + 1, \dots, 2n.
	\end{cases}
\end{align*}
If the transformation is canonical, then the same should be true after the transformation. On the other hand, the chain rule yields
\begin{align*}
	\dot{y}_{i} = \del{x_{j}}{y_{i}}J_{ik}\del{x_{k}}{y_{m}}\del{y_{m}}{H}.
\end{align*}
Comparing this with the Jacobian $\mathcal{J}$ yields
\begin{align*}
	J = \mathcal{J}J\mathcal{J}^{T}.
\end{align*}

\paragraph{Generators of canonical transformations}
Consider a transformation of the form
\begin{align*}
	q_{i} \to q_{i} + \alpha F_{i},\ p_{i} \to p_{i} + \alpha E_{i}.
\end{align*}
Computing the Poisson brackets of the new coordinates and momenta yields
%Show this with Jacobian instead?
\begin{align*}
	\pob{Q_{i}}{P_{j}} &= \pob{q_{i}}{p_{j}} + \pob{q_{i}}{\alpha E_{j}} + \pob{\alpha F_{i}}{p_{j}} + \pob{\alpha F_{i}}{\alpha E_{j}} \\
	                   &= \delta_{ij} + \alpha(\pob{q_{i}}{E_{j}} + \pob{F_{i}}{p_{j}}) + \alpha^{2}\pob{F_{i}}{E_{j}} \\
	                   &= \delta_{ij} + \alpha(\delta_{ik}\del{p_{k}}{E_{j}} + \delta_{jk}\del{q_{k}}{F_{i}}) + \alpha^{2}\pob{F_{i}}{E_{j}} \\
	                   &= \delta_{ij} + \alpha(\del{p_{i}}{E_{j}} + \del{q_{j}}{F_{i}}) + \dots
\end{align*}
and the requirement
\begin{align*}
	\del{p_{i}}{E_{j}} = -\del{q_{j}}{F_{i}}.
\end{align*}
A simple choice of solution is
\begin{align*}
	E_{j} = -\del{q_{j}}{G},\ F_{i} = \del{p_{i}}{G}
\end{align*}
for some (smooth) function $G$. We dub this a generating function, and say that $G$ generates the transformation.

We now reinsert this into the coordinate transformations. By considering the transformation as a map onto the same phase space, we obtain
\begin{align*}
	\del{\alpha}{q_{i}} = \del{p_{i}}{G},\ \del{\alpha}{p_{i}} = -\del{q_{i}}{G}.
\end{align*}
We notice the strong analogy with Hamilton's equations. This kind of transformations can be thought of as flows in phase space. 

\paragraph{Symmetries and infinitesimal transformations}
Suppose now that we perform an infinitesimal transformation generated by $G$. We then obtain
%TODO: SHOW
\begin{align*}
	\dv{\ham}{\alpha} = \pob{\ham}{G}.
\end{align*}
%TODO: SHOW
We have seen that a symmetry of the Hamiltonian is a transformation such that $\var{\ham} = 0$. Supposing this to be true, we have $\dot{G} = \pob{\ham}{G} = 0$, and $G$ is conserved. In other words, if $G$ generates a symmetry, then it is conserved.

\paragraph{Generators from Hamilton's principle}
We return to the principle of least action. Symmetries of the Lagrangian were on the form $\lag\to\lag + \dv{F}{t}$. We can now see the correspondence between $F$ and the generators $G$ of symmetries, and choose to ignore any differences. The principle of least action can be stated as a variational problem on phase space instead, where we seek the extrema of
\begin{align*}
	S = \integ{}{}{t}{p_{i}\dot{q}_{i} - \ham}.
\end{align*}
Recalling that the equations of motion must be preserved under canonical transformations, the action under a canonical transformation can be written as
\begin{align*}
	\integ{}{}{t}{P_{i}\dot{Q}_{i} - H},
\end{align*}
and this is still extremal under the transformation. One way for this to be true is if
\begin{align*}
	p_{i}\dot{q}_{i} - \ham = P_{i}\dot{Q}_{i} - H + \dv{F}{t}.
\end{align*}

The time derivative of $F$ is
\begin{align*}
	\dv{F}{t} = \del{q_{i}}{F}\dot{q}_{i} + \del{p_{i}}{F}\dot{p}_{i} + \del{Q_{i}}{F}\dot{Q}_{i} + \del{P_{i}}{F}\dot{P}_{i} + \del{t}{F}.
\end{align*}
Inserting this into the requirement above gives
\begin{align*}
	p_{i}\dot{q}_{i} - \ham = P_{i}\dot{Q}_{i} - H + \del{q_{i}}{F}\dot{q}_{i} + \del{p_{i}}{F}\dot{p}_{i} + \del{Q_{i}}{F}\dot{Q}_{i} + \del{P_{i}}{F}\dot{P}_{i} + \del{t}{F}.
\end{align*}
Comparing similar terms yields
\begin{align*}
	p_{i}\dot{q}_{i} = \del{q_{i}}{F}\dot{q}_{i},\ 0 = \del{p_{i}}{F}\dot{p}_{i},\ 0 = P_{i}\dot{Q}_{i} + \del{Q_{i}}{F}\dot{Q}_{i},\ 0 = \del{P_{i}}{F}\dot{P}_{i},\ -\ham = -H - \del{t}{F}, 
\end{align*}
and finally
\begin{align*}
	\del{q_{i}}{F} = p_{i},\ \del{p_{i}}{F} = 0,\ \del{Q_{i}}{F} = -P_{i},\ \del{P_{i}}{F} = 0,\ \ham = H - \del{t}{F}.
\end{align*}
The solution to this is $F = F_{1}(q, Q, t)$.

Another choice is a function $F = F_{2} - P_{i}Q_{i}$. Inserting this into the above criterion yields
\begin{align*}
	\del{q_{i}}{F_{2}} = p_{i},\ \del{p_{i}}{F_{2}} = 0,\ \del{Q_{i}}{F_{2}} = 0,\ \del{P_{i}}{F} = Q_{i},\ \ham = H - \del{t}{F_{2}}.
\end{align*}
The solution to this is $F_{2} = F_{2}(q, P, t)$.

A third choice is a function $F = F_{3} + p_{i}q_{i}$. Inserting this into the above criterion yields
\begin{align*}
	\del{q_{i}}{F_{3}} = 0,\ \del{p_{i}}{F} = -q_{i},\ \del{Q_{i}}{F} = -P_{i},\ \del{P_{i}}{F} = 0,\ \ham = H - \del{t}{F_{3}}.
\end{align*}
The solution to this is $F_{3} = F_{3}(p, Q, t)$.

A fourth choice is a function $F = F_{4} + p_{i}q_{i} - P_{i}Q_{i}$. Inserting this into the above criterion yields
\begin{align*}
	\del{q_{i}}{F_{4}} = 0,\ \del{p_{i}}{F_{4}} = -q_{i},\ \del{Q_{i}}{F_{4}} = 0,\ \del{P_{i}}{F_{4}} = Q_{i},\ \ham = H - \del{t}{F_{4}}.
\end{align*}

Going from one type of generator to another looks very similar to a Legendre transform, and computationally is a (somewhat) clear demonstration of what the Legendre transform does. However, the one thing separating it from a Legendre transform is the fact that performing this transformation is not always possible. For instance, it might not be possible to find a generator of a certain kind, in which case performing a transformation to or from that kind is meaningless.

\example{A failed generator transform}
Suppose that we want to perform a canonical transform that preserves the first coordinate. Looking for a generator of the first kind, we find that it must satisfy
\begin{align*}
	\del{q_{1}}{F} = p_{1},\ \del{Q_{1}}{F} = -P_{1}.
\end{align*}
However, as the two coordinates are equal, the partial derivatives represent equivalent operations. The only way to resolve this is for the transformation to satisfy $p_{1} = -P_{1}$ - otherwise you cannot find a generator of the first kind.

\paragraph{Hamilton-Jacobi theory}
Suppose that we could perform a transformation on the form $F = F_{2}(q, P, t) - P_{i}Q_{i}$ such that $H = 0$. This would imply that all $Q_{i}$ and $P_{i}$ were to be constant. The equation
\begin{align*}
	\ham + \del{t}{F_{2}} = 0
\end{align*}
would thus define a differential equation in $F_{2}$. We can now, with some reasoning behind it, I guess, call $F_{2}$ the action $S$. This yields the Hamilton-Jacobi equation
\begin{align*}
	H + \del{t}{S} = 0,
\end{align*}
where the Hamiltonian, according to previous arguments, would depend on the time derivative of $S$.

%Prove S action

\paragraph{The Schrödinger equation from Hamilton-Jacobi theory}

\paragraph{Quantum mechanics and the action}

\paragraph{Integrable systems}
Consider a system with some Hamiltonian. This system is integrable if there is a canonical transformation $(q, p)\to (\theta, I)$ suc that the transformed Hamiltonian only depends on the momenta. For such a system, the equations of motion become
\begin{align*}
	\dot{\theta}_{i} = \del{I_{i}}{H} = \omega_{i},\ \dot{I}_{i} = 0.
\end{align*}

\paragraph{Structures of theory}
A theory in physics contain
\begin{itemize}
	\item some notion of states.
	\item observables.
	\item a description of the dynamics of the system.
	\item predictions of experiments.
\end{itemize}

\example{Hamiltonian mechanics}