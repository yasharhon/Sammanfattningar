\section{Classical mechanics}
In classical mechanics, configuration space is the space of all possible configurations of a system. We can impose coordinates $\chi^{a}$ on this space in order to use what we know.

\paragraph{Kinetic energy}
Kinetic energy is defined by a rank $2$ tensor as
\begin{align*}
	E_{\text{k}} = \frac{1}{2}T_{ab}\dot{\chi}^{a}\dot{\chi}^{b},
\end{align*}
where the dot now really represents the time derivative.

\paragraph{Hamilton's principle}
We define the Lagrangian of a system as $\mathcal{L} = E_{\text{k}} - V$, where $V$ is the potential energy and taken to be a function on coordinate space. The action of a system over time is defined as
\begin{align*}
	S = \integ{}{}{t}{\mathcal{L}}.
\end{align*}
Hamilton's principle states that for the motion of the system in configuration space, $\var{S} = 0$. This can be expressed as
\begin{align*}
	\var{S} = \integ{}{}{t}{\var{\mathcal{L}}} = \integ{}{}{t}{\left(\del{\chi^{a}}{\mathcal{L}} - \dv{t}\del{\dot{\chi}^{a}}{\mathcal{L}}\right)\var{\chi^{a}}} = 0.
\end{align*}

\paragraph{The kinetic metric}
Consider a system with no potential energy. The Lagrangian simply becomes $\mathcal{L} = \frac{1}{2}T_{ab}\dot{\chi}^{a}\dot{\chi}^{b}$. This is very similar to the integral of curve length (or, rather its square, the extremum of which was noted to be the same), except $g_{ab}$ has been replaced by $T_{ab}$. This inspires us to define $T_{ab}$ as the kinetic metric, with corresponding Christoffel symbols.

\paragraph{Motion of a classical system}
By defining $a^{b} = \dot{\chi}^{a}\dcov{a}{\dot{\chi}^{b}}$, the previous work leads us to a system with no potential satisfying $a^{b} = \ddot{\chi}^{b} + \chris{b}{a}{c}\dot{\chi}^{a}\dot{\chi}^{c} = 0$. In other words, a system with no potential moves along the geodesics of the kinetic metric.

For a system with a potential, only the $\del{\chi^{a}}{\mathcal{L}}$ term is affected, and
\begin{align*}
	a^{b} = - T^{ba}\del{a}{V} = T^{ba}F,
\end{align*}
which is a generalization of Newton's second law.