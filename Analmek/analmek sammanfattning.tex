\documentclass[a4paper, 11pt]{article}
\usepackage[T1]{fontenc}
\usepackage[swedish]{babel}
\usepackage{amssymb}
\usepackage{hyperref}
\usepackage[margin=0.5in]{geometry}

\usepackage[arrowdel]{physics}
\usepackage{tensor}
\usepackage{pgffor}

% Structure and symbols
\newcommand{\example}[1]{\subparagraph{Example: #1}}
\newcommand{\nosum}{\ensuremath\ \text{(no sum)}}
\newcommand{\R}{\mathbb{R}}
\renewcommand{\O}[2]{\ensuremath{\text{O}(#1, #2)}}
\newcommand{\SU}[1]{\ensuremath{\text{SU}(#1)}}
\newcommand{\SO}[2]{\ensuremath{\text{SO}(#1, #2)}}
\newcommand{\GL}[2]{\ensuremath{\text{GL}(#1, #2)}}

% Math
\newcommand{\del}[3][]{\partial^{#1}_{#2}#3}
\newcommand{\deval}[4][]{\eval{\dv[#1]{#2}{#3}}_{#4}}
\newcommand{\integ}[5][]{\int\limits_{#2}^{#3}\dd[#1]{#4}#5}
\newcommand{\vinteg}[4]{\int\limits_{#1}^{#2}\dd{\vb{#3}}\cdot #4}
\newcommand{\lieb}[2]{\left[#1, #2\right]}

% Differential geometry and tensors
\newcommand{\dcov}[2]{\grad_{#1}{#2}}
\newcommand{\db}[1]{\vb{E}^{#1}}
\newcommand{\tb}[1]{\vb{E}_{#1}}
\newcommand{\kdelta}[2]{\delta_{#1}^{#2}}
\newcommand{\leci}[1]{\varepsilon_{#1}}
\newcommand{\tenprod}[1]
{
	\foreach \variable [count = \count] in #1
	{
		\ifnum\count > 1
			\otimes
		\fi
		\variable
	}
}
\newcommand{\chris}[3]{\Gamma^{#1}_{\;#2#3}}

% Analytical mechanics
\newcommand{\lag}{\ensuremath{\mathcal{L}}}
\newcommand{\ham}{\ensuremath{\mathcal{H}}}
\newcommand{\J}{\mathcal{J}}
\newcommand{\pob}[3][]{\left\{#2, #3\right\}_{#1}}

% Field theory
\newcommand{\bvar}[1]{\bar{\delta}#1}

\title{Summary of SI2360 Analytical Mechanics and Classical Field Theory}
\author{Yashar Honarmandi \\ yasharh@kth.se}
\date{\today}

\begin{document}

\maketitle

\begin{abstract}
	Detta ær en sammanfattning av kursen SH1014 Modern fysik.
\end{abstract}

\pagenumbering{roman}
\thispagestyle{empty}

\newpage

\tableofcontents

\newpage

\pagenumbering{arabic}

\section{Variationsanalys}

\paragraph{Funktionaler}
Variationsanalys handlar om funktionaler. Detta är avbildningar från funktioner till skalärer.

\paragraph{Extrempunkter och variationer}
Att analytiskt hitta en funktion som är en extrempunkt för en given funktional är allmänt inte enkelt. Den typiska strategin är att i stället anta att man har hittat en funktion som minimerar funktionalen, och introducera en variationsfunktion och en parameter $\varepsilon$ multiplicerad med den. Då vet man enligt antagandet att $\varepsilon = 0$ motsvarar en extrempunkt.

\paragraph{Tillväxt av funktionalen}
För att få en ide om hur en funktional beter sig beroende på $\varepsilon$, antag att $y$ minimerar funktionalen $J$, diskretisera $y$ i $N$ punkter och introducera variationen $\eta$. Detta ger
\begin{align*}
	J(y + \varepsilon\eta) - J(y) = \sum\limits_{i = 1}^{N}\pdv{J}{y_{i}} + O(\varepsilon^{2}).
\end{align*}
De partiella derivatorna

\paragraph{Variationsproblem typ 1}
Vi har en funktional $J$ som är extremal och funktionen $y$ är fixerad i randpunkterna.

\paragraph{Variationsproblem typ 2}
Vi har en funktional som är extremal och funktionen $y$ är fix i en punkt. Andra villkoret kommer från funktionalen.

\paragraph{Variationsproblem på integralform}
För att illustrera hur man gör i variationsanalys, försöker vi hitta en funktion som är en extrempunkt till funktionalen
\begin{align*}
	J(u) = \inteval{x_{0}}{x_{1}}{x}{F(u, u', x)}.
\end{align*}
För att göra detta, antag att det finns ett minimum $y$, och introducera dens variation. Låt nu $f(\varepsilon) = J(y + \varepsilon\eta)$. Då gäller dett att $\dv{f}{\varepsilon} (0) = 0$. Denna derivatan ges av
\begin{align*}
	\dv{f}{\varepsilon} (\varepsilon) =& \dv{\varepsilon}\inteval{x_{0}}{x_{1}}{x}{F(y + \varepsilon\eta, y' + \varepsilon\eta', x)} \\
	                                  =& \inteval{x_{0}}{x_{1}}{x}{\eta\del{u}{F}(y + \varepsilon\eta, y' + \varepsilon\eta', x) + \eta'\del{u'}{F}(y + \varepsilon\eta, y' + \varepsilon\eta', x)} \\
	                                  =& \inteval{x_{0}}{x_{1}}{x}{\eta\del{u}{F}(y + \varepsilon\eta, y' + \varepsilon\eta', x)} + \eta(x_{1})\del{u'}{F(y + \varepsilon\eta, y' + \varepsilon\eta', x_{1})} \\
	                                   &- \eta(x_{0})\del{u'}{F(y + \varepsilon\eta, y' + \varepsilon\eta', x_{0})} - \inteval{x_{0}}{x_{1}}{x}{\eta\dv{x}\del{u'}{F}(y + \varepsilon\eta, y' + \varepsilon\eta', x)} \\
	                                  =& \inteval{x_{0}}{x_{1}}{x}{\eta\left(\del{u}{F}(y + \varepsilon\eta, y' + \varepsilon\eta', x) - \dv{x}\del{u'}{F}(y + \varepsilon\eta, y' + \varepsilon\eta', x)\right)} \\
	                                   &+ \del{u'}{\eta(x_{1})F(y + \varepsilon\eta, y' + \varepsilon\eta', x_{1})} - \eta(x_{0})\del{u'}{F(y + \varepsilon\eta, y' + \varepsilon\eta', x_{0})}.
\end{align*}

Om detta skall vara lika med $0$, oberoende av $\eta$, kräver vi
\begin{align*}
	\del{u}{F} - \dv{x}\del{u'}{F} = 0.
\end{align*}
Om vi har bra randvillkor, kan de två återstående termerna försvinna. Annars måste dessa också vara lika med $0$.

Beräkningen är analog om man har en vektorvärd funktion man minimerar funktionalen med avseende på - det kommer bara dyka upp flera termer. Den är även analog om man minimerar med avseende på en funktion av flera variabler. Då dyker det upp bidrag från de olika partialderivatorna. Om det finns beroende av högre ordningens derivator, får man göra flera partiella integrationer.

\paragraph{Variationsproblem med funktionaler som bivillkor}
Betrakta problemet att hitta en funktion som är en extrempunkt till funktionalen
\begin{align*}
	J(u) = \inteval{x_{0}}{x_{1}}{x}{F(u, u', x)}
\end{align*}
med bivillkoret
\begin{align*}
	K(u) = \inteval{x_{0}}{x_{1}}{x}{G(u, u', x)} = K_{0}.
\end{align*}
Man kan visa att detta är ekvivalent med att hitta en extremalpunkt till $J - \lambda K$ som uppfyller bivillkoret.

\section{Group Theory}

\paragraph{Definition of a Group}
A group is a set of objects $G$ with an operation $G\times G\to G,\ (a, b)\to ab$ such that
\begin{itemize}
	\item If $a, b\in G$ then $ab\in G$.
	\item $a(bc) = (ab)c$ for all $a, b, c\in G$.
	\item There exists an identity $e$ such that $ae = ea = a$ for all $a\in G$.
	\item There exists for every element $a$ an inverse $a^{-1}\in G$ such that $aa^{-1} = a^{-1}a = e$.
\end{itemize}
Groups can be
\begin{itemize}
	\item cyclic, i.e. all elements in the group are powers of a single element.
	\item finitie, i.e. groups containing a finite number of elements, or infinte.
	\item discrete, i.e. all elements in the group can be labelled with some index, por continuous.
	\item commutative, i.e. $ab = ba$ for all elements in the group, or non-commutative.
\end{itemize}

\paragraph{Subgroups}
If $G = \{g_{\alpha}\}$ and the subset $H = \{h_{\alpha}\}$ is also a group, we call $H$ a subgroup of $G$ and write $H < G$.

\paragraph{Conjugacy Classes}
Two group elements $a$ and $b$ are conjugate if there exists an element $g$ such that
\begin{align*}
a = gbg^{-1}.
\end{align*}
We write $a\sim b$.

\paragraph{Equivalence Relations}
An equivalence relation is a relation (here denoted $=$) between two things such that
\begin{itemize}
	\item $a = b \equiv b = a$.
	\item $a = b,\ b = c\implies a = c$.
\end{itemize}

\example{Conjugacy as an Equivalence Relation}

\paragraph{Homomorphisms and Isomorphisms}
A homomorphisms is a map $f: G\to H$ such that $f(g_{1})f(g_{2}) = f(g_{1}g_{2})$. If the map is bijective, $f$ is called an isomorphism.

\paragraph{Direct Products}
GIven two groups $F$ and $G$, we define $F\times G$ as the set of ordered pairs of elements of the two groups. The group action of $F\times G$ is the group actions of $F$ and $G$ separately on the elements in the ordered pair.

\paragraph{Generators}
For discrete groups, the generators of a group is the smallest set of elements in the group such that all other elements in the group can be composed by the elements in the set. For continuous groups, we will use the term generators to refer to elements such that any group element can be written as real powers of this element.

\paragraph{Point Groups}
Point groups are symmetries of, for instance, a crystal structure that leave at least one point in the structure invariant. Examples include
\begin{itemize}
	\item rotations.
	\item reflections.
	\item spatial inversions.
\end{itemize}
Combining these with certain discrete translation, you obtain the space groups of the crystal. Space groups are the groups of all symmetries of a crystal.

\paragraph{Dihedral Groups}
The dihedral group $D_{n}$ is the group of transformations that leave an $n$-sided polygon invariant.

\paragraph{Lie Groups}
A Lie group is a group containing a manifold with the group operation and inverse operation being smooth maps. My current understanding of the significance of this is that it allows us to differentiate and expand the group elements with respect to certain parameters.

The group elements are denoted $g(\vb*{\theta})$, where $g(\vb{0}) = 1$. We write them as
\begin{align*}
	g(\vb*{\theta}) = e^{\theta_{a}T_{a}}
\end{align*}
where we have introduced the generators $T_{a}$. This is reasonable, partially because it is reasonable for a smooth operation to have an addition operation (I think), and in this case the series definition of the exponential function yields exactly that any element is generated by powers of the generators. Note however that the generators themselves are typically not group elements, which might leave a hole in this reasoning. In addition, the question of whether the group is Abelian leaves it questionable whether the addition of generators makes sense, but I will return to this concern. There is probably a deeper understanding to this, and what I am saying may even be completely incorrect. Depending on the context, the exponent may also contain a factor $-i$, but this discussion will omit it.

\paragraph{The Lie Algebra}
Expanding an element around the identity yields
\begin{align*}
	g(\vb*{\theta}) \approx 1 + \theta_{a}T_{a}.
\end{align*}
First of all, we note that performing this for exponentials of a single generator yields
\begin{align*}
	e^{\theta_{i}T_{i}}e^{\theta_{j}T_{j}} \approx (1 + \theta_{i}T_{i})(1 + \theta_{j}T_{j}),
\end{align*}
which is equal to $e^{\theta_{i}T_{i} + \theta_{j}T_{j}}$ to first order even for a non-commutative group. Hence the addition of generators is reasonable. Furthermore, this implies that set of generators is a vector space, termed the Lie algebra.

\paragraph{The Lie Bracket}
Having seen that the exponential notation makes sense, we study it for a non-commutative group. The element $e^{-\theta_{i}T_{i}}e^{-\theta_{j}T_{j}}e^{\theta_{i}T_{i}}e^{\theta_{j}T_{j}}$ is equal to the identity for a commutative group, and we would like to study this in the general case. We have
\begin{align*}
	e^{\theta_{i}T_{i}}                    &\approx 1 + \theta_{i}T_{i} + \frac{1}{2}\theta_{i}^{2}T_{i}^{2}, \\
	e^{\theta_{i}T_{i}}e^{\theta_{j}T_{j}} &\approx \left(1 + \theta_{i}T_{i} + \frac{1}{2}\theta_{i}^{2}T_{i}^{2}\right)\left(1 + \theta_{j}T_{j} + \frac{1}{2}\theta_{j}^{2}T_{j}^{2}\right) \approx 1 + \theta_{i}T_{i} + \theta_{j}T_{j} + \frac{1}{2}\left(\theta_{i}^{2}T_{i}^{2} + \theta_{j}^{2}T_{j}^{2}\right) + \theta_{i}\theta_{j}T_{i}T_{j}.
\end{align*}
We thus obtain
\begin{align*}
	e^{-\theta_{i}T_{i}}e^{-\theta_{j}T_{j}}e^{\theta_{i}T_{i}}e^{\theta_{j}T_{j}} \approx& 1 + \theta_{i}T_{i} + \theta_{j}T_{j} + \frac{1}{2}\left(\theta_{i}^{2}T_{i}^{2} + \theta_{j}^{2}T_{j}^{2}\right) + \theta_{i}\theta_{j}T_{i}T_{j} - \theta_{i}T_{i}\left(1 + \theta_{i}T_{i} + \theta_{j}T_{j}\right) \\
	 &- \theta_{j}T_{j}\left(1 + \theta_{i}T_{i} + \theta_{j}T_{j}\right) + \frac{1}{2}\left(\theta_{i}^{2}T_{i}^{2} + \theta_{j}^{2}T_{j}^{2}\right) + \theta_{i}\theta_{j}T_{i}T_{j} \\
	=& 1 + \theta_{i}\theta_{j}T_{i}T_{j} - \theta_{i}\theta_{j}T_{i}T_{j} - \theta_{i}\theta_{j}T_{j}T_{i} + \theta_{i}\theta_{j}T_{i}T_{j} \\
	=& 1 + \theta_{i}\theta_{j}\comm{T_{a}}{T_{b}}
\end{align*}
to second order, where we have introduced the Lie bracket
\begin{align*}
	\comm{T_{i}}{T_{j}} = T_{i}T_{j} - T_{j}T_{i}.
\end{align*}
Hence, the non-commutativity of Lie groups close to the identity and the structure of the Lie algebra is described by the Lie brackets, which is why we study them. Furthermore, we have
\begin{align*}
	e^{-\theta_{i}T_{i}}e^{-\theta_{j}T_{j}}e^{\theta_{i}T_{i}}e^{\theta_{j}T_{j}} \approx e^{\theta_{i}\theta_{j}\comm{T_{a}}{T_{b}}},
\end{align*}
implying that the Lie bracket belongs to the vector space spanned by the generators and allowing us to write
\begin{align*}
	\comm{T_{a}}{T_{b}} = f_{a, b, c}T_{c}.
\end{align*}
The constants $f_{a, b, c}$ are called structure constants.

\example{Rotations in Two Dimensions}
Consider a rotation of an infinitesimal displacement $\dd{\vb{x}}$ with a rotation $R$. The requirement for length to be preserved implies $R^{T}R = 1$.

Consider now a rotation by a small angle $\var{\theta}$. Taylor expanding it in terms of the angle yields
\begin{align*}
	R(\var{\theta}) \approx 1 + A\var{\theta}.
\end{align*}
The requirement for $R$ to be orthogonal yields $A^{T} = -A$. We choose the solution
\begin{align*}
	J =
	\mqty[
		0  & 1 \\
		-1 & 0
	].
\end{align*}
We can now write the rotation matrix as
\begin{align*}
	R(\var{\theta}) =
	\mqty[
		1             & \var{\theta} \\
		-\var{\theta} & 1
	].
\end{align*}

We would now like to construct a large rotation in terms of smaller rotations as
\begin{align*}
	R(\theta) = \lim\limits_{N\to\infty}\left(1 + \frac{\theta}{N}J\right)^{N} = e^{\theta J}.
\end{align*}
We can write this as an infinite series and use the fact that $J^{2} = -1$ to obtain
\begin{align*}
	R(\theta) = \cos{\theta} + J\sin{\theta}.
\end{align*}

\example{Rotations in Three Dimensions}
The argument done for two dimensions does not use the dimensionality, so we conclude that even for higher dimensions, $R^{T}R = 1$. Expanding a small rotation around the identity yields that the first-order term must include an antisymmetric matrix. The space of antisymmetric $3\times 3$ matrices is three-dimensional. We thus choose the basis
\begin{align*}
	J_{x} =
	\mqty[
		0 & 0  & 0 \\
		0 & 0  & 1 \\
		0 & -1 & 0
	],
	J_{y} =
	\mqty[
		0 & 0 & -1 \\
		0 & 0 & 0  \\
		1 & 0 & 0
	],
	J_{z} =
	\mqty[
		0  & 1 & 0 \\
		-1 & 0 & 0 \\
		0  & 0 & 0
	].
\end{align*}
Exponentiating yields
\begin{align*}
	R(\theta) = e^{\sum \theta_{i}J_{i}} = e^{\vb{\theta}\cdot\vb{J}}.
\end{align*}
In physics we usually extract a factor $i$ such that the basis matrices are Hermitian, and the rotation becomes
\begin{align*}
	R(\theta) = e^{i\vb{\theta}\cdot\vb{J}}.
\end{align*}

The set of generators of these rotations constitutes the Lie algebra.

We know in general that rotations in three dimensions do not commute. In fact, we obtain in general that
\begin{align*}
	R(\vb{\theta})R(\vb{\theta}')R^{-1}(\vb{\theta}) = \theta_{a}\theta_{b}'\comm{J_{a}}{J_{b}},
\end{align*}
where $\comm{J_{a}}{J_{b}}$ is the commutator. This commutator satisfies
\begin{align*}
	\comm{J_{a}}{J_{b}}^{T} = \comm{J_{b}^{T}}{J_{a}^{T}} = \comm{-J_{b}}{-J_{a}} = -\comm{J_{a}}{J_{b}},
\end{align*}
which implies
\begin{align*}
	\comm{J_{a}}{J_{b}} = f_{a,b,c}J_{c}.
\end{align*}
It can be shown that
\begin{align*}
	\comm{J_{i}}{J_{j}} = \varepsilon_{i,j,k}J_{k},
\end{align*}
or in a physics context (where a factor $i$ is extracted):
\begin{align*}
	\comm{J_{i}}{J_{j}} = i\varepsilon_{i,j,k}J_{k}.
\end{align*}

\paragraph{Representations}
A representation of a group on a vector space is a homomorphism $D: G\to\GL{V}$, where $\GL{V}$ is the group of all invertible linear transformations on $V$. The dimension of the representation is defined as the dimension of $V$.

\paragraph{Representations on Direct Product Spaces}
Suppose that there exists representations $\rho_{V}$ and $\rho_{W}$ of $G$ on $V$ and $W$ respectively. The representation $\rho_{V\oplus W}$ of $G$ on $V\oplus W$ is then defined by
\begin{align*}
	\rho_{V\oplus W}(a)(\vb{v}, \vb{w}) = (\rho_{V}(a)(\vb{v}), \rho_{W}(a)(\vb{w})).
\end{align*}

\paragraph{Representations on Tensor Product Spaces}
Suppose that there exists representations $\rho_{V}$ and $\rho_{W}$ of $G$ on $V$ and $W$ respectively. The representation $\rho_{V\otimes W}$ of $G$ on $V\otimes W$ is then defined by
\begin{align*}
	\rho_{V\otimes W}(a)(\vb{v}\otimes\vb{w}) = (\rho_{V}(a)(\vb{v}))\otimes(\rho_{W}(a)(\vb{w})).
\end{align*}

\paragraph{The Unitarity Theorem}
Finite groups have unitary representations - with some inner product, at least. Such representations will be the primary concern of the following discussion.

\paragraph{Reducible Representations}
Two representations are equivalent if they satisfy $U\rho(a)U^{-1} = \rho^{\prime}(a)$ for any group element $a$ and some linear operator $U$. $\rho$ is reducible if it is equivalent to a representation of the form
\begin{align*}
	\mqty[
		\rho_{1} & \sigma \\
		0        & \rho_{2}
	],
\end{align*}
where $\rho_{1}$ and $\rho_{2}$ are representations of dimension $n_{1}$ and $n_{2}$ respectively and $\sigma$ is an $n_{1}\times n_{2}$ block satisfying $\sigma(ab) = \rho_{1}(a)\sigma(b) + \sigma(a)\rho_{1}(b)$. A basis can often be chosen such that $\sigma = 0$.

\example{$\SO{3}$}
As we know, \SO{3} may be represented by rotations in three dimensions. In particular, we may choose a basis for $\R^{3}$ such that
\begin{align*}
	\rho(R_{\vb{e}_{z}}^{3}) = 
	\mqty[
		\cos(\theta) & -\sin(\theta) & 0 \\
		\sin(\theta) & \cos(\theta)  & 0 \\
		0            & 0             & 1
	].
\end{align*}
The upper left block is a representation of \SO{2}, and the lower right block is the trivial representation $\rho(a) = 1$. Hence $\rho$ is reducible.

Note that the upper right block is zero. For finite-dimensional groups, one can apparently always choose a basis such that this is possible. In this case we may write $\rho(R_{\vb{e}_{z}}^{3}) = \rho_{2}\oplus 1$, where $\rho_{2}$ is the representation of \SO{2}.

\example{Tensor Product Representations}
Given some representation $\rho$ on $V$, we consider the representation $\rho_{V\otimes V} = \rho\otimes\rho$. Next, we split a tensor $T$ into its symmetric components $T_{\{ij\}}$ and antisymmetric components $T_[ij]$ such that $T_{ij} = T_{\{ij\}} + T_[ij]$. Next, we have for a symmetric tensor $S$
\begin{align*}
	(\rho\otimes\rho S)_{ij} = \rho_{ik}\rho_{jl}S_{kl} = \rho_{jl}\rho_{ik}S_{lk} = (\rho\otimes\rho S)_{ji},
\end{align*}
i.e. the representation preserves the symmetry of $S$. The same holds for antisymmetric tensors. Denoting the symmetric and antisymmetric subspaces as $(V\otimes V)_{\pm}$ we have $(V\otimes V)_{+}\oplus(V\otimes V)_{-}$, we thus have
\begin{align*}
	\rho\otimes\rho =
	\mqty[
		\rho_{(V\otimes V)_{+}} & 0 \\
		0                       & \rho_{(V\otimes V)_{-}}
	],
\end{align*}
and $\rho\otimes\rho$ has been reduced to $\rho_{(V\otimes V)_{+}}\oplus\rho_{(V\otimes V)_{-}}$, which is the direct sum of its symmetric and antisymmetric representation.

\paragraph{Irreducible Representations}
If a representation cannot be reduced, it is irreducible. Such representations are often called irreps.

By their very nature, all representations leave at least some part of the vector space invariant - all of it, in the worst case. Suppose now that there is a part of the vector space that is left invariant by the representation and another that is not. In this case, the representation would necessarily take the above block diagonal form, meaning that it can be reduced. The consequence of this is that for any non-zero $v\in V$, the fact that $\rho$ is an irrep implies that $V$ is spanned by elements of the form $\rho(a)v$.

\paragraph{Schur's Lemma}
Schur's lemma has two statements:
\begin{itemize}
	\item Given an irrep $\rho$ on $V$, any operator $A$ that commutes with $\rho$ is a multiple of the identity on $V$.
	\item Given two inequivalent irreps $\rho_{1}$ and $\rho_{2}$ on vector spaces $V_{1}$ and $V_{2}$, the only linear transformation $B$ such that $B\rho_{1} = \rho_{2}B$ is the zero transformations.
\end{itemize}

To prove the first, consider an eigenvector $v$ of $A$ with eigenvalue $\lambda$, the existence of which is guaranteed by the fundamental theorem of algebra. Assuming $A$ to commute with $\rho$, we have
\begin{align*}
	A\rho(a)v = \rho(a)Av = \lambda\rho(a)v,
\end{align*}
hence $\rho$ leaves eigenspaces of $A$ invariant. As $V$ is spanned by elements of the form $\rho(a)v$, we conclude that $A = \lambda$ on all of $V$.

To prove the second, assume the two representations to satisfy the above and let the $V_{i}$ having dimensionalities $n_{i}$. The first case is $n_{2} < n_{1}$. The kernel of $B$ is then non-trivial, and for any vector $v$ in it we find
\begin{align*}
	B\rho_{1}v = \rho_{2}Bv = 0,
\end{align*}
hence $\rho_{1}$ leaves the kernel of $B$ invariant. This implies, however, that all of $V_{1}$ must be the kernel of $B$ and $B = 0$ in this case. The second case is $n_{2} \geq n_{1}$, for which we study vectors $w = Bv$. We find
\begin{align*}
	\rho_{2}w = \rho_{2}Bv = B\rho_{1}v,
\end{align*}
hence $\rho_{2}$ preserves the image of $B$. The possible consequences of this are either that $B = 0$ or that the image of $B$ is all of $V_{2}$. The latter is impossible if $n_{2} > n_{1}$, meaning that $B = 0$ in this case. Finally, if $n_{1} = n_{2}$ and the image of $B$ is all of $V_{2}$, that would imply that $B$ is bijective, This would, however, imply that $\rho_{1}$ and $\rho_{2}$ are equivalent, which was assumed not to be the case, hence $B = 0$ for this case too.

\paragraph{The Orthogonality Theorem}
For a group $G$ we can number the possible inequivalent irreps $\rho_{\mu}$, as well as the vector spaces and corresponding dimensionalities $V_{\mu}$ and $n_{\mu}$. Given an operator $T: V_{\nu}\to V_{\mu}$ we may then define the operator
\begin{align*}
	B = \sum\limits_{a}\rho_{\mu}(a)T\rho_{\nu}(a^{-1}).
\end{align*}
We then find
\begin{align*}
	\rho_{\mu}(b)B &= \rho_{\mu}(b)\sum\limits_{a}\rho_{\mu}(a)T\rho_{\nu}(a^{-1}) \\
	               &= \sum\limits_{a}\rho_{\mu}(ba)T\rho_{\nu}(a^{-1}) \\
	               &= \sum\limits_{c}\rho_{\mu}(c)T\rho_{\nu}(cb^{-1}) \\
	               &= \sum\limits_{c}\rho_{\mu}(c)T\rho_{\nu}(c)\rho_{\nu}(b^{-1}) \\
	               &= B\rho_{\nu}(b^{-1}).
\end{align*}
Schur's lemma implies that $B = \lambda_{\mu, T}$ (where we have labelled the eigenvalue with the necessary information) if $\mu = \nu$ and $B = 0$ otherwise. We may summarize this as
\begin{align*}
	B = \lambda_{\mu, T}\delta_{\mu\nu},
\end{align*}
where we forego Einstein notation for the rest of this group theory discussion.

In particular, we consider the operator such that $T_{ij} = \delta_{ip}\delta_{jq}$ for some particular $p, q$. Relabelling the eigenvalue as $\lambda_{\mu, pq}$ we find
\begin{align*}
	\sum\limits_{k, l}\sum\limits_{a}\rho_{\mu, ik}(a)T_{kl}\rho_{\nu, lj}(a^{-1}) = \sum\limits_{k, l}\sum\limits_{a}\rho_{\mu, ik}(a)\delta_{kp}\delta_{lq}\rho_{\nu, lj}(a^{-1}) = \sum\limits_{a}\rho_{\mu, ip}(a)\rho_{\nu, qj}(a^{-1}) = \lambda_{pq, \mu}\delta_{\mu\nu}\delta_{ij}.
\end{align*}
We will try to simplify this by computing the eigenvalue. In order for the above to be non-trivial, we must set $\mu = \nu$. Computing the trace of either side we find
\begin{align*}
	\lambda_{pq, \mu}n_{\mu} = \sum\limits_{i}\sum\limits_{a}\rho_{\mu, ip}(a)\rho_{\mu, qi}(a^{-1}) = \sum\limits_{a}(\rho_{\mu}(a^{-1})\rho_{\mu}(a))_{qp} = \sum\limits_{a}(\rho_{\mu}(e))_{qp} = \sum\limits_{a}\delta_{qp} = \delta_{qp}N_{G},
\end{align*}
where $N_{G}$ is the order of the group. Thus we have the orthogonality relation
\begin{align*}
	\sum\limits_{a}\rho_{\mu, ip}(a)\rho_{\nu, qj}(a^{-1}) = \frac{N_{G}}{n_{\mu}}\delta_{qp}\delta_{\mu\nu}\delta_{ij}.
\end{align*}

\paragraph{The Character}
We will need to classify representations in terms of something that is basis-independent. One such property is the trace, as if $\rho_{1}$ and $\rho_{2}$ are equivalent we have
\begin{align*}
	\tr(\rho_{2}) = \tr(U\rho_{2}U^{-1}) = \tr(U^{-1}U\rho_{2}) = \tr(\rho_{2}).
\end{align*}
Hence we define the character
\begin{align*}
	\chi_{\rho}(a) = \tr(\rho(a)),
\end{align*}
which is the same for all equivalent representations.

\paragraph{Properties of the Character}
We have
\begin{align*}
	\chi_{\rho}(e) = n_{\rho},
\end{align*}
where $n_{\rho}$ is the dimension of the representation. Next, if $a$ and $b$ belong to the same conjugacy class we have
\begin{align*}
	\chi_{\rho}(a) = \chi_{\rho}(gbg^{-1}) = \tr(\rho(gbg^{-1})) = \tr(\rho(g)\rho(b)\rho(g^{-1})) = \chi_{\rho}(b).
\end{align*}
Finally, if
\begin{align*}
	\rho = \bigoplus\limits_{\mu}\rho_{\mu}
\end{align*}
we have
\begin{align*}
	\chi_{\rho} = \sum\limits_{\mu}\chi_{\rho_{\mu}}.
\end{align*}

\paragraph{The Character inner Product}
We define the character inner product
\begin{align*}
	\braket{\chi_{\mu}}{\chi_{\nu}} = \frac{1}{N_{G}}\sum\limits_{a}\chi_{\mu}(a)\chi_{\nu}(a^{-1}) = \frac{1}{N_{G}}\sum\limits_{a}\chi_{\mu}(a)\cc{\chi_{\nu}(a)},
\end{align*}
where the last step follows from assuming unitarity. it may also be computed as
\begin{align*}
	\braket{\chi_{\mu}}{\chi_{\nu}} = \frac{1}{N_{G}}\sum\limits_{C_{i}}k_{i}\chi_{\mu}(C_{i})\cc{\chi_{\nu}(C_{i})},
\end{align*}
where the $C_{i}$ are the different conjugacy classes and $k_{i}$ their order.

\paragraph{Orthogonality of Characters}
Starting from the relation we may consider element $pq$ and sum over $p$ and $q$ to find
\begin{align*}
	\sum\limits_{p, q}\sum\limits_{a}\rho_{\mu, pp}(a)\rho_{\nu, qq}(a^{-1}) = \sum\limits_{a}\chi_{\mu}(a)\chi_{\nu}(a^{-1}) = N_{G}\delta_{\mu\nu},
\end{align*}
where we have introduced the shorthand $\chi_{\mu} = \chi_{\rho_{\mu}}$. We can now use the character inner product to arrive at the orthogonality relation
\begin{align*}
	\braket{\chi_{\mu}}{\chi_{\nu}} = \delta_{\mu\nu}.
\end{align*}

\paragraph{Decomposition of Representations}
Any representations may be written as a direct sum over irreps according to
\begin{align*}
	\rho = \bigoplus\limits_{k}\rho_{\mu_{k}},
\end{align*}
where the $\rho_{\mu_{k}}$ are the different irreps. Alternatively we may sum over the various types of irreps according to
\begin{align*}
	\rho = \bigoplus\limits_{\mu}k_{\mu}\rho_{\mu},
\end{align*}
where the $k_{\mu}$ are the numbers of times each individual irrep occurs. The character of this representation is
\begin{align*}
	\chi_{\rho} = \sum\limits_{\mu}k_{\mu}\chi_{\mu}.
\end{align*}
Projection onto a particular $\chi_{\nu}$ yields
\begin{align*}
	\braket{\chi_{\rho}}{\chi_{\mu}} = \sum\limits_{\mu}k_{\mu}\braket{\chi_{\mu}}{\chi_{\nu}} = \sum\limits_{\mu}k_{\mu}\delta_{\mu\nu} = k_{\nu}.
\end{align*}
This implies that the decomposition may be specified by computing the inner products $\braket{\chi_{\rho}}{\chi_{\mu}}$.

\section{Differential Geometry on Manifolds}
This is a very brief summary of choice results in differential geometry. For details, please consult my summary of SI2360 Analytical Mechanics and Classical Field Theory.

\paragraph{Manifolds}
A manifold is a set which is locally isomorphic to $\R^{n}$. We will take this to mean that we can locally impose coordinates $\chi^{a}$ on the manifold.

More formally, a manifold is described by a number of sets $U_{i}\subset\R^{n}$ called charts. To each chart belongs a set of coordinate functions $\chi_{i}$ which map from a subset $M_{i}\subset M$ to $U_{i}$ such that $\chi_{i}$ is a smooth bijection. A set of charts such that every point $p\in M$ is found in at least one chart is called an atlas.

\paragraph{Tangent Vectors}
Tangent vectors describe how scalar fields change with displacement along a curve. In Euclidean space the tangent basis was composed of derivatives with respect to the set of coordinates. In general curved spaces, we define
\begin{align*}
	\tb{a} = \del{a}{}.
\end{align*}
Derivatives are linear operators, so at least the set of tangent bases span some vector space and it makes sense to call a derivative a vector. A general tangent vector is now
\begin{align*}
	X = X^{a}\tb{a} = X^{a}\del{a}{}.
\end{align*}
These live in the tangent space $\ts{p}{M}$ of the manifold $M$ at the point $p$.

To get more of a sense of how this can be related to vectors, consider the directional derivative
\begin{align*}
	\grad_{\vb{n}} = \vb{n}\cdot\grad = n^{a}\del{a}{}.
\end{align*}
When applied to Euclidean space, there is a direct correspondence between $\vb{n}$ and the directional derivative, as $\grad_{\vb{n}}{\vb{x}} = \vb{n}$. For more general manifolds, tangent vectors are defined to be directional derivatives. Note that this definition carries with it the same dependence on position as was previously warned about.

Tangent vectors transform according to
\begin{align*}
	X^{a}\del{a}{} = X^{a}\del{a}{(\chi^{\prime})^{b}}\del[\prime]{b}{},
\end{align*}
implying the transformation rule
\begin{align*}
	(X^{\prime})^{a} = \del{b}{(\chi^{\prime})^{a}}X^{b},
\end{align*}
which is the same as the transformation rule for contravariant vector components in Euclidean space.

\paragraph{Dual Vectors}
To define dual vectors, we first introduce the dual space as the set of all linear operations from the tangent space to real numbers. This is also a vector space. The basis for the space is defined such that
\begin{align*}
	\db{a}(\del{b}{}) = \kdelta{b}{a}.
\end{align*}

In Euclidean space the dual basis was constructed from the gradient. The only concept here that carries over to manifolds is a definition based on small changes in the coordinates. More specifically, for any smooth scalar field $f$ we define a dual vector field according to
\begin{align*}
	df(X) = Xf = X^{a}\del{a}{f}
\end{align*}
and call it the differential. This has a similar structure to an inner product if the dual vector field has components $df_{a} = \del{a}{f}$. These components correspond to those of the gradient in Euclidean space. The basis we desire is $\db{a} = d\chi^{a}$. These live in the dual space $\ds{p}{M}$ of the manifold $M$ at the point $p$.

The dual basis satisfies satisfies
\begin{align*}
	d\chi^{a}(\del{b}{}) = \del{b}{\chi^{a}} = \kdelta{b}{a},
\end{align*}
as expected. Using this, we obtain
\begin{align*}
	df = (\del{a}{f})d\chi^{a},
\end{align*}
which at least looks like the differential of a function.

The components transform according to
\begin{align*}
	\del{a}{f} = \del[\prime]{b}{f}\del{a}{(\chi^{\prime})^{b}},
\end{align*}
which is the transformation rule for covariant vector components.

\paragraph{Tensors}
Having identified a basis for the tangent and dual spaces, we may now construct tensors similarly to what we have previously done. Note now that as the tangent and dual vectors belong to different vector spaces, the notion of type $(n, m)$ tensors is more clear. This also explains why we needed to be careful with indices being up or down when studying Euclidean space, as the difference is huge for manifolds.

\paragraph{Flow of Vector Fields}
The tangent bundle of a manifold is defined as $TM = \bigcup\limits_{p}\ts{p}{M}$. A vector field is a map $X: M\to\tbun{M}$ such that $X(p)\in\ts{p}{M}$. Given this, we may define the flow of a vector field as a collection of curves $\gamma_{X}$ which given some starting point $p$ satisfy
\begin{align*}
	\deval{\gamma_{X}}{\tau}{p, s} = \eval{X}_{\gamma_{X}(p, s)}.
\end{align*}
We may for a fixed $s$ define the function $\gamma_{sX}(p) = \gamma_{X}(p, s)$, which maps $M$ to itself.

\paragraph{Pushforwards and Pullbacks}
Consider some function $f$ which maps a manifold $M_{1}$ to another manifold $M_{2}$, as well as a function $g: M_{2}\to\R$. We then define the pullback of $g$ to $M_{1}$ by $f$ as $\pub{f}{g} = g\circ f$. We also define the pushforward of a vector $V\in\ts{p}{M_{1}}$ as $(\puf{f}{V})\phi = V(\pub{f}{\phi})$.

\paragraph{Tangents and the Pushforward}
$f$ maps the cooordinates $\chi^{a}$ of $M_{1}$ to the coordinates $\eta^{\mu}$ of $M_{2}$. By definition we have
\begin{align*}
	(\puf{f}{V})\phi &= V(\pub{f}{\phi}) \\
	&= V^{a}\del{a}{(\phi\circ f)} \\
	&= V^{a}\del{\mu}{\phi}\del{a}{\eta^{\mu}},
\end{align*}
meaning $\puf{f}{V} = V^{a}\del{a}{\eta^{\mu}}\del{\mu}{}$.

How do we interpret this? Consider some curve $\gamma$ in $M_{1}$ which is mapped to a curve $\alpha$ $M_{2}$ by $f$. If $V$ is the tangent of $\gamma$ at some particular point, we have
\begin{align*}
	\dot{\alpha} = \dot{\eta^{\mu}}\del{\mu}{} = \del{a}{\eta^{\mu}}\dot{\chi^{a}}\del{\mu}{} = \del{a}{\eta^{\mu}}V^{a}\del{\mu}{},
\end{align*}
meaning that the pushforward of a tangent by $f$ is the tangent of the curve produced by $f$.

\paragraph{The Pullback of Tensors}
We can now define the pullback of a $(0, m)$ tensor on $M_{2}$ according to
\begin{align*}
	\pub{f}{\omega}(V_{1}, \dots, V_{m}) = \omega(\puf{f}{V_{1}}, \dots, \puf{f}{V_{m}}).
\end{align*}
If $f$ is a bijection we may also define the more general pullback of a $(n, m)$ tensor on $M_{2}$ as
\begin{align*}
	\pub{f}{T}(V_{1}, \dots, V_{m}, \omega_{1}, \dots, \omega_{n}) = T(\puf{f}{V_{1}}, \dots, \puf{f}{V_{m}}, \pub{(f^{-1})}{\omega_{1}}, \dots, \pub{(f^{-1})}{\omega_{n}}).
\end{align*}

\paragraph{The Lie Derivative}
For a tensor field $T$ we define the Lie derivative in the $X$-direction as
\begin{align*}
	\lied{X}{T} = \lim\limits_{\varepsilon\to 0}\frac{1}{\varepsilon}\left(\pub{\gamma_{\varepsilon X}}{T} - T\right).
\end{align*}

\paragraph{An Expression for the Lie Derivative}
For a general tensor field we have
\begin{align*}
	\tensor*{(\lied{X}{T})}{^{a_{1}\dots a_{n}}_{b_{1}\dots b_{m}}} &= X^{a}\del{a}{\tensor*{T}{^{a_{1}\dots a_{n}}_{b_{1}\dots b_{m}}}} - \sum\limits_{i = 1}^{n}\tensor*{T}{^{a_{1}\dots a_{i - 1}aa_{i + 1}\dots a_{n}}_{b_{1}\dots b_{m}}}\del{a}{X^{a_{i}}} + \sum\limits_{j = 1}^{m}\tensor*{T}{^{a_{1}\dots a_{n}}_{b_{1}\dots b_{i - 1}ab_{i + 1}\dots b_{m}}}\del{b_{j}}{X^{a}}.
\end{align*}

\paragraph{Connections}
A connection is an operator on a tensor space that satisfies the following:
\begin{itemize}
	\item $\dcov{X}{f} = Xf = X^{a}\del{a}{f}$ for a scalar field $f$.
	\item $\dcov{X + Y}{T} = \dcov{X}{T} + \dcov{Y}{T}$.
	\item $\dcov{fX}{T} = f\dcov{X}{T}$.
	\item $\dcov{X}{(TS)} = (\dcov{X}{T})S + T\dcov{X}{S}$.
\end{itemize}

\paragraph{Connections on Manifolds}
On a manifold, a connection is specified by choosing $n$ independent vectors $X_{i}$ and defining
\begin{align*}
	\dcov{X_{i}}{X_{j}} = \chris{k}{i}{j}X_{k},
\end{align*}
where the expansion coefficients are called connection coefficients or Christoffel symbols. There is no unique way to do this, as the connection then depends on the choice of vectors.

\paragraph{The Connection of a Tensor Field}
Specify the connection according to $\dcov{\del{a}{}}{\del{b}{}} = \chris{c}{a}{b}\del{c}{}$. We then have
\begin{align*}
	\tensor*{(\dcov{X}{T})}{^{a_{1}\dots a_{n}}_{b_{1}\dots b_{m}}} &= X^{a}\left(\del{a}{\tensor*{T}{^{a_{1}\dots a_{n}}_{b_{1}\dots b_{m}}}} + \sum\limits_{i = 1}^{n}\chris{a_{i}}{a}{b}\tensor*{T}{^{a_{1}\dots a_{i - 1}ba_{i + 1}\dots a_{n}}_{b_{1}\dots b_{m}}} - \sum\limits_{j = 1}^{m}\chris{b}{a}{b_{j}}\tensor*{T}{^{a_{1}\dots a_{n}}_{b_{1}\dots b_{j - 1}bb_{j + 1}\dots b_{m}}}\right).
\end{align*}

\paragraph{The Difference Between Two Connections}
Given two different connection $\dcov{}{}$ and $\square$ with connection coefficients $\chris{}{}{}$ and $C$ we have
\begin{align*}
	(\dcov{a}{X} - \square_{a}X)^{b} = \del{a}{X^{b}} + \chris{b}{a}{c}X^{c} - (\del{a}{X^{b}} + C^{b}_{ac}X^{c}) = (\chris{b}{a}{c} - C^{b}_{ac})X^{c},
\end{align*}
and by the tensor quotient rule, the difference in the connection coefficients must transform as a tensor.

\paragraph{Parallel Transport}
A vector $X$ is termed parallel if $\dcov{a}{X} = 0$. This defines $n^{2}$ equations for the $n$ components of $X$, meaning that the system is overdetermined, and generally has no solution on a manifold. We may, however, define $X$ to be parallel along a curve $\gamma$ if
\begin{align*}
	\dcov{\dot{\gamma}}{X} = 0.
\end{align*}
This allows us to define the parallel transport as the vector field that solves the above equation with the vector $X$ as its initial condition. This defines $n$ equations for the $n$ components, and the system is solvable.

In particular, using the properties of the connection we find
\begin{align*}
	\dcov{\dot{\gamma}}{X} &= \dcov{\dot{\chi}^{a}\del{a}{}}{X^{c}\del{c}{}} \\
	&= \dot{\chi}^{a}\dcov{a}{X^{c}\del{c}{}} \\
	&= \dot{\chi}^{a}((\dcov{a}{X^{c}})\del{c}{} + X^{c}\dcov{a}{\del{c}{}}) \\
	&= \dot{\chi}^{a}(\del{a}{X^{b}} + X^{c}\chris{b}{a}{c})\del{b}{}.
\end{align*}

\paragraph{Geodesics and the Geodesic Equation}
A geodesic is defined as a curve with a tangent vector that is parallel along itself.

By definition a geodesic satisfies
\begin{align*}
	\dcov{\dot{\gamma}}{\dot{\gamma}} = \dot{\chi}^{a}(\del{a}{\dot{\chi}^{b}} + \dot{\chi}^{c}\chris{b}{a}{c})\del{b}{} = (\ddot{\chi}^{b} + \dot{\chi}^{a}\dot{\chi}^{c}\chris{b}{a}{c})\del{b}{} = 0,
\end{align*}
and thus
\begin{align*}
	\ddot{\chi}^{b} + \dot{\chi}^{a}\dot{\chi}^{c}\chris{b}{a}{c} = 0.
\end{align*}
This is the geodesic equation. Given a starting point and a tangent vector, it is solvable.

\paragraph{Torsion}
The torsion tensor is a $(1, 2)$ tensor defined as
\begin{align*}
	T(X, Y) = \dcov{X}{Y} - \dcov{Y}{X} - \comm{X}{Y}.
\end{align*}
To find its components, we note that
\begin{align*}
	T_{ab} = T(\del{a}{}, \del{b}{}) &= \dcov{a}{\del{b}{}} - \dcov{b}{\del{a}{}} - \comm{\del{a}{}}{\del{b}{}} \\
	&= \chris{c}{a}{b}\del{c}{} - \chris{c}{b}{a}\del{c}{} - (\del{a}{\del{b}{}} - \del{b}{\del{a}{}}) \\
	&= (\chris{c}{a}{b} - \chris{c}{b}{a})\del{c}{},
\end{align*}
implying
\begin{align*}
	T_{ab}^{c} &= \chris{c}{a}{b} - \chris{c}{b}{a}.
\end{align*}

\paragraph{Curvature}
Consider some vector $Z$ parallel transported along a small closed loop. The parallel transport is linear, so the result of this process must be connected to some $(1, 1)$ tensor. Supposing that the loop is spanned by $X$ and $Y$, we have
\begin{align*}
	Z^{\prime} - Z = R(X, Y)Z = \dcov{X}{\dcov{Y}{Z}} - \dcov{Y}{\dcov{X}{Z}} - \dcov{\comm{X}{Y}}{Z}.
\end{align*}
We define $R(X, Y)Z$ as the Riemann curvature tensor. It is a $(1, 3)$ tensor. Its components are defined by
\begin{align*}
	R(\del{a}{}, \del{b}{})\del{c}{} &= \tensor{R}{^{d}_{cab}}\del{d}{} \\
	&= \dcov{a}{\dcov{b}{\del{c}{}}} - \dcov{b}{\dcov{a}{\del{c}{}}} \\
	&= \dcov{a}{\chris{f}{b}{c}\del{f}{}} - \dcov{b}{\chris{f}{a}{c}\del{f}{}} \\
	&= (\dcov{a}{\chris{f}{b}{c}})\del{f}{} + \chris{f}{b}{c}\dcov{a}{\del{f}{}} - (\dcov{b}{\chris{f}{a}{c}})\del{f}{} - \chris{f}{a}{c}\dcov{b}{\del{f}{}} \\
	&= (\del{a}{\chris{d}{b}{c}} + \chris{f}{b}{c}\chris{d}{a}{f} - \del{b}{\chris{d}{a}{c}} - \chris{f}{a}{c}\chris{d}{b}{f})\del{d}{}, 
\end{align*}
and thus
\begin{align*}
	\tensor{R}{^{d}_{cab}} = \del{a}{\chris{d}{b}{c}} - \del{b}{\chris{d}{a}{c}} + \chris{f}{b}{c}\chris{d}{a}{f} - \chris{f}{a}{c}\chris{d}{b}{f}.
\end{align*}
Note the placements of the indices.

\paragraph{The Metric Tensor}
The metric tensor will be taken as the $(0, 2)$ tensor that defines inner products on manifolds. The inner product, and therefore also the metric tensor, is a map from $\ts{p}{M}\times\ts{p}{M}$ that is symmetric and positive definite. Using this we may extend more of the previously performed work, for instance on curve length.

\paragraph{Metric Compatibility}
A connection is metric compatible if $\dcov{X}{g} = 0$ for all vectors $X$.

\paragraph{The Levi-Civita Connection}
For any manifold with some metric there exists a unique connection that is both metric compatible and torsion free. This connection is termed the Levi-Civita connection.

\paragraph{Curves of Minimal Length}
As the metric defines length, we define the curve length as
\begin{align*}
	l_{\gamma} = \integ{\gamma}{}{s}{} = \integ{0}{1}{t}{\sqrt{g_{ab}\dot{\chi}^{a}\dot{\chi}^{b}}}.
\end{align*}
Defining $\sqrt{\lag} = g_{ab}\dot{\chi}^{a}\dot{\chi}^{b}$, the curve that minimizes the distance between the start and end points satisfies
\begin{align*}
	\del{a}{\sqrt{\lag}} - \dv{t}\pdv{\sqrt{\lag}}{\dot{\chi}^{a}} = \frac{1}{2\sqrt{\lag}}\left(\del{a}{\lag} - \sqrt{\lag}\dv{t}\left(\frac{1}{\sqrt{\lag}}\pdv{\lag}{\dot{\chi}^{a}}\right)\right) = 0.
\end{align*}
One can always choose a parametrization such that $\sqrt{\lag} = 1$ (the arc length parametrization is one example), yielding
\begin{align*}
	\frac{1}{2\sqrt{\lag}}\left(\del{a}{\lag} - \dv{t}\pdv{\lag}{\dot{\chi}^{a}}\right) = 0,
\end{align*}
which is equivalent to extremizing the integral of $\lag$. In terms of the coordinate functions we thus have
\begin{align*}
	\del{a}{g_{bc}}\dot{\chi}^{b}\dot{\chi}^{c} - \dv{t}\pdv{\lag}{\dot{\chi}^{a}} = \del{a}{g_{bc}}\dot{\chi}^{b}\dot{\chi}^{c} - \dv{t}(2g_{ab}\dot{\chi}^{b}) = \del{a}{g_{bc}}\dot{\chi}^{b}\dot{\chi}^{c} - 2g_{ab}\ddot{\chi}^{b} - 2\del{c}{g_{ab}}\dot{\chi}^{b}\dot{\chi}^{c} = 0.
\end{align*}
Multiplying by $-\frac{1}{2}g^{da}$ we find
\begin{align*}
	g^{da}g_{ab}\ddot{\chi}^{b} - \frac{1}{2}g^{da}\del{a}{g_{bc}}\dot{\chi}^{b}\dot{\chi}^{c} + g^{da}\del{c}{g_{ab}}\dot{\chi}^{b}\dot{\chi}^{c} = \ddot{\chi}^{d} + \frac{1}{2}g^{da}(2\del{c}{g_{ab}} - \del{a}{g_{bc}})\dot{\chi}^{b}\dot{\chi}^{c} = 0.
\end{align*}
Renaming indices for convenience we find
\begin{align*}
	\ddot{\chi}^{b} + \frac{1}{2}g^{bd}(2\del{c}{g_{da}} - \del{d}{g_{ac}})\dot{\chi}^{a}\dot{\chi}^{c} = 0.
\end{align*}

\paragraph{Geodesics and Minimum-Length Curves}
Geodesics and curves of minimal length coincide if
\begin{align*}
	\chris{b}{a}{c} = \frac{1}{2}g^{bd}(2\del{c}{g_{da}} - \del{d}{g_{ac}}).
\end{align*}
As the above identification is done based on a quantity that is symmetric in the lower indices, we cannot find any information about the antisymmetric part of the connection coefficients from this.

\paragraph{The Levi-Civita Connection and Geodesics}
The connection coefficients (or Christoffel symbols) defined by the Levi-Civita connection are symmetric due to the torsion being zero. This implies
\begin{align*}
	\chris{d}{a}{b} = \frac{1}{2}g^{dc}(\del{b}{g_{ac}} + \del{a}{g_{cb}} - \del{c}{g_{ab}}).
\end{align*}

\paragraph{The Induced Metric}
Given some immersion $f$ of $M_{1}$ into $M_{2}$ and supposing that the metric $g$ exists on $M_{2}$, this induces a metric $\tilde{g} = \pub{f}{g}$ on $M_{1}$.

\paragraph{Components of the Induced Metric}
Suppose that there is an immersion $f: \chi^{a}\to\eta^{\mu}$ of one manifold into another, and that the metric is $g$ in the outer manifold. By definition the induced metric satisfies $\tilde{g}(U, V) = \pub{f}{g}(U, V) = g(\puf{f}{U}, \puf{f}{V})$, implying
\begin{align*}
	\tilde{g}_{ab}U^{a}V^{b} = g_{\mu\nu}U^{a}\del{a}{\eta^{\mu}}V^{b}\del{b}{\eta^{\nu}},
\end{align*}
and as this is true for any pair of vectors we recognize
\begin{align*}
	\tilde{g}_{ab} = g_{\mu\nu}\del{a}{\eta^{\mu}}\del{b}{\eta^{\nu}}.
\end{align*}

\paragraph{Contravariant Curvature Components}
Using the Levi-Civita connection, we may introduce the contravariant curvature components
\begin{align*}
	R_{abcd} &= \frac{1}{2}\left(\del{a}{\del{d}{g_{bc}}} + \del{b}{\del{c}{g_{ad}}} - \del{a}{\del{c}{g_{bd}}} - \del{b}{\del{d}{g_{ac}}}\right) + g_{fh}(\chris{f}{b}{c}\chris{h}{a}{d} - \chris{f}{b}{d}\chris{h}{a}{c}).
\end{align*}
As we see, the contravariant components are antisymmetric in the two first and two last indices. Furthermore, a simultaneous switch of the pairs of first and last indices leaves the components invariant, hence $R_{abcd} = R_{cdab} = -R_{abdc} = -R_{bacd}$. In addition we have
\begin{align*}
	R(X, Y)Z + R(Y, Z)X + R(Z, X)Y =& \dcov{X}{\dcov{Y}{Z}} - \dcov{Y}{\dcov{X}{Z}} - \dcov{\comm{X}{Y}}{Z} + \dcov{Y}{\dcov{Z}{X}} - \dcov{Z}{\dcov{Y}{X}} - \dcov{\comm{Y}{Z}}{X} \\
	                                &+ \dcov{Z}{\dcov{X}{Y}} - \dcov{X}{\dcov{Z}{Y}} - \dcov{\comm{Z}{X}}{Y} \\
	                               =& \dots = 0.
\end{align*}
In component form this becomes $R_{abcd} + R_{acdb} + R_{adbc} = 0$. There also exists a differential Bianchi identity
\begin{align*}
	(\dcov{Z}{R})(X, Y) + (\dcov{X}{R})(Y, Z) + (\dcov{Y}{R})(Z, X) = 0,
\end{align*}
which in coordinate form becoms $\dcov{f}{R_{abcd}} + \dcov{c}{R_{abdf}} + \dcov{d}{R_{abfc}} = 0$, meaning that the total number of independent components is $\frac{1}{12}n^{2}(n^{2} - 1)$.

\paragraph{Killing Fields}
$K$ is a Killing field if $\lied{K}{g} = 0$.

\paragraph{The Lie Derivative with Killing Fields}
Let $K$ be a Killing field. We then obtain
\begin{align*}
	\lied{K}{g_{ab}} &= \dcov{b}{K_{a}} + \dcov{a}{K_{b}} = 0,
\end{align*}
and all Killing fields must satisfy this relation.

\paragraph{The Ricci Tensor}
The Ricci tensor is defined as $R_{ab} = \tensor{R}{^{c}_{acb}}$. Its components are
\begin{align*}
	R_{ab} = \del{c}{\chris{c}{b}{a}} - \del{b}{\chris{c}{c}{a}} + \chris{f}{b}{a}\chris{c}{c}{f} - \chris{f}{c}{a}\chris{c}{b}{f} .
\end{align*}

\paragraph{The Ricci Scalar}
The Ricci scalar is defined as the contraction $\mathcal{R} = g^{ab}R_{ab} = g^{ab}\tensor{R}{^{c}_{acb}}$.

\paragraph{The Einstein Tensor}
The Einstein tensor is defined as $G_{ab} = R_{ab} - \frac{1}{2}g_{ab}\mathcal{R}$.

\paragraph{The Divergence of the Einstein Tensor}
With the Levi-Civita connection we have
\begin{align*}
	g^{ac}g^{bd}(\dcov{f}{R_{abcd}} + \dcov{c}{R_{abdf}} + \dcov{d}{R_{abfc}}) &= \dcov{f}{(g^{ac}g^{bd}R_{abcd})} + \dcov{c}{(g^{ac}g^{bd}R_{abdf})} + \dcov{d}{(g^{ac}g^{bd}R_{abfc})} \\
	&= \dcov{f}{(g^{bd}\tensor{R}{^{c}_{bcd}})} - \dcov{c}{(g^{ac}\tensor{R}{^{d}_{adf}})} - \dcov{d}{(g^{bd}\tensor{R}{^{c}_{bcf}})} \\
	&= \dcov{f}{\R} - \dcov{c}{(g^{ac}\tensor{R}{_{af}})} - \dcov{d}{(g^{bd}\tensor{R}{_{bf}})} \\
	&= \dcov{f}{\R} - 2\dcov{c}{\tensor{R}{^{c}_{f}}} = 0.
\end{align*}
By comparison we have
\begin{align*}
	\dcov{a}{G^{ab}} &= \dcov{a}{\left(R^{ab} - \frac{1}{2}g^{ab}\R\right)},
\end{align*}
and by lowering the index $b$ we find that this must be zero.

\paragraph{Differential Forms}
The set of $p$-forms, or differential forms, is the set of $(0, p)$ tensors that are completely antisymmetric. They are constructed using the wedge product, defined as
\begin{align*}
	\bigwedge\limits_{k = 1}^{p}d\chi^{a_{k}} = \sum\limits_{\sigma\in S_{p}}\text{sgn}(\sigma)\bigotimes_{k = 1}^{p}d\chi^{a_{\sigma(k)}}.
\end{align*}
Here $S_{p}$ is the set of permutations of $p$ elements. There exists
\begin{align*}
	n_{p}^{N} = {N\choose k}
\end{align*}
basis elements. We note that the wedge product is antisymmetric under the exchange of two basis elements. Hence, once an ordering of indices has been chosen, any permutation will simply create a linearly dependent map.

Consider now some antisymmetric tensor $\omega$. Introducing the antisymmetrizer
\begin{align*}
	\bigotimes_{k = 1}^{p}d\chi^{[a_{\sigma(k)}]} = \frac{1}{p!}\sum\limits_{\sigma\in S_{p}}\text{sgn}(\sigma)\bigotimes_{k = 1}^{p}d\chi^{a_{\sigma(k)}},
\end{align*}
the symmetry yields
\begin{align*}
	\omega = \omega_{a_{1}\dots a_{p}}\bigotimes_{k = 1}^{p}d\chi^{a_{\sigma(k)}} = \omega_{a_{1}\dots a_{p}}\bigotimes_{k = 1}^{p}d\chi^{[a_{\sigma(k)}]} = \frac{1}{p!}\omega_{a_{1}\dots a_{p}}\bigwedge\limits_{k = 1}^{p}d\chi^{a_{k}}.
\end{align*}

\paragraph{The Exterior Derivative}
We define the exterior derivative of a differential form according to
\begin{align*}
	d\omega = \frac{1}{p!}\del{a_{1}}{\omega_{a_{2}\dots a_{p + 1}}}\bigwedge\limits_{k = 1}^{p + 1}d\chi^{a_{k}},
\end{align*}
which is a $p + 1$-form. This notation makes sense, as at least in the case of a $0$-form, we obtain
\begin{align*}
	d\omega = \del{a}{\omega}d\chi^{a},
\end{align*}
which is exactly the form of a dual vector. Somehow this transforms as a tensor.

\paragraph{Integration of Differential Forms}
Consider a set of $p$ tangent vectors $X_{i}$. The corresponding coordinate displacements are $\dd{\chi_{i}^{a}} = X_{i}^{a}\dd{t_{i}}$, with no sum over $i$. We would now like to compute the $p$-dimensional volume defined by the $X_{i}$ and $\dd{t_{i}}$. We expect that if any of the $X_{i}$ are linearly dependent the volume should be zero. We also expect that the volume be linear in the $X_{i}$. This implies
\begin{align*}
	\dd{V_{p}} = \omega(X_{1}, \dots, X_{p})\dd{t_{1}}\dots\dd{t_{p}}
\end{align*}
for some differential form $\omega$. We now define the integral over the $p$-volume $S$ over the $p$-form $\omega$ as
\begin{align*}
	\minteg{S}{\omega} = \integ{}{}{t_{1}}{\dots \integ{}{}{t_{p}}{\omega(\dot{\gamma}_{1},\dots, \dot{\gamma}_{p})}}.
\end{align*}
Here the $\gamma_{i}$ are the set of curves that span $S$, the dot symbolizes the derivative with respect to the individual curve parameters and the right-hand integration is performed over the appropriate set of parameter values.

\paragraph{Stokes' Theorem}
Stokes' theorem relates the integral of a differential form $d\omega$ over some subset $V$ of a manifold to an integral over \bound{V} of another differential form.
It states
\begin{align*}
	\minteg{V}{d\omega} = \oint\limits_{\bound{V}}\omega.
\end{align*}

\paragraph{Total Derivatives}
In the context of a general manifold, a total derivative is of the form $\dcov{a}{T^{aa_{1}\dots}_{b_{1}\dots}}$. Stokes' theorem states
\begin{align*}
	\minteg{V}{\dcov{a}{T^{aa_{1}\dots}_{b_{1}\dots}}} = \oint\limits_{\bound{V}}T^{aa_{1}\dots}_{b_{1}\dots}.
\end{align*}

\section{Differentiation and Integration in Orthogonal Coordinates}

To tie together what we have learned thus far with what we studied in Vector Calculus, we will study differentiation and integration in orthogonal coordinate systems. For this part of the summary we will take a break from the oh-so strict indexing rules established above.

\paragraph{Defining Relation}
Orthogonal coordinate systems are defined by the relation
\begin{align*}
	\tb{a}\cdot\tb{b} = h_{a}^{2}\delta_{ab}\nosum .
\end{align*}

\paragraph{Orthonormal Basis}
Based on the orthogonality conditions, we define the orthogonal basis vectors
\begin{align*}
	\vb{e}_{a} = \frac{1}{h_{a}}\tb{a}\nosum .
\end{align*}
Normeringsvillkoret ger då direkt
\begin{align*}
	h_{a} = \sqrt{\sum\limits_{i}\del{a}{x^{i}}}.
\end{align*}

\paragraph{Physical Components}
The physical components of a vector is its projection onto the orthonormal basis vectors, denoted with a tilde.

\paragraph{Relation to Dual Basis}
By expanding the dual basis vectors in terms of their physical components, we obtain
\begin{align*}
	\tilde{E}_{a}^{b} = \vb{e}_{b}\cdot\db{a} = \frac{1}{h_{b}}\kdelta{b}{a}\nosum .
\end{align*}
This implies
\begin{align*}
	\db{a} = \tilde{E}_{b}^{a}\vb{e}_{b} = \frac{1}{h_{a}}\kdelta{a}{b}\vb{e}_{b} = \frac{1}{h_{a}}\vb{e}_{a}\nosum ,
\end{align*}
and thus
\begin{align*}
	\vb{e}_{a} = h_{a}\db{a}.
\end{align*}
We see that the dual basis would have been an equally good starting point for describing orthogonal systems.

\section{Tensors}

\paragraph{Definition}
A tensor of rank $N$ is a multilinear map from $N$ vectors to a scalar.

\paragraph{Components of a tensor}
The components of a tensor are defined by
\begin{align*}
	T(\vb{E}^{a_{1}}, \dots, \vb{E}^{a_{N}}) = T^{a_{1}, \dots, a_{N}}.
\end{align*}
These are called the contravariant components of the tensor, and the covariant components are defined similarly. Mixed components can also be defined.

\paragraph{Rules for tensors}
Tensors obey the following rules:
\begin{align*}
	(T_{1} + T_{2})(\vb{w}_{1}, \dots, \vb{w}_{n_{2}}) &= T_{1}(\vb{w}_{1}, \dots, \vb{w}_{n_{2}}) + T_{2}(\vb{w}_{1}, \dots, \vb{w}_{n_{2}}), \\
	(kT)(\vb{w}_{1}, \dots, \vb{w}_{n_{2}})            &= kT(\vb{w}_{1}, \dots, \vb{w}_{n_{2}}).
\end{align*}
In component form:
\begin{align*}
	(T_{1} + T_{2})^{a_{1}\dots a_{n}} &= T_{1}^{a_{1}\dots a_{n}} + T_{2}^{a_{1}\dots a_{n}}, \\
	(kT)^{a_{1}\dots a_{n}}            &= kT^{a_{1}\dots a_{n}}.
\end{align*}

\paragraph{The metric tensor}
The metric tensor $g$ is a rank $2$ tensor defined by $g(\vb{v}, \vb{w}) = \vb{v}\cdot\vb{w}$. Its components satisfy
\begin{align*}
	v_{a} = \vb{E}_{a}\cdot v^{b}\vb{E}_{b} = g(\vb{E}_{a}, \vb{E}_{b})v_{b} = g_{ab}v^{b},
\end{align*}
and likewise
\begin{align*}
	v^{a} = g^{ab}v_{b}.
\end{align*}

We note that
\begin{align*}
	v_{a} = g_{ab}v^{b} = g_{ab}g^{bc}v_{c},
\end{align*}
which implies $g_{ab}g^{bc} = \kdelta{a}{c}$.

\paragraph{Tensor product}
Given two tensors $T_{1}$ and $T_{2}$ of ranks $n_{1}$ and $n_{2}$, we kan define the rank $n_{1} + n_{2}$ tensor $\tenprod{{T_{1}, T_{2}}}$ as
\begin{align*}
	(\tenprod{{T_{1}, T_{2}}})(\vb{v}_{1}, \dots, \vb{v}_{n_{1}}, \vb{w}_{1}, \dots, \vb{w}_{n_{2}}) = T_{1}(\vb{v}_{1}, \dots, \vb{v}_{n_{1}})T_{2}(\vb{w}_{1}, \dots, \vb{w}_{n_{2}}).
\end{align*}
In component form:
\begin{align*}
	(\tenprod{{T_{1}, T_{2}}})^{a_{1}\dots a_{n_{1} + n_{2}} = T_{1}^{a_{1}\dots a_{n_{1}}}T_{2}^{a_{n_{1} + 1}}\dots a_{n_{1} + n_{2}}}.
\end{align*}

\paragraph{Tensors as linear combinations}
Using the tensor product, all tensors can be written as linear combinations of certain basis elements due to their bilinearity. Define
\begin{align*}
	e_{a_{1}\dots a_{n}} = \tenprod{{\vb{E}_{a_{1}}, \dots, \vb{E}_{a_{n}}}}
\end{align*}
to be the tensor that satisfies
\begin{align*}
	e_{a_{1}\dots a_{n}}(\vb{E}^{b_{1}}, \dots, \vb{E}^{b_{n}}) = (\vb{E}_{a_{1}}\cdot\vb{E}^{b_{1}})\dots (\vb{E}_{a_{n}}\cdot\vb{E}^{b_{n}}) = \kdelta{a_{1}}{b_{1}}\dots\kdelta{a_{n}}{b_{n}}.
\end{align*}
Then any tensor can be written as
\begin{align*}
	T = T^{a_{1}\dots a_{n}}e_{a_{1}\dots a_{n}}
\end{align*}
where the $T^{a_{1}\dots a_{n}}$ are exactly the contravariant components of $T$.

\paragraph{Tensors as linear transforms on tensors}
A rank $n$ tensor can also be viewed as a linear map from rank $m$ tensors to rank $n - m$ tensors. To do this, we first define, given $T$, the rank $n - m$ tensor $\tilde{T}(\tenprod{{\vb{w}_{1}, \dots, \vb{w}_{m}}})$ such that
\begin{align*}
	(\tilde{T}(\tenprod{{\vb{w}_{1}, \dots, \vb{w}_{m}}}))(\vb{v}_{1}, \dots, \vb{v}_{n - m}) = T(\vb{w}_{1}, \dots, \vb{w}_{m}, \vb{v}_{1}, \dots, \vb{v}_{n - m}).
\end{align*}
This map is also linear in all the $\vb{w}_{i}$. Next, given a rank $n - m$ tensor $\tilde{T}$, one can define the rank $n - m$ tensor $T(\vb{w}_{1}, \dots, \vb{w}_{m})$ such that
\begin{align*}
	T(\vb{w}_{1}, \dots, \vb{w}_{m}, \vb{v}_{1}, \dots, \vb{v}_{n - m}) = (\tilde{T}(\tenprod{{\vb{w}_{1}, \dots, \vb{w}_{m}}}))(\vb{v}_{1}, \dots, \vb{v}_{n - m}).
\end{align*}
This is a linear rank $n$ tensor.

\paragraph{Tensor contraction}
Given a complete set of vectors $\vb{v}_{i}$ and their dual $\vb{v}^{i}$ such that $\vb{v}_{i}\cdot\vb{v}^{i} = \kdelta{i}{j}$, the contraction $e_{12}T$ of two arguments of a rank $n$ tensor is the tensor of rank $n - 2$ satisfying
\begin{align*}
	(e_{12}T)(\vb{w}_{1}, \dots, \vb{w}_{n - 2}) = T(\vb{v}_{i}, \vb{v}^{i}, \vb{w}_{1}, \dots, \vb{w}_{n - 2}).
\end{align*}
In component form:
\begin{align*}
	(e_{12}T)^{a_{1}\dots a_{n - 2}} = T^{c\;a_{1}\dots a_{n - 2}}_{\;c}.
\end{align*}
The definition is similar (I assume) for the contraction of other arguments.

\section{Advanced Differential Geometry}

In this part we will expand on the previously discussed concepts of differential geometry, mainly by incorporating our knowledge of tensors into it.

\paragraph{The Metric Tensor}
The metric tensor $g$ is a rank $2$ tensor. We start by defining it as $g(\vb{v}, \vb{w}) = \vb{v}\cdot\vb{w}$, but more generally the metric tensor defines the inner product.

The metric tensor is symmetric. Its components satisfy
\begin{align*}
	v_{a} = \vb{E}_{a}\cdot v^{b}\vb{E}_{b} = g(\vb{E}_{a}, \vb{E}_{b})v^{b} = g_{ab}v^{b},
\end{align*}
and likewise
\begin{align*}
	v^{a} = g^{ab}v_{b}
\end{align*}
where $\vb{v}$ is a vector. This demonstrates the capabilities of the metric to raise and lower indices.

We note that
\begin{align*}
	v_{a} = g_{ab}v^{b} = g_{ab}g^{bc}v_{c},
\end{align*}
which implies $g_{ab}g^{bc} = \kdelta{a}{c}$.

\example{The Metric in Polar Coordinates}
The contravariant components of the metric tensor, according to the definition, are
\begin{align*}
	g_{rr} = \tb{r}\cdot\tb{r} = 1,\ g_{r\phi} = g_{\phi r} = \tb{r}\cdot\tb{\phi} = 0,\ g_{\phi\phi} = \tb{\phi}\cdot\tb{\phi} = r^{2}.
\end{align*}
Likewise, the covariant components are
\begin{align*}
	g^{rr} = \db{r}\cdot\db{r} = 1,\ g_{r\phi} = g_{\phi r} = \db{r}\cdot\db{\phi} = 0,\ g_{\phi\phi} = \db{\phi}\cdot\db{\phi} = \frac{1}{r^{2}}.
\end{align*}

\paragraph{Christoffel Symbols}
When computing the derivative of a vector quantity, one must account both for the change in the quantity itself and the change in the basis vectors. We define the Christoffel symbols according to
\begin{align*}
	\del{b}{\tb{a}} = \chris{c}{b}{a}\tb{c}.
\end{align*}
These can be computed according to
\begin{align*}
	\db{c}\cdot\del{b}{\tb{a}} = \db{c}\cdot\chris{d}{b}{a}\tb{d} = \kdelta{d}{c}\chris{d}{b}{a} = \chris{c}{b}{a}.
\end{align*}
Note that
\begin{align*}
	\del{a}{\tb{b}} = \del{a}{\del{b}{\vb{r}}} = \del{b}{\del{a}{\vb{r}}} = \del{b}{\tb{a}},
\end{align*}
which implies
\begin{align*}
	\chris{c}{b}{a} = \chris{c}{a}{b}.
\end{align*}

Do the Christoffel symbols define a tensor? Clearly they do not. One simple counterexample is when converting from Cartesian coordinates to any non-trivial coordinate system. In Cartesian coordinates all Christoffel symbols are zero, and no linear combination of these could possibly produce non-zero values. There is a transformation rule, however. To find it, we study
\begin{align*}
	\tensor{(\Gamma^{\prime})}{^{a}_{bc}} &= (\db{a})^{\prime}\cdot\del[\prime]{b}{(\tb{c})^{\prime}} \\
	                                      &= \del{d}{(\chi')^{a}}\db{d}\cdot\del[\prime]{b}{(\del[\prime]{c}{\chi^{f}}\tb{f})} \\
	                                      &= \del{d}{(\chi')^{a}}\db{d}\cdot(\tb{f}\del[\prime]{b}{\del[\prime]{c}{\chi^{f}}} + \del[\prime]{c}{\chi^{f}}\del[\prime]{b}{\tb{f}}) \\
	                                      &= \del{d}{(\chi')^{a}}(\kdelta{f}{d}\del[\prime]{b}{\del[\prime]{c}{\chi^{f}}} + \del[\prime]{c}{\chi^{f}}\db{d}\cdot\del[\prime]{b}{\chi^{g}}\del{g}{\tb{f}}) \\
	                                      &= \del{d}{(\chi')^{a}}(\del[\prime]{b}{\del[\prime]{c}{\chi^{d}}} + \del[\prime]{c}{\chi^{f}}\del[\prime]{b}{\chi^{g}}\chris{d}{g}{f}) \\
	                                      &= \del{d}{(\chi')^{a}}\del[\prime]{c}{\chi^{f}}\del[\prime]{b}{\chi^{g}}\chris{d}{g}{f} + \del{d}{(\chi')^{a}}\del[\prime]{b}{\del[\prime]{c}{\chi^{d}}}.
\end{align*}

\example{Christoffel Symbols in Polar Coordinates}
To compute these, we need partial derivative of the basis vectors. We have
\begin{align*}
	\del{r}{\tb{r}} = \vb{0},\ \del{\phi}{\tb{r}} = \del{r}{\tb{\phi}} = \frac{1}{r}\tb{\phi},\ \del{\phi}{\tb{\phi}} = -r\tb{r}.
\end{align*}
We thus obtain
\begin{align*}
	\chris{a}{r}{r} = 0,\ \chris{r}{r}{\phi} = 0,\ \chris{\phi}{r}{\phi} = \frac{1}{r},\ \chris{r}{\phi}{\phi} = -r,\ \chris{\phi}{\phi}{\phi} = 0.
\end{align*}

\paragraph{Covariant Derivatives}
The partial derivate of $\vb{v} = v^{a}\tb{a}$ with respect to $\chi^{a}$ is given by
\begin{align*}
	\del{a}{\vb{v}} = \tb{b}\del{a}{v^{b}} + v^{b}\del{a}{\tb{b}} = \tb{b}\del{a}{v^{b}} + v^{b}\chris{c}{a}{b}\tb{c}.
\end{align*}
Renaming the summation indices yields
\begin{align*}
	\del{a}{\vb{v}} = \tb{b}(\del{a}{v^{b}} + v^{c}\chris{b}{a}{c}),
\end{align*}
which contains one term from the change in the coordinates and one term from the change in basis.

Realizing that derivatives of vector quantities must take both of these into account in order to transform like a tensor, we would like to define a differentiation operation that takes both of these to account when differentiating vector components. This is the covariant derivative. We define its action on contravariant vector components as
\begin{align*}
	\dcov{a}{v^{b}} = \del{a}{v^{b}} + v^{c}\chris{b}{a}{c},
\end{align*}
such that
\begin{align*}
	\del{a}{\vb{v}} = E_{b}\dcov{a}{v^{a}}.
\end{align*}
In a similar fashion we would like to define its action on covariant vector components. To do this, we use the fact that
\begin{align*}
	\del{a}{(\tb{b}\cdot\db{c})} = \del{a}{\kdelta{b}{c}} = 0.
\end{align*}
The product rule yields
\begin{align*}
	\tb{b}\cdot\del{a}{\db{c}} + \db{c}\cdot\del{a}{\tb{b}} = \tb{b}\cdot\del{a}{\db{c}} + \db{c}\cdot\chris{d}{a}{b}\tb{d} = \tb{b}\cdot\del{a}{\db{c}} + \kdelta{d}{c}\cdot\chris{d}{a}{b} = \tb{b}\cdot\del{a}{\db{c}} + \chris{c}{a}{b},
\end{align*}
which implies
\begin{align*}
	\del{a}{\db{c}} = -\chris{c}{a}{b}\db{b}.
\end{align*}
Repeating the steps above now yields
\begin{align*}
	\dcov{a}{v_{b}} = \del{a}{v_{b}} - \chris{c}{a}{b}v_{c}.
\end{align*}

\paragraph{Covariant Derivatives of Tensor Fields}
Next we study the derivatives of a tensor field
\begin{align*}
	\tau = \tensor*{\tau}{^{a_{1}\dots a_{n}}_{b_{1}\dots b_{m}}}\tensor*{e}{^{b_{1}\dots b_{m}}_{a_{1}\dots a_{n}}}.
\end{align*}
The tensor basis element differentiates according to the product rule, but with multiplication replaced by the tensor product. Hence
\begin{align*}
	\del{a}{\tau} &= \tensor*{e}{^{b_{1}\dots b_{m}}_{a_{1}\dots a_{n}}}\del{a}{\tensor*{\tau}{^{a_{1}\dots a_{n}}_{b_{1}\dots b_{m}}}} + \tensor*{\tau}{^{a_{1}\dots a_{n}}_{b_{1}\dots b_{m}}}\del{a}{\tensor*{e}{^{b_{1}\dots b_{m}}_{a_{1}\dots a_{n}}}} \\
	              &= \tensor*{e}{^{b_{1}\dots b_{m}}_{a_{1}\dots a_{n}}}\del{a}{\tensor*{\tau}{^{a_{1}\dots a_{n}}_{b_{1}\dots b_{m}}}} + \tensor*{\tau}{^{a_{1}\dots a_{n}}_{b_{1}\dots b_{m}}}\left(\sum\limits_{k = 1}^{n}\chris{c_{k}}{a}{a_{k}}\tensor*{e}{^{b_{1}\dots b_{m}}_{c_{1}\dots c_{n}}} - \sum\limits_{l = 1}^{m}\chris{b_{l}}{a}{d_{l}}\tensor*{e}{^{d_{1}\dots d_{m}}_{a_{1}\dots a_{n}}}\right) \\
	              &= \tensor*{e}{^{b_{1}\dots b_{m}}_{a_{1}\dots a_{n}}}\del{a}{\tensor*{\tau}{^{a_{1}\dots a_{n}}_{b_{1}\dots b_{m}}}} + \sum\limits_{k = 1}^{n}\tensor*{\tau}{^{c_{1}\dots c_{n}}_{b_{1}\dots b_{m}}}\chris{a_{k}}{a}{c_{k}}\tensor*{e}{^{b_{1}\dots b_{m}}_{a_{1}\dots a_{n}}} - \sum\limits_{l = 1}^{m}\tensor*{\tau}{^{a_{1}\dots a_{n}}_{d_{1}\dots d_{m}}}\chris{d_{l}}{a}{b_{l}}\tensor*{e}{^{b_{1}\dots b_{m}}_{a_{1}\dots a_{n}}} \\
	              &= \tensor*{e}{^{b_{1}\dots b_{m}}_{a_{1}\dots a_{n}}}\left(\del{a}{\tensor*{\tau}{^{a_{1}\dots a_{n}}_{b_{1}\dots b_{m}}}} + \sum\limits_{k = 1}^{n}\tensor*{\tau}{^{c_{1}\dots c_{n}}_{b_{1}\dots b_{m}}}\chris{a_{k}}{a}{c_{k}} - \sum\limits_{l = 1}^{m}\tensor*{\tau}{^{a_{1}\dots a_{n}}_{d_{1}\dots d_{m}}}\chris{d_{l}}{a}{b_{l}}\right).
\end{align*}
We thus define
\begin{align*}
	\dcov{a}{\tensor*{\tau}{^{a_{1}\dots a_{n}}_{b_{1}\dots b_{m}}}} = \del{a}{\tensor*{\tau}{^{a_{1}\dots a_{n}}_{b_{1}\dots b_{m}}}} + \sum\limits_{k = 1}^{n}\tensor*{\tau}{^{c_{1}\dots c_{n}}_{b_{1}\dots b_{m}}}\chris{a_{k}}{a}{c_{k}} - \sum\limits_{l = 1}^{m}\tensor*{\tau}{^{a_{1}\dots a_{n}}_{d_{1}\dots d_{m}}}\chris{d_{l}}{a}{b_{l}}.
\end{align*}

\paragraph{The Gradient of a Tensor Field}
Based on the above, the directional derivative of a tensor field is
\begin{align*}
	\grad_{\vb{n}}{\tau} = n^{a}\del{a}{\tau}.
\end{align*}
This is equal to the contraction of $\vb{n}$ with the object
\begin{align*}
	\grad{\tau} = \db{c}\otimes\del{c}{\tau},
\end{align*}
which is defined as the gradient of $\tau$. More explicitly, we have
\begin{align*}
	\grad{\tau} = \tensor*{e}{^{cb_{1}\dots b_{m}}_{a_{1}\dots a_{n}}}\dcov{c}{\tensor*{\tau}{^{a_{1}\dots a_{n}}_{b_{1}\dots b_{m}}}}.
\end{align*}
It has the interesting property of having a rank one higher than $\tau$, which matches what we know - for instance, the gradient transforms a scalar into a vector.

\paragraph{Christoffel Symbols and the Metric}
The derivatives of the metric tensor are given by
\begin{align*}
	\del{c}{g_{ab}} = \tb{a}\cdot\del{c}{\tb{b}} + \tb{b}\cdot\del{c}{\tb{a}} = \tb{a}\cdot\chris{d}{c}{b}\tb{d} + \tb{b}\cdot\chris{d}{c}{a}\tb{d} = \chris{d}{c}{b}g_{ad} + \chris{d}{c}{a}g_{bd}.
\end{align*}
Multiplying by $g^{ea}$ and summing over $a$ yields
\begin{align*}
	g^{ea}\del{c}{g_{ab}} = \chris{d}{c}{b}g_{ad}g^{ea} + \chris{d}{c}{a}g_{bd}g^{ea} = \chris{d}{c}{b}g_{da}g^{ae} + \chris{d}{c}{a}g_{bd}g^{ea} = \chris{e}{c}{b} + \chris{d}{c}{a}g_{bd}g^{ea}.
\end{align*}
The hope is that this can be used to obtain an expression for the Christoffel symbols. To try to do that, we will compare this to the expression obtained by switching $c$ and $b$. This expression is
\begin{align*}
	g^{ea}\del{b}{g_{ac}} = \chris{e}{b}{c} + \chris{d}{b}{a}g_{cd}g^{ea} = \chris{e}{c}{b} + \chris{d}{b}{a}g_{cd}g^{ea},
\end{align*}
yielding
\begin{align*}
	\chris{e}{c}{b} &= \frac{1}{2}\left(g^{ea}\del{c}{g_{ab}} + g^{ea}\del{b}{g_{ac}} - \chris{d}{c}{a}g_{bd}g^{ea} - \chris{d}{b}{a}g_{cd}g^{ea}\right) \\
	                &= \frac{1}{2}g^{ea}\left(\del{c}{g_{ab}} + \del{b}{g_{ac}} - \chris{d}{a}{c}g_{bd} - \chris{d}{a}{c}g_{cd}\right) \\
	                &= \frac{1}{2}g^{ea}\left(\del{c}{g_{ab}} + \del{b}{g_{ac}} - \del{a}{g_{bc}}\right).
\end{align*}

\paragraph{Curve Length}
Consider some curve parametrized by $t$, and let $\dot{\vb{\gamma}}$ denote its tangent. The curve length is given by
\begin{align*}
	\dd{s}^{2} = \dd{\vb{x}}\cdot\dd{\vb{x}} = g(\dot{\vb{\gamma}}, \dot{\vb{\gamma}})\dd{t}^{2} = g_{ab}\dot{\chi^{a}}\dot{\chi^{b}}\dd{t}^{2}.
\end{align*}
The curve length is now given by
\begin{align*}
	L = \integ{}{}{t}{\sqrt{g_{ab}\dot{\chi^{a}}\dot{\chi^{b}}}}.
\end{align*}

\paragraph{Geodesics}
A geodesic is a curve that extremises the curve length between two points. From variational calculus, it is known that such curves satisfy the Euler-Lagrange equations, and we would like a differential equation that describes such a curve. By defining $\mathcal{L} = \sqrt{g_{ab}\dot{\chi}^{a}\dot{\chi}^{b}}$, the Euler-Lagrange equations for the curve length becomes
\begin{align*}
	\del{\chi^{a}}{\mathcal{L}} - \dv{t}\del{\dot{\chi}^{a}}{\mathcal{L}} = 0.
\end{align*}
The Euler-Lagrange equation thus becomes
\begin{align*}
	&\frac{1}{2\mathcal{L}}\dot{\chi}^{b}\dot{\chi}^{c}\del{a}{g_{bc}} - \dv{t}\left(\frac{1}{2\mathcal{L}}g_{bc}(\dot{\chi}^{b}\kdelta{a}{c} + \dot{\chi}^{c}\kdelta{a}{b})\right) = 0, \\
	&\frac{1}{2\mathcal{L}}\dot{\chi}^{b}\dot{\chi}^{c}\del{a}{g_{bc}} - \dv{t}\left(\frac{1}{2\mathcal{L}}(g_{ba}\dot{\chi}^{b} + g_{ac}\dot{\chi}^{c}\right) = 0, \\
	&\frac{1}{2\mathcal{L}}\dot{\chi}^{b}\dot{\chi}^{c}\del{a}{g_{bc}} - \dv{t}\left(\frac{1}{\mathcal{L}}g_{ac}\dot{\chi}^{c}\right) = 0.
\end{align*}
Expanding the time derivative yields
\begin{align*}
	\frac{1}{2\mathcal{L}}\dot{\chi}^{b}\dot{\chi}^{c}\del{a}{g_{bc}} - \frac{1}{\mathcal{L}}\dv{t}(g_{ac}\dot{\chi}^{c}) + g_{ac}\dot{\chi}^{c}\frac{1}{\mathcal{L}^{2}}\dv{\lag}{t} = \frac{1}{2\mathcal{L}}\dot{\chi}^{b}\dot{\chi}^{c}\del{a}{g_{bc}} - \frac{1}{\lag}\dv{t}(g_{ac}\dot{\chi}^{c}) + \frac{1}{\lag}g_{ac}\dot{\chi}^{c}\dv{\ln{\lag}}{t} = 0.
\end{align*}
The curve may be reparametrized such that $\lag$ is equal to $1$ everywhere, yielding
\begin{align*}
	\frac{1}{2\mathcal{L}}\left(\dot{\chi}^{a}\dot{\chi}^{b}\del{c}{g_{ab}} - \dv{t}(2\dot{\chi}^{a}g_{ac})\right) = 0.
\end{align*}
We note that the expression in the paranthesis is the Euler-Lagrange equation for the integral of $\mathcal{L}^{2}$, a nice fact for the future. Expanding the derivative yields
\begin{align*}
	\frac{1}{\mathcal{L}}\left(\frac{1}{2}\dot{\chi}^{a}\dot{\chi}^{b}\del{c}{g_{ab}} - g_{ac}\ddot{\chi}^{a} - \dot{\chi}^{a}\dot{\chi}^{b}\del{b}{g_{ac}}\right) = 0.
\end{align*}
To remove the metric from the second derivative, we multiply by $-g^{cd}\mathcal{L}$ to obtain
\begin{align*}
	&g_{ac}g^{cd}\ddot{\chi}^{a} + \frac{1}{2}\dot{\chi}^{a}\dot{\chi}^{b}g^{cd}(2\del{b}{g_{ac}} - \del{c}{g_{ab}}) = 0, \\
	&g_{ac}g^{cd}\ddot{\chi}^{a} + \frac{1}{2}\dot{\chi}^{a}\dot{\chi}^{b}g^{cd}(\del{b}{g_{ac}} + \del{a}{g_{bc}} - \del{c}{g_{ab}}) = 0, \\
	&\ddot{\chi}^{d} + \frac{1}{2}\dot{\chi}^{a}\dot{\chi}^{b}g^{cd}(\del{b}{g_{ac}} + \del{a}{g_{bc}} - \del{c}{g_{ab}}) = 0.
\end{align*}
This is the geodesic equation. It may alternatively be written in terms of the Christoffel symbols as
\begin{align*}
	\ddot{\chi}^{d} + \chris{d}{a}{b}\dot{\chi}^{a}\dot{\chi}^{b} = 0.
\end{align*}

\paragraph{Christoffel Symbols and the Geodesic Equation}
Consider a straight line with a tangent vector of constant magnitude. In euclidean space, this is a geodesic. This curve satisfies
\begin{align*}
	\vb{0} = \dv{\dot{\vb*{\gamma}}}{t} = (\dot{\vb*{\gamma}}\cdot\grad)\dot{\vb*{\gamma}} = \dot{\chi}^{a}\del{a}{\dot{\vb*{\gamma}}} = \dot{\chi}^{a}(\dcov{a}{\dot{\chi}^{d}})\tb{d} = (\dot{\chi}^{a}\del{a}{\dot{\chi}^{d}} + \dot{\chi}^{a}\dot{\chi}^{c}\chris{d}{a}{c})\tb{d}.
\end{align*}
Comparing this to the geodesic equation yields
\begin{align*}
	\chris{d}{a}{b} = \frac{1}{2}g^{dc}(\del{b}{g_{ac}} + \del{a}{g_{cb}} - \del{c}{g_{ab}}).
\end{align*}
A better approach would have been to go through the derivation of the geodesic equation again, identifying the Christoffel symbols as you go, but I am not sure if that is what I did in the previous paragraph. In any case we have already obtained this result.

\section{Classical mechanics}
In classical mechanics, configuration space is the space of all possible configurations of a system. We can impose coordinates $\chi^{a}$ on this space in order to use what we know.

\paragraph{Kinetic energy}
Kinetic energy is defined by a rank $2$ tensor as
\begin{align*}
	E_{\text{k}} = \frac{1}{2}T_{ab}\dot{\chi}^{a}\dot{\chi}^{b},
\end{align*}
where the dot now really represents the time derivative.

%Can be used to show H = T + V

\paragraph{Hamilton's principle}
We define the Lagrangian of a system as $\lag = E_{\text{k}} - V$, where $V$ is the potential energy and taken to be a function on coordinate space. The action of a system over time is defined as
\begin{align*}
	S = \integ{}{}{t}{\lag}.
\end{align*}
Hamilton's principle states that for the motion of the system in configuration space, $\var{S} = 0$. This can be expressed as
\begin{align*}
	\var{S} = \integ{}{}{t}{\var{\lag}} = \integ{}{}{t}{\left(\del{\chi^{a}}{\lag} - \dv{t}\del{\dot{\chi}^{a}}{\lag}\right)\var{\chi^{a}}} = 0.
\end{align*}

\paragraph{The kinetic metric}
Consider a system with no potential energy. The Lagrangian simply becomes $\lag = \frac{1}{2}T_{ab}\dot{\chi}^{a}\dot{\chi}^{b}$. This is very similar to the integral of curve length (or, rather its square, the extremum of which was noted to be the same), except $g_{ab}$ has been replaced by $T_{ab}$. This inspires us to define $T_{ab}$ as the kinetic metric, with corresponding Christoffel symbols.

\paragraph{Motion of a classical system}
By defining $a^{b} = \dot{\chi}^{a}\dcov{a}{\dot{\chi}^{b}}$, the previous work leads us to a system with no potential satisfying $a^{b} = \ddot{\chi}^{b} + \chris{b}{a}{c}\dot{\chi}^{a}\dot{\chi}^{c} = 0$. In other words, a system with no potential moves along the geodesics of the kinetic metric.

For a system with a potential, only the $\del{\chi^{a}}{\lag}$ term is affected, and
\begin{align*}
	a^{b} = - T^{ba}\del{a}{V} = T^{ba}F,
\end{align*}
which is a generalization of Newton's second law.

\paragraph{Noether's theorem}
Noether's theorem relates symmetries of physical systems to conservation laws.

What is a symmetry, then? Consider a one-parameter transformation $t\to\tau(t, s),\ q^{a}\to Q^{a}(q, s)$, where $s$ is the parameter with respect to which the system is transformed, such that $\tau(t, 0) = t,\ Q^{a}(q, s) = q^{a}$ and for small $s = \varepsilon$ that $t\to t + \varepsilon\var{t},\ q^{a}\to q^{a} + \varepsilon\var{q^{a}}$. This is assumed to be normalized such that $\var{t}$ is either $0$ or $1$. How? Don't ask. A quasi-symmetry of a system with Lagrangian $\lag$ is a transformation such that
\begin{align*}
	\varepsilon\var{\lag} = \lag(Q, \dot{Q}, \tau) - \lag(q, \dot{q}, t) = \varepsilon\dv{F}{t}
\end{align*}
for some $F$. The variation of the Lagrangian can be written as
\begin{align*}
	\var{\lag} = \del{q^{a}}{\lag}\var{q^{a}} + \del{\dot{q}^{a}}{\lag}\var{\dot{q}^{a}} + \del{t}{\lag}\var{t}.
\end{align*}
The total time derivative of the Lagrangian is given by
\begin{align*}
	\dv{\lag}{t} = \del{t}{\lag} + \del{q^{a}}{\lag}\dot{q}^{a} + \del{\dot{q}^{a}}{\lag}\ddot{q}^{a},
\end{align*}
which yields
\begin{align*}
	\var{\lag} = \del{q^{a}}{\lag}(\var{q^{a}} - \dot{q}^{a}\var{t}) + \del{\dot{q}^{a}}{\lag}(\var{\dot{q}^{a}} - \ddot{q}^{a}\var{t}) + \dv{\lag}{t}\var{t}.
\end{align*}
The equations of motion are $\del{q^{a}}{\lag} = \dv{t}\del{\dot{q}^{a}}{\lag}$. For a set of coordinates that satisfy this - a so-called on-shell solution - we have
\begin{align*}
	\var{\lag} &= \dv{t}\del{\dot{q}^{a}}{\lag}(\var{q^{a}} - \dot{q}^{a}\var{t}) + \del{\dot{q}^{a}}{\lag}(\var{\dot{q}^{a}} - \ddot{q}^{a}\var{t}) + \dv{\lag}{t}\var{t} \\
	           &= \dv{t}\left(\del{\dot{q}^{a}}{\lag}(\var{q^{a}} - \dot{q}^{a}\var{t}) + \lag\var{t}\right).
\end{align*}
If the transformation is a quasi-symmetry of the system, then this is equal to a total time derivative of $F$, and the quantity
\begin{align*}
	J = F - \del{\dot{q}^{a}}{\lag}\var{q^{a}} + (\dot{q}^{a}\del{\dot{q}^{a}}{\lag} - \lag)\var{t}
\end{align*}
thus satisfies $\dv{J}{t} = 0$. We can introduce the general momenta $p_{a} = \del{\dot{q}^{a}}{\lag}$ and the Hamiltonian $\ham = p_{a}\dot{q}^{a} - \lag$ to write
\begin{align*}
	J = F - p_{a}\var{q^{a}} + \ham\var{t}.
\end{align*}

We arrive at the conclusion that $J$ is a conserved quantity under a quasi-symmetry of the system. Identifying the conservation laws of a system is thus a matter of identifying the quasi-symmetries of a system and computing $J$ under that transformation.

\example{A free particle in space}
Consider a free particle in space. Its Lagrangian is given by $\lag = \frac{1}{2}m\dot{\vb{x}}^{2}$, and the variation of this is
\begin{align*}
	\var{\lag} = m\dot{\vb{x}}\cdot\var{\dot{\vb{x}}}.
\end{align*}
Its general momentum is
\begin{align*}
	\vb{p} = \del{\dot{\vb{x}}}{\lag} = m\dot{\vb{x}}.
\end{align*}
The Hamiltonian is
\begin{align*}
	\ham = \vb{p}\cdot\dot{\vb{x}} - \lag = \frac{1}{2}m\dot{\vb{x}}^{2}.
\end{align*}
We now want to identify quasi-symmetries of the system that make the variation of the Lagrangian either zero or the time derivative of some quantity. A key idea here is that we are only allowed to change the variations (or so I think).

A first attempt is keeping $\var{\vb{x}}$ constant and not varying thiime(a spatial translation), which implies $\var{\dot{\vb{x}}} = \vb{0}$ and $\var{\lag} = 0$. This implies that $F$ is constant. The conserved quantity is thus
\begin{align*}
	J = F - \vb{p}\cdot\var{\vb{x}} = F - \vb{p}\cdot\vb{c},
\end{align*}
i.e. the momentum of the system is conserved. We also note that the constant $F$ in this case is arbitrary, and we might as well have set it to $0$. This will be the case at least sometimes.

A second attempt is varying time, i.e. $\var{t} = 1$, but keeping the coordinates fixed, i.e. $\var{\vb{x}} = 0$ (a time translation). This yields $\var{\dot{\vb{x}}} = \vb{0}$ and $\var{\lag} = 0$. Once again $F$ is constant and taken to be zero, and the conserved quantity is thus $J = H$, i.e. the Hamiltonian of the system is conserved.

A third attempt is to somehow make the scalar product in the variation of the Lagrangian zero, without varying time. An option is $\var{\vb{x}} = \vb{\omega}\times\vb{x}$, where $\vb{\omega}$ is a constant vector. This yields $\var{\dot{\vb{x}}} = \vb{\omega}\times\dot{\vb{x}}$ and $\var{\lag} = 0$. The conserved quantity is thus
\begin{align*}
	J &= -\vb{p}\cdot(\vb{\omega}\times\vb{x}) \\
	  &= -\vb{\omega}\cdot(\vb{x}\times\vb{p}).
\end{align*}
Since $\vb{\omega}$ is constant, that means that $\vb{x}\times\vb{p}$, i.e. the angular momentum, is conserved.

\paragraph{Hamiltonian mechanics}
Hamiltonian mechanics starts with trying to transform $\lag(q, \dot{q}, t)$ to $\ham(q, p, t)$. To start with this, we reintroduce the general momenta $p_{i} = \del{\dot{q}_{i}}{\lag}$ and define the Hamiltonian
\begin{align*}
	\ham = p_{i}\dot{q}_{i} - \lag.
\end{align*}
This is a Legendre transform of the Hamiltonian, which is discussed below. In Lagrangian mechanics, we considered paths in configuration space. In Hamiltonian mechanics, we instead consider paths in phase space, i.e. a space where the points are $(q, t)$. In this space, paths do not intersect as the system is deterministic. The new equations of motion in this formalism is
\begin{align*}
	\dot{p}_{i} = -\del{q_{i}}{\ham},\ \dot{q}_{i} = \del{p_{i}}{\ham}.
\end{align*}

Paths in phase space are periodic for integrable systems and fill out the accessible parts of phase space for chaotic systems.

\paragraph{Legendre transforms}
To illustrate the Legendre transform, consider a function $f(x, y)$ and $g(x, y, u) = ux - f(x, y)$. Its total derivative is given by
\begin{align*}
	\dd{g} = u\dd{x} + x\dd{u} - \del{x}{f}\dd{x} - \del{y}{f}\dd{y}.
\end{align*}
By choosing $u = \del{x}{f}$, we obtain
\begin{align*}
	\dd{g} = x\dd{u} - \del{y}{f}\dd{y},
\end{align*}
implying that $g$ is only a function of $u$ and $y$. To obtain $g$, invert the definition of $u$ to obtain $x(u, y)$.

\paragraph{Equations of motion}
The variation of the Hamiltonian is given by
\begin{align*}
	\dd{\ham} &= \dot{q}_{i}\dd{p_{i}} + p_{i}\dd{\dot{q}_{i}} - \del{q_{i}}{\lag}\dd{q_{i}} - \del{\dot{q}_{i}}{\lag}\dd{\dot{q}_{i}} - \del{t}{\lag}\dd{t} \\
	          &= \dot{q}_{i}\dd{p_{i}} - \del{q_{i}}{\lag}\dd{q_{i}} - \del{t}{\lag}\dd{t}.
\end{align*}
Combining this with the equations of motion
\begin{align*}
	\del{q_{i}}{\lag} = \dv{t}\left(\del{\dot{q}_{i}}{\lag}\right) = \dv{t}\left(p_{i}\right) = \dot{p}_{i}
\end{align*}
yields
\begin{align*}
	\dot{q}_{i} = \del{p_{i}}{\ham},\ \dot{p}_{i} = -\del{q_{i}}{\ham},\ \del{t}{\lag} = -\del{t}{\ham}.
\end{align*}

We also have
\begin{align*}
	\dv{\ham}{t} &= \del{q_{i}}{\ham}\dot{q}_{i} + \del{p_{i}}{\ham}\dot{p}_{i} + del{t}{\ham} \\
	             &= \del{t}{\ham},
\end{align*}
and so the Hamiltonian is conserved if it has no explicit time dependance.

\paragraph{Liouville's theorem}
As paths in phase space do not cross, we can think of the time evolution of a system as a flow in phase space. The volume element is $\dd{V} = \dd{q}\dd{p}$. Liouville's theorem states that flow in phase space is incompressible.

To show this, consider the state at some point in time and after some infinitesimal time $\dd{t}$. Denote the point in phase space at the start as $(q, p)$ and after $\dd{t}$ as $(q', p').$ To first order in time we have
\begin{align*}
	q_{i}' = q_{i} + \dot{q}_{i}\dd{t} = q_{i} + \del{p_{i}}{\ham}\dd{t},\ p_{i}' = p_{i} + \dot{p}_{i}\dd{t} = p_{i} - \del{q_{i}}{\ham}\dd{t}.
\end{align*}
The volume element is given by
\begin{align*}
	\dd{V}' &= \left(\dd{q} +  \left(\del{q}{\del{p}{\ham}}\dd{q} + \del[2]{p}{\ham}\dd{p}\right)\dd{t}\right)\left(\dd{p} -  \left(\del[2]{q}{\ham}\dd{q} + \del{p}{\del{q}{\ham}}\dd{p}\right)\dd{t}\right) \\
	        &= \dd{q}\dd{p} + \left(-\dd{q}\left(\del[2]{q}{\ham}\dd{q} + \del{p}{\del{q}{\ham}}\dd{p}\right) + \dd{p\left(\del{q}{\del{p}{\ham}}\dd{q} + \del[2]{p}{\ham}\dd{p}\right)}\right)\dd{t} \\
	        &= \dd{q}\dd{p} + \left(-\del[2]{q}{\ham}(\dd{q})^{2} + (\del{q}{\del{p}{\ham}} - \del{p}{\del{q}{\ham}})\dd{q}\dd{p} + \del[2]{p}{\ham}(\dd{p})^{2}\right)\dd{t}.
\end{align*}
The equations of motion imply that the terms containing two consecutive derivatives with respect to the same variable are equal to zero. Assuming the Hamiltonian to be sufficiently smooth, the cross-derivatives are equal. This implies
\begin{align*}
	\dd{V}' = \dd{V}.
\end{align*}

\paragraph{Poisson brackets}
Consider a function $f(q, p, t)$. Its time derivative is given by
\begin{align*}
	\dv{f}{t} &= \del{q_{i}}{f}\dot{q}_{i} + \del{p_{i}}{f}\dot{p}_{i} + \del{t}{f} \\
	          &= \del{q_{i}}{f}\del{p_{i}}{\ham} - \del{p_{i}}{f}\del{q_{i}}{\ham} + \del{t}{f} \\
	          &= \pob{f}{\ham} + \del{t}{f},
\end{align*}
where we now have defined the Poisson bracket. It is bilinear and satisfies
\begin{align*}
	\pob{f}{g}          &= -\pob{g}{f}, \\
	\pob{fg}{h}         &= f\pob{g}{h} + \pob{f}{h}g, \\
	\pob{f}{\pob{g}{h}} &+ \pob{g}{\pob{h}{f}} + \pob{h}{\pob{f}{g}} = 0.
\end{align*}
The expression above implies that if $\pob{f}{\ham} = 0$ and $f$ does not depend explicitly on time, then it is a constant of motion.

\paragraph{Restatement of Liouville's theorem}
We define $\rho(q, p, t)$ as the probability that a particle is close to $(q, p)$. Alternatively, for a large number of particles, we can define it as the number of particles close to $(q, p)$.

We have
\begin{align*}
	\dv{\rho}{t} = 0,
\end{align*}
implying
\begin{align*}
	\del{t}{\rho} = -\pob{\rho}{\ham}.
\end{align*}
This is an equivalent statement of Liouville's theorem.

\paragraph{Canonical transformation}
A canonical transformation is a transformation $(\vb{q}, \vb{p})\to (\vb{Q}, \vb{P})$ such that the Hamiltonian $H$ expressed in these new coordinates also satisfies Hamilton's equations in the new coordinates, i.e.
\begin{align*}
	\dot{Q}_{i} = \del{P_{i}}{H},\ \dot{P}_{i} = -\del{Q_{i}}{\ham}.
\end{align*}

\paragraph{Canonical transformations and Poisson brackets}
It turns out that a transformation in phase space is canonical if and only if they preserve the following equations:
\begin{align*}
	\pob{q_{i}}{q_{j}} = \pob{p_{i}}{p_{j}} = 0,\ \pob{q_{i}}{p_{j}} = \delta_{ij}.
\end{align*}

To show this, consider some point $x$ in phase space. Under the canonical transformation, it transforms to $y$. Suppose now that the relation
\begin{align*}
	\dot{x}_{i} = J_{ij}\del{x_{j}}{\ham}
\end{align*}
to be true. According to the equations of motion, this would imply
\begin{align*}
	J_{ij} = 
	\begin{cases}
		 1, &j = i + n, i = 1, 2, \dots, n, \\
		-1, &j = i - n, i = n + 1, \dots, 2n.
	\end{cases}
\end{align*}
If the transformation is canonical, then the same should be true after the transformation. On the other hand, the chain rule yields
\begin{align*}
	\dot{y}_{i} = \del{x_{j}}{y_{i}}J_{ik}\del{x_{k}}{y_{m}}\del{y_{m}}{H}.
\end{align*}
Comparing this with the Jacobian $\mathcal{J}$ yields
\begin{align*}
	J = \mathcal{J}J\mathcal{J}^{T}.
\end{align*}

\section{Relativity}

\paragraph{The Galilean group}
The Galilean group is the group of transformations between frames of reference under which the laws of physics are invariant. It consists of:
\begin{itemize}
	\item Translations by a constant vector.
	\item Rotations of the coordinate axes.
	\item Boosts, i.e. translating the coordinates along a line with a constants speed.
\end{itemize}
It is based on a concept of absolute time. It turns out that the arc element $\dd[2]{s} = \dd{x}^{2} + \dd{y}^{2} + \dd{z}^{2}$ at a given time is preserved under all of these transformations.

The invariance of the laws of physics under these transformations corresponds to there being no special position or direction in the universe, and no special velocity. At least two of these claims have thus far not been disproved.

\paragraph{The emergence of special relativity}
It turned out that Maxwell's equations were not invariant under Galilean transformations. 

DISCLAIMER: THIS IS SERIOUS HEAD CANON! The issue with Maxwell's equations is that they predict that electromagnetic waves travel at speed $c$. This should of course be the same in all frames of reference, according to Galilean relativity. No other wave phenomena had previously raised an issue as they travel through a medium. This medium naturally defines a certain frame of reference in which the physics are special, namely the rest frame of the medium. Only by transporting the medium with you when doing the boost will you reobtain the same physics. A natural idea to follow from this is that electromagnetic waves travel in a medium, so physicists started searching for it. After having found no evidence of its existence, most notably through Michelson and Morley's experiment, the conclusion was that there was no medium in which electromagnetic waves travelled, and thus the speed of light had to be one of the invariant properties under transformation between inertial frames of reference.

The constancy of the speed of light implies that the infinitesimal quantity
\begin{align*}
	\dd[2]{s} = c^{2}\dd{t}^{2} - \dd{x}^{2} - \dd{y}^{2} - \dd{z}^{2}
\end{align*}
is constant. We will soon replace the elements of the Galilean group with elements that keep this quantity.

\paragraph{Four-vectors and the Minkowski metric}
We now define the four-vector $x^{\mu}$, where $\mu = 0, 1, 2, 3$ and $\mu = 0$ corresponds to $ct$ and an inner product with the metric $\eta$. The metric is diagonal with $\eta_{11} = 1$ and $\eta_{ii} = -1$ otherwise. This is called the Minkowski metric.

\paragraph{Lorentz transformations}
We are now interested in transformations that preserve the new arc length. If the transformation is on the form $\vb{x}' = \Lambda\vb{x}$. Computing the arc length yields
\begin{align*}
	\Lambda^{T}\nu\Lambda = \nu.
\end{align*}
The transformations satisfying this constitute the Lorentz group, or $\O{1}{3}$. Computing the determinant on either side yields $\det(\Lambda)^{2} = 1$. The subgroup with determinant $1$ (which preserve the direction of time) is the special Lorentz group \SO{1}{3}.

The equation defining the group elements is symmetric, which imposes constraints on the elements of the matrix. The matrix has $16$ elements, so the defining equation places $10$ constraints on the coefficients of $\Lambda$. With $10$ equations and $16$ unknowns, we expect $6$ linearly independent solutions.

The first three are rotations of space, written in block diagonal form as
\begin{align*}
	\Lambda =
	\mqty[
		1 & 0 \\
		0 & R
	].
\end{align*}
Inserted into the defining equation, we obtain
\begin{align*}
	\mqty[
		1 & 0 \\
		0 & -R^{T}R
	] =
	\mqty[
		1 & 0 \\
		0 & -1
	].
\end{align*}
This yields the familiar requirement $R^{T}R = 1$.

The remaining three transforms are Lorentz boosts corresponding to each axis. This can be shown explicitly for $x$, and a permutation of coordinates will yield the same result for a boost along any other axis. We believe it to be reasonable that such a transformation should not affect any other coordinates than the boosted coordinate and time. This means that the matrix will be on the form
\begin{align*}
	\Lambda =
	\mqty[
		\Lambda_{x} & 0 \\
		0           & 1
	].
\end{align*}
The defining equation now yields
\begin{align*}
	\Lambda_{x}^{T}\sigma_{z}\Lambda_{x} = \sigma_{z}, \
	\sigma_{z} =
	\mqty[
		1 & 0 \\
		0 & -1
	].
\end{align*}
We expand $\Lambda_{x}$ around the identity as $\Lambda_{x} = 1 - \phi K$, where $\phi$ is independent of both coordinates and time. Inserting this into the above equation yields
\begin{align*}
	(1 - \phi K^{T})\sigma_{z}(1 - \phi K) = \sigma_{z}.
\end{align*}
Expanding the bracket yields
\begin{align*}
	(1 - \phi K^{T})(\sigma_{z} - \phi\sigma_{z}K)                                 &= \sigma_{z}, \\
	\sigma_{z} - \phi\sigma_{z}K - \phi K^{T}\sigma_{z} + \phi^{2}K^{T}\sigma_{z}K &= \sigma_{z}.
\end{align*}
Ignoring higher-order terms yields
\begin{align*}
	\sigma_{z}K + K^{T}\sigma_{z} &= 0, \\
	(\sigma_{z}K)^{T}             &= -\sigma_{z}K.
\end{align*}
The generator $K$ must therefore be
\begin{align*}
	K = \sigma_{x} =
	\mqty[
		0 & 1 \\
		1 & 0
	].
\end{align*}
Now a transformation correpsonding to an arbitrary $\phi$ can be written as
\begin{align*}
	\Lambda_{x} = e^{-\phi K} = \cosh{\phi} - \sinh{\phi}\sigma_{x},
\end{align*}
where the last equality comes from writing the exponential as an infinite series and using the fact that $K^{2} = 1$. Perhaps someone should do this explicitly.

To identify the transformation more exactly, we consider two frames of reference in which the origins coincide at $t = 0$. Under such a transformation, we require that $(ct, vt)$ map to $(ct', 0)$. This yields
\begin{align*}
	-\sinh{\phi}ct + \cosh{\phi}vt = 0.
\end{align*}
Defining $\gamma = \cosh{\phi}$ and applying hyperbolic identities yields
\begin{align*}
	-\sqrt{\gamma^{2} - 1}ct + \gamma vt &= ct\left(\frac{v}{c}\gamma - \sqrt{\gamma^{2} - 1}\right) = 0, \\
	\gamma                               &= \sqrt{\frac{1}{1 - \frac{v^{2}}{c^{2}}}}.
\end{align*}
The transformation can now be written as
\begin{align*}
	\Lambda_{x} = 
	\mqty[
		\gamma             & -\frac{v}{c}\gamma \\
		-\frac{v}{c}\gamma & \gamma
	].
\end{align*}
The total matrix for a boost along any other coordinate axis can be found by permuting the elements in the transformation matrix for the $x$ boost. This yields a basis of matrices, and a boost along an arbitrary direction can be found by taking linear combinations of these.

\paragraph{Adding velocities}
The product of two boosts is another boost. For two boosts along the same direction, we obtain $\Lambda(\phi_{1})\Lambda(\phi_{2}) = \Lambda(\phi_{1} + \phi_{2})$. This can be used to show that the total boosted velocity is
\begin{align*}
	v_{3} = \frac{v_{1} + v_{2}}{1 + \frac{v_{1}v_{2}}{c^{2}}}.
\end{align*}

\paragraph{Proper time}
Consider a particle at the origin in its rest frame. The arc length becomes $\dd{s}^{2} = c^{2}\dd{t}^{2}$. As the left-hand side is invariant, so must the right-hand side be. This makes it natural to define the proper time
\begin{align*}
	\dd{\tau} = \frac{1}{c}\dd{s}.
\end{align*}

\paragraph{Relativistic kinematics}
Suppose that you wanted to define $\vb{u} = \dv{\vb{x}}{t}$ as the spatial part of velocity. Well, too bad, cause time transforms under a Lorentz transformation, so this thing will not behave linearly under Lorentz transformation. We need a better alternative.

Consider instead the rest frame $S'$ of the particle, where it is resting at the origin. Its trajectory in the original inertial frame can be parametrized in terms of the proper time. Along a small trajectory we have
\begin{align*}
	\dd{\tau} = \frac{1}{c}\dd{s} = \frac{1}{c}\sqrt{c^{2}\dd{t}^{2} - \dd{\vb{x}}^{2}} = \dd{t}\sqrt{1 - \left(\dv{\vb{x}}{t}\right)^{2}} \implies \dv{t}{\tau} = \gamma.
\end{align*}
We can now define the four velocity
\begin{align*}
	U = \dv{x}{\tau} = \dv{\tau}
	\mqty[
		ct \\
		\vb{x}
	]
	=
	\mqty[
		c\dv{t}{\tau} \\
		\dv{\vb{x}}{\tau}
	]
	= \gamma
	\mqty[
		c \\
		\vb{u}
	]
\end{align*}
where $\vb{u} = \dv{\vb{x}}{t}$. This quantity transforms like a four-vector, and is therefore the four-velocity.

\section{Classical field theory}

Classical field theory can be considered a limit of classical dynamics when the number of particles is infinite. The system obtains new ``coordinates'' $\phi$, which are functions of poisition and time. Summations over coordinates now become integrals over space.

\paragraph{Lagrangian dynamics}
The Lagrangian in a field theory now becomes
\begin{align*}
	L = \integ[D]{}{}{\vb{r}}{\lag}
\end{align*}
where $\lag$ is the Lagrangian density. From this we can obtain the action, and extremize it to obtain the equations for the time evolution of the system.

\paragraph{Solving models}
To solve models, we usually allow for periodic boundary conditions. The field is then expanded as a Fourier series, or a Fourier transform in the limit of a large domain or small lattice constant. We will in any case find that the system is compact in Fourier space, i.e. there are only non-zero contributions within some compact region.

\paragraph{Nöether's theorem}
In this context Nöether's theorem states that symmetries of a system are associated with conservative current. In field theory, a symmetry is a transformation $\phi\to\phi_{a}$, where $a$ is some continuous transformation parameter, such that for the quantity
\begin{align*}
	\dv{\lag}{a} = \dv{V^{\mu}}{x^{\mu}}
\end{align*}
there are quantities $j^{\mu}$ such that
\begin{align*}
	\dv{j^{\mu}}{x^{\mu}} = 0.
\end{align*}

\end{document}
