\section{Integraler och derivator}

\paragraph{Linjeintegraler}
En linjeintegral skrivs på formen
\begin{align*}
	\vinteg{C}{\vb{v}}{\vb{r}}.
\end{align*}
Det representerar hur mycket av ett vektorfält som är parallellt med en bana i rummet. Om det låter oklart, tänk att vektorfältet $\vb{v}$ puttar på en partikel som rär sig längs med banan $C$.

\paragraph{Rotation}
Från en linjeintegral kan rotationen definieras som
\begin{align*}
	\curl{\vb{v}}\cdot\vb{n} = \lim\limits_{A\to 0}\frac{1}{A}\vinteg{C}{\vb{v}}{\vb{r}},
\end{align*}
där $A$ är arean som omslutas av kurvan $C$ och $\vb{n}$ är normal på $C$. Denna tolkas fysikalisk som tätheten av virvlar i fältet $\vb{v}$ som roterar normalt på $\vb{n}$.

\paragraph{Flödesintegraler}
En flödesintegral skrivs på formen
\begin{align*}
	\vinteg{S}{\vb{v}}{\vb{S}}.
\end{align*}
Den representerar hur mycket av ett vektorfält som flöder genom ytan $S$.

\paragraph{Divergens}
Från en flödesintegral kan divergensen definieras som
\begin{align*}
	\div{\vb{v}} = \lim\limits_{V\to 0}\frac{1}{V}\vinteg{V}{\vb{v}}{\vb{S}},
\end{align*}
där $V$ är volymen som omslutas av ytan $S$. Denna tolkas fysikalisk som tätheten av källor till fältet $\vb{v}$.

\paragraph{Potentialer}
Potentialer förekommer i två former: skalärpotentialer och vektorpotentialer.

Ett vektorfält har ett skalärpotential om det kan skrivas som $\grad{f}$ för någon funktion $f$, som då betecknas som potentialet. För sådana fält gäller att $\curl{\vb{v}} = \vb{0}$. Ett vektorfält har ett vektorpotentiale om det kan skrivas som $\curl{\vb{A}}$ för något vektorfält $\vb{A}$, som betecknas vektorpotentialet. För sådana fält gäller att $\div{\vb{v}} = 0$.

Om ett vektorfält kan skrivas som en derivata på några av dessa två sätten, är det ekvivalent med att fältet har en potential.

\paragraph{Laplaceoperatorn}
Vi definierar Laplaceoperatorn för skalärer och vektorer som
\begin{align*}
	\laplace{f}      &= \div{\grad{f}}, \\
	\laplace{\vb{v}} &= \grad{\div{\vb{v}}} - \curl{\curl{\vb{v}}}.
\end{align*}

\paragraph{Gauss' sats}
\begin{align*}
	\vinteg{\bound{V}}{\vb{v}}{\vb{S}} = \integ{V}{V}{\div{\vb{v}}}
\end{align*}

\paragraph{Stokes' sats}
\begin{align*}
	\vinteg{\bound{S}}{\vb{v}}{\vb{r}} = \vinteg{S}{\curl{\vb{v}}}{\vb{S}}
\end{align*}

\paragraph{Gauss' sats på universalform}
\begin{align*}
	\integ{\bound{V}}{S_i}{f} = \integ{V}{V}{\del{i}{f}}.
\end{align*}
$f$ kan nu vara vad som helst.

\paragraph{Stokes' sats på universalform}
\begin{align*}
	\integ{S}{S_i}{\leci{ijk}\del{j}{f}} = \integ{\bound{S}}{x_k}{f}.
\end{align*}
$f$ kan vara vad som helst.