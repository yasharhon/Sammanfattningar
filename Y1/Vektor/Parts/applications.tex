\section{Tillämpningar}

Denna sektionen kommer fokusera på tillämpningar av teorin som har diskuterats i kursen.

\subsection{Värme}
Värmemängden i kroppen $V$ fås vid en given tid $t$ som
\begin{align*}
	\integ{V}{V}{Q},
\end{align*}
där $Q$ är värmetätheten. Värmemängden per tid som flödar ut ur $V$ ges av
\begin{align*}
	\vinteg{\bound{V}}{\vb{j}}{S}
\end{align*}
där $\vb{j}$ är värmeströmtätheten. Denna följer Fouriers lag, som säjer att
\begin{align*}
	\vb{j} = -\lambda\grad{T}.
\end{align*}
Värmemängden per tid som genereras i $V$ ges av
\begin{align*}
	\integ{V}{V}{\kappa},
\end{align*}
där $\kappa$ är tätheten av värmekällor. Balansering av värme som genereras, finns och flödar ut ger
\begin{align*}
	\dv{t}\integ{V}{V}{Q} = \integ{V}{V}{\kappa} - \vinteg{\bound{V}}{\vb{j}}{S} = \integ{V}{V}{\kappa} - \integ{V}{V}{\div{\vb{j}}}.
\end{align*}
Vid att derivera under integraltecknet får man kontinuitetsekvationen
\begin{align*}
	\dv{Q}{t} + \div{\vb{j}} = \kappa
\end{align*}
för värmeflödet.

\subsection{Elektromagnetism}
Elektromagnetismen beskrivs av Maxwells ekvationer:
\begin{align*}
	\div{\vb{E}}  &= 4\pi\rho, \\
	\div{\vb{B}}  &= 0, \\
	\curl{\vb{E}} &= -\frac{1}{c}\dv{\vb{B}}{t}, \\
	\curl{\vb{B}} &= \frac{4\pi\vb{j}}{c} + \frac{1}{c}\dv{\vb{E}}{t}.
\end{align*}
Alternativt kan den skrivas som
\begin{align*}
	\del{\mu}{F^{mu\gamma}} = j^{\gamma},
\end{align*}
där $\vb{F}$ är den elektromagnetiska tensorn, men detta är överkurs.