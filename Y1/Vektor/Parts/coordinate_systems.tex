\section{Koordinatsystem}

Ett koordinatsystem kan tänkas på som en avbildning från $n$-dimensionella kartesiska koordinater till andra koordinater som beskrivs av parametrar $u_1, \dots, u_n$. Vi krävjer att avbildningen ska vara inverterbar.

\paragraph{Ortogonalitet}
Vi säjer att koordinatsystemet är ortogonalt om $\dv{\vb{r}}{u_i}\cdot\dv{\vb{r}}{u_j} = 0$ för alla $i\neq j$. En ekvivalent definition är att alla koordinatytor ska vara normala.

\paragraph{Skalfaktorer}
Koordinatsystemets skalfaktorer definieras som
\begin{align*}
	h_i = \abs{\dv{\vb{r}}{u_i}}.
\end{align*}

\paragraph{Enhetsvektorer}
Koordinatsystemets enhetsvektorer definieras som
\begin{align*}
	\vb{e}_{u_i} = \frac{1}{h_i}\del{u_i}{\vb{r}}
\end{align*}
utan summation.

\paragraph{Tillämpningar på linjeintegraler}
Med detta kan vi skriva
\begin{align*}
	\dd{\vb{r}} = h_i\vb{e}_{u_i}\dd{u_i}.
\end{align*}

\paragraph{Tillämpningar på ytintegraler}
Med detta kan vi skriva
\begin{align*}
	\dd{\vb{S}} = \pm\vb{e}_{u_3}hu_{1}\dd{u_{1}}hu_{2}\dd{u_{2}}.
\end{align*}

\paragraph{Tillämpningar på volymintegraler}
Med detta kan vi skriva
\begin{align*}
	\dd{V} = \dd^3{\vb{r}} = \pdv{(x, y, z)}{(u_1, u_2, u_3)}\dd{u_1}\dd{u_2}\dd{u_3} = \prod\limits_{i = 1}^{3}h_{i}\dd{u_{i}}.
\end{align*}

\paragraph{Gradient i godtyckligt koordinatsystem}
\begin{align*}
	[\grad{f}]_i = \frac{1}{h_i}\del{u_i}{f}.
\end{align*}

Vi kan även tillämpa detta för att få
\begin{align*}
	h_i = \frac{1}{\abs{\grad{u_i}}},\ \vb{e}_{u_i} = \frac{1}{\abs{\grad{u_i}}}\grad{u_i}.
\end{align*}

\paragraph{Divergens i godtyckligt koordinatsystem}
\begin{align*}
	\div{\vb{v}} = \frac{1}{h_1h_2h_3}(\del{u_1}{v_{u_1}h_2h_3} + \del{u_2}{h_1v_{u_2}h_3} + \del{u_3}{h_1h_2v_{u_3}}).
\end{align*}

\paragraph{Rotation i godtyckligt koordinatsystem}
\begin{align*}
	[\curl{\vb{v}}]_i = \frac{1}{h_jh_k}\leci{ijk}\del{u_j}{h_kA_k}.
\end{align*}

\paragraph{Allmäna koordinatsystem}
I allmäna koordinatsystem inför vi $\vb{r} = \vb{r}(u^1, \dots, u^n)$, med indexet som superskript i stället för subskript. Vi definierar då vektorerna
\begin{align*}
	\vb{E}_{i} &= \dv{\vb{r}}{u^i} = \del{i}{\vb{r}}, \\
	\vb{E}^{i} &= \grad{u^i},
\end{align*}
där dessa ej normeras. Det gäller att
\begin{align*}
	\vb{E}_{i} = h_i\vb{e}_{i},\ \vb{E}^{i} = \frac{1}{h_i}\vb{e}_{i}.
\end{align*}
Vi skriver då
\begin{align*}
	\vb{v} = v^i\vb{E}_i = v_i\vb{E}^i.
\end{align*}
Vi definierar även den metriska tensorn som
\begin{align*}
	g_{ij} = \vb{E}_i\cdot\vb{E}_i,\ g^{ij} = \vb{E}^i\cdot\vb{E}^i.
\end{align*}
I ortogonala koordinatsystem gäller att
\begin{align*}
	g_{ij} = h_i^2\delta_{ij},\ g^{ij} = \frac{1}{h_i^2}\delta_{ij}.
\end{align*}