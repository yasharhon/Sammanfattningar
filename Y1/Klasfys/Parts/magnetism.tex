\section{Magnetism}

\subsection{Ekvationer}

\paragraph{Kraft på partikel i magnetfält}
\begin{align*}
	\vect{F} = q\vect{v}\times\vect{B}
\end{align*}

\paragraph{Flödet av magnetfält}
\begin{align*}
	\oint\vect{B}\cdot\dd{\vect{A}} = 0
\end{align*}

\paragraph{Cirkelbana för ladd partikel i magnetfält}
\begin{align*}
	R = \frac{mv}{\abs{q}B}
\end{align*}

\paragraph{Hastighetsfiltrering för vinkelrät elektriskt och magnetiskt fält}
\begin{align*}
	v = \frac{E}{B}
\end{align*}

\paragraph{Kraft på ledare i magnetfält}
\begin{align*}
	\dd{\vect{F}} = I\dd{\vect{l}}\vect{B}
\end{align*}

\paragraph{Magnetfält från laddning i rörelse}
\begin{align*}
	B = \frac{\mu_0q\vect{v}\times\vect{e}_{\vect{r}}}{4\pi r^2}
\end{align*}

\paragraph{Magnetfält kring ledare}
\begin{align*}
	\dd{B} = \frac{\mu_0I\dd{\vect{l}}\times\vect{e}_{\vect{r}}}{4\pi r^2}
\end{align*}

\paragraph{Ampères lag}
\begin{align*}
	\oint\limits_{\partial S}\vect{B}\cdot\dd{\vect{l}} = \mu_0\int\limits_{S}\vect{J}\cdot\dd{\vect{A}}
\end{align*}

\paragraph{Magnetfält kring oändlig ledare}
\begin{align*}
	B = \frac{\mu_0I}{2\pi r}
\end{align*}

\paragraph{Kraft per längd mellan två ledare}
\begin{align*}
	\frac{F}{L} = \frac{\mu_0I_1I_2}{2\pi r}
\end{align*}

\paragraph{Magnetfält i mitten av cirkulär ledare}
\begin{align*}
	B = \frac{\mu_0Ia^2}{2(x^2 + R^2)^\frac{3}{2}}
\end{align*}

\paragraph{Moment på kretsslinga i magnetfält}
\begin{align*}
	\vect{\tau} = \vect{\mu}\times\vect{B}
\end{align*}

\paragraph{Potensiell energi för kretsslinga i magnetfält}
\begin{align*}
	U = -\vect{\mu}\times\vect{B}
\end{align*}

\paragraph{Hall-effekt}
\begin{align*}
	nq = -\frac{J_xB_y}{E_z}
\end{align*}
$n$ representerar här tätheten av laddningsbärare.

\paragraph{Energi i en spola}
\begin{align*}
	U = \frac{1}{2}LI^2
\end{align*}

\paragraph{Energitäthet i magnetiska fält}
\begin{align*}
	u = \frac{B^2}{2\mu}
\end{align*}

\subsection{Principer}

\paragraph{Magnetiska dipoler}
Maxwells ekvationer förutspår att magnetism endast förekommer som dipoler.

\paragraph{Magnetiska material}
Elektroners bana och spinn ger upphov till magnetiska dipoler i material. I magnetiska material är dessa dipolerna i någon grad orienterade och ger upphov till ett makroskopiskt magnetiskt moment.