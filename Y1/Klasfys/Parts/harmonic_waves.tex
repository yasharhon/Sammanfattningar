\section{Harmoniska vågor}
Harmoniska vågor, även kallad plana vågor, är periodiska störningar i ett medium, och beskrivs typisk av en funktion på formen $e^{i\phi}$. För dessa kan man identifiera vissa storheter, somm vi kommer göra här.

\paragraph{Perioden}
Perioden är tiden det tar för en harmonisk våg att gå genom en cykel.

\paragraph{Frekvens och vinkelfrekvens}
Av större intresse är frekvens, som ges av
\begin{align*}
	f = \frac{1}{T},
\end{align*}
som då är antalet cykler vågen går genom per enhet tid. Av ännu större intresse är vinkelfrekvensen, som ges av
\begin{align*}
	\omega = 2\pi f = \frac{2\pi}{T},
\end{align*}
och ger antalet radianer vågen går genom per enhet tid.

\paragraph{Våglängd}
Våglängden är det minsta avståndet mellan två punkter som är i fas.

\paragraph{Vågvektor och vågtal}
För vågor i flera dimensioner är det smartare att använda en vågvektor. Denna har riktning motsvarande riktningen vågen propagerar i och längd
\begin{align*}
	k = \frac{2\pi}{\lambda}.
\end{align*}
$k$ kallas även vågtalet. Anledningen till att man heller använder vågvektorn är exakt den samma som att man heller använder frekvens än period.

\paragraph{Amplitud}
Harmoniska vågor representerar periodiska störningar. Denna störningen uppnår vid vissa tider ett maxvärde. Detta är vågens amplitud.