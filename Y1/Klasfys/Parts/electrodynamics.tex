\section{Elektrodynamik}

\subsection{Ekvationer}

\paragraph{Maxwells ekvationer}
\begin{align*}
	\div{\vect{E}}  &= \frac{\rho}{\varepsilon_0}, \\
	\div{\vect{B}}  &= 0, \\
	\curl{\vect{E}} &= -\pdv{\vect{B}}{t}, \\
	\curl{\vect{B}} &= \mu_0\vect{J} + \varepsilon_0\mu_0\pdv{\vect{E}}{t}.
\end{align*}
Alternativt på integralform:
\begin{align*}
	&\oint_{\partial V}\vect{E}\cdot\dd{\vect{A}} = \frac{1}{\varepsilon_0}\int_{V}\dd{V}\rho, \\
	&\oint\limits_{A}\vect{B}\cdot\dd{\vect{A}} = 0, \\
	&\oint_{\partial A}\vect{E}\cdot\dd{\vect{l}} = -\dv{t}\int\limits_{A}\vect{B}\cdot\dd{\vect{A}}, \\
	&\oint_{\partial A}\vect{B}\cdot\dd{\vect{l}} = \mu_0\left(\int_{A}\vect{J}\cdot\dd{\vect{A}} + \varepsilon_0\dv{t}\int\limits_{A}\vect{E}\cdot\dd{\vect{A}}\right).
\end{align*}
Man skulle kunna skriva en hel paragraf om varje ekvation för sig, men jag tyckte det blev snyggare att presentera de så här. $\Phi$ är flödet av fältet indikerat av subskriptet.

Den första ekvationen är Gauss' lag som vi känner den.

Den andra är Gauss' teorem för magnetfältet, även från statiken.

Den tredje ekvationen är Faradays lag. I praktiken betyder den att man kan inducera spänningar i kretsslingar.

Den fjärde ekvationen är Ampère-Maxwells lag, som även finns i statiken, men modifieras med en extra term.

Alla ekvationerna är formulerade i vakuum.

\paragraph{Inducerad spänning i kretsslinga}
\begin{align*}
	\varepsilon = \oint(\vect{v}\times\vect{B})\cdot\dd{\vect{l}}
\end{align*}

\deriv

\paragraph{Induktion mellan två spolar}
Två spolar inducerar en EMK i varandra. Dens storhet ges av
\begin{align*}
	\varepsilon_2 = -M\dv{I_1}{t}
\end{align*}
för spola $2$ och motsvarande för den andra.

\deriv

\paragraph{Definitionen av ömsesidig induktans}
\begin{align*}
	M = \frac{N_2\Phi_{B_2}}{I_1} = \frac{N_1\Phi_{B_1}}{I_2}
\end{align*}
Den ömsesidiga induktansen mellan två spolar är lika för de två spolarna.

\paragraph{Självinduktans}
Spolar inducerar även en EMK i sig själv. Denna ges av
\begin{align*}
	\varepsilon = -L\dv{I}{t}.
\end{align*}

\deriv

\paragraph{Definitionen av självinduktans}
Självinduktansen till en spola ges av
\begin{align*}
	L = \frac{N\Phi_{B}}{I}.
\end{align*}

\deriv

\subsection{Principer}

