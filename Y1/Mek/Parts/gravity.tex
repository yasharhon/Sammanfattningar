\section{Allmäna gravitationslagen och dens konsekvenser}

Vi vill i denna del försöka visa Keplers lagar, tagna från Tycho Brahes observationer, från Newtons allmäna gravitationslag $\vb{F} = -\frac{GmM}{r^2}\vb{e}_{r}$. För att göra detta betraktar vi en planet i bana kring någon annan planet, negligerar den sista planetens rörelse och ignorerar andra krafter. Detta är ett exempel på rörelse under en centralkraft, som är en kraft vars verkningslinje alltid går igenom någon punkt $O$.

Definiera ett koordinatsystem med origo i den ena planeten, och låt den andra gå i bana kring denna. Från definitionen av kraftmoment ser vi att tyngdkraften ej har ett kraftmoment, och därmed ändras $\vb{H}_{O}$ ej under rörelsen. Vidare vet vi att $\vb{H}_{O}$ är normal på ortsvektorn, vilket ger $\vb{r}\times\vb{H}_{O} = 0$. Med konstant rörelsemängdsmoment är då rörelsen i ett fixt plan.

Den infinitesimala arean $\dd{A}$ som sveps ut av partikelns bana under den infinitesimala tiden $\dd{t}$ ges av
\begin{align*}
	\dd{A} = \frac{1}{2}\abs{\vb{r}\times\dd{\vb{r}}}.
\end{align*}
Rörelsemängdsmomentet ges av $\vb{H}_{O} = \vb{r}\times m\vb{v}$, och vi skriver då
\begin{align*}
	\dd{A} = \frac{1}{2}\abs{\vb{r}\times\dd{\vb{r}}} = \frac{1}{2}\abs{\frac{1}{m}\vb{H}_{O}\dd{t}}.
\end{align*}
Eftersom $\vb{H}_{O}$ är konstant, är även den utsvepta arean konstant under samma tidsintervall $\dd{t}$. Detta är Keplers andra lag.

Eftersom arean som sveps ut per tid är konstant, tillämpar vi cylindriska koordinater för att skriva
\begin{align*}
	\dd{A} = \frac{1}{2}r^2\dv{\phi}{t}\dd{t} = \frac{1}{2}h\dd{t}.
\end{align*}
Vi skriver nu Newtons andra lag i cylindriska koordinater. Den radiella komponenten ges av
\begin{align*}
	\dv[2]{r}{t} - r\left(\dv{\theta}{t}\right)^2 = -\frac{GM}{r^2}.
\end{align*}
Vi byter variabel från $t$ till $\phi$ och får
\begin{align*}
	\dv{r}{t} = \dv{r}{\phi}\dv{\phi}{t}.
\end{align*}
Med uttrycket för arean skriver vi detta som
\begin{align*}
	\dv{r}{t} = \frac{h}{r^2}\dv{r}{\phi}.
\end{align*}
Vidare substituerar vi $u = \frac{1}{r}$ och får
\begin{align*}
	\dv{r}{t} = hu^2\dv{\phi}\left(\frac{1}{u}\right) = hu^2\cdot -\frac{1}{u^2}\dv{u}{\phi} = -h\dv{u}{\phi}.
\end{align*}
En andra derivation med avseende på $t$ ger
\begin{align*}
	\dv[2]{r}{t} = \dv{t}\left(-h\dv{u}{\phi}\right) = -h\dv{t}\left(\dv{u}{\phi}\right) = -h\dv{\phi}{t}\dv[2]{u}{\phi} = -h^2u^2\dv[2]{u}{\phi}.
\end{align*}
Nu kan man skriva Newtons andra lag som
\begin{align*}
	-h^2u^2\dv[2]{u}{\phi} - \frac{1}{u}h^2u^4 &= -h^2u^2\left(\dv[2]{u}{\phi} + u\right).
\end{align*}
Detta är Binets formel. Vi skriver nu Newtons andra lag som
\begin{align*}
	\dv[2]{u}{\phi} + u = \frac{GM}{h^2}.
\end{align*}
Vi löser denna med vilkoret att $\theta = 0$ när $r$ är minimal och får lösningar på formen
\begin{align*}
	r = \frac{\frac{h^2}{GM}}{1 + \frac{h^2C}{GM}\cos{\theta}}.
\end{align*}
Detta är ekvationen för en konisk sektion, där den stabila lösningen är en ellips med den ena planeten i brännpunkten, vilket är Keplers första lag.

Vi använder nu att rörelsemängden är bevarat och att kvoten $e = \frac{h^2}{GM}$ mäter avståndet från origo till den elliptiska banans centrum och får
\begin{align*}
	v_{\text{min}} = \frac{1 - e}{1 + e}v_{\text{max}}
\end{align*}
där den maximala hastigheten nås längst bort från origo och den minimala uppnås närmast origo. Vid att jämföra energin mellan dessa två punkterna får man
\begin{align*}
	v_{\text{max}}^2 - v_{\text{min}}^2 = \frac{2GM}{a}\left(\frac{1}{1 - e} - \frac{1}{1 + e}\right),
\end{align*}
där $a$ är längden på ellipsen längsta halvaxel. Dessa två kan nu kombineras för att få
\begin{align*}
	\frac{(1 + e)^2 - (1 - e)^2}{(1 - e)^2}v_{\text{max}}^2 = \frac{2GM}{a}\frac{2e}{(1-e)(1 + e)},
\end{align*}
vilket ger
\begin{align*}
	v_{\text{min}} = \sqrt{\frac{GM}{a}\frac{1 - e}{1 + e}},\ v_{\text{max}} = \sqrt{\frac{GM}{a}\frac{1 + e}{1 - e}}.
\end{align*}
Eftersom arean som sveps ut över tid är konstant, kan vi beräkna omloppstiden som
\begin{align*}
	\tau = \frac{A}{\dv{A}{t}} = \frac{\pi ab}{\frac{1}{2}h} = \frac{2\pi a^2\sqrt{1 - e^2}}{(1 - e)av_{\text{max}}} = \frac{2\pi a\sqrt{1 - e^2}}{(1 - e)a\sqrt{\frac{GM}{a}\frac{1 + e}{1 - e}}} = 2\pi\frac{a^{\frac{3}{2}}}{\sqrt{GM}},
\end{align*}
där vi har användt att $b = a\sqrt{1 - e^2}$. Denna proportionaliteten mellan $\tau$ och $a$ är Keplers tredje lag.

Slutligen tillämpar vi energilagen för att hitta hastigheten i en godtycklig punkt. Vi har att
\begin{align*}
	E = \frac{1}{2}mv_{\text{max}}^2 - \frac{GmM}{(1 - e)a}
\end{align*}
är konstant. Vi använder resultatet från innan och får
\begin{align*}
	E = -\frac{GmM}{2a}.
\end{align*}
Vi har nu
\begin{align*}
	E = \frac{1}{2}mv^2 - \frac{GmM}{r},
\end{align*}
vilket ger
\begin{align*}
	v = \sqrt{2GM\left(\frac{1}{r} - \frac{1}{2a}\right)}.
\end{align*}