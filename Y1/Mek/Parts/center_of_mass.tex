\section{Masscentrum}
Masscentrum till ett system av partikler definieras av
\begin{align*}
	\vect{r}_G = \frac{\sum m_i\vect{r}_i}{\sum m_i}.
\end{align*}
För en kontinuerlik kropp går detta mot
\begin{align*}
	\vect{r}_G = \frac{\int\dd{m}\vect{r}}{\int\dd{m}}.
\end{align*}
Detta kan skrivas som
\begin{align*}
	\vect{r}_G &= \frac{1}{M}\sum\int\limits_{V_i}\dd{m}\vect{r} \\
	           &= \frac{1}{M}\sum\frac{M_i}{M_i}\int\limits_{V_i}\dd{m}\vect{r} \\
	           &= \frac{1}{M}\sum M_i\vect{r}_{G, i}.
\end{align*}
Kroppen kan då partitioneras på lämpliga sätt, och man kan använda enkla resultat för att beräkna mer komplicerade masscentra.

Masscentrumets läge är så att om en kraft verkar i ett partikelsystems masscentrum, kommer dets verkan kunna reduceras till en enkraftsresultant.

%Detta är en punkt så att i ett homogent kraftfält är fältets verkan på partiklerna ekvivalent med att kraftsumman verkar i masscentrum.

\paragraph{Motivering: Tyngdkraft}
Betrakta ett system av partikler i ett homogent tyngdfält som verkar i $-z$-riktning. Kraftsumman ges av
\begin{align*}
	\vect{F} = \sum\vect{F}_i = -\sum m_ig\vect{e}_z = -mg\vect{e}_z
\end{align*}
där $m$ är den totala massan till partiklerna. Det totala kraftmomentet kring någon punkt $O$ är
\begin{align*}
	\vect{M}_O = \sum\vect{r}_i\times\vect{F}_i = -\sum \vect{r}_i\times m_ig\vect{e}_z = -\sum m_ig\vect{r}_i\times\vect{e}_z.
\end{align*}
Detta är ett system av parallella krafter, vilket alltid har ett enkraftsresulat. Låt enkraftsresulantens angrespunkt ges av ortsvektorn $\vect{r}$ relativt $O$. Eftersom enkraftsresultanten är ekvimoment med det ursprungliga kraftsystemet, får man
\begin{align*}
	\vect{r}\times\vect{F}                                       &= \vect{M}_O \\
	\vect{r}\times (-mg\vect{e}_z)                               &= -\sum m_ig\vect{r}_i\times\vect{e}_z \\
	\left(m\vect{r} - \sum m_i\vect{r}_i\right)\times\vect{e}_z &= 0
\end{align*}
Detta innebär att uttrycket i parentesen är parallellt med $\vect{e}_z$, vilket ger
\begin{align*}
	m\vect{r} - \sum m_i\vect{r}_i = \mu\vect{e}_z
\end{align*}
och slutligen
\begin{align*}
	\vect{r} &= \frac{1}{m}\sum m_i\vect{r}_i + \frac{\mu}{m}\vect{e}_z \\
	         &= \vect{r}_{G} + \frac{\mu}{m}\vect{e}_z
\end{align*}
där vi har introducerat masscentrum. Enkraftsresultanten verkar då i en linje som går genom masscentrum.

\paragraph{Tyngdpunkt}
Tyngdpunktet spelar samma roll som masscentrum, fast i det allmäna tyngdfältet.