\section{Fundamentala koncepter och begrepp}

\paragraph{Krafter}
En kraft $\vect{F}$ beskrivs av en vektor med belopp och rikting, samt en angrepspunkt.

\paragraph{Kraftmoment}
En kraft kan ha en viss vridningsförmåga med avseende på en punkt. Detta är kraftens kraftmoment. Om en kraft vrider kring punkten $O$, ges kraftmomentet av
\begin{align*}
	\vect{M}_O = \vect{r}\times\vect{F},
\end{align*}
där $\vect{r}$ är vektorn från $O$ till $\vect{F}$:s angrepspunkt och $\vect{F}$ är själva kraften.

Riktingen till kraftmomentet anger den positiva rotationsriktningen. Vad betyder detta? Jo, låt en linje gå genom $O$ och parallellt med $\vect{M}$. Då skapar $\vect{M}$ en vridning mot klockan kring denna linjen.

Kraftmomentet ändras inte av att kraften förskjutas längs med dens verkningslinje. Detta ser man vid att låta den angripa i två punkter $A, B$ på verkningslinjen.
\begin{align*}
	\vect{M}_O  &= \vect{r}_{OA}\times\vect{F} \\
	\vect{M}'_O &= \vect{r}_{OB}\times\vect{F} \\
	            &= (\vect{r}_{OA} + \vect{r}_{AB})\times\vect{F} \\
	            &= \vect{r}_{OA}\times\vect{F} + \vect{r}_{AB}\times\vect{F}\\
	            &= \vect{M},
\end{align*}
då den andra vektoren är parallell med $\vect{F}$.

Från detta kan vi räkna ut ett kraftmoments komponent med avseende på en axel vid att välja en punkt $P$ på axeln och beräkna kraftmomentet med avseende på denna punkten. Kraftmomentets komponent med avseende på axeln ges då av
\begin{align*}
	M_\lambda = \vect{M}_P\cdot\vect{e}_\lambda
\end{align*}
där $\vect{e}_\lambda$ är en enhetsvektor parallell med axeln.

Denna komponenten är oberoende av valet av $P$. Detta ser man vid att välja en ny punkt $Q$ och beräkna
\begin{align*}
	\vect{M}_P &= \vect{r}_{PA}\times\vect{F} \\
	\vect{M}_Q &= \vect{r}_{QA}\times\vect{F} \\
	           &= (\vect{r}_{QP} + \vect{r}_{PA})\times\vect{F} \\
	           &= \vect{r}_{QP}\times\vect{F} + \vect{r}_{PA}\times\vect{F}
\end{align*}
Komponenten med avseende på axeln ges då av
\begin{align*}
	\vect{M}_Q\cdot\vect{e}_\lambda &= (\vect{r}_{QP}\times\vect{F})\cdot\vect{e}_\lambda + (\vect{r}_{PA}\times\vect{F})\cdot\vect{e}_\lambda \\
	                                &= (\vect{r}_{QP}\times\vect{F})\cdot\vect{e}_\lambda \\
	                                &= \vect{M}_Q\cdot\vect{e}_\lambda,
\end{align*}
eftersom $\vect{r}_{QP}$ är parallell med $\vect{e}_\lambda$, och kryssprodukten vi beräknar då måste vara normal på båda dessa.

\paragraph{Stel kropp}
En stel kropp är en kropp som uppfyller att
\begin{align*}
	\dv{\abs{\vect{r}_{AB}}}{t} = 0\ \forall\ A, B. 
\end{align*}

\paragraph{Friläggning}
Att frilägga en kropp består i att betrakta hela eller delar av system och alla krafter och moment som verkar på varje del. Här kan även inre krafter uppstå för delsystemerna som ignoreras för det totala systemet.

\paragraph{Friktionskraft}
Friktionskraften är en speciell kraft, och förtjäner en lite mer noggrann beskrivning.

Friktionskraften är en kontaktkraft som uppstår när två ytor är i kontakt. Kraften uppstår på grund av interaktioner mellan materian i de två ytorna. Friktion förekommer som statisk eller kinematisk friktion, beroende på om objektet som upplever friktion rör på sig eller inte. Friktionskraftens belopp beror av friktionskoefficienten $\mu$. Denna beror igen på t.ex. om friktionen är statisk eller kinematisk och egenskaperna till ytorna som är i kontakt. Friktionskraften pekar alltid i motsatt riktning av ytornas relativa rörsle.

Vad är så beloppet av friktionskraften? I det statiska fallet är friktionskraften alltid så att kraftsumman på objektet är noll, och dens belopp är mindre än eller lika med $\mu N$, där $N$ är normalkraften på den ena ytan från den andra. I det kinematiska fallet är friktionskraftens belopp lika med $\mu N$.