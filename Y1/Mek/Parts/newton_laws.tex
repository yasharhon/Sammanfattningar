\section{Newtons lagar}

\paragraph{Första lagen}
En partikel förblir i sitt tillstånd av vila eller rätlinjig likformig rörelse då och endast då summan av alla krafter som verkar på partikeln är noll.

Denna formuleras separat av två årsaker:
\begin{itemize}
	\item Historiska årsaker, då det var denna lagen som först stod i strid med Aristoteles' värdsuppfattning.
	\item Teoretiska årsakar, då den ger ett enkelt sätt att definiera interialramer på.
\end{itemize}

\paragraph{Andra lagen}
\begin{align*}
	\vect{F} = \dv{\vect{p}}{t}
\end{align*}

\paragraph{Tredje lagen}
Om en kropp utövar en kraft på en annan, kommer denna utöva en lika stor och motsatt riktad kraft på den första kroppen.

\paragraph{Inertialsystem}
Ett inertialsystem är ett koordinatsystem där Newtons andra lag gäller. Motsatsen är ett icke-inertialsystem.

\paragraph{Galileitransformationer}
Vi söker ett samband mellan två referensramer. Låt en ram $S$ vara en intertialram och låt $S'$ röra sig relativt $S$ så att
\begin{itemize}
	\item hastigheten till origo i $S'$ relativt $S$, betecknad $\vb{v}_{O'}$, är konstant.
	\item $S'$ ej roterar relativt $S$.
\end{itemize}

Enligt definitionen kan man skriva ortsvektorer i $S$ som
\begin{align*}
	\vb{r} = \vb{r}_{O'} + \vb{r}'
\end{align*}
där $\vb{r}'$ är ortsvektorn i $S'$. Derivation med avseende på tid ger
\begin{align*}
	\vb{v} &= \vb{v}_{O'} + \dv{t}\left(x'\vb{e}_{x'} + y'\vb{e}_{y'} + z'\vb{e}_{z'}\right) \\
	       &= \vb{v}_{O'} + \dv{x'}{t}\vb{e}_{x'} + \dv{y'}{t}\vb{e}_{y'} + \dv{z'}{t}\vb{e}_{z'} \\
	       &= \vb{v}_{O'} + \vb{v}'.
\end{align*}
Derivation en gång till ger
\begin{align*}
	\vb{a} = \vb{a}'.
\end{align*}

Konsekvensen är att Newtons andra lag ser exakt likadan ut i $S'$ som i $S$.

\paragraph{Momentekvationen}
Tidsderivatan av rörelsemängdsmomentet ges av
\begin{align*}
	\dv{\vb{H}_{O}}{t} = \dv{t}(\vb{r}_{O}\times m\vb{v}) = \dv{\vb{r}}{t}\times m\vb{v} + \vb{r}\times m\dv{\vb{v}}{t}.
\end{align*}
Den första termen ger inget bidrag då de två vektorerna är parallella. Newtons andra lag ger då
\begin{align*}
	\dv{\vb{H}_{O}}{t} = \vb{r}\times\vb{F} = \vb{M}_{O}.
\end{align*}
Detta är momentekvationen.

\paragraph{Impulslagarna}
Integration av Newtons andra lag och momentekvationen ger
\begin{align*}
	\Delta\vb{p} = \integ{t_1}{t_2}{t}{\vb{F}},\ \Delta\vb{H}_{O} = \integ{t_1}{t_2}{t}{\vb{M}_{O}}.
\end{align*}
$\integ{t_1}{t_2}{t}{\vb{F}}$ kallas för impulsen och $\integ{t_1}{t_2}{t}{\vb{M}_{O}}$ kallas för momentimpuls.