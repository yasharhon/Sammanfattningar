\subsection{Definitioner}

\paragraph{Delmängder}

Låt $A, B$ vara mängder. $A$ är en delmängd av $B$ om det för varje $x \in A$ gäller att $x \in B$. Notation: $A \subset B$.

\paragraph{Union och snitt}

Låt $A, B$ vara mängder. Unionen $A \cup B$ består av de element som ligger i någon av mängderna. Snittet $A \cap B$ består av de element som är i båda.

\paragraph{Övre och undra begränsningar}

Ett tal $m$ är en övre begränsning av en mängd $A$ om $x \leq m$ för varje $x \in A$, och en undra begränsning om $x \geq m$ för varje $x \in A$.

\paragraph{Supremum och infimum}

Ett tal $m$ är supremum till en mängd $A$ om $m$ är den minsta övre begränsningen till $A$. $m$ är infimum till $A$ om $m$ är den största undra begränsningen till $A$. Notation: $\textnormal{sup}A, \textnormal{inf}A$.

\subsection{Satser}

\paragraph{Supremumsegenskapen}

Varje uppåt begränsade delmängd av $\mathbb{R}$ har en minsta övre begränsning.

\proof
Överkurs.