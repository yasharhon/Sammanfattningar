\section{Talföljder}

\subsection{Definitioner}

\paragraph{Definitionen av en talföjld}
En talföljd är en följd av tal $a_1, a_2, ...$ och betecknas $\left(a_n\right)_{n = 1}^\infty$.

\paragraph{Delföljder}
En delföljd av en talföljd är en selektion av tal i följden som fortfarande är oändligt stor.

\paragraph{Växande och avtagande talföljder}
En talföljd är växande om $a_{n + 1} \geq a_n$ för varje $n \geq 1$. Avtagande talföljder definieras analogt.

\paragraph{Uppåt och nedåt begränsade talföljder}
En talföljd är uppåt begränsad om det finns ett $M$ så att $a_n \leq M$ för alla $n \geq 1$. Nedåt begränsade talföljder definieras analogt.

\paragraph{Begränsade talföljder}
En talföljd är begränsad om den är både uppåt och nedåt begränsad.

\paragraph{Konvergens av talföljder}
En talföljd konvergerar mot ett gränsvärde $A$ om det för alla $\varepsilon > 0$ finns ett $N$ sådant att $\abs{a_n - A} < \varepsilon$ för varje $n > N$. Detta beteendet betecknas
\begin{align*}
	\lim_{n\to\infty} a_n = A.
\end{align*}

\paragraph{Divergenta talföljder}
En divergent talföljd är inte konvergent.

\paragraph{Binomialkoefficienter}
\begin{align*}
	\left(\frac{n}{k}\right) = \frac{n!}{(n-k)!k!}
\end{align*}

\paragraph{$e$, Eulers tal}
\begin{align*}
	e = \lim_{n\to\infty}\left(1 + \frac{1}{n}\right)^{n}
\end{align*}

\subsection{Satser}

\paragraph{Gränsvärden för kombinationer av talföljder}
Låt $\left(a_n\right)_{n = 1}^\infty, \left(b_n\right)_{n = 1}^\infty$ vara talföljder med gränsvärden $A$ och $B$. Då följer att
\begin{enumerate}
	\item[a)] $\left(a_n + b_n\right)_{n = 1}^\infty$ är konvergent med gränsvärdet $A + B$.
	\item[b)] $\left(a_n b_n\right)_{n = 1}^\infty$ är konvergent med gränsvärdet $AB$.
	\item[c)] om $B \neq 0$ är $\left(\frac{a_n}{b_n}\right)_{n = 1}^\infty$ konvergent med gränsvärdet $\frac{A}{B}$.
	\item[d)] om $a_n \leq b_n$ för varje $n$ så gäller att $A \leq B$.
\end{enumerate}

\proof

\paragraph{Växande och uppåt begränsade talföljder}
Om $\left(a_n\right)_{n = 1}^\infty$ är en växande och uppåt begränsad talföljd så är den konvergent och
\begin{align*}
	\lim_{n\to\infty} a_n = \sup{\{a_n: n \geq 1\}}
\end{align*}
Det analoga gäller för avtagande och nedåt begränsade mängder.

\proof
Enligt supremumsegenskapen finns det ett $K = \sup{\left(a_n\right)_{n = 1}^\infty}$. Då finns det även $a_i$ godtyckligt nära $K$ - med andra ord finns det ett $N$ så att $\abs{a_N - K} < \varepsilon$ för något $\varepsilon > 0$. Eftersom talföljden är växande, är detta även sant när $n > N$, vilket fullbördar beviset.

\paragraph{Binomialsatsen}
För $n\in\Z$ har man
\begin{align*}
	(a + b)^n = \sum\limits_{k = 0}^{n}\left(\frac{n}{k}\right) a^kb^{n-k}.
\end{align*}

\proof

\paragraph{Gränsvärde för potenser}
\begin{align*}
	\lim_{n\to\infty} n^p =
	\begin{cases}
		\infty, & p > 0\\
		0,      & p < 0
	\end{cases}
\end{align*}

\proof

\paragraph{Standardgränsvärden}
Låt $a > 1$ och $b > 0$. Då gäller att
\begin{align*}
	\lim_{n\to\infty}\frac{a^n}{n^b} = \infty \\
	\lim_{n\to\infty}\frac{n!}{b^n} = \infty
\end{align*}

\proof

\paragraph{Ändligt värde av $e$}
Talföljden $\left(a_n\right)_{n = 1}^\infty$ med
\begin{align*}
	a_n = \left(1 + \frac{1}{n}\right)^{n}
\end{align*}
är konvergent.

\proof

\paragraph{Bolzano-Weierstrass' sats}
Låt $\left(a_n\right)_{n = 1}^\infty$ vara en begränsad talföljd. Då finns det konvergent delföljd.

\proof