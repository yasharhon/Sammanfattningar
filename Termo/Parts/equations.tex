\section{Ekvationer}

Om inte annat specifieras, kommer alla ekvationer följa symbolkonvention enligt denna tabellen.

\begin{table}[!h]
	\centering
	\begin{tabular}{| l | c |}
		\hline
		\textbf{Storhet} & \multicolumn{1}{|l|}{\textbf{Symbol}}\\
		\hline
		Tryck           & $p$ \\
		\hline
		Volym           & $V$ \\
		\hline
		Temperatur      & $T$ \\
		\hline
		Antal partiklar & $N$ \\
		\hline
		Antal mol       & $\nu$ \\
		\hline
		Inre energi     & $U$ \\
		\hline
		Värme           & $Q$ \\
		\hline
		Arbete          & $W$ \\
		\hline
	\end{tabular}
\end{table}

\twocolumn

\subsection{Allmänna ekvationer}

\paragraph{Konversion mellan $m$, $\nu$ och $N$}
\begin{align*}
	\nu = \frac{m}{M} = \frac{N}{N_\text{A}}.
\end{align*}
$M$ är gasens molara massa, massan per mol partiklar. Flera relationer kan härledas vid att använda $R = N_\text{A}k$.

\paragraph{Täthet}
\begin{align*}
	\rho = \frac{m}{V}
\end{align*}
Tätheten av en substans kan även definieras som
\begin{align*}
	\rho = \frac{N}{V}.
\end{align*}

\subsection{Termodynamikens huvudsatser}

\paragraph{Första huvudsatsen}
\begin{align*}
	\dd{U} = \idd Q - \idd W
\end{align*}
Vid att definiera första huvudsatsen så, definieras arbete gjort på systemet implisitt som positivt. Arbetet ges av
\begin{align*}
	\idd W = -p\dd{V}
\end{align*}

\paragraph{Andra huvudsatsen}
\begin{align*}
	\dd{S} = \frac{\idd Q}{T} \geq 0
\end{align*}
Likheten gäller för reversibla processer, och olikheten gäller för andra processer.

\paragraph{Tredje huvudsatsen}
\begin{align*}
	S(T = 0) = 0
\end{align*}

\paragraph{Potensialer, derivator och annat skit}

\paragraph{Inre energiens differensial}
Vid att kombinera termodynamikens första och andra huvudsats får man
\begin{align*}
	\dd{U} = T\dd{S} - p\dd{V}
\end{align*}

\paragraph{Fri energi}
\begin{align*}
	& F = U -TS\\
	& \dd{F} = -S\dd{T} - p\dd{V}
\end{align*}
Vid isoterma processer är $\idd W = \dd{F}$.

\paragraph{Entalpi}
\begin{align*}
	& H = U + pV \\
	& \dd{H} = T\dd{S} + V\dd{p}
\end{align*}

\paragraph{Fri entalpi}
\begin{align*}
	& G = H - TS \\	
	& \dd{G} = -S\dd{T} + V \dd{p}
\end{align*}

\subsection{Gaser}

\paragraph{Ideala gaslagen}
\begin{align*}
	pV = \nu RT = NkT.
\end{align*}
$N$ är antalet partiklar i gasen och $\nu$ är antalet mol partiklar i gasen.

\paragraph{van der Waals' tillståndsekvation}
\begin{align*}
	p = \frac{NkT}{V - Nb} - a\left(\frac{N}{V}\right)^2\\
	\left(p + \frac{a_0}{v^2}\right)(v - b_0) = RT
\end{align*}
Dessa är båda ekvivalenta versioner av van der Waals' tillståndsekvation, var man introduserar $a_0 = a N_\text{A}^2$, $b_0 = bN_\text{A}$ och $v = \frac{V}{\nu}$. $a$ innehåller information om växelverkan mellan partiklarna och $b$ innehåller information om partiklarnas volym.

\paragraph{Maxwell-Boltzmann-fördelingen}
Partiklarna i en ideal gas har olik fart. Antalet partiklar med en given fart $v$ per volym är fördelad enligt
\begin{align*}
	n(v) = Cv^2e^{-\frac{mv^2}{2kT}},
\end{align*}
var $m$ är en partikels massa. Vi krävjer att fördelingen är normaliserad, dvs.
\begin{align*}
	\int_0^{\infty}\dd{v}n(v) = \frac{N}{V},
\end{align*}
som ger
\begin{align*}
	K = 4\pi n \left(\frac{m}{2\pi kT}\right)^\frac{3}{2}.
\end{align*}
Från detta kan man räkna ut en mest sannolik fart $v_\text{p}$, en förväntad fart $<v>$ och en RMS-fart $v_\text{RMS}$. Dessa är
\begin{align*}
	& v_\text{p} = \sqrt{\frac{2kT}{m}}\\
	& \expval{v} = \sqrt{\frac{8kT}{\pi m}}\\
	& v_\text{p} = \sqrt{\expval{v^2}} = \sqrt{\frac{3kT}{m}}.
\end{align*}
Man kan även räkna ut en medelenergi per partikel, som är
\begin{align*}
	\expval{E} = \frac{n}{2}kT,
\end{align*}
var $n$ är antalet kvadratiska frihetsgrader per partikel. För en enatomig ideal gas är det
\begin{align*}
	\expval{E} = \frac{3kT}{2}.
\end{align*}

\paragraph{Medelfri väg}
Medelavståndet mellan två kollisioner i en gas är
\begin{align*}
	l = \frac{kT}{p\pi d^2\sqrt{2}} = \frac{1}{n\pi d^2\sqrt{2}}
\end{align*}
var $n$ är partikeltätheten och $d$ är partiklernas diameter.

\paragraph{Stöttal}
Stöttalet är antalet partikler som kolliderar med en yta per enhet area och tid, och fås som
\begin{align*}
	\nu * = \frac{1}{4}n\expval{v} = \frac{p}{\sqrt{2\pi mkT}}
\end{align*}
var $n$ är partikeltätheten.