\section{Ekvationer}

\subsection{Allmänna ekvationer}

\paragraph{Konversion mellan $m$, $\nu$ och $N$}

\begin{align*}
	\nu = \frac{m}{M} = \frac{N}{N_\text{A}}
\end{align*}

$M$ är gasens molara massa, massan per mol partiklar. Flera relationer kan härledas vid att använda $R = N_\text{A}k$.

\paragraph{Täthet}

\begin{align*}
	\rho = \frac{m}{V}
\end{align*}

Tätheten av en substans kan även definieras som

\begin{align*}
	\rho = \frac{N}{V}
\end{align*}

\subsection{Gaser}

\paragraph{Ideala gaslagen}

\begin{align*}
	pV = \nu RT = NkT
\end{align*}

$p$ är gasens tryck, $V$ är gasens volym, $T$ är gasens temperatur, $N$ är antalet partiklar i gasen och $\nu$ är antalet mol partiklar i gasen.

\paragraph{van der Waals' tillståndsekvation}

\begin{align*}
	p = \frac{NkT}{V - Nb} - a\left(\frac{N}{V}\right)^2\\
	\left(p + \frac{a_0}{v^2}\right)(v - b_0) = RT
\end{align*}

Dessa är båda ekvivalenta versioner av van der Waals' tillståndsekvation, var man introduserar $a_0 = a N_\text{A}^2$, $b_0 = bN_\text{A}$ och $v = \frac{V}{\nu}$. $a$ innehåller information om växelverkan mellan partiklarna och $b$ innehåller information om partiklarnas volym.

\paragraph{Maxwell-Boltzmann-fördelingen}

Partiklarna i en ideal gas har olik fart. Antalet partiklar med en given fart $v$ per volym är fördelad enligt
\begin{align*}
	n(v) = Cv^2e^{-\frac{mv^2}{2kT}}
\end{align*}
var $m$ är en partikels massa. Vi krävjer att fördelingen är normaliserad, dvs.
\begin{align*}
	\int_0^{\infty}\dd{v}n(v) = \frac{N}{V}
\end{align*}
som ger
\begin{align*}
	K = 4\pi n \left(\frac{m}{2\pi kT}\right)^\frac{3}{2}
\end{align*}

Från detta kan man hitta en mest sannolik fart $v_\text{p}$, en förväntad fart $<v>$ och en RMS-fart $v_\text{RMS}$. Dessa är
\begin{align*}
	& v_\text{p} = \sqrt{\frac{2kT}{m}}\\
	& \expval{v} = \sqrt{\frac{8kT}{\pi m}}\\
	& v_\text{p} = \sqrt{\expval{v^2}} = \sqrt{\frac{3kT}{m}}\\
\end{align*}