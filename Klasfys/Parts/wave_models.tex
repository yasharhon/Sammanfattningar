\section{Modellering av systemer}
Syftet i denna delen är att diskutera systemer som kan modelleras med teorin bak klassiska vågor.

\subsection{Vätskor och gaser}
Vi tänker oss att vi trycker på vätskan eller gasen i en behållare med tvärsnittsarea $A$ och volym $V$. Detta får mediet att komprimeras. För $\dd{p} << p_0$ har man
\begin{align*}
	\dd{p} = -B\frac{\dd{V}}{V},
\end{align*}
där $B$ är mediets bulkmodul. Kompressionen kan tänkas få oändligt tunna volymelementer i mediet att förflytta sig ett avstånd $s(x)$, där $x$ är en koordinat som indikerar hur djupt i behållaren volymelementet är.

Betrakta nu vätskan eller gasen mellan två punkter $x$ och $x + \dd{x}$. Tryckskillnaden mellan dessa två punkterna ges av
\begin{align*}
	\dd{p} = -B\frac{\dd{V}}{V} = -B\frac{A(s(x + \dd{x}) - s(x)}{A\dd{x}}\to -B\dv{s}{x},
\end{align*}
där vi förutsätter att $s(x), s(x + \dd{x}) << \dd{x}$. Newtons andra lag ger då
\begin{align*}
	ma = A\Delta p. 
\end{align*}
Om vi stoppar in för $p, m, a$ får vi
\begin{align*}
	&\rho A\dd{x}\pdv[2]{s}{t} = AB\Delta\pdv{s}{x} \\
	&\pdv[2]{s}{t} = \frac{B}{\rho}\frac{\pdv{s}{x}(x + \dd{x}) - \pdv{s}{x}(x)}{\dd{x}} = \frac{B}{\rho}\pdv[2]{s}{x},
\end{align*}
och vågfarten är $c = \sqrt{\frac{B}{\rho}}$.

Om vi även antar att vågen propagerar via adiabatiska kompressioner, har vi
\begin{align*}
	&pV^{\gamma} = C \\
	&\ln{p} + \gamma\ln{V} = c \\
	&\frac{\dd{p}}{p} + \gamma\frac{\dd{V}}{V} = 0,
\end{align*}
vilket ger
\begin{align*}
	B = -\frac{\dd{p}}{\left(\frac{\dd{V}}{V}\right)} = \gamma p.
\end{align*}
Om man använder ideala gaslagen på detta, får man
\begin{align*}
	c = \sqrt{\frac{\gamma RT}{M}},
\end{align*}
där $M$ är gasens molära massa.

\paragraph{Fasta materialer i bulk}
Betrakta ett material som sträcks mellan $x, x + \dd{x}$. Materialet sträcks då en längd $s(x, t)$. Om $s(x, t)$ är litet jämfört med $\dd{x}$ upplevs samma kraft i båda punkter. Spänningen i materialet är då
\begin{align*}
	\sigma = \frac{F}{A} = \frac{E(s(x + \dd{x})  s(x)}{\dd{x}}\to E\pdv{s}{x},
\end{align*}
där vi har användt Hookes lag för fasta materialer, som ger $\frac{F}{A} = E\frac{\Delta l}{l_0}$, där $E$ är materialets elasticitetsmodul.

Materialet mellan $x, x + \dd{x}$ har enligt ovan en position som ges av $s$. Då ger Newtons andra lag
\begin{align*}
	\rho A\dd{x}\pdv[2]{s}{t} &= \Delta F \\
	\rho A\pdv[2]{s}{t}       &= \pdv{F}{x} = AE\pdv[2]{s}{x} \\
	\pdv[2]{s}{x}             &= \frac{\rho}{E}\pdv[2]{s}{t}
\end{align*}
och vågfarten är
\begin{align*}
	c = \sqrt{\frac{E}{\rho}}.
\end{align*}

Märk att resonnemanget ovan är för longitudinella vågor. För transversella vågor ges farten av
\begin{align*}
	c = \sqrt{\frac{G}{\rho}}
\end{align*}
där $G$ är materialets skjuvmodul.

\paragraph{Transversella vågor i snören}
Av någon magisk årsak ges vågfarten av
\begin{align*}
	c = \sqrt{\frac{T}{\mu}}
\end{align*}
där $T$ är snörets spänning och $\mu$ är massan per längdenhet.