\section{Modellering av systemer}
Syftet i denna delen är att diskutera systemer som kan modelleras med teorin bak klassiska vågor.

\subsection{Vätskor och gaser}
Vi tänker oss att vi trycker på vätskan eller gasen i en behållare med tvärsnittsarea $A$ och volym $V$. Detta får mediet att komprimeras. För $\dd{p} << p_0$ har man
\begin{align*}
	\dd{p} = -B\frac{\dd{V}}{V},
\end{align*}
där $B$ är mediets bulkmodul. Kompressionen kan tänkas få oändligt tunna volymelementer i mediet att förflytta sig ett avstånd $s(x)$, där $x$ är en koordinat som indikerar hur djupt i behållaren volymelementet är.

Betrakta nu vätskan eller gasen mellan två punkter $x$ och $x + \dd{x}$. Tryckskillnaden mellan dessa två punkterna ges av
\begin{align*}
	\dd{p} = -B\frac{\dd{V}}{V} = \dd{p} = -B\frac{A(s(x + \dd{x}) - s(x)}{A\dd{x}}\to -B\dv{s}{x}
\end{align*}