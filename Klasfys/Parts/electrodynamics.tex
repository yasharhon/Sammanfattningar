\section{Elektrodynamik}

\subsection{Ekvationer}

\paragraph{Maxwells ekvationer}
\begin{align*}
	&\oint_{\partial V}\vect{E}\cdot\dd{\vect{A}} = \frac{1}{\varepsilon_0}\int_{V}\dd{V}\rho, \\
	&\oint\vect{B}\cdot\dd{\vect{A}} = 0, \\
	&\oint_{\partial A}\vect{E}\cdot\dd{\vect{l}} = -\dv{\Phi_{B}}{t}, \\
	&\oint_{\partial A}\vect{B}\cdot\dd{\vect{l}} = \mu_0\left(\int_{A}\vect{J}\cdot\dd{\vect{A}} + \varepsilon\dv{\Phi_{E}}{t}\right).
\end{align*}
Man skulle kunna skriva en hel paragraf om varje ekvation för sig, men jag tyckte det blev snyggare att presentera de så här. $\Phi$ är flödet av fältet indikerat av subskriptet.

Den första ekvationen är Gauss' lag som vi känner den.

Den andra är Gauss' teorem för magnetfältet, även från statiken.

Den tredje ekvationen är Faradays lag. I praktiken betyder den att man kan inducera spänningar i kretsslingar.

Den fjärde ekvationen är Ampère-Maxwells lag, som även finns i statiken, men modifieras med en extra term.

\paragraph{Inducerad spänning i kretsslinga}
\begin{align*}
	\varepsilon = \oint(\vect{v}\times\vect{B})\cdot\dd{\vect{l}}
\end{align*} 

\subsection{Principer}

