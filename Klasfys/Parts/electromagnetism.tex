\section{Elektromagnetism}

\subsection{Ekvationer}

\paragraph{Coulombs lag}
\begin{align*}
	\vect{F} = \frac{q_1q_2}{4\pi\varepsilon_0 r^2}\vect{\bar{r}}
\end{align*}
Coulombs lag ger den elektriska kraften mellan två laddningar. $\vect{\bar{r}}$ är en enhetsvektor mellan laddningarna.

\deriv
Ursprungligt ett experimentellt resultat, fås alternativt via Maxwells ekvationer.

\paragraph{Elektriskt fält}
\begin{align*}
	\vect{E} = \frac{1}{q_1}\vect{F}
\end{align*}
Detta definierar det elektriska fältet som upplevs av en testladdning $q_1$.

\paragraph{Elektriskt potensiale}
Om en kropp påverkas av en kraft medan den förflyttar sig längs en bana $C$, gör den arbetet
\begin{align*}
	W = -\int_{C}\vect{F}\cdot\dd{\vect{s}}
\end{align*}
mot kraften. Om man definierar potensialet $V = \frac{1}{q_1}W$ får man
\begin{align*}
	V = -\frac{1}{q_1}\int_{C}\vect{F}\cdot\dd{\vect{s}} = \int_{C}\vect{E}\cdot\dd{\vect{s}},
\end{align*}
vilket vi tar som vår definition på potensialet.

Man måste alltdi välja någon referens för potensialet. Om man väljer denna i oändligheten, ges potensialet från en punktladdning av
\begin{align*}
	V = \frac{q}{4\pi\varepsilon_0 r}.
\end{align*}
Enligt superpositionsprincipet ges potensialet från en laddningstäthet $\rho$ av
\begin{align*}
	V = \int\dd{V}\frac{\rho}{4\pi\varepsilon_0 r}.
\end{align*}

\paragraph{Kraftmoment på dipol}
\begin{align*}
	\vect{\tau} = \vect{p}\times\vect{E}
\end{align*}
Detta är ekvationen för kraftmomentet på en dipol. $\vect{p}$ pekar från den negativa till den positiva laddningen, medan dens längd beror på hur stora laddningarna i dipolen är.

\deriv

\paragraph{Gauss' lag}
\begin{align*}
	\oint_{\partial V}\vect{E}\cdot\dd{\vect{A}} = \frac{1}{\varepsilon_0}\int_{V}\dd{V}\rho
\end{align*}
Gauss' lag relaterar flödet av det elektriska fältet genom en yta till mängden laddning innanför ytan.

I ett dielektrikum blir Gauss' lag
\begin{align*}
	\oint_{\partial V}\varepsilon\vect{E}\cdot\dd{\vect{A}} = \int_{V}\dd{V}\rho,
\end{align*}
där vi nu integrerar över den fria laddningstätheten.

\deriv

\paragraph{Elektrisk energitäthet}
\begin{align*}
	u = \frac{1}{2}\varepsilon_0E^2
\end{align*}

\deriv

\paragraph{Strömtäthet}
\begin{align*}
	\vect{J} = nq\vect{v}_d
\end{align*}
$n$ är tätheten av laddade partiklar, $q$ är laddningen till en partikel och $\vect{v}_d$ är partikelns drifthastighet (?).

\deriv

\paragraph{Ledning i metaller}
\begin{align*}
	\rho = \frac{m_e}{ne^2\tau}
\end{align*}
$m_e$ är elektronmassan, $n$ är elektrontätheten och $\tau$ är medeltiden mellan kollisioner mellan två elektroner.

\subsection{Principer}

\paragraph{Metaller}
I metaller är laddningar fria, vilket ger att det elektriska fältet inuti metallet alltid är noll. Då är potensialet inuti metallet konstant. Om det finns någon laddningstäthet i metallet, inducerad eller ej, kommer den vara lokaliserad på metallets yta. Detta utnyttjas i t.ex. Faradaybur, som är behållare av metall som även har noll elektriskt fält inuti sig.

\paragraph{Superpositionsprincipet}
Potensialet från flera laddningar kan adderas.

\paragraph{Dielektrikas påverkan av elektriska fältet}
Dielektrika består av massa dipoler. När ett dielektrikum placeras i ett elektriskt fält, kommer dipolerna orienteras med fältet, vilket kommer skapa en laddningstäthet på ytan. Eftersom elektriska fält får de positiva änderna av dipolerna att peka bort, kommer de inducerade laddningstätheterna att skapa ett elektriskt fält riktad mot det yttra fältet.