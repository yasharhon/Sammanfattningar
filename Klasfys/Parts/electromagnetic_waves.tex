\section{Elektromagnetiska vågor}
Från Maxwells ekvationer kan man visa att flera storheter kopplade till elektromagnetism följer vågekvationen. Dessa vågor kallas för elektromagnetiska vågor. Experimenter visade att ljus propagerade med samma farten som den teoretiska farten till elektromagnetiska vågor, och därmed blev det snabbt etablerat att ljus är elektromagnetiska vågor. Därmed kommer vi ägna en hel sektion åt att diskutera de.

\subsection{Principer}

\paragraph{Essensiell information från Maxwells ekvationer}
Från Maxwells ekvation får man veta att
\begin{itemize}
	\item elektromagnetiska vågor är transversella.
	\item elektromagnetiska vågor utbredar sig med ljusfarten $c = \frac{1}{\sqrt{\varepsilon_0\mu_0}}$ i vakuum, och motsvarande i andra media.
	\item $\vect{B}\perp\vect{E}$.
	\item $B = \frac{1}{c}E$.
	\item $\vect{E}\times\vect{B}$ pekar i utbridningsriktningen.
\end{itemize}

\paragraph{Dispersion}
Dispersion är när ett mediums brytningsindex beror på vågens våglängd.

\paragraph{Polarisation}
Elektriskt och magnetiskt fält är vektorstorheter, och det betyder att de kan svänga i olika plan. Ljusets polarisation refererar till hur fälterna svängar i planet normalt på utbridgningen.

Ljus kan vara
\begin{itemize}
	\item linjärpolariserat, så att fältet oscillerar på en linje i det normala planet.
	\item elliptiskt polariserat, så att fältet oscillerar på en ellips i det normala planet.
	\item opolariserat, så att fältet oscillerar godtyckligt utan preferens för någon riktning.
\end{itemize}

\subsection{Ekvationer}

\paragraph{Vågekvationen för $\vect{E}$}
\begin{align*}
	\laplacian{\vect{E}} = \varepsilon\mu\pdv{\vect{E}}{t}
\end{align*}

\deriv

\paragraph{Definitionen av brytningsindex}
\begin{align*}
	n = \frac{c_0}{c}
\end{align*}
Här är $c_0$ ljusfarten i vakuum. Per definition är $n\geq 1$ i linjära materialer. Definitionen ger
\begin{align*}
	n = \sqrt{\varepsilon_{\text{r}}\mu_{\text{r}}}
\end{align*}
där $\varepsilon_{\text{r}}, \mu_{\text{r}}$ är relativa permeabiliteter och permittiviteter.

\paragraph{Reflektionskoefficient för elektromagnetiska vågor}
\begin{align*}
	R = \left(\frac{n_1 - n_2}{n_1 + n_2}\right)^2
\end{align*}

\paragraph{Våglängd i medium}
\begin{align*}
	\lambda = \frac{\lambda_0}{n}
\end{align*}
$\lambda_0$ är våglängden i vakuum. En direkt konsekvens av detta är att
\begin{align*}
	k = nk_0
\end{align*}
för vågvektorn, där $k_0$ är vågvektorns längd i vakuum.

\deriv
Vi har från innan att vid transmission mellan två medier ändras inte frekvens. Detta ger att
\begin{align*}
	\lambda = \frac{c}{f} = \frac{c_0}{nf} = \frac{\lambda_0}{n}
\end{align*}
där $c_0$ är vågfarten i vakuum.

\paragraph{Definitionen av optisk väg}
För en given bana definieras den optiska vägen ljus tar som
\begin{align*}
	L = \int\limits_{C}\dd{l}n,
\end{align*}
som för materialer med konstant brytningsindex reduceras till
\begin{align*}
	L = \sum L_in_i,
\end{align*}
där $L_i$ är sträckan vägen rör sig i materialet med brytningsindex $n_i$.

\paragraph{Poyntingvektorn}
\begin{align*}
	\vect{S} = \frac{1}{\mu_0}\vect{E}\times\vect{B}
\end{align*}
Poyntingvektorn ger information om energitransport från en elektromagnetisk våg. Dens riktning indikerar i vilken riktning energin transporteras, och dens belopp indikerar hur mycket energi som transporteras.

\deriv
I ett litet tidsintervall flödar energin
\begin{align*}
	\dd{U} = u\dd{V} = uAc\dd{t}
\end{align*}
genom arean $A$. För en elektromagnetisk våg har man
\begin{align*}
	u = \frac{1}{2}\varepsilon_0E^2 + \frac{1}{2\mu_0}B^2.
\end{align*}
Med relationen $B = \frac{E}{c}$ får man
\begin{align*}
	u = \varepsilon_0E^2,
\end{align*}
vilket ger
\begin{align*}
	\dd{U} = u\dd{V} = \varepsilon_0E^2Ac\dd{t}.
\end{align*}
Flödet per tidsenhet och area ges av
\begin{align*}
	S = \frac{1}{A}\dv{U}{t} = \frac{EB}{\mu_0}.
\end{align*}
Givet detta, definierar man
\begin{align*}
	\vect{S} = \frac{1}{\mu_0}\vect{E}\times\vect{B}
\end{align*}
som uppfyller
\begin{align*}
	\abs{\vect{S}} = \frac{EB}{\mu_0}
\end{align*}
och vars riktning även ger information om i vilken riktning energin transporteras.

\paragraph{Intensitet för elektromagnetisk våg}
\begin{align*}
	I = \frac{1}{2}\varepsilon_0cE^2
\end{align*}

\deriv'
Enligt definitionen är
\begin{align*}
	I = \expval{S}
\end{align*}