\section{Elektricitet}

\subsection{Ekvationer}

\paragraph{Definitionen av resistivitet}
\begin{align*}
	\rho = \frac{\abs{\vect{E}}}{\abs{\vect{J}}}
\end{align*}

\paragraph{Motstånd i en ledare}
\begin{align*}
	R = \frac{\rho L}{A}
\end{align*}

\paragraph{Seriekoppling av resistorer}
\begin{align*}
	R_\text{t} = \sum R_i
\end{align*}

\deriv

\paragraph{Parallellkoppling av resistorer}
\begin{align*}
	\frac{1}{R_\text{t}} = \sum \frac{1}{R_i}
\end{align*}

\deriv

\paragraph{Definitionen av kapacitans}
\begin{align*}
	C = \frac{Q}{V}
\end{align*}
$Q$ är beloppet av laddningen som lagras i kondensatorn (endast den positiva eller endast den negativa), och $V$ är spänningen som upprätthållas av kondensatorn.

\paragraph{Seriekoppling av kondensatorer}
\begin{align*}
	\frac{1}{C_\text{t}} = \sum\frac{1}{C_i}
\end{align*}

\deriv

\paragraph{Parallellkoppling av kondensatorer}
\begin{align*}
	C_\text{t} = \sum C_i
\end{align*}

\paragraph{Dielektrisk konstant}
\begin{align*}
	K = \frac{C}{C_0}
\end{align*}
$C_0$ betecknar kondensatorns kapacitans utan dielektrikum mellan plattarna. För en parallellplattkondensator har vi $K = \varepsilon_\text{r}$.

\paragraph{Kretser och elektromotiv kraft}
\begin{align*}
	V = \varepsilon - IR_\text{i}
\end{align*}
Den elektromotiva kraften $\varepsilon$ är spänningen som driver ström i kretsen. $R_\text{i}$ är kretsens inra motstånd, och då ges den faktiska spänningen i kretsen av denna formeln.

\deriv

\paragraph{Effektutveckling i en krets}
\begin{align*}
	P = VI
\end{align*}

\paragraph{Effektutveckling i resistiv komponent}
\begin{align*}
	P = I^2R = \frac{(\Delta V)^2}{R}
\end{align*}

\paragraph{Uppladdning av kondensator i RC-krets}
\begin{align*}
	&q = CV(1 - e^{-\frac{t}{RC}}), \\
	&I = \frac{V}{R}e^{-\frac{t}{RC}}
\end{align*}

\deriv

\paragraph{Utladdning av kondensator i RC-krets}
\begin{align*}
	&q = Q_0e^{-\frac{t}{RC}}), \\
	&I = -\frac{Q_0}{RC}e^{-\frac{t}{RC}}
\end{align*}

\subsection{Principer}