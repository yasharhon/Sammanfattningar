\section{Teckenkonvention i optik}
Utseendet till ekvationerna vi användar i optik ska beror på teckenkonventionen man användar. En möjlighet är att använda kartesisk teckenkonvention. Denna baseras på att
\begin{itemize}
	\item allt ljus kommer från höger mot vänster.
	\item Koordinatsystemet definieras med origo i centrum av den optiska komponenten, $x$-axeln med positiv riktning mot höger och $y$-axeln med positiv riktning uppåt.
\end{itemize}
Detta implicerar följande konvention:
\begin{table}[!ht]
	\makebox[\textwidth][c]
	{\begin{tabular}{| l | l | l |}
		\hline
		                           & +                                  & - \\
		\hline
		Objektavstånd              & Objekt till höger om optisk objekt & Objekt till vänster om optisk objekt \\
		\hline
		Bildavstånd                & Bild till höger om optisk objekt   & Bild till bänster om optisk objekt \\
		\hline
		Fokallängd för lins        & Samlar ljus till höger (konvex)    & Samlar ljus till vänster (konkav) \\
		\hline
		Fokallängd för spegel      & Centrum till höger (konvex)        & Centrum till vänster (konkav) \\
		\hline
		Fokallängd för sfärisk yta & Centrum till höger (konvex)        & Centrum till vänster (konkav) \\
		\hline
	\end{tabular}}
\end{table}

%Räkna på sfärisk yta

Alternativt kan man använda den så kallade "Real Is Positive"- konventionen (R.I.P) som används i kurslitteraturen. Denna definieras av följande tabell:
\begin{table}[!ht]
	\makebox[\textwidth][c]
	{\begin{tabular}{| l | l | l |}
		\hline
		            & + & - \\
		\hline
		Objektavstånd för linser       & Objekt till vänster om lins     & Objekt till höger om lins \\
		\hline
		Bildavstånd för linser         & Bild till höger om lins         & Bild till vänster om lins \\
		\hline
		Fokallängd för lins            & Samlar ljus till höger (konvex) & Samlar ljus till vänster (konkav) \\
		\hline
		Objektavstånd för speglar      & Objekt till vänster om spegel   & Objekt till höger om spegel \\
		\hline
		Bildavstånd för speglar        & Bild till vänster om spegel     & Bild till höger om spegel \\
		\hline
		Fokallängd för spegel          & Centrum till vänster (konkav)   & Centrum till höger (konvex) \\
		\hline
		Objektavstånd för sfärisk yta  & Objekt till vänster om yta      & Objekt till höger om yta \\
		\hline
		Bildavstånd för sfärisk yta    & Bild till höger om spegel       & Bild till vänster om yta \\
		\hline
		Krökningsradie för sfärisk yta & Centrum till höger              & Centrum till vänster \\
		\hline
	\end{tabular}}
\end{table}

Varför en konvention är klart överlegen är triviellt och lämnas som en övning till läsaren.