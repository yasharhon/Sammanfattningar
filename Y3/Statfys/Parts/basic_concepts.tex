\section{Basic Concepts in Statistical Physics}

\paragraph{The Thermodynamic Limit}
The thermodynamic limit is the limit of the statistical consideration of a system when the number of particle is large. In this limit, quantities such as temperature, pressure and density can be defined as we know them and macroscopic equilibria can be achieved.

\paragraph{Microstates}
A microstate of a system is any complete description of all particles in a system, for instance a specification of all positions and velocities of the particles in a gas.

\paragraph{Macrostates}
A macrostate of a system is a description of the macroscopic properties of a system.

\paragraph{Multiplicity}
The multiplicity of a macrostate is the number of microstates that yield the same macrostate.

\paragraph{The Fundamental Postulate}
The fundamental postulate of statistical mechanics is that all microstates available to a system are observed with equal probability.

\paragraph{Equilibrium and Multiplicity}
Combining the fundamental postulate with our knowledge of thermodynamics, it is clear that a system in thermal equilibrium is in the macrostate corresponding to maximal multiplicity.

\paragraph{The Gibbs Factor}
Consider a system of $N$ particles with internal energy $U$ exchanging energy and particles with a smaller system with energy $\varepsilon$ and $n$ particles. The entropy of the reservoir is
\begin{align*}
	S \approx S_{0} - \varepsilon\fix{\pdv{S}{U}}{N, V} - n\fix{\pdv{S}{N}}{U, V} = S_{0} - \frac{1}{T}\left(\varepsilon - \mu n\right).
\end{align*}
Writing this in terms of the multiplicity we have
\begin{align*}
	\ln{\Omega(E - \varepsilon, N - n)} &= \ln{\Omega(E, N)} - \frac{\varepsilon - \mu n}{\kb T}, \\
	\Omega(E - \varepsilon, N - n)      &= \Omega(E, N)e^{-\frac{\varepsilon - \mu n}{\kb T}},
\end{align*}
implying
\begin{align*}
	P(\varepsilon, n) \propto e^{-\beta\left(\varepsilon - \mu n\right)}.
\end{align*}