\section{Grunderna i elektrodynamik}

\paragraph{Lite om kvasistatiska fält}
Vi kommer här betrakta kvasistatiska fall, alltså fall där $\rho$ och $\vb{J}$ ändras långsamt. I såna fall kommer vi försöka bilda en dynamisk teori genom att extrapolera Biot-Savarts och Coulombs lagar trivialt, fast nu med integration av tidsberoende källor. Notera att det totala elektriska fältet även får en extra term från ändringen av magnetiska fältet, som vi kommer se.

\paragraph{Elektromagnetisk induktion}
Betrakta en strömslinga $C$ som rör sig godtyckligt och ändrar form under en liten tid $\dd{t}$. Vi är nu intresserade av ändringen i magnetiska flödet
\begin{align*}
	\dv{\Phi}{t} = \dv{t}\vinteg{S(t)}{}{a}{\vb{B}} = \lim\limits_{\dd{t}\to 0}\frac{1}{\dd{t}}\left(\vinteg{S(t + \dd{t})}{}{a}{\vb{B}(t + \dd{t})} - \vinteg{S(t)}{}{a}{\vb{B}(t)}\right).
\end{align*}
Vi serieutvecklar magnetfältet med avseende på tid (den följande variationen av koordinater tas med i det faktum att vi integrerar över olika ytor) för att få
\begin{align*}
	\dv{\Phi}{t} \approx \lim\limits_{\dd{t}\to 0}\frac{1}{\dd{t}}\left(\vinteg{S(t + \dd{t})}{}{a}{\vb{B}(t)} - \vinteg{S(t)}{}{a}{\vb{B}(t)} + \dd{t}\vinteg{S(t + \dd{t})}{}{a}{\del{t}{\vb{B}}(t)}\right).
\end{align*}
Låt nu kurvan vid $t$ och $t + \dd{t}$ förbindas av ytan $S_{\text{o}}$. Då fås
\begin{align*}
	\vinteg{S(t + \dd{t})}{}{a}{\vb{B}(t)} - \vinteg{S(t)}{}{a}{\vb{B}(t)} + \vinteg{S_{\text{o}}}{}{a}{\vb{B}(t)} = \integ{V}{}{\tau}{\div{\vb{B}}} = 0,
\end{align*}
där vi har lagt på ett minustecken för att ändra orienteringen på ena ytan. Om varje punkt på $C$ rör sig med en hastighet $\vb{v}$ fås
\begin{align*}
	\vinteg{S(t + \dd{t})}{}{a}{\vb{B}(t)} - \vinteg{S(t)}{}{a}{\vb{B}(t)} = -\integ{C(t)}{}{}{\vb{B}\cdot(\dd{\vb{l}}\times\vb{v}\dd{t})}.
\end{align*}
Vi har
\begin{align*}
	\vb{B}\cdot(\dd{\vb{l}}\times\vb{v}) = B_{i}\varepsilon_{ijk}\dd{l}_{j}v_{k} = \dd{l}_{j}\varepsilon_{jki}v_{k}B_{i} = \dd{\vb{l}}\cdot(\vb{v}\times\vb{B}).
\end{align*}
Detta ger slutligen
\begin{align*}
	\dv{\Phi}{t} = \vinteg{S(t + \dd{t})}{}{a}{\del{t}{\vb{B}}(t)} - \int\limits_{C(t)}{\dd{\vb{l}}\cdot(\vb{v}\times\vb{B})}.
\end{align*}

\paragraph{Rörlig slinga i statiskt fält}
Betrakta fallet då en slinga rör sig i ett statiskt fält. Vi får då
\begin{align*}
	\int\limits_{C(t)}{\dd{\vb{l}}\cdot(\vb{v}\times\vb{B})} = -\dv{\Phi}{t}.
\end{align*}
Vi känner igen vänstra termen som elektromotoriska kraften, alltså linjeintegralen av kraft per laddning över slingan, och får
\begin{align*}
	\emf = -\dv{\Phi}{t}.
\end{align*}

\paragraph{Varierande magnetiskt fält och fältlagar}
Betrakta en statisk strömslinga i ett varierande magnetiskt fält. Michael Faraday upptäckte att även en sån upplever en kraft. Hans hypotes var att detta berodde på att det inducerades en elektromotorisk spänning på grund av ett elektriskt fält. Mer specifikt,
\begin{align*}
	\vinteg{}{}{l}{\vb{E}} = -\vinteg{S}{}{a}{\del{t}{\vb{B}}(t)}.
\end{align*}

Maxwell generaliserade detta genom att ta bort den fysikaliska slingan och i stället betrakta en integrationsbana i rummet. Om Faradays hypotes stämde, skulle Stokes' sats ge
\begin{align*}
	\vinteg{S}{}{a}{\left(\curl{\vb{E}} + \del{t}{\vb{B}}(t)\right)} = 0.
\end{align*}
Om integrationsbanan är godtycklig, ger det
\begin{align*}
	\curl{\vb{E}} + \del{t}{\vb{B}}(t) = \vb{0}.
\end{align*}
Detta är en dynamisk fältlag för det elektriska och magnetiska fältet.

\paragraph{Elektriska fältets nya term}
Vi ser att elektriska fältet har nollskild rotation i dynamiken, så vi delar upp det i en rotationsfri del och en divergensfri del. Om vi nu tittar på den divergensfria delen, fås
\begin{align*}
	\vb{E} = -\frac{1}{4\pi}\integ{}{}{V}{\del{t}{\vb{B}}\times\frac{1}{R^{2}}\ub{R}}
\end{align*}
i analogi med Biot-Savarts lag. Den rotationsfria har redan beskrivits.

\paragraph{Potentialer för dynamiska fält}
Vi har för den rotationsfria terman att
\begin{align*}
	\curl{\vb{E}} + \del{t}{\vb{B}}(t) = \curl(\vb{E} + \del{t}{\vb{A}}) = \vb{0}.
\end{align*}
Detta ger
\begin{align*}
	\div(\vb{E} + \del{t}{\vb{A}}) = 0,\ \curl(\vb{E} + \del{t}{\vb{A}}) = \vb{0},
\end{align*}
och därmed måste
\begin{align*}
	\vb{E} = -\del{t}{\vb{A}}.
\end{align*}
Med generaliseringen av $\vb{A}$ fås
\begin{align*}
	\vb{E} = -\frac{\mu_{0}}{4\pi}\integ{}{}{\tau'}{\frac{1}{R}\del{t}{\vb{J}}}.
\end{align*}
Vi ser då att i dynamiken ges sambandet mellan fält och potentialer av
\begin{align*}
	\vb{B} = \curl{\vb{A}},\ \vb{E} = -\grad{V} - \del{t}{\vb{A}}.
\end{align*}

\paragraph{EMK från induktans}
Vi får i ett system med $n$ statiska slingor
\begin{align*}
	\emf_{i} = -\sum\limits_{j}M_{ij}\dv{I_{j}}{t} = -L_{i}\dv{I_{i}}{t} + \emf_{\text{övriga}}.
\end{align*}
Vi kan ställa upp detta med hjälp av egeninduktansen som
\begin{align*}
	\dv{I_{i}}{t} + \frac{R_{i}I_{i}}{L_{i}} = \frac{\emf_{\text{övriga}}}{L_{i}}.
\end{align*}

\paragraph{Magnetisk energi}
Den magnetiska energin $W_{\text{m}}$ är arbetet som krävs för att starta en strömkälla $\vb{J}$. Magnetiska fältet gör inget arbete i statiska fall, och därmed kan vi endast betrakta magnetisk energi i dynamiken.

Under igångsättningen finns ett inducerat elektriskt fält $\vb{E} = -\del{t}{\vb{A}}$. För att upprätthålla $\vb{J}$ mot det elektriska fältet tillförs effekten
\begin{align*}
	P = -\integ{}{}{\tau}{\vb{E}\cdot\vb{J}} = \integ{}{}{\tau}{\vb{J}\cdot\del{t}{\vb{A}}}.
\end{align*}
Med Ampères lag skriver vi detta som
\begin{align*}
	P &= \frac{1}{\mu_{0}}\integ{}{}{\tau}{(\curl{\vb{B}})\cdot\del{t}{\vb{A}}} \\
	  &= \frac{1}{\mu_{0}}\integ{}{}{\tau}{\div(\vb{B}\times\del{t}{\vb{A}}) + \vb{B}\cdot\curl(\del{t}{\vb{A}})} \\
	  &= \frac{1}{\mu_{0}}\vinteg{}{}{a}{\vb{B}\times\del{t}{\vb{A}}} + \frac{1}{\mu_{0}}\integ{}{}{\tau}{\vb{B}\cdot\del{t}{\vb{B}}}.
\end{align*}
När vi utvidgar integrationsvolymen mot oändligheten försvinner ytintegralen, och kvar står
\begin{align*}
	P = \frac{1}{2\mu_{0}}\dv{t}\integ{}{}{\tau}{B^{2}}.
\end{align*}
Därmed fås
\begin{align*}
	W_{\text{m}} = \frac{1}{2\mu_{0}}\integ{}{}{\tau}{B^{2}},
\end{align*}
som alternativt kan skrivas som
\begin{align*}
	W_{\text{m}} = \frac{1}{2}\integ{}{}{\tau}{\vb{J}\cdot\vb{A}}.
\end{align*}

\paragraph{Magnetisk energi i material}

\paragraph{Magnetisk energi för en slinga med tjocklek}
Vi delar strömslingan i små delar $V_{i}$ med tvärsnitt $S_{i}$. Magnetiska energin ges av
\begin{align*}
	W_{\text{m}} &= \frac{1}{2}\sum\limits_{i}\integ{V_{i}}{}{\tau}{\vb{J}_{i}\cdot\vb{A}}.
\end{align*}
Varje element för en ström $\vb{J}_{i}$, och bidrar då med fält $\vb{A}_{i}$ respektiva $\vb{B}_{i}$. Detta ger
\begin{align*}
	\frac{1}{2}\sum\limits_{i}\sum\limits_{j}\integ{V_{i}}{}{\tau}{\vb{J}_{i}\cdot\vb{A}_{j}} = \sum\limits_{i}\sum\limits_{j}W_{\text{m}, ij}.
\end{align*}
Egenenergierna $W_{\text{m}, ii}$ motsvarar uttrycken vi har härlett innan, alltså integralen av kvadratet av något bidrag till magnetiska fältet. Detta sättet är att föredra framför att räkna på flödet i slingor (vilket också skulle kunna funka om man delar upp strömtätheten i skivor).

Om vi vidare antar att $\vb{A}$ är ungefär konstant över varje tvärsnitt fås
\begin{align*}
	\vb{J}_{i}\dd{\tau} = \vb{J}_{i}S_{i}\cdot\dd{vb{l}} = I_{i}\dd{\vb{l}}.
\end{align*}
Därmed kan vi skriva
\begin{align*}
	W_{\text{m}, ij} = \frac{1}{2}I_{i}\vinteg{C_{i}}{}{l}{\vb{A}_{j}} = \frac{1}{2}I_{i}\Phi_{ij} = \frac{1}{2}I_{i}M_{ij}I_{j}.
\end{align*}
För systemet fås då
\begin{align*}
	W_{\text{m}} = \frac{1}{2}\sum\limits_{i}\sum\limits_{j}I_{i}M_{ij}I_{j}.
\end{align*}
Vi vet att detta är strikt positivt, så induktansmatrisen måste vara positivt definit.

\paragraph{Induktanser från energitermer}
Med detta kan vi skriva
\begin{align*}
	M_{ij} = \frac{2W_{\text{m}, ij}}{I_{i}I_{j}},\ L_{i} = \frac{2W_{\text{m}, ii}}{I_{i}^{2}}.
\end{align*}

\paragraph{Krafter på magneter}
Betrakta en magnet någonstans i närheten av en strömtäthet $\vb{J}$. Magnetens position beskrivs av någon referensvektor $\vb{r}$.

Betrakta magneten först. Om det förflyttas en sträcka $\dd{\vb{r}}$, görs ett arbete $\dd{W}$ av kraften $\vb{F}$. Kraftjämvikten ger
\begin{align*}
	\vb{F} + \vb{F}_{\text{m}} = \vb{0},
\end{align*}
och därmed
\begin{align*}
	\dd{W} = \vb{F}\cdot\dd{\vb{r}} = -\vb{F}_{\text{m}}\cdot\dd{\vb{r}},\ \vb{F}_{\text{m}} = -\grad{W}.
\end{align*}

Om systemet är isolerad får man att upplagrad magnetisk energi är enda energikällan som finns, vilket ger $\dd{W} = \dd{W}_{\text{m}}$ och
\begin{align*}
	\vb{F}_{\text{m}} = -\grad{W_{\text{m}}}.
\end{align*}
Jag tror att detta motsvarar att alla magnetiska flöden bevaras. Om detta är sant får man
\begin{align*}
	\vb{F}_{\text{m}} = \frac{1}{2}\frac{\Phi^{2}}{L^{2}}\grad{L} = \frac{1}{2}I^{2}\grad{L}.
\end{align*}

Om systemet i stället är anslutet till strömkällor så att alla strömmar hålls konstanta fås
\begin{align*}
	\grad{W_{\text{m}}} = \frac{1}{2}I^{2}\grad{L}.
\end{align*}
Samtidigt gör källan ett arbete
\begin{align*}
	\dd{W} = -I\dd{\Phi} + \dd{W_{\text{m}}}
\end{align*}
för att upprätthålla konstanta strömmar. Eftersom $\Phi = IL$ drar vi slutsatsen att $W = -W_{\text{m}}$, vilket ger
\begin{align*}
	\vb{F}_{\text{m}} = \grad{W_{\text{m}}} = \frac{1}{2}I^{2}\grad{L}.
\end{align*}

\paragraph{Generalisering av Ampères lag}
I statiska situationer har vi sett att Ampères lag är konsekvent med att strömmarna är divergensfria. I dynamiska situationer är inte strömmarna nödvändigtvis divergensfria, så vi kommer försöka göra en ny ad hoc dynamisk Ampères lag.

Genom att utgå från vår generalisering av Biot-Savarts lag fås
\begin{align*}
	\curl{\vb{B}} &= \frac{\mu_{0}}{4\pi}\integ{}{}{\tau'}{\curl(\vb{J}\times\frac{1}{R^{2}}\ub{R})} \\
	              &= \frac{\mu_{0}}{4\pi}\integ{}{}{\tau'}{\div(\frac{1}{R^{2}}\ub{R})\vb{J} - (\vb{J}\cdot\grad)\frac{1}{R^{2}}\ub{R}} \\
	              &= \mu_{0}\vb{J} + \frac{\mu_{0}}{4\pi}\integ{}{}{\tau'}{(\vb{J}\cdot\grad')\frac{1}{R^{2}}\ub{R}}.
\end{align*}
Vi betecknar den andra termen som $\vb{I}$. Komponentvis har vi
\begin{align*}
	I_{i} &= \frac{\mu_{0}}{4\pi}\integ{}{}{\tau'}{J_{j}\del{j}{\frac{1}{R^{3}}R_{i}}} \\
	      &= \frac{\mu_{0}}{4\pi}\integ{}{}{\tau'}{\del{j}{\left(\frac{1}{R^{3}}J_{j}R_{i}\right)} - \frac{1}{R^{3}}R_{i}\del{j}{J_{j}}} \\
	      &= \frac{\mu_{0}}{4\pi}\integ{}{}{a_{i}'}{\left(\frac{1}{R^{3}}J_{j}R_{i}\right)} - \frac{\mu_{0}}{4\pi}\integ{}{}{\tau'}{\frac{1}{R^{3}}R_{i}\div{\vb{J}}}.
\end{align*}
Genom att utvidga integrationsområdet mot oändligheten försvinner första termen, vilket ger
\begin{align*}
	\curl{\vb{B}} &= \mu_{0}\vb{J} - \frac{\mu_{0}}{4\pi}\integ{}{}{\tau'}{\div{\vb{J}}\frac{1}{R^{2}}\ub{R}}.
\end{align*}
Med kontinuitetsekvationen fås
\begin{align*}
	\curl{\vb{B}} &= \mu_{0}\vb{J} + \frac{\mu_{0}}{4\pi}\integ{}{}{\tau'}{\del{t}{\rho}\frac{1}{R^{2}}\ub{R}}.
\end{align*}
Vi känner igen andra termen som proportionell mot en tidsderivata av elektriska fältet, och en möjlig kandidat till en ny Ampères lag är
\begin{align*}
	\curl{\vb{B}} - \mu_{0}\varepsilon_{0}\del{t}{\vb{E}} = \mu_{0}\vb{J}.
\end{align*}

Observera att detta inte är en härledning, då vi har utgått från dessa generaliserade varianterna av Coulombs och Biot-Savarts lagar. Dessa uppfyller till exempel inte $\curl{\vb{E}} = -\del{t}{\vb{B}}$. För att uppfylla detta måste induktiva korrektionstermen läggas till, men då kommer inte den nya Ampères lag att gälla, så man får en jobbig iterativ process med korrektioner.

\paragraph{Maxwells ekvationer}
All vår kunnskap om elektrostatik- och dynamik slås nu ihop för att skriva ned de ekvationerna som beskriver det vi vet:
\begin{align*}
	\curl{\vb{E}} + \del{t}{\vb{B}} = \vb{0}, \\
	\div{\vb{B}} = 0, \\
	\curl{\vb{B}} - \mu_{0}\varepsilon_{0}\del{t}{\vb{E}} = \mu_{0}\vb{J}, \\
	\div{\vb{E}} = \frac{\rho}{\varepsilon_{0}}.
\end{align*}
Dessa kallas för Maxwells ekvationer.

\paragraph{Poyntings sats}
Betrakta en volym $V$ som omkransas av en yta $S$. Joules lag ger att $V$ tillförs effekten
\begin{align*}
	P_{\text{mek}} = \integ{V}{}{}{\vb{J}\cdot\vb{E}}.
\end{align*}
Vi kan tolka detta som
\begin{align*}
	\vb{J}\cdot\vb{E} = \del{t}{w_{\text{mek}}},
\end{align*}
där $w_{\text{mek}}$ är den mekaniska energitätheten. Med hjälp av Maxwells ekvationer skriver vi
\begin{align*}
	\del{t}{w_{\text{mek}}} &= \frac{1}{\mu_{0}}\vb{E}\cdot\curl{\vb{B}} - \varepsilon_{0}\vb{E}\cdot\del{t}{\vb{E}} \\
	                        &= \frac{1}{\mu_{0}}\left(\vb{B}\cdot\curl{\vb{B}} - \div(\vb{E}\times\vb{B})\right) - \frac{1}{2}\varepsilon_{0}\cdot\del{t}{E^{2}} \\
	                        &= -\frac{1}{\mu_{0}}\div(\vb{E}\times\vb{B}) - \del{t}{\left(\frac{1}{2}\varepsilon_{0}\cdot\del{t}{E^{2}} + \frac{1}{2\mu_{0}}B^{2}\right)}.
\end{align*}
Vi har en fältenergitäthet från två av termerna, som vi förstår väl, men det finns även en extra term här. Den är en flödestäthet av energi. Vi kallar den Poyntings vektor
\begin{align*}
	\vb{S} = \frac{1}{\mu_{0}}\vb{E}\times\vb{B}.
\end{align*}
Den uppfyller konserveringslagen
\begin{align*}
	\div{\vb{S}} + \del{t}{(w_{\text{mek}} + w_{\text{em}})} = 0.
\end{align*}

\paragraph{Rörelsemängd i elektromagnetism}
Betrakta ett föremål med laddningstäthet $\rho$ och strömtäthet $\vb{J}$. Detta har rörelsemängd som kommer från laddningarnas rörelse. Om vi inför rörelsemängdstätheten $\vb{g}_{\text{mek}}$ ger Newtons andra lag
\begin{align*}
	\del{t}{\vb{g}_{\text{mek}}} = \rho\vb{E} + \vb{J}\times\vb{B}.
\end{align*}
Maxwells ekvationer ger
\begin{align*}
	\del{t}{\vb{g}_{\text{mek}}} = \varepsilon_{0}\vb{E}(\div{\vb{E}}) + \frac{1}{\mu_{0}}(\curl{\vb{B}})\times\vb{B} - \varepsilon_{0}\del{t}{\vb{E}}\times\vb{B}.
\end{align*}
Vi skriver med hjälp av Maxwells ekvationer
\begin{align*}
	(\curl{\vb{B}})\times\vb{B} &= -\frac{1}{2}\grad{B^{2}} + (\vb{B}\cdot\grad)\vb{B}, \\
	\del{t}{\vb{E}}\times\vb{B} &= \del{t}{(\vb{E}\times\vb{B})} - \vb{E}\times\del{t}{\vb{B}} \\
	                            &= \del{t}{(\vb{E}\times\vb{B})} + \vb{E}\times(\curl{\vb{E}}) \\
	                            &= \del{t}{(\vb{E}\times\vb{B})} + \frac{1}{2}\grad{E^{2}} - (\vb{E}\cdot\grad)\vb{E},
\end{align*}
vilket, till sammans med Maxwells ekvationer, ger
\begin{align*}
	\del{t}{\vb{g}_{\text{mek}}} &= \varepsilon_{0}\vb{E}(\div{\vb{E}}) + \frac{1}{\mu_{0}}\left(-\frac{1}{2}\grad{B^{2}} + (\vb{B}\cdot\grad)\vb{B}\right) - \varepsilon_{0}\left(\del{t}{(\vb{E}\times\vb{B})} + \frac{1}{2}\grad{E^{2}} - (\vb{E}\cdot\grad)\vb{E}\right) \\
	                             &= \varepsilon_{0}\vb{E}(\div{\vb{E}}) + \frac{1}{\mu_{0}}(\vb{B}\cdot\grad)\vb{B} - \varepsilon_{0}\left(\del{t}{(\vb{E}\times\vb{B})} - (\vb{E}\cdot\grad)\vb{E}\right) - \frac{1}{2}\grad(\varepsilon_{0}E^{2} + \frac{1}{\mu_{0}}B^{2}) \\
	                             &= \varepsilon_{0}(\vb{E}(\div{\vb{E}}) + (\vb{E}\cdot\grad)\vb{E}) + \frac{1}{\mu_{0}}(\vb{B}\div{\vb{B}} + (\vb{B}\cdot\grad)\vb{B}) - \grad(\frac{1}{2}\varepsilon_{0}E^{2} + \frac{1}{2\mu_{0}}B^{2}) - \varepsilon_{0}\mu_{0}\del{t}{\left(\frac{1}{\mu_{0}}\vb{E}\times\vb{B}\right)} \\
	                             &= \varepsilon_{0}(\vb{E}(\div{\vb{E}}) + (\vb{E}\cdot\grad)\vb{E}) + \frac{1}{\mu_{0}}(\vb{B}\div{\vb{B}} + (\vb{B}\cdot\grad)\vb{B}) - \grad(w_{\text{e}} + w_{\text{m}}) - \varepsilon_{0}\mu_{0}\del{t}{\vb{S}}.
\end{align*}
Notera att vi adderade en term som är $\vb{0}$.

\paragraph{Elektromagnetisk rörelsemängd och Maxwells spänningstensor}
Vi betraktar nu den mekaniska rörelsemängden komponentvis:
\begin{align*}
	\del{t}{g_{\text{mek}, i}} &= \varepsilon_{0}(E_{i}\del{j}{E_{j}} + E_{j}\del{j}{E_{i}}) + \frac{1}{\mu_{0}}(B_{i}\del{j}{B_{j}} + B_{j}\del{j}{B_{i}}) - \del{i}{w_{\text{em}}} - \varepsilon_{0}\mu_{0}\del{t}{S_{i}} \\
	                                &= \del{j}{\left(\varepsilon_{0}E_{i}E_{j} + \frac{1}{\mu_{0}}B_{i}B_{j} - w_{\text{em}}\delta_{ij}\right)} - \varepsilon_{0}\mu_{0}\del{t}{S_{i}}.
\end{align*}
Vi definierar då elektromagnetiska rörelsemängdstätheten
\begin{align*}
	\vb{g}_{\text{em}} = \mu_{0}\varepsilon_{0}\vb{S}
\end{align*}
och Maxwells spänningstensor
\begin{align*}
	T_{ij} = \varepsilon_{0}E_{i}E_{j} + \frac{1}{\mu_{0}}B_{i}B_{j} - \frac{1}{2}\delta_{ij}\left(\varepsilon_{0}E^{2} + \frac{1}{\mu_{0}}B^{2}\right).
\end{align*}
Då kan vi skriva
\begin{align*}
	\del{t}{(g_{\text{mek}, i} + g_{\text{em}, i})} + \del{j}{(-T_{ij})} = 0,
\end{align*}
alltså en kontinuitetsekvation.

Om man vill kan man skriva
\begin{align*}
	\del{j}{T_{ij}} = \delta_{jk}\del{k}{T_{ij}} = \ub{j}\cdot\ub{k}\del{k}{T_{ij}} = (\ub{k}\del{k}){}\cdot({T_{ij}\ub{j}}) = \div{{T_{ij}\ub{j}}},
\end{align*}
och vektorfältet som transporterar komponent $i$ av rörelsemängden är $-\div{{T_{ij}\ub{j}}}$. Men att skriva så är lite fult, och ekvationen vi började med är en fullgod kontinuitetsekvation.

\paragraph{Flöde av rörelsemängd och ytspänningsvektorer}
I ett område har vi
\begin{align*}
	\dv{t}(p_{\text{mek}, i} + p_{\text{em}, i}) &= \integ{}{}{\tau}{\del{t}{g_{\text{mek}, i} + g_{\text{em}, i}}} \\
	                                             &= \integ{}{}{\tau}{\del{j}{T_{ij}}} \\
	                                             &= \integ{}{}{a_{j}}{T_{ij}}.
\end{align*}
Vi skriver integranden som
\begin{align*}
	T_{ij}n_{j} &= \varepsilon_{0}E_{i}E_{j}n_{j} + \frac{1}{\mu_{0}}B_{i}B_{j}n_{j} - \frac{1}{2}\delta_{ij}\left(\varepsilon_{0}E^{2} + \frac{1}{\mu_{0}}B^{2}\right)n_{j} \\
	            &= \varepsilon_{0}\vb{n}\cdot\vb{E}E_{i} + \frac{1}{\mu_{0}}\vb{n}\cdot\vb{B}B_{i} - \frac{1}{2}\left(\varepsilon_{0}E^{2} + \frac{1}{\mu_{0}}B^{2}\right)n_{i}.
\end{align*}
Vi definierar den elektriska och magnetiska ytspänningsvektorn som
\begin{align*}
	\vb{T}_{\text{e}} = \varepsilon_{0}(\vb{n}\cdot\vb{E})\vb{E} - \frac{1}{2}\varepsilon_{0}E^{2}\vb{n},\ \vb{T}_{\text{m}} = \frac{1}{\mu_{0}}(\vb{n}\cdot\vb{B})\vb{B} - \frac{1}{2\mu_{0}}B^{2}\vb{n}.
\end{align*}
Vi ser då att flödet av dessa ut genom området ger ändringen av total rörelsemängd. I många praktiska fall är tidsderivatan av elektromagnetisk rörelsemängd försumbar, och kvar står en flödesintegral på ena sidan och summan av alla krafter på andra.

\paragraph{Geometrisk konstruktion av ytspänningsvektorerna}
Det gäller att
\begin{align*}
	T_{\text{e}} = w_{\text{e}}
\end{align*}
och att om vinkeln mellan $\vb{E}$ och $\vb{n}$ är $\alpha$, bildar ytspänningsvektorn vinkeln $\alpha$ med $\vb{E}$ och $2\alpha$ med $\vb{n}$. Den ligger även i samma plan som $\vb{E}$ och $\vb{n}$. Det samma gäller för den magnetiska ytspänningsvektorn.

Vi visar detta så rakt fram som vi kan för den elektriska ytspänningsvektorn. Det sista påståendets sannhet är uppenbar eftersom $\vb{T}_{\text{e}}$ är en linjärkombination av $\vb{E}$ och $\vb{n}$. Vi har vidare
\begin{align*}
	T_{\text{e}}^{2} &= \varepsilon_{0}^{2}\left((\vb{n}\cdot\vb{E})\vb{E} - \frac{1}{2}E^{2}\vb{n}\right)^{2} \\
	                 &= \varepsilon_{0}^{2}\left((\vb{n}\cdot\vb{E})^{2}E^{2} + \frac{1}{4}E^{4}\vb{n}^{2} - (\vb{n}\cdot\vb{E})\vb{E}\cdot E^{2}\vb{n}\right) \\
	                 &= \varepsilon_{0}^{2}\left((\vb{n}\cdot\vb{E})^{2}E^{2} + \frac{1}{4}E^{4} - (\vb{n}\cdot\vb{E})^{2}E^{2}\right) \\
	                 &= \frac{1}{4}\varepsilon_{0}^{2}E^{4} \\
	                 &= w_{\text{e}}^{2},
\end{align*}
vilket visar det första påståendet. Vi har vidare
\begin{align*}
	\vb{T}_{\text{e}}\cdot\vb{E} = \varepsilon_{0}(\vb{n}\cdot\vb{E})E^{2} - \frac{1}{2}\varepsilon_{0}E^{2}(\vb{n}\cdot\vb{E}) = \frac{1}{2}\varepsilon_{0}E^{2}(\vb{n}\cdot\vb{E}) = (\vb{n}\cdot\vb{E})w_{\text{e}},
\end{align*}
varför
\begin{align*}
	\cos{\alpha} = \frac{\vb{n}\cdot\vb{E}}{E}.
\end{align*}
Jämför detta med
\begin{align*}
	\vb{T}_{\text{e}}\cdot\vb{n} = \varepsilon_{0}(\vb{n}\cdot\vb{E})^{2} - \frac{1}{2}\varepsilon_{0}E^{2}n^{2} = \frac{1}{2}\varepsilon_{0}E^{2}(2\cos[2](\alpha) - 1) = T_{\text{e}}\cos{2\alpha}.
\end{align*}
Beviset är helt analogt för den magnetiska ytspänningsvektorn.

\paragraph{Elektromagnetiskt rörelsemängdsmoment}
Vi har
\begin{align*}
	\vb{N}_{\text{mek}} = \integ{}{}{\tau}{\vb{r}\times\del{t}{\vb{g}_{\text{mek}}}} = \integ{}{}{\tau}{\del{t}{\vb{l}_{\text{mek}}}},
\end{align*}
där vi har infört tätheten av rörelsemängdsmoment
\begin{align*}
	\vb{l}_{\text{em}} = \vb{r}\times\vb{g}_{\text{em}}.
\end{align*}
Vi vill gärna ha en konserveringslag för denna, och använder konserveringslagen för rörelsemängd för att skriva
\begin{align*}
	\del{t}{(l_{\text{mek}, i} + l_{\text{em}, i})} &= \varepsilon_{ijk}r_{j}\del{t}{(g_{\text{mek}, k} + g_{\text{em}, k})} \\
	                                                &= \varepsilon_{ijk}r_{j}\del{p}{T_{kp}} \\
	                                                &= \del{p}{(\varepsilon_{ijk}r_{j}T_{kp})} - \varepsilon_{ijk}T_{kp}\del{p}{r_{j}} \\
	                                                &= \del{p}{(\varepsilon_{ijk}r_{j}T_{kp})} - \varepsilon_{ijk}T_{kj}.
\end{align*}
Eftersom Maxwells spänningstensor är symmetrisk, försvinner andra termen. Vi skriver först om
\begin{align*}
	\del{p}{(\varepsilon_{ijk}r_{j}T_{kp})} = \del{j}{(\varepsilon_{ipk}r_{p}T_{kj})} = -\del{j}{(\varepsilon_{ikp}r_{p}T_{kj})}
\end{align*}
och definierar
\begin{align*}
	M_{ij} = \varepsilon_{ikp}r_{p}T_{kj}.
\end{align*}
Detta ger
\begin{align*}
	\del{t}{(l_{\text{mek}, i} + l_{\text{em}, i})} + \del{j}{M_{ij}} = 0.
\end{align*}

\paragraph{Flöde av elektromagnetiskt rörelsemängdsmoment}
Vi har
\begin{align*}
	\dv{t}(L_{\text{mek}, i} + L_{\text{em}, i}) &= \integ{}{}{\tau}{\del{t}{(l_{\text{mek}, i} + l_{\text{em}, i})}} \\
	                                             &= -\integ{}{}{\tau}{\del{j}{M_{ij}}} \\
	                                             &= -\integ{}{}{a_{j}}{M_{ij}} \\
	                                             &= -\integ{}{}{a_{j}}{\varepsilon_{ikp}r_{p}T_{kj}},
\end{align*}
vilket ger
\begin{align*}
	\dv{t}\vb{L} = -\integ{}{}{a}{(\vb{T}_{\text{e}} + \vb{T}_{\text{m}})\times\vb{r}}.
\end{align*}