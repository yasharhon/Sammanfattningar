\section{Antenner}

\paragraph{Elektriska dipolantenner}
En enkel form för antenn konstrueras genom att placera två motsatta variabla punktladdningar på ett avstånd $d$ från varandra och förbinda dessa med en ström $I$. Genom att orientera $z$-axeln parallell med vektorn som förbinder punktladdningarna fås ett dipolmoment och en ström
\begin{align*}
	\vb{p} = q(t)d\ub{z},\ I = \del{t}{q}.
\end{align*}
Från detta kan man i gränsen $d\to 0,\ qd\to p$ visa att
\begin{align*}
	\rho = -\vb{p}\cdot\grad{\delta(\vb{r})},\ \vb{J} = \vb{p}\delta(\vb{r}).
\end{align*}
Genom att studera fältet från en dipol i Lorenzgaugen fås
\begin{align*}
	\vb{E}(\vb{r}, t) &= \frac{3}{4\pi\varepsilon_{0}}\frac{\ub{r}\cdot(\vb{p} + \frac{r}{c}\dot{\vb{p}})\ub{r} - (\vb{p} + \frac{r}{v}\dot{\vb{p}})}{r^{3}} + \frac{\mu_{0}}{4\pi}\frac{(\dot{\vb{p}}\cdot\ub{r})\ub{r} - \ddot{\vb{p}}}{r}, \\
	\vb{B}(\vb{r}, t) &= \frac{\mu_{0}}{4\pi r^{2}}\left(\dot{\vb{p}} + \frac{r}{c}\ddot{p}\right)\times\ub{r},
\end{align*}
där alla $\vb{p}$ och derivator av denna evalueras vid $t_{\text{r}} = t - \frac{r}{c}$.

Fälten uppfyller
\begin{align*}
	\vb{B} = \frac{1}{c}\ub{r}\times\vb{E},\ \vb{E} = -c\ub{r}\times\vb{B}.
\end{align*}

\paragraph{Effekttransport från en elektrisk dipolantenn}
Man kan visa att genom en sfär med radien $R$ som är koncentrisk med dipolen fås
\begin{align*}
	P = \dv{W_{\text{när}}}{t} + P_{\text{strål}},
\end{align*}
där
\begin{align*}
	W_{\text{när}} = \frac{\mu_{0}}{6\pi c}\left(\frac{c^{3}p^{2}}{2R^{3}} + \frac{c^{2}}{R^{2}}(\vb{p}\cdot\dot{\vb{p}}) + \frac{c}{R}\dot{p}^{2}\right),\ P_{\text{strål}} = \frac{\mu_{0}}{6\pi\varepsilon_{0}c}\ddot{p}^{2}.
\end{align*}
Det är bara den sista termen, strålningstermen, som når ut till oändligheten.

\paragraph{Magnetiska dipolantenner}
En annan enkel form för antenn konstrueras genom att lägga en sladd i en cirkulär slinga. Om cirkeln har radie $a$ och för strömmen $I$ fås dipolmomentet
\begin{align*}
	\vb{m} = \pi a^{2}I\ub{z}.
\end{align*}
I gränsen $a\to 0, \pi a^{2}I\to m$ fås magnetiseringen
\begin{align*}
	\vb{M} = \vb{m}\delta(\vb{r}).
\end{align*}
Den motsvarande strömtätheten är
\begin{align*}
	\vb{J} = \curl{\vb{M}} = \grad{\delta}\times\vb{m}.
\end{align*}
Från detta kan man visa att
\begin{align*}
	\vb{B} = -\frac{\mu_{0}}{4\pi}\frac{\del{t}{\vb{m}} + \frac{r}{c}\del[2]{t}{\vb{m}}}{r^{2}}\times\ub{r},\ \vb{B} = \frac{\mu_{0}}{4\pi r}\left(\frac{3\left(\ub{r}\cdot\left(\vb{m} + \frac{r}{c}\del{t}{\vb{m}}\right)\right)\ub{r}}{r^{2}} + \frac{(\ub{r}\cdot\del[2]{t}{\vb{m}}))\ub{r} - \del[2]{t}{\vb{m}}}{c^{2}}\right).
\end{align*}

\paragraph{Raka trådantenner}
En annan enkel antenn kan konstrueras genom att föra en varierande ström genom en rak tråd med längd $h$ (orienterad längs med $z$-axeln).

\paragraph{Raka trådantenner med tidsharmonisk ström}
Vi antar att strömmen i tråden är tidsharmonisk, och söker elektriska och magnetiska fältet för $r \gg h, \lambda$. Vi kan då använda resultaten från analyser av tidsharmoniska strömfördelningar för att få
\begin{align*}
	\vb{B} = j\frac{\mu_{0}}{4\pi}\frac{e^{-jkr}}{r}\fou{\vb{J}}\times\vb{k},
\end{align*}
där strömtätheten ges av
\begin{align*}
	\vb{J}(\vb{r}') = I(z')\delta(\vb{s}')(H(z' + \frac{1}{2}h) - H(z' - \frac{1}{2}h))\ub{z}.
\end{align*}
Genom att Fouriertransformera strömtätheten fås
\begin{align*}
	\vb{B} = j\frac{\mu_{0}\omega}{4\pi c}\frac{e^{-jkr}}{r}\sin{\phi}\integ{-\frac{1}{2}h}{\frac{1}{2}h}{z'}{I(z')e^{j\frac{z'}{\lambda}2\pi\cos{\theta}}}\ub{\phi}.
\end{align*}

\paragraph{Mittpunktsmatade trådantenner}
För en mittpunktsmatad trådantenn antar vi att strömmen tar slut i änderna av tråden. Den approximativa strömtätheten ges av
\begin{align*}
	I(z') = I_{0}\sin(\frac{2\pi}{\lambda}\left(\frac{1}{2}h - \abs{z'}\right)).
\end{align*}
Det motsvarande magnetfältet blir då
\begin{align*}
	\vb{B} = j\frac{\mu_{0}I_{0}}{2\pi}\frac{e^{-jkr}}{r}\frac{\cos(\pi\cos{\theta}\frac{h}{\lambda}) - \cos(\pi\frac{h}{\lambda})}{\sin{\theta}}\ub{\phi}.
\end{align*}

\paragraph{Gruppantenner}
En gruppantenn är en mängd av $N$ strömfördelningar som alla befinner sig inom ett avstånd $w_{\text{max}}$ från varandra.

\paragraph{Fjärrfält från en gruppantenn av dipoler}
Betrakta en uppställning av dipoler med dipolmoment $\vb{p}_{n}$ centrerade i punkterna $\vb{w}_{n}$. I fjärrzonen till samtliga dipoler ger superposition
\begin{align*}
	\vb{B} = \frac{\mu_{0}}{4\pi c}\sum\limits_{n}\frac{1}{R_{n}}\del[2]{t}{\vb{p}_{n}}(t_{n})\times\ub{R_{n}},\ t_{n} = t - \frac{R_ {n}}{c}.
\end{align*}
I fjärrzonen kan vi använda parallellapproximationen $R_{n} \approx r - \ub{r}\cdot\vb{w}_{n},\ \ub{R_{n}} \approx \ub{r}$ för att få
\begin{align*}
	\vb{B} \approx -\frac{\mu_{0}}{4\pi c}\ub{r}\times\left(\sum\limits_{n}\frac{1}{R_{n}}\del[2]{t}{\vb{p}_{n}}\left(t_{\text{r}} + \frac{\ub{r}\cdot\vb{w}_{n}}{c}\right)\right) \approx -\frac{\mu_{0}}{4\pi cr}\ub{r}\times\left(\sum\limits_{n}\del[2]{t}{\vb{p}_{n}}\left(t_{\text{r}} + \frac{\ub{r}\cdot\vb{w}_{n}}{c}\right)\right),
\end{align*}
där vi har infört retarderade tiden för origo $t_{\text{r}} = t - \frac{r}{c}$.

Om dipolen har harmoniskt tidsberoende $\vb{p}_{n} = p_{n}\ub{p_{n}}e^{j(\omega t + \alpha_{n})}$ kan vi skriva
\begin{align*}
	\vb{B} &= -\frac{\mu_{0}}{4\pi cr}\ub{r}\times\left(\sum\limits_{n}-\omega^{2}p_{n}\ub{p_{n}}e^{j\alpha_{n} + j\omega\left(t_{\text{r}} + \frac{\ub{r}\cdot\vb{w}_{n}}{c}\right)}\right) \\
	       &= \frac{\mu_{0}\omega^{2}}{4\pi cr}e^{j\omega t_{\text{r}}}\ub{r}\times\left(\sum\limits_{n}p_{n}\ub{p_{n}}e^{j\alpha_{n} + j\omega\frac{\ub{r}\cdot\vb{w}_{n}}{c}}\right).
\end{align*}
Om man inför vågvektorn $k = \frac{\omega}{c}\ub{r}$ och bortser från harmoniska tidsberoendet kan detta skrivas som
\begin{align*}
	\vb{B} = \frac{\mu_{0}\omega^{2}}{4\pi cr}e^{-jkr}\sum\limits_{n}e^{j\alpha_{n} + j\vb{k}\cdot\vb{w}_{n}}\vb{f}_{n}(\theta, \phi).
\end{align*}
Om alla delarna är lika och likadant orienterade fås $\vb{f}_{n} = a_{n}\ub{f}$, vilket ger
\begin{align*}
	\vb{B} = \frac{\mu_{0}\omega^{2}}{4\pi cr}e^{-jkr}\ub{f}\sum\limits_{n}a_{n}e^{j\alpha_{n} + j\vb{k}\cdot\vb{w}_{n}}.
\end{align*}
Här dyker det upp en elementfaktor $\ub{f}$ som beskriver antennens orientering och en gruppfaktor $G = \sum\limits_{n}a_{n}e^{j\alpha_{n} + j\vb{k}\cdot\vb{w}_{n}}$ som beskriver hur de samverkar.