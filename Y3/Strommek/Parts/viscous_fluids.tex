\section{Viskösa fluider}

\paragraph{Töjningstensorn}
Betrakta ett litet fluidelement som rör sig i någon riktning $x_{i}$. Över elementets längd $\cha{x}$ ändras hastigheten med $\del{i}{u_{i}}\cha{x_{i}}$. Över ett litet tidsintervall $\cha{t}$ kommer då fluidelementet att förlängas med $\cha{s} = \del{i}{u_{i}}\cha{x_{i}}\cha{t}$. Den linjära töjningen definieras som förlängningen per längd och tid, och ges här i $x_{i}$-riktningen av $\del{i}{u_{i}}$.

Betrakta vidare ett fluidelement i ett hastighetsfält i $x_{1}x_{2}$-planet. Det kommer skjuvas över en tid $\cha{t}$ så att det bildar en vinkel $\cha{\alpha}$ med $x_{2}$-axeln och $\cha{\beta}$ med $x_{1}$-axeln. Trigonometri ger
\begin{align*}
	\cha{\alpha} = \frac{(u_{1} + \del{2}{u_{1}}\cha{x_{2}} - u_{1})\cha{t}}{\cha{x_{2}}} = \del{2}{u_{1}}\cha{t},\ \cha{\beta} = \del{1}{u_{2}}\cha{t}.
\end{align*}
Skjuvningen per tid ges av $\del{1}{u_{2}} + \del{2}{u_{1}}$.

Vi definierar nu töjningstensorn
\begin{align*}
	e_{ij} = \del{i}{u_{j}} + \del{j}{u_{i}}.
\end{align*}
Från denna kan vi få töjningen av ett givet element.

Vi noterar två saker: Töjningstensorn är symmetrisk, och för inkompressibla vätskor är den spårlös. Som med andra tensorer finns det även ett huvudaxelsystem där töjningstensorn är diagonal.

\paragraph{Friktion i fluider}
Friktion i fluider uppkommer vid töjning. Vi kommer behandla den som om den är proportionell mot töjningen.

\paragraph{Konstitutiva relationer och Newtonska fluider}
För en inkompressibel fluid kommer vi arbeta med den konstitutiva relationen
\begin{align*}
	\tau_{ij} = -p\delta_{ij} + 2\mu e_{ij}.
\end{align*}
En inkompressibel vätska som uppfyller denna konstitutiva relationen kallas för en Newtonsk fluid.

\paragraph{Viskositet}
I relationen ovan införde vi viskositeten $\mu$.

\paragraph{Navier-Stokes' ekvation för en Newtonsk fluid}.
Vi har
\begin{align*}
	\del{j}{\tau_{ij}} = -\del{i}{p} + \mu(\del{j}{\del{j}{u_{i}}} + \del{j}{\del{i}{u_{j}}}).
\end{align*}
Genom att ordna om derivatorna innehåller den sista termen en derivata av hastighetsfältets divergens. Eftersom vi studerar Newtonska fluider är denna $0$, och
\begin{align*}
	\del{j}{\tau_{ij}} = -\del{i}{p} + \mu\del{j}{\del{j}{u_{i}}}.
\end{align*}
Vi inför nu kinematiska ekvationen $\nu = \frac{\mu}{\rho}$, och skriver då kraftekvationen som
\begin{align*}
	\mdv{u_{i}} = -\frac{1}{\rho}\del{i}{p} + \nu\del{j}{\del{j}{u_{i}}} + g_{i}.
\end{align*}
Detta är Navier-Stokes' ekvation(er) för en Newtonsk fluid. På vektorform är den
\begin{align*}
	\del{t}{\vb{u}} + (\vb{u}\cdot\grad)\vb{u} = -\frac{1}{\rho}\grad{p} + \nu\laplacian{\vb{u}} + \vb{g}.
\end{align*}

Eftersom det finns friktion i vätskan, har ekvationen som randvillkor att $\vb{u} = \vb{0}$ på fasta ränder, eftersom vätskan kommer röra sig med randen.

\paragraph{Förenkling genom borttagning av kraftterm}
Antag att gravitationen inte driver flödet av en vätska utan bara sätter upp ett tryckfält i vätskan, och att detta är enda yttre kraften på vätskan. Då kan vi skriva $p = p' + p_{g}$, där $p_{g}$ är trycket som uppstår på grund av gravitationen. Denna termen uppfyller $\rho\vb{g} - \grad{p_{g}} = 0$. Då blir Navier-Stokes' ekvation
\begin{align*}
	\del{t}{\vb{u}} + (\vb{u}\cdot\grad)\vb{u} = -\frac{1}{\rho}\grad{p'} + \nu\laplacian{\vb{u}}.
\end{align*}
Primmet tas oftast ej med.