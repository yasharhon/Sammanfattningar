\section{Tillståndsrepresentationer}

\paragraph{Ideen}
Den fundamentala ideen vi vill åt nu är att representera ett systems tillstånd på ett annat sätt än bara utsignalens tidsutveckling. Vi vill även inkludera andra storheter i systemet.

\paragraph{Kort om notation}
$y$ kommer fortfarande att representera utsignalen och $u$ insignalen. $x$ kommer vara systemets tillstånd, karakteriserad av ett visst antal storheter. Däremot skulle en eller flera av dessa vara vektorer. Just för att detta kan variera från system till system och betraktning till betraktning, väljer jag att vara ambivalent med notationen.

\paragraph{Linjarisering}
Verkliga system är ofta olinjära, men vi ska försöka behandla dem som linjära ändå.

Betrakta ett system
\begin{align*}
	\dot{x} = f(x, u),\ \dot{y} = g(x, u).
\end{align*}
Anta konstant styrsignal $u_{0}$, och antag att systemet då tenderar mot ett stationärt tillstånd $\vb{x}_{0}$. Denna punkten uppfyller då
\begin{align*}
	f(x_{0}, u_{0}) = \vb{0},\ h(x_{0}, u_{0}) = y_{0}.
\end{align*}
När vi linjäriserar, betraktar vi små variationer $\Delta x, \Delta u, \Delta y$ kring denna punkten. Vi får
\begin{align*}
	\dv{\Delta x}{t} = f(x_{0} + \Delta x, u_{0} + \Delta u) = \vb{f}(x_{0}, u_{0}) + \pdv{f}{x}\Delta x + \del{u}{f}\Delta u = \pdv{f}{x}\Delta x + \del{u}{f}\Delta u.
\end{align*}
Observera att $\pdv{f}{x}$ allmänt är en matris.

På samma sätt fås
\begin{align*}
	\dv{\Delta y}{t} = h(x_{0} + \Delta x, u_{0} + \Delta u) - y_{0} = h(x_{0}, u_{0}) - y_{0} + \grad_{x}{h}\Delta x + \del{u}{h}\Delta u = \grad_{x}{h}\Delta x + \del{u}{h}\Delta u.
\end{align*}
Det totala systemet
\begin{align*}
	\dv{\Delta x}{t} &= \pdv{f}{x}\Delta x + \del{u}{f}\Delta u, \\
	\dv{\Delta y}{t}     &= \grad_{x}{h}\Delta x + \del{u}{h}\Delta u
\end{align*}
är alltså linjärt för små ändringar.

\paragraph{Representation av linjära system}
Tidsutvecklingen av ett system tillstånd kan i linjära fall skrivas som
\begin{align*}
	\dot{x} = Ax + Bu,
\end{align*}
där $u$ beskriver en styrsignal. Utsignalen är typiskt på formen
\begin{align*}
	y = Cx + Du.
\end{align*}

\paragraph{Linjära system i Laplacedomänet}
Genom att Laplacetransformera ekvationen som beskriver ett systems tillstånd fås
\begin{align*}
	sX = AX + BU,
\end{align*}
givet att systemets starttillstånd är $0$. Detta kan skrivas som
\begin{align*}
	X = (sI - A)^{-1}BU.
\end{align*}
Insatt i uttrycket för $Y$ fås
\begin{align*}
	Y = CX + DU = (C(sI - A)^{-1}B + D)U,
\end{align*}
och vi identifierar överförningsfunktionen som
\begin{align*}
	G = C(sI - A)^{-1}B + D.
\end{align*}
Man kan visa att
\begin{align*}
	A^{-1} = \frac{1}{\mdet{A}}A^{\dagger},
\end{align*}
vilket ger
\begin{align*}
	G = \frac{1}{\mdet{sI - A}}C(sI - A)^{\dagger}B + D
\end{align*}

\paragraph{Poler i tillståndsrepresentation}
Om inte $\mdet{sI - A}$ och $C(sI - A)^{\dagger}$ delar faktorer, blir då systemets poler egenvärdena till $A$.

\paragraph{Stabilitet}
Baserad på detta kan vi använda kunskapen om polernas inverkan på systemets stabilitet för att dra följande slutsats: Ett linjärt system är stabilt om och endast om alla egenvärden till $A$ har negativa realdelar.

\paragraph{Lösning av system i representation}
Betrakta ett system på formen
\begin{align*}
	\dot{x} = Ax + Bu,\ x(0) = x_{0},\ y = Cx.
\end{align*}
Vi använder att
\begin{align*}
	\dv{t}e^{-At} = -Ae^{-At},
\end{align*}
vilket ger
\begin{align*}
	e^{-At}\dot{x}              &= e^{-At}Ax + e^{-At}Bu, \\
	\dv{t}\left(e^{-At}x\right) &= e^{-At}Bu, \\
	x                           &= x_{0}e^{At} + \integ{0}{t}{\tau}{e^{-A(t - \tau)}Bu}.
\end{align*}

\paragraph{Styrbarhet}
Ett tillstånd $x$ är styrbart om det finns en insignal som kan styra det motsvarande systemet från $0$ till $x$ på ändlig tid. Ett system är styrbart om alla tillståndsvektorer är styrbara.

\paragraph{Test av styrbarhet}
Vi noterar först att Cayley-Hamiltons sats ger
\begin{align*}
	A^{n} + \sum\limits_{i = 1}^{n}a_{i}A^{n - i} = 0,
\end{align*}
där $a_{i}$ är koefficienterna i $A$:s karakteristiska polynom. Därmed kan alla potenser av $A$ av högre ordning skrivas som en linjärkombination av potenser av $A$ upp till och med $n - 1$, vilket betyder att
\begin{align*}
	e^{At} = \sum\limits_{i = 0}^{n - 1}f_{i}(t)A^{i}
\end{align*}
för några funktioner $f_{i}$.

Betrakta nu ett system och dets representation. Om systemets starttillstånd är $0$, medför detta
\begin{align*}
	x = \left(\sum\limits_{i = 0}^{n - 1}\gamma_{i}(t)A^{i}\right)B
\end{align*}
där
\begin{align*}
	\gamma_{i} = \integ{0}{t}{\tau}{f_{i}(\tau)u(\tau)}.
\end{align*}
Om $x$ är styrbart, måste det enligt resonnemanget ovan vara en linjärkombination av de olika $A^{i}B$. Detta betyder att $x$ måste ligga i bildrummet till styrbarhetsmatrisen
\begin{align*}
	\S = [B, AB, \dots, A^{n - 1}B].
\end{align*}
Systemet är styrbart om $\det(\S) \neq 0$.

\paragraph{Observerbarhet}
Ett tillstånd $x$ är icke observerbart om utsignalen $y$ är identiskt noll då initialvärdet är $x$ och insignalen identisk noll. Ett system är observerbart om inga tillståndsvektorer är icke-observerbara.

\paragraph{Test av observerbarhet}
Vi vill nu testa om ett tillstånd $x_{0}$ är observerbart. Om vi har styrsignal $u = 0$, gäller det att
\begin{align*}
	x = e^{At}x_{0}, y = Ce^{At}x_{0}, \dv[n]{y}{t} = CA^{n}e^{At}x_{0}.
\end{align*}
Speciellt är $y = 0$ för alla $t$ om
\begin{align*}
	y(0) = Cx_{0} = 0, \dv{y}{t} = CAx_{0} = 0,\ \dots, \dv[n - 1]{y}{t} = CA^{n - 1}x_{0} = 0,
\end{align*}
eftersom om detta stämmer, ger Cayley-Hamiltons sats att även högre ordningens derivator av $y$ kommer vara $0$. Därmed ligger de icke-observerbara tillstånden i nollrummet till observerbarhetsmatrisen
\begin{align*}
	\O =
	\mqty[
		C \\
		CA \\
		\vdots \\
		CA^{n - 1}
	].
\end{align*}
Systemet är observerbart om $\det(\O) \neq 0$.

\paragraph{Minimalhet}
En tillståndsbeskrivning är en minimal realisation av ett insignal-utsignalsamband om och endast om den är både styrbar och observerbar.

\paragraph{Proportionell återkoppling}
Antag att vi återkopplar systemet med $u = l_{0}r - Lx$. Om utsignalen ej beror av $u$ kan det återkopplade systemet skrivas som
\begin{align*}
	\dot{x} = (A - BL)x + Bl_{0}r,\ y = Cx.
\end{align*}

Detta ger att systemets poler ges av egenvärdena till $A - BL$. Det finns $n$ såna, och $L$ har $n$ parametrar. Om systemet är styrbart kan dets poler därmed placeras godtyckligt vid lämpligt val av $L$.

Vidare studerar vi ett statiskt scenario, som uppfyller
\begin{align*}
	x &= (BL - A)^{-1}Bl_{0}r, \\
	y &= C(BL - A)^{-1}Bl_{0}r + Dl_{0}r - DL(BL - A)^{-1}Bl_{0}r \\
	  &= \left((C - DL)(BL - A)^{-1}B + D\right)l_{0}r,
\end{align*}
som är proportionellt mot $l_{0}$. Genom att välja
\begin{align*}
	l_{0} = \left((C - DL)(BL - A)^{-1}B + D\right)^{-1}
\end{align*}
fås att $y = r$ när systemet är statiskt. Detta kräver dock att man känner de involverade matriserna och att inga störningar påverkar systemet. Därför inför vi I-reglering.

\paragraph{I-reglering}
När vi I-reglerar, inför vi extra tillstånd
\begin{align*}
	x_{n + 1} = \integ{0}{t}{\tau}{e}.
\end{align*}
Detta ger
\begin{align*}
	\dot{x}_{n + 1} = r - y = r - Cx.
\end{align*}
Då kan vi utvidga modellen till
\begin{align*}
	\dot{\vb{x}} =
	\mqty[
		\dot{x} \\
		\dot{x}_{n + 1}
	]
	=
	\mqty[
		A  & 0 \\
		-C & 0
	]
	\mqty[
		x \\
		x_{n + 1}
	] +
	\mqty[
		B \\
		0
	]
	u
	+
	\mqty[
		0 \\
		1
	]
	r = A\vb{x} + Bu + 
	\mqty[
		0 \\
		1
	]
	r.
\end{align*}
Strategin är nu att återkoppla det nya systemet med återkoppling på formen
\begin{align*}
	u = -Lx - l_{n + 1}x_{n + 1} = -L\vb{x}.
\end{align*}
Då kan $L$ väljas så att $A - BL$ får önskade egenvärden. Stationärt har vi
\begin{align*}
	\dot{x} = 0,\ 	\dot{x}_{n + 1} = 0.
\end{align*}

\paragraph{Skattning av tillstånd}
Betrakta ett system där man endast kan mäta utsignal och insignal. I såna fall kan man inte göra tillståndsåterkoppling direkt. Däremot kan man försöka göra detta genom att skatta systemets tillstånd.

Mer precist, antag att vi har en modell
\begin{align*}
	\dot{x} = Ax + Bu,\ y = Cx,\ x(0) = x_{0}
\end{align*}
som vi simulerar enligt
\begin{align*}
	\dot{\hat{x}} = A\hat{x} + Bu,\  \hat{y} = C\hat{x},\ \hat{x}(0) = \hat{x}_{0}.
\end{align*}
Simuleringsfelet är
\begin{align*}
	y - \hat{y} = y - C\hat{x}.
\end{align*}
Vi försöker återkoppla det simulerade systemet med hjälp av simuleringsfelet. Detta beskrivs av
\begin{align*}
	\dot{\hat{x}} = A\hat{x} + Bu + K(y- C\hat{x}) = (A - KC)\hat{x} + Bu + Ky.
\end{align*}
Systemet ovan kallas för en observatör för det ursprungliga systemet.

Skattningsfelet $\tilde{x}$ beskrivs av
\begin{align*}
	\dot{\tilde{x}} = \dot{x} - \dot{\hat{x}} = Ax + Bu - A\hat{x} - Bu - K(Cx - C\hat{x}) = (A - KC)\tilde{x}.
\end{align*}
Detta har lösning
\begin{align*}
	\tilde{x} = e^{(A - KC)t}\tilde{x}(0).
\end{align*}
Skattningsfelet tenderar alltså mot $0$ om egenvärdena till $A - KC$ är negativa, och hur snabbt det tenderar mot $0$ beror av egenvärdenas belopp. Det finns en komrpomiss mellan hur snabbt skattningsfelet avtar och hur känsligt systemet är för störningar.

För att studera huruvida skattningsfelet kan fås att avta godtyckligt snabbt, studerar vi egenvärdena till $A - KC$. Detta är samma som för $A^{T} - C^{T}K^{T}$. Att få detta system att avta godtyckligt snabbt är en fråga om att kunna placera polerna för matrisen ovan godtyckligt. Detta är exakt samma problem som vi har studerat med proportionell återkoppling, och vi vet då att systemet kan fås att avta godtyckligt snabbt om systemet baserad på $A^{T}$ och $C^{T}$ är styrbart.

\paragraph{Kalmanfilter}
Betrakta ett system vars utsignal störs av ett mätfel $v$ och vars tillstånd störs av ett fel $v$. Skattningsfelet beskrivs av
\begin{align*}
	\dot{\tilde{x}} = (A - KC)\tilde{x} + e - Kv.
\end{align*}
Stora $K$ ger som vi ser höga mätfel, men det ger även snabba system. Optimeringen där görs av ett Kalmanfilter. Vi väljer då systemet så att egenvärdena till $A - BL$ har lägre belopp än egenvärdena till $A - KC$.

Vid återkoppling av såna system är överförningsfunktionen för det slutna systemet
\begin{align*}
	G = C(sI - (A - BL))^{-1}Bl_{0},
\end{align*}
alltså likadant som tidigare.