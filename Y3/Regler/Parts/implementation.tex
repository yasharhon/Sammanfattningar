\section{Implementering}
Mycket av modern reglerteknik görs på dator, så här kommer vi diskutera verktyg för att göra en datorbaserad implementering av en regulator. I en dator tidsdiskretiseras allt, och det kommer därför vara viktigt att kunna approximera derivator.

\paragraph{Sampling}
Sampling är processen att mäta en signal med jämna intervall. Processen karakteriseras av en frekvens $f = \frac{1}{T}$, som är hur ofta man samplar.

\paragraph{Nyquistfrekvensen}
Det visar sig att signaler med högre frekvens än $f = \frac{1}{2T}$ ej kan skiljas från signaler med lägre frekvens. Denna frekvensen kalls Nyquistfrekvensen.

\paragraph{Introduktion av operatorer}
Vi definierar
\begin{align*}
	py = \dot{y},\ q_{T}y = y(t + T).
\end{align*}
Dessa operatorerna kommer vara viktiga när vi gör diskretisering.

\paragraph{Eulers formel bakåt}
Eulers formel bakåt gör approximationen
\begin{align*}
	\dot{y} \approx \Delta_{\text{E}}y = \frac{1}{T}(y(t) - y(t - T)).
\end{align*}
I termer av operatorer kan Eulers formel bakåt skrivas som
\begin{align*}
	p \approx \Delta_{\text{E}} = \frac{1}{T}(1 - q_{T}^{-1}).
\end{align*}

\paragraph{Tustins formel}
Tustins formel gör den implicita approximationen
\begin{align*}
	\dot{y} \approx \Delta_{\text{T}}y
\end{align*}
så att
\begin{align*}
	\frac{1}{2}(\Delta_{\text{T}}y(t) + \delta_{\text{T}}y(t - T)) = \frac{1}{T}(y(t) - y(t - T)).
\end{align*}
I termer av operatorer kan vi skriva
\begin{align*}
	\frac{1}{2}\Delta_{\text{T}}(1 + q_{T}^{-1}) = \frac{1}{T}(1 - q_{T}^{-1}),
\end{align*}
och därmed
\begin{align*}
	p \approx \Delta_{\text{T}} = \frac{2}{T}\frac{1 - q_{T}^{-1}}{1 + q_{T}^{-1}}.
\end{align*}