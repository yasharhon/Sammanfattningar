\documentclass[a4paper, 11pt]{article}
\usepackage[T1]{fontenc}
\usepackage[swedish]{babel}
\usepackage{amssymb}
\usepackage{hyperref}
\usepackage[margin=0.5in]{geometry}

% Package import
\usepackage{amssymb}
\usepackage{amsmath}
\usepackage[arrowdel]{physics}
\usepackage{tensor}
\usepackage{pgffor}
\usepackage{slashed}
\usepackage{tikz-feynman}
\usepackage{simplewick}

% Symbols and notation
\newcommand{\nosum}{\ensuremath{\ \text{(no sum)}}}
\newcommand{\R}{\mathbb{R}}
\newcommand{\p}{\ensuremath{^{\prime}}}
\newcommand{\cc}{\ensuremath{^{\star}}}
\newcommand{\adj}{\ensuremath{^{\dagger}}}
\renewcommand{\O}[1]{\ensuremath{\text{O}(#1)}}
\newcommand{\SU}[1]{\ensuremath{\text{SU}(#1)}}
\newcommand{\SO}[1]{\ensuremath{\text{SO}(#1)}}
\newcommand{\GL}[1]{\ensuremath{\text{GL}(#1)}}

% Calculus
\newcommand{\bound}[1]{\ensuremath{\partial #1}}
\newcommand{\del}[2]{\ensuremath{\partial^{#1}_{#2}}}
\newcommand{\delp}[2][]{\ensuremath{(\partial^{\prime})^{#1}_{#2}}}
\newcommand{\padd}[1]{\ensuremath{\left[D#1\right]}}
\newcommand{\cdel}[2]{\ensuremath{D^{#1}_{#2}}}
\newcommand{\delsw}[2]{\ensuremath{\overleftrightarrow{\partial^{#1}_{#2}}}}
\newcommand{\dalem}{\ensuremath{\square}}
\newcommand{\inte}[2]{\int\limits_{#1}^{#2}}
\newcommand{\integ}[5][]{\int\limits_{#2}^{#3}\dd[#1]{#4}#5}
\newcommand{\pinte}[2]{\ensuremath{\int\limits_{#1}\padd{#2}}}

% Variational calculus
\newcommand{\bvar}[1]{\bar{\delta}#1}

% Functional integration
\newcommand{\finteg}[4]{\ensuremath{\int\limits_{#1}\text{D}\left[#2(#3)\right]#4}}

% Differential geometry and tensors
\newcommand{\dcov}[2]{\grad_{#1}{#2}}
\newcommand{\db}[1]{\vb{E}^{#1}}
\newcommand{\tb}[1]{\vb{E}_{#1}}
\newcommand{\kdelta}[2]{\tensor*{\delta}{^{#1}_{#2}}}
\newcommand{\leci}[2]{\ensuremath{\tensor{\varepsilon}{^{#1}_{#2}}}}
\newcommand{\tenprod}[1]
{
	\foreach \variable [count = \count] in #1
	{
		\ifnum\count > 1
			\otimes
		\fi
		\variable
	}
}
\newcommand{\chris}[3]{\tensor*{\Gamma}{^{#1}_{#2#3}}}
\newcommand{\ts}[2]{T_{#1}#2}
\newcommand{\ds}[2]{T_{#1}^{\star}#2}
\newcommand{\tbun}[1]{T#1}
\newcommand{\lied}[2]{\ensuremath{\mathcal{L}_{#1}#2}}
\newcommand{\df}[1]{\ensuremath{d#1}}
\newcommand{\puf}[2]{#1_{\star}#2}
\newcommand{\pub}[2]{#1^{\star}#2}

% Classical mechanics
\newcommand{\lag}{\ensuremath{\mathcal{L}}}
\newcommand{\ham}{\ensuremath{\mathcal{H}}}

% Quantum mechanics
\newcommand{\fsl}[1]{\ensuremath{\slashed{#1}}}
\newcommand{\norp}{\ensuremath{N}}
\newcommand{\torp}{\ensuremath{T}}
\newcommand{\kgprop}[2]{\ensuremath{\Delta_{\text{#1}}\left(#2\right)}}
\newcommand{\dirprop}[4][\text{F}]{\ensuremath{S_{#1, #2#3}\left(#4\right)}}
\newcommand{\phprop}[4][\text{F}]{\ensuremath{D_{#1, #2#3}\left(#4\right)}}

% Structure and symbols
\newcommand{\example}[1]{\subparagraph{Example: #1}}
\renewcommand{\O}[2]{\ensuremath{\text{O}(#1, #2)}}

% Math
\newcommand{\deval}[4][]{\eval{\dv[#1]{#2}{#3}}_{#4}}
\newcommand{\vinteg}[4]{\int\limits_{#1}^{#2}\dd{\vb{#3}}\cdot #4}

% Analytical mechanics
\newcommand{\J}{\mathcal{J}}
\newcommand{\pob}[3][]{\left\{#2, #3\right\}_{#1}}

\title{Summary of SI2360 Analytical Mechanics and Classical Field Theory}
\author{Yashar Honarmandi \\ yasharh@kth.se}
\date{\today}

\begin{document}

\maketitle

\begin{abstract}
	Denna sammanfattningen innehåller centrala definitioner och satser i SF1672 Flervariabelanalys.
\end{abstract}

\pagenumbering{roman}
\thispagestyle{empty}

\newpage

\tableofcontents

\newpage

\pagenumbering{arabic}

\section{Variational Calculus}

\paragraph{Functionals}
A functional is a mapping from a function space to (real) numbers. The most general functional is composed of two components: One which takes a function in some function space and maps it to a function in another (these two spaces might, but need not, coincide), and one that takes a function in the second space and maps it to a number.

\paragraph{Functional Derivatives}
We will be concerned with optimizing functionals, hence we need some notion of derivatives of functionals with respect to their arguments. To express this in a way that is manageable with our knowledge of mathematics, we will transform the functional to a function of one variable. We do this by considering paths in function space parametrized by some parameter $\alpha$ and define the functional derivative as a directional derivative along that path. The paths of interest will be those where the shift is by some Dirac delta function. More explicitly, for a functional $S(\phi)$ we define
\begin{align*}
	\eval{\fdv{S}{\phi}}_{\phi = \phi_{\alpha}} = \lim\limits_{\epsilon\to 0}\frac{S(\phi_{\alpha + \epsilon}) - S(\phi_{\alpha})}{\epsilon}.
\end{align*}
For a functional depending on multiple fields, we compute the functional derivative along a path where only one function is varied. A similar thing may be done for a function map.

Because we have reduced the problem to a single-variable calculus one, theorems such as the chain rule still apply. We therefore only really need to calculate the functional derivative of a few basic function maps. Let us thus consider the function map
\begin{align*}
	\lag(\phi^{a})(x) = (\del{}{\mu})^{n}\phi^{a_{0}}(x).
\end{align*}
By our previous claim the only sensible definition is
\begin{align*}
	\fdv{\lag(\phi^{a})(x)}{\phi^{b}(y)} = \delta_{ab}(\del{}{\mu})^{n}\delta^{d}(x - y).
\end{align*}
This can now be used to compute all sorts of functional derivatives.

\paragraph{Functional Optimization}
Consider a function (or set of functions) $q$ and some functional $S$ of those $q$. Describe the $q$ such that $S$ has an extremum.

What kinds of problems are interesting? Problems containing arbitrarily high derivatives of $q$ are of course important. Problems with $q$ having fixed value at the boundary, different kinds of boundary conditions, no boundary conditions or even with a functional as a boundary condition are also of interest. However, to introduce the involved techniques, we will first be studying problems of the above form.

\paragraph{The Variation of a Function}
We define
\begin{align*}
	\var{q} = \deval{q}{\alpha}{0}\dd{\alpha}..
\end{align*}

\paragraph{The Variation of a Functional}
We now in a similar way define the variation of a functional as
\begin{align*}
	\var{S} = \deval{S}{\alpha}{0}\dd{\alpha}.
\end{align*}

We see that the variation operation behaves exactly like a derivative, and according to the definition, it commutes with all other derivatives that may be involved, assuming sufficient smoothness. These results will therefore be used without further argument.

In these functional derivatives will appear the partial derivative of a function map with respect to a function. What in god's name is that? And what about a derivative with respect to a derivative? And should the derivative with respect to a derivative not contain some information about the derivative of the function the derivative of which we are differentiating with respect to? These are all very good and somewhat complex questions.

Mathematically, it is somewhat beyond me to give a proper answer. But physically, we can think of it the following way: The functions $q^{i}$, as we will see later, will represent the path of the system, and $\tau$ will be replaced with the time $t$. As such, by varying $q^{i}$ we vary the path of the system, and by varying $\dot{q}^{i}$ we vary the velocity with which the system traverses the path. What we are doing with variational calculus is testing out an infinite number of paths to find the extremum of the action, and surely we must both try out different paths and traverse them with different velocities in order to find the extremum. That is why the functions and their derivatives can be treated separately.

What is a derivative with respect to a function? To answer this, recall that at every point $\tau$, the functions are just numbers. When studying the variation of a functional, we are in truth studying its dependence on $\alpha$. In order to do this, the chain rule states that we must first compute an outer partial derivative, which is exactly what the derivative with respect to a function is. This can in turn be indirectly translated to say something about how the functional changes when the function is varied.

\paragraph{The Integral Functional}
The integral functional is the functional
\begin{align*}
	S(q) = \integ{a}{b}{\tau}{F(q, \dot{q})},
\end{align*}
and will be the subject matter of this text from now on.

\paragraph{Solving the Variational Problem}
We have now tried varying the functional around the extremum. Single-variable calculus gives the condition that $\dv{S}{\alpha} = 0$ at the extremum, which is equivalent to $\var{S} = 0$. In other words,
\begin{align*}
	\integ{a}{b}{\tau}{\del{q^{i}}{F}\var{q^{i}} + \del{\dot{q}^{i}}{F}\var{\dot{q}^{i}}} = 0.
\end{align*}
We can integrate this by parts to obtain
\begin{align*}
	\left[\del{\dot{q}^{i}}{F}\var{q^{i}}\right]_{a}^{b} + \integ{a}{b}{\tau}{\left(\del{q^{i}}{F} - \dv{\tau}\del{\dot{q}^{i}}{F}\right)\var{q^{i}}} = 0.
\end{align*}

The first term from the integration by parts can be handled in two ways. If the variational problem has fixed boundary conditions, the families $q^{i}$ must have been chosen such that all functions in the family satisfy the boundary conditions. Thus the variations of these at the endpoints vanish. Otherwise, the two arising terms might be used as boundary conditions themselves (the need for this arises due to the problem in question being second-order, and thus requiring two conditions). As $q^{i}$ may be varied in any way possible, it is thus clear that when using these boundary terms as conditions, they must be set equal to zero.

The remaining integral might of course happen to be zero for the given choice of families of $q^{i}$. But the extremum is an extremum no matter what choice I make. So by changing up the problem - for instance, by reparametrizing the $q^{i}$ or study an entirely different family in a similar way - I still obtain the same results. This must imply that the integral is zero no matter what $\var{q^{i}}$ is. And for this to be true, the only possibility is for the integrand to always be zero. To be absolutely sure, we can try varying only one coordinate at a time. This implies that
\begin{align*}
	\pdv{F}{q^{i}} - \dv{\tau}\pdv{F}{\dot{q}^{i}} = 0
\end{align*}
for all $i$, always. Solutions to this set of equations are thus our extrema, and are called the Euler-Lagrange equations.

\paragraph{Variational Problems With Higher-Order Derivatives}
What if the integrand also involves higher-order derivatives of the $q^{i}$? We can retrace the above steps mostly, but we will have to perform an extra (series of) integration(s) by parts. For instance, by including the second derivative and adding one extra integration, you should be able to show (unless this is wrong) that the extremum solution satisfies
\begin{align*}
	\pdv{F}{q^{i}} - \dv{\tau}\pdv{F}{\dot{q}^{i}} + \dv[2]{\tau}\pdv{F}{\ddot{q}^{i}} = 0.
\end{align*}

\paragraph{The Functional Derivative of the Integral Functional}
For a variational problem with fixed boundary conditions we obtained
\begin{align*}
	\var{S} = \integ{a}{b}{\tau}{\left(\pdv{F}{q^{i}} - \dv{\tau}\pdv{F}{\dot{q}^{i}}\right)\var{q^{i}}}.
\end{align*}
The functional derivative is then
\begin{align*}
	\fdv{S}{q^{i}(t)} = \eval{\pdv{F}{q^{i}} - \dv{\tau}\pdv{F}{\dot{q}^{i}}}_{t}.
\end{align*}
Using this definition, the Euler-Lagrange equations may be written as
\begin{align*}
	\fdv{S}{q^{i}(t)} = 0.
\end{align*}

\section{Group theory}

\paragraph{Definition of a group}
A grou is a set of objects $G$ with an operation $G\times G\to G,\ (a, b)\to ab$ such that
\begin{itemize}
	\item If $a, b\in G$ then $ab\in G$.
	\item $a(bc) = (ab)c$ for all $a, b, c\in G$.
	\item There exists an identity $e$ such that $ae = ea = a$ for all $a\in G$.
	\item There exists for every element $a$ an inverse $a^{-1}\in G$ such that $aa^{-1} = a^{-1}a = e$.
\end{itemize}
Groups can be
\begin{itemize}
	\item cyclic, i.e. all elements in the group are powers of a single element.
	\item finitie, i.e. groups containing a finite number of elements.
	\item infinite.
	\item discrete, i.e. all elements in the group can be labeled with some index.
	\item continuous.
	\item commutative, i.e. $ab = ba$ for all elements in the group.
\end{itemize}

\paragraph{Subgroups}
If $G = \{g_{\alpha}\}$ and the subset $H = \{h_{\alpha}\}$ is also a group, we call $H$ a subgroup of $G$ and write $H < G$.

\paragraph{Generators}
The generators of a group is the smallest set of elements in the group such that all other elements in the group can be composed by the elements in the set. In this context, we will use the generators in a wider context - for instance, the matrix $J$ that is used to create rotation matrices is said to be a generator of the group.

\paragraph{Direct products}
GIven two groups $F$ and $G$, we define $F\times G$ as the set of ordered pairs of elements of the two groups. The group action of $F\times G$ is the group actions of $F$ and $G$ separately on the elements in the ordered pair.

\paragraph{Homomorphisms and isomorphisms}
A homomorphisms is a map $f: G\to H$ such that $f(g_{1})f(g_{2}) = f(g_{1}g_{2})$. If the map is bijective, $f$ is called an isomorphism.

\paragraph{Point groups}
Point groups are symmetries of, for instance, a crystal structure that leave at least one point in the structure invariant. Examples include
\begin{itemize}
	\item rotations.
	\item reflections.
	\item spatial inversions.
\end{itemize}
Combined with certain discrete translation, these are the space groups of the crystal. Space groups are the groups of all symmetries of a crystal.

\paragraph{Dihedral groups}
The dihedral group $D_{n}$ is the group of transformations that leave an $n$-sided polygon invariant.

\paragraph{Lie groups}
Formally, a Lie group is a group containing a manifold and the group operation and inverse operation being smooth maps on the manifold. Its elements are $g(\vb{\theta})$, where $g(\vb{0}) = 1$. We can expand the map as $g(\vb{\theta})\approx 1 + A$, where
\begin{align*}
	A = i\theta_{a}T_{a},
\end{align*}
where the $T_{a}$ are the generators. That means that close to the identity, the non-commutativity of such maps is captured by the commutators, or Lie brackets:
\begin{align*}
	\lieb{T_{a}}{T_{b}} = if_{a, b, c}T_{c}.
\end{align*}
The generators are sel-adjoint, so the constants $f_{a, b, c}$ are real.

\paragraph{Representations}
A representation is a homomorphism $D: G\to GL(V)$, where $GL(V)$ is the group of all invertible linear transformations on $V$. The group elements thus act on $V$ according to
\begin{align*}
	D(g_{1})D(g_{2})v = D(g_{1}g_{2})v,\ v\in V.
\end{align*}

\paragraph{Reducible and irreducible representations}
Two representations are equivalent if they satisfy $S^{-1}DS = D'$, where $S$ is a matrix representing a change of basis. Some representations can be written as direct sums in certain bases. For these, there is a basis where the representation is block diagonal. These are reducible. Those that cannot are irreducible.

\paragraph{Small and large rotations in two dimensions}
Consider a rotation of an infinitesimal displacement $\dd{\vb{x}}$ with a rotation $R$. The requirement for length to be preserved implies $R^{T}R = 1$.

Consider now a rotation by a small angle $\var{\theta}$. Taylor expanding it in terms of the angle yields
\begin{align*}
	R(\var{\theta}) \approx 1 + A\var{\theta}.
\end{align*}
The requirement for $R$ to be orthogonal yields $A^{T} = -A$. We choose the solution
\begin{align*}
	J =
	\mqty[
		0  & 1 \\
		-1 & 0
	].
\end{align*}
We can now write the rotation matrix as
\begin{align*}
	R(\var{\theta}) =
	\mqty[
		1             & \var{\theta} \\
		-\var{\theta} & 1
	].
\end{align*}

We would now like to construct a large rotation in terms of smaller rotations as
\begin{align*}
	R(\theta) = \lim\limits_{N\to\infty}\left(1 + \frac{\theta}{N}J\right)^{N} = e^{\theta J}.
\end{align*}
We can write this as an infinite series and use the fact that $J^{2} = -1$ to obtain
\begin{align*}
	R(\theta) = \cos{\theta} + J\sin{\theta}.
\end{align*}

\paragraph{Rotatins in three dimensions}
The argument done for two dimensions does not use the dimensionality, so we conclude that even for higher dimensions, $R^{T}R = 1$. Expanding a small rotation around the identity yields that the first-order term must include an antisymmetric matrix. The space of antisymmetric $3\times 3$ matrices is three-dimensional. We thus choose the basis
\begin{align*}
	J_{x} =
	\mqty[
		0 & 0  & 0 \\
		0 & 0  & 1 \\
		0 & -1 & 0
	],
	J_{y} =
	\mqty[
		0 & 0 & -1 \\
		0 & 0 & 0  \\
		1 & 0 & 0
	],
	J_{z} =
	\mqty[
		0  & 1 & 0 \\
		-1 & 0 & 0 \\
		0  & 0 & 0
	].
\end{align*}
Exponentiating yields
\begin{align*}
	R(\theta) = e^{\sum \theta_{i}J_{i}} = e^{\vb{\theta}\cdot\vb{J}}.
\end{align*}
In physics we usually extract a factor $i$ such that the basis matrices are Hermitian, and the rotation becomes
\begin{align*}
	R(\theta) = e^{i\vb{\theta}\cdot\vb{J}}.
\end{align*}

The set of generators of these rotations constitutes the Lie algebra.

We know in general that rotations in three dimensions do not commute. In fact, we obtain in general that
\begin{align*}
	R(\vb{\theta})R(\vb{\theta}')R^{-1}(\vb{\theta}) = \theta_{a}\theta_{b}'\lieb{J_{a}}{J_{b}},
\end{align*}
where $\lieb{J_{a}}{J_{b}}$ is the commutator. This commutator satisfies
\begin{align*}
	\lieb{J_{a}}{J_{b}}^{T} = \lieb{J_{b}^{T}}{J_{a}^{T}} = \lieb{-J_{b}}{-J_{a}} = -\lieb{J_{a}}{J_{b}},
\end{align*}
which implies
\begin{align*}
	\lieb{J_{a}}{J_{b}} = f_{a,b,c}J_{c}.
\end{align*}
It can be shown that
\begin{align*}
	\lieb{J_{i}}{J_{j}} = \varepsilon_{i,j,k}J_{k},
\end{align*}
or in a physics context (where a factor $i$ is extracted):
\begin{align*}
	\lieb{J_{i}}{J_{j}} = i\varepsilon_{i,j,k}J_{k}.
\end{align*}

\paragraph{Symmetries in classical mechanics}

\example{Newton's second law}
Newton's second law $m\ddot{\vb{x}} = -\grad{V}$, assuming the potential to be fixed, has certain symmetry properties:
\begin{itemize}
	\item The transformation $t\to t' = t + t_{0}$ is a symmetry, as $\dv{t} = \dv{t'}$ and $V$ is not changed under the transformation.
	\item The transformation $t\to\tau = -t$ is a symmetry as $\dv{t} = \dv{\tau}{t}\dv{\tau} = -\dv{\tau}$, which implies $\dv[2]{t} = \dv[2]{\tau}$ and $V$ is not changed under the transformation.
	\item Considering a system of particles, if the forces between these only depend on differences between the position vectors, the translation $\vb{x}_{i}\to\vb{y}_{i} = \vb{x}_{i} + \vb{x}_{0}$ is a symmetry as it does not change any differences.
\end{itemize}

\example{Constraining solutions using symmetries}
If a system is invariant under some transformation $\vb{x}\to\vb{R}(\vb{x})$, then any property $u$ dependant on those coordinates satisfies $u(\vb{x}) = u(\vb{R}(\vb{x}))$.

\paragraph{Connection to Noether's theorem}
We defined symmetries of the action as transformations that satisy $\var{\lag} = 0$. In particular, we can construct a set of transformations such that $\del{s}{t} = \var{t},\ \del{s}{q^{a}} = \var{q^{a}}$, where $s$ is the symmetry parameter. This is a one-parameter family of symmetries. By defining $T_{s}q(t, 0) = q(t, s)$, these symmetries satisfy
\begin{align*}
	T_{s_{2}}T_{s_{1}}q(t, 0) = T_{s_{1} + s_{2}}q(t, 0).
\end{align*}
We see that these symmetries define a group.

\example{A particle in a moving potential}

\section{Differential Geometry}

For details on much of this, notably the early parts on Euclidean space, please consult my summary of SI2360 Analytical Mechanics and Classical Field Theory.

\paragraph{Euclidean and Affine Spaces}
A Euclidean space is a set of points such that there to each point can be assigned a position vector. To such spaces we may assign a set of $n$ coordinates $\chi^{a}$ which together uniquely describe each point in the space locally.

\paragraph{Tangent and Dual Bases}
The tangent and dual bases are defined by
\begin{align*}
	\tb{a} = \del{\chi^{a}}{\vb{r}} = \del{a}{\vb{r}},\ \db{a} = \grad{\chi^{a}}.
\end{align*}
Using such bases, we may write
\begin{align*}
	\vb{v} = v^{a}\tb{a} = v_{a}\db{a}.
\end{align*}
The components of these vectors are called contravariant and covariant components respectively.

\paragraph{Christoffel Symbols}
When computing the derivative of a vector quantity, one must account both for the change in the quantity itself and the change in the basis vectors. We define the Christoffel symbols according to
\begin{align*}
	\del{b}{\tb{a}} = \chris{c}{b}{a}\tb{c}.
\end{align*}
These can be computed according to
\begin{align*}
	\db{c}\cdot\del{b}{\tb{a}} = \db{c}\cdot\chris{d}{b}{a}\tb{d} = \kdelta{d}{c}\chris{d}{b}{a} = \chris{c}{b}{a}.
\end{align*}
Note that
\begin{align*}
	\del{a}{\tb{b}} = \del{a}{\del{b}{\vb{r}}} = \del{b}{\del{a}{\vb{r}}} = \del{b}{\tb{a}},
\end{align*}
which implies
\begin{align*}
	\chris{c}{b}{a} = \chris{c}{a}{b}.
\end{align*}
Similarly, we might want to consider $\del{b}{\db{a}}$, which might introduce new symbols. We find, however, that
\begin{align*}
	\del{a}{\db{b}\cdot\tb{c}} = \db{b}\cdot\del{a}{\tb{c}} + \tb{c}\cdot\del{a}{\db{b}} = 0,
\end{align*}
which implies
\begin{align*}
	\del{a}{\db{b}} = -\chris{b}{a}{c}\db{c}.
\end{align*}

\paragraph{Covariant Derivatives}
Covariant derivatives are defined by
\begin{align*}
	\dcov{a}{v^{b}} = \del{a}{v^{b}} + \chris{b}{a}{c}v^{c},
\end{align*}
and thus satisfy
\begin{align*}
	\del{a}{\vb{v}} = \tb{b}\dcov{a}{v^{b}}.
\end{align*}

\paragraph{Tensors}
To define tensors, we first define tensors of the kind $(0, n)$ as maps from $n$ vectors to scalars. Using this, we define tensors of the kind $(n, m)$ as linear maps from $(0, n)$ tensors to $(0, m)$ tensors.

\section{Differentiation and Integration in Orthogonal Coordinates}

To tie together what we have learned thus far with what we studied in Vector Calculus, we will study differentiation and integration in orthogonal coordinate systems. For this part of the summary we will take a break from the oh-so strict indexing rules established above.

\paragraph{Defining Relation}
Orthogonal coordinate systems are defined by the relation
\begin{align*}
	\tb{a}\cdot\tb{b} = h_{a}^{2}\delta_{ab}\nosum .
\end{align*}

\paragraph{Orthonormal Basis}
Based on the orthogonality conditions, we define the orthogonal basis vectors
\begin{align*}
	\vb{e}_{a} = \frac{1}{h_{a}}\tb{a}\nosum .
\end{align*}
The normalization thus implies
\begin{align*}
	h_{a} = \sqrt{\sum\limits_{i}(\del{a}{x^{i}})^{2}}.
\end{align*}

\paragraph{Physical Components}
The physical components of a vector is its projection onto the orthonormal basis vectors, denoted with a tilde.

\paragraph{Relation to Dual Basis}
By expanding the dual basis vectors in terms of their physical components, we obtain
\begin{align*}
	\tilde{E}_{a}^{b} = \vb{e}_{b}\cdot\db{a} = \frac{1}{h_{b}}\kdelta{b}{a}\nosum .
\end{align*}
This implies
\begin{align*}
	\db{a} = \tilde{E}_{b}^{a}\vb{e}_{b} = \frac{1}{h_{a}}\kdelta{a}{b}\vb{e}_{b} = \frac{1}{h_{a}}\vb{e}_{a}\nosum ,
\end{align*}
and thus
\begin{align*}
	\vb{e}_{a} = h_{a}\db{a}.
\end{align*}
We see that the dual basis would have been an equally good starting point for describing orthogonal systems.

\paragraph{Line Integrals}
Using our previous knowledge of rates of change along a curve, we have
\begin{align*}
	\vinteg{\Gamma}{}{r}{\vb{v}} &= \integ{\Gamma}{}{\chi^{a}}{\tb{a}\cdot\vb{v}} \\
	                             &= \integ{\Gamma}{}{\tau}{\dot{\chi}^{a}\tb{a}\cdot\vb{v}} \\
	                             &= \integ{\Gamma}{}{\tau}{\sum\limits_{a}\dot{\chi}^{a}h_{a}\tilde{v}_{a}}.
\end{align*}
Specifically, when integrating along a $\chi^{c}$ coordinate line, we can use the coordinate as a parameter, yielding
\begin{align*}
	\vinteg{\Gamma}{}{r}{\vb{v}} = \integ{\Gamma}{}{\tau}{\dv{\chi^{a}}{\chi^{c}}\tb{a}\cdot\vb{v}} = \integ{\Gamma}{}{\chi^{c}}{h_{c}v_{c}}.
\end{align*}

\paragraph{Surface Integrals}
Consider a coordinate level surface $S_{c}$. In three dimensions we have
\begin{align*}
	\dd{\vb{S}} &= \del{a}{\vb{r}}\times\del{b}{\vb{r}}\dd{\chi^{a}}\dd{\chi^{b}} \\
	            &= h_{a}h_{b}\vb{e}_{a}\vb{e}_{b}\dd{\chi^{a}}\dd{\chi^{b}} \\
	            &= \pm h_{a}h_{b}\dd{\chi^{a}}\dd{\chi^{b}}\vb{e}_{c}.
\end{align*}
We immediately identify the unit normal and area element.

The final results (hopefully) generalize to other dimensionalities, but I could not see any way of bypassing the need for the cross product in three dimensions.

\paragraph{Volume Integrals}
Consider an infinitesimal volume element separated by $2n$ coordinate surfaces corresponding to coordinate values $\chi^{a}$ and $\chi^{a} + \dd{\chi^{a}}$. Using what we did with line integrals
\begin{align*}
	\dd{V} = \prod\limits_{i}h_{i}\dd{\chi^{i}}.
\end{align*}
We identify the Jacobian as $\J = \prod\limits_{i}h_{i}$.

\section{Tensors}

\paragraph{Definition}
A tensor of rank $N$ is a linear map from $N$ vectors to a scalar.

\paragraph{Components of a tensor}
The components of a tensor are defined by
\begin{align*}
	T(\vb{E}^{a_{1}}, \dots, \vb{E}^{a_{N}}) = T^{a_{1}, \dots, a_{N}}.
\end{align*}

\section{Advanced Differential Geometry}

In this part we will expand on the previously discussed concepts of differential geometry, mainly by incorporating our knowledge of tensors into it.

\paragraph{The Metric Tensor}
The metric tensor $g$ is a rank $2$ tensor. We start by defining it as $g(\vb{v}, \vb{w}) = \vb{v}\cdot\vb{w}$, but more generally the metric tensor defines the inner product.

The metric tensor is symmetric. Its components satisfy
\begin{align*}
	v_{a} = \vb{E}_{a}\cdot v^{b}\vb{E}_{b} = g(\vb{E}_{a}, \vb{E}_{b})v^{b} = g_{ab}v^{b},
\end{align*}
and likewise
\begin{align*}
	v^{a} = g^{ab}v_{b}
\end{align*}
where $\vb{v}$ is a vector. This demonstrates the capabilities of the metric to raise and lower indices.

We note that
\begin{align*}
	v_{a} = g_{ab}v^{b} = g_{ab}g^{bc}v_{c},
\end{align*}
which implies $g_{ab}g^{bc} = \kdelta{a}{c}$.

\example{The Metric in Polar Coordinates}
The contravariant components of the metric tensor, according to the definition, are
\begin{align*}
	g_{rr} = \tb{r}\cdot\tb{r} = 1,\ g_{r\phi} = g_{\phi r} = \tb{r}\cdot\tb{\phi} = 0,\ g_{\phi\phi} = \tb{\phi}\cdot\tb{\phi} = r^{2}.
\end{align*}
Likewise, the covariant components are
\begin{align*}
	g^{rr} = \db{r}\cdot\db{r} = 1,\ g_{r\phi} = g_{\phi r} = \db{r}\cdot\db{\phi} = 0,\ g_{\phi\phi} = \db{\phi}\cdot\db{\phi} = \frac{1}{r^{2}}.
\end{align*}

\paragraph{Christoffel Symbols}
When computing the derivative of a vector quantity, one must account both for the change in the quantity itself and the change in the basis vectors. We define the Christoffel symbols according to
\begin{align*}
	\del{b}{\tb{a}} = \chris{c}{b}{a}\tb{c}.
\end{align*}
These can be computed according to
\begin{align*}
	\db{c}\cdot\del{b}{\tb{a}} = \db{c}\cdot\chris{d}{b}{a}\tb{d} = \kdelta{d}{c}\chris{d}{b}{a} = \chris{c}{b}{a}.
\end{align*}
Note that
\begin{align*}
	\del{a}{\tb{b}} = \del{a}{\del{b}{\vb{r}}} = \del{b}{\del{a}{\vb{r}}} = \del{b}{\tb{a}},
\end{align*}
which implies
\begin{align*}
	\chris{c}{b}{a} = \chris{c}{a}{b}.
\end{align*}

Do the Christoffel symbols define a tensor? Clearly they do not. One simple counterexample is when converting from Cartesian coordinates to any non-trivial coordinate system. In Cartesian coordinates all Christoffel symbols are zero, and no linear combination of these could possibly produce non-zero values. There is a transformation rule, however. To find it, we study
\begin{align*}
	\tensor{(\Gamma^{\prime})}{^{a}_{bc}} &= (\db{a})^{\prime}\cdot\del[\prime]{b}{(\tb{c})^{\prime}} \\
	                                      &= \del{d}{(\chi')^{a}}\db{d}\cdot\del[\prime]{b}{(\del[\prime]{c}{\chi^{f}}\tb{f})} \\
	                                      &= \del{d}{(\chi')^{a}}\db{d}\cdot(\tb{f}\del[\prime]{b}{\del[\prime]{c}{\chi^{f}}} + \del[\prime]{c}{\chi^{f}}\del[\prime]{b}{\tb{f}}) \\
	                                      &= \del{d}{(\chi')^{a}}(\kdelta{f}{d}\del[\prime]{b}{\del[\prime]{c}{\chi^{f}}} + \del[\prime]{c}{\chi^{f}}\db{d}\cdot\del[\prime]{b}{\chi^{g}}\del{g}{\tb{f}}) \\
	                                      &= \del{d}{(\chi')^{a}}(\del[\prime]{b}{\del[\prime]{c}{\chi^{d}}} + \del[\prime]{c}{\chi^{f}}\del[\prime]{b}{\chi^{g}}\chris{d}{g}{f}) \\
	                                      &= \del{d}{(\chi')^{a}}\del[\prime]{c}{\chi^{f}}\del[\prime]{b}{\chi^{g}}\chris{d}{g}{f} + \del{d}{(\chi')^{a}}\del[\prime]{b}{\del[\prime]{c}{\chi^{d}}}.
\end{align*}

\example{Christoffel Symbols in Polar Coordinates}
To compute these, we need partial derivative of the basis vectors. We have
\begin{align*}
	\del{r}{\tb{r}} = \vb{0},\ \del{\phi}{\tb{r}} = \del{r}{\tb{\phi}} = \frac{1}{r}\tb{\phi},\ \del{\phi}{\tb{\phi}} = -r\tb{r}.
\end{align*}
We thus obtain
\begin{align*}
	\chris{a}{r}{r} = 0,\ \chris{r}{r}{\phi} = 0,\ \chris{\phi}{r}{\phi} = \frac{1}{r},\ \chris{r}{\phi}{\phi} = -r,\ \chris{\phi}{\phi}{\phi} = 0.
\end{align*}

\paragraph{Covariant Derivatives}
The partial derivate of $\vb{v} = v^{a}\tb{a}$ with respect to $\chi^{a}$ is given by
\begin{align*}
	\del{a}{\vb{v}} = \tb{b}\del{a}{v^{b}} + v^{b}\del{a}{\tb{b}} = \tb{b}\del{a}{v^{b}} + v^{b}\chris{c}{a}{b}\tb{c}.
\end{align*}
Renaming the summation indices yields
\begin{align*}
	\del{a}{\vb{v}} = \tb{b}(\del{a}{v^{b}} + v^{c}\chris{b}{a}{c}),
\end{align*}
which contains one term from the change in the coordinates and one term from the change in basis.

Realizing that derivatives of vector quantities must take both of these into account in order to transform like a tensor, we would like to define a differentiation operation that takes both of these to account when differentiating vector components. This is the covariant derivative. We define its action on contravariant vector components as
\begin{align*}
	\dcov{a}{v^{b}} = \del{a}{v^{b}} + v^{c}\chris{b}{a}{c},
\end{align*}
such that
\begin{align*}
	\del{a}{\vb{v}} = E_{b}\dcov{a}{v^{a}}.
\end{align*}
In a similar fashion we would like to define its action on covariant vector components. To do this, we use the fact that
\begin{align*}
	\del{a}{(\tb{b}\cdot\db{c})} = \del{a}{\kdelta{b}{c}} = 0.
\end{align*}
The product rule yields
\begin{align*}
	\tb{b}\cdot\del{a}{\db{c}} + \db{c}\cdot\del{a}{\tb{b}} = \tb{b}\cdot\del{a}{\db{c}} + \db{c}\cdot\chris{d}{a}{b}\tb{d} = \tb{b}\cdot\del{a}{\db{c}} + \kdelta{d}{c}\cdot\chris{d}{a}{b} = \tb{b}\cdot\del{a}{\db{c}} + \chris{c}{a}{b},
\end{align*}
which implies
\begin{align*}
	\del{a}{\db{c}} = -\chris{c}{a}{b}\db{b}.
\end{align*}
Repeating the steps above now yields
\begin{align*}
	\dcov{a}{v_{b}} = \del{a}{v_{b}} - \chris{c}{a}{b}v_{c}.
\end{align*}

\paragraph{Covariant Derivatives of Tensor Fields}
Next we study the derivatives of a tensor field
\begin{align*}
	\tau = \tensor*{\tau}{^{a_{1}\dots a_{n}}_{b_{1}\dots b_{m}}}\tensor*{e}{^{b_{1}\dots b_{m}}_{a_{1}\dots a_{n}}}.
\end{align*}
The tensor basis element differentiates according to the product rule, but with multiplication replaced by the tensor product. Hence
\begin{align*}
	\del{a}{\tau} &= \tensor*{e}{^{b_{1}\dots b_{m}}_{a_{1}\dots a_{n}}}\del{a}{\tensor*{\tau}{^{a_{1}\dots a_{n}}_{b_{1}\dots b_{m}}}} + \tensor*{\tau}{^{a_{1}\dots a_{n}}_{b_{1}\dots b_{m}}}\del{a}{\tensor*{e}{^{b_{1}\dots b_{m}}_{a_{1}\dots a_{n}}}} \\
	              &= \tensor*{e}{^{b_{1}\dots b_{m}}_{a_{1}\dots a_{n}}}\del{a}{\tensor*{\tau}{^{a_{1}\dots a_{n}}_{b_{1}\dots b_{m}}}} + \tensor*{\tau}{^{a_{1}\dots a_{n}}_{b_{1}\dots b_{m}}}\left(\sum\limits_{k = 1}^{n}\chris{c_{k}}{a}{a_{k}}\tensor*{e}{^{b_{1}\dots b_{m}}_{c_{1}\dots c_{n}}} - \sum\limits_{l = 1}^{m}\chris{b_{l}}{a}{d_{l}}\tensor*{e}{^{d_{1}\dots d_{m}}_{a_{1}\dots a_{n}}}\right) \\
	              &= \tensor*{e}{^{b_{1}\dots b_{m}}_{a_{1}\dots a_{n}}}\del{a}{\tensor*{\tau}{^{a_{1}\dots a_{n}}_{b_{1}\dots b_{m}}}} + \sum\limits_{k = 1}^{n}\tensor*{\tau}{^{c_{1}\dots c_{n}}_{b_{1}\dots b_{m}}}\chris{a_{k}}{a}{c_{k}}\tensor*{e}{^{b_{1}\dots b_{m}}_{a_{1}\dots a_{n}}} - \sum\limits_{l = 1}^{m}\tensor*{\tau}{^{a_{1}\dots a_{n}}_{d_{1}\dots d_{m}}}\chris{d_{l}}{a}{b_{l}}\tensor*{e}{^{b_{1}\dots b_{m}}_{a_{1}\dots a_{n}}} \\
	              &= \tensor*{e}{^{b_{1}\dots b_{m}}_{a_{1}\dots a_{n}}}\left(\del{a}{\tensor*{\tau}{^{a_{1}\dots a_{n}}_{b_{1}\dots b_{m}}}} + \sum\limits_{k = 1}^{n}\tensor*{\tau}{^{c_{1}\dots c_{n}}_{b_{1}\dots b_{m}}}\chris{a_{k}}{a}{c_{k}} - \sum\limits_{l = 1}^{m}\tensor*{\tau}{^{a_{1}\dots a_{n}}_{d_{1}\dots d_{m}}}\chris{d_{l}}{a}{b_{l}}\right).
\end{align*}
We thus define
\begin{align*}
	\dcov{a}{\tensor*{\tau}{^{a_{1}\dots a_{n}}_{b_{1}\dots b_{m}}}} = \del{a}{\tensor*{\tau}{^{a_{1}\dots a_{n}}_{b_{1}\dots b_{m}}}} + \sum\limits_{k = 1}^{n}\tensor*{\tau}{^{c_{1}\dots c_{n}}_{b_{1}\dots b_{m}}}\chris{a_{k}}{a}{c_{k}} - \sum\limits_{l = 1}^{m}\tensor*{\tau}{^{a_{1}\dots a_{n}}_{d_{1}\dots d_{m}}}\chris{d_{l}}{a}{b_{l}}.
\end{align*}

\paragraph{The Gradient of a Tensor Field}
Based on the above, the directional derivative of a tensor field is
\begin{align*}
	\grad_{\vb{n}}{\tau} = n^{a}\del{a}{\tau}.
\end{align*}
This is equal to the contraction of $\vb{n}$ with the object
\begin{align*}
	\grad{\tau} = \db{c}\otimes\del{c}{\tau},
\end{align*}
which is defined as the gradient of $\tau$. More explicitly, we have
\begin{align*}
	\grad{\tau} = \tensor*{e}{^{cb_{1}\dots b_{m}}_{a_{1}\dots a_{n}}}\dcov{c}{\tensor*{\tau}{^{a_{1}\dots a_{n}}_{b_{1}\dots b_{m}}}}.
\end{align*}
It has the interesting property of having a rank one higher than $\tau$, which matches what we know - for instance, the gradient transforms a scalar into a vector.

\paragraph{Christoffel Symbols and the Metric}
The derivatives of the metric tensor are given by
\begin{align*}
	\del{c}{g_{ab}} = \tb{a}\cdot\del{c}{\tb{b}} + \tb{b}\cdot\del{c}{\tb{a}} = \tb{a}\cdot\chris{d}{c}{b}\tb{d} + \tb{b}\cdot\chris{d}{c}{a}\tb{d} = \chris{d}{c}{b}g_{ad} + \chris{d}{c}{a}g_{bd}.
\end{align*}
Multiplying by $g^{ea}$ and summing over $a$ yields
\begin{align*}
	g^{ea}\del{c}{g_{ab}} = \chris{d}{c}{b}g_{ad}g^{ea} + \chris{d}{c}{a}g_{bd}g^{ea} = \chris{d}{c}{b}g_{da}g^{ae} + \chris{d}{c}{a}g_{bd}g^{ea} = \chris{e}{c}{b} + \chris{d}{c}{a}g_{bd}g^{ea}.
\end{align*}
The hope is that this can be used to obtain an expression for the Christoffel symbols. To try to do that, we will compare this to the expression obtained by switching $c$ and $b$. This expression is
\begin{align*}
	g^{ea}\del{b}{g_{ac}} = \chris{e}{b}{c} + \chris{d}{b}{a}g_{cd}g^{ea} = \chris{e}{c}{b} + \chris{d}{b}{a}g_{cd}g^{ea},
\end{align*}
yielding
\begin{align*}
	\chris{e}{c}{b} &= \frac{1}{2}\left(g^{ea}\del{c}{g_{ab}} + g^{ea}\del{b}{g_{ac}} - \chris{d}{c}{a}g_{bd}g^{ea} - \chris{d}{b}{a}g_{cd}g^{ea}\right) \\
	                &= \frac{1}{2}g^{ea}\left(\del{c}{g_{ab}} + \del{b}{g_{ac}} - \chris{d}{a}{c}g_{bd} - \chris{d}{a}{c}g_{cd}\right) \\
	                &= \frac{1}{2}g^{ea}\left(\del{c}{g_{ab}} + \del{b}{g_{ac}} - \del{a}{g_{bc}}\right).
\end{align*}

\paragraph{Curve Length}
Consider some curve parametrized by $t$, and let $\dot{\vb{\gamma}}$ denote its tangent. The curve length is given by
\begin{align*}
	\dd{s}^{2} = \dd{\vb{x}}\cdot\dd{\vb{x}} = g(\dot{\vb{\gamma}}, \dot{\vb{\gamma}})\dd{t}^{2} = g_{ab}\dot{\chi^{a}}\dot{\chi^{b}}\dd{t}^{2}.
\end{align*}
The curve length is now given by
\begin{align*}
	L = \integ{}{}{t}{\sqrt{g_{ab}\dot{\chi^{a}}\dot{\chi^{b}}}}.
\end{align*}

\paragraph{Geodesics}
A geodesic is a curve that extremises the curve length between two points. From variational calculus, it is known that such curves satisfy the Euler-Lagrange equations, and we would like a differential equation that describes such a curve. By defining $\mathcal{L} = \sqrt{g_{ab}\dot{\chi}^{a}\dot{\chi}^{b}}$, the Euler-Lagrange equations for the curve length becomes
\begin{align*}
	\del{\chi^{a}}{\mathcal{L}} - \dv{t}\del{\dot{\chi}^{a}}{\mathcal{L}} = 0.
\end{align*}
The Euler-Lagrange equation thus becomes
\begin{align*}
	&\frac{1}{2\mathcal{L}}\dot{\chi}^{b}\dot{\chi}^{c}\del{a}{g_{bc}} - \dv{t}\left(\frac{1}{2\mathcal{L}}g_{bc}(\dot{\chi}^{b}\kdelta{a}{c} + \dot{\chi}^{c}\kdelta{a}{b})\right) = 0, \\
	&\frac{1}{2\mathcal{L}}\dot{\chi}^{b}\dot{\chi}^{c}\del{a}{g_{bc}} - \dv{t}\left(\frac{1}{2\mathcal{L}}(g_{ba}\dot{\chi}^{b} + g_{ac}\dot{\chi}^{c}\right) = 0, \\
	&\frac{1}{2\mathcal{L}}\dot{\chi}^{b}\dot{\chi}^{c}\del{a}{g_{bc}} - \dv{t}\left(\frac{1}{\mathcal{L}}g_{ac}\dot{\chi}^{c}\right) = 0.
\end{align*}
Expanding the time derivative yields
\begin{align*}
	\frac{1}{2\mathcal{L}}\dot{\chi}^{b}\dot{\chi}^{c}\del{a}{g_{bc}} - \frac{1}{\mathcal{L}}\dv{t}(g_{ac}\dot{\chi}^{c}) + g_{ac}\dot{\chi}^{c}\frac{1}{\mathcal{L}^{2}}\dv{\lag}{t} = \frac{1}{2\mathcal{L}}\dot{\chi}^{b}\dot{\chi}^{c}\del{a}{g_{bc}} - \frac{1}{\lag}\dv{t}(g_{ac}\dot{\chi}^{c}) + \frac{1}{\lag}g_{ac}\dot{\chi}^{c}\dv{\ln{\lag}}{t} = 0.
\end{align*}
The curve may be reparametrized such that $\lag$ is equal to $1$ everywhere, yielding
\begin{align*}
	\frac{1}{2\mathcal{L}}\left(\dot{\chi}^{a}\dot{\chi}^{b}\del{c}{g_{ab}} - \dv{t}(2\dot{\chi}^{a}g_{ac})\right) = 0.
\end{align*}
We note that the expression in the paranthesis is the Euler-Lagrange equation for the integral of $\mathcal{L}^{2}$, a nice fact for the future. Expanding the derivative yields
\begin{align*}
	\frac{1}{\mathcal{L}}\left(\frac{1}{2}\dot{\chi}^{a}\dot{\chi}^{b}\del{c}{g_{ab}} - g_{ac}\ddot{\chi}^{a} - \dot{\chi}^{a}\dot{\chi}^{b}\del{b}{g_{ac}}\right) = 0.
\end{align*}
To remove the metric from the second derivative, we multiply by $-g^{cd}\mathcal{L}$ to obtain
\begin{align*}
	&g_{ac}g^{cd}\ddot{\chi}^{a} + \frac{1}{2}\dot{\chi}^{a}\dot{\chi}^{b}g^{cd}(2\del{b}{g_{ac}} - \del{c}{g_{ab}}) = 0, \\
	&g_{ac}g^{cd}\ddot{\chi}^{a} + \frac{1}{2}\dot{\chi}^{a}\dot{\chi}^{b}g^{cd}(\del{b}{g_{ac}} + \del{a}{g_{bc}} - \del{c}{g_{ab}}) = 0, \\
	&\ddot{\chi}^{d} + \frac{1}{2}\dot{\chi}^{a}\dot{\chi}^{b}g^{cd}(\del{b}{g_{ac}} + \del{a}{g_{bc}} - \del{c}{g_{ab}}) = 0.
\end{align*}
This is the geodesic equation. It may alternatively be written in terms of the Christoffel symbols as
\begin{align*}
	\ddot{\chi}^{d} + \chris{d}{a}{b}\dot{\chi}^{a}\dot{\chi}^{b} = 0.
\end{align*}

\paragraph{Christoffel Symbols and the Geodesic Equation}
Consider a straight line with a tangent vector of constant magnitude. In euclidean space, this is a geodesic. This curve satisfies
\begin{align*}
	\vb{0} = \dv{\dot{\vb*{\gamma}}}{t} = (\dot{\vb*{\gamma}}\cdot\grad)\dot{\vb*{\gamma}} = \dot{\chi}^{a}\del{a}{\dot{\vb*{\gamma}}} = \dot{\chi}^{a}(\dcov{a}{\dot{\chi}^{d}})\tb{d} = (\dot{\chi}^{a}\del{a}{\dot{\chi}^{d}} + \dot{\chi}^{a}\dot{\chi}^{c}\chris{d}{a}{c})\tb{d}.
\end{align*}
Comparing this to the geodesic equation yields
\begin{align*}
	\chris{d}{a}{b} = \frac{1}{2}g^{dc}(\del{b}{g_{ac}} + \del{a}{g_{cb}} - \del{c}{g_{ab}}).
\end{align*}
A better approach would have been to go through the derivation of the geodesic equation again, identifying the Christoffel symbols as you go, but I am not sure if that is what I did in the previous paragraph. In any case we have already obtained this result.

\paragraph{Manifolds}
A manifold is a set which is locally isomorphic to $\R^{n}$. We will take this to mean that we can locally impose coordinates $\chi^{a}$ on the manifold.

More formally, a manifold is described by a number of sets $U_{i}\subset\R^{n}$ called charts. To each chart belongs a set of coordinate functions $\chi_{i}$ which map from a subset $M_{i}\subset M$ to $U_{i}$ such that $\chi_{i}$ is a smooth bijection. A set of charts such that every point $p\in M$ is found in at least one chart is called an atlas.

\paragraph{Manifolds and Vectors}
Even though manifolds are locally isomorphic to Euclidean space, the vectors that were previously developed do not make sense when applied to this Euclidean space.

\example{Tangent Vectors on $S_{2}$}
Consider $S_{2}$, the unit sphere in $\R^{3}$, and suppose you cover it with a layer of water like an ocean, introduce north and south poles, place two sailors on opposite sides of the equator and tell both of them to sail south at some given speed. In practice, this means that they should both travel in their local $-y$ direction. Assuming vectors in the two spaces to make sense, you would conclude that the sailors are sailing in the same direction at the same speed and could not possibly hit each other. The accident which would occur at the south pole would of course prove you wrong. This example is one, very verbose, way of expressing why the vectors in the local Euclidean spaces do not make sense.

This argument seems to have one hole in it, namely that $S_{2}$ is implicitly embedded in $\R^{3}$. Using this fact, the collision between the sailors could be deduced using the previously developed concepts of vectors. The reason why this would work is that you could impose a position vector in $\R^{3}$ onto every point on $S_{2}$. This is not a feature of more general manifolds, meaning that this hole does not exist for more general manifolds.

\paragraph{Tangent Vectors}
Tangent vectors describe how scalar fields change with displacement along a curve. In Euclidean space the tangent basis was composed of derivatives with respect to the set of coordinates. In general curved spaces, we define
\begin{align*}
	\tb{a} = \del{a}{}.
\end{align*}
Derivatives are linear operators, so at least the set of tangent bases span some vector space and it makes sense to call a derivative a vector. A general tangent vector is now
\begin{align*}
	X = X^{a}\tb{a} = X^{a}\del{a}{}.
\end{align*}
These live in the tangent space $\ts{p}{M}$ of the manifold $M$ at the point $p$.

To get more of a sense of how this can be related to vectors, consider the directional derivative
\begin{align*}
	\grad_{\vb{n}} = \vb{n}\cdot\grad = n^{a}\del{a}{}.
\end{align*}
When applied to Euclidean space, there is a direct correspondence between $\vb{n}$ and the directional derivative, as $\grad_{\vb{n}}{\vb{x}} = \vb{n}$. For more general manifolds, tangent vectors are defined to be directional derivatives. Note that this definition carries with it the same dependence on position as was previously warned about.

Tangent vectors transform according to
\begin{align*}
	X^{a}\del{a}{} = X^{a}\del{a}{(\chi^{\prime})^{b}}\del[\prime]{b}{},
\end{align*}
implying the transformation rule
\begin{align*}
	(X^{\prime})^{a} = \del{b}{(\chi^{\prime})^{a}}X^{b},
\end{align*}
which is the same as the transformation rule for contravariant vector components in Euclidean space.

\paragraph{Dual Vectors}
To define dual vectors, we first introduce the dual space as the set of all linear operations from the tangent space to real numbers. This is also a vector space. The basis for the space is defined such that
\begin{align*}
	\db{a}(\del{b}{}) = \kdelta{b}{a}.
\end{align*}

In Euclidean space the dual basis was constructed from the gradient. The only concept here that carries over to manifolds is a definition based on small changes in the coordinates. More specifically, for any smooth scalar field $f$ we define a dual vector field according to
\begin{align*}
	df(X) = Xf = X^{a}\del{a}{f}
\end{align*}
and call it the differential. This has a similar structure to an inner product if the dual vector field has components $df_{a} = \del{a}{f}$. These components correspond to those of the gradient in Euclidean space. The basis we desire is $\db{a} = d\chi^{a}$. These live in the dual space $\ds{p}{M}$ of the manifold $M$ at the point $p$.

The dual basis satisfies satisfies
\begin{align*}
	d\chi^{a}(\del{b}{}) = \del{b}{\chi^{a}} = \kdelta{b}{a},
\end{align*}
as expected. Using this, we obtain
\begin{align*}
	df = (\del{a}{f})d\chi^{a},
\end{align*}
which at least looks like the differential of a function.

The components transform according to
\begin{align*}
	\del{a}{f} = \del[\prime]{b}{f}\del{a}{(\chi^{\prime})^{b}},
\end{align*}
which is the transformation rule for covariant vector components.

\paragraph{Tensors}
Having identified a basis for the tangent and dual spaces, we may now construct tensors similarly to what we have previously done. Note now that as the tangent and dual vectors belong to different vector spaces, the notion of type $(n, m)$ tensors is more clear. This also explains why we needed to be careful with indices being up or down when studying Euclidean space, as the difference is huge for manifolds.

\paragraph{Flow of Vector Fields}
The tangent bundle of a manifold is defined as $TM = \bigcup\limits_{p}\ts{p}{M}$. A vector field is a map $X: M\to\tbun{M}$ such that $X(p)\in\ts{p}{M}$. Given this, we may define the flow of a vector field as a collection of curves $\gamma_{X}$ which given some starting point $p$ satisfy
\begin{align*}
	\deval{\gamma_{X}}{\tau}{p, s} = \eval{X}_{\gamma_{X}(p, s)}.
\end{align*}

\paragraph{Pushforwards and Pullbacks}
Consider some function $f$ which maps a manifold $M_{1}$ to another manifold $M_{2}$, as well as a function $g: M_{2}\to\R$. We then define the pullback of $g$ to $M_{1}$ by $f$ as $\pub{f}{g} = g\circ f$. We also define the pushforward of a vector $V\in\ts{p}{M}$ as $(\puf{f}{V})\phi = V(\pub{f}{\phi})$.

\paragraph{Tangents and the Pushforward}
$f$ maps the cooordinates $\chi^{a}$ of $M_{1}$ to the coordinates $\eta^{\mu}$ of $M_{2}$. By definition we have
\begin{align*}
	(\puf{f}{V})\phi &= V(\pub{f}{\phi}) \\
	                 &= V^{a}\del{a}{(\phi\circ f)} \\
	                 &= V^{a}\del{\mu}{\phi}\del{a}{\eta^{\mu}},
\end{align*}
meaning $\puf{f}{V} = V^{a}\del{a}{\eta^{\mu}}\del{\mu}{}$.

How do we interpret this? Consider some curve $\gamma$ in $M_{1}$ which is mapped to a curve $\alpha$ $M_{2}$ by $f$. If $V$ is the tangent of $\gamma$ at some particular point, we have
\begin{align*}
	\dot{\alpha} = \dot{\eta^{\mu}}\del{\mu}{} = \del{a}{\eta^{\mu}}\dot{\chi^{a}}\del{\mu}{} = \del{a}{\eta^{\mu}}V^{a}\del{\mu}{},
\end{align*}
meaning that the pushforward of a tangent by $f$ is the tangent of the curve produced by $f$.

\paragraph{The Pullback of Tensors}
We can now define the pullback of a $(0, m)$ tensor on $M_{2}$ according to
\begin{align*}
	\pub{f}{\omega}(V_{1}, \dots, V_{m}) = \omega(\puf{f}{V_{1}}, \dots, \puf{f}{V_{m}}).
\end{align*}
If $f$ is a bijection we may also define the more general pullback of a $(n, m)$ tensor on $M_{2}$ as
\begin{align*}
	\pub{f}{T}(V_{1}, \dots, V_{m}, \omega_{1}, \dots, \omega_{n}) = T(\puf{f}{V_{1}}, \dots, \puf{f}{V_{m}}, \pub{(f^{-1})}{\omega_{1}}, \dots, \pub{(f^{-1})}{\omega_{n}}).
\end{align*}

\paragraph{The Lie Derivative}
The composition $XY$ of two tangent vectors does not produce a new vector, hence this is not a way to produce a derivative on manifolds. Instead we introduce the Lie bracket
\begin{align*}
	\lied{X}{Y} =\comm{X}{Y} = XY - YX &= X^{a}\del{a}{(Y^{b}\del{b}{})} - Y^{b}\del{b}{(X^{a}\del{a}{})} \\
	                                   &= X^{a}\del{a}{(Y^{b})}\del{b}{} + X^{a}Y^{b}\del{a}{\del{b}{}} - Y^{b}\del{b}{(X^{a})}\del{a}{} - Y^{b}X^{a}\del{b}{\del{a}{}} \\
	                                   &= (X^{b}\del{b}{(Y^{a})} - Y^{b}\del{b}{(X^{a})})\del{a}{},
\end{align*}
which is a new vector. This derivative satisfies many properties that are desirable for a derivative - in particular the product rule
\begin{align*}
	\lied{X}{fY} &= (X^{b}\del{b}{(fY^{a})} - fY^{b}\del{b}{(X^{a})})\del{a}{} \\
	             &= (X^{b}(Y^{a}\del{b}{(f)} + f\del{b}{(Y^{a})}) - fY^{b}\del{b}{(X^{a})})\del{a}{} \\
	             &= X^{b}\del{b}{(f)}Y^{a}\del{a}{} + f(X^{b}\del{b}{(Y^{a})} - Y^{b}\del{b}{(X^{a})})\del{a}{} \\
	             &= X(f)Y + f\lied{X}{Y}.
\end{align*}

\paragraph{Differential Forms}
The set of $p$-forms, or differential forms, is the set of $(0, p)$ tensors that are completely antisymmetric. They are constructed using the wedge product, defined as
\begin{align*}
	\bigwedge\limits_{k = 1}^{p}d\chi^{a_{k}} = \sum\limits_{\sigma\in S_{p}}\text{sgn}(\sigma)\bigotimes_{k = 1}^{p}d\chi^{a_{\sigma(k)}}.
\end{align*}
Here $S_{p}$ is the set of permutations of $p$ elements. There exists
\begin{align*}
	n_{p}^{N} = {N\choose k}
\end{align*}
basis elements. We note that the wedge product is antisymmetric under the exchange of two basis elements. Hence, once an ordering of indices has been chosen, any permutation will simply create a linearly dependent map.

Consider now some antisymmetric tensor $\omega$. Introducing the antisymmetrizer
\begin{align*}
	\bigotimes_{k = 1}^{p}d\chi^{[a_{\sigma(k)}]} = \frac{1}{p!}\sum\limits_{\sigma\in S_{p}}\text{sgn}(\sigma)\bigotimes_{k = 1}^{p}d\chi^{a_{\sigma(k)}},
\end{align*}
the symmetry yields
\begin{align*}
	\omega = \omega_{a_{1}\dots a_{p}}\bigotimes_{k = 1}^{p}d\chi^{a_{\sigma(k)}} = \omega_{a_{1}\dots a_{p}}\bigotimes_{k = 1}^{p}d\chi^{[a_{\sigma(k)}]} = \frac{1}{p!}\omega_{a_{1}\dots a_{p}}\bigwedge\limits_{k = 1}^{p}d\chi^{a_{k}}.
\end{align*}

\paragraph{The Exterior Derivative}
We define the exterior derivative of a differential form according to
\begin{align*}
	d\omega = \frac{1}{p!}\del{a_{1}}{\omega_{a_{2}\dots a_{p + 1}}}\bigwedge\limits_{k = 1}^{p + 1}d\chi^{a_{k}},
\end{align*}
which is a $p + 1$-form. This notation makes sense, as at least in the case of a $0$-form, we obtain
\begin{align*}
	d\omega = \del{a}{\omega}d\chi^{a},
\end{align*}
which is exactly the form of a dual vector. Somehow this transforms as a tensor.

\paragraph{Integration of Differential Forms}
Consider a set of $p$ tangent vectors $X_{i}$. The corresponding coordinate displacements are $\dd{\chi_{i}^{a}} = X_{i}^{a}\dd{t_{i}}$, with no sum over $i$. We would now like to compute the $p$-dimensional volume defined by the $X_{i}$ and $\dd{t_{i}}$. We expect that if any of the $X_{i}$ are linearly dependent the volume should be zero. We also expect that the volume be linear in the $X_{i}$. This implies
\begin{align*}
	\dd{V_{p}} = \omega(X_{1}, \dots, X_{p})\dd{t_{1}}\dots\dd{t_{p}}
\end{align*}
for some differential form $\omega$. We now define the integral over the $p$-volume $S$ over the $p$-form $\omega$ as
\begin{align*}
	\minteg{S}{\omega} = \integ{}{}{t_{1}}{\dots \integ{}{}{t_{p}}{\omega(\dot{\gamma}_{1},\dots, \dot{\gamma}_{p})}}.
\end{align*}
Here the $\gamma_{i}$ are the set of curves that span $S$, the dot symbolizes the derivative with respect to the individual curve parameters and the right-hand integration is performed over the appropriate set of parameter values.

\paragraph{Stokes' Theorem}
Stokes' theorem relates the integral of a differential form $d\omega$ over some subset $V$ of a manifold to an integral over \bound{V} of another differential form.

To derive it, consider a $p + 1$-volume parametrized such that all $t_{i}$ range from 0 to 1 and such that for any fixed $t_{p + 1}$, the remaining $t_{i}$ parametrize a $p$-dimensional surface $V_{p}$ with a boundary independent of $t_{p + 1}$. This construction is somewhat restrictive, but only necessary in the derivation.

For some $p$-form $\omega$ and $p + 1$-volume $V$ we have
\begin{align*}
	\minteg{V}{d\omega} &= \integ{}{}{t_{1}}{\dots \integ{}{}{t_{p + 1}}{d\omega(\dot{\gamma}_{1},\dots, \dot{\gamma}_{p + 1})}} \\
	                    &= \integ{}{}{t_{1}}{\dots \integ{}{}{t_{p + 1}}{\left(\del{a_{1}}{\omega_{a_{2}\dots a_{p + 1}}}\sum\limits_{\sigma\in S_{p + 1}}\text{sgn}(\sigma)\bigotimes_{k = 1}^{p + 1}d\chi^{a_{\sigma(k)}}\right)(\dot{\gamma}_{1},\dots, \dot{\gamma}_{p + 1})}} \\
	                    &= \sum\limits_{\sigma\in S_{p + 1}}\frac{\text{sgn}(\sigma)}{p!}\integ{}{}{t_{1}}{\dots \integ{}{}{t_{p + 1}}{\left(\del{a_{1}}{\omega_{a_{2}\dots a_{p + 1}}}\bigotimes_{k = 1}^{p + 1}d\chi^{a_{\sigma(k)}}\right)(\dot{\gamma}_{1},\dots, \dot{\gamma}_{p + 1})}} \\
	                    &= \sum\limits_{\sigma\in S_{p + 1}}\frac{\text{sgn}(\sigma)}{p!}\integ{}{}{t_{1}}{\dots \integ{}{}{t_{p + 1}}{\left(\del{a_{p + 1}}{\omega_{a_{1}\dots a_{p}}}\bigotimes_{k = 1}^{p + 1}d\chi^{a_{\sigma(k)}}\right)(\dot{\gamma}_{1},\dots, \dot{\gamma}_{p + 1})}},
\end{align*}
where the latter follows from the cyclicity imposed by the summation. We have
\begin{align*}
	d\chi^{a_{\sigma(k)}}{\dot{\gamma}_{k}} = \dv{\chi^{a}}{t_{k}}\del{a}{\chi^{a_{\sigma(k)}}} = \dv{\chi^{a_{\sigma(k)}}}{t_{k}},
\end{align*}
and thus
\begin{align*}
	\minteg{V}{d\omega} &= \sum\limits_{\sigma\in S_{p + 1}}\frac{\text{sgn}(\sigma)}{p!}\integ{}{}{t_{1}}{\dots \integ{}{}{t_{p + 1}}{\del{a_{p + 1}}{\omega_{a_{1}\dots a_{p}}}\prod\limits_{k = 1}^{p + 1}\del{t_{k}}{\chi^{a_{\sigma(k)}}}}} \\
	                    &= \sum\limits_{\sigma\in S_{p + 1}}\frac{\text{sgn}(\sigma)}{p!}\integ{}{}{t_{1}}{\dots \integ{}{}{t_{p + 1}}{\del{a_{p + 1}}{\omega_{a_{1}\dots a_{p}}}\prod\limits_{k = 1}^{p + 1}\del{t_{\sigma(k)}}{\chi^{a_{k}}}}},
\end{align*}
where we have once again utilized the cyclicity. Denote the integral inside the sum as $I(\sigma, \omega)$. We have
\begin{align*}
	I(\sigma, \omega) &= \integ{}{}{t_{1}}{\dots \integ{}{}{t_{p + 1}}{\del{a_{p + 1}}{\omega_{a_{1}\dots a_{p}}}\prod\limits_{k = 1}^{p + 1}\del{t_{\sigma(k)}}{\chi^{a_{k}}}}} \\
	                  &= \integ{}{}{t_{1}}{\dots \integ{}{}{t_{p + 1}}{\del{t_{\sigma(p + 1)}}{\omega_{a_{1}\dots a_{p}}}\prod\limits_{k = 1}^{p}\del{t_{\sigma(k)}}{\chi^{a_{k}}}}}.
\end{align*}
To proceed, we integrate by parts and obtain
\begin{align*}
	I(\sigma, \omega) &= \eval{\integ{}{}{t_{1}}{\dots \integ{}{}{t_{p + 1}}{\omega_{a_{1}\dots a_{p}}\prod\limits_{k = 1}^{p}\del{t_{\sigma(k)}}{\chi^{a_{k}}}}}}_{t_{\sigma(p + 1)} = t_{\sigma(p + 1)}^{-}}^{t_{\sigma(p + 1)} = t_{\sigma(p + 1)}^{+}} - \integ{}{}{t_{1}}{\dots \integ{}{}{t_{p + 1}}{\omega_{a_{1}\dots a_{p}}\del{t_{\sigma(p + 1)}}{\prod\limits_{k = 1}^{p}\del{t_{\sigma(k)}}{\chi^{a_{k}}}}}},
\end{align*}
where the former terms contains no integration over $t_{\sigma(p + 1)}$, and is instead evaluated at the maximal and minimal values of $t_{\sigma(p + 1)}$ given the values of the other parameters.

Let us proceed to simplify this. For starters, if $\sigma(p + 1) \neq p + 1$, the remaining integration in the first term is done over the boundary of $V_{p}$. Furthermore, there exists a $k$ such that $\sigma(k) = p + 1$, and as we are at the boundary of $V_{p}$ we must have
\begin{align*}
	\del{t_{\sigma(k)}}{\chi^{a_{k}}} = 0.
\end{align*}
Otherwise, the remaining integration domain is $V_{p}$. Next, the latter integral contains a sequence of terms proportional to
\begin{align*}
	\del{t_{\sigma(p + 1)}}{\del{t_{\sigma(k)}}{\chi^{a_{k}}}},
\end{align*}
which is symmetric with respect to exchanging $p + 1$ and $k$. These terms therefore cancel in the sum, and we are left with
\begin{align*}
	\minteg{V}{d\omega} &= \sum\limits_{\sigma\in S_{p}}\frac{\text{sgn}(\sigma)}{p!}\eval{\integ{}{}{t_{1}}{\dots \integ{}{}{t_{p}}{\omega_{a_{1}\dots a_{p}}\prod\limits_{k = 1}^{p}\del{t_{\sigma(k)}}{\chi^{a_{k}}}}}}_{t_{\sigma(p + 1)} = 0}^{t_{\sigma(p + 1)} = 1} \\
	                    &= \eval{\integ{}{}{t_{1}}{\dots \integ{}{}{t_{p}}{\omega}}}_{t_{\sigma(p + 1)} = 0}^{t_{\sigma(p + 1)} = 1}
\end{align*}
where the condition that $\sigma(p + 1) = p + 1$ restricts the summation to $S_{p}$.

The remaining integration domain is, as stated before, a parametrization of $V_{p}$, which at the extremal values of $t_{p + 1}$ must be at the boundary of $V$. The minus sign from the integral evaluation tells us that the integrals are taken with opposite orientation, meaning that together they indeed form an integration over the (closed) boundary of $V$. We thus arrive at Stokes' theorem,
\begin{align*}
	\minteg{V}{d\omega} = \oint\limits_{\bound{V}}\omega.
\end{align*}

\example{Reobtaining Familiar Theorems}
Many familiar integration theorems are in fact consequences of Stokes' theorem. Let us rederive them.

We start with a $1$-form $d\omega$, which will be integrated over a $1$-dimensional volume, i.e. a curve. We have
\begin{align*}
	\minteg{\gamma}{d\omega} = \omega(p) - \omega(q),
\end{align*}
where $q$ and $p$ are the start and end points of $\gamma$. This is the analogue of integrating a vector field along a curve.

Next, we consider a $2$-form written as the exterior derivative of a $1$-form. Writing $\omega = \omega_{i}d\chi^{i}$ we have
\begin{align*}
	d\omega = \del{j}{\omega_{i}}d\chi^{j}\wedge d\chi^{i}.
\end{align*}
Using Cartesian coordinates and restricting ourselves to two dimensions we have
\begin{align*}
	\df{\chi^{j}}\wedge\df{\chi^{i}}(\dot{\gamma}_{s}, \dot{\gamma}_{t})\dd{s}\dd{t} &= (\df{\chi^{j}}\otimes\df{\chi^{i}} - \df{\chi^{i}}\otimes\df{\chi^{j}})(\dot{\gamma}_{s}, \dot{\gamma}_{t})\dd{s}\dd{t} \\
	&= (\del{s}{\chi^{j}}\del{t}{\chi^{i}} - \del{s}{\chi^{i}}\del{t}{\chi^{j}})\dd{s}\dd{t} \\
	&= (\delta_{jk}\delta_{im} - \delta_{ik}\delta_{jm})\del{s}{\chi^{k}}\del{t}{\chi^{m}}\dd{s}\dd{t} \\
	&= \varepsilon_{jin}\varepsilon_{nkm}\del{s}{\chi^{k}}\del{t}{\chi^{m}}\dd{s}\dd{t} \\
	&= \varepsilon_{jik}\dd{S_{k}}.
\end{align*}
Thus we have
\begin{align*}
	\minteg{S}{d\omega} = \integ{S}{}{S_{k}}{\varepsilon_{ijk}\del{i}{\omega_{j}}} = \integ{S}{}{\vb{S}}{\cdot\curl{\vb*{\omega}}}.
\end{align*}
At the same time we have
\begin{align*}
	\minteg{\bound{S}}{\omega} &= \integ{\bound{S}}{}{t}{\omega_{a}\df{\chi^{a}}(\dot{\gamma}_{t})} \\
	                           &= \integ{\bound{S}}{}{t}{\omega_{a}\dv{\chi^{a}}{t}} \\
	                           &= \integ{\bound{S}}{}{\vb{r}}{\cdot\vb*{\omega}},
\end{align*}
hence
\begin{align*}
	\integ{S}{}{\vb{S}}{\cdot\curl{\vb*{\omega}}} = \integ{\bound{S}}{}{\vb{r}}{\cdot\vb*{\omega}},
\end{align*}
which is the more boring version of Stokes' theorem.

Next we consider a $2$-form and its exterior derivative. We have
\begin{align*}
	\minteg{V}{\df{\omega}} &= \frac{1}{2}\minteg{V}{\del{a}{\omega_{bc}}\df{\chi^{a}}\wedge\df{\chi^{b}}\wedge\df{\chi^{c}}}.
\end{align*}
As the components of $\omega$ may be chosen such that $\omega_{ab} = -\omega_{ba}$ we may write $\omega_{ab} = \varepsilon_{abc}\omega_{c}$ for some suitable (and arbitrary) choice of $\omega_{c}$. We thus have
\begin{align*}
	\minteg{V}{\df{\omega}} &= \frac{1}{2}\minteg{V}{\varepsilon_{bcd}\del{a}{\omega_{d}}\df{\chi^{a}}\wedge\df{\chi^{b}}\wedge\df{\chi^{c}}} \\
	                        &= \frac{1}{2}\integ{}{}{x^{1}}{\integ{}{}{x^{2}}{\integ{}{}{x^{3}}{\varepsilon_{bcd}\varepsilon_{abc}\del{a}{\omega_{d}}}}} \\
	                        &= \frac{1}{2}\integ{}{}{x^{1}}{\integ{}{}{x^{2}}{\integ{}{}{x^{3}}{(\delta_{ad}\delta_{bb} - \delta_{ab}\delta_{bd})\del{a}{\omega_{d}}}}} \\
	                        &= \integ{}{}{x^{1}}{\integ{}{}{x^{2}}{\integ{}{}{x^{3}}{\del{a}{\omega_{a}}}}} \\
	                        &= \integ{}{}{x^{1}}{\integ{}{}{x^{2}}{\integ{}{}{x^{3}}{\div{\vb*{\omega}}}}}.
\end{align*}
At the same time we have
\begin{align*}
	\minteg{\bound{V}}{\omega} &= \frac{1}{2}\minteg{\bound{V}}{\varepsilon_{ijk}\omega_{i}\df{\chi^{j}}\wedge\df{\chi^{k}}} \\
	                           &= \frac{1}{2}\integ{\bound{V}}{}{S_{m}}{\varepsilon_{ijk}\varepsilon_{jkm}\omega_{i}} \\
	                           &= \frac{1}{2}\integ{\bound{V}}{}{S_{m}}{(\delta_{im}\delta_{jj} - \delta_{ij}\delta_{jm})\omega_{i}} \\
	                           &= \integ{\bound{V}}{}{S_{i}}{\omega_{i}} \\
	                           &= \integ{\bound{V}}{}{\vb{S}}{\cdot\vb*{\omega}}.
\end{align*}
Hence we have
\begin{align*}
	\integ{}{}{x^{1}}{\integ{}{}{x^{2}}{\integ{}{}{x^{3}}{\div{\vb*{\omega}}}}} = \integ{\bound{V}}{}{\vb{S}}{\cdot\vb*{\omega}},
\end{align*}
which is the divergence theorem.

\example{The $n$-Dimensional Divergence Theorem}

\paragraph{The Geometry of Curved Space}
We can also impose a metric tensor such that $\vb{v}\cdot\vb{w} = g_{ab}v^{a}w^{b}$, where the metric tensor is symmetric and positive definite.

Dual vectors can be defined as linear maps from tangent vectors to scalars, i. e. on the form
\begin{align*}
	V(\vb{w}) = V_{a}w^{a}.
\end{align*}
In particular, the dual vector $\dd{f}$ can be defined as
\begin{align*}
	\dd{f}(\vb{v}) = v^{a}\del{a}{f} = \dv{f}{t}
\end{align*}
along a curve with $\vb{v}$ as a tangent. A basis for the space of dual vectors is $e^{a} = \dd{\chi^{a}}$. The tangent and dual spaces, if a metric exists, are related by $v_{a} = g_{ab}v^{b}$.

Curve lengths are defined and computed as before. By defining geodesics as curves that extremize path length, this gives a set of Christoffel symbols and therefore a covariant derivative and a sense of what it means for a vector to change along a curve.

\section{Differential Geometry on Manifolds}

\paragraph{Manifolds}
A manifold is a set which is locally isomorphic to $\R^{n}$. We will take this to mean that we can locally impose coordinates $\chi^{a}$ on the manifold.

More formally, a manifold is described by a number of sets $U_{i}\subset\R^{n}$ called charts. To each chart belongs a set of coordinate functions $\chi_{i}$ which map from a subset $M_{i}\subset M$ to $U_{i}$ such that $\chi_{i}$ is a smooth bijection. A set of charts such that every point $p\in M$ is found in at least one chart is called an atlas.

\paragraph{Manifolds and Vectors}
Even though manifolds are locally isomorphic to Euclidean space, the vectors that were previously developed do not make sense when applied to this Euclidean space.

\example{Tangent Vectors on $S_{2}$}
Consider $S_{2}$, the unit sphere in $\R^{3}$, and suppose you cover it with a layer of water like an ocean, introduce north and south poles, place two sailors on opposite sides of the equator and tell both of them to sail south at some given speed. In practice, this means that they should both travel in their local $-y$ direction. Assuming vectors in the two spaces to make sense, you would conclude that the sailors are sailing in the same direction at the same speed and could not possibly hit each other. The accident which would occur at the south pole would of course prove you wrong. This example is one, very verbose, way of expressing why the vectors in the local Euclidean spaces do not make sense.

This argument seems to have one hole in it, namely that $S_{2}$ is implicitly embedded in $\R^{3}$. Using this fact, the collision between the sailors could be deduced using the previously developed concepts of vectors. The reason why this would work is that you could impose a position vector in $\R^{3}$ onto every point on $S_{2}$. This is not a feature of more general manifolds, meaning that this hole does not exist for more general manifolds.

\paragraph{Tangent Vectors}
Tangent vectors describe how scalar fields change with displacement along a curve. In Euclidean space the tangent basis was composed of derivatives with respect to the set of coordinates. In general curved spaces, we define
\begin{align*}
	\tb{a} = \del{}{a}.
\end{align*}
Derivatives are linear operators, so at least the set of tangent bases span some vector space and it makes sense to call a derivative a vector. A general tangent vector is now
\begin{align*}
	X = X^{a}\tb{a} = X^{a}\del{}{a}.
\end{align*}
These live in the tangent space $\ts{p}{M}$ of the manifold $M$ at the point $p$.

To get more of a sense of how this can be related to vectors, consider the directional derivative
\begin{align*}
	\grad_{\vb{n}} = \vb{n}\cdot\grad = n^{a}\del{}{a}.
\end{align*}
When applied to Euclidean space, there is a direct correspondence between $\vb{n}$ and the directional derivative, as $\grad_{\vb{n}}{\vb{x}} = \vb{n}$. For more general manifolds, tangent vectors are defined to be directional derivatives. Note that this definition carries with it the same dependence on position as was previously warned about.

Tangent vectors transform according to
\begin{align*}
	X^{a}\del{}{a} = X^{a}\del{}{a}{(\chi^{\prime})^{b}}\delp{b},
\end{align*}
implying the transformation rule
\begin{align*}
	(X^{\prime})^{a} = \del{}{b}{(\chi^{\prime})^{a}}X^{b},
\end{align*}
which is the same as the transformation rule for contravariant vector components in Euclidean space.

\paragraph{Dual Vectors}
To define dual vectors, we first introduce the dual space as the set of all linear operations from the tangent space to real numbers. This is also a vector space. The basis for the space is defined such that
\begin{align*}
	\db{a}(\del{}{b}) = \kdelta{a}{b}.
\end{align*}

In Euclidean space the dual basis was constructed from the gradient. The only concept here that carries over to manifolds is a definition based on small changes in the coordinates. More specifically, for any smooth scalar field $f$ we define a dual vector field according to
\begin{align*}
	\df{f}(X) = Xf = X^{a}\del{}{a}{f}
\end{align*}
and call it the differential. This has a similar structure to an inner product if the dual vector field has components $df_{a} = \del{a}{f}$. These components correspond to those of the gradient in Euclidean space. The basis we desire is $\db{a} = d\chi^{a}$. These live in the dual space $\ds{p}{M}$ of the manifold $M$ at the point $p$.

The dual basis satisfies satisfies
\begin{align*}
	\df{\chi^{a}}(\del{}{b}) = \del{}{b}{\chi^{a}} = \kdelta{b}{a},
\end{align*}
as expected. Using this, we obtain
\begin{align*}
	\df{f} = (\del{a}{f})\df{\chi^{a}},
\end{align*}
which at least looks like the differential of a function.

The components transform according to
\begin{align*}
	\del{}{a}f = \delp{b}f\del{}{a}{(\chi^{\prime})^{b}},
\end{align*}
which is the transformation rule for covariant vector components.

\paragraph{Tensors}
Having identified a basis for the tangent and dual spaces, we may now construct tensors similarly to what we have previously done. Note now that as the tangent and dual vectors belong to different vector spaces, the notion of type $(n, m)$ tensors is more clear. This also explains why we needed to be careful with indices being up or down when studying Euclidean space, as the difference is huge for manifolds.

\paragraph{Flow of Vector Fields}
The tangent bundle of a manifold is defined as $TM = \bigcup\limits_{p}\ts{p}{M}$. A vector field is a map $X: M\to\tbun{M}$ such that $X(p)\in\ts{p}{M}$. Given this, we may define the flow of a vector field as a collection of curves $\gamma_{X}$ which given some starting point $p$ satisfy
\begin{align*}
\deval{\gamma_{X}}{\tau}{p, s} = \eval{X}_{\gamma_{X}(p, s)}.
\end{align*}
We may for a fixed $s$ define the function $\gamma_{sX}(p) = \gamma_{X}(p, s)$, which maps $M$ to itself.

\paragraph{Pushforwards and Pullbacks}
Consider some function $f$ which maps a manifold $M_{1}$ to another manifold $M_{2}$, as well as a function $g: M_{2}\to\R$. We then define the pullback of $g$ to $M_{1}$ by $f$ as $\pub{f}{g} = g\circ f$. We also define the pushforward of a vector $V\in\ts{p}{M_{1}}$ by $f$ as the map $\ts{p}{M_{1}}\to\ts{p}{M_{2}}$ such that $(\puf{f}{V})\phi = V(\pub{f}{\phi})$.

\paragraph{Tangents and the Pushforward}
Taking $f$ to map the cooordinates $\chi^{a}$ of $M_{1}$ to the coordinates $\eta^{\mu}$ of $M_{2}$, we have by definition that
\begin{align*}
	(\puf{f}{V})\phi &= V(\pub{f}{\phi}) \\
	                 &= V^{a}\del{}{a}{(\phi\circ f)} \\
	                 &= V^{a}\del{}{a}{f^{\mu}}\del{}{\mu}{\phi},
\end{align*}
meaning $\puf{f}{V} = V^{a}\del{}{a}f^{\mu}\del{}{\mu}$.

How do we interpret this? Consider some curve $\gamma$ in $M_{1}$ which is mapped to a curve $\alpha$ in $M_{2}$ by $f$. If $V$ is the tangent of $\gamma$ at some particular point, we have
\begin{align*}
	\dot{\alpha} = \dot{f^{\mu}}\del{}{\mu} = \del{}{a}{f^{\mu}}\dot{\chi^{a}}\del{}{\mu} = V^{a}\del{}{a}{f^{\mu}}\del{}{\mu},
\end{align*}
meaning that the pushforward of a tangent by $f$ is the tangent of the curve produced by $f$.

\paragraph{The Pullback of Tensors}
We can now define the pullback of a $(0, m)$ tensor on $M_{2}$ according to
\begin{align*}
	\pub{f}{\omega}(V_{1}, \dots, V_{m}) = \omega(\puf{f}{V_{1}}, \dots, \puf{f}{V_{m}}).
\end{align*}
Its components are
\begin{align*}
	\tensor{(\pub{f}{\omega})}{_{a_{1}\dots a_{m}}} =& \omega_{\mu_{1}\dots\mu_{m}}\prod\limits_{i = 1}^{m}\del{}{a_{i}}{f^{\mu_{i}}}.
\end{align*}

If $f$ is a bijection we may also define the more general pullback of a $(n, m)$ tensor on $M_{2}$ as
\begin{align*}
	\pub{f}{T}(V_{1}, \dots, V_{m}, \omega_{1}, \dots, \omega_{n}) = T(\puf{f}{V_{1}}, \dots, \puf{f}{V_{m}}, \pub{(f^{-1})}{\omega_{1}}, \dots, \pub{(f^{-1})}{\omega_{n}}).
\end{align*}
Its components are
\begin{align*}
	\tensor*{(\pub{f}{T})}{^{a_{1}\dots a_{n}}_{b_{1}\dots b_{m}}} = T^{\mu_{1}\dots \mu_{n}}_{\nu_{1}\dots\nu_{m}}\prod\limits_{i = 1}^{m}\del{}{a_{i}}f^{\nu_{i}}\prod\limits_{j = 1}^{n}\del{}{\mu_{j}}(f^{-1})^{b_{j}}.
\end{align*}

\paragraph{The Lie Derivative}
For a tensor field $T$ we define the Lie derivative in the $X$-direction as
\begin{align*}
	\lied{X}{T} = \lim\limits_{\varepsilon\to 0}\frac{1}{\varepsilon}\left(\pub{\gamma_{\varepsilon X}}{T} - T\right).
\end{align*}

\paragraph{An Expression for the Lie Derivative}
As the definition of the Lie derivative is similar to that of the usual derivative, it follows that it is linear in its argument and satisfies the product rule, where the product in question is now the tensor product. This means that we can find a general expression for the Lie derivative of a tensor field by considering its action on tangent and dual vectors.

First, for scalars we simply have $\lied{X}{f} = X^{a}\del{}{a}{f}$.

When studying the effect on tangent vectors we will have to be extra careful. Let $\gamma_{\varepsilon X}$ map $p$ to $q$, and let $\eval{Y}_{p}(\omega) = \eval{Y^{a}}_{p}\eval{\del{}{a}{\omega}}_{p}$ to clarify at what point we are working. To first order in $\varepsilon$ we then have
\begin{align*}
	\eval{\pub{\gamma_{\varepsilon X}}{Y}}_{q}(\omega) &= \eval{Y}_{q}(\pub{\gamma_{-\varepsilon X}}{\omega}) \\
	                                                   &= \eval{Y^{a}}_{q}\eval{\del{}{a}{(\omega\circ\gamma_{-\varepsilon X})}}_{q} \\
	                                                   &\approx \left(\eval{Y^{a}}_{p} + \varepsilon\eval{ X^{b}\del{}{b}{Y^{a}}}_{p}\right) \eval{\del{}{c}{\omega}}_{\gamma_{-\varepsilon X}(q)}\eval{\del{}{a}{(\gamma_{-\varepsilon X})^{c}}}_{q} \\
	                                                   &\approx \left(\eval{Y^{a}}_{p} + \varepsilon\eval{ X^{b}\del{}{b}{Y^{a}}}_{p}\right) \left(\kdelta{a}{c} - \varepsilon\del{}{a}{X^{c}}\right)\eval{\del{}{c}{\omega}}_{p} \\
	                                                   &\approx \left(\kdelta{a}{c}\eval{Y^{a}}_{p} + \varepsilon\kdelta{a}{c}\eval{ X^{b}\del{}{b}{Y^{a}}}_{p} - \varepsilon\del{}{a}{X^{c}}\eval{Y^{a}}_{p}\right)\eval{\del{}{c}{\omega}}_{p} \\
	                                                   &= \eval{Y^{a}}_{p}\eval{\del{}{a}{\omega}}_{p} + \varepsilon\eval{ X^{b}\del{}{b}{Y^{a}}}_{p}\eval{\del{}{a}{\omega}}_{p} - \eval{Y^{a}}_{p}\varepsilon\del{}{a}{X^{c}}\eval{\del{}{c}{\omega}}_{p},
\end{align*}
and thus $\lied{X}{Y} = \comm{X}{Y}$. In particular, we have $\lied{X}{\del{}{a}{}} = -(\del{}{a}{X^{b}})\del{}{b}{}$.

To find the Lie derivative of dual vectors, we will prove that the Lie derivative respects contradiction. This is evidently the case if the pullback does, so we will consider the pullback of the function $\omega_{a}V^{a}$. Introducing the isomorphism $f: \chi^{a}\to\eta^{\mu}$, where we use latin letters in the new coordinates to distinguish the two, we have
\begin{align*}
	\pub{f}{(\omega_{\mu}V^{\mu})}(\chi^{a}) = \omega_{\mu}V^{\mu}(\eta^{\mu}(\chi^{a})).
\end{align*}
Next we have, by definition, $\pub{f}{\omega}(U) = \omega(\puf{f}{U})$, and by a previously obtained relation we find
\begin{align*}
	(\pub{f}{\omega})_{a} = \omega_{\mu}\del{}{a}{\eta^{\mu}}.
\end{align*}
Finally we have by definition $\pub{f}{V}(\gamma) = V(\pub{(f^{-1})}{\gamma})$. As has been reasoned above, we have $(\pub{(f^{-1})}{\gamma})_{\mu} = \gamma_{a}\del{}{\mu}{\chi^{a}}$, this implies $\pub{f}{V}(\gamma) = V^{\mu}\gamma_{a}\del{}{\mu}{\chi^{a}}$, and finally
\begin{align*}
	(\pub{f}{V})^{a} = V^{\mu}\del{}{\mu}{\chi^{a}}.
\end{align*}
Putting this together we have
\begin{align*}
	\pub{f}{V}(\pub{f}{\omega}) = \omega_{\mu}\del{}{a}{\eta^{\mu}}V^{\nu}\del{}{\nu}{\chi^{a}} = \omega_{\mu}V^{\mu}.
\end{align*}
Thus the pullback respects contraction, and so must the Lie derivative. This implies
\begin{align*}
	\lied{X}{\omega(Y)} &= (\lied{X}{\omega})(Y) + \omega(\lied{X}{Y}) \\
	                    &= (\lied{X}{\omega})_{a}Y^{a} + \omega_{a}(X^{b}\del{}{b}{Y^{a}} - Y^{b}\del{}{b}{X^{a}}).
\end{align*}
As the left-hand side is simply equal to $X^{a}\del{}{a}{(\omega_{b}Y^{b})}$, we obtain
\begin{align*}
	(\lied{X}{\omega})_{a}Y^{a} &= X^{a}\del{}{a}{(\omega_{b}Y^{b})} - \omega_{b}X^{a}\del{}{a}{Y^{b}} + \omega_{a}Y^{b}\del{}{b}{X^{a}} \\
&= Y^{a}X^{b}\del{}{b}{\omega_{a}} + \omega_{b}Y^{a}\del{}{a}{X^{b}},
\end{align*}
and thus
\begin{align*}
	(\lied{X}{\omega})_{a} &= X^{b}\del{}{b}{\omega_{a}} + \omega_{b}\del{}{a}{X^{b}}.
\end{align*}
In particular, we have $\lied{X}{\df{\chi^{a}}} = \del{}{c}{X^{a}}\df{\chi^{c}}$.

We are now ready to compute the Lie derivative of a tensor. Writing
\begin{align*}
	T = \tensor*{T}{^{a_{1}\dots a_{n}}_{b_{1}\dots b_{m}}}\tensor*{e}{^{b_{1}\dots b_{m}}_{a_{1}\dots a_{n}}},
\end{align*}
the product rule implies
\begin{align*}
	\lied{X}{T} &= (\lied{X}{\tensor*{T}{^{a_{1}\dots a_{n}}_{b_{1}\dots b_{m}}}})\tensor*{e}{^{b_{1}\dots b_{m}}_{a_{1}\dots a_{n}}} + \tensor*{T}{^{a_{1}\dots a_{n}}_{b_{1}\dots b_{m}}}\sum\limits_{i = 1}^{n}\tensor*{e}{^{b_{1}\dots b_{m}}_{a_{1}\dots a_{i - 1}}}\otimes\lied{X}{\tensor*{e}{_{a_{i}}}}\otimes\tensor*{e}{_{a_{i + 1}\dots a_{n}}} + \tensor*{T}{^{a_{1}\dots a_{n}}_{b_{1}\dots b_{m}}}\sum\limits_{j = 1}^{m}\tensor*{e}{^{b_{1}\dots b_{j - 1}}}\otimes\lied{X}{\tensor*{e}{^{b_{j}}}}\otimes\tensor*{e}{^{b_{j + 1}\dots b_{m}}_{a_{1}\dots a_{n}}} \\
	            &= X^{a}\del{}{a}{\tensor*{T}{^{a_{1}\dots a_{n}}_{b_{1}\dots b_{m}}}}\tensor*{e}{^{b_{1}\dots b_{m}}_{a_{1}\dots a_{n}}} - \tensor*{T}{^{a_{1}\dots a_{n}}_{b_{1}\dots b_{m}}}\sum\limits_{i = 1}^{n}\tensor*{e}{^{b_{1}\dots b_{m}}_{a_{1}\dots a_{i - 1}}}\otimes \del{}{a_{i}}{X^{a}}\tensor*{e}{_{a}}\otimes\tensor*{e}{_{a_{i + 1}\dots a_{n}}} + \tensor*{T}{^{a_{1}\dots a_{n}}_{b_{1}\dots b_{m}}}\sum\limits_{j = 1}^{m}\tensor*{e}{^{b_{1}\dots b_{j - 1}}}\otimes \del{}{a}{X^{b_{j}}}\tensor*{e}{^{a}}\otimes\tensor*{e}{^{b_{j + 1}\dots b_{m}}_{a_{1}\dots a_{n}}} \\
	            &= X^{a}\del{}{a}{\tensor*{T}{^{a_{1}\dots a_{n}}_{b_{1}\dots b_{m}}}}\tensor*{e}{^{b_{1}\dots b_{m}}_{a_{1}\dots a_{n}}} - \sum\limits_{i = 1}^{n}\tensor*{T}{^{a_{1}\dots a_{n}}_{b_{1}\dots b_{m}}}\del{}{a_{i}}{X^{a}}\tensor*{e}{^{b_{1}\dots b_{m}}_{a_{1}\dots a_{i - 1}}}\otimes\tensor*{e}{_{a}}\otimes\tensor*{e}{_{a_{i + 1}\dots a_{n}}} + \sum\limits_{j = 1}^{m}\tensor*{T}{^{a_{1}\dots a_{n}}_{b_{1}\dots b_{m}}}\del{}{a}{X^{b_{j}}}\tensor*{e}{^{b_{1}\dots b_{j - 1}}}\otimes\tensor*{e}{^{a}}\otimes\tensor*{e}{^{b_{j + 1}\dots b_{m}}_{a_{1}\dots a_{n}}}.
\end{align*}
There are no free indices, so we may switch the places of the numbered indices and $a$ to obtain
\begin{align*}
	\lied{X}{T} =& X^{a}\del{}{a}{\tensor*{T}{^{a_{1}\dots a_{n}}_{b_{1}\dots b_{m}}}}\tensor*{e}{^{b_{1}\dots b_{m}}_{a_{1}\dots a_{n}}} - \sum\limits_{i = 1}^{n}\tensor*{T}{^{a_{1}\dots a_{i - 1}aa_{i + 1}\dots a_{n}}_{b_{1}\dots b_{m}}}\del{}{a}{X^{a_{i}}}\tensor*{e}{^{b_{1}\dots b_{m}}_{a_{1}\dots a_{i - 1}}}\otimes\tensor*{e}{_{a_{i}}}\otimes\tensor*{e}{_{a_{i + 1}\dots a_{n}}} \\
	             &+ \sum\limits_{j = 1}^{m}\tensor*{T}{^{a_{1}\dots a_{n}}_{b_{1}\dots b_{i - 1}ab_{i + 1}\dots b_{m}}}\del{}{b_{j}}{X^{a}}\tensor*{e}{^{b_{1}\dots b_{j - 1}}}\otimes\tensor*{e}{^{b_{j}}}\otimes\tensor*{e}{^{b_{j + 1}\dots b_{m}}_{a_{1}\dots a_{n}}} \\
	            =& \left(X^{a}\del{}{a}{\tensor*{T}{^{a_{1}\dots a_{n}}_{b_{1}\dots b_{m}}}} - \sum\limits_{i = 1}^{n}\tensor*{T}{^{a_{1}\dots a_{i - 1}aa_{i + 1}\dots a_{n}}_{b_{1}\dots b_{m}}}\del{}{a}{X^{a_{i}}} + \sum\limits_{j = 1}^{m}\tensor*{T}{^{a_{1}\dots a_{n}}_{b_{1}\dots b_{i - 1}ab_{i + 1}\dots b_{m}}}\del{}{b_{j}}{X^{a}}\right)\tensor*{e}{^{b_{1}\dots b_{m}}_{a_{1}\dots a_{n}}},
\end{align*}
allowing us to identify
\begin{align*}
	\tensor*{(\lied{X}{T})}{^{a_{1}\dots a_{n}}_{b_{1}\dots b_{m}}} &= X^{a}\del{}{a}{\tensor*{T}{^{a_{1}\dots a_{n}}_{b_{1}\dots b_{m}}}} - \sum\limits_{i = 1}^{n}\tensor*{T}{^{a_{1}\dots a_{i - 1}aa_{i + 1}\dots a_{n}}_{b_{1}\dots b_{m}}}\del{}{a}{X^{a_{i}}} + \sum\limits_{j = 1}^{m}\tensor*{T}{^{a_{1}\dots a_{n}}_{b_{1}\dots b_{i - 1}ab_{i + 1}\dots b_{m}}}\del{}{b_{j}}{X^{a}}.
\end{align*}

\paragraph{Connections}
A connection is an operator on a tensor space that satisfies the following:
\begin{itemize}
	\item $\dcov{X}{f} = Xf = X^{a}\del{}{a}{f}$ for a scalar field $f$.
	\item $\dcov{X + Y}{T} = \dcov{X}{T} + \dcov{Y}{T}$.
	\item $\dcov{fX}{T} = f\dcov{X}{T}$.
	\item $\dcov{X}{(TS)} = (\dcov{X}{T})S + T\dcov{X}{S}$.
\end{itemize}

\paragraph{Connections on Manifolds}
On a manifold, a connection is specified by choosing $n$ independent vectors $X_{i}$ and defining
\begin{align*}
	\dcov{X_{i}}{X_{j}} = \chris{k}{i}{j}X_{k},
\end{align*}
where the expansion coefficients are called connection coefficients or Christoffel symbols. There is no unique way to do this, as the connection then depends on the choice of vectors.

\paragraph{The Connection of a Tensor Field}
Specify the connection according to $\dcov{\del{}{a}{}}{\del{}{b}{}} = \chris{c}{a}{b}\del{}{c}{}$. For a tensor
\begin{align*}
	T = \tensor*{T}{^{a_{1}\dots a_{n}}_{b_{1}\dots b_{m}}}\tensor*{e}{^{b_{1}\dots b_{m}}_{a_{1}\dots a_{n}}}
\end{align*}
we find using the product rule
\begin{align*}
	\dcov{X}{T} =& (\dcov{X}{\tensor*{T}{^{a_{1}\dots a_{n}}_{b_{1}\dots b_{m}}}})\tensor*{e}{^{b_{1}\dots b_{m}}_{a_{1}\dots a_{n}}} + \tensor*{T}{^{a_{1}\dots a_{n}}_{b_{1}\dots b_{m}}}\sum\limits_{i = 1}^{n}\tensor*{e}{^{b_{1}\dots b_{m}}_{a_{1}\dots a_{i - 1}}}\otimes\dcov{X}{\tensor*{e}{_{a_{i}}}}\otimes\tensor*{e}{_{a_{i + 1}\dots a_{n}}} + \tensor*{T}{^{a_{1}\dots a_{n}}_{b_{1}\dots b_{m}}}\sum\limits_{j = 1}^{m}\tensor*{e}{^{b_{1}\dots b_{j - 1}}}\otimes\dcov{X}{\tensor*{e}{^{b_{j}}}}\otimes\tensor*{e}{^{b_{j + 1}\dots b_{m}}_{a_{1}\dots a_{n}}}.
\end{align*}
To proceed we will need to require that the connection respect contraction, yielding
\begin{align*}
	\dcov{X}{\omega(Y)} &= \dcov{X}{\omega}(Y) + \omega(\dcov{X}{Y}) \\
	                    &= (\dcov{X}{\omega})_{a}Y^{a} + \omega_{a}(X^{b}\dcov{b}{(Y^{c}\del{}{c}{})})^{a} \\
	                    &= (X^{b}\dcov{b}{\omega})_{a}Y^{a} + \omega_{a}(X^{b}(\del{}{b}{Y^{c}}\del{}{c}{} + Y^{c}\chris{d}{b}{c}\del{}{d}{}))^{a} \\
	                    &= X^{b}(\dcov{b}{\omega})_{a}Y^{a} + \omega_{a}X^{b}\del{}{b}{Y^{a}} + \omega_{a}X^{b}Y^{c}\chris{a}{b}{c}.
\end{align*}
On the other side we have
\begin{align*}
	\dcov{X}{\omega(Y)} = X^{a}(Y^{b}\del{}{a}{\omega_{b}} + \omega_{b}\del{}{a}{Y^{b}}),
\end{align*}
implying
\begin{align*}
	Y^{a}X^{b}(\dcov{b}{\omega})_{a} &= X^{a}(Y^{b}\del{}{a}{\omega_{b}} + \omega_{b}\del{}{a}{Y^{b}}) - \omega_{a}X^{b}\del{}{b}{Y^{a}} - \omega_{a}X^{b}Y^{c}\chris{a}{b}{c} \\
	                                 &= Y^{a}X^{b}\del{}{b}{\omega_{a}} - Y^{a}X^{b}\omega_{c}\chris{c}{b}{a},
\end{align*}
and as this must be true for all $X$ and $Y$ we have
\begin{align*}
	\dcov{b}{\omega} &= (\del{}{b}{\omega_{a}} - \omega_{c}\chris{c}{b}{a})\df{\chi^{a}}.
\end{align*}

We now proceed according to
\begin{align*}
	\dcov{X}{T} =& X^{a}\del{}{a}{\tensor*{T}{^{a_{1}\dots a_{n}}_{b_{1}\dots b_{m}}}}\tensor*{e}{^{b_{1}\dots b_{m}}_{a_{1}\dots a_{n}}} + \tensor*{T}{^{a_{1}\dots a_{n}}_{b_{1}\dots b_{m}}}\sum\limits_{i = 1}^{n}X^{a}\chris{b}{a}{a_{i}}\tensor*{e}{^{b_{1}\dots b_{m}}_{a_{1}\dots a_{i - 1}}}\otimes\tensor*{e}{_{b}}\otimes\tensor*{e}{_{a_{i + 1}\dots a_{n}}} - \tensor*{T}{^{a_{1}\dots a_{n}}_{b_{1}\dots b_{m}}}\sum\limits_{j = 1}^{m}X^{a}\chris{b_{j}}{a}{b}\tensor*{e}{^{b_{1}\dots b_{j - 1}}}\otimes\tensor*{e}{^{b}}\otimes\tensor*{e}{^{b_{j + 1}\dots b_{m}}_{a_{1}\dots a_{n}}} \\
	            =& X^{a}\left(\del{}{a}{\tensor*{T}{^{a_{1}\dots a_{n}}_{b_{1}\dots b_{m}}}} + \sum\limits_{i = 1}^{n}\chris{a_{i}}{a}{b}\tensor*{T}{^{a_{1}\dots a_{i - 1}ba_{i + 1}\dots a_{n}}_{b_{1}\dots b_{m}}} - \sum\limits_{j = 1}^{m}\chris{b}{a}{b_{j}}\tensor*{T}{^{a_{1}\dots a_{n}}_{b_{1}\dots b_{j - 1}bb_{j + 1}\dots b_{m}}}\right)\tensor*{e}{^{b_{1}\dots b_{m}}_{a_{1}\dots a_{n}}},
\end{align*}
and we recognize
\begin{align*}
	\tensor*{(\dcov{X}{T})}{^{a_{1}\dots a_{n}}_{b_{1}\dots b_{m}}} &= X^{a}\left(\del{}{a}{\tensor*{T}{^{a_{1}\dots a_{n}}_{b_{1}\dots b_{m}}}} + \sum\limits_{i = 1}^{n}\chris{a_{i}}{a}{b}\tensor*{T}{^{a_{1}\dots a_{i - 1}ba_{i + 1}\dots a_{n}}_{b_{1}\dots b_{m}}} - \sum\limits_{j = 1}^{m}\chris{b}{a}{b_{j}}\tensor*{T}{^{a_{1}\dots a_{n}}_{b_{1}\dots b_{j - 1}bb_{j + 1}\dots b_{m}}}\right).
\end{align*}

\paragraph{Parallel Transport}
A vector $X$ is termed parallel if $\dcov{a}{X} = 0$. This defines $n^{2}$ equations for the $n$ components of $X$, meaning that the system is overdetermined, and generally has no solution on a manifold. We may, however, define $X$ to be parallel along a curve $\gamma$ if
\begin{align*}
	\dcov{\dot{\gamma}}{X} = 0.
\end{align*}
This allows us to define the parallel transport as the vector field that solves the above equation with the vector $X$ as its initial condition. This defines $n$ equations for the $n$ components, and the system is solvable.

In particular, using the properties of the connection we find
\begin{align*}
	\dcov{\dot{\gamma}}{X} &= \dcov{\dot{\chi}^{a}\del{}{a}{}}{X^{c}\del{}{c}{}} \\
	                       &= \dot{\chi}^{a}\dcov{a}{X^{c}\del{}{c}{}} \\
	                       &= \dot{\chi}^{a}((\dcov{a}{X^{c}})\del{}{c}{} + X^{c}\dcov{a}{\del{}{c}{}}) \\
	                       &= \dot{\chi}^{a}(\del{}{a}{X^{b}} + X^{c}\chris{b}{a}{c})\del{}{b}{}.
\end{align*}

\paragraph{Geodesics and the Geodesic Equation}
A geodesic is defined as a curve with a tangent vector that is parallel along itself.

By definition a geodesic satisfies
\begin{align*}
	\dcov{\dot{\gamma}}{\dot{\gamma}} = \dot{\chi}^{a}(\del{}{a}{\dot{\chi}^{b}} + \dot{\chi}^{c}\chris{b}{a}{c})\del{}{b}{} = (\ddot{\chi}^{b} + \dot{\chi}^{a}\dot{\chi}^{c}\chris{b}{a}{c})\del{}{b}{} = 0,
\end{align*}
and thus
\begin{align*}
	\ddot{\chi}^{b} + \dot{\chi}^{a}\dot{\chi}^{c}\chris{b}{a}{c} = 0.
\end{align*}
This is the geodesic equation. Given a starting point and a tangent vector, it is solvable.

\paragraph{Torsion}
The torsion tensor is a $(1, 2)$ tensor defined as
\begin{align*}
	T(X, Y) = \dcov{X}{Y} - \dcov{Y}{X} - \comm{X}{Y}.
\end{align*}
To find its components, we note that
\begin{align*}
	T_{ab} = T(\del{}{a}{}, \del{}{b}{}) &= \dcov{a}{\del{}{b}{}} - \dcov{b}{\del{}{a}{}} - \comm{\del{}{a}{}}{\del{}{b}{}} \\
	                                 &= \chris{c}{a}{b}\del{}{c}{} - \chris{c}{b}{a}\del{}{c}{} - (\del{}{a}{\del{}{b}{}} - \del{}{b}{\del{}{a}{}}) \\
	                                 &= (\chris{c}{a}{b} - \chris{c}{b}{a})\del{}{c}{},
\end{align*}
implying
\begin{align*}
	T_{ab}^{c} &= \chris{c}{a}{b} - \chris{c}{b}{a}.
\end{align*}

\paragraph{Curvature}
Consider some vector $Z$ parallel transported along a small closed loop. The parallel transport is linear, so the result of this process must be connected to some $(1, 1)$ tensor. Supposing that the loop is spanned by $X$ and $Y$, we have
\begin{align*}
	Z^{\prime} - Z = R(X, Y)Z = \dcov{X}{\dcov{Y}{Z}} - \dcov{Y}{\dcov{X}{Z}} - \dcov{\comm{X}{Y}}{Z}.
\end{align*}
We define $R(X, Y)Z$ as the Riemann curvature tensor. It is a $(1, 3)$ tensor. Its components are defined by
\begin{align*}
	R(\del{}{a}{}, \del{}{b}{})\del{}{c}{} &= \tensor{R}{^{d}_{cab}}\del{}{d}{} \\
	                                 &= \dcov{a}{\dcov{b}{\del{}{c}{}}} - \dcov{b}{\dcov{a}{\del{}{c}{}}} \\
	                                 &= \dcov{a}{\chris{f}{b}{c}\del{}{f}{}} - \dcov{b}{\chris{f}{a}{c}\del{}{f}{}} \\
	                                 &= (\dcov{a}{\chris{f}{b}{c}})\del{}{f}{} + \chris{f}{b}{c}\dcov{a}{\del{}{f}{}} - (\dcov{b}{\chris{f}{a}{c}})\del{}{f}{} - \chris{f}{a}{c}\dcov{b}{\del{}{f}{}} \\
	                                 &= (\del{}{a}{\chris{d}{b}{c}} + \chris{f}{b}{c}\chris{d}{a}{f} - \del{}{b}{\chris{d}{a}{c}} - \chris{f}{a}{c}\chris{d}{b}{f})\del{}{d}{}, 
\end{align*}
and thus
\begin{align*}
	\tensor{R}{^{d}_{cab}} = \del{}{a}{\chris{d}{b}{c}} - \del{}{b}{\chris{d}{a}{c}} + \chris{f}{b}{c}\chris{d}{a}{f} - \chris{f}{a}{c}\chris{d}{b}{f}.
\end{align*}
Note the placements of the indices.

\paragraph{The Metric Tensor}
The metric tensor will be taken as the $(0, 2)$ tensor that defines inner products on manifolds. The inner product, and therefore also the metric tensor, is a map from $\ts{p}{M}\times\ts{p}{M}$ that is symmetric and positive definite. Using this we may extend more of the previously performed work, for instance on curve length.

\paragraph{Metric Compatibility}
A connection is metric compatible if $\dcov{X}{g} = 0$ for all vectors $X$.

\paragraph{The Levi-Civita Connection}
For any manifold with some metric there exists a unique connection that is both metric compatible and torsion free. This connection is termed the Levi-Civita connection.

To prove that it is unique, suppose that there exists two different metric compatible connections with connection coefficients $\chris{c}{a}{b}$ and $C^{c}_{ab}$. For the connection to be metric compatible, we must have $(\dcov{a}{g})_{bc} = 0$ for both connections, and thus
\begin{align*}
	\del{}{a}{g_{bc}} - \chris{d}{a}{b}g_{dc} - \chris{d}{a}{c}g_{bd} = 0,\ \del{}{a}{g_{bc}} - C^{d}_{ab}g_{dc} - C^{d}_{ac}g_{bd} = 0,
\end{align*}
and we find
\begin{align*}
	\chris{d}{a}{b}g_{dc} + \chris{d}{a}{c}g_{bd} = C^{d}_{ab}g_{dc} + C^{d}_{ac}g_{bd}.
\end{align*}
We may now construct the difference $\chris{d}{a}{b} - C^{d}_{ab} = D^{d}_{ab}$ to write
\begin{align*}
	D^{d}_{ab}g_{dc} + D^{d}_{ac}g_{bd} = 0.
\end{align*}
If both connections are torsion free, we may switch the lower indices in every term to find
\begin{align*}
	D^{d}_{ac}g_{bd} = D^{d}_{ca}g_{bd} = -D^{d}_{cb}g_{da} = -D^{d}_{bc}g_{da} = D^{d}_{ba}g_{cd} = D^{d}_{ab}g_{dc},
\end{align*}
implying $D^{d}_{ab}g_{dc} = 0$. Multiplying by $g^{cf}$ yields $D^{f}_{ab} = 0$, and so the Levi-Civita connection is unique.

\paragraph{Curves of Minimal Length}
As the metric defines length, we define the curve length as
\begin{align*}
	l_{\gamma} = \inte{\gamma}{}\dd{s}{} = \inte{0}{1}\dd{t}{\sqrt{g_{ab}\dot{\chi}^{a}\dot{\chi}^{b}}}.
\end{align*}
Defining $\sqrt{\lag} = g_{ab}\dot{\chi}^{a}\dot{\chi}^{b}$, the curve that minimizes the distance between the start and end points satisfies
\begin{align*}
	\del{}{a}{\sqrt{\lag}} - \dv{t}\pdv{\sqrt{\lag}}{\dot{\chi}^{a}} = \frac{1}{2\sqrt{\lag}}\left(\del{}{a}{\lag} - \sqrt{\lag}\dv{t}\left(\frac{1}{\sqrt{\lag}}\pdv{\lag}{\dot{\chi}^{a}}\right)\right) = 0.
\end{align*}
One can always choose a parametrization such that $\sqrt{\lag} = 1$ (the arc length parametrization is one example), yielding
\begin{align*}
	\frac{1}{2\sqrt{\lag}}\left(\del{}{a}{\lag} - \dv{t}\pdv{\lag}{\dot{\chi}^{a}}\right) = 0,
\end{align*}
which is equivalent to extremizing the integral of $\lag$. In terms of the coordinate functions we thus have
\begin{align*}
	\del{}{a}{g_{bc}}\dot{\chi}^{b}\dot{\chi}^{c} - \dv{t}\pdv{\lag}{\dot{\chi}^{a}} = \del{}{a}{g_{bc}}\dot{\chi}^{b}\dot{\chi}^{c} - \dv{t}(2g_{ab}\dot{\chi}^{b}) = \del{}{a}{g_{bc}}\dot{\chi}^{b}\dot{\chi}^{c} - 2g_{ab}\ddot{\chi}^{b} - 2\del{}{c}{g_{ab}}\dot{\chi}^{b}\dot{\chi}^{c} = 0.
\end{align*}
Multiplying by $-\frac{1}{2}g^{da}$ we find
\begin{align*}
	g^{da}g_{ab}\ddot{\chi}^{b} - \frac{1}{2}g^{da}\del{}{a}{g_{bc}}\dot{\chi}^{b}\dot{\chi}^{c} + g^{da}\del{}{c}{g_{ab}}\dot{\chi}^{b}\dot{\chi}^{c} = \ddot{\chi}^{d} + \frac{1}{2}g^{da}(2\del{}{c}{g_{ab}} - \del{}{a}{g_{bc}})\dot{\chi}^{b}\dot{\chi}^{c} = 0.
\end{align*}
Renaming indices for convenience we find
\begin{align*}
	\ddot{\chi}^{b} + \frac{1}{2}g^{bd}(2\del{}{c}{g_{da}} - \del{}{d}{g_{ac}})\dot{\chi}^{a}\dot{\chi}^{c} = 0.
\end{align*}

\paragraph{Geodesics and Minimum-Length Curves}
Geodesics and curves of minimal length coincide if
\begin{align*}
	\chris{b}{a}{c} = \frac{1}{2}g^{bd}(2\del{}{c}{g_{da}} - \del{}{d}{g_{ac}}).
\end{align*}
As the above identification is done based on a quantity that is symmetric in the lower indices, we cannot find any information about the antisymmetric part of the connection coefficients from this.

\paragraph{The Levi-Civita Connection and Geodesics}
The connection coefficients (or Christoffel symbols) defined by the Levi-Civita connection are symmetric due to the torsion being zero. This implies
\begin{align*}
	\chris{d}{a}{b} = \frac{1}{2}(\chris{d}{a}{b} + \chris{d}{b}{a}) = \frac{1}{4}g^{dc}(2\del{}{b}{g_{ca}} - \del{}{c}{g_{ab}} + 2\del{}{a}{g_{cb}} - \del{}{c}{g_{ba}}) = \frac{1}{2}g^{dc}(\del{}{b}{g_{ac}} + \del{}{a}{g_{cb}} - \del{}{c}{g_{ab}}).
\end{align*}

\paragraph{The Induced Metric}
Given some immersion $f$ of $M_{1}$ into $M_{2}$ and supposing that the metric $g$ exists on $M_{2}$, this induces a metric $\tilde{g} = \pub{f}{g}$ on $M_{1}$.

\paragraph{Components of the Induced Metric}
Suppose that there is an immersion $f: \chi^{a}\to\eta^{\mu}$ of one manifold into another, and that the metric is $g$ in the outer manifold. By definition the induced metric satisfies $\tilde{g}(U, V) = \pub{f}{g}(U, V) = g(\puf{f}{U}, \puf{f}{V})$, implying
\begin{align*}
	\tilde{g}_{ab}U^{a}V^{b} = g_{\mu\nu}U^{a}\del{}{a}{\eta^{\mu}}V^{b}\del{}{b}{\eta^{\nu}},
\end{align*}
and as this is true for any pair of vectors we recognize
\begin{align*}
	\tilde{g}_{ab} = g_{\mu\nu}\del{}{a}{\eta^{\mu}}\del{}{b}{\eta^{\nu}}.
\end{align*}

\example{A Function Surface In Euclidean Space}
Consider a surface $\chi^{n} = h(\chi^{\mu})$, where $h$ is independent of $\chi^{n}$, and suppose that we know the metric in these coordinates. This surface is a manifold with the $\chi^{\mu}$ being a natural choice of coordinates, and there is also a natural choice of immersion. All derivatives of the new coordinates reduce to Kronecker deltas for all but the last coordinates. For the final coordinate we find
\begin{align*}
	\del{}{a}{\eta^{n}} = \del{}{a}{h},
\end{align*}
implying
\begin{align*}
	\tilde{g}_{ab} = g_{ab} + g_{nn}\del{}{a}{h}\del{}{b}{h}.
\end{align*}
In particular, if choosing Cartesian coordinates we find
\begin{align*}
	\dd{s}^{2} = \sum\limits_{a}(1 + (\del{}{i}{h})^{2})(\dd{x}^{i})^{2},
\end{align*}
which is the expected result.

\paragraph{Curvature and the Metric}
Using a Levi-Civita connection, we can write the curvature as
\begin{align*}
	\tensor{R}{^{d}_{cab}} =& \frac{1}{2}\del{}{a}{\left(g^{df}(\del{}{c}{g_{bf}} + \del{}{b}{g_{fc}} - \del{}{f}{g_{bc}})\right)} - \frac{1}{2}\del{}{b}{\left(g^{df}(\del{}{c}{g_{af}} + \del{}{a}{g_{fc}} - \del{}{f}{g_{ac}})\right)} + \chris{f}{b}{c}\chris{d}{a}{f} - \chris{f}{a}{c}\chris{d}{b}{f}.
\end{align*}
We may then introduce the contravariant curvature components
%d to m
%c to b
%a to c
%b to d
\begin{align*}
	R_{abcd} &= g_{am}\tensor{R}{^{m}_{bcd}} \\
	         &= g_{am}\left(\frac{1}{2}\del{}{c}{\left(g^{mf}(\del{}{b}{g_{df}} + \del{}{d}{g_{fb}} - \del{}{f}{g_{db}})\right)} - \frac{1}{2}\del{}{d}{\left(g^{mf}(\del{}{b}{g_{cf}} + \del{}{c}{g_{fb}} - \del{}{f}{g_{cb}})\right)} + \chris{f}{d}{b}\chris{m}{c}{f} - \chris{f}{c}{b}\chris{m}{d}{f}\right) \\
	         &= g_{am}\left(\frac{1}{2}\del{}{c}{\left(g^{mf}(\del{}{b}{g_{df}} + \del{}{d}{g_{fb}} - \del{}{f}{g_{db}})\right)} - \frac{1}{2}\del{}{d}{\left(g^{mf}(\del{}{b}{g_{cf}} + \del{}{c}{g_{fb}} - \del{}{f}{g_{cb}})\right)}\right) + g_{am}\left(\chris{f}{d}{b}\chris{m}{c}{f} - \chris{f}{c}{b}\chris{m}{d}{f}\right) \\
	R_{abcd} &= \frac{1}{2}\left(\del{}{a}{\del{}{d}{g_{bc}}} + \del{}{b}{\del{}{c}{g_{ad}}} - \del{}{a}{\del{}{c}{g_{bd}}} - \del{}{b}{\del{}{d}{g_{ac}}}\right) + g_{fh}(\chris{f}{b}{c}\chris{h}{a}{d} - \chris{f}{b}{d}\chris{h}{a}{c}).
\end{align*}
We find $R_{abcd} = R_{cdab} = -R_{abdc} = -R_{bacd}$. In addition, the curvature satisfies the Bianchi identity $R(X, Y)Z + R(Y, Z)X + R(Z, X)Y = 0$, implying $R_{abcd} + R_{acdb} + R_{adbc} = 0$, meaning that the total number of independent components is $\frac{1}{12}n^{2}(n^{2} - 1)$.

\paragraph{The Ricci Tensor}
The Ricci tensor is defined as $R_{ab} = \tensor{R}{^{c}_{acb}}$.

\paragraph{The Ricci Scalar}
The Ricci scalar is defined as the trace of the Ricci scalar: $\mathcal{R} = g^{ab}R_{ab} = g^{ab}\tensor{R}{^{c}_{acb}}$.

\paragraph{The Einstein Tensor}
The Einstein tensor is defined as $G_{ab} = R_{ab} - \frac{1}{2}g_{ab}\mathcal{R}$.

\paragraph{Killing Fields}
$K$ is a Killing field (be careful looking this up on the internet) if $\lied{K}{g} = 0$.

\paragraph{The Lie Derivative with Killing Fields}
Let $K$ be a Killing field. We then obtain
\begin{align*}
	\lied{K}{g_{ab}} &= K^{c}\del{}{c}{g_{ab}} + g_{ac}\del{}{b}{K^{c}} + g_{cb}\del{}{a}{K^{c}}.
\end{align*}
Using the Levi-Civita connection we find
\begin{align*}
	\del{}{b}{g_{ac}} + \del{}{a}{g_{cb}} - \del{}{c}{g_{ab}} = 2g_{cd}\chris{d}{a}{b},
\end{align*}
hence
\begin{align*}
	\del{}{c}{g_{ab}} = g_{bd}\chris{d}{a}{c} + g_{ad}\chris{d}{b}{c},
\end{align*}
yielding
\begin{align*}
	\lied{K}{g_{ab}} &= K^{c}(\chris{d}{a}{c}g_{bd} + \chris{d}{b}{c}g_{ad}) + g_{ac}\del{}{b}{K^{c}} + g_{cb}\del{}{a}{K^{c}} \\
	                 &= g_{cb}(K^{d}\chris{c}{a}{d} + \del{}{a}{K^{c}}) + g_{ac}(K^{d}\chris{c}{b}{d} + \del{}{b}{K^{c}}) \\
	                 &= g_{cb}(\del{}{a}{K^{c}} + K^{d}\chris{c}{d}{a}) + g_{ac}(\del{}{b}{K^{c}} + K^{d}\chris{c}{d}{b}) \\
	                 &= g_{cb}\dcov{a}{K^{c}} + g_{ac}\dcov{b}{K^{c}} \\
	                 &= \dcov{b}{K_{a}} + \dcov{a}{K_{b}} = 0,
\end{align*}
and all Killing fields must satisfy this relation.

\paragraph{Differential Forms}
The set of $p$-forms, or differential forms, is the set of $(0, p)$ tensors that are completely antisymmetric. They are constructed using the wedge product, defined as
\begin{align*}
	\bigwedge\limits_{k = 1}^{p}d\chi^{a_{k}} = \sum\limits_{\sigma\in S_{p}}\text{sgn}(\sigma)\bigotimes_{k = 1}^{p}d\chi^{a_{\sigma(k)}}.
\end{align*}
Here $S_{p}$ is the set of permutations of $p$ elements. There exists
\begin{align*}
	n_{p}^{N} = {N\choose k}
\end{align*}
basis elements. We note that the wedge product is antisymmetric under the exchange of two basis elements. Hence, once an ordering of indices has been chosen, any permutation will simply create a linearly dependent map.

Consider now some antisymmetric tensor $\omega$. Introducing the antisymmetrizer
\begin{align*}
\bigotimes_{k = 1}^{p}d\chi^{[a_{\sigma(k)}]} = \frac{1}{p!}\sum\limits_{\sigma\in S_{p}}\text{sgn}(\sigma)\bigotimes_{k = 1}^{p}d\chi^{a_{\sigma(k)}},
\end{align*}
the symmetry yields
\begin{align*}
\omega = \omega_{a_{1}\dots a_{p}}\bigotimes_{k = 1}^{p}d\chi^{a_{\sigma(k)}} = \omega_{a_{1}\dots a_{p}}\bigotimes_{k = 1}^{p}d\chi^{[a_{\sigma(k)}]} = \frac{1}{p!}\omega_{a_{1}\dots a_{p}}\bigwedge\limits_{k = 1}^{p}d\chi^{a_{k}}.
\end{align*}

\paragraph{The Exterior Derivative}
We define the exterior derivative of a differential form according to
\begin{align*}
	d\omega = \frac{1}{p!}\del{}{a_{1}}{\omega_{a_{2}\dots a_{p + 1}}}\bigwedge\limits_{k = 1}^{p + 1}d\chi^{a_{k}},
\end{align*}
which is a $p + 1$-form. This notation makes sense, as at least in the case of a $0$-form, we obtain
\begin{align*}
	d\omega = \del{}{a}{\omega}d\chi^{a},
\end{align*}
which is exactly the form of a dual vector. Somehow this transforms as a tensor.

\paragraph{Integration of Differential Forms}
Consider a set of $p$ tangent vectors $X_{i}$. The corresponding coordinate displacements are $\dd{\chi_{i}^{a}} = X_{i}^{a}\dd{t_{i}}$, with no sum over $i$. We would now like to compute the $p$-dimensional volume defined by the $X_{i}$ and $\dd{t_{i}}$. We expect that if any of the $X_{i}$ are linearly dependent the volume should be zero. We also expect that the volume be linear in the $X_{i}$. This implies
\begin{align*}
	\dd{V_{p}} = \omega(X_{1}, \dots, X_{p})\dd{t_{1}}\dots\dd{t_{p}}
\end{align*}
for some differential form $\omega$. We now define the integral over the $p$-volume $S$ over the $p$-form $\omega$ as
\begin{align*}
	\inte{S}{}{\omega} = \inte{}{}\dd{t_{1}}{\dots \inte{}{}\dd{t_{p}}{\omega(\dot{\gamma}_{1},\dots, \dot{\gamma}_{p})}}.
\end{align*}
Here the $\gamma_{i}$ are the set of curves that span $S$, the dot symbolizes the derivative with respect to the individual curve parameters and the right-hand integration is performed over the appropriate set of parameter values.

\paragraph{Stokes' Theorem}
Stokes' theorem relates the integral of a differential form $d\omega$ over some subset $V$ of a manifold to an integral over \bound{V} of another differential form.

To derive it, consider a $p + 1$-volume parametrized such that all $t_{i}$ range from 0 to 1 and such that for any fixed $t_{p + 1}$, the remaining $t_{i}$ parametrize a $p$-dimensional surface $V_{p}$ with a boundary independent of $t_{p + 1}$. This construction is somewhat restrictive, but only necessary in the derivation.

For some $p$-form $\omega$ and $p + 1$-volume $V$ we have
\begin{align*}
	\inte{V}{}\df{\omega} &= \inte{}{}\dd{t_{1}}{\dots \inte{}{}\dd{t_{p + 1}}{d\omega(\dot{\gamma}_{1},\dots, \dot{\gamma}_{p + 1})}} \\
	                      &= \inte{}{}\dd{t_{1}}{\dots \inte{}{}\dd{t_{p + 1}}{\left(\del{}{a_{1}}{\omega_{a_{2}\dots a_{p + 1}}}\sum\limits_{\sigma\in S_{p + 1}}\text{sgn}(\sigma)\bigotimes_{k = 1}^{p + 1}d\chi^{a_{\sigma(k)}}\right)(\dot{\gamma}_{1},\dots, \dot{\gamma}_{p + 1})}} \\
	                      &= \sum\limits_{\sigma\in S_{p + 1}}\frac{\text{sgn}(\sigma)}{p!}\inte{}{}\dd{t_{1}}{\dots \inte{}{}\dd{t_{p + 1}}{\left(\del{}{a_{1}}{\omega_{a_{2}\dots a_{p + 1}}}\bigotimes_{k = 1}^{p + 1}d\chi^{a_{\sigma(k)}}\right)(\dot{\gamma}_{1},\dots, \dot{\gamma}_{p + 1})}} \\
	                      &= \sum\limits_{\sigma\in S_{p + 1}}\frac{\text{sgn}(\sigma)}{p!}\inte{}{}\dd{t_{1}}{\dots \inte{}{}\dd{t_{p + 1}}{\left(\del{}{a_{p + 1}}{\omega_{a_{1}\dots a_{p}}}\bigotimes_{k = 1}^{p + 1}d\chi^{a_{\sigma(k)}}\right)(\dot{\gamma}_{1},\dots, \dot{\gamma}_{p + 1})}},
\end{align*}
where the latter follows from the cyclicity imposed by the summation. We have
\begin{align*}
d\chi^{a_{\sigma(k)}}{\dot{\gamma}_{k}} = \dv{\chi^{a}}{t_{k}}\del{}{a}{\chi^{a_{\sigma(k)}}} = \dv{\chi^{a_{\sigma(k)}}}{t_{k}},
\end{align*}
and thus
\begin{align*}
	\inte{V}{}\df{\omega} &= \sum\limits_{\sigma\in S_{p + 1}}\frac{\text{sgn}(\sigma)}{p!}\inte{}{}\dd{t_{1}}{\dots \inte{}{}\dd{t_{p + 1}}{\del{}{a_{p + 1}}{\omega_{a_{1}\dots a_{p}}}\prod\limits_{k = 1}^{p + 1}\del{}{t_{k}}{\chi^{a_{\sigma(k)}}}}} \\
	                      &= \sum\limits_{\sigma\in S_{p + 1}}\frac{\text{sgn}(\sigma)}{p!}\inte{}{}\dd{t_{1}}{\dots \inte{}{}\dd{t_{p + 1}}{\del{}{a_{p + 1}}{\omega_{a_{1}\dots a_{p}}}\prod\limits_{k = 1}^{p + 1}\del{}{t_{\sigma(k)}}{\chi^{a_{k}}}}},
\end{align*}
where we have once again utilized the cyclicity. Denote the integral inside the sum as $I(\sigma, \omega)$. We have
\begin{align*}
	I(\sigma, \omega) &= \inte{}{}\dd{t_{1}}{\dots \inte{}{}\dd{t_{p + 1}}{\del{}{a_{p + 1}}{\omega_{a_{1}\dots a_{p}}}\prod\limits_{k = 1}^{p + 1}\del{}{t_{\sigma(k)}}{\chi^{a_{k}}}}} \\
	                  &= \inte{}{}\dd{t_{1}}{\dots \inte{}{}\dd{t_{p + 1}}{\del{}{t_{\sigma(p + 1)}}{\omega_{a_{1}\dots a_{p}}}\prod\limits_{k = 1}^{p}\del{}{t_{\sigma(k)}}{\chi^{a_{k}}}}}.
\end{align*}
To proceed, we integrate by parts and obtain
\begin{align*}
	I(\sigma, \omega) &= \eval{\inte{}{}\dd{t_{1}}{\dots \inte{}{}\dd{t_{p + 1}}{\omega_{a_{1}\dots a_{p}}\prod\limits_{k = 1}^{p}\del{}{t_{\sigma(k)}}{\chi^{a_{k}}}}}}_{t_{\sigma(p + 1)} = t_{\sigma(p + 1)}^{-}}^{t_{\sigma(p + 1)} = t_{\sigma(p + 1)}^{+}} - \inte{}{}\dd{t_{1}}{\dots \inte{}{}\dd{t_{p + 1}}{\omega_{a_{1}\dots a_{p}}\del{}{t_{\sigma(p + 1)}}{\prod\limits_{k = 1}^{p}\del{}{t_{\sigma(k)}}{\chi^{a_{k}}}}}},
\end{align*}
where the former terms contains no integration over $t_{\sigma(p + 1)}$, and is instead evaluated at the maximal and minimal values of $t_{\sigma(p + 1)}$ given the values of the other parameters.

Let us proceed to simplify this. For starters, if $\sigma(p + 1) \neq p + 1$, the remaining integration in the first term is done over the boundary of $V_{p}$. Furthermore, there exists a $k$ such that $\sigma(k) = p + 1$, and as we are at the boundary of $V_{p}$ we must have
\begin{align*}
	\del{}{t_{\sigma(k)}}{\chi^{a_{k}}} = 0.
\end{align*}
Otherwise, the remaining integration domain is $V_{p}$. Next, the latter integral contains a sequence of terms proportional to
\begin{align*}
	\del{}{t_{\sigma(p + 1)}}{\del{}{t_{\sigma(k)}}{\chi^{a_{k}}}},
\end{align*}
which is symmetric with respect to exchanging $p + 1$ and $k$. These terms therefore cancel in the sum, and we are left with
\begin{align*}
	\inte{V}{}\df{\omega} &= \sum\limits_{\sigma\in S_{p}}\frac{\text{sgn}(\sigma)}{p!}\eval{\inte{}{}\dd{t_{1}}{\dots \inte{}{}\dd{t_{p}}{\omega_{a_{1}\dots a_{p}}\prod\limits_{k = 1}^{p}\del{}{t_{\sigma(k)}}{\chi^{a_{k}}}}}}_{t_{\sigma(p + 1)} = 0}^{t_{\sigma(p + 1)} = 1} \\
	                      &= \eval{\inte{}{}\dd{t_{1}}{\dots \inte{}{}\dd{t_{p}}{\omega}}}_{t_{\sigma(p + 1)} = 0}^{t_{\sigma(p + 1)} = 1}
\end{align*}
where the condition that $\sigma(p + 1) = p + 1$ restricts the summation to $S_{p}$.

The remaining integration domain is, as stated before, a parametrization of $V_{p}$, which at the extremal values of $t_{p + 1}$ must be at the boundary of $V$. The minus sign from the integral evaluation tells us that the integrals are taken with opposite orientation, meaning that together they indeed form an integration over the (closed) boundary of $V$. We thus arrive at Stokes' theorem,
\begin{align*}
	\inte{V}{}\df{\omega} = \oint\limits_{\bound{V}}\omega.
\end{align*}

\example{Reobtaining Familiar Theorems}
Many familiar integration theorems are in fact consequences of Stokes' theorem. Let us rederive them.

We start with a $1$-form $d\omega$, which will be integrated over a $1$-dimensional volume, i.e. a curve. We have
\begin{align*}
	\inte{\gamma}{}\df{\omega} = \omega(p) - \omega(q),
\end{align*}
where $q$ and $p$ are the start and end points of $\gamma$. This is the analogue of integrating a vector field along a curve.

Next, we consider a $2$-form written as the exterior derivative of a $1$-form. Writing $\omega = \omega_{i}d\chi^{i}$ we have
\begin{align*}
d\omega = \del{}{j}{\omega_{i}}d\chi^{j}\wedge d\chi^{i}.
\end{align*}
Using Cartesian coordinates and restricting ourselves to two dimensions we have
\begin{align*}
	\df{\chi^{j}}\wedge\df{\chi^{i}}(\dot{\gamma}_{s}, \dot{\gamma}_{t})\dd{s}\dd{t} &= (\df{\chi^{j}}\otimes\df{\chi^{i}} - \df{\chi^{i}}\otimes\df{\chi^{j}})(\dot{\gamma}_{s}, \dot{\gamma}_{t})\dd{s}\dd{t} \\
	                                                                                 &= (\del{}{s}{\chi^{j}}\del{}{t}{\chi^{i}} - \del{}{s}{\chi^{i}}\del{}{t}{\chi^{j}})\dd{s}\dd{t} \\
	                                                                                 &= (\del{}ta_{jk}\del{}ta_{im} - \del{}ta_{ik}\del{}ta_{jm})\del{}{s}{\chi^{k}}\del{}{t}{\chi^{m}}\dd{s}\dd{t} \\
	                                                                                 &= \varepsilon_{jin}\varepsilon_{nkm}\del{}{s}{\chi^{k}}\del{}{t}{\chi^{m}}\dd{s}\dd{t} \\
	                                                                                 &= \varepsilon_{jik}\dd{S_{k}}.
\end{align*}
Thus we have
\begin{align*}
	\inte{S}{}\df{\omega} = \inte{S}{}\dd{S_{k}}{\varepsilon_{ijk}\del{}{i}{\omega_{j}}} = \inte{S}{}\dd{\vb{S}}{\cdot\curl{\vb*{\omega}}}.
\end{align*}
At the same time we have
\begin{align*}
	\inte{\bound{S}}{}{\omega} &= \inte{\bound{S}}{}\dd{t}{\omega_{a}\df{\chi^{a}}(\dot{\gamma}_{t})} \\
	                           &= \inte{\bound{S}}{}\dd{t}{\omega_{a}\dv{\chi^{a}}{t}} \\
	                           &= \inte{\bound{S}}{}\dd{\vb{r}}{\cdot\vb*{\omega}},
\end{align*}
hence
\begin{align*}
	\inte{S}{}\dd{\vb{S}}{\cdot\curl{\vb*{\omega}}} = \inte{\bound{S}}{}\dd{\vb{r}}{\cdot\vb*{\omega}},
\end{align*}
which is the more boring version of Stokes' theorem.

Next we consider a $2$-form and its exterior derivative. We have
\begin{align*}
	\inte{V}{}{\df{\omega}} &= \frac{1}{2}\inte{V}{}{\del{}{a}{\omega_{bc}}\df{\chi^{a}}\wedge\df{\chi^{b}}\wedge\df{\chi^{c}}}.
\end{align*}
As the components of $\omega$ may be chosen such that $\omega_{ab} = -\omega_{ba}$ we may write $\omega_{ab} = \varepsilon_{abc}\omega_{c}$ for some suitable (and arbitrary) choice of $\omega_{c}$. We thus have
\begin{align*}
	\inte{V}{}{\df{\omega}} &= \frac{1}{2}\inte{V}{}{\varepsilon_{bcd}\del{}{a}{\omega_{d}}\df{\chi^{a}}\wedge\df{\chi^{b}}\wedge\df{\chi^{c}}} \\
	                        &= \frac{1}{2}\inte{}{}\dd{x^{1}}{\inte{}{}\dd{x^{2}}{\inte{}{}\dd{x^{3}}{\varepsilon_{bcd}\varepsilon_{abc}\del{}{a}{\omega_{d}}}}} \\
	                        &= \frac{1}{2}\inte{}{}\dd{x^{1}}{\inte{}{}\dd{x^{2}}{\inte{}{}\dd{x^{3}}{(\del{}ta_{ad}\del{}ta_{bb} - \del{}ta_{ab}\del{}ta_{bd})\del{}{a}{\omega_{d}}}}} \\
	                        &= \inte{}{}\dd{x^{1}}{\inte{}{}\dd{x^{2}}{\inte{}{}\dd{x^{3}}{\del{}{a}{\omega_{a}}}}} \\
	                        &= \inte{}{}{x^{1}}{\inte{}{}\dd{x^{2}}{\inte{}{}\dd{x^{3}}{\div{\vb*{\omega}}}}}.
\end{align*}
At the same time we have
\begin{align*}
	\inte{\bound{V}}{}{\omega} &= \frac{1}{2}\inte{\bound{V}}{}{\varepsilon_{ijk}\omega_{i}\df{\chi^{j}}\wedge\df{\chi^{k}}} \\
	                           &= \frac{1}{2}\inte{\bound{V}}{}\dd{S_{m}}{\varepsilon_{ijk}\varepsilon_{jkm}\omega_{i}} \\
	                           &= \frac{1}{2}\inte{\bound{V}}{}\dd{S_{m}}{(\del{}ta_{im}\del{}ta_{jj} - \del{}ta_{ij}\del{}ta_{jm})\omega_{i}} \\
	                           &= \inte{\bound{V}}{}\dd{S_{i}}{\omega_{i}} \\
	                           &= \inte{\bound{V}}{}\dd{\vb{S}}{\cdot\vb*{\omega}}.
\end{align*}
Hence we have
\begin{align*}
	\inte{}{}\dd{x^{1}}{\inte{}{}\dd{x^{2}}{\inte{}{}\dd{x^{3}}{\div{\vb*{\omega}}}}} = \inte{\bound{V}}{}\dd{\vb{S}}{\cdot\vb*{\omega}},
\end{align*}
which is the divergence theorem.

\example{The $n$-Dimensional Divergence Theorem}
Let us generalize the latter to an $n$-dimensional case. Consider a $n - 1$-form and its exterior derivative. We have
\begin{align*}
	\inte{V}{}{\df{\omega}} &= \frac{1}{(n - 1)!}\inte{V}{}{\del{}{a_{1}}{\omega_{a_{2}\dots a_{n}}}\bigwedge\limits_{k = 1}^{n}d\chi^{a_{k}}}.
\end{align*}
We may once again choose coordinates such that the components of $\omega$ are completely antisymmetric, yielding
\begin{align*}
	\inte{V}{\df{\omega}} &= \frac{1}{(n - 1)!}\inte{V}{}{\varepsilon_{a_{2}\dots a_{n}a_{n + 1}}\del{}{a_{1}}{\omega_{a_{n + 1}}}\bigwedge\limits_{k = 1}^{n}d\chi^{a_{k}}} \\
	                      &= \frac{1}{(n - 1)!}\inte[n]{}{}\dd{x}{\varepsilon_{a_{2}\dots a_{n}a_{n + 1}}\varepsilon_{a_{1}\dots a_{n}}\del{}{a_{1}}{\omega_{a_{n + 1}}}} \\
	                      &= \frac{1}{(n - 1)!}\inte[n]{}{}\dd{x}{\varepsilon_{a_{2}\dots a_{n + 1}}\varepsilon_{a_{2}\dots a_{n}a_{1}}\del{}{a_{1}}{\omega_{a_{n + 1}}}}.
\end{align*}
Summing over all but the first and last index will give a factor $(n - 1)!$, as this is the number of permutations of $n - 1$ unique indices. The remaining product is non-zero only when the last indices are equal, and is in this case equal to $1$, hence
\begin{align*}
	\inte{V}{}{\df{\omega}} &= \inte[n]{}{}\dd{x}{\del{}ta_{a_{n + 1}a_{1}}\del{}{a_{1}}{\omega_{a_{n + 1}}}} \\
	                        &= \inte[n]{}{}\dd{x}{\del{}{a}{\omega_{a}}}.
\end{align*}

At the same time we have
\begin{align*}
	\inte{\bound{V}}{}{\omega} &= \frac{1}{(n - 1)!}\inte{\bound{V}}{\varepsilon_{a_{1}\dots a_{n}}\omega_{a_{n}}\bigwedge\limits_{k = 1}^{n - 1}d\chi^{a_{k}}} \\
	                           &= \frac{1}{(n - 1)!}\inte{\bound{V}}{}\dd{S_{m}}{\varepsilon_{a_{1}\dots a_{n}}\varepsilon_{a_{1}\dots a_{n - 1}m}\omega_{a_{n}}} \\
	                           &= \inte{\bound{V}}{}\dd{S_{a}}{\omega_{a}}.
\end{align*}
Hence we have
\begin{align*}
	\inte[n]{}{}\dd{x}{\del{}{a}{\omega_{a}}} = \inte{\bound{V}}{}\dd{S_{a}}{\omega_{a}},
\end{align*}
which is a generalization of the divergence theorem. Note that this only applies if the coordinates of the manifold are akin to Cartesian coordinates.

\section{Classical mechanics}
In classical mechanics, configuration space is the space of all possible configurations of a system. We can impose coordinates $\chi^{a}$ on this space in order to use what we know.

\paragraph{Kinetic energy}
Kinetic energy is defined by a rank $2$ tensor as
\begin{align*}
	E_{\text{k}} = \frac{1}{2}T_{ab}\dot{\chi}^{a}\dot{\chi}^{b},
\end{align*}
where the dot now really represents the time derivative.

%Can be used to show H = T + V

\paragraph{Hamilton's principle}
We define the Lagrangian of a system as $\lag = E_{\text{k}} - V$, where $V$ is the potential energy and taken to be a function on coordinate space. The action of a system over time is defined as
\begin{align*}
	S = \integ{}{}{t}{\lag}.
\end{align*}
Hamilton's principle states that for the motion of the system in configuration space, $\var{S} = 0$. This can be expressed as
\begin{align*}
	\var{S} = \integ{}{}{t}{\var{\lag}} = \integ{}{}{t}{\left(\del{\chi^{a}}{\lag} - \dv{t}\del{\dot{\chi}^{a}}{\lag}\right)\var{\chi^{a}}} = 0.
\end{align*}

\paragraph{The kinetic metric}
Consider a system with no potential energy. The Lagrangian simply becomes $\lag = \frac{1}{2}T_{ab}\dot{\chi}^{a}\dot{\chi}^{b}$. This is very similar to the integral of curve length (or, rather its square, the extremum of which was noted to be the same), except $g_{ab}$ has been replaced by $T_{ab}$. This inspires us to define $T_{ab}$ as the kinetic metric, with corresponding Christoffel symbols.

\paragraph{Motion of a classical system}
By defining $a^{b} = \dot{\chi}^{a}\dcov{a}{\dot{\chi}^{b}}$, the previous work leads us to a system with no potential satisfying $a^{b} = \ddot{\chi}^{b} + \chris{b}{a}{c}\dot{\chi}^{a}\dot{\chi}^{c} = 0$. In other words, a system with no potential moves along the geodesics of the kinetic metric.

For a system with a potential, only the $\del{\chi^{a}}{\lag}$ term is affected, and
\begin{align*}
	a^{b} = - T^{ba}\del{a}{V} = T^{ba}F,
\end{align*}
which is a generalization of Newton's second law.

\paragraph{Noether's theorem}
Noether's theorem relates symmetries of physical systems to conservation laws.

What is a symmetry, then? Consider a one-parameter transformation $t\to\tau(t, s),\ q^{a}\to Q^{a}(q, s)$, where $s$ is the parameter with respect to which the system is transformed, such that $\tau(t, 0) = t,\ Q^{a}(q, s) = q^{a}$ and for small $s = \varepsilon$ that $t\to t + \varepsilon\var{t},\ q^{a}\to q^{a} + \varepsilon\var{q^{a}}$. This is assumed to be normalized such that $\var{t}$ is either $0$ or $1$. How? Don't ask. A quasi-symmetry of a system with Lagrangian $\lag$ is a transformation such that
\begin{align*}
	\varepsilon\var{\lag} = \lag(Q, \dot{Q}, \tau) - \lag(q, \dot{q}, t) = \varepsilon\dv{F}{t}
\end{align*}
for some $F$. The variation of the Lagrangian can be written as
\begin{align*}
	\var{\lag} = \del{q^{a}}{\lag}\var{q^{a}} + \del{\dot{q}^{a}}{\lag}\var{\dot{q}^{a}} + \del{t}{\lag}\var{t}.
\end{align*}
The total time derivative of the Lagrangian is given by
\begin{align*}
	\dv{\lag}{t} = \del{t}{\lag} + \del{q^{a}}{\lag}\dot{q}^{a} + \del{\dot{q}^{a}}{\lag}\ddot{q}^{a},
\end{align*}
which yields
\begin{align*}
	\var{\lag} = \del{q^{a}}{\lag}(\var{q^{a}} - \dot{q}^{a}\var{t}) + \del{\dot{q}^{a}}{\lag}(\var{\dot{q}^{a}} - \ddot{q}^{a}\var{t}) + \dv{\lag}{t}\var{t}.
\end{align*}
The equations of motion are $\del{q^{a}}{\lag} = \dv{t}\del{\dot{q}^{a}}{\lag}$. For a set of coordinates that satisfy this - a so-called on-shell solution - we have
\begin{align*}
	\var{\lag} &= \dv{t}\del{\dot{q}^{a}}{\lag}(\var{q^{a}} - \dot{q}^{a}\var{t}) + \del{\dot{q}^{a}}{\lag}(\var{\dot{q}^{a}} - \ddot{q}^{a}\var{t}) + \dv{\lag}{t}\var{t} \\
	           &= \dv{t}\left(\del{\dot{q}^{a}}{\lag}(\var{q^{a}} - \dot{q}^{a}\var{t}) + \lag\var{t}\right).
\end{align*}
If the transformation is a quasi-symmetry of the system, then this is equal to a total time derivative of $F$, and the quantity
\begin{align*}
	J = F - \del{\dot{q}^{a}}{\lag}\var{q^{a}} + (\dot{q}^{a}\del{\dot{q}^{a}}{\lag} - \lag)\var{t}
\end{align*}
thus satisfies $\dv{J}{t} = 0$. We can introduce the general momenta $p_{a} = \del{\dot{q}^{a}}{\lag}$ and the Hamiltonian $\ham = p_{a}\dot{q}^{a} - \lag$ to write
\begin{align*}
	J = F - p_{a}\var{q^{a}} + \ham\var{t}.
\end{align*}

We arrive at the conclusion that $J$ is a conserved quantity under a quasi-symmetry of the system. Identifying the conservation laws of a system is thus a matter of identifying the quasi-symmetries of a system and computing $J$ under that transformation.

\example{A free particle in space}
Consider a free particle in space. Its Lagrangian is given by $\lag = \frac{1}{2}m\dot{\vb{x}}^{2}$, and the variation of this is
\begin{align*}
	\var{\lag} = m\dot{\vb{x}}\cdot\var{\dot{\vb{x}}}.
\end{align*}
Its general momentum is
\begin{align*}
	\vb{p} = \del{\dot{\vb{x}}}{\lag} = m\dot{\vb{x}}.
\end{align*}
The Hamiltonian is
\begin{align*}
	\ham = \vb{p}\cdot\dot{\vb{x}} - \lag = \frac{1}{2}m\dot{\vb{x}}^{2}.
\end{align*}
We now want to identify quasi-symmetries of the system that make the variation of the Lagrangian either zero or the time derivative of some quantity. A key idea here is that we are only allowed to change the variations (or so I think).

A first attempt is keeping $\var{\vb{x}}$ constant and not varying thiime(a spatial translation), which implies $\var{\dot{\vb{x}}} = \vb{0}$ and $\var{\lag} = 0$. This implies that $F$ is constant. The conserved quantity is thus
\begin{align*}
	J = F - \vb{p}\cdot\var{\vb{x}} = F - \vb{p}\cdot\vb{c},
\end{align*}
i.e. the momentum of the system is conserved. We also note that the constant $F$ in this case is arbitrary, and we might as well have set it to $0$. This will be the case at least sometimes.

A second attempt is varying time, i.e. $\var{t} = 1$, but keeping the coordinates fixed, i.e. $\var{\vb{x}} = 0$ (a time translation). This yields $\var{\dot{\vb{x}}} = \vb{0}$ and $\var{\lag} = 0$. Once again $F$ is constant and taken to be zero, and the conserved quantity is thus $J = H$, i.e. the Hamiltonian of the system is conserved.

A third attempt is to somehow make the scalar product in the variation of the Lagrangian zero, without varying time. An option is $\var{\vb{x}} = \vb{\omega}\times\vb{x}$, where $\vb{\omega}$ is a constant vector. This yields $\var{\dot{\vb{x}}} = \vb{\omega}\times\dot{\vb{x}}$ and $\var{\lag} = 0$. The conserved quantity is thus
\begin{align*}
	J &= -\vb{p}\cdot(\vb{\omega}\times\vb{x}) \\
	  &= -\vb{\omega}\cdot(\vb{x}\times\vb{p}).
\end{align*}
Since $\vb{\omega}$ is constant, that means that $\vb{x}\times\vb{p}$, i.e. the angular momentum, is conserved.

\paragraph{Hamiltonian mechanics}
Hamiltonian mechanics starts with trying to transform $\lag(q, \dot{q}, t)$ to $\ham(q, p, t)$. To start with this, we reintroduce the general momenta $p_{i} = \del{\dot{q}_{i}}{\lag}$ and define the Hamiltonian
\begin{align*}
	\ham = p_{i}\dot{q}_{i} - \lag.
\end{align*}
This is a Legendre transform of the Hamiltonian, which is discussed below. In Lagrangian mechanics, we considered paths in configuration space. In Hamiltonian mechanics, we instead consider paths in phase space, i.e. a space where the points are $(q, t)$. In this space, paths do not intersect as the system is deterministic. The new equations of motion in this formalism is
\begin{align*}
	\dot{p}_{i} = -\del{q_{i}}{\ham},\ \dot{q}_{i} = \del{p_{i}}{\ham}.
\end{align*}

Paths in phase space are periodic for integrable systems and fill out the accessible parts of phase space for chaotic systems.

\paragraph{Legendre transforms}
To illustrate the Legendre transform, consider a function $f(x, y)$ and $g(x, y, u) = ux - f(x, y)$. Its total derivative is given by
\begin{align*}
	\dd{g} = u\dd{x} + x\dd{u} - \del{x}{f}\dd{x} - \del{y}{f}\dd{y}.
\end{align*}
By choosing $u = \del{x}{f}$, we obtain
\begin{align*}
	\dd{g} = x\dd{u} - \del{y}{f}\dd{y},
\end{align*}
implying that $g$ is only a function of $u$ and $y$. To obtain $g$, invert the definition of $u$ to obtain $x(u, y)$.

\paragraph{Equations of motion}
The variation of the Hamiltonian is given by
\begin{align*}
	\dd{\ham} &= \dot{q}_{i}\dd{p_{i}} + p_{i}\dd{\dot{q}_{i}} - \del{q_{i}}{\lag}\dd{q_{i}} - \del{\dot{q}_{i}}{\lag}\dd{\dot{q}_{i}} - \del{t}{\lag}\dd{t} \\
	          &= \dot{q}_{i}\dd{p_{i}} - \del{q_{i}}{\lag}\dd{q_{i}} - \del{t}{\lag}\dd{t}.
\end{align*}
Combining this with the equations of motion
\begin{align*}
	\del{q_{i}}{\lag} = \dv{t}\left(\del{\dot{q}_{i}}{\lag}\right) = \dv{t}\left(p_{i}\right) = \dot{p}_{i}
\end{align*}
yields
\begin{align*}
	\dot{q}_{i} = \del{p_{i}}{\ham},\ \dot{p}_{i} = -\del{q_{i}}{\ham},\ \del{t}{\lag} = -\del{t}{\ham}.
\end{align*}

We also have
\begin{align*}
	\dv{\ham}{t} &= \del{q_{i}}{\ham}\dot{q}_{i} + \del{p_{i}}{\ham}\dot{p}_{i} + del{t}{\ham} \\
	             &= \del{t}{\ham},
\end{align*}
and so the Hamiltonian is conserved if it has no explicit time dependance.

\paragraph{Liouville's theorem}
As paths in phase space do not cross, we can think of the time evolution of a system as a flow in phase space. The volume element is $\dd{V} = \dd{q}\dd{p}$. Liouville's theorem states that flow in phase space is incompressible.

To show this, consider the state at some point in time and after some infinitesimal time $\dd{t}$. Denote the point in phase space at the start as $(q, p)$ and after $\dd{t}$ as $(q', p').$ To first order in time we have
\begin{align*}
	q_{i}' = q_{i} + \dot{q}_{i}\dd{t} = q_{i} + \del{p_{i}}{\ham}\dd{t},\ p_{i}' = p_{i} + \dot{p}_{i}\dd{t} = p_{i} - \del{q_{i}}{\ham}\dd{t}.
\end{align*}
The volume element is given by
\begin{align*}
	\dd{V}' &= \left(\dd{q} +  \left(\del{q}{\del{p}{\ham}}\dd{q} + \del[2]{p}{\ham}\dd{p}\right)\dd{t}\right)\left(\dd{p} -  \left(\del[2]{q}{\ham}\dd{q} + \del{p}{\del{q}{\ham}}\dd{p}\right)\dd{t}\right) \\
	        &= \dd{q}\dd{p} + \left(-\dd{q}\left(\del[2]{q}{\ham}\dd{q} + \del{p}{\del{q}{\ham}}\dd{p}\right) + \dd{p\left(\del{q}{\del{p}{\ham}}\dd{q} + \del[2]{p}{\ham}\dd{p}\right)}\right)\dd{t} \\
	        &= \dd{q}\dd{p} + \left(-\del[2]{q}{\ham}(\dd{q})^{2} + (\del{q}{\del{p}{\ham}} - \del{p}{\del{q}{\ham}})\dd{q}\dd{p} + \del[2]{p}{\ham}(\dd{p})^{2}\right)\dd{t}.
\end{align*}
The equations of motion imply that the terms containing two consecutive derivatives with respect to the same variable are equal to zero. Assuming the Hamiltonian to be sufficiently smooth, the cross-derivatives are equal. This implies
\begin{align*}
	\dd{V}' = \dd{V}.
\end{align*}

\paragraph{Poisson brackets}
Consider a function $f(q, p, t)$. Its time derivative is given by
\begin{align*}
	\dv{f}{t} &= \del{q_{i}}{f}\dot{q}_{i} + \del{p_{i}}{f}\dot{p}_{i} + \del{t}{f} \\
	          &= \del{q_{i}}{f}\del{p_{i}}{\ham} - \del{p_{i}}{f}\del{q_{i}}{\ham} + \del{t}{f} \\
	          &= \pob{f}{\ham} + \del{t}{f},
\end{align*}
where we now have defined the Poisson bracket. It is bilinear and satisfies
\begin{align*}
	\pob{f}{g}          &= -\pob{g}{f}, \\
	\pob{fg}{h}         &= f\pob{g}{h} + \pob{f}{h}g, \\
	\pob{f}{\pob{g}{h}} &+ \pob{g}{\pob{h}{f}} + \pob{h}{\pob{f}{g}} = 0.
\end{align*}
The expression above implies that if $\pob{f}{\ham} = 0$ and $f$ does not depend explicitly on time, then it is a constant of motion.

\paragraph{Restatement of Liouville's theorem}
We define $\rho(q, p, t)$ as the probability that a particle is close to $(q, p)$. Alternatively, for a large number of particles, we can define it as the number of particles close to $(q, p)$.

We have
\begin{align*}
	\dv{\rho}{t} = 0,
\end{align*}
implying
\begin{align*}
	\del{t}{\rho} = -\pob{\rho}{\ham}.
\end{align*}
This is an equivalent statement of Liouville's theorem.

\section{Relativitet och elektromagnetism}

\paragraph{Lorentztransformen}
Betrakta två inertialsystem $S$ och $S'$ med sammanfallande koordinataxlar, där $S'$ rör sig med hastigheten $v\ub{x}$ relativt $S$. Lorentztransformen mellan koordinaterna i de två systemen är
\begin{align*}
	t' = \gamma\left(t - \frac{\beta}{c}x\right),\ x' = \gamma(x - vt),
\end{align*}
med
\begin{align*}
	\beta = \frac{v}{c},\ \gamma = \frac{1}{\sqrt{1 - \beta^{2}}}.
\end{align*}
Övriga koordinater påverkas inte. Inversion av transformen fås genom att byta tecken på $\beta$ och byta plats på primmade och oprimmade koordinater. På matrisform skriver vi
\begin{align*}
	(x')^{\mu} = \Lambda_{\nu}^{\mu}x^{\nu},
\end{align*}
där $\nu$ och $\mu$ löper från $0$ till $3$ och $x^{0} = ct$. Konventionen i speciell relativitet är att grekiska index löper över alla dimensioner medan latinska index bara löper över rumsindex.

Mer allmänt kan vi skriva transformationsmatrisen som
\begin{align*}
	\lambda_{\nu}^{\mu} = \del{\nu}{(x')^{\mu}},
\end{align*}
med invers vars komponenter ges av $\delp{\nu}{x^{\mu}}$. Kontravarianta vektorer transformeras med vanliga transformationsmatrisen och kovarianta vektorer med den inversa transformationsmatrisen. Alla vektorer kan uttryckas i kontravarianta och kovarianta komponenter, och dessa uppfyller
\begin{align*}
	A_{\mu} = g_{\mu\nu}A^{\nu},\ A^{\mu} = g^{\mu\nu}A_{\nu}
\end{align*}
där $g$ är den metriska tensorn. Metriken vi använder ges av $g_{ij} = \delta_{ij},\ g_{0\mu} = -\delta_{0\mu}$.

\paragraph{Derivator}
Kedjeregeln ger att derivationsoperatorn $\del{0}{} = \frac{1}{c}\del{t}{},\ \del{i}{} = \del{i}{}$ transformeras kovariant. Dens kontravarianta motsvarighet tas enkelt fram med metriken.

\paragraph{Källtensorn}
Betrakta en laddningsfördelning som rör sig med konstant hastighet relativt ett inertialsystem $S$. I dens vilosystem $S'$ är $\vb{J}' = \vb{0}$. I $S$ fås
\begin{align*}
	\rho = \gamma \rho',\ \vb{J} = \rho v\ub{x} = \gamma\rho' v\ub{x}.
\end{align*}
Mer allmänt kan man bilda $4$-vektorn
\begin{align*}
	J^{0} = c\rho, J^{i} = J_{i}
\end{align*}
och visa att denna transformeras enligt Lorentztransformationen.

\paragraph{Potentialtensorn}
Vi har visat att potentialen från en punktladdning i icke-accelererad rörelse ges av
\begin{align*}
	V(x^{\mu}) = \gamma\frac{q}{4\pi\varepsilon_{0}}\frac{1}{\sqrt{\gamma(x^{1} - \beta x^{0})^{2} + (x^{2})^{2} + (x^{3})^{2}}},\ \vb{A}(x^{\mu}) = \frac{V}{c}\vb*{\beta}.
\end{align*}
I laddningens inertialsystem fås
\begin{align*}
	V((x')^{\mu}) = \frac{q}{4\pi\varepsilon_{0}}\frac{1}{\sqrt{(x')^{i}(x')^{i}}},\ \vb{A}'((x')^{\mu}) = \vb{0}.
\end{align*}
Då definierar vi $4$-vektorn
\begin{align*}
	(A')^{0} = \frac{V}{c},\ (A')^{i} = A_{i}.
\end{align*}
Denna transformeras enligt Lorentztransformationen.

\paragraph{Lorentzskalärer}
En Lorentsskalär är en skalär som är invariant under Lorentztransformationen. Den mest allmänna Lorentzskalären är skalärprodukten mellan $4$-vektorer
\begin{align*}
	\vb{A}\cdot\vb{B} = A_{\mu}B^{\mu}.
\end{align*}
Detta ger direkt att $\del{\mu}{J^{\mu}}$ är en Lorentzskalar, vilket motsvarar kontinuitetsekvationen. I Lorenzgaugen är $\del{\mu}{A^{\mu}}$ en Lorentzskalär. d'Alembertoperatorn $\del{\mu}{\delc{\mu}{}}$ är även invariant under Lorentztransformen.

\paragraph{Faradaytensorn}
Elektriska och magnetiska fältet kan användas för att konstruera den antisymmetriska tensorn
\begin{align*}
	F^{\mu\nu} = 
	\mqty
	[
		0                & \frac{E_{x}}{c} & \frac{E_{y}}{c} & \frac{E_{z}}{c} \\
		-\frac{E_{x}}{c} & 0               & B_{z}           & -B_{y} \\
		-\frac{E_{y}}{c} & -B_{z}          & 0               & -B_{x}  \\
		-\frac{E_{z}}{c} & B_{y}           & B_{x}           & 0
	].
\end{align*}
Genom att Lorentztransformera denna fås
\begin{align*}
	E_{x}' = E_{x},\ E_{y}' = \gamma(E_{y} - \beta cB_{z}),\ E_{z}' = \gamma(E_{z} + \beta cB_{y}), \\
	B_{x}' = B_{x},\ B_{y}' = \gamma\left(B_{y} + \frac{\beta}{c}B_{z}\right),\ B_{z}' = \left(B_{z} - \frac{\beta}{c}E_{y}\right).
\end{align*}

\paragraph{Duala Faradaytensorn}
Duala Faradaytensorn fås genom att göra bytet $E_{i}\leftrightarrow -cB_{i}$ i Faradaytensorn. På matrisform blir den
\begin{align*}
	G^{\mu\nu} = 
	\mqty
	[
		0     & -B_{x}           & -B_{y}           & -B_{z} \\
		B_{x} & 0                & \frac{E_{z}}{c}  & -\frac{E_{y}}{c} \\
		B_{y} & -\frac{E_{z}}{c} & 0                & \frac{E_{x}}{c}  \\
		B_{z} & \frac{E_{y}}{c}  & -\frac{E_{x}}{c} & 0
	].
\end{align*}

\paragraph{Maxwells ekvationer}
Två av Maxwells ekvationer kan nu skrivas
\begin{align*}
	\del{\nu}{F^{\mu\nu}} = \mu_{0}J^{\mu}.
\end{align*}
Om man gör det samma med duala fälttensorn fås
\begin{align*}
	\del{\nu}{G^{\mu\nu}} = 0,
\end{align*}
vilket ger de två andra.

\section{Classical Field theory}

Classical field theory can be considered a limit of classical dynamics when the number of particles is infinite. The system obtains new ``coordinates'' $\phi^{a}$, which are functions of position and time. Summations over coordinates now become integrals over space.

\paragraph{Lagrangian Formulation of Field Theory}
The Lagrangian in a field theory now becomes
\begin{align*}
	L = \integ[d]{}{}{\vb{r}}{\lag}
\end{align*}
where $\lag$ is the Lagrangian density, which in general depends on all involved fields, their derivatives with respect to both space and time and space and time themselves. From this we can obtain the action, and extremize it to obtain the equations for the time evolution of the system. The equations of motion are of the form
\begin{align*}
	\pdv{\lag}{\phi^{a}} - \pdv{t}\pdv{\lag}{(\del{t}{\phi^{a}})} - \pdv{x^{i}}\pdv{\lag}{(\del{i}{\phi^{a}})} = 0.
\end{align*}
Alternatively, by defining $x^{0} = ct$ for some speed $c$ and extending the summation, we can write
\begin{align*}
	\pdv{\lag}{\phi^{a}} - \pdv{x^{\mu}}\pdv{\lag}{(\del{\mu}{\phi^{a}})} = 0.
\end{align*}

\example{A String}

\example{The Electromagnetic Field}

\example{The Schrödinger Equation}

\paragraph{Hamiltonian Formulation}
In the Hamiltonian formalism, we define the momentum density
\begin{align*}
	\pi_{a} = \del{\phi^{a}}{\lag}.
\end{align*}
The Hamiltonian is now given by
\begin{align*}
	H = \integ[D]{}{}{\vb{r}}{\ham},
\end{align*}
where $\ham = \pi_{a}\del{t}{\phi^{a}} - \lag$. The Hamiltonian equations of motion become
\begin{align*}
	\dot{\phi} = \fdv{H}{\pi},\ \dot{\pi} = -\fdv{H}{\phi}.
\end{align*}
While the Hamiltonian formalism carries no issues with it in classical contexts, it does not generalize well to relativity due to the fact that it treats the time derivative differently to the spatial derivatives, which is a big no-no.

\example{A String}

\paragraph{Reduction to Discrete Problems}
For a problem on a compact domain, one can Fourier expand the fields (and the momentum densities) to obtain a discrete set of Fourier coefficients, the dynamics of which can be studied. It is this approach which will be the basis for quantum mechanics, where the coefficients will be replaced by occupation operators.

For problems on non-compact domains, we instead employ the Fourier transform as a tool. However, we have not really helped ourselves in this case.

\paragraph{Symmetries of Field Theories}
Consider a field theory (on Euclidean space) described by the Lagrangian density \lag. A symmetry of the system is a transformation of all involved coordinates and fields such that:
\begin{enumerate}
	\item \lag retains its functional form under the transformation - in other words, the expression for the Lagrangian density is unchanged.
	\item The action is unchanged by the transformation.
\end{enumerate}

Before proceeding, it would also be useful to clarify what kinds of transformation we are considering. Transformations in field theory concern both transformations of the coordinates according to
\begin{align*}
	(x^{\prime})^{\mu} = x^{\mu} + \var{x^{\mu}}
\end{align*}
and of the fields according to
\begin{align*}
	(\phi^{\prime})^{a}((x^{\prime})^{\mu}) = \phi^{a}(x^{\mu}) + \var{\phi^{a}}.
\end{align*}
We will distinguish between the transformed fields and the change in the field at a particular point, given by
\begin{align*}
	(\phi^{\prime})^{a}(x^{\mu}) = \phi^{a}(x^{\mu}) + \bvar{\phi^{a}}
\end{align*}

\paragraph{Nöether's Theorem}
Field theory also carries with it a version of Nöether's theorem, which will be covered here. A version will be presented here which is somewhat more restricted than the version presented for systems with discrete degrees of freedom - if you wanted to compare the two, you could say that this version only contains symmetries.

Consider the action of a symmetry on a given system. The requirement that the action be unchanged can be written as
\begin{align*}
	\integ{\Omega^{\prime}}{}{(x^{\prime})^{\mu}}{\lag^{\prime}} - \integ{\Omega}{}{x^{\mu}}{\lag} = 0.
\end{align*}
The functional form of the Lagrangian density is unchanged, which also carries the consequence that the integration variables may be renamed. This yields
\begin{align*}
	\integ{\Omega^{\prime}}{}{x^{\mu}}{\lag((\phi^{\prime})^{a}, x^{\mu})} - \integ{\Omega}{}{x^{\mu}}{\lag(\phi^{a}, x^{\mu})} = 0.
\end{align*}

%TODO: Extend to multiple dimensions
To illustrate what happens to this, consider the analogous scenario in one dimension, for which we have
\begin{align*}
	\integ{a + \var{a}}{b + \var{b}}{x}{f + \var{f}} - \integ{a}{b}{x}{f} = 0.
\end{align*}
This can be rewritten as
\begin{align*}
	\integ{a + \var{a}}{b + \var{b}}{x}{\var{f}} + \integ{b}{b + \var{b}}{x}{f} - \integ{a}{a + \var{a}}{x}{f} = 0.
\end{align*}
The terms containing an integral of a variation over an interval containing a variation will be of order greater than one, and may thus be ignored. In addition, the two latter integrals may be linearized to obtain
\begin{align*}
	\integ{a}{b}{x}{\var{f}} + f(b)\var{b} - f(a)\var{a} = \integ{a}{b}{x}{\var{f} + \dv{x}(f\var{x})} = 0.
\end{align*}
Returning to the case which we wanted to study, we obtain
\begin{align*}
	\integ{\Omega}{}{x^{\mu}}{\lag((\phi^{\prime})^{a}, x^{\mu}) - \lag(\phi^{a}, x^{\mu})} + \integ{S}{}{S_{\mu}}{\lag(\phi^{a}, x^{\mu})\var{x^{\mu}}} = \integ{\Omega}{}{x^{\mu}}{\lag((\phi^{\prime})^{a}, x^{\mu}) - \lag(\phi^{a}, x^{\mu}) + \pdv{x^{\nu}}(\lag(\phi^{a}, x^{\mu})\var{x^{\nu}})} = 0,
\end{align*}
where $S$ is the boundary of $\Omega$ and we have made (hopefully proper) use of the $n$-dimensional divergence theorem. The difference in the first two terms can be expanded to first order as
\begin{align*}
	\lag((\phi^{\prime})^{a}, x^{\mu}) - \lag(\phi^{a}, x^{\mu}) &= \pdv{\lag}{\phi^{a}}\bvar{\phi}^{a} + \pdv{\lag}{(\del{\nu}{\phi^{a}})}\bvar{(\del{\nu}{\phi^{a}})}.
\end{align*}
The use of the variation at a specific point is due to the fact that both Lagrangians are now evaluated at the same points. This is significant because while the total variation does not commute with the differentiation operators, this one does. Using the equations of motion, we additionally obtain
\begin{align*}
	\pdv{\lag}{\phi^{a}}\bvar{\phi}^{a} + \pdv{\lag}{(\del{\nu}{\phi^{a}})}\del{\nu}{\bvar{\phi^{a}}} = \bvar{\phi}^{a}\pdv{x^{\nu}}\pdv{\lag}{(\del{\nu}{\phi^{a}})} + \pdv{\lag}{(\del{\nu}{\phi^{a}})}\del{\nu}{\bvar{\phi^{a}}} = \pdv{x^{\nu}}\left(\bvar{\phi}^{a}\pdv{\lag}{(\del{\nu}{\phi^{a}})}\right).
\end{align*}
Hence we have
\begin{align*}
	\integ{\Omega}{}{x^{\mu}}{\pdv{x^{\nu}}\left(\bvar{\phi}^{a}\pdv{\lag}{(\del{\nu}{\phi^{a}})} + \lag\var{x^{\nu}}\right)} = 0,
\end{align*}
which is already in the form of a conservation law for the quantities
\begin{align*}
	\bvar{\phi}^{a}\pdv{\lag}{(\del{0}{\phi^{a}})} + \lag\var{x^{0}}
\end{align*}
and the corresponding currents
\begin{align*}
	\bvar{\phi}^{a}\pdv{\lag}{(\del{i}{\phi^{a}})} + \lag\var{x^{i}}.
\end{align*}

To obtain a neater form, we specify the form of the transformation as
\begin{align*}
	\var{x}^{\mu} = \alpha^{r}X_{r}^{\mu},\ \var{\phi}^{a} = \alpha^{r}\Phi_{r}^{a}
\end{align*}
where the $\alpha^{r}$ are a set of transformation parameters. To first order, we have
\begin{align*}
	\var{\phi}^{a}  &= \bvar{\phi}^{a} + \del{\mu}{\phi^{a}}\var{x}^{\mu}, \\
	\bvar{\phi}^{a} &= \alpha^{r}(\Phi_{r}^{a} - \del{\mu}{\phi^{a}}X_{r}^{\mu}).
\end{align*}
For this kind of transformation, one thus obtains
\begin{align*}
	\integ{\Omega}{}{x^{\mu}}{\pdv{x^{\nu}}\left(\alpha^{r}(\Phi_{r}^{a} - \del{\sigma}{\phi^{a}}X_{r}^{\sigma})\pdv{\lag}{(\del{\nu}{\phi^{a}})} + \alpha^{r}X_{r}^{\nu}\lag\right)} &= \integ{\Omega}{}{x^{\mu}}{\alpha^{r}\pdv{x^{\nu}}\left(\left(\lag\kdelta{\sigma}{\nu} - \del{\sigma}{\phi^{a}}\pdv{\lag}{(\del{\nu}{\phi^{a}})}\right)X_{r}^{\sigma} + \Phi_{r}^{a}\pdv{\lag}{(\del{\nu}{\phi^{a}})}\right)}.
\end{align*}
Setting this equal to zero and changing the sign of the integrand yields
\begin{align*}
	\integ{\Omega}{}{x^{\mu}}{\alpha^{r}\pdv{x^{\nu}}\left(\left(\del{\sigma}{\phi^{a}}\pdv{\lag}{(\del{\nu}{\phi^{a}})} - \lag\kdelta{\sigma}{\nu}\right)X_{r}^{\sigma} - \Phi_{r}^{a}\pdv{\lag}{(\del{\nu}{\phi^{a}})}\right)} = 0,
\end{align*}
which is the final form of Nöether's theorem. It states that
\begin{align*}
	\pdv{x^{\nu}}\left(\left(\del{\sigma}{\phi^{a}}\pdv{\lag}{(\del{\nu}{\phi^{a}})} - \lag\kdelta{\sigma}{\nu}\right)X_{r}^{\sigma} - \Phi_{r}^{a}\pdv{\lag}{(\del{\nu}{\phi^{a}})}\right) = 0,
\end{align*}
which is a conservation law for the quantities corresponding to $\nu = 0$.

A more general proof can be performed for a transformation such that
\begin{align*}
	\var{\lag} = \dv{\alpha}\dv{V^{\mu}}{x^{\mu}},
\end{align*}
i.e. a divergence, but I will leave this case for the future me.

\example{Charge Conservation}

\paragraph{Poisson Brackets}
Poisson brackets of two functionals on phase space are defined as
\begin{align*}
	\pob{F}{G} = \integ[D]{}{}{\vb{r}}{\del{\phi}{F}\del{\pi}{G} - \del{\pi}{F}\del{\phi}{G}}
\end{align*}
We can somehow show that
\begin{align*}
	\pob{\phi(x)}{\phi(y)} = \pob{\pi(x)}{\pi(y)} = 0,\ \pob{\phi(x)}{\pi(y)} = \delta(x - y).
\end{align*}

\end{document}
