\section{Relativity}

\paragraph{The Galilean group}
The Galilean group is the group of transformations between frames of reference under which the laws of physics are invariant. It consists of:
\begin{itemize}
	\item Translations by a constant vector.
	\item Rotations of the coordinate axes.
	\item Boosts, i.e. translating the coordinates along a line with a constants speed.
\end{itemize}
It is based on a concept of absolute time. It turns out that the arc element $\dd[2]{s} = \dd{x}^{2} + \dd{y}^{2} + \dd{z}^{2}$ at a given time is preserved under all of these transformations.

The invariance of the laws of physics under these transformations corresponds to there being no special position or direction in the universe, and no special velocity. At least two of these claims have thus far not been disproved.

\paragraph{The emergence of special relativity}
It turned out that Maxwell's equations were not invariant under Galilean transformations. 

DISCLAIMER: THIS IS SERIOUS HEAD CANON! The issue with Maxwell's equations is that they predict that electromagnetic waves travel at speed $c$. This should of course be the same in all frames of reference, according to Galilean relativity. No other wave phenomena had previously raised an issue as they travel through a medium. This medium naturally defines a certain frame of reference in which the physics are special, namely the rest frame of the medium. Only by transporting the medium with you when doing the boost will you reobtain the same physics. A natural idea to follow from this is that electromagnetic waves travel in a medium, so physicists started searching for it. After having found no evidence of its existence, most notably through Michelson and Morley's experiment, the conclusion was that there was no medium in which electromagnetic waves travelled, and thus the speed of light had to be one of the invariant properties under transformation between inertial frames of reference.

The constancy of the speed of light implies that the infinitesimal quantity
\begin{align*}
	\dd[2]{s} = c^{2}\dd{t}^{2} - \dd{x}^{2} - \dd{y}^{2} - \dd{z}^{2}
\end{align*}
is constant. We will soon replace the elements of the Galilean group with elements that keep this quantity.

\paragraph{Four-vectors and the Minkowski metric}
We now define the four-vector $x^{\mu}$, where $\mu = 0, 1, 2, 3$ and $\mu = 0$ corresponds to $ct$ and an inner product with the metric $\eta$. The metric is diagonal with $\eta_{11} = 1$ and $\eta_{ii} = -1$ otherwise. This is called the Minkowski metric.

\paragraph{Lorentz transformations}
We are now interested in transformations that preserve the new arc length. If the transformation is on the form $\vb{x}' = \Lambda\vb{x}$. Computing the arc length yields
\begin{align*}
	\Lambda^{T}\nu\Lambda = \nu.
\end{align*}
The transformations satisfying this constitute the Lorentz group, or $\O{1}{3}$. Computing the determinant on either side yields $\det(\Lambda)^{2} = 1$. The subgroup with determinant $1$ (which preserve the direction of time) is the special Lorentz group \SO{1}{3}.

The equation defining the group elements is symmetric, which imposes constraints on the elements of the matrix. The matrix has $16$ elements, so the defining equation places $10$ constraints on the coefficients of $\Lambda$. With $10$ equations and $16$ unknowns, we expect $6$ linearly independent solutions.

The first three are rotations of space, written in block diagonal form as
\begin{align*}
	\Lambda =
	\mqty[
		1 & 0 \\
		0 & R
	].
\end{align*}
Inserted into the defining equation, we obtain
\begin{align*}
	\mqty[
		1 & 0 \\
		0 & -R^{T}R
	] =
	\mqty[
		1 & 0 \\
		0 & -1
	].
\end{align*}
This yields the familiar requirement $R^{T}R = 1$.

The remaining three transforms are Lorentz boosts corresponding to each axis. This can be shown explicitly for $x$, and a permutation of coordinates will yield the same result for a boost along any other axis. We believe it to be reasonable that such a transformation should not affect any other coordinates than the boosted coordinate and time. This means that the matrix will be on the form
\begin{align*}
	\Lambda =
	\mqty[
		\Lambda_{x} & 0 \\
		0           & 1
	].
\end{align*}
The defining equation now yields
\begin{align*}
	\Lambda_{x}^{T}\sigma_{z}\Lambda_{x} = \sigma_{z}, \
	\sigma_{z} =
	\mqty[
		1 & 0 \\
		0 & -1
	].
\end{align*}
We expand $\Lambda_{x}$ around the identity as $\Lambda_{x} = 1 - \phi K$, where $\phi$ is independent of both coordinates and time. Inserting this into the above equation yields
\begin{align*}
	(1 - \phi K^{T})\sigma_{z}(1 - \phi K) = \sigma_{z}.
\end{align*}
Expanding the bracket yields
\begin{align*}
	(1 - \phi K^{T})(\sigma_{z} - \phi\sigma_{z}K)                                 &= \sigma_{z}, \\
	\sigma_{z} - \phi\sigma_{z}K - \phi K^{T}\sigma_{z} + \phi^{2}K^{T}\sigma_{z}K &= \sigma_{z}.
\end{align*}
Ignoring higher-order terms yields
\begin{align*}
	\sigma_{z}K + K^{T}\sigma_{z} &= 0, \\
	(\sigma_{z}K)^{T}             &= -\sigma_{z}K.
\end{align*}
The generator $K$ must therefore be
\begin{align*}
	K = \sigma_{x} =
	\mqty[
		0 & 1 \\
		1 & 0
	].
\end{align*}
Now a transformation correpsonding to an arbitrary $\phi$ can be written as
\begin{align*}
	\Lambda_{x} = e^{-\phi K} = \cosh{\phi} - \sinh{\phi}\sigma_{x},
\end{align*}
where the last equality comes from writing the exponential as an infinite series and using the fact that $K^{2} = 1$. Perhaps someone should do this explicitly.

To identify the transformation more exactly, we consider two frames of reference in which the origins coincide at $t = 0$. Under such a transformation, we require that $(ct, vt)$ map to $(ct', 0)$. This yields
\begin{align*}
	-\sinh{\phi}ct + \cosh{\phi}vt = 0.
\end{align*}
Defining $\gamma = \cosh{\phi}$ and applying hyperbolic identities yields
\begin{align*}
	-\sqrt{\gamma^{2} - 1}ct + \gamma vt &= ct\left(\frac{v}{c}\gamma - \sqrt{\gamma^{2} - 1}\right) = 0, \\
	\gamma                               &= \sqrt{\frac{1}{1 - \frac{v^{2}}{c^{2}}}}.
\end{align*}
The transformation can now be written as
\begin{align*}
	\Lambda_{x} = 
	\mqty[
		\gamma             & -\frac{v}{c}\gamma \\
		-\frac{v}{c}\gamma & \gamma
	].
\end{align*}
The total matrix for a boost along any other coordinate axis can be found by permuting the elements in the transformation matrix for the $x$ boost. This yields a basis of matrices, and a boost along an arbitrary direction can be found by taking linear combinations of these.

\paragraph{Adding velocities}
The product of two boosts is another boost. For two boosts along the same direction, we obtain $\Lambda(\phi_{1})\Lambda(\phi_{2}) = \Lambda(\phi_{1} + \phi_{2})$. This can be used to show that the total boosted velocity is
\begin{align*}
	v_{3} = \frac{v_{1} + v_{2}}{1 + \frac{v_{1}v_{2}}{c^{2}}}.
\end{align*}

\paragraph{Proper time}
Consider a particle at the origin in its rest frame. The arc length becomes $\dd{s}^{2} = c^{2}\dd{t}^{2}$. As the left-hand side is invariant, so must the right-hand side be. This makes it natural to define the proper time
\begin{align*}
	\dd{\tau} = \frac{1}{c}\dd{s}.
\end{align*}

\paragraph{Relativistic kinematics}
Suppose that you wanted to define $\vb{u} = \dv{\vb{x}}{t}$ as the spatial part of velocity. Well, too bad, cause time transforms under a Lorentz transformation, so this thing will not behave linearly under Lorentz transformation. We need a better alternative.

Consider instead the rest frame $S'$ of the particle, where it is resting at the origin. Its trajectory in the original inertial frame can be parametrized in terms of the proper time. Along a small trajectory we have
\begin{align*}
	\dd{\tau} = \frac{1}{c}\dd{s} = \frac{1}{c}\sqrt{c^{2}\dd{t}^{2} - \dd{\vb{x}}^{2}} = \dd{t}\sqrt{1 - \left(\dv{\vb{x}}{t}\right)^{2}} \implies \dv{t}{\tau} = \gamma.
\end{align*}
We can now define the four velocity
\begin{align*}
	U = \dv{x}{\tau} = \dv{\tau}
	\mqty[
		ct \\
		\vb{x}
	]
	=
	\mqty[
		c\dv{t}{\tau} \\
		\dv{\vb{x}}{\tau}
	]
	= \gamma
	\mqty[
		c \\
		\vb{u}
	]
\end{align*}
where $\vb{u} = \dv{\vb{x}}{t}$. This quantity transforms like a four-vector, and is therefore the four-velocity.