\section{Group Theory}

\paragraph{Definition of a Group}
A group is a set of objects $G$ with an operation $G\times G\to G,\ (a, b)\to ab$ such that
\begin{itemize}
	\item If $a, b\in G$ then $ab\in G$.
	\item $a(bc) = (ab)c$ for all $a, b, c\in G$.
	\item There exists an identity $e$ such that $ae = ea = a$ for all $a\in G$.
	\item There exists for every element $a$ an inverse $a^{-1}\in G$ such that $aa^{-1} = a^{-1}a = e$.
\end{itemize}
Groups can be
\begin{itemize}
	\item cyclic, i.e. all elements in the group are powers of a single element.
	\item finitie, i.e. groups containing a finite number of elements, or infinte.
	\item discrete, i.e. all elements in the group can be labelled with some index, por continuous.
	\item commutative, i.e. $ab = ba$ for all elements in the group, or non-commutative.
\end{itemize}

\paragraph{Subgroups}
If $G = \{g_{\alpha}\}$ and the subset $H = \{h_{\alpha}\}$ is also a group, we call $H$ a subgroup of $G$ and write $H < G$.

\paragraph{Conjugacy Classes}
Two group elements $a$ and $b$ are conjugate if there exists an element $g$ such that
\begin{align*}
a = gbg^{-1}.
\end{align*}
We write $a\sim b$.

\paragraph{Equivalence Relations}
An equivalence relation is a relation (here denoted $=$) between two things such that
\begin{itemize}
	\item $a = b \equiv b = a$.
	\item $a = b,\ b = c\implies a = c$.
\end{itemize}

\example{Conjugacy as an Equivalence Relation}

\paragraph{Homomorphisms and Isomorphisms}
A homomorphisms is a map $f: G\to H$ such that $f(g_{1})f(g_{2}) = f(g_{1}g_{2})$. If the map is bijective, $f$ is called an isomorphism.

\paragraph{Direct Products}
GIven two groups $F$ and $G$, we define $F\times G$ as the set of ordered pairs of elements of the two groups. The group action of $F\times G$ is the group actions of $F$ and $G$ separately on the elements in the ordered pair.

\paragraph{Generators}
For discrete groups, the generators of a group is the smallest set of elements in the group such that all other elements in the group can be composed by the elements in the set. For continuous groups, we will use the term generators to refer to elements such that any group element can be written as real powers of this element.

\paragraph{Point Groups}
Point groups are symmetries of, for instance, a crystal structure that leave at least one point in the structure invariant. Examples include
\begin{itemize}
	\item rotations.
	\item reflections.
	\item spatial inversions.
\end{itemize}
Combining these with certain discrete translation, you obtain the space groups of the crystal. Space groups are the groups of all symmetries of a crystal.

\paragraph{Dihedral Groups}
The dihedral group $D_{n}$ is the group of transformations that leave an $n$-sided polygon invariant.

\paragraph{Lie Groups}
A Lie group is a group containing a manifold with the group operation and inverse operation being smooth maps. My current understanding of the significance of this is that it allows us to differentiate and expand the group elements with respect to certain parameters.

The group elements are denoted $g(\vb*{\theta})$, where $g(\vb{0}) = 1$. We write them as
\begin{align*}
	g(\vb*{\theta}) = e^{\theta_{a}T_{a}}
\end{align*}
where we have introduced the generators $T_{a}$. This is reasonable, partially because it is reasonable for a smooth operation to have an addition operation (I think), and in this case the series definition of the exponential function yields exactly that any element is generated by powers of the generators. Note however that the generators themselves are typically not group elements, which might leave a hole in this reasoning. In addition, the question of whether the group is Abelian leaves it questionable whether the addition of generators makes sense, but I will return to this concern. There is probably a deeper understanding to this, and what I am saying may even be completely incorrect. Depending on the context, the exponent may also contain a factor $-i$, but this discussion will omit it.

\paragraph{The Lie Algebra}
Expanding an element around the identity yields
\begin{align*}
	g(\vb*{\theta}) \approx 1 + \theta_{a}T_{a}.
\end{align*}
First of all, we note that performing this for exponentials of a single generator yields
\begin{align*}
	e^{\theta_{i}T_{i}}e^{\theta_{j}T_{j}} \approx (1 + \theta_{i}T_{i})(1 + \theta_{j}T_{j}),
\end{align*}
which is equal to $e^{\theta_{i}T_{i} + \theta_{j}T_{j}}$ to first order even for a non-commutative group. Hence the addition of generators is reasonable. Furthermore, this implies that set of generators is a vector space, termed the Lie algebra.

\paragraph{The Lie Bracket}
Having seen that the exponential notation makes sense, we study it for a non-commutative group. The element $e^{-\theta_{i}T_{i}}e^{-\theta_{j}T_{j}}e^{\theta_{i}T_{i}}e^{\theta_{j}T_{j}}$ is equal to the identity for a commutative group, and we would like to study this in the general case. We have
\begin{align*}
	e^{\theta_{i}T_{i}}                    &\approx 1 + \theta_{i}T_{i} + \frac{1}{2}\theta_{i}^{2}T_{i}^{2}, \\
	e^{\theta_{i}T_{i}}e^{\theta_{j}T_{j}} &\approx \left(1 + \theta_{i}T_{i} + \frac{1}{2}\theta_{i}^{2}T_{i}^{2}\right)\left(1 + \theta_{j}T_{j} + \frac{1}{2}\theta_{j}^{2}T_{j}^{2}\right) \approx 1 + \theta_{i}T_{i} + \theta_{j}T_{j} + \frac{1}{2}\left(\theta_{i}^{2}T_{i}^{2} + \theta_{j}^{2}T_{j}^{2}\right) + \theta_{i}\theta_{j}T_{i}T_{j}.
\end{align*}
We thus obtain
\begin{align*}
	e^{-\theta_{i}T_{i}}e^{-\theta_{j}T_{j}}e^{\theta_{i}T_{i}}e^{\theta_{j}T_{j}} \approx& 1 + \theta_{i}T_{i} + \theta_{j}T_{j} + \frac{1}{2}\left(\theta_{i}^{2}T_{i}^{2} + \theta_{j}^{2}T_{j}^{2}\right) + \theta_{i}\theta_{j}T_{i}T_{j} - \theta_{i}T_{i}\left(1 + \theta_{i}T_{i} + \theta_{j}T_{j}\right) \\
	 &- \theta_{j}T_{j}\left(1 + \theta_{i}T_{i} + \theta_{j}T_{j}\right) + \frac{1}{2}\left(\theta_{i}^{2}T_{i}^{2} + \theta_{j}^{2}T_{j}^{2}\right) + \theta_{i}\theta_{j}T_{i}T_{j} \\
	=& 1 + \theta_{i}\theta_{j}T_{i}T_{j} - \theta_{i}\theta_{j}T_{i}T_{j} - \theta_{i}\theta_{j}T_{j}T_{i} + \theta_{i}\theta_{j}T_{i}T_{j} \\
	=& 1 + \theta_{i}\theta_{j}\comm{T_{a}}{T_{b}}
\end{align*}
to second order, where we have introduced the Lie bracket
\begin{align*}
	\comm{T_{i}}{T_{j}} = T_{i}T_{j} - T_{j}T_{i}.
\end{align*}
Hence, the non-commutativity of Lie groups close to the identity is described by the Lie brackets, which is why we study them. Furthermore, we have
\begin{align*}
	e^{-\theta_{i}T_{i}}e^{-\theta_{j}T_{j}}e^{\theta_{i}T_{i}}e^{\theta_{j}T_{j}} \approx e^{\theta_{i}\theta_{j}\comm{T_{a}}{T_{b}}},
\end{align*}
implying that the Lie bracket belongs to the vector space spanned by the generators and allowing us to write
\begin{align*}
	\comm{T_{a}}{T_{b}} = f_{a, b, c}T_{c}.
\end{align*}
The constants $f_{a, b, c}$ are called structure constants.

\paragraph{Representations}
A representation is a homomorphism $D: G\to GL(V)$, where $GL(V)$ is the group of all invertible linear transformations on $V$. The group elements thus act on $V$ according to
\begin{align*}
D(g_{1})D(g_{2})v = D(g_{1}g_{2})v,\ v\in V.
\end{align*}

\paragraph{Reducible and irreducible representations}
Two representations are equivalent if they satisfy $S^{-1}DS = D'$, where $S$ is a matrix representing a change of basis. Some representations can be written as direct sums in certain bases. For these, there is a basis where the representation is block diagonal. These are reducible. Those that cannot are irreducible.

\example{Rotations in Two Dimensions}
Consider a rotation of an infinitesimal displacement $\dd{\vb{x}}$ with a rotation $R$. The requirement for length to be preserved implies $R^{T}R = 1$.

Consider now a rotation by a small angle $\var{\theta}$. Taylor expanding it in terms of the angle yields
\begin{align*}
R(\var{\theta}) \approx 1 + A\var{\theta}.
\end{align*}
The requirement for $R$ to be orthogonal yields $A^{T} = -A$. We choose the solution
\begin{align*}
J =
\mqty[
0  & 1 \\
-1 & 0
].
\end{align*}
We can now write the rotation matrix as
\begin{align*}
R(\var{\theta}) =
\mqty[
1             & \var{\theta} \\
-\var{\theta} & 1
].
\end{align*}

We would now like to construct a large rotation in terms of smaller rotations as
\begin{align*}
R(\theta) = \lim\limits_{N\to\infty}\left(1 + \frac{\theta}{N}J\right)^{N} = e^{\theta J}.
\end{align*}
We can write this as an infinite series and use the fact that $J^{2} = -1$ to obtain
\begin{align*}
R(\theta) = \cos{\theta} + J\sin{\theta}.
\end{align*}

\example{Rotations in Three Dimensions}
The argument done for two dimensions does not use the dimensionality, so we conclude that even for higher dimensions, $R^{T}R = 1$. Expanding a small rotation around the identity yields that the first-order term must include an antisymmetric matrix. The space of antisymmetric $3\times 3$ matrices is three-dimensional. We thus choose the basis
\begin{align*}
J_{x} =
\mqty[
0 & 0  & 0 \\
0 & 0  & 1 \\
0 & -1 & 0
],
J_{y} =
\mqty[
0 & 0 & -1 \\
0 & 0 & 0  \\
1 & 0 & 0
],
J_{z} =
\mqty[
0  & 1 & 0 \\
-1 & 0 & 0 \\
0  & 0 & 0
].
\end{align*}
Exponentiating yields
\begin{align*}
R(\theta) = e^{\sum \theta_{i}J_{i}} = e^{\vb{\theta}\cdot\vb{J}}.
\end{align*}
In physics we usually extract a factor $i$ such that the basis matrices are Hermitian, and the rotation becomes
\begin{align*}
R(\theta) = e^{i\vb{\theta}\cdot\vb{J}}.
\end{align*}

The set of generators of these rotations constitutes the Lie algebra.

We know in general that rotations in three dimensions do not commute. In fact, we obtain in general that
\begin{align*}
R(\vb{\theta})R(\vb{\theta}')R^{-1}(\vb{\theta}) = \theta_{a}\theta_{b}'\comm{J_{a}}{J_{b}},
\end{align*}
where $\comm{J_{a}}{J_{b}}$ is the commutator. This commutator satisfies
\begin{align*}
\comm{J_{a}}{J_{b}}^{T} = \comm{J_{b}^{T}}{J_{a}^{T}} = \comm{-J_{b}}{-J_{a}} = -\comm{J_{a}}{J_{b}},
\end{align*}
which implies
\begin{align*}
\comm{J_{a}}{J_{b}} = f_{a,b,c}J_{c}.
\end{align*}
It can be shown that
\begin{align*}
\comm{J_{i}}{J_{j}} = \varepsilon_{i,j,k}J_{k},
\end{align*}
or in a physics context (where a factor $i$ is extracted):
\begin{align*}
\comm{J_{i}}{J_{j}} = i\varepsilon_{i,j,k}J_{k}.
\end{align*}