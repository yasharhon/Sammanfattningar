\section{Statistical Physics}

\paragraph{The Partition Function}
The partition function is defined as
\begin{align*}
	Z = \sum e^{-\beta E_{i}},
\end{align*}
where the $E_{i}$ are possible values of the energy (or, rather, the Hamiltonian) and we now formally introduce $\beta = \frac{1}{\kb T}$.

\paragraph{Properties of the Partition Function}

\paragraph{Internal Energy}
In a statistical context, we define the internal energy as $U = \expval{E}$. We have
\begin{align*}
	U &= \frac{\sum E_{i}e^{-\beta E_{i}}}{\sum e^{-\beta E_{i}}} \\
	  &= \frac{-\dv{Z}{\beta}}{Z} \\
	  &= -\pdv{\ln{Z}}{\beta}.
\end{align*}
Alternatively, in terms of temperature,
\begin{align*}
	U = -\pdv{\ln{Z}}{T}\pdv{T}{\beta} = \frac{1}{\kb\beta^{2}}\pdv{\ln{Z}}{T} = \kb T^{2}\pdv{\ln{Z}}{T}.
\end{align*}

\paragraph{Entropy}
We have
\begin{align*}
	\ln{P_{i}} = \ln(\frac{e^{-\beta E_{i}}}{Z}) = -\beta E_{i} - \ln{Z}.
\end{align*}
The definition of entropy from probability thus implies
\begin{align*}
	S = -\kb\sum P_{i}\ln{P_{i}} = \kb\sum P_{i}(\beta E_{i} + \ln{Z}) = \kb\ln{Z} + \kb\beta U.
\end{align*}

\paragraph{Helmholtz Free Energy}
We have
\begin{align*}
	F = U - TS = U - \frac{1}{\kb\beta}S = U - U - \frac{1}{\beta}\ln{Z} = -\frac{1}{\beta}\ln{Z},
\end{align*}
or
\begin{align*}
	Z = e^{-\beta F}.
\end{align*}

\paragraph{Heat Capacity}
The heat capacity is given by
\begin{align*}
	C = \pdv{U}{T} = 2\kb T\pdv{\ln{Z}}{T} + \kb T^{2}\pdv[2]{\ln{Z}}{T}.
\end{align*}

\paragraph{Pressure}
The pressure is given by
\begin{align*}
	p = -\pdv{F}{V} = \frac{1}{\beta}\pdv{\ln{Z}}{V}.
\end{align*}

\paragraph{Enthalpy}
The enthalpy is given by
\begin{align*}
	H = \kb T^{2}\pdv{\ln{Z}}{T} + \frac{V}{\beta}\pdv{\ln{Z}}{V} = \kb T\left(T\pdv{\ln{Z}}{T} + V\pdv{\ln{Z}}{V}\right).
\end{align*}

\paragraph{Gibbs Free Energy}
The Gibbs free energy is given by
\begin{align*}
	G = \kb T\left(-\ln{Z} + V\pdv{\ln{Z}}{V}\right).
\end{align*}

\paragraph{The Equipartition Theorem}
Suppose that some degree of freedom $x$, which may take any value, of a system only appears in the Hamiltonian as a quadratic term $\alpha x^{2}$. Its contribution to the internal energy is
\begin{align*}
	\expval{E} = \frac{\integ{-\infty}{\infty}{x}{\alpha x^{2}e^{-\beta\alpha x^{2}}}}{\integ{-\infty}{\infty}{x}{e^{-\beta\alpha x^{2}}}}.
\end{align*}
One can show that this is equal to
\begin{align*}
	\expval{E} = \frac{1}{2}\kb T.
\end{align*}
Examples of such degrees of freedom are
\begin{itemize*}
	\item translations, rotations and vibrations in a gas.
	\item vibrations in a crystal.
\end{itemize*}