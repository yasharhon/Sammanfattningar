\section{Quantum Statistical Mechanics}

\paragraph{Distinguishability}
The concepts developed in this chapter will be based on the concepts of distinguishability. Classical particles are distinguishable, but quantum particles are not. As is postulated ad hoc in non-relativistic quantum mechanics, the operation of changing the positions of two quantum particles in a quantum state has a certain symmetry depending on the type of considered particles. Boson states are symmetric under exchange, and fermions are antisymmetric.

\paragraph{The Pauli Exclusion Principle}
Consider a many-particle state of fermions where two particles occupy the same state. Exchanging these two particles do not modify the total state. Combining this with the requirement that the state be antisymmetric under particle exchange implies that such a state cannot exist for fermions. This gives rise to the Pauli exclusion principle, which states that two indistinguishable fermions cannot occupy the same quantum state.

\paragraph{Quantum Grand Partition Functions}
Consider a system with a set of quantum states with energies $E_{i}$ occupied by $n_{i}$ particles each. Such a system would correspond to a set of non-interacting particles. If this system is set in thermal contact with a heat bath, this implies that the occupancy of each state will vary due to the heat exchange, which is why such systems belong to the grand canonical ensemble. The grand canonical partition function for such a system is of the form
\begin{align*}
	\Z = \sum\limits_{N = 0}^{\infty}\sum\limits_{\text{States}}e^{-\beta(E - \mu N)}.
\end{align*}
A first question might be what happened to the factor $N!$ signifying that the particles are indistinguishable - after all, we said that distinguishability would be important. The answer is that the occupancy numbers themselves are not quantum numbers directly defining any state, but rather numbers defining the composition of the states. Due to the (anti)symmetry of the total state, it must be that any set of occupancy numbers can only correspond to a single quantum state, and thus there is no overcounting.

The sum over states is a sum over all configurations such that the sum of all occupancy numbers is equal to $N$. Combining this with the summation over $N$ yields that this sum may be replaced by a sum over all conceivable occupancy configurations. The total energy and the number of particles may be expressed as sums over particle numbers, allowing us to write
\begin{align*}
	\Z &= \sum\limits_{n_{1}}\dots\sum\limits_{n_{\infty}}e^{n_{i}\beta(\mu - E_{i})} \\
	   &= \prod\limits_{i}\left(\sum\limits_{n_{i}}e^{n_{i}\beta(\mu - E_{i})}\right),
\end{align*}
where the summation is performed over all possible values of the quantum numbers.

At this point we must specify what kinds of particles are discussed in order to proceed. For bosons $n_{i}$ may be any positive integer, yielding
\begin{align*}
	\Z = \prod\limits_{i}\left(\frac{1}{1 - e^{\beta(\mu - E_{i})}}\right),
\end{align*}
whereas for fermions $n_{i}$ may only be $0$ or $1$, yielding
\begin{align*}
	\Z = \prod\limits_{i}\left(1 + e^{\beta(\mu - E_{i})}\right).
\end{align*}
We can also write
\begin{align*}
	\ln{\Z} = \pm\sum\limits_{i}\ln(1 \pm e^{\beta(\mu - E_{i})}),
\end{align*}
where the $+$ is for fermions and the $-$ is for bosons.

\paragraph{The Bose-Einstein and Fermi-Dirac Distributions}
When considering the statistics of quantum gases, we will base our work on distribution functions describing the mean occupancy of a given energy level. This choice is highly analogous to, say, converting from the Maxwell-Boltzmann velocity distribution to a corresponding speed distribution for ideal gases. We have
\begin{align*}
	\expval{n_{i}} = -\frac{1}{\beta}\pdv{\ln{\Z}}{E_{i}} = \frac{e^{\beta(\mu - E_{i})}}{1 \pm e^{\beta(\mu - E_{i})}} = \frac{1}{e^{\beta(E_{i} - \mu)} \pm 1}.
\end{align*}
As there is a one-to-one correspondence between a given occupancy number and the corresponding energy, we can now write the distribution functions as
\begin{align*}
	f(E) = \frac{1}{e^{\beta(E_{i} - \mu)} - 1}
\end{align*}
for bosons and
\begin{align*}
	f(E) = \frac{1}{e^{\beta(E_{i} - \mu)} + 1}
\end{align*}
for fermions. These functions are called the Bose-Einstein and Fermi-Dirac distribution functions.

\paragraph{The Non-Interacting Quantum Gas}
Consider a fluid of non-interacting quantum particles. The state of the system is characterized by a set of possible wave vectors, and each wave vector state has a degeneracy of $2s + 1$, where $s$ is the spin of each particle. The partition function can thus be written as
\begin{align*}
	\Z = \prod\limits_{\vb{k}}\Z_{\vb{k}}^{2s + 1}
\end{align*}
where
\begin{align*}
	\Z_{\vb{k}} = (1 \pm e^{-\beta(E_{\vb{k}} - \mu)})^{\pm 1}.
\end{align*}

%TODO: Compute grand potential and thermodynamic properties

We continue our study by considering non-relativistic particles in a periodic cube. For such particles the components of the wave vector are of the form
\begin{align*}
	k_{i} = \frac{2\pi}{L}n,\ n = 0, \pm 1, \pm 2, \dots
\end{align*}
where $L$ is the dimension of the cube. In addition, there is a spin degeneracy, hence termed $\eta(s)$, such that the density of states in k-space is
\begin{align*}
	\rho(\vb{k}) = \eta(s)\frac{1}{\left(\frac{2\pi}{L}\right)^{3}} = \eta(s)\frac{V}{8\pi^{3}}.
\end{align*}
In terms of the length of the wave vector we have
\begin{align*}
	\rho(k) = \eta(s)\frac{V}{8\pi^{3}}\cdot 4\pi k^{2} = \eta(s)\frac{V}{2\pi^{2}}k^{2}.
\end{align*}
To express this in terms of the energy, we use the dispersion relation
\begin{align*}
	E = \frac{\hbar^{2}k^{2}}{2m},
\end{align*}
which yields that the density of states is
\begin{align*}
	\rho(E) = \frac{1}{\frac{\hbar^{2}k(E)}{m}}\cdot \eta(s)\frac{V}{2\pi^{2}}k^{2}(E) = \frac{\eta(s)}{2\pi^{2}}\frac{mV}{\hbar^{2}}\frac{\sqrt{2mE}}{\hbar} = \frac{\eta(s)}{(2\pi)^{2}}V\left(\frac{2m}{\hbar^{2}}\right)^{\frac{3}{2}}\sqrt{E}.
\end{align*}
The number of states with energy less than $E$ is thus
\begin{align*}
	\sigma(E) &= \integ{0}{E}{\varepsilon}{\rho(\varepsilon)} \\
	          &= \frac{\eta(s)}{(2\pi)^{2}}V\left(\frac{2m}{\hbar^{2}}\right)^{\frac{3}{2}}\integ{0}{E}{\varepsilon}{\sqrt{\varepsilon}} \\
	          &= \frac{2}{3}\frac{\eta(s)}{(2\pi)^{2}}V\left(\frac{2mE}{\hbar^{2}}\right)^{\frac{3}{2}}.
\end{align*}

The partition function of a quantum gas is thus given by
\begin{align*}
	\ln{\Z} &= \pm\sum\limits_{\vb{k}}\ln(1 \pm e^{-\beta(E_{\vb{k}} - \mu)}) \\
	        &= \pm\integ{0}{\infty}{E}{\rho(E)\ln(1 \pm e^{-\beta(E - \mu)}}) \\
	        &= \mp\integ{0}{\infty}{E}{\sigma(E)\frac{\mp\beta e^{-\beta(E - \mu)}}{1 \pm e^{-\beta(E - \mu)}}} \\
	        &= \beta\integ{0}{\infty}{E}{\frac{\sigma(E)}{e^{\beta(E - \mu)} \pm 1}},
\end{align*}
where the top sign is for fermions and the bottom for bosons.

The grand potential of such a quantum gas is given by
\begin{align*}
	\GP = -\kb T\ln{\Z} = -\integ{0}{\infty}{E}{\frac{\sigma(E)}{e^{\beta(E - \mu)} \pm 1}},
\end{align*}
or more explicitly
\begin{align*}
	\GP = &=  -\frac{2}{3}\frac{\eta(s)}{(2\pi)^{2}}V\left(\frac{2m}{\hbar^{2}}\right)^{\frac{3}{2}}\integ{0}{\infty}{E}{\frac{E^{\frac{3}{2}}}{e^{\beta(E - \mu)} \pm 1}}.
\end{align*}

The total number of occupied states may be computed in two ways. The first is
\begin{align*}
	N &= \sum\limits_{\vb{k}}n_{\vb{k}} \\
	  &= \sum\limits_{\vb{k}}\kb T\pdv{\ln{Z_{\vb{k}}}}{\mu} \\
	  &= \kb T\integ{0}{\infty}{E}{\rho(E)\pdv{\ln{Z_{\vb{k}}}}{\mu}}.
\end{align*}
The second is
\begin{align*}
	N &= \pdv{\ln{\Z}}{\beta\mu}.
\end{align*}
We now use the fact that the integrand in $\ln{Z}$ is of the form $\sigma(E)f(\beta(E - \mu))$. For such a function we have
\begin{align*}
	\del{\beta\mu}{} = -\del{\beta E}{} = -\kb T\del{E}{},
\end{align*}
yielding
\begin{align*}
	N &= \pdv{\beta\mu}\left(\beta\integ{0}{\infty}{E}{\frac{\sigma(E)}{e^{\beta(E - \mu)} \pm 1}}\right) \\
	  &= \beta\integ{0}{\infty}{E}{\sigma(E)\pdv{\beta\mu}\left(\frac{1}{e^{\beta(E - \mu)} \pm 1}\right)} \\
	  &= -\integ{0}{\infty}{E}{\sigma(E)\pdv{E}\left(\frac{1}{e^{\beta(E - \mu)} \pm 1}\right)} \\
	  &= \integ{0}{\infty}{E}{\frac{\rho(E)}{e^{\beta(E - \mu)} \pm 1}}.
\end{align*}

The internal energy is given by
\begin{align*}
	U &= \mu N - \pdv{\ln{Z}}{\beta} \\
	  &= \mu\integ{0}{\infty}{E}{\frac{\rho(E)}{e^{\beta(E - \mu)} \pm 1}} - \pdv{\beta}\left(\beta\integ{0}{\infty}{E}{\frac{\sigma(E)}{e^{\beta(E - \mu)} \pm 1}}\right) \\
	  &= \integ{0}{\infty}{E}{\frac{\mu\rho(E)}{e^{\beta(E - \mu)} \pm 1}} - \integ{0}{\infty}{E}{\frac{\sigma(E)}{e^{\beta(E - \mu)} \pm 1}} - \beta\integ{0}{\infty}{E}{\sigma(E)\cdot -\frac{(E - \mu)e^{\beta(E - \mu)}}{(e^{\beta(E - \mu)} \pm 1)^{2}}} \\
	  &= \integ{0}{\infty}{E}{\frac{\mu\rho(E)}{e^{\beta(E - \mu)} \pm 1}} - \integ{0}{\infty}{E}{\frac{\sigma(E)}{e^{\beta(E - \mu)} \pm 1}} + \beta\integ{0}{\infty}{E}{\sigma(E)(E - \mu)\frac{e^{\beta(E - \mu)}}{(e^{\beta(E - \mu)} \pm 1)^{2}}} \\
	  &= \integ{0}{\infty}{E}{\frac{\mu\rho(E)}{e^{\beta(E - \mu)} \pm 1}} - \integ{0}{\infty}{E}{\frac{\sigma(E)}{e^{\beta(E - \mu)} \pm 1}} + \integ{0}{\infty}{E}{(\rho(E)(E - \mu) + \sigma(E))\frac{1}{e^{\beta(E - \mu)} \pm 1}} \\
	  &= \integ{0}{\infty}{E}{\frac{E\rho(E)}{e^{\beta(E - \mu)} \pm 1}},
\end{align*}
or alternatively as
\begin{align*}
	U = \sum\limits_{\vb{k}}n_{\vb{k}}E_{\vb{k}}.
\end{align*}

It can be shown that all these quantities are proportional to
\begin{align*}
	\mp(\kb T)^{n}\Gamma(n)\Li(\mp z)
\end{align*}
where $n$ is the power of $E$ that is integrated.

\paragraph{The Fermi Energy}
Consider a gas of fermions, a Fermi gas, at $T = 0$. At this temperature, the fermions will fill up states from the lowest energy, $2s + 1$ at a time, until they hit a certain maximal energy, defined to be the Fermi energy. Based on this a more explicit definition of the Fermi energy is
\begin{align*}
	E_{\text{F}} = \eval{\mu}_{T = 0}.
\end{align*}
%TODO: Why does this make sense?

At $T = 0$, $\beta$ is very large. This means that the occupation number for a quantum gas is
\begin{align*}
	n_{\vb{k}} = \frac{1}{e^{\beta(E_{\vb{k}} - \mu)} + 1} \to \theta(\mu - E_{\vb{k}}) = \theta(E_{\text{F}} - E_{\vb{k}})
\end{align*}
where $\theta$ is the Heaviside function. We thus obtain
\begin{align*}
	N = \integ[3]{}{}{\vb{k}}{g(\vb{k})}
\end{align*}
where we integrate over a sphere of radius $k_{\text{F}} = k(E_{\text{F}})$. This wave vector is called the Fermi wave vector.

\paragraph{The Fermi Temperature}
The Fermi temperature is defined as
\begin{align*}
	T_{\text{F}} = \frac{E_{\text{F}}}{\kb}.
\end{align*}

\paragraph{The Fermi Surface}
The Fermi surface is the surface in k-space made up of points corresponding to states with energy equal to that of the chemical potential.

\paragraph{The Sommerfeld Formula}
Consider a integral of the form
\begin{align*}
	I = \integ{0}{\infty}{E}{\phi(E)f(E)},\ f(E) = \frac{1}{e^{\beta(E - \mu)} + 1}.
\end{align*}
Introduce the antiderivative $\psi$ of $\phi$ such that $\psi(0) = 0$. We thus have
\begin{align*}
	I = -\integ{0}{\infty}{E}{\psi(E)\dv{f}{E}}.
\end{align*}
Introducing $x = \beta(E - \mu)$, we can somehow expand $\psi$ as
\begin{align*}
	\psi = \sum\limits_{s = 0}^{\infty}\eval{\dv[s]{\psi}{x}}_{x = 0}\frac{x^{s}}{s!},
\end{align*}
compute
\begin{align*}
	\dv{f}{E} = -\beta\frac{e^{x}}{(e^{x} + 1)^{2}}
\end{align*}
and write
\begin{align*}
	I &= \sum\limits_{s = 0}^{\infty}\frac{\beta}{s!}\eval{\dv[s]{\psi}{x}}_{x = 0}\integ{0}{\infty}{E}{x^{s}\frac{e^{x}}{(e^{x} + 1)^{2}}} \\
	  &= \sum\limits_{s = 0}^{\infty}\frac{1}{s!}\eval{\dv[s]{\psi}{x}}_{x = 0}\integ{-\beta\mu}{\infty}{x}{\frac{x^{s}e^{x}}{(e^{x} + 1)^{2}}}.
\end{align*}
If the temperature is much lower than the Fermi temperature, we can extend the integral to $-\infty$, where the odd terms disappear and the even terms take the form
\begin{align*}
	\integ{-\infty}{\infty}{x}{\frac{x^{s}e^{x}}{(e^{x} + 1)^{2}}} &= 2\integ{0}{\infty}{x}{\frac{x^{s}e^{x}}{(e^{x} + 1)^{2}}} \\
	                                                               &= 2\integ{0}{\infty}{x}{\frac{x^{s}e^{-x}}{(e^{-x} + 1)^{2}}} \\
	                                                               &= 2\integ{0}{\infty}{x}{x^{s}e^{-x}\sum\limits_{m = 0}^{\infty}(-1)^{m}(m + 1)e^{-mx}} \\
	                                                               &= 2\sum\limits_{m = 0}^{\infty}(-1)^{m}(m + 1)\integ{0}{\infty}{x}{x^{s}e^{-(m + 1)x}} \\
	                                                               &= 2\sum\limits_{m = 1}^{\infty}(-1)^{m + 1}m\integ{0}{\infty}{x}{x^{s}e^{-mx}} \\
	                                                               &= 2(s!)\sum\limits_{m = 1}^{\infty}\frac{(-1)^{m + 1}}{m^{s}} \\
	                                                               &= 2(s!)(1 - 2^{1 - s})\zeta(s).
\end{align*}
We thus finally obtain the Sommerfeld formula
\begin{align*}
	I &\approx \sum\limits_{s = 0}^{\infty}\frac{1}{s!}\eval{\dv[s]{\psi}{x}}_{x = 0}\integ{-\infty}{\infty}{x}{\frac{x^{s}e^{x}}{(e^{x} + 1)^{2}}} \\
	  &= \sum\limits_{s = 0}^{\infty}\frac{1}{(2s)!}\eval{\dv[2s]{\psi}{x}}_{x = 0}2(2s!)(1 - 2^{1 - 2s})\zeta(2s) \\
	  &= \psi(x = 0) + \frac{\pi^{2}}{6}\eval{\dv[2]{\psi}{x}}_{x = 0} + \frac{7\pi^{4}}{360}\eval{\dv[4]{\psi}{x}}_{x = 0} + \dots \\
	  &= \integ{0}{\mu}{E}{\phi} + \frac{\pi^{2}}{6}\eval{\dv{\phi}{x}}_{E = \mu} + \frac{7\pi^{4}}{360}\eval{\dv[3]{\phi}{x}}_{E = \mu} + \dots
\end{align*}