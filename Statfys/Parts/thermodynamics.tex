\section{Thermodynamics}

\paragraph{Functions of State}
A quantity $f$ is a function of state if its equilibrium value is a function of the equilibrium values of the variables describing the state.

\paragraph{Internal Energy}
The internal energy $U$ of a system is the sum of the energy of all the internal degrees of freedom of a system.

\paragraph{Quasistatic Processes}
A process is quasistatic if the system is in equilibrium at each point during the process.

\paragraph{The First Law}
Energy is conserved and heat and work are both forms of energy. In mathematical form:
\begin{align*}
	\dd{U} = \din{Q} + \din{W}.
\end{align*}
This implies the convention that positive differentials correspond to energy supplied to the system.

\paragraph{Heat Capacity of a Gas}
The work required to compress a gas is $\din{W} = -p\dd{V}$, meaning that the total change in internal energy for a gas is
\begin{align*}
	\dd{U} = \fix{\pdv{U}{T}}{V}\dd{T} + \fix{\pdv{U}{V}}{T}\dd{V}
\end{align*}
where the subscript indicates which variables are constant when the derivatives are computed. The first law gives
\begin{align*}
	\din{Q} = \fix{\pdv{U}{T}}{V}\dd{T} + \left(\fix{\pdv{U}{V}}{T} + p\right)\dd{V}.
\end{align*}
Heat capacitites are defined as derivatives of $Q$ with respect to temperature, imploring us to divide by the differential change in temperature. We thus obtain
\begin{align*}
	C_{V} = \fix{\pdv{Q}{T}}{V} = \fix{\pdv{Q}{T}}{V},\ C_{p} = \fix{\pdv{Q}{T}}{p} = \fix{\pdv{U}{T}}{V} + \left(\fix{\pdv{U}{V}}{T} + p\right)\fix{\pdv{V}{T}}{p}.
\end{align*}

\paragraph{Molar Heat Capacities}
Molar heat capacities, denoted with a small $c$, are heat capacities per molar mass.

\paragraph{Adiabatic Index}
The adiabatic index is defined as
\begin{align*}
	\gamma = \frac{C_{p}}{C_{V}}.
\end{align*}

\paragraph{The Second Law of Thermodynamics}
The second law comes in two different statements:
\begin{itemize}
	\item \textbf{Clausius' statement:} No process is possible whose sole result is the transfer of heat from a colder to a hotter body.
	\item \textbf{Kelvin's statement:} No process is possible whose sole result is the complete conversion of heat into work.
\end{itemize}
%TODO: Equivalence

\paragraph{The Carnot Cycle}
Consider a machine performing work based on the energy transfer between two heat reservoirs. One way to extract the energy is by using a Carnot process, which uses an ideal gas. The cycle connects four different points in a $pV$ diagram with two adiabatics and two isothermals.

\paragraph{Efficiency}
The efficiency of a machine has various definitions, but the general definition is the ratio between the energy you get out and the energy you put in.

\paragraph{Carnot's Theorem}
No heat engine working between two heat reservoirs is more efficient than a Carnot engine.

To prove this, suppose that you create an engine producing the same work $W$ as a Carnot heat engine, but for an energy $Q'$ as opposed to the energy $Q$ needed to run the Carnot engine. Now let your new engine be used to power a reversed Carnot engine. The statement that this new engine is more efficient is expressed as
\begin{align*}
	\frac{W}{Q'} > \frac{W}{Q} \implies Q > Q'.
\end{align*}
During the process each engine gives off some heat $Q_{\text{l}}$ to the cold reservoir. The first law of thermodynamics implies
\begin{align*}
	W = Q' - Q_{\text{l}}' = Q - Q_{\text{l}}.
\end{align*}
As $Q - Q' > 0$, so is $Q_{\text{l}} - Q_{\text{l}}'$. Now, $Q - Q'$ is the net energy left in the hot reservoir during a cycle and $Q_{\text{l}} - Q_{\text{l}}'$ is the energy extracted from the cold reservoir. The sole result of this process is thus to extract energy from a cold reservoir to a hot one, which violates the second law.

%TODO: Corollary

\paragraph{Clausius' Theorem}
Consider some arbitrary cyclic process. We can describe this as a sequence of the system being connected to temperatures $T_{i}$, with heats $\din{Q_{i}}$ being supplied at every stage. The total work performed during the cycle is given by
\begin{align*}
	W = \sum\din{Q_{i}}.
\end{align*}
Now suppose that the heat at each point is supplied by a Carnot engine operating between temperatures $T$ and $T_{i}$. One can show that for a Carnot engine the ratio of heat to temperature at which the heat is exchanged is constant. This implies
\begin{align*}
	\frac{\din{Q_{i}}}{T_{i}} = \frac{\din{Q_{i}} + \din{W_{i}}}{T}
\end{align*}
where $\din{W_{i}}$ is the work done by the Carnot engine. This process cannot violate the second law of thermodynamics, hence
\begin{align*}
	W + \sum\din{W_{i}} \leq 0.
\end{align*}
Hence
\begin{align*}
	\sum\din{Q_{i}} + \sum\din{Q_{i}}\left(\frac{T}{T_{i}} - 1\right) = T\sum\frac{\din{Q_{i}}}{T_{i}} \leq 0.
\end{align*}
The temperature $T$ is constant and positive, and can thus be ignored. In the limit of considering the process in a contiuous manner, this becomes
\begin{align*}
	\intin{}{}{Q}{\frac{1}{T}} \leq 0.
\end{align*}
This is Clausius' theorem, where equality necessarily holds for a reversible cycle.

\paragraph{Entropy}
As we have seen, the integral
\begin{align*}
	\intin{}{}{Q}{\frac{1}{T}}
\end{align*}
is path independent for a reversible process. We thus define the state function
\begin{align*}
	\dd{S} = \frac{\din{Q}}{T}
\end{align*}
to be the entropy.

\paragraph{The Second Law and Entropy}
Consider two points connected by a reversible and an irreversible process such that the two form a cycle. Clausius' theorem yields
\begin{align*}
	\intin{A}{B}{Q}{\frac{1}{T}} + \intin{B}{A}{Q_{\text{rev}}}{\frac{1}{T}} \leq 0 \implies \intin{A}{B}{Q}{\frac{1}{T}} \leq \intin{A}{B}{Q_{\text{rev}}}{\frac{1}{T}}.
\end{align*}
This holds for two arbitrary points, meaning
\begin{align*}
	\dd{S} \geq \frac{\din{Q}}{T}.
\end{align*}
Considering a thermally isolated system, we obtain
\begin{align*}
	\dd{S} \geq 0.
\end{align*}
This is a restatement of the second law of thermodynamics.

\paragraph{The First Law and Entropy}
For a reversible process we obtain
\begin{align*}
	\dd{U} = T\dd{S} - p\dd{V}.
\end{align*}
However, as all involved quantities are functions of state, this must hold even if the process in question is irreversible.

\paragraph{Natural Variables For Internal Energy}
Our restatement of the first law implies that $U$ can be written as a function of $S$ and $V$.

%TODO: Derivatives of U and S

\paragraph{Enthalpy}
The enthalpy is defined as 
\begin{align*}
	H = U + pV.
\end{align*}
We have
\begin{align*}
	\dd{H} = \dd{U} + p\dd{V} + V\dd{p} = T\dd{S} - p\dd{V} + p\dd{V} + V\dd{p} = T\dd{S} + V\dd{p},
\end{align*}
implying that $H$ is a function of $S$ and $p$, as well as
\begin{align*}
	\fix{\pdv{H}{S}}{p} = T,\ \fix{\pdv{H}{p}}{S} = V.
\end{align*}

\paragraph{Helmholtz Free Energy}
The Helmholtz free energy is defined as
\begin{align*}
	F = U - TS.
\end{align*}
We have
\begin{align*}
	\dd{F} = \dd{U} - T\dd{S} - S\dd{T} = T\dd{S} - p\dd{V} - T\dd{S} - S\dd{T} = -p\dd{V} - S\dd{T},
\end{align*}
implying that $H$ is a function of $T$ and $p$, as well as
\begin{align*}
	\fix{\pdv{F}{T}}{V} = -S,\ \fix{\pdv{F}{V}}{T} = -p.
\end{align*}

\paragraph{Gibbs Free Energy}
The Gibbs free energy is defined as
\begin{align*}
	G = H - TS.
\end{align*}
We have
\begin{align*}
	\dd{G} = \dd{H} - T\dd{S} - S\dd{T} = T\dd{S} + V\dd{p} - T\dd{S} - S\dd{T} = V\dd{p} - S\dd{T},
\end{align*}
implying that $G$ is a function of $T$ and $p$, as well as
\begin{align*}
	\fix{\pdv{G}{T}}{p} = -S,\ \fix{\pdv{G}{p}}{T} = V.
\end{align*}

\paragraph{Availability and Equilibrium}
Consider a system in contact with surroundings at temperature $T_{0}$ and pressure $p_{0}$. If heat $\din{Q}$ is supplied to the system during some process, the entropy change satisfies $T_{0}\dd{S} \geq \din{Q}$. The first law implies
\begin{align*}
	\din{Q} = \dd{U} - \din{W} - (-p_{0}\dd{V}),
\end{align*}
where we have separated the work into a term arising due to the change in volume and a term describing other sources. Combining this yields
\begin{align*}
	T_{0}\dd{S} &\geq \dd{U} - \din{W} - (-p_{0}\dd{V}), \\
	\din{W}     &\geq \dd{U} - T_{0}\dd{S}  + p_{0}\dd{V}.
\end{align*}
Defining the availability for this case as
\begin{align*}
	A = U + p_{0}V - T_{0}S
\end{align*}
implies
\begin{align*}
	\din{W} \geq \dd{A}.
\end{align*}
If the system is mechanically isolated from the surroundings, such that only volume-changing work can be performed, we have
\begin{align*}
	\dd{A} \leq 0.
\end{align*}
This is a general equilibrium condition which applies to systems with many different constraints, depending on a proper definition of the availability.

\paragraph{Maxwell Relations}
Using the symmetry of partial derivatives for a state function, one can obtain relations between derivatives of other state functions.

As an example, the derivatives of $F$ are $-S$ and $-p$, computed with respect to $T$ and $V$. This implies
\begin{align*}
	\fix{\pdv{S}{V}}{T} = \fix{\pdv{p}{T}}{V}.
\end{align*}