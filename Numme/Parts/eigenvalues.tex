\section{Egenvärdesberäkningar}

\paragraph{Potensmetoden}
Potensmetoden är ett sätt att hitta det största egenvärdet till en matris på. Algoritmen går ut på att välja en startvektor $x^{0}$ och iterera enligt
\begin{align*}
	x^{i + 1} = \frac{1}{\norm{Ax^{i}}}Ax^{i}, \lambda^{i + 1} = (x^{i + 1})^{T}Ax^{i + 1}.
\end{align*}
Detta kan alternativt skrivas som
\begin{align*}
	v^{i + 1} = Ax^{i}, \lambda^{i} = (x^{i})^{T}v^{i + 1}, x^{i + 1} = \frac{1}{\norm{v^{i + 1}}}v^{i + 1}.
\end{align*}