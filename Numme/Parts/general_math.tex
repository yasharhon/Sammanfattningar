\section{Användbar matte}

\paragraph{Allmän begränsning av globalt fel}
Betrakta begynnelsevärdesproblemet
\begin{align*}
	\deval{y}{t}{t} &= f(t, y(t)), \\
	y(a)            &= b,
\end{align*}
löst på $[a, T]$, där $f$ är Lipschitzkontinuerlig. Betrakta en numerisk lösning med lokalt fel begränsad av $Mh^{p + 1}$. Då begränsas det globala felet av
\begin{align*}
	\abs{y(T) - y_{N}} \leq \frac{e^{L(T - a)}M}{L}h^{p}.
\end{align*}

\proof
Vi inför $y(t; t_{n})$ som den exakta lösningen som startar i $(t_{n}, y_{n})$. Det globala felet ges då av
\begin{align*}
	\abs{y(T) - y_{N}} &= \abs{y(T) - y(T; t_{N - 1}) + y(T; t_{N - 1}) + \dots - y(T; t_{1}) + y(T; t_{1}) - y_{N}} \\
	                   &\leq \abs{y(T) - y(T; t_{N - 1})} + \abs{y(T; t_{N - 1}) - y(T; t_{N - 2})} + \dots + \abs{y(T; t_{1}) - y_{N}}.
\end{align*}
Den första termen ges simpelthen av det lokala felet. Satsen om entydighet av lösning för en sådan differentialekvation ger vidare
\begin{align*}
	\abs{y(T; t_{i}) - y(T; t_{i - 1})} \leq e^{L(T - t_{i})}\abs{y(t_{i}; t_{i}) - y(t_{i}; t_{i - 1})}.
\end{align*}
Det som står kvar i absolutbeloppstecknet är det lokala felet, eftersom den vänstra termen är exakt och den högra kommer från en iteration. Detta ger
\begin{align*}
	\abs{y(T; t_{i}) - y(T; t_{i - 1})} \leq e^{L(T - t_{i})}Mh^{p + 1} = e^{L(N - i)h}Mh^{p + 1}
\end{align*}
och vidare
\begin{align*}
	\abs{y_{N} - y(T)} &\leq Mh^{p + 1} + Mh^{p + 1}e^{Lh} + \dots + Mh^{p + 1}e^{L(N - 1)h}. \\
	                   &= Mh^{p + 1}\frac{1 - e^{LNh}}{1 - e^{Lh}} \\
	                   &= Mh^{p + 1}\frac{e^{LNh} - 1}{e^{Lh} - 1} \\
	                   &\leq Mh^{p + 1}\frac{e^{LNh}}{Lh},
\end{align*}
och beviset är klart.

\paragraph{Möjlighet för polynominterpolation}
Givet $n + 1$ punkter $(x_{i}, y_{i})$, där $x_{i}\neq x_{j}$ om $i\neq j$, finns det ett entydigt polynom $p$ av grad (högst) $n$ så att $p(x_{i}) = y_{i}\ \forall i= 1, \dots, n + 1$.

\proof
För att visa existensen av $p$, kan vi välja det som
\begin{align*}
	p(x) = \sum\limits_{i = 1}^{n + 1}y_{i}\prod\limits_{j\neq i}\frac{x - x_{j}}{x_{i} - x_{j}}.
\end{align*}
Denna metoden kallas för Lagrangeinterpolation. Vi ser att $p$ har grad $n$ och
\begin{align*}
	p(x_{k}) = \sum\limits_{i = 1}^{n + 1}y_{i}\prod\limits_{j\neq i}\frac{x_{k} - x_{j}}{x_{i} - x_{j}}.
\end{align*}
För alla termer i summan där $i\neq k$ kommer det finnas en faktor $x_{k} - x_{k}$ i nämnaren, och dessa ger inget bidrag. För $i = k$ blir produkten bara $1$, och $p(x_{k}) = y_{k}$.

För att visa att $p$ är entydig, antag att $q$ är ett annat interpolationspolynom och bilda $v = q - p$, med grad högst $n$. Detta ger att $v$ har $n + 1$ nollställen. Detta är endast möjligt om $v = 0$, och därmed måste $p$ vara unikt.

\paragraph{Fel för linjär interpolation}
Antag att $f\in C^{2}$. Låt $p$ vara det linjära polynomet som interpolerar punkterna $(x_{0}, f(x_{0})$ och $(x_{0} + h, f(x_{0} + h)$. Då gäller $\abs{f(x) - p(x)} \leq Ch^{2},\ x_{0} \leq x \leq x_{0} + h$, där
\begin{align*}
	C = \max\limits_{x_{0} \leq z \leq x_{0} + h}\frac{\abs{\deval[2]{f}{x}{z}}}{8}.
\end{align*}

\proof
Antag $x_{0} = 0$, och bilda $g = f - p$. $g$ är kontinuerlig på $[0, h]$, och antar därmed ett största och minsta värde på detta intervallet. Antag att den antar sitt största värde i $x = a$. Taylorutveckling kring $a$ ger
\begin{align*}
	g(x) = g(a) + \deval{g}{x}{a}(x - a) + \frac{1}{2}\deval[2]{g}{x}{c}(x - a)^{2}
\end{align*}
för något $c\in [a, x]$. Vi ser att andra termen måstse bli $0$, ty $a$ är en maxpunkt. Evaluering av Taylorutvecklingen i $x = 0$ eller $x = h$ (den som är närmast $a$) ger $g(x) = 0$ och
\begin{align*}
	g(a)       &= -\frac{1}{2}\deval[2]{g}{x}{c}(x - a)^{2},
	\abs{g(a)} &\leq \frac{1}{2}\max\limits_{x_{0} \leq z \leq x_{0} + h}\abs{\deval[2]{g}{x}{z}}\left(\frac{1}{2}h\right)^{2}.
\end{align*}
Vi vet att $\deval[2]{f}{x}{x} = \deval[2]{g}{x}{x}$ eftersom $p$ är linjär, vilket ger
\begin{align*}
	\abs{g(a)} &\leq \max\limits_{x_{0} \leq z \leq x_{0} + h}\frac{\abs{\deval[2]{f}{x}{z}}}{8}h^{2}.
\end{align*}