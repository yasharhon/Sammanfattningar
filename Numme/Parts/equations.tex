\section{Lösning av ekvationer}

\paragraph{Fixpunktsmetoden}
Betrakta ekvationen
\begin{align*}
	x = g(x).
\end{align*}

Fixpunktsmetoden är en enkel iterationsmetod för att lösa denna ekvationen, med den enkla iterationsformeln
\begin{align*}
	x_{n + 1} = g(x_{n}).
\end{align*}

En pseudokod-beskrivning av en lösning med startvillkor \verb|x0| där det itereras tills lösningen stämmer med en tolerans \verb|t| är:
\begin{lstlisting}
	define g(x)
	input x0
	input t
	while abs(x - g(x)) > t
		x = g(x)
	end
	return x
\end{lstlisting}

%Bli klar med detta
\subparagraph{Konvergens}
Om $g\in C^{1}$, $\abs{\deval{g}{x}{\alpha}} < 1$ och $\alpha$ är en fixpunkt. finns det en omgivning till $\alpha$ så att om $x_{0}$ är i denna omgivningen, går $x_{n}\to\alpha$. Metoden konvergerar linjärt med reduktionsfaktor $S = \abs{\deval{g}{x}{\alpha}}$.

För att visa detta, skriver vi
\begin{align*}
	x_{n + 1} - \alpha = g(x_{n}) - g(\alpha) = \deval{g}{x}{c}(x_{n} - \alpha),
\end{align*}
där vi har användt medelvärdesatsen och det faktum att $\alpha$ är en fixpunkt. Vidare, eftersom $g\in C^{1}$ finns det en omgivning till $\alpha$ så att $\abs{\deval{g}{x}{x}}\leq \frac{S + 1}{2}$. Om $x_{0}$ är i denna, är 
\begin{align*}
	e_{n + 1} \leq \frac{S + 1}{2}e_{n}.
\end{align*}
Detta implicerar att $x_{n} \to \alpha$ och att
\begin{align*}
	\frac{e_{n + 1}}{e_{n}} \to S.
\end{align*}

\paragraph{Fixpunktsmetoden för system}
Betrakta ekvationssystemet
\begin{align*}
	\vb{x} = g(\vb{x}).
\end{align*}

Vi löser även detta med fixpunktsmetoden, och använder iterationsformeln
\begin{align*}
	\vb{x}_{n + 1} = g(\vb{x}_{n}).
\end{align*}

En pseudokod-beskrivning av en lösning med startvillkor \verb|x0| där det itereras tills lösningen stämmer med en tolerans \verb|t| är:
\begin{lstlisting}
	define g(x)
	input x0
	input t
	while abs(x - g(x)) > t
		x = g(x)
	end
	return x
\end{lstlisting}
där allt nu är listor. Toleransen ser kanske lite annorlunda ut, vafan vet jag.

\paragraph{Intervallhalveringsalgoritmen}
Betrakta ekvationen
\begin{align*}
	f(x) = 0.
\end{align*}

Intervallhalveringsalgoritmen utgår från två punkter $a, b$ så att $f(a)f(b) < 0$, och gör följande:
\begin{enumerate}
	\item Beräkna funktionsvärdet i punkten $m = \frac{b + a}{2}$.
	\item Om $f(a)f(m) < 0$, sätt $a = m$. Annars, sätt $b = m$.
	\item Sluta iterationen när intervallbredden $\frac{b - a}{2}$ är mindre än den givna toleransen.
	\item Returnera $\frac{b + a}{2}$.
\end{enumerate}

En pseudokod-beskrivning av en lösning med startvillkor \verb|a| och \verb|a| där det itereras tills lösningen stämmer med en tolerans \verb|t| är:
\begin{lstlisting}
	define f(x)
	input a, b
	input t
	while (b - a)/2 > t
		m = (a + b)/2
		if f(a)f(m) < 0
			a = m
		else
			b = m
		end
	end
	return (a + b)/2
\end{lstlisting}

\paragraph{Newton-Rhapsonsmetoden}
Betrakta ekvationen
\begin{align*}
	f(x) = 0.
\end{align*}
Newton-Rhapsons metod utgår från tangenten till $f$. Om man startar i $x_{0}$, har tangenten ekvation $t(x) = \deval{f}{x}{x_{0}}(x - x_{0}) + f(x_{0})$. Dens nollställe ges av
\begin{align*}
	x = x_{0} - \frac{f(x_{0})}{\deval{f}{x}{x_{0}}}.
\end{align*}
Från detta gör vi iterationen
\begin{align*}
	x_{n + 1} = x_{n} - \frac{f(x_{n})}{\deval{f}{x}{x_{n}}}
\end{align*}
som avslutas när $\abs{x_{n + 1} - x_{n}}$ är mindre än någon tolerans. Vi ser att detta är en variant av fixpunktsmetoden med $g(x) = x - \frac{f(x)}{\deval{f}{x}{x}}$.

En pseudokod-beskrivning av en lösning med startvillkor \verb|x_0| där det itereras tills lösningen stämmer med en tolerans \verb|t| är:
\begin{lstlisting}
	define f(x)
	define fderiv(x)
	input x_0
	input t
	while f(x_0) > t
		x = x - f(x)/fderiv(x)
	end
	return x
\end{lstlisting}

\subparagraph{Konvergens}
Om $\alpha$ är ett nollställe till $f$, är det även ett nollställe till $g$. Vi har för $f\in C^{2}$ och $\deval{f}{x}{\alpha} \neq 0$ att
\begin{align*}
	\deval{g}{x}{\alpha} = 1 - \frac{\deval{f}{x}{\alpha}}{\deval{f}{x}{\alpha}} + \frac{f(\alpha)\deval[2]{f}{x}{\alpha}}{\deval{f}{x}{\alpha}^{2}} = 0.
\end{align*}
Därmed, om dessa villkor uppfylls, är Netwon-Rhapsons metod kvadratiskt konvergent med konstant $M = \frac{\deval[2]{f}{x}{\alpha}}{2\deval{f}{x}{\alpha}}$.

För att visa detta, konstaterar vi att lokal konvergens följer av beviset som gjordes för fixpunktsmetoden. Vi Taylorutvecklar vidare nära $x_{n}$ och får
\begin{align*}
	f(\alpha)                            &= f(x_{n}) + \deval{f}{x}{x_{n}}(\alpha - x_{n}) + \frac{1}{2}\deval[2]{f}{x}{c_{n}}(\alpha - x_{n})^{2}, \\
	\frac{f(\alpha)}{\deval{f}{x}{x_{n}}} &= \frac{f(x_{n})}{\deval{f}{x}{x_{n}}} + \alpha - x_{n} + \frac{1}{2}\frac{\deval[2]{f}{x}{c_{n}}}{\deval{f}{x}{x_{n}}}(\alpha - x_{n})^{2} = 0.
\end{align*}
Detta skriver vi om till
\begin{align*}
	x_{n} - \frac{f(x_{n})}{\deval{f}{x}{x_{n}}} - \alpha = \frac{1}{2}\frac{\deval[2]{f}{x}{c_{n}}}{\deval{f}{x}{x_{n}}}(\alpha - x_{n})^{2}.
\end{align*}
Detta implicerar
\begin{align*}
	\frac{e_{n + 1}}{e_{n}^{2}} = \abs{\frac{\deval[2]{f}{x}{c_{n}}}{2\deval{f}{x}{x_{n}}}} \to M,
\end{align*}
och beviset är klart.

\paragraph{Newton-Rhapsons metod för system}
Betrakta ekvationen
\begin{align*}
	\vb{f}(\vb{x}) = 0.
\end{align*}
Metoden är den samma för ett enda system. Den utgår från den linjära approximationen
\begin{align*}
	\vb{f}(\vb{x}) = \vb{f}(\vb{x}_{0}) + \dd{\vb{f}}(\vb{x}_{0})(\vb{x} - \vb{x}_{0})
\end{align*}
och ger iterationen
\begin{align*}
	\vb{x}_{n + 1} = \vb{x}_{n} - \dd{\vb{f}}(\vb{x}_{n})^{-1}f(\vb{x}_{n}).
\end{align*}
Notera att beräkningsmässigt är det svårt att hitta en inversmatris, så det är smartare att hitta en vektor $\vb{h}$ så att $\dd{\vb{f}}(x_{n})\vb{h} = \vb{f}(x_{n})$ och använda den i stället.

\paragraph{Linjära system}
Lösning av ett $n\times n$ linjärt system $Ax = b$ görs med ett antal beräkningar som är $\Ordo{n^{p}}$ för något $p$. För Gausselimination är $p = 3$.

\subparagraph{Störningsanalys}
Antag att man försöker lösa ett system $Ax = b$. Givet approximativ indata $\tilde{b}$ med en given felgräns, dvs. en given gräns för $\abs{b - \tilde{b}}$, hur stor felgräns fås för den approximativa lösningen $\tilde{x}$?

Man kan skriva
\begin{align*}
	\abs{x - \tilde{x}} = A^{-1}(b - \tilde{b}),
\end{align*}
och normen av detta blir
\begin{align*}
	\norm{x - \tilde{x}} \leq \norm{A^{-1}}\norm{b - \tilde{b}},
\end{align*}
där vi har infört matrisnormen
\begin{align*}
	\norm{A} = \max\limits_{x \neq 0}\frac{\norm{Ax}}{\norm{x}}
\end{align*}
som uppfyller
\begin{align*}
	\norm{Ax} \leq \norm{A}\norm{x}.
\end{align*}

Uttrycket ovan innehåller absoluta fel, men ofta vet vi bara relativa fel på formen $\frac{\norm{b - \tilde{b}}}{\norm{b}}$. Med hjälp av matrisnormen kan vi dock skriva
\begin{align*}
	\norm{b} \leq \norm{A}\norm{x},
\end{align*}
vilket kan kombineras med uttrycket ovan för att ge
\begin{align*}
	\frac{\norm{x - \tilde{x}}}{\norm{x}} \leq \norm{A^{-1}}\norm{A}\frac{\norm{b - \tilde{b}}}{\norm{b}}.
\end{align*}
Vi kan då definiera konditionstalet $\kappa = \norm{A^{-1}}\norm{A}$, vilket ger
\begin{align*}
	R_{x} \leq \kappa R_{b}.
\end{align*}