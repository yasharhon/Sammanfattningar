\section{Integration}

\paragraph{Eulers metod}
Utgå från differentialekvationen
\begin{align*}
	\dv{y}{x} = f, y(a) = y_{0},
\end{align*}
och antag att vi vill integrera $f$ över $[a, b]$. Analysens huvudsats ger
\begin{align*}
	\inteval{a}{b}{f(x)}{x} = y(b) - y(a).
\end{align*}
Del nu upp $[a, b]$ i $N$ steg, och låt $x_{0} = a, x_{N} = b$. Eulers metod ger
\begin{align*}
	y_{N} &= y_{N - 1} + hf(x_{N - 1}) \\
	      &= y_{N - 2} + hf(x_{N - 2}) + hf(x_{N - 1}) \\
	      &\vdots \\
	      &= y_{0} + h\sum\limits_{i = 0}^{N - 1}f(x_{i}).
\end{align*}
Vi får slutligen approximationen
\begin{align*}
	\inteval{a}{b}{f(x)}{x} \approx h\sum\limits_{i = 0}^{N - 1}f(x_{i}).
\end{align*}

\paragraph{Trapetsmetoden}
Utgå från differentialekvationen
\begin{align*}
	\dv{y}{x} = f, y(a) = y_{0},
\end{align*}
och antag att vi vill integrera $f$ över $[a, b]$. Del nu upp $[a, b]$ i $N$ steg, och låt $x_{0} = a, x_{N} = b$. I trapetsmetoden interpolerar vi $f$ i $x_{i}$ och $x_{i + 1}$ och integrerar denna räta linjen. Integralen är lika med arean av en trapets, vilket vi enkelt kan beräkna. En approximation av integralen över $[x_{i}, x_{i + 1}]$ är därmed
\begin{align*}
	h\frac{f(x_{i + 1}) + f(x_{i})}{2}.
\end{align*}
En approximation av hela integralen är då
\begin{align*}
	\inteval{a}{b}{f(x)}{x} &\approx h\sum\limits_{i = 0}^{N - 1}\frac{f(x_{i + 1}) + f(x_{i})}{2} \\
	                        &= h\left(\frac{f(x_{0}) + f(x_{N})}{2} + \sum\limits_{i = 1}^{N - 1}f(x_{i})\right).
\end{align*}

\subparagraph{Fel}
Låt $T_{h}$ vara skattningsfelet av integralen med trapetsmetoden med steglängd $h$. Felet ges då av
\begin{align*}
	\abs{\inteval{a}{b}{f(x)}{x} - T_{h}} \leq \max\limits_{x\in [a, b]}\frac{1}{12}\abs{\deval[2]{f}{x}{x}}(b - a)h^{2}.
\end{align*}

\paragraph{Simpsons metod}
Använder kvadratisk interpolation.

\subparagraph{Fel}
Låt $S_{h}$ vara skattningsfelet av integralen med Simpsons metod med steglängd $h$. Felet ges då av
\begin{align*}
	\abs{\inteval{a}{b}{f(x)}{x} - S_{h}} \leq \max\limits_{x\in [a, b]}\frac{1}{80}\abs{\deval[4]{f}{x}{x}}(b - a)h^{4}.
\end{align*}

\paragraph{Monte Carlo-integration}
Medelvärdesatsen för integraler ger att
\begin{align*}
	\frac{1}{b - a}\inteval{a}{b}{f(x)}{x} = f(c),
\end{align*}
där $f(c)$ är funktionens medelvärde på $[a, b]$. Vi kan skatta medelvärdet som
\begin{align*}
	\frac{1}{N}\sum\limits_{i = 1}{N}f(x_{i}),
\end{align*}
där alla $x_{i}$ är dragna oberoende från en likformig fördelning på $[a, b]$. Detta kallas för Monte Carlo-integration.

Metoden är helt analog i högre dimensioner.

\paragraph{Fel}
Om vi antar att alla $x_{i}$ dras från en likformig sannolikhetstäthet, kan vi skriva
\begin{align*}
	\frac{1}{b - a}\inteval{a}{b}{f(x)}{x} = \inteval{a}{b}{f(x)p(x)}{x},
\end{align*}
där $p$ är sannolikhetstätheten. Vänstersidan kan även skrivas som $\expec{f(X)}$.

Vi inför den stokastiska variabeln
\begin{align*}
	\varepsilon_{n} = \frac{1}{N}\sum\limits_{i = 1}{N}f(X_{i}) - \expec{f(X)}.
\end{align*}
Dens väntevärde ges av
\begin{align*}
	\expec{\varepsilon_{n}} &= \frac{1}{N}\sum\limits_{i = 1}{N}\expec{f(X_{i})} - \expec{f(X)} \\
	                         &= \expec{f(X)} - \expec{f(X)} \\
	                         &= 0.
\end{align*}
Vi har vidare
\begin{align*}
	\expec{\varepsilon_{n}^{2}} &= \expec{\left(\frac{1}{N}\sum\limits_{i = 1}^{N}f(X_{i}) - \expec{f(X)}\right)^{2}} \\
	                         &= \frac{1}{N^{2}}\expec{\left(\sum\limits_{i = 1}^{N}(f(X_{i}) - \expec{f(X)})\right)^{2}} \\
	                         &= \frac{1}{N^{2}}\expec{\left(\sum\limits_{i = 1}^{N}(f(X_{i}) - \expec{f(X)})\right)^{2}} \\
	                         &= \frac{1}{N^{2}}\expec{\left(\sum\limits_{i = 1}^{N}(f(X_{i}) - \expec{f(X)})\right)^{2}} \\
	                         &= \frac{1}{N^{2}}\expec{\sum\limits_{i = 1}^{N}(f(X_{i}) - \expec{f(X)})^{2} + \sum\limits_{n\neq m}(f(X_{n}) - \expec{f(X)})(f(X_{m}) - \expec{f(X)})} \\
	                         &= \frac{1}{N^{2}}\sum\limits_{i = 1}^{N}\expec{(f(X_{i}) - \expec{f(X)})^{2}} + \expec{\sum\limits_{n\neq m}(f(X_{n}) - \expec{f(X)})(f(X_{m}) - \expec{f(X)})} \\
	                         &= \frac{\sigma_{f}^{2}}{N} + 0.
\end{align*}
Korstermerna har väntevärde
\begin{align*}
	\expec{(f(X_{n}) - \expec{f(X)})(f(X_{m}) - \expec{f(X)})} &= \expec{f(X_{n}) - \expec{f(X)}}\expec{f(X_{m}) - \expec{f(X)}} \\
	                                                           &= 0,
\end{align*}
då de olika $X_{i}$ är oberoende. Därmed är standardavvikelsen av $\varepsilon_{N}$ lika med $\frac{\sigma}{\sqrt{N}}$, och felet i metoden är proportionellt mot $\frac{1}{\sqrt{N}}$.

Vi noterar att centrala gränsvärdesatsen ger att $\nu = \frac{\sqrt{N}}{\sigma}\varepsilon_{N}$ är standard normalfördelat.

I $d$ dimensioner är felet proportionellt mot $N^{-\frac{d}{2}}$, och vi ser att den är bättre än trapetsmetoden för $d > 4$.