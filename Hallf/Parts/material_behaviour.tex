\section{Materialers beteende}

\paragraph{Idealplastisk deformation}
De flesta material beter sig så att när de deformeras förbi en viss punkt, deformeras de plastiskt i stället för elastiskt. En approximation för att beskriva detta beteendet är att låta deformationen vara elastisk upp till en töjningsgräns $\varepsilon_{\text{s}}$, och låta $\sigma$ vara konstant lika med en sträckgräns $\sigma_{\text{s}}$ för större töjningar.

Om materialet komprimeras, visar det sig att det elastiska beteendet ofta är likt.

När lasten sedan tas bort, kommer stången förkortas igen tills lasten blir lika med noll. Denna kontraktionen är parallell med det elastiska regimet, och konsekvensen är att man får en permanent deformation.