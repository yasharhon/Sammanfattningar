\section{Energi}

\paragraph{Töjningsenergi för en stång}
Vid axial belastning av en stång med längd $l$ och tvärsnittsarea $A$ med en kraft $F$ görs ett arbete
\begin{align*}
	W = \integ{0}{u}{x}{F}.
\end{align*}
Vi kan introducera definitionen av spänning och göra en substitution där vi inför töjningen som ny integrationsvariabel för att få
\begin{align*}
	W = AL\integ{0}{\varepsilon}{\varepsilon}{\sigma}.
\end{align*}
Vi definierar då töjningsenergin per volym
\begin{align*}
	w_{V} = \integ{0}{\varepsilon}{\varepsilon}{\sigma}.
\end{align*}
Eftersom stången har volym $Al$, ger detta
\begin{align*}
	W = Vw_{V}.
\end{align*}

Vi kan nu introducera Hookes lag, vilket ger
\begin{align*}
	w_{W} = \frac{1}{2}E\varepsilon^{2}.
\end{align*}

\paragraph{Skjuvenergi}
Betrakta ett rätblock med höjd $h$, brädd $b$ och längd $L$ som skjuvs så att höjdriktningen bildar vinkeln $\gamma$ med vertikalen. Det infinitesimala arbetet vid en liten skjuvning ges av
\begin{align*}
	\dd{W} = \tau bt\cdot h\dd{\gamma},
\end{align*}
och det totala arbetet ges av
\begin{align*}
	W = \integ{0}{\gamma}{\gamma}{\tau bth},
\end{align*}
vilket implicerar
\begin{align*}
	w_{V} = \integ{0}{\gamma}{\gamma}{\tau}.
\end{align*}

Vid att införa Hookes lag kan man visa att
\begin{align*}
	W = \frac{1}{2}VG\gamma^{2}.
\end{align*}

\paragraph{Energi för böjning av en balk}
Om balken har utböjning $w$, kommer man få
\begin{align*}
	W = \integ{0}{L}{x}{\frac{1}{2}EI\left(\dv[2]{w}{x}\right)^{2}}.
\end{align*}

\paragraph{Energi för vridning av en balk}
Om balken vris i axelriktning en vinkel $\phi$, får man
\begin{align*}
	W = \integ{0}{L}{x}{\frac{1}{2}GK\left(\dv{\phi}{x}\right)^{2}}.
\end{align*}

\paragraph{Allmän töjningsenergi}
Helt allmänt har man
\begin{align*}
	W = \integ[3]{V}{}{r}{\frac{1}{2}\sigma_{ij}\varepsilon_{ij}}.
\end{align*}

\paragraph{Axialbelastad stång och virtuellt arbete}
Betrakta en axialbelastad stång som belastas med en kraft $F$ som producerar volymskraften $K_{x}$ i axelriktningen. Kraftjämvikt, till sammmans med definitionen av töjning, ger
\begin{align*}
	\dv{x}\left(EA\dv{u}{x}\right) + K_{x} = 0.
\end{align*}
Om balken är inspänd i ena ändan, fås
\begin{align*}
	u(0) = 0,\ EA\deval{u}{x}{L} = F.
\end{align*}
Enligt argumentet ovan har balken energi
\begin{align*}
	W = \integ{0}{L}{x}{\frac{1}{2}EA\left(\dv{u}{x}\right)^{2}},
\end{align*}
som är en funktional av deformationen $u$. Allmänt är jämviktsläget för balken ett extremum till denna funktionalen.

Vi betraktar elastiska fall nu, och antar att $u$ uppfyller differentialekvationen som beskriver balken och alla randvillkor. Vi introducerar variationen $\var{u}$ av $u$. Ändring av $u$ med denna variation ger en variation av energin
\begin{align*}
	\var{W} &= \integ{0}{L}{x}{\frac{1}{2}EA\var{\left(\dv{u}{x}\right)^{2}}} \\
	        &= \integ{0}{L}{x}{\frac{1}{2}EA2\dv{u}{x}\var{\dv{u}{x}}}.
\end{align*}
Vi antar nu att variationen uppfyller de kinematiska randvillkoren, vilket ger $\var{u}(0) = 0$. Partiell integration ger
\begin{align*}
	\var{W} &= EA\deval{u}{x}{L}\var{u}(L) - \integ{0}{L}{x}{\var{\dv{u}{x}}\dv{x}\left(EA\dv{u}{x}\right)} \\
	        &= F\var{u}(L) - \integ{0}{L}{x}{\var{\dv{u}{x}}\dv{x}K_{x}}.
\end{align*}

\paragraph{Potentiell energi för en axialbelastad balk}
I vissa fall, typiskt elastiska fall, kan vi nu införa potentiella energin
\begin{align*}
	U = W - Fu(L) - \integ{0}{l}{x}{K_{x}}.
\end{align*}
En funktion som är ett extremum till $U$ ger $\var{U} = 0$. Det kan visas att $U$ är positivt definit, så extremumet måste vara ett minimum, och ämviktsläget är ett minimum för potentiella energin, som väntat. Vi ser att detta kriteriet är ekvivalent med att $\var{W}$ ges av uttrycket ovan. Alltså är variationsprincipen ett specialfall av variationsprincipen för $W$.

\paragraph{Potentiell energi för en böjd balk}
Vi kan även definiera en potential
\begin{align*}
	U =& \integ{0}{L}{x}{\frac{1}{2}EI\left(\dv[2]{w}{x}\right)^{2}} + M(L)\deval{w}{x}{L} - M(0)\deval{w}{x}{0} - T(L)w(L) + T(0)w(0)\\
	   &- \sum F_{i}w(x_{i}) + \sum M_{i}\deval{w}{x}{x_{i}} - \integ{0}{L}{x}{qw},
\end{align*}
där summorna görs över alla yttre krafter och moment.

\paragraph{Energimetoder i hållfasthetslära}
Man vill inte alltid lösa variationsproblemen som dyker upp i hållfasthetslära, utan man gör ansatsen
\begin{align*}
	u = \sum a_{i}u_{i},
\end{align*}
där alla $u_{i}$ är kända, och ser vilka koefficienter $a_{i}$ som minimerar de olika funktionalerna. Om man väljer de olika $u_{i}$ smart, blir lösningen bättre desto fler termer man har.