\section{Analys i tre dimensioner}

\paragraph{Spänning}
Vi definierar spänningsvektorn
\begin{align*}
	\vb{s} = \lim\limits_{A \to 0}\frac{\vb{F}}{A}
\end{align*}
där $\vb{F}$ är kraften på det lilla arealementet. Den har en komponent normalt på ytan, som är normalspänningen, och en komponent som är parallel med ytan, som är skjuvspänningen. Vi får
\begin{align*}
	\sigma = \vb{s}\cdot\vb{n}, \tau^{2} = \abs{\vb{s}}^{2} - \sigma^{2}.
\end{align*}

\paragraph{Skjuvspänningar i tre dimensioner}
Snitta nu ut en infinitesimal kub. På ytorna, till exempel ytan som är normal på $x$-axeln, kan man dekomponera spänningsvektorn i en normalspänning $\sigma_{x}$ och två skjuvspänningar $\tau_{xy}, \tau_{xz}$, och motsvarande i andra riktningar. Om vi tittar på momentjämvikt kring kubens centrum i $x$-riktning fås
\begin{align*}
	2\tau_{yz}\dd{x}\dd{z}\cdot\frac{1}{2}\dd{y} - 2\tau_{zy}\dd{x}\dd{y}\cdot\frac{1}{2}\dd{z}        &= 0, \\
	\tau_{yz} &= \tau_{zy}.
\end{align*}
En motsvarande härledning kan göras för de andra sidorna. Tillkommer det andra termer om skjuvspänningen varierar? Ja, men dessa kommer vara av högre ordning, och kan försummas.

\paragraph{Spänningsmatris}
Vi kan nu definiera en spänningsmatris
\begin{align*}
	S =
	\mqty[
		\sigma_{x} & \tau_{yx}  & \tau_{zx} \\
		\tau_{xy}  & \sigma_{y} & \tau_{zy} \\
		\tau_{xz}  & \tau_{yz}  & \sigma_{z}
	].
\end{align*}
Enligt argumentet ovan är denna symmetrisk.

\paragraph{Spänningar på godtycklig yta}
Om man har en godtycklig yta med normalvektor $\vb{n}$, kan det visas att spänningarna på ytan ges av
\begin{align*}
	\vb{s} = S\vb{n}.
\end{align*}
Vi kan då skriva
\begin{align*}
	\sigma &= \vb{n}^{T}S\vb{n}, \\
	\tau   &= \abs{S\vb{n}}^{2} - (\vb{n}^{T}S\vb{n})^{2}.
\end{align*}

\paragraph{Huvudspänningar}
Finns det orienteringar sådana att $\vb{s} = \sigma\vb{n}$? Att hitta sådana är ett egenvärdesproblem. Matematiken ger att det finns sådana orienteringar, och att de är ortogonala mot varandra.

\paragraph{Plana tillstånd}
Ett specialfall är när $z$-riktningen är en huvudriktning för spänningen. Då är skjuvspänningarna i $xy$-planet, dvs. $\tau_{zx} = \tau_{zy} = 0$. Om $\sigma_{z} = 0$, har man plan spänning.

Betrakta plan spänning på ett plan som bildar en vinkel $\phi$ med $y$-axeln. Normalvektorn ges av
\begin{align*}
	\vb{n} = \cos{\phi}\vb{e}_{x} + \sin{\phi}\vb{e}_{y}.
\end{align*}
Vi får då
\begin{align*}
	\sigma &= \sigma_{x}\cos[2]{\phi} + \sigma_{y}\sin[2]{\phi} + 2\tau_{xy}\sin{\phi}\cos{\phi}, \\
	\tau   &= \tau_{xy}\cos{2\phi} + \frac{\sigma_{y} - \sigma_{x}}{2}\sin{2\phi}.
\end{align*}

\paragraph{Mohrs spänningscirkel}
Mohrs spänningscirkel är ett sätt att grafiskt ta fram plana spänningar vid rotation av ett plan. För att konstruera cirkeln, rita upp ett $\sigma, \tau$-koordinatsystem och två punkter $(sigma_{x}, \tau_{x, y})$ och $(sigma_{y}, -\tau_{x, y})$, där dessa tas från något givet tillstånd. Dessa punkter skall vara i motstående änder av cirkeln, och från detta kan cirkeln ritas. En rotation moturs av planet med en vinkel $\phi$ motsvarar en rotation medurs av tillståndet på cirkeln med en vinkel $2\phi$.

\paragraph{Jämvikt i tre dimensioner}
Snitta ut en liten kub. Jämvikt i $x$-riktning ger
\begin{align*}
	\sigma(\vb{x} + \dd{x}\vb{e}_{x})\dd{y}\dd{z} - \sigma(\vb{x})\dd{y}\dd{z} + \tau_{zx}(\vb{x} + \dd{z}\vb{e}_{z})\dd{x}\dd{y} - \tau_{zx}(\vb{x})\dd{x}\dd{y} + \tau_{yx}(\vb{x} + \dd{y}\vb{e}_{y})\dd{x}\dd{z} - \tau_{yx}(\vb{x})\dd{x}\dd{z} = 0,
\end{align*}
vilket implicerar
\begin{align*}
	\del{x}{\sigma_{x}} + \del{y}{\tau_{xy}} + \del{z}{tau_{zx}} = 0.
\end{align*}
