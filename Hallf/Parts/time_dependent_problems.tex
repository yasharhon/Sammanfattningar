\section{Tidsberoende problem}

\paragraph{Diskreta problem}
Betrakta ett problem där vi har $n$ punktmassor på en linje som rör sig normalt på linjens riktning, och har en förskjutning $w_{i}$ från jämviktsläget. Newtons andra lag ger
\begin{align*}
	m_{i}\ddot{w}_{i} = P_{i} - F_{i}.
\end{align*}
Här är $P_{i}$ den yttre kraften massa $i$ utsätts för och $F_{i}$ en sorts motståndsterm. Denna kan till exempel uppstå på grund av styvhet, och blir då på formen
\begin{align*}
	F_{i} = K_{ij}w_{j}.
\end{align*}
Problemet kan formuleras på matrisform som
\begin{align*}
	M\ddot{\vb{w}} + K\vb{w} = \vb{P}.
\end{align*}

Det homogena problemet kan lösas med en ansats $\vb{w} = \vb{a}\sin{\omega t}$. Detta ger upphov till ett egenvärdesproblem i egenfrekvenserna $\omega$.

Notera att 
\begin{align*}
	F_{i} = K_{ij}w_{j}.
\end{align*}
ger matrisrelationen
\begin{align*}
	\vb{F} = K\vb{w},
\end{align*}
där $K$ är styvhetsmatrisen. Vi hade alternativt kunnat ställa upp flexibilitetsmatrisen $\alpha$ för att få
\begin{align*}
	\vb{w} = \alpha\vb{F}.
\end{align*}
Det visar sig att både styvhetsmatrisen och flexibilitetsmatrisen är symmetriska.