\section{Finita elementmetoden}

Finita elementmetoden (FEM) är en metod för att numeriskt lösa partiella differentialekvationer. Den kan tillämpas på problem inom allt från strömningmekanik och hållfasthetslära till kvantmekanik.

Grunden för FEM är att lösa problem med hjälp av potentiella energins minimum. Vi kunde formulera en potential $U = W_{\text{e}} - A$, där
\begin{align*}
	W_{\text{e}} = \integ{0}{l}{x}{\frac{1}{2}EA\left(\dv{u}{x}\right)^{2}}
\end{align*}
för en balk och
\begin{align*}
	A = \integ[3]{0}{l}{x}{K} + \sum F_{i}u(x_{i}
\end{align*}
är bidraget från yttre krafter. För att hitta en approximativ lösning, gör man ansatsen
\begin{align*}
	u = \sum a_{i}f_{i},
\end{align*}
där alla $f_{i}$ är kända. Nu kommer potentialen bero av de olika koefficienterna, och att minimera potentiella energin innebär nu att hitta koefficienter i summan så att potentiella energin minimeras. I FEM använder man speciella val av såna ansatsfunktioner.

Mer specifikt diskretiserar man först problemet, och skriver
\begin{align*}
	u(x) = \sum d_{i}N_{i}(x)
\end{align*}
där $d_{i}$ är förskjutningen i den $i$:te punkten och $N_{i}$ är formfunktioner. Ett enkelt val av formfunktioner är $N_{i}(x_{j}) = \delta_{ij}$. Denna summan kan man alternativt skriva som en skalärprodukt $u = Nd = d^{T}N^{T}$, där $N$ är en radvektor med alla $N_{i}$ och $d$ en kolumnvektor med alla $d_{i}$. Vi måste även införa $B = \dv{N}{x}$ Nu kan vi skriva potentiella energin som
\begin{align*}
	U &= \integ{0}{l}{x}{\frac{1}{2}d^{T}B^{T}EABd - K_{x}d^{T}N^{T}} - \sum F_{i}d^{T}N^{T} \\
	  &= \frac{1}{2}d^{T}Kd - d^{T}F
\end{align*}
där vi har infört styvhetsmatrisen
\begin{align*}
	K = \integ{0}{l}{x}{\frac{1}{2}B^{T}EAB}
\end{align*}
och matrisen
\begin{align*}
	F = \integ{0}{l}{x}{K_{x}N^{T}} - \sum F_{i}N^{T}.
\end{align*}
Potentiella energins minimum ges då av
\begin{align*}
	Kd - F = 0.
\end{align*}