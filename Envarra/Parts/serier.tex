\section{Serier}

\subsection{Definitioner}

\paragraph{Delsummor}
Låt $\left(a_i\right)_{i = 1}^\infty$ vara en talföljd. Den motsvarande delsumman är
\begin{align*}
	s_n = \sum\limits_{i = 1}^{n}a_i.
\end{align*}

\paragraph{Serier}
En serie definieras som
\begin{align*}
	\sum\limits_{i = 1}^{\infty}a_i = \limit{n}{\infty}s_n.
\end{align*}

\paragraph{Konvergens}
Om $\limit{n}{\infty}s_n$ existerar, är serien konvergent mot dens summa. Annars är den divergent.

\paragraph{Geometriska serier}
En geometrisk serie är på formen $a_i = x^i$.

\paragraph{Absolut konvergens}
Serien $\sum\limits_{i = 1}^{\infty}a_i$ är absolutt konvergent om $\sum\limits_{i = 1}^{\infty}\abs{a_i}$ är konvergent.

\paragraph{Taylorserier}
Låt $f$ vara oändligt deriverbar. Funktionens Taylorserie kring $a$ är
\begin{align*}
	s = \sum\limits_{i = 1}^{\infty}\frac{\dv[i]{f}{x}(a)}{i!}(x - a)^i.
\end{align*}

\paragraph{Konvergensradie}
Enligt ekvation \ref{eq:taylor_formula} är
\begin{align*}
	f(x) - p_{n-1}(x) = R_n(x) = \frac{\dv[n]{f}{x} (\alpha)}{n!}(x - a)^n.
\end{align*}
$f$ konvergerar mot sin Taylorserie om denna resttermen går mot $0$ när $n\to\infty$ för ett givet $x$. Detta händer för $x$ så att $\abs{x -a} < r$, där $r$ är Taylorseriens konvergensradie.

\subsection{Satser}

\paragraph{Seriers egenskaper}
Låt $\sum\limits_{i = 1}^{\infty}a_i, \sum\limits_{i = 1}^{\infty}b_i$ vara två konvergenta serier. Då gäller
\begin{align*}
	&\sum\limits_{i = 1}^{\infty}(a_i + b_i) = \sum\limits_{i = 1}^{\infty}a_i + \sum\limits_{i = 1}^{\infty}b_i, \\
	&\sum\limits_{i = 1}^{\infty}ca_i = c\sum\limits_{i = 1}^{\infty}a_i, c\in\R.
\end{align*}

\proof

\paragraph{Konvergens och termernas beteende}
Om $\sum\limits_{i = 1}^{\infty}a_i$ är konvergent är $\limit{i}{\infty}a_i = 0$.

\proof
Låt $s_n$ beteckna seriens delsumma och $S$ dens summa. Vi har
\begin{align*}
	a_n = s_n - s_{n - 1}.
\end{align*}
Om serien är konvergent, kan vi räkna ut gränsvärdet enligt
\begin{align*}
	\limit{n}{\infty}a_n = \limit{n}{\infty}(s_n - s_{n - 1}) = S - S = 0.
\end{align*}

\paragraph{Summan av en geometrisk serie}
Om $\abs{x} < 1$ är
\begin{align*}
	\sum\limits_{i = 1}^{\infty}x^i = \frac{1}{1 - x}.
\end{align*}

\proof
Betrakta $s_n - xs_n = 1 - x^{n + 1}$. Detta ger
\begin{align*}
	\sum\limits_{i = 1}^{n}x^i = \frac{1 - x^{n + 1}}{1 - x}.
\end{align*}
Om $\abs{x} < 1$ har man
\begin{align*}
	\sum\limits_{i = 1}^{\infty}x^i = \frac{1}{1 - x}.
\end{align*}

\paragraph{Jamförelse av termer och konvergens}
Låt $0\leq a_i\leq b_i$ för alla $i$. Då gäller att
\begin{itemize}
	\item om $\sum\limits_{i = 1}^{\infty}b_i$ är konvergent är $\sum\limits_{i = 1}^{\infty}a_i$ konvergent.
	\item om $\sum\limits_{i = 1}^{\infty}a_i$ är divergent är $\sum\limits_{i = 1}^{\infty}b_i$ divergent.
\end{itemize}

\proof

\paragraph{Kvoten av termer och konvergens}
Låt $\sum\limits_{i = 1}^{\infty}a_i, \sum\limits_{i = 1}^{\infty}b_i$ vara två positiva serier vars termer uppfyller
\begin{align*}
	\limit{i}{\infty}\frac{a_i}{b_i} = K\neq 0.
\end{align*}
Då konvergerar $\sum\limits_{i = 1}^{\infty}a_i$ om och endast om $\sum\limits_{i = 1}^{\infty}b_i$ konvergerar.

\proof

\paragraph{Absolut konvergens och konvergens}
En absolut konvergent serie är konvergent.

\proof

\paragraph{Summan av potenser}
Serien
\begin{align*}
	\sum\limits_{i = 1}^{\infty}\frac{1}{i^p}
\end{align*}
är konvergent om och endast om $p > 1$.

\proof