\section{Integraler}

\subsection{Definitioner}

\paragraph{Trappfunktioner}
En trappfunktion på intervallet $[a, b]$ är på formen
\begin{align*}
	\Psi(x) =
	\begin{cases}
		c_1, a\leq x\leq x_1 \\
		c_2, x_1\leq x\leq x_2 \\
		\vdots \\
		c_n, x_{n - 1}\leq x\leq x_n \\
	\end{cases}
\end{align*}
Mängden av alla $x_i$ kallas en uppdelning av intervallet och intervallerna $[x_{i - 1}, x_i]$ kallas delintervall av uppdelningen.

\paragraph{Integralen av en trappfunktion}
Låt $\Psi$ vara en trappfunktion. Då definieras integralen av denna som
\begin{align*}
	\integ{a}{b}{\Psi(x)}{x} = \sum\limits_{i = 1}^{n}c_i(x_i - x_{i - 1}).
\end{align*}

\paragraph{Övertrappor och undertrappor}
En övertrappa $\Psi$ för en funktion $f$ är en funktion så att
\begin{align*}
	f(x)\leq \Psi(x).
\end{align*}
Undertrappor definieras analogt. Integralerna av dessa kallas översummor och undersummor.

\paragraph{Integrerbarhet}
Låt $f$, definierad på $[a, b]$, vara en begränsad funktion, $L(f)$ vara mängden av alla undersummor till $f$ och $U(f)$ mängden av alla översummor till $f$. $L(f)$ är uppåt begränsad av talen i $U(f)$ och vice versa, så $\sup{L(f)}, \inf{U(f)}$ existerar. Om
\begin{align*}
	\sup{L(f)} = \inf{U(f)}
\end{align*}
är $f$ integrerbar.

\paragraph{Integralen}
Låt $f$ vara integrerbar på $[a, b]$. Då definieras integralen av $f$ på intervallet som
\begin{align*}
	\integ{a}{b}{f(x)}{x} = \sup{L(f)}.
\end{align*}

\paragraph{Byte av ordning av integrationsgränser}
\begin{align*}
	\integ{a}{b}{f(x)}{x} = -\integ{b}{a}{f(x)}{x}.
\end{align*}

\paragraph{Integration med oändliga gränser}
Låt $f$ vara integrerbar på $[a, R]$ för varje $R > a$. Då definieras
\begin{align*}
	\integ{a}{\infty}{f(x)}{x} = \limit{R}{\infty}\integ{a}{R}{f(x)}{x},
\end{align*}
och integration från $-\infty$ analogt. Om gränsvärdet existerar, är integralet konvergent. Annars är det divergent.

\paragraph{Lokala integraler}
Låt $f: (a, b]\to\R$ vara integrerbar i $[a + \varepsilon, b]$, för varje litet $\varepsilon$. Då är
\begin{align*}
	\integ{a}{b}{f(x)}{x} = \limit{\varepsilon}{0}\integ{a + \varepsilon}{b}{f(x)}{x},
\end{align*}
och analogt om $f$ ej är definierad i $b$ eller i någon inre punkt av $[a, b]$. Om sådana gränsvärden existerar är integralen konvergent. Annars är den divergent.

\paragraph{Riemannsummor}
Låt $P_n = \left\{x_{n, i}\right\}_{i = 0}^{N(n)}$ vara en uppdelning av $[a, b]$ för ett givet $n$ med $N(n)$ delintervall. Låt $\alpha_{n, i}\in [x_{n, i - 1}, x_{n, i}], \Delta_{n, i} = x_{n, i} - x_{n, i - 1}$. Summan
\begin{align*}
	\sum\limits_{i = 1}^{N(n)}f(\alpha_{n, i})\Delta_{n, i}
\end{align*}
är en Riemannsumma för $f$ i $[a, b]$.

\subsection{Satser}

\paragraph{Integralen och $\varepsilon$}
Låt $f$ vara begränsad på $[a, b]$. Då är $f$ integrerbar om och endast om det för varje $\varepsilon$ finns en övertrappa $\Psi$ och en undertrappa $\Phi$ till $f$ sådana att
\begin{align*}
	\integ{a}{b}{\Psi(x)}{x} - \integ{a}{b}{\Phi(x)}{x} < \varepsilon.
\end{align*}

\proof

\paragraph{Summor mot integraler}
Låt $f$ vara kontinuerlig på $[a, b]$, $\left\{x_i\right\}_{i = 0}^n$ vara en uppdelning, $\Delta_i = x_i - x_{i - 1}$ och $M_i = \max{f(x)}, m_i = \min{f(x)}$ på $[x_{i - 1}, x_i]$. Då gäller att
\begin{align*}
	&\sum\limits_{i = 0}^n M_i\Delta_i\to\integ{a}{b}{f(x)}{x}, \\
	&\sum\limits_{i = 0}^n m_i\Delta_i\to\integ{a}{b}{f(x)}{x}
\end{align*}
då $\max{\Delta_i}\to 0$.

\proof

\paragraph{Integralens egenskaper}
Låt $f$ vara integrerbar på $[a, b]$. Då gäller
\begin{align*}
	&\integ{a}{b}{(f(x) + g(x))}{x} = \integ{a}{b}{f(x)}{x} + \integ{a}{b}{g(x)}{x}, \\
	&\integ{a}{b}{cf(x)}{x} = c\integ{a}{b}{f(x)}{x}, \\
	&\integ{a}{b}{f(x)}{x} = \integ{a}{c}{f(x)}{x} + \integ{a}{b}{f(x)}{x}, \\
	&\abs{\integ{a}{b}{f(x)}{x}}\leq \integ{a}{b}{\abs{f(x)}}{x}.
\end{align*}
Om $f(x)\leq g(x)$ på $[a, b]$ gäller
\begin{align*}
	\integ{a}{b}{f(x)}{x}\leq \integ{a}{b}{g(x)}{x}.
\end{align*}

\proof

\paragraph{Medelvärdesatsen för integraler}
Låt $f, g$ vara kontinuerliga på $[a, b]$ och $g\geq 0$. Då finns det ett $\alpha\in (a, b)$ sådant att
\begin{align*}
	\integ{a}{b}{f(x)g(x)}{x} = f(\alpha)\integ{a}{b}{g(x)}{x}.
\end{align*}

\subparagraph{Specialfall}
Välj $g(x) = 1$. Då blir satsen
\begin{align*}
	\integ{a}{b}{f(x)}{x} = f(\alpha)(b - a).
\end{align*}

\proof

\paragraph{Analysens huvudsats}
Låt $f$ vara kontinuerlig på $[a, b]$. Då är
\begin{align*}
	F(x) = \integ{a}{x}{f(t)}{t}
\end{align*}
en primitiv funktion till $f$ på $[a, b]$.

\proof
\begin{align*}
	\frac{F(x + h) - F(x)}{h} = \frac{\integ{a}{x + h}{f(t)}{t} - \integ{a}{x}{f(t)}{t}}{h} = \frac{1}{h}\integ{x}{x + h}{f(t)}{t}.
\end{align*}
Enligt specialfallet av medelvärdesatsen finns ett $\alpha\in (x, x +h)$ sådant att
\begin{align*}
	\frac{1}{h}\integ{x}{x + h}{f(t)}{t} = \frac{x + h - x}{h}f(\alpha) = f(\alpha).
\end{align*}
När $h$ går mot $0$ går vänstersidan mot $\deval{F}{x}{x}$ och högersidan mot $f(x)$, och beviset är klart.

\paragraph{Primitiva funktioner och integralers värde}
Låt $f$ vara kontinuerlig på $[a, b]$ och låt $F$ vara en primitiv funktion till $f$ på $[a, b]$. Då är
\begin{align*}
	\integ{a}{b}{f(t)}{t} = F(b) - F(a).
\end{align*}

\proof
Låt $G(x) = F(a) + \integ{a}{x}{f(t)}{t}$, där $F$ är någon primitiv funktion till $f$. Enligt analysens huvudsats är $\dv{G}{x} = f$. Vi ser även att $G(x) = F(x) + C$. Vi jämförar de två uttryckerna vi har för $G$ i $a$ och ser att $C = 0$. Då är $F(b) = \integ{a}{b}{f(t)}{t} + F(a)$, och beviset är klart.

\paragraph{Partiell integration}
Låt $u, v, \dv{v}{x}$ vara kontinuerliga på $[a, b]$ och $v$ vara en primitiv funktion till $\dv{v}{x}$. Då gäller:
\begin{align*}
	\integ{a}{b}{u\deval{v}{x}{x}}{x} = uv(b) - uv(a) - \integ{a}{b}{v\deval{u}{x}{x}}{x}.
\end{align*}

\proof
Där $u, v$ är deriverbara har vi
\begin{align*}
	\dv{(uv)}{x} = u\dv{v}{x} + v\dv{u}{x}, \\
	u\dv{v}{x} = \dv{(uv)}{x} - v\dv{u}{x}.
\end{align*}
Båda sider är integrerbara enligt sats som inte finns i boken, så vi integrerar och användar analysens huvudsats:
\begin{align*}
	\integ{a}{b}{u\deval{v}{x}{x}}{x} &= \integ{a}{b}{\deval{(uv)}{x}{x}}{x} - \integ{a}{b}{v\deval{u}{x}{x}}{x} \\
	                                  &= uv(b) - uv(a) - \integ{a}{b}{v\deval{u}{x}{x}}{x}.
\end{align*}

\paragraph{Variabelbyte}
Låt $g, \dv{g}{x}$ vara kontinuerliga i $[a, b]$ och låt $f$ vara kontinuerlig i $(g(a), g(b))$. Då gäller:
\begin{align*}
	\integ{a}{b}{f(g(x))\deval{g}{x}{x}}{x} = \integ{g(a)}{g(b)}{f(x)}{x}.
\end{align*}

\proof
Enligt kedjeregeln har vi
\begin{align*}
	\deval{(F\circ g)}{x}{x} = f(g(x))\deval{g}{x}{x},
\end{align*}
där $F$ är en primitiv funktion till $f$. Vänstersidan är integrerbar eftersom $f$ är kontinuerlig, och högersidan är integrerbar eftersom $\dv{g}{x}$ även är kontinuerlig, så vi integrerar på båda sidor.
\begin{align*}
	\integ{a}{b}{\deval{F\circ g}{x}{x}}{x} = \integ{a}{b}{f(g(x))\deval{g}{x}{x}}{x}.
\end{align*}
Vänstersidan är
\begin{align*}
	\integ{a}{b}{\deval{(F\circ g)}{x}{x}}{x} = F(g(b)) - F(g(a)) = \integ{g(a)}{g(b)}{f(x)}{x},
\end{align*}
och beviset är klart.

\paragraph{Integration av potenser över obegränsade intervall}
Integralen
\begin{align*}
	\integ{1}{\infty}{\frac{1}{x^p}}{x}
\end{align*}
är konvergent om och endast om $p > 1$.

\proof

\paragraph{Jämförelsessats för två integraler}
Låt $f, g$ vara integrerbara i $[a, R]$ för alla $R > a$, sådana $0\neq f(x)\neq g(x)$ för varje $x > a$. Då gäller att om $\integ{a}{\infty}{g(x)}{x}$ är konvergent är även $\integ{a}{\infty}{f(x)}{x}$ konvergent, och om $\integ{a}{\infty}{f(x)}{x}$ är divergent är även $\integ{a}{\infty}{g(x)}{x}$ divergent.

\proof
Övning till läsaren.

\paragraph{Gräns av kvot och integral}
Låt $f, g$ vara positiva och integrerbara i $[a, R]$ för alla $R > a$ sådana att
\begin{align*}
	\limit{x}{\infty}\frac{f(x)}{g(x)} = K\neq 0.
\end{align*}
Då är $\integ{a}{\infty}{g(x)}{x}$ konvergent om och endast om $\integ{a}{\infty}{f(x)}{x}$ är konvergent.

\proof
Övning till läsaren.

\paragraph{Samband mellan summor och integraler}
Låt $f$ vara en avtagande funktion i $[m, n]$, där $m, n$ är heltal sådana att $m < n$. Då gäller:
\begin{align*}
	\sum\limits_{i = m + 1}^{n}f(i)\leq \integ{m}{n}{f(x)}{x}\leq \sum\limits_{i = m}^{n - 1}f(i).
\end{align*}

Detta kan omformuleras till
\begin{align*}
	\integ{m}{n}{f(x)}{x} + f(n)\leq \sum\limits_{i = m}^{n}f(i)\leq \integ{m}{n}{f(x)}{x} + f(m).
\end{align*}

Låt $f$ vara en växande funktion i $[m, n]$, där $m, n$ är heltal sådana att $m < n$. Då gäller:
\begin{align*}
	\sum\limits_{i = m}^{n - 1}f(i)\leq \integ{m}{n}{f(x)}{x}\leq \sum\limits_{i = m + 1}^{n}f(i).
\end{align*}

Detta kan omformuleras till
\begin{align*}
	\integ{m}{n}{f(x)}{x} + f(m)\leq \sum\limits_{i = m}^{n}f(i)\leq \integ{m}{n}{f(x)}{x} + f(n).
\end{align*}

\proof
De två summorna i det första sättet av olikheter representerar areaor av $n - m - 1$ rektangler med brädd $1$ jämnt fördelade under grafen till $f$ på intervallet. I det avtagande fallet bestämms höjden till rektanglerna i den första summan av funktionsvärdet vid rektanglets högra kant, medan höjden till rektanglerna i den andra summan bestämms av funktionsvärdet vid rektanglets vänstra kant. Eftersom $f$ är avtagande är den andra summan störra. Eftersom $f$ är avtagande, har integralet av $f$ på intervallet ett värde mellan dessa två summorna. Den andra olikheten fås vid att betrakta de två första och två sista delarna av den första olikheten och addera en lämplig term för att få samma summa. Beviset är analogt i det växande fallet.

\paragraph{Cauchys integralkriterium}
Låt $f$ vara positiv och avtagande i $(m, \infty)$. Då är $\sum\limits_{i = m}^{\infty}f(i)$ konvergent om och endast om $\integ{m}{\infty}{f(x)}{x}$ är konvergent.

\proof

\paragraph{Integral av potenser över $0$}
Integralen
\begin{align*}
	\integ{0}{1}{\frac{1}{x^q}}{x}
\end{align*}
är konvergent om och endast om $q < 1$.

\proof

\paragraph{Jämförelsessatser för lokala integraler}
Låt $f, g$ vara integrerbara i $[a + \varepsilon, b]$ för alla $\varepsilon > 0$, sådana att $0\neq f(x)\neq g(x)$ för varje $x\in (a, b]$. Då gäller att om $\integ{a}{b}{g(x)}{x}$ är konvergent är även $\integ{a}{b}{f(x)}{x}$ konvergent, och om $\integ{a}{b}{f(x)}{x}$ är divergent är även $\integ{a}{b}{g(x)}{x}$ divergent.

\proof
Övning till läsaren.

\paragraph{Gräns av kvot och lokalt integral}
Låt $f, g$ vara positiva och integrerbara i $[a + \varepsilon, b]$ för alla $\varepsilon > 0$ sådana att
\begin{align*}
	\limit{x}{\infty}\frac{f(x)}{g(x)} = K\neq 0.
\end{align*}
Då är $\integ{a}{b}{g(x)}{x}$ konvergent om och endast om $\integ{a}{b}{f(x)}{x}$ är konvergent.

\proof
Övning till läsaren.

\paragraph{Riemannsummor och integraler}
Låt $f$ vara kontinuerlig på $[a, b]$ och låt $(P_n)_{n = 1}^{\infty}$ vara en följd av uppdelningar av intervallet sådana att
\begin{align*}
	\max{\{\Delta_{n, i}:\ 1\leq i\leq N(n)\}}\to 0
\end{align*}
då $n\to\infty$. Då gäller att
\begin{align*}
	\sum\limits_{i = 1}^{N(n)}f(\alpha_{n, i})\Delta_{n, i}\to \integ{a}{b}{f(x)}{x}
\end{align*}
då $n\to\infty$.

\proof