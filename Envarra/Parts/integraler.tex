\section{Integraler}

\subsection{Definitioner}

\paragraph{Trappfunktioner}
En trappfunktion på intervallet $[a, b]$ är på formen
\begin{align*}
	\Psi(x) =
	\begin{cases}
		c_1, a\leq x\leq x_1 \\
		c_2, x_1\leq x\leq x_2 \\
		\vdots \\
		c_n, x_{n - 1}\leq x\leq x_n \\
	\end{cases}
\end{align*}
Mängden av alla $x_i$ kallas en uppdelning av intervallet och intervallerna $[x_{i - 1}, x_i]$ kallas delintervall av uppdelningen.

\paragraph{Integralen av en trappfunktion}
Låt $\Psi$ vara en trappfunktion. Då definieras integralen av denna som
\begin{align*}
	\integ{a}{b}{\Psi(x)}{x} = \sum\limits_{i = 1}^{n}c_i{x_i - x_{i - 1}}.
\end{align*}

\paragraph{Övertrappor och undertrappor}
En övertrappa $\Psi$ för en funktion $f$ är en funktion så att
\begin{align*}
	f(x)\leq \Psi(x).
\end{align*}
Undertrappor definieras analogt. Integralerna av dessa kallas översummor och undersummor.

\paragraph{Integrerbarhet}
Låt $f$, definierad på $[a, b]$, vara en begränsad funktion, $L(f)$ vara mängden av alla undersummor till $f$ och $U(f)$ mängden av alla översummor till $f$. $L(f)$ är uppåt begränsad av talen i $U(f)$ och vice versa, så $\sup{L(f)}, \inf{U(f)}$ existerar. Om
\begin{align*}
	\sup{L(f)} = \inf{U(f)}
\end{align*}
är $f$ integrerbar.

\paragraph{Integralen}
Låt $f$ vara integrerbar på $[a, b]$. Då definieras integralen av $f$ på intervallet som
\begin{align*}
	\integ{a}{b}{f(x)}{x} = \sup{L(f)}.
\end{align*}

\paragraph{Byte av integrationsgränser}
\begin{align*}
	\integ{a}{b}{f(x)}{x} = -\integ{b}{a}{f(x)}{x}.
\end{align*}

\subsection{Satser}

\paragraph{Integralen och $\varepsilon$}
Låt $f$ vara begränsad på $[a, b]$. Då är $f$ integrerbar om och endast om det för varje $\varepsilon$ finns en övertrappa $\Psi$ och en undertrappa $\Phi$ till $f$ sådana att
\begin{align*}
	\integ{a}{b}{\Psi(x)}{x} - \integ{a}{b}{\Phi(x)}{x} < \varepsilon.
\end{align*}

\proof

\paragraph{Summor mot integraler}
Låt $f$ vara kontinuerlig på $[a, b]$, $\left\{x_i\right\}_{i = 0}^n$ vara en uppdelning, $\Delta_i = x_i - x_{i - 1}$ och $M_i = \max{f(x)}, m_i = \min{f(x)}$ på $[x_{i - 1}, x_i]$. Då gäller att
\begin{align*}
	&\sum\limits_{i = 0}^n M_i\Delta_i\to\integ{a}{b}{f(x)}{x}, \\
	&\sum\limits_{i = 0}^n m_i\Delta_i\to\integ{a}{b}{f(x)}{x}
\end{align*}
då $\max{\Delta_i}\to 0$.

\proof

\paragraph{Integralens egenskaper}
Låt $f$ vara integrerbar på $[a, b]$. Då gäller
\begin{align*}
	&\integ{a}{b}{(f(x) + g(x))}{x} = \integ{a}{b}{f(x)}{x} + \integ{a}{b}{g(x)}{x}, \\
	&\integ{a}{b}{cf(x)}{x} = c\integ{a}{b}{f(x)}{x}, \\
	&\integ{a}{b}{f(x)}{x} = \integ{a}{c}{f(x)}{x} + \integ{a}{b}{f(x)}{x}, \\
	&\abs{\integ{a}{b}{f(x)}{x}}\leq \integ{a}{b}{\abs{f(x)}}{x}.
\end{align*}
Om $f(x)\leq g(x)$ på $[a, b]$ gäller
\begin{align*}
	\integ{a}{b}{f(x)}{x}\leq \integ{a}{b}{g(x)}{x}.
\end{align*}

\proof

\paragraph{Medelvärdesatsen för integraler}
Låt $f, g$ vara kontinuerliga på $[a, b]$ och $g\geq 0$. Då finns det ett $\alpha\in (a, b)$ sådant att
\begin{align*}
	\integ{a}{b}{f(x)g(x)}{x} = f(\alpha)\integ{a}{b}{g(x)}{x}.
\end{align*}

\subparagraph{Specialfall}
Välj $g(x) = 1$. Då blir satsen
\begin{align*}
	\integ{a}{b}{f(x)}{x} = f(\alpha)(b - a).
\end{align*}

\proof

\paragraph{Analysens huvudsats}
Låt $f$ vara kontinuerlig på $[a, b]$. Då är
\begin{align*}
	F(x) = \integ{a}{x}{f(t)}{t}
\end{align*}
en primitiv funktion till $f$ på $[a, b]$.

\proof

\paragraph{Primitva funktioner och integralers värde}
Låt $f$ vara kontinuerlig på $[a, b]$ och låt $F$ vara en primitiv funktion till $f$ på $[a, b]$. Då är
\begin{align*}
	\integ{a}{b}{f(t)}{t} = F(b) - F(a).
\end{align*}

\proof