\section{Derivata}

\subsection{Definitioner}

\paragraph{Derivatans definition}
Låt $f$ vara en funktion definierad i en omgvning krin $x_0$. $f$ är deriverbar i $x_0$ om
\begin{align*}
	\eval{\dv{f}{x}}_{x_0} &= \dv{f}{x} (x_0) = f'(x_0) \\
	                           &= \lim\limits_{h\to 0}\frac{f(x_0 + h) - f(x_0)}{h}
\end{align*}
existerar. Värdet kallas derivatan i $x_0$.

\paragraph{Deriverbara funktioner}
Om en funktion $f$ är deriverbar i alla punkter i definitionsmängden, är funktionen deriverbar. Funktionen $f' = \dv{f}{x}$ med $D_{f'} = D_f$ kallas derivatan.

\paragraph{Minima och maxima}
En funktion $f$ har ett lokalt maximum i $x_0$ om det finns en omgivning $I$ till $x_0$ så att $f(x)\leq f(x_0)$ för alla $x\in I\cap D_f$. Det analoga gäller för ett lokalt minimum. Om $f$ har antingen ett lokalt maximum eller minimum i $x_0$ har $f$ ett lokalt extrempunkt i $f$.

\paragraph{Stationära punkt}
En funktion $f$ har ett stationärt punkt $x_0$ om $\dv{f}{x}\eval_{x_0} = 0$.

\paragraph{Inflexionspunkt}
Låt $f$ vara en funktion definierad på ett intervall $I$. En punkt $x_0\in I$ sägs vara en inflexionspunkt till $f$ om det finns ett $\delta > 0$ sådan att $f$ är konvex i $[x_0 - \delta, x_0]$ eller $[x_0, x_0 + \delta]$ och konkav i det andra.

\paragraph{Globala maxima och minima}
En funktion $f$ har ett globalt maximum i $x_0$ om $f(x)\leq f(x_0$ för varje $x\in D_f$.

\paragraph{Taylorpolynomet}
Låt $f$ vara $n$ gånger deriverbar. Polynomet
\begin{align*}
	p_n(x) = \sum\limits_{i = 0}{n}\frac{\dv[i]{f}{x} (0)}{i!}(x - a)^i
\end{align*}
kallas Taylorpolynomet av grad $n$ till $f$ kring $a$. Specialfallet där $a = 0$ kallas Maclaurinpolynomet till $f$ av grad $n$.

\subsection{Satser}

\paragraph{Derivata och kontinuitet}
Låt $f$ vara deriverbar i $(a, b)$. Då är $f$ kontinuerlig i $(a, b)$.

\proof
Kan man tänka.

\paragraph{Derivationsregler}
Låt $f, g$ vara deriverbara i punkten $x$. Då följer att $f + g, fg$ är deriverbara i $x$. Derivatorna har sambandet
\begin{align*}
	\eval{\dv{x}(f + g)}_x &= \left(\dv{f}{x} + \dv{g}{x}\right)\eval_x, \\
	\eval{\dv{x}(af)}_x    &= a\dv{f}{x}\eval_x, a\in\R, \\
	\eval{\dv{x}(fg)}_x    &= \left(f\dv{g}{x} + g\dv{f}{x}\right)\eval_x. \\
\end{align*}
Om $g(x)\neq 0$ är även $\frac{f}{g}$ deriverbar i $x$ och
\begin{align*}
	\dv{x}\frac{f}{g}\eval_x = \frac{\left(g\dv{f}{x} - f\dv{g}{x}\right)\eval_x}{g^2(x)}.
\end{align*}

\proof
Inte omöjligt.

\paragraph{Kedjeregeln}
Låt $f$ vara deriverbar i $y$, $g$ deriverbar i $x$ och $y = g(x)$. Då är den sammansatta funktionen $f\circ g$ deriverbar och
\begin{align*}
	\dv{x}(f\circ g)\eval_x = \dv{x}f\eval_{g(x)}\cdot\dv{x}g\eval_x.
\end{align*}

\proof

\paragraph{Derivatan av inversfunktioner}
Låt $f$ vara en deriverbar och inverterbar funktion. Då är inversen $f^{-1}$ deriverbar i alla punkter $y = \dv{x}f\eval_x$ där $\dv{x}f\eval_x\neq 0$ med derivatan
\begin{align*}
	\dv{y}f^{-1}\eval_y = \frac{1}{\dv{x}f\eval_x}.
\end{align*}

\proof

\paragraph{Extrempunkt och derivata}
Låt $f$ vara deriverbar i $x_0$ och ha en lokal extrempunkt i $x_0$. Då är $\dv{f}{x}\eval_{x_0} = 0$.

\proof

\paragraph{Rolles sats}
Låt $f$ vara kontinuerlig på $[a, b]$, deriverbar på $(a, b)$ så att $f(a) = f(b)$. Då existerar $p\in (a, b)$ så att $\dv{f}{x}\eval_p = 0$.

\proof

\paragraph{Generaliserade medelvärdessatsen}
Låt $f$ och $g$ vara reellvärda, kontinuerliga på $[a, b]$ och deriverbara på $(a, b)$. Då existerar $p\in (a, b)$ så att
\begin{align*}
	\dv{f}{x}\eval_p(g(b) - g(a)) = \dv{g}{x}\eval_p(f(b) - f(a)).
\end{align*}
Om $g(a)\neq g(b)$ och $\dv{g}{x}\eval_p\neq 0$, gäller
\begin{align*}
	\frac{\dv{f}{x}\eval_p}{\dv{g}{x}\eval_p} = \frac{f(b) - f(a)}{g(b) - g(a)}.
\end{align*}

\subparagraph{Medelvärdesatsen}
Välj $g(x) = x$. Detta ger
\begin{align*}
	\dv{f}{x}\eval_p(b - a) = f(b) - f(a).
\end{align*}

\proof
Använd Rolles sats.

\paragraph{Följder av dessa satser}
Låt $f$ vara deriverbar på $(a, b)$. Då gäller:
\begin{itemize}
	\item $\dv{f}{x}(x) = 0$ för varje $x\in (a, b)$ omm $f$ är konstant på $(a, b)$.
	\item $\dv{f}{x}(x)\geq 0$ för varje $x\in (a, b)$ omm $f$ är växande på $(a, b)$.
	\item $\dv{f}{x}(x) > 0$ implicerar att $f$ är strängt växande på $(a, b)$.
	\item $\dv{f}{x}(x)\leq 0$ för varje $x\in (a, b)$ omm $f$ är avtagande på $(a, b)$.
	\item $\dv{f}{x}(x) < 0$ implicerar att $f$ är strängt avtagande på $(a, b)$.
\end{itemize}

\proof

\paragraph{L'Hôpitals regel}
Låt $f, g$ vara reellvärda, deriverbara funktioner i en omgivning $I$ av $a$ sådana att
\begin{align*}
	\lim\limits_{x\to a}f(x) = \lim\limits_{x\to a}g(x) = 0.
\end{align*}
Då gäller att
\begin{align*}
	\lim\limits_{x\to a}\frac{f(x)}{g(x)} = \lim\limits_{x\to a}\frac{\dv{f}{x} (x)}{\dv{g}{x} (x)}.
\end{align*}

\proof

\paragraph{Oändliga kvoter}
Låt
\begin{align*}
	\lim\limits_{x\to a}\frac{\dv{f}{x} (x)}{\dv{g}{x} (x)} &= L, \\
	\limit{x}{a}f(x)                                        &= \pm\infty, \\
	\limit{x}{a}g(x)                                        &= \pm\infty.
\end{align*}
Då gäller att
\begin{align*}
	\limit{x}{a}\frac{f(x)}{g(x)} = L.
\end{align*}

\proof

\paragraph{Konvexitet och derivata}
Låt $f$ vara deriverbar i $(a, b)$. Då är $f$ konvex i $(a, b)$ omm $\dv{f}{x}$ är växande i $(a, b)$.

\proof

\paragraph{Andrederivata och konvexitet}
Låt $f$ vara två gånger deriverbar i $(a, b)$. Då är $\dv[2]{f}{x} (x)\geq 0$ för varje $x\in (a, b)$ omm $f$ är konvex.

\proof

\paragraph{Andrederivata och inflexionspunkt}
Låt $f$ vara två gånger deriverbar och låt $\dv[2]{f}{x}$ vara kontinuerlig. Om $f$ har en inflexionspunkt i $x_0$ så är $\dv[2]{f}{x} (x_0) = 0$.

\proof

\paragraph{Taylors formel}
Låt $f$ vara $n$ gånger deriverbar och definierad i en omgivning av $0$, sådan att $\dv[n]{f}{x}$ är kontinuerlig. Då är
\begin{align*}
	f(x) = \sum\limits_{i = 0}^{n - 1}\frac{\dv[i]{f}{x} (0)}{i!}x^i + \frac{\dv[n]{f}{x} (\alpha)}{n!}x^n
\end{align*}
för något $\alpha\in [0, x]$. Kring en godtycklig punkt $a$ blir formeln
\begin{align*}
	f(x) = \sum\limits_{i = 0}^{n - 1}\frac{\dv[i]{f}{x} (a)}{i!}(x - a)^i + \frac{\dv[n]{f}{x} (\alpha)}{n!}(x - a)^n
\end{align*}
för något $\alpha\in (a, x)$.

\proof

\paragraph{Taylors formel och stora ordo}
Låt $f$ vara $n$ gånger deriverbar och $\dv[n]{f}{x}$ vara kontinuerlig i en omgivning av $0$. Då är
\begin{align*}
	f(x) = \sum\limits_{i = 0}^{n - 1}\frac{\dv[i]{f}{x} (0)}{i!}x^i + \Ordo{x^n}.
\end{align*}

\proof