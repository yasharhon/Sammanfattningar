\section{Derivata}

\subsection{Definitioner}

\paragraph{Derivatans definition}
Låt $f$ vara en funktion definierad i en omgvning krin $x_0$. $f$ är deriverbar i $x_0$ om
\begin{align*}
	\eval{\dv{f}{x}}_{x_0} &= \dv{f}{x} (x_0) = f'(x_0) \\
	                           &= \lim\limits_{h\to 0}\frac{f(x_0 + h) - f(x_0)}{h}
\end{align*}
existerar. Värdet kallas derivatan i $x_0$.

\paragraph{Deriverbara funktioner}
Om en funktion $f$ är deriverbar i alla punkter i definitionsmängden, är funktionen deriverbar. Funktionen $f' = \dv{f}{x}$ med $D_{f'} = D_f$ kallas derivatan.

\paragraph{Stationära punkt}
En funktion $f$ har ett stationärt punkt $x_0$ om $\dv{f}{x}\eval_{x_0} = 0$.

\paragraph{Taylorpolynomet}
Låt $f$ vara $n$ gånger deriverbar. Polynomet
\begin{align*}
	p_n(x) = \sum\limits_{i = 0}{n}\frac{\dv[i]{f}{x} (0)}{i!}(x - a)^i
\end{align*}
kallas Taylorpolynomet av grad $n$ till $f$ kring $a$. Specialfallet där $a = 0$ kallas Maclaurinpolynomet till $f$ av grad $n$.

\paragraph{Primitiva funktioner}
Låt $f$ vara definierad på $[a, b]$ och $F$ vara kontinuerlig på $[a, b]$. $F$ är en primitiv funktion till $f$ om $\deval{F}{x}{x} = f(x)$ för varje $x\in (a, b)$.

\subsection{Satser}

\paragraph{Derivata och kontinuitet}
Låt $f$ vara deriverbar i $(a, b)$. Då är $f$ kontinuerlig i $(a, b)$.

\proof
Kan man tänka.

\paragraph{Derivationsregler}
Låt $f, g$ vara deriverbara i punkten $x$. Då följer att $f + g, fg$ är deriverbara i $x$. Derivatorna har sambandet
\begin{align*}
	&\deval{(f + g)}{x}{x} = \deval{f}{x}{x} + \deval{g}{x}{x}, \\
	&\deval{(af)}{x}{x} = a\deval{f}{x}{x}, a\in\R, \\
	&\deval{(fg)}{x}{x} = f(x)\deval{g}{x}{x} + g(x)\deval{f}{x}{x}. \\
\end{align*}
Om $g(x)\neq 0$ är även $\frac{f}{g}$ deriverbar i $x$ och
\begin{align*}
	\dv{x}\frac{f}{g}\eval_x = \frac{\left(g\dv{f}{x} - f\dv{g}{x}\right)\eval_x}{g^2(x)}.
\end{align*}

\proof
De två första följer nästan direkt från definitionen.

\begin{align*}
	                 &\frac{f(x + h)g(x + h) - f(x)g(x)}{h} \\
	                 &= \frac{f(x + h)g(x + h) - f(x + h)g(x) + f(x + h)g(x) - f(x)g(x)}{h} \\
	                 &= \left(\frac{f(x + h)g(x + h) - f(x + h)g(x)}{h} + \frac{f(x + h)g(x) - f(x)g(x)}{h}\right) \\
	\deval{fg}{x}{x} &= \limit{h}{0}\frac{f(x + h)g(x + h) - f(x)g(x)}{h} \\
	                 &= \limit{h}{0}\left(\frac{f(x + h)g(x + h) - f(x + h)g(x)}{h} + \frac{f(x + h)g(x) - f(x)g(x)}{h}\right) \\
	                 &= f(x)\deval{g}{x}{x} + g(x)\deval{f}{x}{x}.
\end{align*}

\paragraph{Kedjeregeln}
Låt $f$ vara deriverbar i $y$, $g$ deriverbar i $x$ och $y = g(x)$. Då är den sammansatta funktionen $f\circ g$ deriverbar och
\begin{align*}
	\dv{x}(f\circ g)\eval_x = \dv{x}f\eval_{g(x)}\cdot\dv{x}g\eval_x.
\end{align*}

\proof

\paragraph{Derivatan av inversfunktioner}
Låt $f$ vara en deriverbar och inverterbar funktion. Då är inversen $f^{-1}$ deriverbar i alla punkter $y = \dv{x}f\eval_x$ där $\dv{x}f\eval_x\neq 0$ med derivatan
\begin{align*}
	\dv{y}f^{-1}\eval_y = \frac{1}{\dv{x}f\eval_x}.
\end{align*}

\proof

\paragraph{Extrempunkt och derivata}
Låt $f$ vara deriverbar i $x_0$ och ha en lokal extrempunkt i $x_0$. Då är $\dv{f}{x} (x_0) = 0$.

\proof
Låt $f$ ha ett maximum i $x_0$. Detta ger $f(x_0)\geq f(x)$ i en omgivning till $x_0$. Betrakta
\begin{align*}
	\dv{f}{x} (x_0) = \limit{h}{0}\frac{f(x_0 + h) - f(x_0)}{h}.
\end{align*}
När $h\to 0$ från det positival hållet har man
\begin{align*}
	\frac{f(x_0 + h) - f(x_0)}{h}\leq 0
\end{align*}
eftersom nämnaren är negativ enligt antagandet. När $h\to 0$ från det negativa hållet har man
\begin{align*}
	\frac{f(x_0 + h) - f(x_0)}{h}\geq 0.
\end{align*}
Vi räknar ut gränsvärdet när $h$ går mot $0$. Eftersom det existerar, måste vi ha att $\dv{f}{x} (x_0) = 0$.

\paragraph{Rolles sats}
Låt $f$ vara kontinuerlig på $[a, b]$, deriverbar på $(a, b)$ så att $f(a) = f(b)$. Då existerar $p\in (a, b)$ så att $\dv{f}{x}\eval_p = 0$.

\proof
Om $f$ är konstant på $[a, b]$ är beviset trivialt.

Annars, låt $f(x) > f(a)$ för något $x\in (a, b)$. Eftersom $f$ är kontinuerlig på $[a, b]$, antar den enligt sats ett största värde. Eftersom $f(a) = f(b)$ måste detta största värdet antas i något $q\in (a, b)$. Då $f$ är deriverbar i $q$, gäller det enligt sats att $\dv{f}{x} (q) = 0$. Detta är punkten vi söker.

Ett analogt bevis gäller om $f(x) < f(a)$ för något $x\in (a, b)$.

\paragraph{Generaliserade medelvärdessatsen}
Låt $f$ och $g$ vara reellvärda, kontinuerliga på $[a, b]$ och deriverbara på $(a, b)$. Då existerar $p\in (a, b)$ så att
\begin{align*}
	\deval{f}{x}{p}(g(b) - g(a)) = \deval{g}{x}{p}(f(b) - f(a)).
\end{align*}
Om $g(a)\neq g(b)$ och $\dv{g}{x}\eval_p\neq 0$, gäller
\begin{align*}
	\frac{\deval{g}{x}{p}}{\deval{g}{x}{p}} = \frac{f(b) - f(a)}{g(b) - g(a)}.
\end{align*}

\subparagraph{Medelvärdesatsen}
Välj $g(x) = x$. Detta ger
\begin{align*}
	\dv{f}{x}\eval_p(b - a) = f(b) - f(a).
\end{align*}

\proof
Bilda
\begin{align*}
	h(x) = f(x)(g(b) - g(a)) - g(x)(f(b) - f(a)),
\end{align*}
som är kontinuerlig och deriverbar på intervallet enligt annan sats. Denna uppfyller $h(a) = h(b)$, och då existerar enligt Rolles sats ett $p\in (a, b)$ så att $\deval{h}{x}{p} = 0$. Vi har
\begin{align*}
	\deval{h}{x}{x} = \deval{f}{x}{x}(g(b) - g(a)) - \deval{g}{x}{x}(f(b) - f(a)),
\end{align*}
vilket ger
\begin{align*}
	\deval{f}{x}{p}(g(b) - g(a)) = \deval{g}{x}{p}(f(b) - f(a)).
\end{align*}

\paragraph{Följder av dessa satser}
Låt $f$ vara deriverbar på $(a, b)$. Då gäller:
\begin{itemize}
	\item $\deval{f}{x}{x} = 0$ för varje $x\in (a, b)$ om och endast om $f$ är konstant på $(a, b)$.
	\item $\deval{f}{x}{x}\geq 0$ för varje $x\in (a, b)$ om och endast om $f$ är växande på $(a, b)$.
	\item $\deval{f}{x}{x} > 0$ implicerar att $f$ är strängt växande på $(a, b)$.
	\item $\deval{f}{x}{x}\leq 0$ för varje $x\in (a, b)$ om och endast om $f$ är avtagande på $(a, b)$.
	\item $\deval{f}{x}{x} < 0$ implicerar att $f$ är strängt avtagande på $(a, b)$.
\end{itemize}

\proof
Om $f$ är konstantfunktionen, är första påståendet triviellt. Om $\dv{f}{x}(x) = 0$ på $(a, b)$, välj $x_0, x_1$ i intervallet så att $x_0 < x_1$. Då ger medelvärdesatsen att $f(x_1) - f(x_0) = \deval{f}{x}{x}(x_1 - x_0) = 0$, med $p\in (x_0, x_1)$, vilket bevisar omvändingen.

Om nu $\deval{f}{x}{x} > 0$ på intervallet, ger medelvärdesatsen på samma sätt $f(x_1) - f(x_0) > 0$, med ett analogt argument om nollan inkluderas. Anta nu att $f$ är växande. Detta ger
\begin{align*}
	\deval{f}{x}{x} = \limit{h}{0}\frac{f(x + h) - f(x)}{h}\geq 0.
\end{align*}
Anledningen till att det inte är en ekvivalens när derivatan är strikt positiv är att detta gränsvärdet kan bli $0$ även om $f$ är växande. Med ett analogt bevis för de två sista påståenden är beviset klart.

\paragraph{L'Hôpitals regel}
Låt $f, g$ vara reellvärda, deriverbara funktioner i en omgivning $I$ av $a$ sådana att
\begin{align*}
	\lim\limits_{x\to a}f(x) = \lim\limits_{x\to a}g(x) = 0.
\end{align*}
Då gäller att
\begin{align*}
	\lim\limits_{x\to a}\frac{f(x)}{g(x)} = \lim\limits_{x\to a}\frac{\dv{f}{x} (x)}{\dv{g}{x} (x)}.
\end{align*}

\proof

\paragraph{Oändliga kvoter}
Låt
\begin{align*}
	\lim\limits_{x\to a}\frac{\dv{f}{x} (x)}{\dv{g}{x} (x)} &= L, \\
	\limit{x}{a}f(x)                                        &= \pm\infty, \\
	\limit{x}{a}g(x)                                        &= \pm\infty.
\end{align*}
Då gäller att
\begin{align*}
	\limit{x}{a}\frac{f(x)}{g(x)} = L.
\end{align*}

\proof

\paragraph{Konvexitet och derivata}
Låt $f$ vara deriverbar i $(a, b)$. Då är $f$ konvex i $(a, b)$ omm $\dv{f}{x}$ är växande i $(a, b)$.

\proof

\paragraph{Andrederivata och konvexitet}
Låt $f$ vara två gånger deriverbar i $(a, b)$. Då är $\dv[2]{f}{x} (x)\geq 0$ för varje $x\in (a, b)$ omm $f$ är konvex.

\proof

\paragraph{Andrederivata och inflexionspunkt}
Låt $f$ vara två gånger deriverbar och låt $\dv[2]{f}{x}$ vara kontinuerlig. Om $f$ har en inflexionspunkt i $x_0$ så är $\dv[2]{f}{x} (x_0) = 0$.

\proof

\paragraph{Taylors formel}
Låt $f$ vara $n$ gånger deriverbar och definierad i en omgivning av $0$, sådan att $\dv[n]{f}{x}$ är kontinuerlig. Då är
\begin{align*}
	f(x) = \sum\limits_{i = 0}^{n - 1}\frac{\deval[i]{f}{x}{0}}{i!}x^i + \frac{\deval[i]{f}{x}{\alpha}}{n!}x^n
\end{align*}
för något $\alpha\in [0, x]$. Kring en godtycklig punkt $a$ blir formeln
\begin{align}
	f(x) = \sum\limits_{i = 0}^{n - 1}\frac{\deval[i]{f}{x}{a}}{i!}(x - a)^i + \frac{\deval[n]{f}{x}{\alpha}}{n!}(x - a)^n
	\label{eq:taylor_formula}
\end{align}
för något $\alpha\in [a, x]$.

\proof
Vi beviser satsen först för $a = 0$. Det är klart att formeln stämmer för $x = 0$, så bilda
\begin{align*}
	C = \frac{f(x) - p(x)}{x^n}, x\neq 0.
\end{align*}
Då är beviset ekvivalent med att visa att $Cn! = \deval[n]{f}{x}{\alpha}$ för et lämpligt $\alpha$.

Notera att $\deval[i]{f}{x}{0} = \deval[i]{p}{x}{0}, i = 0, \dots, n - 1$, och bilda
\begin{align*}
	g(t) = f(t) - p(t) - Ct^n\implies \deval[i]{g}{x}{0} = 0, i = 0, \dots, n - 1.
\end{align*}
Från definitionen är även $g(x) = 0$, och eftersom $g$ är kontinuerlig finns det enligt Rolles sats $x_1\in (0, x)$ så att $\deval{g}{x}{x_1} = 0$. Et motsvarande argument användt flera gånger ger att det finns $x_n\in (0, x_{n - 1})\subseteq [0, x]$ så att $\deval[n]{g}{x}{x_n} = 0$.
\begin{align*}
	\deval[n]{g}{x}{x_n} = \deval[n]{f}{x}{x_n} - Cn!,
\end{align*}
och nollstället ger önskad likhet.

För att visa satsen kring något $a\neq 0$, bilda $g(t) = f(t + a)$. Denna uppfyller förutsättningarna för formeln vi har bevist, vilket ger
\begin{align*}
	g(t) = \sum\limits_{i = 0}^{n - 1}\frac{\deval[i]{g}{x}{0}}{i!}t^i + \frac{\deval[i]{g}{x}{\alpha_0}}{n!}t^n = f(t + a), \alpha_0\in [0, t].
\end{align*}
Vi använder att $\deval[i]{g}{t}{t} = \deval[i]{f}{t}{t + a}, i = 0, \dots, n$ för att få
\begin{align*}
	f(t + a) = \sum\limits_{i = 0}^{n - 1}\frac{\deval[i]{f}{t}{a}}{i!}t^i + \frac{\deval[i]{f}{t}{\alpha_0 + a}}{n!}t^n, \alpha\in [0, t].
\end{align*}
Definiera $x = t + a$ och $\alpha = \alpha_0 + a\in [a, x]$ för att få
\begin{align*}
	f(x) = \sum\limits_{i = 0}^{n - 1}\frac{\deval[i]{f}{t}{a}}{i!}t^i + \frac{\deval[i]{f}{t}{\alpha}}{n!}t^n, \alpha\in [a, x].
\end{align*}

\paragraph{Taylors formel och stora ordo}
Låt $f$ vara $n$ gånger deriverbar och $\dv[n]{f}{x}$ vara kontinuerlig i en omgivning av $0$. Då är
\begin{align*}
	f(x) = \sum\limits_{i = 0}^{n - 1}\frac{\dv[i]{f}{x} (0)}{i!}x^i + \Ordo{x^n}.
\end{align*}

\proof