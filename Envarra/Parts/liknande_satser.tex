\section{Liknande satser}
Denna delen av sammanfattningen kombinerar analoga satser från diskreta och kontinuerliga fall, mer spesifikt funktioner och talföljder respektiva integraler och summor.

\subsection{Funktioner och talföljder}
Här kommer funktionen eller talföljden omtalas som grejen, med symbolet $a$, och indexet $n$ eller variabeln $x$ omtalas som variablen med symbol $v$.

\paragraph{Uppåt och nedåt begränsadhet}
En grej $a$ är uppåt begränsad om det finns ett $M$ så att $a\leq M$ för alla $v \geq 1$. Nedåt begränsade grejer definieras analogt.

\paragraph{Begränsade grejer}
En talföljd är begränsad om den är både uppåt och nedåt begränsad.

\paragraph{Gränsvärde vid oändligheten}
$a$ konvergerar mot $A$ när $v\to\infty$ om det for varje $\varepsilon > 0$ finns ett $N$ sådant att $\abs{a - A} < \varepsilon$ för varje $v > N$. Detta skrivs
\begin{align*}
	\limit{a}{\infty}a = A
\end{align*}
eller $a\to A$ när $v\to\infty$. Om det för en grej $a$ inte finns ett sådant $A$, sägs $a$ vara divergent då $v\to\infty$.

\paragraph{Det oegentliga gränsvärdet}
$a$ har det oegentliga gränsvärdet $\infty$ då $x\to\infty$ om det för varje $M$ finns ett $N$ sådant att $a > M$ för varje $v > N$. Detta skrivs
\begin{align*}
	\limit{x}{\infty}a = \infty.
\end{align*}

\paragraph{Standardgränsvärden}
Låt $a > 1, b > 0$. Då gäller att
\begin{align*}
	\limit{v}{\infty}\frac{a^v}{v^b} = \infty.
\end{align*}

\subsection{Integraler och summor}
Här kommer funktionen eller talföljden som integreras eller summeras omtalas som funktionen, med symbol $a$, integralet eller summan från $a$ till $b$ omtalas som grejen, med symbolet $S_a^b v$, och indexet eller variabeln omtalas variabeln, med symbolet $v$.

\paragraph{Integration med oändliga gränser}
Grejen definieras som
\begin{align*}
	S_a^{\infty} v = \limit{R}{\infty}S_a^R.
\end{align*}
och integration från $-\infty$ analogt. Om gränsvärdet existerar, är grejen konvergent. Annars är den divergent.

\paragraph{Jämförelsessats för två integraler}
Låt $a, b$ vara funktioner sådana att $0\leq a\leq b$ för varje $v > x$. Då gäller att om $S_x^{\infty}v$ är konvergent är även $S_x^{\infty}u$ konvergent, och om $S_x^{\infty}u$ är divergent är även $S_x^{\infty}v$ divergent.

\paragraph{Kvoten av termer och konvergens}
Låt $a, b$ uppfylla
\begin{align*}
	\limit{v}{\infty}\frac{a}{b} = K\neq 0.
\end{align*}
för $v > a$. Då konvergerar $S_x^{\infty}a$ om och endast om $S_x^{\infty}b$ konvergerar.

\paragraph{Summan av potenser}
Serien
\begin{align*}
	S_1^{\infty}a\frac{1}{v^p}
\end{align*}
är konvergent om och endast om $p > 1$.

\subsection{Lokala integraler och integraler mot oändligheten}

Här kommer integralet över något domän $D$, av ändlig eller oändlig storlek att skrivas som $\int\limits_{D}$. Att en funktion är integrerbar i detta domänet kommer här bety att den är integrerbar godtyckligt nära gränserna, och $x\in D$ kommer bety att $x$ kan vara godtyckligt nära gränserna.

\paragraph{Jämförelsessats för två integraler}
Låt $f, g$ vara integrerbara i $D$ för alla , sådana $0\neq f(x)\neq g(x)$ för varje $x\in D$. Då gäller att om $\integ{D}{}{g(x)}{x}$ är konvergent är även $\integ{D}{}{f(x)}{x}$ konvergent, och om $\integ{D}{}{f(x)}{x}$ är divergent är även $\integ{D}{}{g(x)}{x}$ divergent.

\paragraph{Gräns av kvot och integral}
Låt $f, g$ vara positiva och integrerbara i $D$ för alla $x\in D$ sådana att
\begin{align*}
	\limit{x}{a}\frac{f(x)}{g(x)} = K\neq 0.
\end{align*}
där $a$ är någon problematisk gräns. Då är $\integ{D}{}{g(x)}{x}$ konvergent om och endast om $\integ{D}{}{f(x)}{x}$ är konvergent.

\subsection{Samband mellan summor och integraler}

Dessa satsar upprepas eftersom jag tyckte de var relevanta.

\paragraph{Samband mellan summor och integraler}
Låt $f$ vara en avtagande funktion i $[m, n]$, där $m, n$ är heltal sådana att $m < n$. Då gäller:
\begin{align*}
	\sum\limits_{i = m + 1}^{n}f(i)\leq \integ{m}{n}{f(x)}{x}\leq \sum\limits_{i = m}^{n - 1}f(i).
\end{align*}

Detta kan omformuleras till
\begin{align*}
	\integ{m}{n}{f(x)}{x} + f(n)\leq \sum\limits_{i = m}^{n}f(i)\leq \integ{m}{n}{f(x)}{x} + f(m).
\end{align*}

Låt $f$ vara en växande funktion i $[m, n]$, där $m, n$ är heltal sådana att $m < n$. Då gäller:
\begin{align*}
	\sum\limits_{i = m}^{n - 1}f(i)\leq \integ{m}{n}{f(x)}{x}\leq \sum\limits_{i = m + 1}^{n}f(i).
\end{align*}

Detta kan omformuleras till
\begin{align*}
	\integ{m}{n}{f(x)}{x} + f(m)\leq \sum\limits_{i = m}^{n}f(i)\leq \integ{m}{n}{f(x)}{x} + f(n).
\end{align*}

\paragraph{Cauchys integralkriterium}
Låt $f$ vara positiv och avtagande i $(m, \infty)$. Då är $\sum\limits_{i = m}^{\infty}f(i)$ konvergent om och endast om $\integ{m}{\infty}{f(x)}{x}$ är konvergent.