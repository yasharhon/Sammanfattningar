\subsection{Definitioner}

\paragraph{Definition av en funktion}

Låt $X, Y$ vara mängder. En funktion $f: X\to Y$ är ett sätt att till varje element $x \in X$ tilldela ett välbestämt element $y\in Y$. Vi säger att $x$ avbildas på $y$ och att $y$ är bilden av $x$. $x$ kallas argumentet till $f$. $X$ kallas funktionens definitionsmängd, och betecknas även $D_f$. $Y$ kallas funktionen målmängd.

\paragraph{Värdemängd}

Värdemängden till $f: X\rightarrow Y$ definieras som:
\begin{align}
	V_f=\left \{y\in Y: y=f(x)\textnormal{ för något }x\in X\right \}\nonumber
\end{align}
alltså alla värden $f$ antar.

\paragraph{Injektivitet}

$f$ är injektiv om det för varje $x_1, x_2 \in X$ gäller att om $f(x_1)=f(x_2)$ så är $x_1=x_2$.

\paragraph{Surjektivitet}

$f$ är surjektiv om $V_f=Y$.

\paragraph{Bijektivitet}

Om $f$ är injektiv och surjektiv, är $f$ bijektiv.

\paragraph{Inversa funktioner}

Låt $f: X \to Y$ vara en bijektiv funktion. Inversen till $f$ är avbildningen $f^{-1}: Y\to X$ som ges av $f^{-1}(y)=x$, där $y=f(x)$. Funktioner som har en invers kallas inverterbara.

\paragraph{Växande och avtagande funktioner}

En funktion $f$ är växande på en mängd $M\in D_f$ om det för varje $x,y\in M: x<y$ gäller att $f(x)\leq f(y)$. Om $M=D_f$ kallas $f$ växande. Avtagande funktioner definieras analogt.

\paragraph{Strängt växande och avtagande funktioner}

En funktion $f$ är strängt växande på en mängd $M\in D_f$ om det för varje $x,y\in M: x<y$ gäller att $f(x)<f(y)$. Om $M=D_f$ kallas $f$ strängt växande. Strängt avtagande funktioner definieras analogt.

\paragraph{Monotona funktioner}

Om en funktioner är antingen strängt växande respektiva strängt avtagande eller växande respektiva avtagande i ett intervall, är den strängt monoton respektiva monoton.

\paragraph{Uppåt och nedåt begränsade funktioner}

En funktion $f$ är uppåt begränsad om $V_f$ är uppåt begränsad. Nedåt begränsade funktioner definieras analogt. Om funktioner saknar övre eller nedra begrensning är den uppåt eller nedåt obegränsad.

