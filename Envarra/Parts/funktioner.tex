\paragraph{Definition av en funktion}

Låt $X, Y$ vara mängder. En funktion $f: X \rightarrow Y$ är ett sätt att till varje element $x \in X$ tilldela ett välbestämt element $y \in Y$. Vi säger att $x$ avbildas på $y$ och att $y$ är bilden av $x$. $x$ kallas argumentet till $f$. $X$ kallas funktionens definitionsmängd, och betecknas även $D_f$. $Y$ kallas funktionen målmängd.

\paragraph{Värdemängd}

Värdemängden till $f: X\rightarrow Y$ definieras som:
\begin{align}
	V_f=\left \{y\in Y: y=f(x)\textnormal{ för något }x\in X\right \}\nonumber
\end{align}
alltså alla värden $f$ antar.

\paragraph{Injektivitet}

$f$ är injektiv om det för varje $x_1, x_2 \in X$ gäller att om $f(x_1)=f(x_2)$ så är $x_1=x_2$.

\paragraph{Surjektivitet}

$f$ är surjektiv om $V_f=Y$.

\paragraph{Bijektivitet}

Om $f$ är injektiv och surjektiv, är $f$ bijektiv.