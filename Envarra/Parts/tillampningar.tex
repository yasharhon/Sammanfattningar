\section{Integralens tillämpningar}

\subsection{Volymberäkning}

\paragraph{Rotation kring $x$-axeln}
Antag att $f$ är kontinuerlig på $[a, b]$. Om vi roterar regionen begränsad av linjerna $x = a, x = b$, $x$-axeln och grafen till $f$ kring $x$-axeln bildas en kropp vars volym vi ska beräkna. Vi approximerar volymet med $n$ cylindrar. Dessa ligger under grafen, är centrerade i $x$-axeln, har lik höjd och är jämnt fördelade på $[a, b]$. Varje cylinder har då radius $f(x_i)$ och höjd $\frac{b - a}{n} = \Delta$. Det approximerade volymet är då
\begin{align*}
	V = \sum\limits_{i = 0}^{n - 1}\pi f(x_i)^2\Delta.
\end{align*}
Denna summan har formen till en Riemannsumma, och går då för stora $n$ mot kroppens volym
\begin{align*}
	V = \pi\integ{a}{b}{f(x)^2}{x}.
\end{align*}

\paragraph{Rotation kring $y$-axeln}
Antag att $f$ är kontinuerlig på $[a, b]$. Om vi roterar regionen begränsad av linjerna $x = a, x = b$, $x$-axeln och grafen till $f$ kring $y$-axeln bildas en kropp vars volym vi ska beräkna. Vi approximerar volymet med $n$ cylinderskal. Dessa är centrerade i $y$-axeln, har lik tjocklek och ligger lag för lag på $[a, b]$ under grafen. Vart skal har då tjocklek $\frac{b - a}{n} = \Delta$ och höjd $f(x_i)$. Det approximerade volymet är då
\begin{align*}
	V &= \sum\limits_{i = 0}^{n - 1}\pi(x_{i + 1}^2 - x_i^2)f(x_i)\\
	  &= \sum\limits_{i = 0}^{n - 1}\pi f(x_i)(x_{i + 1} + x_i)(x_{i + 1} - x_i) \\
	  &= \sum\limits_{i = 0}^{n - 1}\pi f(x_i)(2x_i + \Delta)\Delta \\
	  &= \sum\limits_{i = 0}^{n - 1}2\pi f(x_i)x_i\Delta + \Delta\sum\limits_{i = 0}^{n - 1}\pi f(x_i)\Delta.
\end{align*}
Den första summan har formen till en Riemannsumma, och går då för stora $n$ mot
\begin{align*}
	2\pi\integ{a}{b}{xf(x)}{x}.
\end{align*}
Den andra summan går för stora $n$ mot $0$ eftersom summan går mot ett ändligt integral medan faktoren framför går mot $0$. Då ges kroppens volym av
\begin{align*}
	V = 2\pi\integ{a}{b}{xf(x)}{x}.
\end{align*}