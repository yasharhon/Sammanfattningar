\section{Gränsvärden}

\subsection{Definitioner}

\paragraph{Gränsvärde vid oändligheten}
Låt $f$ vara en funktion definierad i $(a, \infty)$. $f$ konvergerar mot gränsvärdet $A$ när $x\to\infty$ om det for varje $\varepsilon > 0$ finns ett $N$ sådant att $\abs{f(x) - A} < \varepsilon$ för varje $x > N$. Detta skrivs
\begin{align*}
	\lim_{x \to \infty} f(x) = A
\end{align*}
eller $f(x) \to A$ när $x \to \infty$.

\paragraph{Divergens}
Om det för en funktion $f$ inte finns ett sådant $A$, sägs $f$ vara divergent då $x \to \ infty$.

\paragraph{Det oegentliga gränsvärdet}
Låt $f$ vara en funktion definierad i $(a, \infty)$. $f$ har det oegentliga gränsvärdet $\infty$ då $x \ to \infty$ om det för varje $M$ finns ett $N$ sådant att $f(x) > M$ för varje $x > N$. Detta skrivs
\begin{align*}
	\lim_{x \to \infty} f(x) = \infty
\end{align*}

\subsection{Satser}