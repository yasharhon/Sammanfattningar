\section{Gränsvärden}

\subsection{Definitioner}

\paragraph{Gränsvärde vid oändligheten}
Låt $f$ vara en funktion definierad i $(a, \infty)$. $f$ konvergerar mot gränsvärdet $A$ när $x\to\infty$ om det for varje $\varepsilon > 0$ finns ett $N$ sådant att $\abs{f(x) - A} < \varepsilon$ för varje $x > N$. Detta skrivs
\begin{align*}
	\lim_{x \to \infty} f(x) = A
\end{align*}
eller $f(x) \to A$ när $x \to \infty$.

\paragraph{Divergens}
Om det för en funktion $f$ inte finns ett sådant $A$, sägs $f$ vara divergent då $x \to \ infty$.

\paragraph{Det oegentliga gränsvärdet}
Låt $f$ vara en funktion definierad i $(a, \infty)$. $f$ har det oegentliga gränsvärdet $\infty$ då $x \ to \infty$ om det för varje $M$ finns ett $N$ sådant att $f(x) > M$ för varje $x > N$. Detta skrivs
\begin{align*}
	\lim_{x \to \infty} f(x) = \infty
\end{align*}

\subsection{Satser}

\paragraph{Gränsvärden för kombinationer av funktioner}
Låt $f,g$ vara kontinuerliga funktioner sådana att $f(x)\to A, g(x)\to B$ när $x\to\infty$. Då gäller att
\begin{itemize}
	\item[a)] $f(x) + g(x)\to A + B$ när $x\to\infty$.
	\item[b)] $f(x)g(x)\to AB$ när $x\to\infty$.
	\item[c)] om $B\neq 0$ så följer att $\frac{f(x)}{g(x)}\to\frac{A}{B}$ när $x\to\infty$.
	\item[d)] om $f(x)\leq g(x)$ för alla $x\in (a,\infty)$ så gäller att $A\leq B$.
\end{itemize}

\subparagraph{Bevis}
Mjo.

\paragraph{Gränsvärden och supremum}
Låt $f: (a, \infty)\to\R$ för något $a\in\R$ vara växande och uppåt begränsad. Då gäller att
\begin{align*}
	\lim\limits_{n\to\infty} = \sup{f(x): x\geq a}.
\end{align*}

\subparagraph{Bevis}
Nä.

\paragraph{Standardgränsvärden}
Låt $a > 1, b > 0$. Då gäller att
\begin{align*}
	\lim\limits_{x\to\infty}\frac{a^x}{x^b} = \infty \\
	\lim\limits_{x\to\infty}\frac{x^b}{\log_a x} = \infty
\end{align*}

\subparagraph{Bevis}
Orkar inte.