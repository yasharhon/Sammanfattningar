\section{Electrons}

\paragraph{Statistics of Electrons}
Properties of solids will also be treated using statistical mechanics. Again, for details, please see my summary of SI1162 Statistical Physics.

\paragraph{The Electron Gas}
The electron gas is a system of non-interacting electrons. The statistical mechanics studied in SI1162 are for electron gases.

\paragraph{Fundamentals}
The density of states for the electron gas is given by
\begin{align*}
	\rho(E) = \frac{V}{2\pi^{2}}\left(\frac{2m}{\hbar^{2}}\right)^{\frac{3}{2}}\sqrt{E}.
\end{align*}
The distribution of energies is according to the Fermi-Dirac distribution
\begin{align*}
	f(E) = \frac{1}{e^{\beta(E - \mu)} + 1}.
\end{align*}

\paragraph{Fermi Energy}
The Fermi energy is the energy of the highest occupied state of a system at $T = 0$. According to the shape of the Fermi-Dirac distribution, $E_{\text{F}} = \eval{\mu}_{T = 0}$.

\paragraph{Heat Capacity}
The molar heat capacity of the free electron gas is given by
\begin{align*}
	c_{V} = \frac{\pi^{2}s^{\prime}}{2}\frac{RT}{T_{\text{F}}}
\end{align*}
where $s^{\prime}$ is the number of electrons per formula unit and $T_{\text{F}}$ is the Fermi temperature.
	
\paragraph{Resistivity}
To treat resistivity in the free electron model, consider an electron in the gas. Its velocity is given by $\vb{v} = \frac{\hbar}{m}\vb{k}$. In the presence of a constant electric field, Newton's second law gives
\begin{align*}
	\dv{\vb{k}}{t} = -\frac{e}{\hbar}\vb{E}.
\end{align*}
The solution to this is unbounded, and this description is thus not complete.

To remedy this, we introduce scattering to the problem. Supposing that electrons are scattered after some characteristic time $\tau$, we introduce the drift velocity
\begin{align*}
	\vb{v}_{\text{d}} = -\frac{e\tau}{m}\vb{E}.
\end{align*}
The current density is thus
\begin{align*}
	\vb{J} = -ne\vb{v}_{\text{d}} = \frac{ne^{2}\tau}{m}\vb{E},
\end{align*}
and thus the resistivity is
\begin{align*}
	\rho = \frac{m}{ne^{2}\tau}.
\end{align*}

An alternative way to derive this is to introduce a scattering term to Newton's law according to
\begin{align*}
	m\dv{\vb{v}}{t} = -e\vb{E} - \frac{m}{\tau}\vb{v},
\end{align*}
with steady-state solution
\begin{align*}
	\vb{v}_{\text{d}} = -\frac{e\tau}{m}\vb{E}.
\end{align*}

\paragraph{Experimental Resistivity}
Experimentally, the temperature dependence of the resistivity is of the form
\begin{align*}
	\rho = \rho_{\text{I}} + \rho_{\text{L}},
\end{align*}
and is called Matthiesen's rule. The first term comes from impurities in the solid, which scatter electrons, and is constant. The second term comes from lattice vibrations, and thus vanishes at low temperatures and is linear at high temperatures.

\paragraph{The Hall Effect}
For a solid in the presence of both an electric and magnetic field, Newton's second law for the electrons becomes
\begin{align*}
	m\dv{\vb{v}}{t} = -e\vb{E} - e\vb{v}\times\vb{B} - \frac{m}{\tau}\vb{v},
\end{align*}
The implicit steady-state solution is
\begin{align*}
	\vb{v} = -\frac{e\tau}{m}\left(\vb{E} + \vb{v}\times\vb{B}\right),
\end{align*}
which must be studied for a specific geometry.

The geometry which is typically considered is a prismatic slab of solid through which a current runs in the $x$-direction and which is exposed to a magnetic field in the $z$-direction. For this geometry, as there is no current in the $y$-direction and the electric field is planar, we have
\begin{align*}
	v_{x} = -\frac{e\tau}{m}E_{x},\ E_{y} = -\frac{e\tau B}{m}E_{x},\ v_{z} = 0.
\end{align*}
Hence the electric field transversal to the current direction is non-zero. This is termed the Hall effect.

Defining the Hall coefficient as
\begin{align*}
	R = \frac{E_{x}}{J_{x}B},
\end{align*}
the previous expression for the current density can be used to obtain
\begin{align*}
	R = \frac{1}{n(-e)}.
\end{align*}
This can be generalized for materials with other charge carriers by replacing $-e$ by the charge of the charge carriers. However, this description, and consequently the free electron model, is lacking for insulators and semiconductors.

\paragraph{The Central Equation}
To flesh out our description of electronic properties, we will start by introducing a potential in which the electrons are moving. The first approximation consists in combining the potential experienced by any single electron from both the atoms in the crystal and other electrons into some potential $V(\vb{r})$, where we require that $V$ have the same periodicity as the crystal. For this case, the Schrödinger equation is separable and given by
\begin{align*}
	-\frac{\hbar^{2}}{2m}\laplacian{\Psi} + V\Psi = E\Psi.
\end{align*}
The electron density is given by $n = N\abs{\Psi}^{2}$, and we will therefore require that the probability density have the same periodicity as the lattice.

To solve the Schrödinger equation, we utilize the periodicity of the potential and the solution to Fourier expand the two as
\begin{align*}
	V = \sum\limits_{\vb{G}}v_{\vb{G}}e^{i\vb{G}\cdot\vb{r}},\ \Psi = \sum\limits_{\vb{k}}c_{\vb{k}}e^{i\vb{k}\cdot\vb{r}}.
\end{align*}
Note that $V$ is expanded in terms of the reciprocal lattice vectors as it must have the periodicity of the lattice, while this is not the case for the electron state. This is due to the fact that the electron density might still have the correct translational symmetry with other Fourier components. Inserting this into the Schrödinger equation yields
\begin{align*}
	\sum\limits_{\vb{k}}\frac{\hbar^{2}k^{2}}{2m}c_{\vb{k}}e^{i\vb{k}\cdot\vb{r}} + \left(\sum\limits_{\vb{G}}v_{\vb{G}}e^{i\vb{G}\cdot\vb{r}}\right)\left(\sum\limits_{\vb{k}}c_{\vb{k}}e^{i\vb{k}\cdot\vb{r}}\right) &= E\sum\limits_{\vb{k}}c_{\vb{k}}e^{i\vb{k}\cdot\vb{r}}, \\
	\sum\limits_{\vb{k}}\frac{\hbar^{2}k^{2}}{2m}c_{\vb{k}}e^{i\vb{k}\cdot\vb{r}} + \sum\limits_{\vb{G}}\sum\limits_{\vb{k}}v_{\vb{G}}c_{\vb{k}}e^{i(\vb{k} + \vb{G})\cdot\vb{r}} &= E\sum\limits_{\vb{k}}c_{\vb{k}}e^{i\vb{k}\cdot\vb{r}}.
\end{align*}
The sum over $\vb{k}$ may be relabelled for each $\vb{G}$ (which are in the set of $\vb{k}$) to yield
\begin{align*}
	\sum\limits_{\vb{k}}\frac{\hbar^{2}k^{2}}{2m}c_{\vb{k}}e^{i\vb{k}\cdot\vb{r}} + \sum\limits_{\vb{G}}\sum\limits_{\vb{k}}v_{\vb{G}}c_{\vb{k} - \vb{G}}e^{i\vb{k}\cdot\vb{r}} &= E\sum\limits_{\vb{k}}c_{\vb{k}}e^{i\vb{k}\cdot\vb{r}}, \\
	\sum\limits_{\vb{k}}e^{i\vb{k}\cdot\vb{r}}\left(\left(\frac{\hbar^{2}k^{2}}{2m} - E\right)c_{\vb{k}} + \sum\limits_{\vb{G}}\sum\limits_{\vb{k}}v_{\vb{G}}c_{\vb{k} - \vb{G}}\right) &= 0.
\end{align*}
Noting that each term in the series is orthogonal to all others, we must require that each term be equal to zero. Introducing the free-electron energy $\lambda_{\vb{k}} = \frac{\hbar^{2}k^{2}}{2m}$ yields the central equation
\begin{align*}
	\left(\lambda_{\vb{k}} - E\right)c_{\vb{k}} + \sum\limits_{\vb{G}}\sum\limits_{\vb{k}}v_{\vb{G}}c_{\vb{k} - \vb{G}} = 0.
\end{align*}

A minor comment about this is that any single coefficient is inversely proportional to $\lambda_{\vb{k}} - E$, implying that some coefficients will be much larger than the others, meaning that in most cases there is no need to include many coefficients.

\paragraph{Bloch's Theorem}
As the wave function may be expanded in terms of plane waves, we will now study each plane wave solution, i.e. studying the solution for a fixed $\vb{k}$. Firstly we note that shifting $\vb{k}$ by a reciprocal lattice vector yields a new equation relating the same set of Fourier coefficients, meaning that if the dispersion relation is non-trivial (in other words, if there is kinetic energy in the system) then there are enough linearly independent equations to find non-trivial solutions. Secondly, we note that generally, allowing one coefficient in the set to be non-trivial implies that the others are too, meaning that the restriction to a single wave vector must in fact be modified to a single set of wavevectors. This set will be denoted by the wave vector in the first Brillouin zone. Assuming the system to have been solved, the state is
\begin{align*}
	\Psi = \sum\limits_{\vb{G}}c_{\vb{k} - \vb{G}}e^{i(\vb{k} - \vb{G})\cdot\vb{r}} = e^{i\vb{k}\cdot\vb{r}}\sum\limits_{\vb{G}}c_{\vb{k} - \vb{G}}e^{-i\vb{G}\cdot\vb{r}}.
\end{align*}
The sum is the Fourier series of a function which shares the periodicity of the lattice, and we dub this a Bloch function and denote it as $u_{\vb{k}}$. We thus arrive at the state
\begin{align*}
	\Psi = u_{\vb{k}}e^{i\vb{k}\cdot\vb{r}}.
\end{align*}

\paragraph{Energy Bands}
Solving the central equation will net you both the Fourier coefficients of the state and the energy of the state. The energies can be plotted as a function of $\vb{k}$ to yield a set of graphs of energies for different $\vb{k}$. These (and, sometimes, the set of energy values corresponding to each graph) are called energy bands. A feature of energy bands is often that certain energies are not found in the bands, meaning that there are gaps between the bands.

\paragraph{The One-Dimensional Nearly-Free Electron Model}
In the one-dimensional nearly-free electron model, electrons in a potential $V = 2V_{0}\cos(Gx),\ G = \frac{2\pi}{a}$ is studied. For this potential, we have only two non-zero Fourier coefficients $v_{G} = v_{-G} = V_{0}$.

To study this model and, in particular, its band gap, we consider electrons at the Brillouin zone boundary, where $k = \pm\frac{1}{2}G = \pm\frac{\pi}{a}$. According to our previous argument, we start by only considering the corresponding coefficients, which are given by
\begin{align*}
	\left(\lambda - E\right)c_{\frac{1}{2}G} + V_{0}c_{-\frac{1}{2}G} = 0,\ \left(\lambda - E\right)c_{-\frac{1}{2}G} + V_{0}c_{\frac{1}{2}G} = 0
\end{align*}
where $\lambda = \frac{1}{4}\frac{\hbar^{2}\pi^{2}}{2ma^{2}}$. The system has non-trivial solutions when
\begin{align*}
	\left(\lambda - E\right)^{2} - V_{0}^{2} = 0\implies E = \lambda \pm V_{0},
\end{align*}
and the corresponding solutions satisfy
\begin{align*}
	\frac{c_{\frac{1}{2}G}}{c_{-\frac{1}{2}G}} = \frac{V_{0}}{E - \lambda} = \pm 1,
\end{align*}
meaning that the low-energy state is proportional to $\sin(\frac{1}{2}Gx)$ and the high-energy state is proportional to $\cos(\frac{1}{2}Gx)$.

The results can be interpreted by considering the corresponding electron densities. For the low-energy state the density is concentrated at the potential minima, whereas for the high-energy state it is concentrated at the maxima, explaining the difference in energy. Furthermore, as the set of states does not continuously transition between the two, there is a band gap - here of size $2\abs{V_{0}}$.

Similarly, studying the solution close to the Brillouin zone boundary nets the energies
\begin{align*}
	E = \frac{\lambda_{k} + \lambda_{k + G}}{2} \pm\sqrt{\left(\frac{\lambda_{k} + \lambda_{k + G}}{2}\right)^{2} + V_{0}^{2}},
\end{align*}
corresponding to a similar, but somewhat larger bandgap.