\section{Superconductivity and Magnetism}

\paragraph{Superconductivity}
Superconductors are materials which exhibit a state in which their resistivity vanishes below measurable levels. Their study started with the discovery of metals which did not follow Mathiessen's rule at low temperatures, but instead saw a sudden drop in resistivity below a certain temperature. Since then, many different superconducting materials have been discovered.

The transition to superconductivity is a phase transition, and can thus only be maintained in a certain regime. This regime is characterized by critical temperatures, magnetic fields and current densities.

\paragraph{The Meissner Effect}
When exposed to an external magnetic field, currents are induced in the superconductor. Because there is no resistance, these currents increase in magnitude until they cancel the external field. This is termed the Meissner effect.

Based on this, we can compute the susceptibility of a superconductor. For a superconductor in an external field $\vb{B}_{\text{a}} = \mu_{0}\vb{H}$ the total magnetic field in the superconductor is
\begin{align*}
	\vb{B} = \mu_{0}(\vb{H} + \vb{M}),
\end{align*}
and the requirement that the flux be zero implies
\begin{align*}
	\vb{M} = -\vb{H}.
\end{align*}
Assuming superconductors to be linear, they satisfy a relation of the form $\vb{M} = \chi\vb{H}$ where $\chi$ is the magnetic susceptibility. We thus recognize that $\chi = -1$.

\paragraph{Type I and II Superconductors}
Type I superconductors exhibit the Meissner effect all the way until their phase transition. Type II superconductors have two phases characterized by different critical fields. For fields below the first critical field, type II superconductors exhibit the Meissner effect, whereas between the two critical fields the magnetization decreases with the external fields until it vanishes at the second critical field. This implies that in this phase external fields can penetrate the material.

\paragraph{Vortices in Type II Superconductors}
The second state for type II superconductors is called a vortex state. In this state, so-called vortices are formed in the material. The vortex is composed of a cylindrical region where current is conducted normally. The size of the cylinder is characterized by a length scale $\xi$, termed the coherence length. Outside the vortices, the material is superconducting and currents arise around the vortices which shield the rest of the superconductor from the magnetic field going through the vortices. The size of this region is characterized by a length scale $\lambda$, termed the penetration depth. Each vortex mediates a flux $\phi_{0} = \frac{h}{2e}$.

It turns out that this allows us to separate between type I and II superconductors, as type I superconductors satisfy $\xi > \sqrt{2}\lambda$, and vice versa for a type II superconductor.

\paragraph{The Isotope Effect}
For a given element, the isotope of which the superconductor is formed determines the critical temperature of the superconductor. The critical temperature is inversely proportional to the mass of the isotope.

\paragraph{London Theory}
The London theory of superconductivity is a phenomenological theory describing superconductivity. By minimizing the free energy of the electrons and fields, one obtains the equation
\begin{align*}
	\lambda_{L}^{2}\curl{\curl{B}} + \vb{B} = \vb{0},
\end{align*}
which, together with Maxwell's equations, give a complete description of the superconductor. $\lambda_{L}$ is called the London penetration depth.

Alternative formulations can be obtained. For instance, in stationary cases, Ampère's law yields
\begin{align*}
	\curl{\vb{H}} = \vb{J}.
\end{align*}
This is combined with $\vb{B} = \mu_{0}\vb{H}$, where there is no magnetization term as all magnetization comes from currents in the superconductor, which are integrated in Ampère's law. Inserting this into the London equation yields
\begin{align*}
	\curl{\vb{J}} = -\frac{1}{\mu_{0}\lambda_{L}^{2}}\vb{B}.
\end{align*}
Alternatively, in terms of the vector potential and in the correct gauge for a homogenous superconductor, we have
\begin{align*}
	\vb{J} = -\frac{1}{\mu_{0}\lambda_{L}^{2}}\vb{A}.
\end{align*}

\paragraph{BCS Theory}
The BCS theory of superconductivity answered two important questions. Firstly, electrons in a superconductor are free electron-like at low temperatures and close to their lowest energy state. This does not fit with the occurrence of a phase transition. Second, the isotope effect was experimentally discovered, but was not incorporated into the theory, as the nucleus does not interact with the outer electrons (somehow).

The idea behind the BCS theory can be summarized by the following example: Consider an electron passing through the lattice. It interacts with the lattice by Coulomb interactions, distorting the lattice as it moves. This, in turn, leaves behind a region where the lattice is positively charged, attracting a new electron to it. Hence BCS theory is based on studying electron-phonon interactions. This reveals the answer to the second question, as phonons are connected to the masses of the lattice atoms. Furthermore, this implies an attraction between electrons. In fact, it turns out that electrons with opposite wave vectors and spins form pairs which are called Cooper pairs. Cooper pairs are quasiparticles with bosonic character, explaining the lowered energy and presence of a phase transition.

This lowering of energy manifests as a change in the density of states, which goes from its typical fermion-like shape to shifting the most energetic electrons to an energy slightly below the Fermi energy. Between this energy and the Fermi energy there will be an energy region which is not occupied, and thus an energy gap. The concentration close to the Fermi level is due to the lowering of energy being comparable to a typical phonon energy, which is small compared to electron energies.

The BCS theory thus describes many-body effects in solids which give rise to exotic material behaviour. Physics of this kind are called emergent physics.

\paragraph{The Heat Capacity of a Superconductor}
When cooling below the critical temperature, the heat capacity of superconductors increases discontinuously and attains a new behaviour, vanishing at low temperatures. This is further evidence of superconductivity being a different phase altogether.

\paragraph{The Condensation Energy of a Superconductor}
The free energy difference between the normal and superconducting states is called the condensation energy. To estimate it, consider a small superconductor at infinite separation from a large magnet which is moved to the point where the magnetic field is equal to the critical magnetic field. The work done on the superconductor is
\begin{align*}
	W = -V\integ{0}{B_{\text{c}}}{\vb{B}_{\text{a}}}{\cdot\vb{M}}.
\end{align*}
This work is equal to the Helmholtz free energy of the superconductor. Using the previously derived relation between the applied field and the magnetization, we obtain
\begin{align*}
	F = F_{0} + \frac{V}{\mu_{0}}\integ{0}{B_{\text{c}}}{\vb{B}_{\text{a}}}{\cdot\vb{B}_{\text{a}}} = F_{0} + \frac{V}{2\mu_{0}}B_{\text{c}}^{2}.
\end{align*}
The free energy $F_{n}$ of the normal conducting state, on the other hand, is approximately constant when the magnetic field is varied. At the phase boundary, the two are equal, meaning that the condensation energy is
\begin{align*}
	F_{n} - F_{0} = \frac{V}{2\mu_{0}}B_{\text{c}}^{2}.
\end{align*}

\paragraph{Magnetic Susceptibility}
The magnetic susceptibility is given by
\begin{align*}
	\vb{M} = \chi\vb{H}.
\end{align*}

\paragraph{Types of Magnetic Response}
The magnetic response of a material is (in many cases) categorized according to the following categories:
\begin{itemize}
	\item Diamagnetism, where $\chi < 0$ and $\abs{\chi} \ll 0$.
	\item Paramagnetism, where $\chi > 0$ and $\abs{\chi} \ll 0$.
	\item Ferromagnetism, where $\chi \gg 0$. This is often accompanied by the presence of non-zero magnetization at zero external field if the temperature is sufficiently low.
\end{itemize}

\paragraph{Origins of Magnetic Response}
Magnetic response in a solid can have many different origins. A few examples are:
\begin{itemize}
	\item Atoms
	\begin{itemize}
		\item Paramagnetism due to redistribution of atomic electrons.
		\item Diamagnetism due to wavefunctions changing such tat they oppose the external field.
	\end{itemize}
	\item Free electrons
	\begin{itemize}
		\item Paramagnetism due to redistribution of free electrons.
		\item Diamagnetism due to wavefunctions changing such tat they oppose the external field.
	\end{itemize}
	\item Spin-spin interactions
	\begin{itemize}
		\item Ferromagnetism or antiferromagnetism, depending on the nature of the interaction.
		\item Other exotic effects, for instance spin waves.
	\end{itemize}
	\item Atomic nuclei, the discussion of which is beyond the scope of this course.
\end{itemize}

\paragraph{Diamagnetism from Atoms}
It turns out that by considering atomic electrons as non-interacting electrons in a Coulomb potential and adding a perturbation Hamiltonian
\begin{align*}
	H_{\text{d}} = \frac{e^{2}B^{2}}{8m}(x^{2} + y^{2}),
\end{align*}
diamagnetic effects are obtained.

We start with a classical description, which yields that electrons in an external magnetic field precess around the nucleus with a frequency
\begin{align*}
	\omega = \frac{eB}{2m},
\end{align*}
termed the Larmor frequency. The total current from all precessing electrons is
\begin{align*}
	I = -Ze\frac{eB}{4\pi m},
\end{align*}
yielding a magnetic moment of
\begin{align*}
	\mu = -\frac{Ze^{2}B}{4\pi m}\expval{\rho^{2}},
\end{align*}
where the latter factor may be taken as a mean over all electrons. This should probably be treated more explicitly, as it clearly motivates the form of the perturbation Hamiltonian.

The expected value of the perturbation Hamiltonian is given by
\begin{align*}
	\expval{H_{\text{d}}} = \frac{e^{2}B^{2}}{8m}\expval{\rho^{2}}.
\end{align*}
For some reason, the corresponding magnetic moment for a single state is
\begin{align*}
	\mu = -\grad_{\vb{B}}{\expval{H_{\text{d}}}} = -\frac{e^{2}B}{4m}\expval{\rho^{2}}.
\end{align*}
The total magnetic moment is thus
\begin{align*}
	\mu = -\frac{Ze^{2}B}{4m}\expval{\rho^{2}}.
\end{align*}
Rewriting this in terms of the atomic radius, we obtain
\begin{align*}
	M = n\mu = -\frac{nZe^{2}B}{6m}\expval{r^{2}},
\end{align*}
and finally the susceptibility
\begin{align*}
	\chi = -\frac{nZe^{2}\mu_{0}}{6m}\expval{r^{2}}.
\end{align*}

\paragraph{Paramagnetism in Atoms}
It turns out that by considering atomic electrons as non-interacting electrons in a Coulomb potential and adding a perturbation Hamiltonian
\begin{align*}
	H_{\text{p}} = \frac{e\hbar}{2m}L_{z},
\end{align*}
paramagnetic effects are obtained.

Paramagnetic effects are often found in atoms with unfilled inner electron shells. The total angular momentum for an atom is given by
\begin{align*}
	\vb{J} = \vb{L} + \vb{S}.
\end{align*}
The corresponding magnetic momentum is
\begin{align*}
	\vb*{\mu} = -g\mu_{\text{B}}\vb{J}
\end{align*}
where
\begin{align*}
	g = 1 + \frac{j(j + 1) + s(s + 1) - l(l + 1)}{2j(j + 1)}
\end{align*}
is called the Landé g-factor and $\mu_{\text{B}}$ is the Bohr magneton $\frac{e\hbar}{2m}$.

Electrons distribute in the shells according to Hund's rules:
\begin{enumerate}
	\item Maximize the total spin $s$.
	\item Maximize the total angular momentum $l$.
	\item Compute the total angular momentum according to $j = \abs{l - s}$ if the shell is less than half-filled and $j = l + s$ if the shell is more than half-filled.
\end{enumerate}

\paragraph{Magnetism in the Electron Gas}
At zero field, free electrons have an equal probability of having either spin. However, the introduction of an external field skews the statistics towards one configuration, splitting the density of states (or, more likely if you ask me, the occupation number) into two terms corresponding to either spin and creating paramagnetism. This effect is very small as the energy change is much smaller than the Fermi level, which is the energy at which electrons might switch energies.

The number of electrons that switch spins is approximately
\begin{align*}
	\Delta N = \mu_{\text{B}}BD_{2}(E_{\text{F}}) \approx \frac{1}{2}\mu_{\text{B}}BD(E_{\text{F}})
\end{align*}
where $D_{2}$ is the density of states with unfavorable spin and $D$ is the total density of states. The total magnetization is thus
\begin{align*}
	M = \frac{2\mu_{\text{B}}\Delta N}{V} = \frac{\mu_{\text{B}^{2}BD(E_{\text{F}})}}{V} = \frac{3\mu_{\text{B}}^{2}BN}{2E_{\text{F}}},
\end{align*}
and the susceptibility is
\begin{align*}
	\chi = \frac{3\mu_{0}\mu_{\text{B}}^{2}N}{2E_{\text{F}}}.
\end{align*}

In addition to this paramagnetic contribution there is a diamagnetic contribution. Landau showed that it was equal to a third of the paramagnetic contribution in magnitude, yielding the total susceptibility
\begin{align*}
	\chi = \frac{\mu_{0}\mu_{\text{B}}^{2}N}{E_{\text{F}}}.
\end{align*}
While we have neglected the temperature dependence along the way, it turns out that the susceptibility is almost temperature-independent.

\paragraph{Ferromagnetism}
Ferromagnetic materials exhibit spontaneous magnetization at zero applied field. Their behaviour is characterized by being ferromagnetic at low temperatures before undergoing a phase transition to a paramagnetic phase at higher temperatures. The critical temperature for ferromagnets is termed the Curie temperature. The existence of ferromagnetism can be explained by introducing coupling between atomic angular momenta in a solid.

\paragraph{The Weiss Model}
The Weiss model is a phenomenological theory of ferromagnetism and, as we will see, antiferromagnetism. It introduces an exchange field
\begin{align*}
	\vb{B}_{\text{E}} = \lambda\mu_{0}\vb{M}
\end{align*}
which must be taken into consideration when considering interactions with the magnetic field, but not when solving Maxwell's equations. $\lambda$ is a dimensionless constant. The effect of this term is to simulate the coupling effects described above.

In the paramagnetic phase, when treating each spin as a two-level system, it can be shown that
\begin{align*}
	\frac{1}{\chi} = \frac{T - T_{\text{C}}}{C},\ C = \frac{\mu_{0}n\mu_{\text{B}}^{2}}{\kb},\ T_{\text{C}} = \lambda C.
\end{align*}
In reality, the temperature at which the susceptibility is expected to be infinite and the onset of ferromagnetism differ slightly, but this is usually neglected.

In the ferromagnetic phase, spontaneous magnetization is expected at zero field. The magnetization at zero field is given by
\begin{align*}
	M = n\mu_{\text{B}}\tanh(\frac{\mu_{0}\mu_{\text{B}}\lambda M}{\kb T}),
\end{align*}
which is an implicit equation in terms of the magnetization. The Curie temperature is the lowest temperature at which the solution is $M = 0$. It can be shown that this corresponds to
\begin{align*}
	T_{\text{C}} = \lambda C
\end{align*}
for the same Curie constant as before, which means that the presence of the phase transition checks out.

In a more general case the orbital angular momenta in the paramagnetic phase cancel out, leaving $J = S$. Performing the same derivation as before (which I should do), the Curie constant
\begin{align*}
	C = \frac{N\mu_{0}g^{2}S(S + 1)\mu_{\text{B}}^{2}}{3\kb}.
\end{align*}

The Weiss model has certain complications. Some solids have weak coupling between angular momenta, as opposed to the strong coupling which is required for the Weiss model to work. For these cases, $J$ is a better quantum number than $S$. In addition, certain materials have interactions between the atomic structure and the band electrons, altering the effectiveness of the interactions and thereby certain parameters in the model. This also has an effect on the ferromagnetic phase, where the saturation magnetization is $M = n_{\text{B}}n\mu_{\text{B}}$, where $n_{\text{B}} \neq p$ is the effective number of Bohr magnetons.

\paragraph{The Heisenberg Model}
The Heisenberg model is a quantum theory of ferromagnetism and antiferromagnetism. It adds the interaction energy
\begin{align*}
	U = -2J\vb{s}_{i}\cdot\vb{s}_{j}
\end{align*}
between two angular momenta, where $J$ is the so-called exchange integral which arises due to overlap of wavefunctions.

To obtain an understanding for how this works, the exchange integral can be derived in the Weiss model. You finally obtain
\begin{align*}
		J = \frac{3\kb T_{\text{C}}}{2zS(S + 1)},
\end{align*}
where $z$ is the number of nearest neighbours.

The sign of the exchange integral need not be positive, and negative exchange integrals describe both antiferromagnetism and frustrated magnets.

\paragraph{Antiferromagnetism}
Antiferromagnetic materials exhibit spontaneous spin ordering at zero applied field. Their behaviour is characterized by being antiferromagnetic at low temperatures before undergoing a phase transition to a paramagnetic phase at higher temperatures. The critical temperature for antiferromagnets is termed the Néel temperature.

\paragraph{The Weiss Model for Antiferromagnets}
To describe antiferromagnets, they are first divided into two sublattices $A$ and $B$ corresponding to each spin configuration and considering only nearest-neighbour interactions. Through some work it can be shown that
\begin{align*}
	\frac{1}{\chi} = \frac{T + T_{\text{N}}}{2C}.
\end{align*}