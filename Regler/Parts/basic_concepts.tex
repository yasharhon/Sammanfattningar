\section{Grundläggande koncept}

\paragraph{Grundläggande begrepp och ideer}
Reglerteknik handlar om att kontrollera olika storheter, ofta betecknad $y$, mot något värde $r$. Dessa påverkas typiskt av en yttre störning $v$, och vi kan kontrollera dem vid att tillföra en påverkan $u$.

\paragraph{Strategi för att förstå}
För att förstå systemet, hittar vi först på en modell som beskriver det. Ur denna modellen fås typiskt en differentialekvation. Denna löser vi med Laplacetransform över tid.

\paragraph{Överförningsfungktionen}
Typiskt (isär om modellen ger en linjär differentialekvation) fås en lösning i Laplacerummet på formen $Y(s) = G(s)U(s)$, där $U$ är Laplacetransformen av $u$. Funktionen $G$ är överförningsfunktionen. Notera att denna lösningsformen beror på att alla initialvärden är $0$!

\paragraph{Poler}
Ett systems poler är rötterna till nämnarpolynomet (som typiskt finns) i överförningsfunktionen.

\paragraph{Stabilitet}
Ett system är stabilt om det tenderar mot ett visst läge. Systemets stabilitet är typiskt kopplad med dets noder. Detta kan man se i enkla fall, till exempel vanliga linjära ordinära differentialekvationer, då systemets polar anger hur snabbt lösningen avtar.

\paragraph{Nollställen}
Ett systems nollställen är rötterna till täljarpolynomet (som typiskt finns) hos överförningsfunktioner. Eftersom vi är intresserade av att styra $y$, är det viktigt hur vi ska välja $u$ för att få det. Därmed är $\frac{1}{G}$ en viktig storhet, och nollställen kan därmed orsaka reglerproblem som är svårlösta.

\paragraph{Impulssvar}
Om lösningen för $Y$ är på formen $Y = GU$, är lösningen för $y$ på formen
\begin{align*}
	y(t) = \integ{0}{t}{\tau}{g(\tau)u(t - \tau)}.
\end{align*}
$g$ kallas för impulssvaret.