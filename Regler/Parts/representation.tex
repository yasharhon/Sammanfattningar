\section{Tillståndsrepresentationer}

\paragraph{Ideen}
Den fundamentala ideen vi vill åt nu är att representera ett systems tillstånd på ett annat sätt än just dets tidsutveckling, till exempel som en vektor.

\paragraph{Representation av linjära system}
Tidsutvecklingen av ett system tillstånd kan i många linjära fall skrivas som
\begin{align*}
	\dot{\vb{x}} = A\vb{x} + \vb{b}u,
\end{align*}
där $\vb{u}$ beskriver en styrsignal. Utsignalen är typiskt på formen
\begin{align*}
	y = \vb{c}\cdot\vb{x} + du.
\end{align*}

\paragraph{Linjära system i Laplacedomänet}
Genom att laplacetransformera ekvationen som beskriver ett systems tillstånd fås
\begin{align*}
	s\vb{X} = A\vb{X} + \vb{b}U,
\end{align*}
givet att systemets starttillstånd är $\vb{0}$. Detta kan skrivas som
\begin{align*}
	\vb{X} = (sI - A)^{-1}B\vb{U}.
\end{align*}
Insatt i uttrycket för $Y$ fås
\begin{align*}
	Y = \vb{c}\cdot\vb{X} + d\vb{U} = (\vb{c}^{T}(sI - A)^{-1}B + d)\vb{U},
\end{align*}
och vi identifierar överförningsfunktionen som
\begin{align*}
	G = \vb{c}^{T}(sI - A)^{-1}\vb{b} + d.
\end{align*}
Av någon anledning är detta lika med
\begin{align*}
	G = \frac{1}{sI - A}\vb{c}^{T}(sI - A)^{\dagger}\vb{b} + d
\end{align*}

\paragraph{Poler i tillståndsrepresentation}
Det visar sig att systemets poler ges av
\begin{align*}
	\det(sI - A) = s^{2}.
\end{align*}
Detta ska tydligen motsvara $A$:s egenvärden.

\paragraph{Linjärisering}
Verkliga system är ofta olinjära, men vi ska försöka behandla dem som linjära ändå.

Betrakta ett system
\begin{align*}
	\dot{\vb{x}} = \vb{f}(\vb{x}, u),\ \dot{y} = g(\vb{x}, u).
\end{align*}
Anta konstant styrsignal $u_{0}$, och antag att systemet då tenderar mot ett stationärt tillstånd $\vb{x}_{0}$. Denna punkten uppfyller då
\begin{align*}
	\vb{f}(\vb{x}_{0}, u_{0}) = \vb{0},\ h(\vb{x}_{0}, u_{0}) = y_{0}.
\end{align*}
När vi linjäriserar, betraktar vi små variationer $\Delta\vb{x}, \Delta u, \Delta y$ kring denna punkten. Vi får
\begin{align*}
	\dv{\Delta\vb{x}}{t} = \vb{f}(\vb{x}_{0} + \Delta\vb{x}, u_{0} + \Delta u) = \vb{f}(\vb{x}_{0}, u_{0}) + \pdv{\vb{f}}{\vb{x}}\Delta\vb{x} + \del{u}{\vb{f}}\Delta u = \pdv{\vb{f}}{\vb{x}}\Delta\vb{x} + \del{u}{\vb{f}}\Delta u.
\end{align*}
På samma sätt fås
\begin{align*}
	\dv{\Delta y}{t} = h(\vb{x}_{0} + \Delta\vb{x}, u_{0} + \Delta u) - y_{0} = h(\vb{x}_{0}, u_{0}) - y_{0} + \grad_{x}{h}\cdot\Delta\vb{x} + \del{u}{h}\Delta u = \grad_{x}{h}\cdot\Delta\vb{x} + \del{u}{h}\Delta u.
\end{align*}
Det totala systemet
\begin{align*}
	\dv{\Delta\vb{x}}{t} &= \pdv{\vb{f}}{\vb{x}}\Delta\vb{x} + \del{u}{\vb{f}}\Delta u, \\
	\dv{\Delta y}{t}     &= \grad_{x}{h}\cdot\Delta\vb{x} + \del{u}{h}\Delta u
\end{align*}
är alltså linjärt för små ändringar.

\paragraph{Lösning av system i representation}
Betrakta ett system på formen
\begin{align*}
	\dot{\vb{x}} = A\vb{x} + \vb{b}u,\ \vb{x}(0) = \vb{x}_{0},\ y = \vb{c}\cdot\vb{x}.
\end{align*}
Vi använder att
\begin{align*}
	\dv{t}e^{-At} = -Ae^{-At},
\end{align*}
varför
\begin{align*}
	e^{-At}\dot{\vb{x}}              &= e^{-At}A\vb{x} + e^{-At}\vb{b}u, \\
	\dv{t}\left(e^{-At}\vb{x}\right) &= e^{-At}\vb{b}u, \\
	\vb{x}                           &= \vb{x}_{0}e^{At} + \integ{0}{t}{\tau}{e^{-A(t - \tau)}\vb{b}u}.
\end{align*}

\paragraph{Styrbarhet}
Ett tillstånd $\vb{x}$ är styrbart om man kan styra det motsvarande systemet från $\vb{0}$ till $\vb{x}$ med hjälp av en styrsignal $u$ på ändlig tid.

\paragraph{Test av styrbarhet}
Vi noterar först att Cayley-Hamiltons sats ger
\begin{align*}
	A^{n} + \sum\limits_{i = 1}^{n}a_{i}A^{n - i} = 0,
\end{align*}
där $a_{i}$ är koefficienterna i $A$:s karakteristiska polynom. Därmed kan alla potenser av $A$ av högre ordning skrivas som en linjärkombination av potenser av $A$ upp till och med $n - 1$. Därmed gäller det att
\begin{align*}
	e^{At} = \sum\limits_{i = 0}^{n - 1}f_{i}(t)A^{i}
\end{align*}
för några funktioner $f_{i}$.

Betrakta nu ett system och dets representation. Om systemets starttillstånd är $\vb{0}$, medför detta
\begin{align*}
	\vb{x} = \left(\sum\limits_{i = 0}^{n - 1}\gamma_{i}(t)A^{i}\right)\vb{b}
\end{align*}
där
\begin{align*}
	\gamma_{i} = \integ{0}{t}{\tau}{f_{i}(\tau)u(\tau)}.
\end{align*}
Med andra ord är de styrbara $\vb{x}$ linjärkombinationer av de olika $A^{i}\vb{b}$, alltså att det ligger i bildrummet till styrbarhetsmatrisen
\begin{align*}
	\S = [\vb{b}, A\vb{b}, \dots, A^{n - 1}\vb{b}].
\end{align*}
Systemet är styrbart om $\det(\S) \neq 0$.

\paragraph{Observerbarhet}
Ett tillstånd $\vb{x}$ är icke observerbart om utsignalen $y$ är identiskt noll då initialvärdet är $\vb{x}$ och insignalen identisk noll.

\paragraph{Test av observerbarhet}
Vi vill nu testa om ett tillstånd $\vb{x}_{0}$ är observerbart. Om vi har styrsignal $u = 0$, gäller det att
\begin{align*}
	\vb{x} = e^{At}\vb{x}_{0}, y = \vb{c}\cdot e^{At}\vb{x}_{0}, \dv[n]{y}{t} = \vb{c}\cdot A^{n}e^{At}\vb{x}_{0}.
\end{align*}
Speciellt är $y = 0$ för alla $t$ om
\begin{align*}
	y(0) = \vb{c}\cdot\vb{x}_{0} = 0, \dv{y}{t} = \vb{c}\cdot A\vb{x}_{0} = 0,\ \dots, \dv[n - 1]{y}{t} = \vb{c}\cdot A^{n - 1}\vb{x}_{0} = 0.
\end{align*}
Givet att detta stämmer, ger Cayley-Hamiltons sats att även högre ordningens derivator av $y$ kommer vara $0$. Därmed ligger de icke-observerbara tillstånden i nollrummet till observerbarhetsmatrisen
\begin{align*}
	\O =
	\mqty[
		\vb{c}^{T} \\
		\vb{c}^{T}A \\
		\vdots \\
		\vb{c}^{T}A^{n - 1}
	].
\end{align*}
Systemet är observerbart $\det(\O) \neq 0$.

\paragraph{Minimalhet}
Ett system är minimalt om och endast om det är både styrbart och observerbart.