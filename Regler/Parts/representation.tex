\section{Tillståndsrepresentationer}

\paragraph{Ideen}
Den fundamentala ideen vi vill åt nu är att representera ett systems tillstånd på ett annat sätt än just dets tidsutveckling, till exempel som en vektor.

\paragraph{Representation av linjära system}
Tidsutvecklingen av ett system tillstånd kan i många linjära fall skrivas som
\begin{align*}
	\dot{\vb{x}} = A\vb{x} + B\vb{u},
\end{align*}
där $\vb{u}$ beskriver en styrsignal. Utsignalen är typiskt på formen
\begin{align*}
	y = \vb{c}\cdot\vb{x} + \vb{d}\cdot\vb{u}.
\end{align*}

\paragraph{Linjära system i Laplacedomänet}
Genom att laplacetransformera ekvationen som beskriver ett systems tillstånd fås
\begin{align*}
	s\vb{X} = A\vb{X} + \vb{b}U,
\end{align*}
givet att systemets starttillstånd är $\vb{0}$. Detta kan skrivas som
\begin{align*}
	\vb{X} = (sI - A)^{-1}B\vb{U}.
\end{align*}
Insatt i uttrycket för $Y$ fås
\begin{align*}
	Y = \vb{c}\cdot\vb{X} + d\vb{U} = (\vb{c}^{T}(sI - A)^{-1}B + d)\vb{U},
\end{align*}
och vi identifierar överförningsfunktionen som
\begin{align*}
	G = \vb{c}^{T}(sI - A)^{-1}\vb{b} + d.
\end{align*}
Av någon anledning är detta lika med
\begin{align*}
	G = \frac{1}{sI - A}\vb{c}^{T}(sI - A)^{\dagger}\vb{b} + d
\end{align*}

\paragraph{Poler i tillståndsrepresentation}
Det visar sig att systemets poler ges av
\begin{align*}
	\det(sI - A) = s^{2}.
\end{align*}
Detta ska tydligen motsvara $A$:s egenvärden.

\paragraph{Linjärisering}
Verkliga system är ofta olinjära, men vi ska försöka behandla dem som linjära ändå.

Betrakta ett system
\begin{align*}
	\dot{\vb{x}} = \vb{f}(\vb{x}, u),\ \dot{y} = g(\vb{x}, u).
\end{align*}
Anta konstant styrsignal $u_{0}$, och antag att systemet då tenderar mot ett stationärt tillstånd $\vb{x}_{0}$. Denna punkten uppfyller då
\begin{align*}
	\vb{f}(\vb{x}_{0}, u_{0}) = \vb{0},\ h(\vb{x}_{0}, u_{0}) = y_{0}.
\end{align*}
När vi linjäriserar, betraktar vi små variationer $\Delta\vb{x}, \Delta u, \Delta y$ kring denna punkten. Vi får
\begin{align*}
	\dv{\Delta\vb{x}}{t} = \vb{f}(\vb{x}_{0} + \Delta\vb{x}, u_{0} + \Delta u) = \vb{f}(\vb{x}_{0}, u_{0}) + \pdv{\vb{f}}{\vb{x}}\Delta\vb{x} + \del{u}{\vb{f}}\Delta u = \pdv{\vb{f}}{\vb{x}}\Delta\vb{x} + \del{u}{\vb{f}}\Delta u.
\end{align*}
På samma sätt fås
\begin{align*}
	\dv{\Delta y}{t} = h(\vb{x}_{0} + \Delta\vb{x}, u_{0} + \Delta u) - y_{0} = h(\vb{x}_{0}, u_{0}) - y_{0} + \grad_{x}{h}\cdot\Delta\vb{x} + \del{u}{h}\Delta u = \grad_{x}{h}\cdot\Delta\vb{x} + \del{u}{h}\Delta u.
\end{align*}
Det totala systemet
\begin{align*}
	\dv{\Delta\vb{x}}{t} &= \pdv{\vb{f}}{\vb{x}}\Delta\vb{x} + \del{u}{\vb{f}}\Delta u, \\
	\dv{\Delta y}{t}     &= \grad_{x}{h}\cdot\Delta\vb{x} + \del{u}{h}\Delta u
\end{align*}
är alltså linjärt för små ändringar.