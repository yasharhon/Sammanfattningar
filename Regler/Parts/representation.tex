\section{Tillståndsrepresentationer}

\paragraph{Ideen}
Den fundamentala ideen vi vill åt nu är att representera ett systems tillstånd på ett annat sätt än just dets tidsutveckling, till exempel som en vektor.

\paragraph{Representation av linjära system}
Tidsutvecklingen av ett system tillstånd kan i många linjära fall skrivas som
\begin{align*}
	\dot{x} = Ax + Bu,
\end{align*}
där $u$ beskriver en styrsignal. Utsignalen är typiskt på formen
\begin{align*}
	y = Cx + Du.
\end{align*}

\paragraph{Linjära system i Laplacedomänet}
Genom att laplacetransformera ekvationen som beskriver ett systems tillstånd fås
\begin{align*}
	sX = AX + BU,
\end{align*}
givet att systemets starttillstånd är $\vb{0}$. Detta kan skrivas som
\begin{align*}
	X = (sI - A)^{-1}BU.
\end{align*}
Insatt i uttrycket för $Y$ fås
\begin{align*}
	Y = CX + DU = (C(sI - A)^{-1}B + D)U,
\end{align*}
och vi identifierar överförningsfunktionen som
\begin{align*}
	G = C(sI - A)^{-1}B + D.
\end{align*}
Av någon anledning är detta lika med
\begin{align*}
	G = \frac{1}{sI - A}C(sI - A)^{\dagger}B + D
\end{align*}

\paragraph{Poler i tillståndsrepresentation}
Det visar sig att systemets poler ges av
\begin{align*}
	\det(sI - A) = s^{2}.
\end{align*}
Detta ska tydligen motsvara $A$:s egenvärden.

\paragraph{Linjarisering}
Verkliga system är ofta olinjära, men vi ska försöka behandla dem som linjära ändå.

Betrakta ett system
\begin{align*}
	\dot{x} = f(x, u),\ \dot{y} = g(x, u).
\end{align*}
Anta konstant styrsignal $u_{0}$, och antag att systemet då tenderar mot ett stationärt tillstånd $\vb{x}_{0}$. Denna punkten uppfyller då
\begin{align*}
	f(x_{0}, u_{0}) = \vb{0},\ h(x_{0}, u_{0}) = y_{0}.
\end{align*}
När vi linjäriserar, betraktar vi små variationer $\Delta x, \Delta u, \Delta y$ kring denna punkten. Vi får
\begin{align*}
	\dv{\Delta x}{t} = f(x_{0} + \Delta x, u_{0} + \Delta u) = \vb{f}(x_{0}, u_{0}) + \pdv{f}{x}\Delta x + \del{u}{f}\Delta u = \pdv{f}{x}\Delta x + \del{u}{f}\Delta u.
\end{align*}
Observera att $\pdv{f}{x}$ allmänt är en matris.

På samma sätt fås
\begin{align*}
	\dv{\Delta y}{t} = h(x_{0} + \Delta x, u_{0} + \Delta u) - y_{0} = h(x_{0}, u_{0}) - y_{0} + \grad_{x}{h}\Delta x + \del{u}{h}\Delta u = \grad_{x}{h}\Delta x + \del{u}{h}\Delta u.
\end{align*}
Det totala systemet
\begin{align*}
	\dv{\Delta x}{t} &= \pdv{f}{x}\Delta x + \del{u}{f}\Delta u, \\
	\dv{\Delta y}{t}     &= \grad_{x}{h}\Delta x + \del{u}{h}\Delta u
\end{align*}
är alltså linjärt för små ändringar.

\paragraph{Lösning av system i representation}
Betrakta ett system på formen
\begin{align*}
	\dot{\vb{x}} = A\vb{x} + Bu,\ \vb{x}(0) = \vb{x}_{0},\ y = C\vb{x}.
\end{align*}
Vi använder att
\begin{align*}
	\dv{t}e^{-At} = -Ae^{-At},
\end{align*}
varför
\begin{align*}
	e^{-At}\dot{\vb{x}}              &= e^{-At}A\vb{x} + e^{-At}Bu, \\
	\dv{t}\left(e^{-At}\vb{x}\right) &= e^{-At}Bu, \\
	\vb{x}                           &= \vb{x}_{0}e^{At} + \integ{0}{t}{\tau}{e^{-A(t - \tau)}Bu}.
\end{align*}

\paragraph{Styrbarhet}
Ett tillstånd $\vb{x}$ är styrbart om man kan styra det motsvarande systemet från $\vb{0}$ till $\vb{x}$ med hjälp av en styrsignal $u$ på ändlig tid.

\paragraph{Test av styrbarhet}
Vi noterar först att Cayley-Hamiltons sats ger
\begin{align*}
	A^{n} + \sum\limits_{i = 1}^{n}a_{i}A^{n - i} = 0,
\end{align*}
där $a_{i}$ är koefficienterna i $A$:s karakteristiska polynom. Därmed kan alla potenser av $A$ av högre ordning skrivas som en linjärkombination av potenser av $A$ upp till och med $n - 1$. Därmed gäller det att
\begin{align*}
	e^{At} = \sum\limits_{i = 0}^{n - 1}f_{i}(t)A^{i}
\end{align*}
för några funktioner $f_{i}$.

Betrakta nu ett system och dets representation. Om systemets starttillstånd är $\vb{0}$, medför detta
\begin{align*}
	\vb{x} = \left(\sum\limits_{i = 0}^{n - 1}\gamma_{i}(t)A^{i}\right)\vb{b}
\end{align*}
där
\begin{align*}
	\gamma_{i} = \integ{0}{t}{\tau}{f_{i}(\tau)u(\tau)}.
\end{align*}
Med andra ord är de styrbara $\vb{x}$ linjärkombinationer av de olika $A^{i}\vb{b}$, alltså att det ligger i bildrummet till styrbarhetsmatrisen
\begin{align*}
	\S = [\vb{b}, A\vb{b}, \dots, A^{n - 1}\vb{b}].
\end{align*}
Systemet är styrbart om $\det(\S) \neq 0$.

\paragraph{Observerbarhet}
Ett tillstånd $\vb{x}$ är icke observerbart om utsignalen $y$ är identiskt noll då initialvärdet är $\vb{x}$ och insignalen identisk noll.

\paragraph{Test av observerbarhet}
Vi vill nu testa om ett tillstånd $\vb{x}_{0}$ är observerbart. Om vi har styrsignal $u = 0$, gäller det att
\begin{align*}
	\vb{x} = e^{At}\vb{x}_{0}, y = \vb{c}\cdot e^{At}\vb{x}_{0}, \dv[n]{y}{t} = \vb{c}\cdot A^{n}e^{At}\vb{x}_{0}.
\end{align*}
Speciellt är $y = 0$ för alla $t$ om
\begin{align*}
	y(0) = \vb{c}\cdot\vb{x}_{0} = 0, \dv{y}{t} = \vb{c}\cdot A\vb{x}_{0} = 0,\ \dots, \dv[n - 1]{y}{t} = \vb{c}\cdot A^{n - 1}\vb{x}_{0} = 0.
\end{align*}
Givet att detta stämmer, ger Cayley-Hamiltons sats att även högre ordningens derivator av $y$ kommer vara $0$. Därmed ligger de icke-observerbara tillstånden i nollrummet till observerbarhetsmatrisen
\begin{align*}
	\O =
	\mqty[
		\vb{c}^{T} \\
		\vb{c}^{T}A \\
		\vdots \\
		\vb{c}^{T}A^{n - 1}
	].
\end{align*}
Systemet är observerbart $\det(\O) \neq 0$.

\paragraph{Minimalhet}
Ett system är minimalt om och endast om det är både styrbart och observerbart.

\paragraph{Proportionell återkoppling}
Antag att vi återkopplar systemet med $\vb{u} = l_{0}\vb{r} - L\vb{x}$. Om utsignalen ej beror av $\vb{u}$ kan det återkopplade systemet skrivas som
\begin{align*}
	\dot{\vb{x}} = (A - BL)\vb{x} + Bl_{0}\vb{r},\ y = C\vb{x}.
\end{align*}

Systemets poler ges av egenvärdena till $A - BL$. Det finns $n$ såna, och $L$ har $n$ parametrar. Om systemet är styrbart kan dets poler därmed placeras godtyckligt vid lämpligt val av $L$.

Valet av $l_{0}$ görs så att $y = r$ när systemet är stationärt. Detta kräver dock att man känner $G(0)$ och att inga störningar påverkar systemet. Därför inför vi I-reglering.

\paragraph{I-reglering}
När vi I-reglerar, inför vi extra tillstånd
\begin{align*}
	x_{n + 1} = \integ{0}{t}{\tau}{e}.
\end{align*}
Detta ger
\begin{align*}
	\dot{x}_{n + 1} = r - y = r - Cx.
\end{align*}
Då kan vi utvidga modellen till
\begin{align*}
	\dot{\vb{x}} =
	\mqty[
		\dot{x} \\
		\dot{x}_{n + 1}
	]
	=
	\mqty[
		A  & 0 \\
		-C & 0
	]
	\mqty[
		x \\
		x_{n + 1}
	] +
	\mqty[
		B \\
		0
	]
	u
	+
	\mqty[
		0 \\
		1
	]
	r = A\vb{x} + Bu + 
	\mqty[
		0 \\
		1
	]
	r.
\end{align*}
Strategin är nu att återkoppla det nya systemet med återkoppling på formen
\begin{align*}
	u = -Lx - l_{n + 1}x_{n + 1} = -L\vb{x}.
\end{align*}
Då kan $L$ väljas så att $A - BL$ får önskade egenvärden. Stationärt har vi
\begin{align*}
	\dot{x} = 0,\ 	\dot{x}_{n + 1} = 0.
\end{align*}

\paragraph{Skattning av tillstånd}
Om man ej kan mäta systemets tillstånd exakt, kan man skatta det.

Mer precist, antag att vi har en modell
\begin{align*}
	\dot{x} = Ax + Bu,\ y = Cx
\end{align*}
som simulerar
\begin{align*}
	\dot{\hat{x}} = A\hat{x} + Bu,\  \hat{y} = C\hat{x}.
\end{align*}
Felsignalen är
\begin{align*}
	y - \hat{y} = y - C\hat{x}.
\end{align*}
Vi försöker återkoppla systemet. Det beskrivs då av
\begin{align*}
	\dot{\hat{x}} = A\hat{x} + Bu + K(y- C\hat{x}) = (A - KC)C\hat{x} + Bu + Ky.
\end{align*}
Skattningsfelet $\tilde{x}$ ges då av
\begin{align*}
	\dot{\tilde{x}} = \dot{x} - \dot{\hat{x}} = Ax + Bu - A\hat{x} - Bu - K(Cx - C\hat{x}) = (A - KC)\tilde{x}.
\end{align*}
Detta har lösning
\begin{align*}
	\tilde{x} = e^{(A - KC)t}\tilde{x}(0).
\end{align*}
Felet tenderar mot $0$ egenvärdena till $A - KC$ är negativa, och hur snabbt det tenderar mot $0$ beror av egenvärdenas belopp.

Mätfelet ges av $y_{\text{m}} = y + e$. Detta ger
\begin{align*}
	\dot{\tilde{x}} = (A - KC)\tilde{x} + Ke.
\end{align*}
Stora $K$ ger som vi ser höga mätfel, men det ger även snabba system. Optimeringen där görs av ett Kalmanfilter. Vi väljer då systemet så att egenvärdena till $A - BL$ har lägre belopp än egenvärdena till $A - KC$.

Vid återkoppling av såna system är överförningsfunktionen för det slutna systemet
\begin{align*}
	G = C(sI - (A - BL))^{-1}Bl_{0},
\end{align*}
alltså likadant som tidigare.