\section{Kompensering}

\paragraph{Ideen}
Vi ser att det enklaste sättet att konstruera en bra regulator på är att ändra konstruktionen av det öppna systemet. Vi bestämmer alltså regulatorn $F$ utifrån krav på
\begin{itemize}
	\item snabbhet, alltså skärfrekvens.
	\item dämpning, alltså fasmarginal.
	\item stationärt fel, alltså krav på $\abs{G_{\text{O}}(0)}$.
\end{itemize}

\paragraph{Kompensation för snabbhet}
För snabbhet räcker det med en P-regulator. Denna flyttar amplitudkurvan, men ändrar ej faskurvan. Alltså hjälper den oss att bestämma skärfrekvensen.

\paragraph{Dämpning}
Här kan man använda en PD-regulator. Den fixar fasmarginalen.

\paragraph{Stationärt fel}
Här kan man använda en PI-regulator. Denna kan dock förstöra fasmarginalen.

\paragraph{Arbetsgång}
Arbetsgången i kompensering är att
\begin{itemize}
	\item bestämma önskad bandbredd.
	\item bestämma önskad fasmarginal för att ge nödvändig fasökning vid skärfrekvensen.
	\item Gör lead- och laggrejer. Jag kanske borde fatta det.
\end{itemize}