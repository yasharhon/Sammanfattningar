\section{Kompensering}

\paragraph{Ideen}
Vi ser att det enklaste sättet att konstruera en bra regulator på är att ändra konstruktionen av det öppna systemet. Vi bestämmer alltså regulatorn $F$ utifrån krav på
\begin{itemize}
	\item snabbhet, alltså skärfrekvens.
	\item dämpning, alltså fasmarginal.
	\item stationärt fel, alltså krav på $\abs{G_{\text{O}}(0)}$.
\end{itemize}

\paragraph{Kompensation för snabbhet}
För snabbhet räcker det med en P-regulator. Denna flyttar amplitudkurvan, men ändrar ej faskurvan. Alltså hjälper den oss att bestämma skärfrekvensen.

\paragraph{Fasavancering}
För att höja fasen kan man använda en deriverande länk, alltså en regulator med överföringsfunktion
\begin{align*}
	F = K(\tau_{\text{D}}s + 1).
\end{align*}
Typiskt kan man inte låta deriveringen verka fullt ut, så överförningsfunktionen blir i stället på formen
\begin{align*}
	F_{\text{lead}} = K\frac{\tau_{\text{D}}s + 1}{\beta\tau_{\text{D}}s + 1}.
\end{align*}
Vi har
\begin{align*}
	\arg{F_{\text{lead}}} = \dots = \arctan{\frac{(1 - \beta)\tau_{\text{D}}\omega}{1 + \beta\tau_{\text{D}}^{2}\omega^{2}}},
\end{align*}
och får därmed att den maximala fasförskjutningen är
\begin{align*}
	\phi_{\text{max}} = \arctan{\frac{1 - \beta}{2\sqrt{\beta}}}
\end{align*}
för frekvensen
\begin{align*}
	\omega = \frac{1}{\sqrt{\beta}\tau_{\text{D}}}.
\end{align*}

\paragraph{Lågfrekvensförstärkning}
Lågfrekvensförstärkning kan ta bort stationärt fel. För att lågfrekvensförstarka kan man använda en integrerande länk, alltså en regulator med överförningsfunktion
\begin{align*}
	F = \frac{\tau_{1}s + 1}{\tau_{\text{D}}s}.
\end{align*}
Denna har dock oändligt hög förstärkning för låga frekvenser, kan man i stället använda en fasretarderande länk, alltså en regulator med överförningsfunktion
\begin{align*}
	F_{\text{lag}} = \frac{\tau_{1}s + 1}{\beta\tau_{1}s + \gamma}.
\end{align*}
Vi har
\begin{align*}
	\arg{F_{\text{lag}}} = \dots = -\arctan{\frac{(1 - \gamma)\tau_{1}\omega_{\text{c}}}{\gamma + \tau_{1}^{2}\omega_{\text{c}}^{2}}}.
\end{align*}
Det kan däremot vara svårt att göra rätt val av parametrar.

\paragraph{Arbetsgång}
Arbetsgången i kompensering är att
\begin{itemize}
	\item bestämma önskad bandbredd.
	\item bestämma önskad fasmarginal för att ge nödvändig fasökning vid skärfrekvensen.
	\item Gör lead- och laggrejer. Jag kanske borde fatta det.
\end{itemize}