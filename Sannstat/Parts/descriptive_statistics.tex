\section{Deskriptiv statistik}
Definitionerna som dyker upp i denna del kan virka redundanta, men det är underförstått att detta är punktskattningar av parametrar och inte själva parametrarna som definieras här.

\subsection{Definitioner}

\paragraph{Punktskattningar}
En punktskattning av en parameter $\theta$ är en funktion av utfallen $x_1, \dots, x_n$ av dom stokastiska variablerna $X_1, \dots, X_n$ vars fördelning beror av $\theta$. Därmed är punktskattningen ett utfall av stickprovsvariabeln $\theta^*$.

\paragraph{Väntevärdesriktighet}
En punktskattning är väntevärdesriktig om $\expec{\theta^*} = \theta$.

\paragraph{Konsistens}
Punktskattningen $\theta^*$ är konsistent om det för varje $\theta$ och $\varepsilon > 0$ gäller att
\begin{align*}
	\lim\limits_{n\to\infty}P(\abs{\theta_{n}^* - \theta} > \varepsilon) = 0.
\end{align*}

\paragraph{Medelkvadratfel}
Medelkvadratfelet definieras som $\expec{(\theta^* - \theta)^2}$.

\paragraph{Medelfel}
Medelfelet definieras som en skattning av $\dev{\theta^*}$, och betecknas $\text{d}(\theta^*)$.

\paragraph{Effektivitet}
Om två skattningar $\theta^*, \hat{\theta}$ uppfyller $\var{\theta^*}\leq\var{\hat{\theta}}$ är $\theta^*$ effektivare än $\hat{\theta}$.

\paragraph{Medelvärde}
Medelvärdet definieras som
\begin{align*}
	\bar{x} = \frac{1}{n}\sum x_i.
\end{align*}

\paragraph{Varians}
Variansen definieras som
\begin{align*}
	s^2 = \frac{1}{n - 1}\sum (x_i - \bar{x})^2,
\end{align*}
med en analog definition av standardavvikelsen $s$.

\paragraph{Kovarians}
Kovariansen definieras som
\begin{align*}
	c_{xy} = \frac{1}{n - 1}\sum (x_i - \bar{x})(y_i - \bar{y}).
\end{align*}

\paragraph{Korrelationskoefficient}
Korrelationskoefficienten definieras som
\begin{align*}
	r = \frac{c_{xy}}{s_{x}s_{y}}.
\end{align*}

\paragraph{Konfidensintervall}
Intervallet $I_{\theta}$ som med sannolikhet $1 - \alpha$ täcker över den okända parametern $\theta$ kallas konfidensintervallet för $\theta$ med konfidensgrad $1 - \alpha$.

\subsection{Satser}

\paragraph{Medelvärdets egenskaper}
Medelvärdet är en konsistent och väntevärdesriktig skattning av en stokastisk variabels väntevärde.

\proof

\paragraph{Variansens egenskaper}
Variansen är en konsistent och väntevärdesriktig skattning av en stokastisk variabels varians.

\proof

\paragraph{Konfidensintervall för väntevärde, känd varians}
Låt $X_1, \dots, X_{n}$ vara normalfördelade med väntevärde $\mu$ och varians $\sigma$. Då är
\begin{align*}
	I_{\mu} = \left[\bar{x} - \lambda_{\frac{\alpha}{2}}\frac{\sigma}{\sqrt{n}}, \bar{x} + \lambda_{\frac{\alpha}{2}}\frac{\sigma}{\sqrt{n}}\right]
\end{align*}
ett konfidensintervall för väntevärdet med konfidensgrad $1 - \alpha$, där $\lambda_{\frac{\alpha}{2}}$ är $\frac{\alpha}{2}$-kvantilen i normalfördelningen.

\proof

\paragraph{Konfidensintervall för väntevärde, okänd varians}
Låt $X_1, \dots, X_{n}$ vara normalfördelade med väntevärde $\mu$. Då är
\begin{align*}
	I_{\mu} = \left[\bar{x} - t_{\frac{\alpha}{2}}(n + 1)\frac{s}{\sqrt{n}}, \bar{x} + t_{\frac{\alpha}{2}}(n + 1)\frac{s}{\sqrt{n}}\right]
\end{align*}
ett konfidensintervall för väntevärdet, där där $t_{\frac{\alpha}{2}}(n + 1)$ är $\frac{\alpha}{2}$-kvantilen i $t$-fördelningen med $n - 1$ frihetsgrader.

\proof