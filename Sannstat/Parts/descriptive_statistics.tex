\section{Deskriptiv statistik}
Definitionerna som dyker upp i denna del kan virka redundanta, men det är underförstått att detta är punktskattningar av parametrar och inte själva parametrarna som definieras här.

\subsection{Definitioner}

\paragraph{Punktskattningar}
En punktskattning av en parameter $\theta$ är en funktion av utfallen $x_1, \dots, x_n$ av de stokastiska variablerna $X_1, \dots, X_n$ vars fördelning beror av $\theta$. Därmed är punktskattningen ett utfall av stickprovsvariabeln $\theta^*$.

\paragraph{Väntevärdesriktighet}
En punktskattning är väntevärdesriktig om $\expec{\theta^*} = \theta$.

\paragraph{Konsistens}
Punktskattningen $\theta^*$ är konsistent om det för varje $\theta$ och $\varepsilon > 0$ gäller att
\begin{align*}
	\lim\limits_{n\to\infty}P(\abs{\theta_{n}^* - \theta} > \varepsilon) = 0.
\end{align*}

\paragraph{Medelkvadratfel}
Medelkvadratfelet definieras som $\expec{(\theta^* - \theta)^2}$.

\paragraph{Medelfel}
Medelfelet definieras som en skattning av $\dev{\theta^*}$, och betecknas $\text{d}(\theta^*)$.

\paragraph{Effektivitet}
Om två skattningar $\theta^*, \hat{\theta}$ uppfyller $\var{\theta^*}\leq\var{\hat{\theta}}$ är $\theta^*$ effektivare än $\hat{\theta}$.

\paragraph{Likelihood-funktionen}
Funktionen
\begin{align*}
	L(\theta) =
	\begin{cases}
		P(X_1 = x_i, \dots, X_n = x_n),\ \text{diskret fall}, \\
		f_{X_1, \dots, X_n}(x_1, \dots, x_1), \ \text{kontinuerligt fall},
	\end{cases}
\end{align*}
givet att $\theta$ är den sanna parametern, är likelihood-funktionen.

\paragraph{Maximum likelihood-skattningar}
Maximumet till $L(\theta)$ är maximum likelihood-skattningen av parametern $\theta$.

\paragraph{Minsta kvadrat-skattningar}
Låt $x_1, \dots, x_n$ vara utfall av $X_1, \dots, X_n$ med väntevärden $\expec{X_i} = \mu_i(\theta)$ och definiera kvadratfelet
\begin{align*}
	Q = \sum\limits_{i = 1}^{n}(x_i - \mu_i(\theta).
\end{align*}
Det värdet av $\theta$ som minimerar $Q$ är minsta kvadrat-skattningen av $\theta$.

\paragraph{Medelvärde}
Medelvärdet definieras som
\begin{align*}
	\bar{x} = \frac{1}{n}\sum x_i.
\end{align*}

\paragraph{Varians}
Variansen definieras som
\begin{align*}
	s^2 = \frac{1}{n - 1}\sum (x_i - \bar{x})^2,
\end{align*}
med en analog definition av standardavvikelsen $s$.

\paragraph{Kovarians}
Kovariansen definieras som
\begin{align*}
	c_{xy} = \frac{1}{n - 1}\sum (x_i - \bar{x})(y_i - \bar{y}).
\end{align*}

\paragraph{Korrelationskoefficient}
Korrelationskoefficienten definieras som
\begin{align*}
	r = \frac{c_{xy}}{s_{x}s_{y}}.
\end{align*}

\paragraph{Konfidensintervall}
Intervallet $I_{\theta}$ som med sannolikhet $1 - \alpha$ täcker över den okända parametern $\theta$ kallas konfidensintervallet för $\theta$ med konfidensgrad $1 - \alpha$.

\subsection{Satser}

\paragraph{Medelvärdets egenskaper}
Medelvärdet är en konsistent och väntevärdesriktig skattning av en stokastisk variabels väntevärde.

\proof
Väntevärdesriktigheten följer direkt från väntevärdets egenskaper.

\paragraph{Variansens egenskaper}
Variansen är en konsistent och väntevärdesriktig skattning av en stokastisk variabels varians.

\proof

\paragraph{Konfidensintervall för väntevärde, känd varians}
Låt $X_1, \dots, X_{n}$ vara normalfördelade med väntevärde $\mu$ och varians $\sigma$. Då är
\begin{align*}
	I_{\mu} = \left[\bar{x} - \lambda_{\frac{\alpha}{2}}\frac{\sigma}{\sqrt{n}}, \bar{x} + \lambda_{\frac{\alpha}{2}}\frac{\sigma}{\sqrt{n}}\right]
\end{align*}
ett konfidensintervall för väntevärdet med konfidensgrad $1 - \alpha$, där $\lambda_{\frac{\alpha}{2}}$ är $\frac{\alpha}{2}$-kvantilen i normalfördelningen.

\proof
Vi har att $\mean{X}\in\text{N}\left(\mu, \frac{\sigma}{\sqrt{n}}\right)$. Detta betyder att
\begin{align*}
	P\left(-\lambda_{-\frac{\alpha}{2}} < \frac{\mean{X} - \mu}{\frac{\sigma}{\sqrt{n}}} < \lambda_{\frac{\alpha}{2}}\right) = 1 - \alpha.
\end{align*}
Från detta får vi två olikheter som ger det givna konfidensintervallet.

\paragraph{Konfidensintervall för väntevärde, okänd varians}
Låt $X_1, \dots, X_{n}$ vara normalfördelade med väntevärde $\mu$. Då är
\begin{align*}
	I_{\mu} = \left[\bar{x} - t_{\frac{\alpha}{2}}(n - 1)\frac{s}{\sqrt{n}}, \bar{x} + t_{\frac{\alpha}{2}}(n - 1)\frac{s}{\sqrt{n}}\right]
\end{align*}
ett konfidensintervall för väntevärdet, där där $t_{\frac{\alpha}{2}}(n + 1)$ är $\frac{\alpha}{2}$-kvantilen i $t$-fördelningen med $n - 1$ frihetsgrader.

\proof
Vi har att
\begin{align*}
	\frac{\mean{X} - \mu}{\frac{\sqrt{\frac{1}{n - 1}\sum (X_i - \mean{X})^2}}{\sqrt{n}}}\in t(n - 1).
\end{align*}
Därmed har vi
\begin{align*}
	P\left(-t_{-\frac{\alpha}{2}} < \frac{\mean{X} - \mu}{\frac{\sqrt{s^2}}{\sqrt{n}}}\in t(n - 1) < t_{\frac{\alpha}{2}}\right) = 1 - \alpha,
\end{align*}
där vi använder beteckningen $s^2$ för skattningen av standardavvikelsen och använder $t$-fördelningens symmetri. Detta ger två olikheter som ger det givna konfidensintervallet.

\paragraph{Konfidensintervall för standardavvikelse, okänd medelvärde}
Låt $X_1, \dots, X_{n}$ vara normalfördelade med standardavvikelse $\sigma$. Då är
\begin{align*}
	I_{\mu} = \left[\sqrt{\frac{n - 1}{\chi_{\frac{\alpha}{2}}^2(n - 1)}}s, \sqrt{\frac{n - 1}{\chi_{1 - \frac{\alpha}{2}}^2(n - 1)}}s\right]
\end{align*}
ett konfidensintervall för väntevärdet, där $\chi_{\frac{\alpha}{2}}^2(n - 1)$ är $\frac{\alpha}{2}$-kvantilen i $\chi^2$-fördelningen med $n - 1$ frihetsgrader.

För stora $n$ kan man skriva intervallen som
\begin{align*}
	\left[1 - \frac{\lambda_{\frac{\alpha}{2}}}{\sqrt{2(n - 1)}}, 1 + \frac{\lambda_{\frac{\alpha}{2}}}{\sqrt{2(n - 1)}}\right]
\end{align*}

\proof

\paragraph{Konfidensintervall för differans mellan väntevärden för olika objekt}
Låt $X_1, \dots, X_{n_1}\in N(\mu_1, \sigma_1)$ och $Y_1, \dots, Y_{n_2}\in N(\mu_2, \sigma_2)$, där dessa kan betraktas som stickprov från två olika objekt. Då gäller att:
\begin{itemize}
	\item Om $\sigma_1, \sigma_2$ är kända är
	\begin{align*}
		\left[\bar{x} - \bar{y} - \lambda_{\frac{\alpha}{2}}D, \bar{x} - \bar{y} + \lambda_{\frac{\alpha}{2}}D\right]
	\end{align*}
	ett konfidensintervall för $\mu_1 - \mu_2$ med konfidensgrad $1 - \alpha$, där $D = \sqrt{\frac{\sigma_1^2}{n_1} + \frac{\sigma_2^2}{n_2}}$.
	
	\item Om $\sigma_1 = \sigma_2 = \sigma$ är okända är
	\begin{align*}
		\left[\bar{x} - \bar{y} - t_{\frac{\alpha}{2}}(f)d, \bar{x} - \bar{y} + t_{\frac{\alpha}{2}}(f)d\right]
	\end{align*}
	ett konfidensintervall för $\mu_1 - \mu_2$ med konfidensgrad $1 - \alpha$, där $d = s\sqrt{\frac{1}{n_1} + \frac{1}{n_2}}$ och $f = n_1 + n_2 - 2$.
\end{itemize}

\proof

\paragraph{Konfidensintervall för differans mellan väntevärden för och efter}
Låt $X_1, \dots, X_{n_1}\in N(\mu_1, \sigma_1)$ och $Y_1, \dots, Y_{n_2}\in N(\mu_2, \sigma_2)$, där samma $i$ motsvarar stickprov från två olika objekt. Då gäller att:
\begin{itemize}
	\item Om vi definierar $Z_i = X_i - Y_i$, är
	\begin{align*}
		I_{\mu} = \left[\bar{z} - t_{\frac{\alpha}{2}}(n - 1)\frac{s}{\sqrt{n}}, \bar{z} + t_{\frac{\alpha}{2}}(n - 1)\frac{s}{\sqrt{n}}\right]
	\end{align*}
	ett konfidensintervall för $\mu_1 - \mu_2$, där $s$ är skattningen av standardavvikelsen från de olika $Z_i$.

	\item Om $\sigma_1, \sigma_2$ är okända är
	\begin{align*}
		\left[\bar{x} - \bar{y} - \lambda_{\frac{\alpha}{2}}d, \bar{x} - \bar{y} + \lambda_{\frac{\alpha}{2}}d\right]
	\end{align*}
	ett konfidensintervall för $\mu_1 - \mu_2$ med approximativ konfidensgrad $1 - \alpha$, där $d = \sqrt{\frac{s_1^2}{n_1} + \frac{s_2^2}{n_2}}$.
\end{itemize}

\proof

\paragraph{Allmän skattning av normalfördelade stokastiska variabler}
Låt skattningen av en parameter $\theta$ vara normalfördelad med väntevärde $\theta$ och standardavvikelse $D$. Då beräknas konfidensintervall med approximativ konfidensgrad $1 - \alpha$ som
\begin{itemize}
	\item
	\begin{align*}
		\left[\theta* - \lambda_{\frac{\alpha}{2}}D, \theta* + \lambda_{\frac{\alpha}{2}}D\right]
	\end{align*}
	om $D$ ej beror av $\theta$.
	
	\item
	\begin{align*}
		\left[\theta* - \lambda_{\frac{\alpha}{2}}d, \theta* + \lambda_{\frac{\alpha}{2}}d\right]
	\end{align*}
	om $d$ beror av $\theta$, för något lämpligt val av $d$.
\end{itemize}

\proof

\paragraph{Felförplantning}
Givet medelfelet till någon skattning av en parameter $\theta$, önskar vi nu att estimera medelfelet och väntevärdet av skattningen av någon funktion av $\theta$. Vi skriver denna som $\psi = g(\theta)$.

Första satsen vi har säjer att om $\theta^{*}$ är en approximativt väntevärdesriktig skattning av $\theta$ med medelfel $d(\theta^{*})$, är $\psi^{*} = g(\theta^{*})$ en approximativt väntevärdesriktig skattning av $\psi = g(\theta)$, eventuellt med korrektionstermen $\frac{1}{2}d^2(\theta^{*})\dv[2]{g}{{\theta^{*}}} (\theta^{*})$. Dens medelfel ges av
\begin{align*}
	d(\psi^{*})\approx\abs{\dv{g}{\theta^{*}} (\theta^{*})}d(\theta^{*}).
\end{align*}
I fallet där $\psi^{*}$ beror av två variabler $\theta^{*}$ och $\eta^{*}$, gäller ett motsvarande kriterie. Om kritereiet uppfylls, ges väntevärdet på motsvarande vis och medelfelet ges då av
\begin{align*}
	d^2(\psi^{*})\approx\left(\pdv[2]{g}{{\theta^{*}}} (\theta^{*}, \eta^{*})\right)^2d^2(\theta^{*}) + \left(\pdv[2]{g}{{\eta^{*}}} (\theta^{*}, \eta^{*})\right)^2d^2(\eta^{*}).
\end{align*}

\proof