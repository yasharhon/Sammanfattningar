\section{Stokastiska variabler}

\subsection{Definitioner}

\paragraph{Stokastiska variabler}
En stokastisk variabel är en funktion definierad på ett utfallsrum.

\paragraph{Diskreta stokastiska variabler}
En stokastisk variabel är diskret om den kan anta ett ändligt eller uppräkneligt oändligt antal värden.

\paragraph{Kontinuerliga stokastiska variabler}
En stokastisk variabel är kontinuerlig om det finns en funktion $f$ så att
\begin{align*}
	P(X\in A) = \integ{A}{}{f}{x}\forall A,
\end{align*}
eller motsvarande i flera variabler.

\paragraph{Sannolikhetsfunktioner}
Låt $X$ vara en diskret stokastisk variabel. Då definieras sannolikhetsfunktionen som
\begin{align*}
	p(k) = P(X = k).
\end{align*}

\paragraph{Täthetsfunktioner}
Låt $X$ vara en kontinuerlig stokastisk variabel. Då definieras täthetsfunktionen som en funktion $f$ som uppfyller
\begin{align*}
	P(X\in A)          &= \integ{A}{}{f}{x}\ \forall\ A, \\
	f(x)               &\geq 0\ \forall\ x, \\
	\integ{\R}{}{f}{x} &= 1.
\end{align*}

\paragraph{Sannolikhetsfunktioner i flera variabler}
Låt $(X, Y)$ vara en diskret stokastisk variabel. Då definieras sannolikhetsfunktionen som
\begin{align*}
	p(j, k) = P(X = j, Y = k).
\end{align*}
För kontinuerliga stokastiska variabler definieras den enligt
\begin{align*}
	P(X\in A) = \integ{A}{}{f}{x}
\end{align*}
och uppfyller $f(x)\geq 0\ \forall\ x$ och
\begin{align*}
	\integ{\R}{}{f}{x} = 1.
\end{align*}

\paragraph{Täthetsfunktioner i flera variabler}
Låt $(X, Y)$ vara en kontinuerlig stokastisk variabel. Då definieras täthetsfunktionen som en funktion $f$ som uppfyller
\begin{align*}
	P(X\in A)            &= \int\limits_{A}f(x, y)\dd{x}\dd{y}\ \forall\ A, \\
	f(x, y)              &\geq 0\ \forall\ x, y, \\
	\integ{\R^2}{}{f}{x} &= 1.
\end{align*}

\paragraph{Fördelningsfunktioner}
Låt $X$ vara en stokastisk variabel. Funktionen $F: x\to P(X\leq x)$ är fördelningsfunktionen för $X$.

\paragraph{Fördelningsfunktioner i flera variabler}
Låt $(X, Y)$ vara en tvådimensionell stokastisk variabel. Funktionen $F_{X, Y}: (x, y)\to P(X\leq x, Y\leq y)$ är den simultana fördelningsfunktionen för $(X, Y)$.

\paragraph{Marginalfördelningar}
Låt $p_{X, Y}$ vara sannolikhetsfunktionen till den stokastiska variabeln $(X, Y)$. Marginalfördelningnen $p_{X}$ till $X$ definieras då som
\begin{align*}
	p_{X}(j) = \sum\limits_{k}p(j, k)
\end{align*}
i det kontinuerliga fallet och
\begin{align*}
	f_{X}(x) = \int\limits_{\R}f(x, y)\dd{y}.
\end{align*}
En konsekvens av definitionen i det kontinuerliga fallet är
\begin{align*}
	F_{X}(x) = \lim\limits_{y\to\infty}f(x, y).
\end{align*}

\paragraph{Oberoende stokastiska variabler}
Variablerna $X, Y$ är oberoende om
\begin{align*}
	P(X\in C, Y\in D) = P(X\in C)P(Y\in D)\ \forall\ C, D.
\end{align*}

\paragraph{Väntevärde}
Låt $X$ vara en stokastisk variabel med sannolikhetsfunktion $p$. Då definieras variabelns väntevärde som
\begin{align*}
	\expec{E} = \sum kp(k).
\end{align*}
För en kontinuerlig stokastisk variabel definieras det som
\begin{align*}
	\expec{E} = \integ{\R}{}{xf(x)}{x}.
\end{align*}

\paragraph{Varians}
Låt $X$ vara en stokastisk variabel med väntevärde $\mu$. Variansen till $X$ definieras som
\begin{align*}
	\sigma^2 = \expec{(X - \mu)^2}.
\end{align*}

\paragraph{Standardavvikelse}
Låt $X$ vara en stokastisk variabel med varians $\sigma^2$. Standardavvikelsen till $X$ definieras som
\begin{align*}
	\sigma = \sqrt{\sigma^2}.
\end{align*}

\paragraph{Variationskoefficient}
Låt $X$ vara en stokastisk variabel med väntevärde $\mu$ och standardavvikelse$\sigma^2$. Variationskoeficienten till $X$ definieras som
\begin{align*}
	R = \frac{\sigma}{\mu}.
\end{align*}

\paragraph{Kovarians}
Låt $X, Y$ vara stokastiska variablerm ed väntevärden $\mu_{X}, \mu_{Y}$. Då definieras kovariansen mellan dessa som
\begin{align*}
	\cov{X}{Y} = \expec{(X - \mu_{X})(Y - \mu_{Y})}.
\end{align*}

\paragraph{Okorrellerade variabler}
$X, Y$ är okorrelerade om $\cov{X}{Y} = 0$.

\paragraph{Korrelationskoefficient}
Låt $X, Y$ vara stokastiska variabler. Då definieras korrelationskoefficienten mellan dessa som
\begin{align*}
	\rho(X, Y) = \frac{\cov{X}{Y}}{\var{X}\var{Y}}.
\end{align*}

\paragraph{Kvantiler}
Lösningen till
\begin{align*}
	F(x) = 1 - \alpha
\end{align*}
kallas $\alpha$-kvantilen till $X$.

\paragraph{Standardiserade stokastiska variabler}
Låt $X$ vara en stokastisk variabel med väntevärde $\mu$ och standardavvikelse $\sigma$. Då är $Y = \frac{X - \mu}{sigma}$ en standardiserad variabel.

\subsection{Satser}

\paragraph{Fördelningsfunktioners egenskaper}
Låt $F$ vara en fördelningsfunktion. Då gäller att
\begin{itemize}
	\item
	\begin{align*}
		F(x)\to
		\begin{cases}
			0, x\to -\infty, \\
			1, x\to\infty.
		\end{cases}
	\end{align*}
	\item $F$ är växande (eller icke-avtagande för kontinuerliga stokastiska variabler).
	\item $F$ är kontinuerlig till höger för varje $X$.
\end{itemize}
Omvänt gäller även att alla funktioner som uppfyller dessa egenskaper är fördelningsfunktioner.

\proof

\paragraph{Fördelningsfunktioner och sannolikheter}
Låt $F$ vara en fördelningsfunktion för variabeln $X$. Då gäller att
\begin{align*}
	F(b) - F(a) = P(a < X\leq b).
\end{align*}

\proof

\paragraph{Fördelningsfunktioner och sannolikhetsfunktioner}
Låt $F$ och $p$ vara fördelnings- respektiva sannolikhetsfunktionen till en diskret stokastisk variabel $X$. Då gäller att
\begin{align*}
	F(x) &= \sum\limits_{j\leq x}p(j), \\
	p(x) &=
	\begin{cases}
		F(x), x = 0, \\
		F(x) - F(x - 1), \text{annars}.
	\end{cases}
\end{align*}
En motsvarande relation till första ekvationen gäller även för sannolikhets- och fördelningsfunktioner i flera variabler.

\proof

\paragraph{Fördelningsfunktioner och täthetsfunktioner}
Låt $F$ och $f$ vara fördelnings- respektiva täthetsfunktionen till en kontinuerlig stokastisk variabel $X$ och låt $f$ vara kontinuerlig i $x$. Då gäller att
\begin{align*}
	F(x)          &= \integ{-\infty}{x}{f}{u}, \\
	\dv{F}{x} (x) &= f(x).
\end{align*}

\paragraph{Fördelningsfunktioner och täthetsfunktioner i flera variabler}
Låt $F$ och $f$ vara fördelnings- respektiva täthetsfunktionen till en kontinuerlig stokastisk variabel $(X, Y)$ och låt $f$ vara kontinuerlig i $(x, y)$. Då gäller att
\begin{align*}
	F(x, y)          &= \int_{-\infty}{x}\int_{-\infty}{y}f(u, v)\dd{u}\dd{v}, \\
	\pdv{F}{x}{y} (x, y) &= f(x, y).
\end{align*}

\paragraph{Normalisering av sannolikhetsfunktioner}
Låt $p$ vara en sannolikhetsfunktion. Då gäller att
\begin{align*}
	\sum p(j) = 1.
\end{align*}

\proof

\paragraph{Sannolikhetsfunktioner och sannolikheter}
Låt $p$ vara en sannolihetsfunktion för den stokastiska variabeln $X$. Då gäller att
\begin{align*}
	P(a\leq X\leq b) = \sum\limits_{i = a}^{b}p(i).
\end{align*}

\proof

\paragraph{Funktioner av stokastiska variabler}
Låt $X$ vara en stokastisk variabel. Då har den stokastiska variabeln $Y = g(X)$ sannolikhetsfuktionen $p_{Y}(k) = \sum\limits_{g(i) = k}p_{X}(i)$.

\proof

\paragraph{Väntevärde för funktioner av stokastiska variabler}
Låt $X$ vara en stokastisk variabel med sannolikhetsfunktion $p_{X}$. Då ges väntevärdet till $g(X)$ av
\begin{align*}
	\expec{g(X)} = \sum g(k)p_{X}(k),
\end{align*}
med en motsvarande relation i det kontinuerliga fallet och i det flerdimensionella fallet.

\proof

\paragraph{Förenklad formel för varians}
Låt $X$ vara en stokastisk variabel med väntevärde $\mu$. Då ges variansen till $X$ av
\begin{align*}
	\sigma^2 = \expec{X^2} - \mu^2.
\end{align*}

\proof

\paragraph{Förenklad formel för kovarians}
Låt $X, Y$ vara stokastiska variabler. Då ges kovariansen till dessa av
\begin{align*}
	\cov{X}{Y} = \expec{XY} - \expec{X}\expec{Y}.
\end{align*}

\proof

\paragraph{Väntevärde för linjärkombination av variabler}
\begin{align*}
	\expec{b + \sum a_iX_i} = b +\sum a_i\expec{X_i}.
\end{align*}

\proof

\paragraph{Varians för linjärkombination av variabler}
\begin{align*}
	\var{b + \sum a_iX_i} = \sum a_i^2\var{X_i} + \sum\limits_{1\leq j < k}a_ja_k\cov{X_j}{X_k}.
\end{align*}

\paragraph{Oberoende variabler och funktioner}
$X, Y$ är oberoende om
\begin{align*}
	F_{X, Y}(x, y) = F_{X}(x)F_{Y}(y)
\end{align*}
eller
\begin{align*}
	p_{X, Y}(j, k) = p_{X}(j)p_{Y}(k)
\end{align*}
i det diskreta fallet och
\begin{align*}
	f_{X, Y}(x, y) = f_{X}(x)f_{Y}(y).
\end{align*}

\proof

\paragraph{Oberoende variabler och väntevärde av produktet}
Låt $X, Y$ vara oberoende. Då gäller att
\begin{align*}
	\expec{XY} = \expec{X}\expec{Y}.
\end{align*}

\proof

\paragraph{Oberoende variabler och kovarians}
Oberoende variabler är okorrelerade.

\proof

\paragraph{Stora talens lag}
Låt $X_1, \dots, X_n$ vara likfördelade stokastiska variabler med samma väntevärde $\mu$ och inför variabeln $\overline{X} = \frac{1}{n}\sum X_i$. Då gäller att
\begin{align*}
	\lim\limits_{n\to\infty}P(\mu - \varepsilon < X < \mu + \varepsilon) = 1\ \forall\ \varepsilon.
\end{align*}

\proof

\paragraph{Markovs olikhet}
Låt $Y$ vara en stokastisk variabel och $a\geq 0, Y\geq 0$. Då gäller att
\begin{align*}
	P(Y\geq a))\leq\frac{\expec{Y}}{a}.
\end{align*}

\proof

\paragraph{Tjebysjobs olikhet}
Låt $X$ vara en stokastisk variabel med väntevärde $\mu$ och standardavvikelse $\sigma$. Då gäller att
\begin{align*}
	P(\abs{X - \mu}\geq k\sigma)\leq\frac{1}{k^2}\ \forall\ k > 0.
\end{align*}