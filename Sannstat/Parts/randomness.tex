\section{Grunläggande koncept inom slump}

\subsection{Definitioner}

\paragraph{Slumpförsök}
Ett slumpförsök är en experiment där resultatet ej kan avgöras på förhand.

\paragraph{Utfall}
Ett utfall är resultatet av ett slumpförsök.

\paragraph{Utfallsrum}
Ett utfallsrum, betecknad $\Omega$, är mängden av alla möjliga utfall för ett givet slumpförsök.

\paragraph{Händelser}
En händelse är en uppsättning intressanta utfall, alltså en delmängd av utfallsrummet, och betecknas $A, B, C, \dots$.

\paragraph{Sannolikheter}
Sannolikheten för en given händelse $A$ uppfyller följande axiom:
\begin{itemize}
	\item För varje $A$ gäller det att $0\leq P(A)\leq 1$.
	\item För hela $\Omega$ gäller att $P(\Omega) = 1$.
	\item Om $A_1, A_2, \dots$ är en följd av parvis disjunkta händelser så gäller att $P(A_1\cup A_2\cup\dots) = \sum P(A_i)$.
\end{itemize}

\paragraph{Disjunkta händelser}
Två händelser $A, B$ är disjunkta, eller parvis oförenliga, om $A\cap B = \emptyset$.

\paragraph{Betingade sannolikheter}
Sannolikheten $P(B\given A)$ är sannolikheten för att $B$ händer givet att $A$ har händt, och definieras som
\begin{align*}
	P(B\given A) = \frac{P(A\cap B)}{P(A)}.
\end{align*}
För tre händelser definieras det som
\begin{align*}
	P(A\cap B\cap C) = P(A)P(B\given A)P(C\given (A\cap B))
\end{align*}
och motsvarande för flere händelser.

\paragraph{Oberoende händelser}
Två händelser är oberoende om $P(A\cap B) = P(A)P(B)$. Detta generaliseras till tre händelser om
\begin{align*}
	P(A\cap B) = P(A)P(B), \\
	P(A\cap C) = P(A)P(C), \\
	P(B\cap C) = P(B)P(C), \\
	P(A\cap B\cap C) = P(A)P(B)P(C).
\end{align*}

\paragraph{Slumpmässiga fel}
Ett slumpmässigt fel är en differans mellan ett enkelt mätvärde och ett väntevärde.

\paragraph{Systematiska fel}
Ett systematiskt fel är en differanse mellan ett väntevärde och ett korrekt värde.

\paragraph{Precision}
Precision är när många mätningar motsvarar väntevärdet bra.

\paragraph{Noggrannhet}
Noggrannhet är när många mätningar motsvarar det korrekta värdet bra.

\subsection{Satser}

\paragraph{de Morgans lagar}
När man ska hitta komplement till komplicerade mängder, byta alla delmängder med deras komplement och alla unioner $(\cup)$ till snitt $(\cap)$, och motsatt.

\paragraph{Regler för sannolikhetskalkyl}
\begin{align*}
	P(\comp{A}) &= 1 - P(A), \\
	P(B)        &= P(B\cap A) + P(B\cap\comp{A}), \\
	P(A\cup B)  &= P(A) + P(B) - P(A\cap B)
\end{align*}

\proof
Följer från mängdlära.

\paragraph{Lagen om total sannolikhet}
Låt $H_1, \dots, H_n$ vara parvis oförenliga och låt $\bigcup\limits_{i = 1}^{n}H_i = \Omega$. Då gäller att
\begin{align*}
	P(A) = \sum\limits_{i = 1}^{n}P(H_i)P(A\given H_i).
\end{align*}

\proof
Från mängdlära har man att
\begin{align*}
	A = \bigcup\limits_{i = 1}^{n}(A\cap H_i).
\end{align*}
Eftersom alla $H_i$ är parvis oförenliga, följer formeln för sannolikheten direkt.

\paragraph{Bayes' sats}
Låt $H_1, \dots, H_n$ vara disjunkta och låt $\bigcup\limits_{i = 1}^{n}H_i = \Omega$. Då gäller att
\begin{align*}
	P(H_i\given A) = \frac{P(H_i)P(A\given H_i)}{\sum\limits_{j} P(H_j)P(A\given H_j)}.
\end{align*}

\proof
Direkt konsekvens av lagen om total sannolikhet och definitionen av betingad sannolikhet.

\paragraph{Oberoende händelser där minst en inträffer}
Låt $A_1, \dots, A_n$ vara oberoende och $P(A_i) = p_i$. Då ges sannolikheten för att minst en av dessa händer av
\begin{align*}
	1 - \prod\limits_{i = 1}^{n}(1 - p_i).
\end{align*}

\proof
Sannolikheten för att inga av de inträffer är
\begin{align*}
	\prod\limits_{i = 1}^{n}(1 - p_i),
\end{align*}
och den givna formeln följer direkt.