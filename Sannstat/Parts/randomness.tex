\section{Grunläggande koncept inom slump}

\subsection{Definitioner}

\paragraph{Slumpförsök}
Ett slumpförsök är en experiment där resultatet ej kan avgöras på förhand.

\paragraph{Utfall}
Ett utfall är resultatet av ett slumpförsök.

\paragraph{Utfallsrum}
Ett utfallsrum, betecknad $\Omega$, är mängden av alla möjliga utfall för ett givet slumpförsök.

\paragraph{Händelser}
En händelse är en uppsättning intressanta utfall, alltså en delmängd av utfallsrummet, och betecknas $A, B, C, \dots$.

\paragraph{Sannolikheter}
Sannolikheten för en given händelse $A$ uppfyller följande axiom:
\begin{itemize}
	\item För varje $A$ gäller det att $0\leq P(A)\leq 1$.
	\item För hela $\Omega$ gäller att $P(\Omega) = 1$.
	\item Om $A_1, A_2, \dots$ är en följd av parvis disjunkta händelser så gäller att $P(A_1\cup A_2\cup\dots) = \sum P(A_i)$.
\end{itemize}

\paragraph{Disjunkta händelser}
Två händelser $A, B$ är disjunkta om $A\cap B = \emptyset$.

\subsection{Satser}

\paragraph{de Morgans lagar}
När man ska hitta komplement till komplicerade mängder, byta alla delmängder med deras komplement och alla unioner $(\cup)$ till snitt $(\cap)$, och motsatt.

\paragraph{Regler för sannolikhetskalkyl}
\begin{align*}
	P(A*) = 1 - P(A), \\
	P(B) = P(B\cap A) + P(B\cap A*), \\
	P(A\cup B) = P(A) + P(B) - P(A\cap B)
\end{align*}

\proof
Följer från mängdlära.