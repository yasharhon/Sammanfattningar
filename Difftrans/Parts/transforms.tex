\section{Transformer}

\subsection{Definitioner}

\paragraph{Fouriertransformen}
Låt $f\in L^{1}(\R, \C)$. Då definieras Fouriertransformen $\fou{f}$ som
\begin{align*}
	\fou{f}(\omega) = \inteval{\R}{}{f(x)e^{-i\omega x}}{x}.
\end{align*}

\paragraph{Inversa Fouriertransformen}
Inversa Fouriertransformen definieras som
\begin{align*}
	\fouinv{f}(x) = \inteval{\R}{}{f(\omega)e^{i\omega x}}{\omega}.
\end{align*}

\paragraph{Faltning}
Låt $f, g\in L^{1}(\R, \C)$. Då definieras faltningen av $f$ och $g$ som
\begin{align*}
	\conv{f}{g}(x) = \inteval{\R}{}{f(x - t)g(t)}{t} = \inteval{\R}{}{f(t)g(x - t)}{t}.
\end{align*}

\paragraph{Laplacetransformen}
Låt $f: [0, \infty)\to\R$ vara integrerbar på varje delintervall av $[0, \infty)$ så att det finns ett $\in R$ så att $\abs{f(t)}\leq Me^{kt}$ för något $M$ och alla $t \geq 0$. Då definieras Laplacetransformen av $f$ som
\begin{align*}
	\lap{f}(s) = \inteval{0}{\infty}{f(t)e^{-st}}{t}
\end{align*}
för alla $s > k$.

\paragraph{Faltning för andra definitionsmängder}
Låt $f$ och $g$ vara Riemannintegrerbara på $[0, \infty)$. Då definieras faltningen av $f$ och $g$ som
\begin{align*}
	\conv{f}{g}(x) = \inteval{0}{\infty}{f(x - t)g(t)}{t} = \inteval{0}{\infty}{f(t)g(x - t)}{t}.
\end{align*}

\subsection{Satser}

\paragraph{Räkneregler för Fouriertransformen}
Fouritertransformen uppfyller:
\begin{itemize}
	\item Fouriertransformen definierar en linjär operator.
	\item Om $f_{a}(x) = f(x - a)$ är $\fou{f_{a}}(\omega) = e^{-ia\omega}\fou{f}(\omega)$.
	\item Om $f_{a}(x) = f(\frac{x}{a})$ är $\fou{f_{a}}(\omega) = \frac{1}{a}\fou{f}(\frac{\omega}{a})$.
	\item Om $f,\dv{f}{x}\in L^{1}(\R, \C)$ är $\fou{\dv{f}{x}}(\omega) = i\omega\fou{f}(\omega)$.
\end{itemize}

\proof
Här visas endast andra och sista påståendet.

Om $f_{a}(x) = f(x - a)$ är 
\begin{align*}
	\fou{f_{a}}(\omega) &= \inteval{-\infty}{\infty}{f(x - a)e^{-i\omega x}}{x} \\
	                    &= \inteval{-\infty}{\infty}{f(x)e^{-i\omega(x + a)}}{x} \\
	                    &= e^{-i\omega a)}\inteval{-\infty}{\infty}{f(x)e^{-i\omega x}}{x} \\
	                    &= e^{-i\omega a)}\fou{f}(\omega),
\end{align*}
vilket skulle visas.

Om $f$ har kompakt stöd, dvs. det finns ett $M$ så att $\abs{x} > M\implies f(x) = 0$,  gäller det vidare att
\begin{align*}
	\fou{\dv{f}{x}}(\omega) &= \inteval{-\infty}{\infty}{\deval{f}{x}{x}e^{-i\omega x}}{x} \\
	                        &= \left[f(x)e^{-i\omega x}\right]_{-2M}^{2M} + i\omega\inteval{-\infty}{\infty}{f(x)e^{-i\omega x}}{x}.
\end{align*}
Jag är inte säker på att detta stämmer, men jag tror gränserna i sista raden valdes godtycklig som något större än $M$ och sedan skulle kunna skickas mot oändligheten. I alla fall ger denna inget bidrag, och beviset är klart.

\paragraph{Fouriertransformens egenskaper}
Låt $f\in L^{1}(\R, \C)$. Då gäller att
\begin{itemize}
	\item $\abs{\fou{f}(\omega)} \leq \inteval{\R}{}{\abs{f(x)}}{x}$.
	\item $\fou{f}$ är kontinuerlig på $\R$.
	\item $\lim\limits_{\abs{\omega}\to\infty}\fou{f}(\omega) = 0$.
\end{itemize}

\proof
Första påståendet följer från Cauchy-Schwarz' olikhet.

Andra påståendet är sant ty
\begin{align*}
	\abs{\fou{f}(\omega + h) - \fou{f}(\omega)} &= \abs{\inteval{-\infty}{\infty}{f(x)e^{-i(\omega + h)x}}{x} - \inteval{-\infty}{\infty}{f(x)e^{-i\omega x}}{x}} \\
	                                            &= \abs{\inteval{-\infty}{\infty}{f(x)e^{-i\omega x}(e^{-ihx} - 1)}{x}} \\
	                                            &\leq \abs{\inteval{\abs{x} < M}{}{f(x)e^{-i\omega x}(e^{-ihx} - 1)}{x}} + \abs{\inteval{\abs{x} \geq M}{}{f(x)e^{-i\omega x}(e^{-ihx} - 1)}{x}}.
\end{align*}
För små $h$ är $\abs{e^{-ihx} - 1} < \varepsilon$

Sista påståendet följer direkt från Riemann-Lebesgues lemma.

\paragraph{Inversionsformeln}
Låt $f\in L^{1}(\R, \C)$ vara styckvist $C^{1}$ och ha generaliserade vänster- och högerderivator i alla punkter. Då gäller att
\begin{align*}
	f(x) = \fouinv{f}(x) = \frac{1}{2\pi}\inteval{\R}{}{\fou{f}(\omega)e^{i\omega x}}{\omega}.
\end{align*}

\proof

\paragraph{Faltningsformeln}
Antag att $f, g\in L^{1}(\R, \C)$. Då gäller att
\begin{align*}
	\fou{\conv{f}{g}}(\omega)           &= \fou{f}(\omega)\fou{g}(\omega), \\
	\fouinv{\conv{\fou{f}}{\fou{g}}}(x) &= f(x)g(x).
\end{align*}

\proof
Ett lite hälft bevis är:
\begin{align*}
	\fou{\conv{f}{g}}(\omega) = \inteval{\R}{}{e^{-i\omega x}\inteval{\R}{}{f(x - t)g(t)}{t}}{x}.
\end{align*}
Utan att motivera varför, byter vi integrationsordning och skriver vi detta som
\begin{align*}
	\fou{\conv{f}{g}}(\omega) &= \inteval{\R}{}{\inteval{\R}{}{e^{-i\omega x}f(x - t)g(t)}{t}}{x} \\
	                          &= \inteval{\R}{}{\inteval{\R}{}{e^{-i\omega(x - t)}f(x - t)e^{-i\omega t}g(t)}{t}}{x} \\
	                          &= \inteval{\R}{}{e^{-i\omega t}g(t)\inteval{\R}{}{e^{-i\omega(x - t)}f(x - t)}{x}}{t}.
\end{align*}
Vid att göra variabelbytet $u = x - t$ tas $t$-beroendet bort, och högra integralen blir $\fou{f}(\omega)$, vilket ger
\begin{align*}
	\fou{\conv{f}{g}}(\omega) &= \inteval{\R}{}{e^{-i\omega t}g(t)\fou{f}(\omega)}{t} \\
	                          &= \fou{f}(\omega)\inteval{\R}{}{e^{-i\omega t}g(t)}{t} \\
	                          &= \fou{f}(\omega)\fou{g}(\omega).
\end{align*}
Ideen är motsvarande med inverstransformen.

\paragraph{Planchevels sats}
Låt $f\in L^{1}(\R, \C)$. Då gäller det att
\begin{align*}
	\inteval{\R}{}{\abs{f(x)}^{2}}{x} = \frac{1}{2\pi}\inteval{\R}{}{\abs{\fou{f}(\omega)}}{\omega}.
\end{align*}

\proof

\paragraph{Räkneregler för Laplacetransformen}
Låt $f: [0, \infty)\to\R$ ha en Laplacetransform. Då gäller att
\begin{itemize}
	\item Laplacetransformationen definierar en linjär operator.
	\item $\lap{e^{at}f(t)}(s) = \lap{f}(s - a)$.
	\item Om $f(t) = 0$ för $t < 0$, gäller det för $g(t) = f(t - a)$ att $\lap{g}{s} = e^{-as}\lap{f}(s)$.
	\item Det gäller för $g(t) = f(at), a > 0$ att $\lap{g}(s) = \frac{1}{a}\lap{f}(\frac{s}{a})$.
	\item $\lap{tf}(s) = \deval{\lap{f}}{s}{s}$.
	\item Om $\dv{f}{t}$ även har en Laplacetransform, är $\lap{\dv{f}{t}}(s) = s\lap{f}(s) - f(0)$.
\end{itemize}

\proof

\paragraph{Laplacetransformens entydighet}
Låt $f$ och $g$ uppfylla $\lap{f} = \lap{g}$. Då är $f = g$.

\proof

\paragraph{Laplacetransformationen av en faltning}
Låt $f$ och $g$ ha Laplacetransformer. Då gäller att
\begin{align*}
	\lap{\conv{f}{g}}(s) = \lap{f}(s)\lap{g}(s).
\end{align*}

\proof