\subsection{Stabilitet}

\paragraph{Jämviktspunkter}
Betrakta
\begin{align*}
	\deval{\vb{x}}{t}{t} = \vb{f}(\vb{x}(t)).
\end{align*}
En jämviktspunkt för detta systemet är en punkt $\vb{x}(t_{0})$ så att $\vb{f}(\vb{x}(t_{0})) = \vb{0}$, med implikationen att $\vb{x}(t)$ är konstant för $t > t_{0}$.

\paragraph{Stabila jämviktspunkter}
En jämviktspunkt $\vb{x}_{0}$ är stabil om det för varje $\varepsilon > 0$ finns ett $\delta > 0$ så att alla lösningar $\vb{x}$ som uppfyller $\abs{\vb{x}(t_{0}) - \vb{x}_{0}} < \delta$, existerar för $t > t_{0}$ och uppfyller $\abs{\vb{x}(t) - \vb{x}_{0}} < \varepsilon,\ t > t_{0}$. En jämviktspunkt som ej uppfyller detta är instabil.

\paragraph{Asymptotiskt stabila jämviktspunkter}
En jämviktspunkt $\vb{x}_{0}$ är asymptotiskt stabil om den är stabil och det finns ett $\delta_{0} > 0$ så att om $\abs{\vb{x}(t_{0}) - \vb{x}_{0}} < \delta_{0}$, gäller det att
\begin{align*}
	\lim\limits_{t\to\infty}\vb{x}(t) = \vb{x}_{0}.
\end{align*}

\paragraph{Stabilitet av autonom ODE}
Betrakta
\begin{align*}
\deval{y}{t}{t} = g(y(t)),\ g(y_{0}) = 0.
\end{align*}
Då gäller att
\begin{itemize}
	\item om $\deval{g}{y}{y_{0}} < 0$, är $y_{0}$ asymptotiskt stabil.
	\item om $\deval{g}{y}{y_{0}} > 0$, är $y_{0}$ instabil.
\end{itemize}

\subparagraph{Bevis}
Här bevisas endast det första fallet.

Betrakta $(y - y_{0})^{2}$. Nära $y_{0}$ gäller att
\begin{align*}
	\dv{t}(y(t) - y_{0})^{2} &= 2(y(t) - y_{0})g(y(t)) \\
	                         &\approx 2(y(t) - y_{0})\left(g(y_{0}) + \deval{g}{y}{y_{0}}(y(t) - y_{0}) + o((y(t) - y_{0})^{2})\right) \\
	                         &= 2(y(t) - y_{0})\left(\deval{g}{y}{y_{0}}(y(t) - y_{0}) + o((y(t) - y_{0})^{2})\right).
\end{align*}
Det gäller att $o((y(t) - y_{0})^{2}) < -\deval{g}{y}{y_{0}}(y(t) - y_{0})^{2}$ tillräckligt nära $y_{0}$ (man skulle även kunna välja en annan nollskild konstant än $-\deval{g}{y}{y_{0}}$, men detta valet gör beviset snyggare). Detta ger
\begin{align*}
	\dv{t}(y(t) - y_{0})^{2} < \deval{g}{y}{y_{0}}(y(t) - y_{0})^{2},
\end{align*}
som kan lösas för att ge
\begin{align*}
	(y(t) - y_{0})^{2} < e^{\deval{g}{y}{y_{0}}t}(y(0) - y_{0})^{2},
\end{align*}
som går mot $0$ för stora $t$ enligt vårt antagande om $g$:s derivata.

\paragraph{Karakterisering av jämviktspunkter för system}
Betrakta systemet
\begin{align*}
	\deval{\vb{x}}{t}{t} = P\vb{x}(t),
\end{align*}
där $P$ är konstant och reellvärd. För enkelhetens skull kommer vi här att låta systemet vara ett system i två variabler. Låt även $P$ ha egenvärden $r_{1}, r_{2}\neq 0$. Då gäller att $\vb{0}$ är en kritisk punkt. Lösningarnas banor kan nu beskrivas på följande sätt:
\begin{itemize}
	\item Om $r_{1}, r_{2} < 0$ går alla lösningar in mot origo, och origo kallas en stabil nod.
	\item Om $r_{1}, r_{2} > 0$ går alla lösningar ut från origo, och origo kallas en instabil nod.
	\item Om egenvärderna har olika tecken går lösningarna in mot origo parallellt med en egenvektor och ut parallellt med den andra, och origo kallas en instabil sadelpunkt.
	\item Om $r_{1} = \alpha + i\beta, r_{2} = \alpha - i\beta$ gäller att:
	\begin{itemize}
		\item Om $\alpha > 0$ går lösningarna i spiraler ut från origo, och origo kallas en instabil spiralpunkt.
		\item Om $\alpha > 0$ går lösningarna i spiraler in mot origo, och origo kallas en stabil spiralpunkt.
		\item Om $\alpha = 0$ går lösningarna i bana kring origo, och origo kallas ett centrum.
	\end{itemize}
	\item Om $r_{1} = r_{2} = r$ och det finns två egenvektorer motsvarande egenvärdet $r$ går banorna i linjer från eller till origo, beroende på tecknet till $r$, och origo är en instabil eller stabil nod.
	\item Om $r_{1} = r_{2} = r$ och det bara finns en egenvektor motsvarande egenvärdet $r$ går lösningarna i kurvade banor ut från eller in mot origo, där dessa banorna blir parallella med egenvektorn långt borta från origo, och origo är en stabil eller instabil degenererad nod.
\end{itemize}

\subparagraph{Slutsats}
Det gäller alltså att
\begin{itemize}
	\item Om alla $P$s egenvärden har negativ realdel, är origo en stabil jämviktspunkt.
	\item Om något av $P$:s egenvärden har positiv realdel, är origo en instabil jämviktspunkt.
\end{itemize}

\paragraph{Stabilitet av jämviktspunkter för icke-linjära system av ODE}
Betrakta
\begin{align*}
	\deval{\vb{x}}{t}{t} = \vb{f}(\vb{x}(t)),
\end{align*}
Låt detta ha en kritisk punkt $\vb{x}_{0}$ och låt $\vb{g}\in C^{1}$ i en öppen mängd kring $\vb{x}_{0}$. Vi linjariserar kring $\vb{x}_{0}$, vilket går om
\begin{align*}
	\lim\limits_{\vb{x}\to\vb{x}_{0}}\frac{\abs{\vb{f}(\vb{x}(t))}}{\abs{\vb{x}(t)}} = 0,
\end{align*}
vilket uppfylls om $\vb{f}\in C^{2}$. Inför funktionalmatrisen aka Jacobimatrisen
\begin{align*}
	J(\vb{x}) = 
	\left[\begin{array}{ccc}
		\deval{f_1}{x_1}{\vb{x}} & \dots  & \deval{f_1}{x_n}{\vb{x}} \\
		\vdots                     & \ddots & \vdots \\
		\deval{f_p}{x_1}{\vb{x}} & \dots  & \deval{f_p}{x_n}{\vb{x}}
	\end{array}\right]
\end{align*}
och betrakta $J(\vb{x}_{0})$. Då gäller att
\begin{itemize}
	\item Om alla $J(\vb{x}_{0})$s egenvärden har negativ realdel, är $\vb{x}_{0}$ en stabil jämviktspunkt.
	\item Om något av $J(\vb{x}_{0})$ egenvärden har positiv realdel, är $\vb{x}_{0}$ en instabil jämviktspunkt.
\end{itemize}

\paragraph{Lyapunovfunktioner}
Betrakta
\begin{align*}
	\deval{\vb{x}}{t}{t} = \vb{f}(\vb{x}(t)).
\end{align*}
Antag att systemet har en kritisk punkt $\vb{0}$. Om det finns en positivt definitiv funktion $V\in C^{1}$ och en negativt definitiv funktion
\begin{align*}
	V' = \pdv{V}{x}f_{1} + \pdv{V}{y}f_{2}
\end{align*}
på någon omgivning av $\vb{0}$, är $\vb{0}$ en stabil jämviktspunkt. Om $V'$ är negativt semidefinitiv, är $\vb{0}$ en stabil jämviktspunkt.