\section{Fouriertransform}

\subsection{Definitioner}

\paragraph{Fouriertransformen}
Låt $L^{1}(\R, \C)$ vara alla absolut integrerbara funktioner och $f\in L^{1}(\R, \C)$. Då definieras Fouriertransformen $\fou{f}$ som
\begin{align*}
	\fou{f}(\omega) = \inteval{-\infty}{\infty}{f(x)e^{-i\omega x}}{x}.
\end{align*}

\subsection{Satser}

\paragraph{Räkneregler för Fouriertransformen}
Fouritertransformen uppfyller:
\begin{itemize}
	\item Fouriertransformen definierar en linjär operator.
	\item Om $f_{a}(x) = f(x - a)$ är $\fou{f_{a}}(\omega) = e^{-ia\omega}\fou{f}(\omega)$.
	\item Om $f_{a}(x) = f(\frac{x}{a})$ är $\fou{f_{a}}(\omega) = \frac{1}{a}\fou{f}(\frac{\omega}{a})$.
	\item Om $f,\dv{f}{x}\in L^{1}(\R, \C)$ är $\fou{\dv{f}{x}}(\omega) = i\omega\fou{f}(\omega)$.
\end{itemize}

\proof
Här visas endast andra och sista påståendet.

Om $f_{a}(x) = f(x - a)$ är 
\begin{align*}
	\fou{f_{a}}(\omega) &= \inteval{-\infty}{\infty}{f(x - a)e^{-i\omega x}}{x} \\
	                    &= \inteval{-\infty}{\infty}{f(x)e^{-i\omega(x + a)}}{x} \\
	                    &= e^{-i\omega a)}\inteval{-\infty}{\infty}{f(x)e^{-i\omega x}}{x} \\
	                    &= e^{-i\omega a)}\fou{f}(\omega),
\end{align*}
vilket skulle visas.

Om $f$ har kompakt stöd, dvs. det finns ett $M$ så att $\abs{x} > M\implies f(x) = 0$,  gäller det vidare att
\begin{align*}
	\fou{\dv{f}{x}}(\omega) &= \inteval{-\infty}{\infty}{\deval{f}{x}{x}e^{-i\omega x}}{x} \\
	                        &= \left[f(x)e^{-i\omega x}\right]_{-2M}^{2M} + i\omega\inteval{-\infty}{\infty}{f(x)e^{-i\omega x}}{x}.
\end{align*}
Jag är inte säker på att detta stämmer, men jag tror gränserna i sista raden valdes godtycklig som något större än $M$ och sedan skulle kunna skickas mot oändligheten. I alla fall ger denna inget bidrag, och beviset är klart.

\paragraph{Fouriertransformens egenskaper}
Låt $f\in L^{1}(\R, \C)$. Då gäller att
\begin{itemize}
	\item $\abs{\fou{f}(\omega)} \leq \inteval{\R}{}{\abs{f(x)}}{x}$.
	\item $\fou{f}$ är kontinuerlig på $\R$.
	\item $\lim\limits_{\abs{\omega}\to\infty}\fou{f}(\omega) = 0$.
\end{itemize}

\proof
Första påståendet följer från Cauchy-Schwarz' olikhet.

Andra påståendet är sant ty
\begin{align*}
	\abs{\fou{f}(\omega + h) - \fou{f}(\omega)} &= \abs{\inteval{-\infty}{\infty}{f(x)e^{-i(\omega + h)x}}{x} - \inteval{-\infty}{\infty}{f(x)e^{-i\omega x}}{x}} \\
	                                            &= \abs{\inteval{-\infty}{\infty}{f(x)e^{-i\omega x}(e^{-ihx} - 1)}{x}} \\
	                                            &\leq \abs{\inteval{\abs{x} < M}{}{f(x)e^{-i\omega x}(e^{-ihx} - 1)}{x}} + \abs{\inteval{\abs{x} \geq M}{}{f(x)e^{-i\omega x}(e^{-ihx} - 1)}{x}}.
\end{align*}
För små $h$ är $\abs{e^{-ihx} - 1} < \varepsilon$

Sista påståendet följer direkt från Riemann-Lebesgues lemma.