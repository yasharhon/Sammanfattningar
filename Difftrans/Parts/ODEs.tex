\section{Ordinarie differentialekvationer (ODE)}

\subsection{Användbara defitioner och satser}

\paragraph{Lipschitzkontinuitet}
En funktion $f$ är Lipschitzkontinuerlig om det finns ett $K$ så att det för varje $x_1, x_2$ gäller att
\begin{align*}
	\abs{f(x_1) - f(x_2)} \leq K\abs{x_1 - x_2}.
\end{align*}

\paragraph{Lipschitzkontinuitet och deriverbarhet}
Låt $f\in C^{1}$. Då är $f$ Lipschitzkontinuerlig.

\paragraph{Grönwalls lemma}
Antag att det finns positiva $A, K$ så att $h: [0, T\to\R]$ uppfyller
\begin{align*}
	h(t) \leq K\inteval{0}{t}{h(s)}{s} + A.
\end{align*}
Då gäller att
\begin{align*}
	h(t) \leq Ae^{Kt}.
\end{align*}

\subparagraph{Bevis}
Definiera
\begin{align*}
	I(t) = \inteval{0}{t}{h(s)}{s}.
\end{align*}
Då gäller att
\begin{align*}
	\deval{I}{t}{t} = h(t) \leq KI(t) + A.
\end{align*}
Denna differentialolikheten kan vi lösa vid att tillämpa integrerande faktor. Detta kommer att ge
\begin{align*}
	\dv{t}\left(e^{-Kt}I(t)\right) \leq Ae^{-Kt}.
\end{align*}
Vi integrerar från $0$ till $r$ och använder att $I(0) = 0$ för att få
\begin{align*}
	I(r) \leq \frac{A}{K}(e^{Kr} - 1).
\end{align*}
Derivation på båda sidor ger
\begin{align*}
	h(r) \leq Ae^{Kr},
\end{align*}
vilket skulle visas.

\paragraph{Linjära differentialekvationer}
Om en differentialekvation kan skrivas på formen $F(t, y, \dv{y}{x}, \dots) = 0$, är den linjär om $F$ är linjär i alla sina argument förutom $t$.

\paragraph{Wronskianen}
Wronskianen definieras som
\begin{align*}
	W(y_1, y_2)(t) = \mdet{y_{1}(t) & y_{2}(t) \\ \deval{y_{1}}{t}{t} & \deval{y_{2}}{t}{t}}.
\end{align*}

\paragraph{Linjärt beroende funktioner}
$f: I\to\R, g: I\to\R$ är linjärt beroende om det finns $k_{1}, k_{2}$ så att
\begin{align*}
	k_{1}f(t) + k_{2}g(t) = 0\ \forall\ t\in I.
\end{align*}

\subsection{Första ordningen}

\paragraph{Entydighet av lösning}
Betrakta differentialekvationen
\begin{align*}
	\deval{y}{t}{t} &= f(y), \\
	y(0)            &= y_0.
\end{align*}
Detta har en unik lösning om $f$ är Lipschitzkontinuerlig.

Observera att beviset kan även göras för en funktion $f(t, y)$ vid att skriva differentialekvationen som ett system och komma med en motsvarande sats för system av differentialekvationer.

\subparagraph{Bevis}
Betrakta två lösningar $y, z$ av differentialekvationen. Vi får
\begin{align*}
	y(\tau) - y_{0} = \inteval{0}{\tau}{f(y)}{t}
\end{align*}
och samma för $z$. Vi subtraherar dessa två resultat och får
\begin{align*}
	y(\tau) - z(\tau) = y_{0} - z_{0} + \inteval{0}{\tau}{f(y) - f(z)}{t}.
\end{align*}
Vid att beräkna absolutbeloppet av båda sidor och använda Cauchy-Schwarz' oliket får man vidare
\begin{align*}
	\abs{y(\tau) - z(\tau)} \leq \abs{y_{0} - z_{0}} + \inteval{0}{\tau}{\abs{f(y) - f(z)}}{t}.
\end{align*}
Kravet om Lipschitzkontinuitet av $f$ ger vidare
\begin{align*}
	\abs{y(\tau) - z(\tau)} \leq \abs{y_{0} - z_{0}} + \inteval{0}{\tau}{K\abs{y(t) - z(t)}}{t}.
\end{align*}
Grönwalls lemma ger slutligen
\begin{align*}
	\abs{y(\tau) - z(\tau)} \leq \abs{y_{0} - z_{0}}e^{K\tau}.
\end{align*}
Om $y_{0} = z_{0}$ är $y = z$, och beviset är klart.

\paragraph{Lösning av linjära ODE av första ordning}
Antag att vi har en differentialekvation på formen
\begin{align*}
	\deval{y}{t}{t} + p(t)y(t) = g(t).
\end{align*}
Beräkna
\begin{align*}
	P(t) = \inteval{a}{t}{p}{x}
\end{align*}
och inför den integrerande faktorn $e^{P(t)}$. Multiplicera med den på båda sidor för att få
\begin{align*}
	e^{P(t)}\deval{y}{t}{t} + p(t)e^{P(t)}y(t) = e^{P(t)}g(t).
\end{align*}
Detta kan skrivas om till
\begin{align*}
	\dv{t}\left(ye^{P}\right)(t) = e^{P(t)}g(t) = \deval{H}{t}{t}.
\end{align*}
Analysens huvudsats ger då
\begin{align*}
	y(t)e^{P(t)} = H(t) + c
\end{align*}
och slutligen
\begin{align*}
	y(t) = ce^{-P(t)} + e^{-P(t)}H(t).
\end{align*}

Låt oss lägga till bivillkoret $y(a) = y_0$. Man kan då visa att lösningen kan skrivas som
\begin{align*}
	y(t) = y_0e^{-\inteval{a}{t}{p}{x}} + \inteval{a}{t}{g(x)e^{-\inteval{x}{t}{p}{s}}}{x}.
\end{align*}

\paragraph{Separabla ODE av första ordning}
Antag att vi har en differentialekvation som kan skrivas på formen
\begin{align*}
	m(x) + n(y(x))\deval{y}{x}{x} = 0.
\end{align*}
Denna betecknas som en separabel ODE av första ordning.

För att lösa den, beräkna primitiv funktion på båda sidor, vilket ger
\begin{align*}
	M(x) + N(y(x)) = c,\ c\in\R.
\end{align*}
Om $N$ är inverterbar, får man då $y$ enligt
\begin{align*}
	y(x) = N^{-1}(c - M(x)).
\end{align*}

\subsection{Andra ordningen}

\paragraph{Entydighet av lösning}
Betrakta den andra ordningens ODE
\begin{align*}
	\deval[2]{y}{t}{t} + p(t)\deval{y}{t}{t} + q(t)y(t) &= g(t),\ y > t_0, \\
	y(t_0)                                              &= y_0, \\
	\deval{y}{t}{t_{0}}                                 &= y'_{0}.
\end{align*}
Den har en entydig lösning om $p, q$ är Lipschitzkontinuerliga.

\paragraph{Form på lösning av andra ordningens ODE}
Betrakta den andra ordningens ODE
\begin{align*}
	\deval[2]{y}{t}{t} + p(t)\deval{y}{t}{t} + q(t)y(t) = L(t, y) = g(t).
\end{align*}
Låt $y_{\text{P}}$ vara en partikulär lösning till denna. Då är $y$ en lösning om och endast om
\begin{align*}
	y = y_{\text{H}} + y_{\text{P}},
\end{align*}
där $y_{\text{H}}$ löser den homogena ekvationen.

\subparagraph{Bevis}
Vi har
\begin{align*}
	L(t, y) = L(t, y_{\text{P}} + y_{\text{H}}) = L(t, y_{\text{P}}) + L(t, y_{\text{H}}) = g(t) + 0 = g(t),
\end{align*}
och därmed löser $y$ differentialekvationen. Vi har även
\begin{align*}
	L(t, y - y_{\text{P}}) = g(t) - g(t) = 0,
\end{align*}
och $y - y_{\text{P}}$ löser den homogena ekvationen. Eftersom detta är sant, har vi visat ekvivalens.

\paragraph{Linjär kombination av lösningar}
Betrakta
\begin{align*}
	\deval[2]{y}{t}{t} + p(t)\deval{y}{t}{t} + q(t)y(t) &= g(t),\ t > t_0, \\
	y(t_0)                                              &= y_0, \\
	\deval{y}{t}{t_{0}}                                 &= y'_{0}
\end{align*}
och anta att $y_{1}, y_{2}$ är lösningar. Då finns det $c_{1}, c_{2}$ så att $y = c_{1}y_{1} + c_{2}y_{2}$ är en lösning om $W(y_{1}, y_{2})(t_{0}) \neq 0$.

\paragraph{Abels sats}
Betrakta
\begin{align*}
	\deval[2]{y}{t}{t} + p(t)\deval{y}{t}{t} + q(t)y(t) &= g(t),\ t\in I, \\
	y(t_0)                                              &= y_0, \\
	\deval{y}{t}{t_{0}}                                 &= y'_{0}
\end{align*}
och anta att $y_{1}, y_{2}$ är lösningar. Då gäller att
\begin{align*}
	W(y_{1}, y_{2})(t) = W(y_{1}, y_{2})(t_{0})e^{-\inteval{t_{0}}{t}{p(s)}{s}}.
\end{align*}

\subparagraph{Bevis}
\begin{align*}
	\deval{W}{t}{t} &= \deval{y_{1}}{t}{t}\deval{y_{2}}{t}{t} - \deval{y_{1}}{t}{t}\deval{y_{2}}{t}{t} + y_{1}\deval[2]{y_{2}}{t}{t} - y_{2}\deval[2]{y_{1}}{t}{t} \\
	                &= y_{1}\left(-p(t)\deval{y_{2}}{t}{t} + q(t)y_{2}(t)\right) - y_{2}\left(-p(t)\deval{y_{1}}{t}{t} + q(t)y_{1}(t)\right) \\
	                &= -p(t)W(y_{1}, y_{2})(t).
\end{align*}
Denna differentialekvationen har lösning
\begin{align*}
	W(y_{1}, y_{2})(t) = W(y_{1}, y_{2})(t_{0})e^{-\inteval{t_{0}}{t}{p(s)}{s}},
\end{align*}
vilket skulle visas.

\paragraph{Linjärt beroende av lösningar}
Betrakta
\begin{align*}
	\deval[2]{y}{t}{t} + p(t)\deval{y}{t}{t} + q(t)y(t) &= g(t),\ t\in I, \\
	y(t_0)                                              &= y_0, \\
	\deval{y}{t}{t_{0}}                                 &= y'_{0}
\end{align*}
och anta att $y_{1}, y_{2}$ är lösningar. Då är dessa linjärt beroende på $I$ om och endast om $W(y_{1}, y_{2})(t) = 0$.

\subparagraph{Bevis}
Om dessa är linjärt beroende, ser man att Wronskianen blir lika med $0$, då kolumnerna i matrisen vars determinant ger Wronskianen kommer vara multipler av varandra.

\paragraph{Lösning av andra ordningens ODE med konstanta koefficienter}
Låt $r_1, r_2$ vara lösningar till
\begin{align*}
	r^2 + pr + q = 0.
\end{align*}
Då ges lösningarna till
\begin{align*}
	\deval[2]{y}{t}{t} + p\deval{y}{t}{t} + qy(t) = L(t, y) = 0
\end{align*}
av
\begin{align*}
	y(t) = 
	\begin{cases}
		c_1e^{r_1t} + c_2e^{r_2t},\ &r_1\neq r_2, \\
		(c_1t + c_2)e^{r_1t},\      &r_1 = r_2.
	\end{cases}
\end{align*}

\subsection{Annat}

\paragraph{Exakta differentialekvationer}
Betrakta ekvationen
\begin{align*}
	M(x, y(x)) + N(x, y(x))\deval{y}{x}{x} = 0.
\end{align*}
Denna är exakt om den kan skrivas på formen
\begin{align*}
	\deval{\psi}{x}{x, y(x)} = 0.
\end{align*}
Det gåller då att
\begin{align*}
	\pdeval{\psi}{x}{x, y(x)} = M(x, y(x)),\ \pdeval{\psi}{y}{x, y(x)} = N(x, y(x)),
\end{align*}
och lösningarna ges implicit av
\begin{align*}
	\psi(x, y(x)) = c.
\end{align*}

\paragraph{Exakthet av differentialekvationer}
Differentialekvationen
\begin{align*}
	M(x, y(x)) + N(x, y(x))\deval{y}{x}{x} = 0
\end{align*}
är exakt om
\begin{align*}
	\pdeval{M}{y}{x, y(x)} = \pdeval{N}{x}{x, y(x)}.
\end{align*}

\paragraph{Eulers metod}
Betrakta differentialekvationen
\begin{align*}
	\deval{y}{t}{t} &= f(t, y),\ 0 < t < T, \\
	y(0)             &= y_0.
\end{align*}
Vi gör indelningen $t_n = n\Delta t, n = 0, 1, \dots, N$ så att $\Delta t = \frac{T}{N}$ och inför $y_n = y(t_n)$. Vidare gör vi approximationen
\begin{align*}
	\frac{y_{n + 1} - y_{n}}{\Delta t} = f(t_n, y).
\end{align*}