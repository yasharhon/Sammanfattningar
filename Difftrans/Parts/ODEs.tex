\section{Ordinarie differentialekvationer}

\paragraph{Linjära differentialekvationer}
Om en differentialekvation kan skrivas på formen $L(y) = g$, är den linjär om
\begin{itemize}
	\item $L(y_1 + y_2) = L(y_1) + L(y_2)$.
	\item $L(\alpha y) = \alpha L(y),\ \alpha\in\R$.
\end{itemize}

\paragraph{Separabla ordinarie differentialekvationer}
Antag att vi har en differentialekvation som kan skrivas på formen
\begin{align*}
	m(x) + n(y(x))\deval{y}{x}{x} = 0.
\end{align*}
Vi beräknar primitiv funktion på båda sidor och får
\begin{align*}
	M(x) + N(y(x)) = c,\ c\in\R.
\end{align*}
Om $N$ är inverterbar, får man då $y$ enligt
\begin{align*}
	y(x) = N^{-1}(c - M(x)).
\end{align*}

\paragraph{Linjära första ordningens ordinarie differentialekvationer}
Antag att vi har en differentialekvation på formen
\begin{align*}
	\deval{y}{t}{t} + p(t)y(t) = g(t).
\end{align*}
Beräkna
\begin{align*}
	P(t) = \inteval{a}{t}{p}{x}
\end{align*}
och inför den integrerande faktorn $e^{P(t)}$. Multiplicera med den på båda sidor för att få
\begin{align*}
	e^{P(t)}\deval{y}{t}{t} + p(t)e^{P(t)}y(t) = e^{P(t)}g(t).
\end{align*}
Detta kan skrivas om till
\begin{align*}
	\dv{t}\left(ye^{P}\right)(t) = e^{P(t)}g(t) = \deval{H}{t}{t}.
\end{align*}
Analysens huvudsats ger då
\begin{align*}
	y(t)e^{P(t)} = H(t) + c
\end{align*}
och slutligen
\begin{align*}
	y(t) = ce^{-P(t)} + e^{-P(t)}H(t).
\end{align*}

Låt oss lägga till bivillkoret $y(a) = y_0$. Man kan då visa att lösningen kan skrivas som
\begin{align*}
	y(t) = y_0e^{-\inteval{a}{t}{p}{x}} + \inteval{a}{t}{g(x)e^{-\inteval{x}{t}{p}{s}}}{x}.
\end{align*}

\paragraph{Lipschitzkontinuerlighet}
En funktion $f$ är Lipschitzkontinuerlig om det finns ett $K$ så att det för varje $x_1, x_2$ gäller att
\begin{align*}
	\abs{f(x_1) - f(x_2)} \leq K\abs{x_1 - x_2}.
\end{align*}

\paragraph{Entydighet av lösning av en första ordnings ordinarie differentialekvation}
Betrakta differentialekvationen
\begin{align*}
	&\deval{y}{t}{t} = f(y(t)),\ 0 < t < \tau, \\
	&y_0 = 0.
\end{align*}
Denna har en entydig lösning om $f$ är Lipschitzkontinuerlig.