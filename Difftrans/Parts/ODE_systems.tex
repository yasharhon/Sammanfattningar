\subsection{System av ODE}

\paragraph{Formulering}
Betrakta ett system av funktioner $x_{1}, x_{2}, \dots$ som beskrivs av systemet
\begin{align*}
	\deval{x_{1}}{t}{t} &= g_{1}(t) + \sum p_{1i}(t)x_{i}, \\
	\deval{x_{2}}{t}{t} &= g_{2}(t) + \sum p_{2i}(t)x_{i}, \\
	\vdots
\end{align*}
av differentialekvationer. Definiera
\begin{align*}
	\vb{x}(t) =\left[
	\begin{array}{c}
		x_{1}(t) \\
		x_{2}(t) \\
		\vdots
	\end{array}\right],
	\vb{g}(t) =\left[
	\begin{array}{c}
		g_{1}(t) \\
		g_{2}(t) \\
		\vdots
	\end{array}\right],
	P(t) =\left[
	\begin{array}{ccc}
		p_{11}(t) & p_{12}(t) & \dots \\
		p_{21}(t) & p_{22}(t) & \dots \\
		\vdots    & \vdots    & \ddots
	\end{array}\right].
\end{align*}
Då kan systemet skrivas som
\begin{align*}
	\deval{\vb{x}}{t}{t} = \vb{g}(t) + P\vb{x}(t).
\end{align*}
Detta kan även generaliseras till
\begin{align*}
	\deval{\vb{x}}{t}{t} = \vb{f}(\vb{x}(t)) + \vb{g}(t).
\end{align*}

\paragraph{Autonoma system}
Ett autonomt system är på formen
\begin{align*}
	\deval{\vb{x}}{t}{t} = \vb{f}(\vb{x}(t)).
\end{align*}

\paragraph{Form på lösning av system av ODE}
Låt $\vb{x}_{\text{p}}$ lösa
\begin{align*}
	\deval{\vb{x}}{t}{t} = \vb{g}(t) + P\vb{x}(t).
\end{align*}
Då är alla lösningar på formen
\begin{align*}
	\vb{x} = \vb{x}_{\text{p}} + \vb{x}_{\text{h}}
\end{align*}
där $\vb{x}_{\text{h}}$ löser det motsvarande homogena systemet.

\paragraph{Fundamentalmatris}
Betrakta
\begin{align*}
	\deval{\vb{x}}{t}{t} = P(t)\vb{x}(t)
\end{align*}
med fundamentalt sätt av lösningar $\vb{x}^{(1)}(t), \dots, \vb{x}^{(n)}(t)$. Då definieras systemets fundamentalmatris som
\begin{align*}
	\Psi = \left[\vb{x}^{(1)}(t) \dots \vb{x}^{(n)}(t)\right].
\end{align*}
Vi definierar även den speciella fundamentalmatrisen $\Phi$, vars kolumner satisfierar begynnelsesvillkoret
\begin{align*}
	\vb{x}^{(1)}(t_{0}) =
	\left[\begin{array}{c}
		1      \\
		0      \\
		\vdots \\
		0
	\end{array}\right],
	\dots,
	\vb{x}^{(n)}(t_{0}) =
	\left[\begin{array}{c}
		0      \\
		0      \\
		\vdots \\
		1
	\end{array}\right].
\end{align*}
Det kan visas att denna ges av
\begin{align*}
	\Phi(t) = e^{A(t)t}.
\end{align*}

\paragraph{Linjär kombination av lösningar}
Låt $\vb{x}^{(1)}, \dots, \vb{x}^{(n)}\in\R,\ 0 < t < T$ vara ett fundamentalt sätt av lösningar till
\begin{align*}
	\deval{\vb{x}}{t}{t} = P(t)\vb{x}(t),\ t > 0,
\end{align*}
där $P$ är kontinuerlig. Då kan varje lösning till ekvationen skrivas som
\begin{align*}
	\vb{x} = \sum c_{i}\vb{x}^{(i)}
\end{align*}
på precis ett sätt. Med fundamentalmatrisen kan detta uttryckas som
\begin{align*}
	\vb{x} = \Psi\vb{c},
\end{align*}
där $\vb{c}$ är en vektor med koefficienter.

\subparagraph{Bevis}
Begynnelsesvärdeproblemet implicerar att vår lösning måste uppfylla
\begin{align*}
	\left[\vb{x}^{(1)}(0) \dots \vb{x}^{(n)}(0)\right]
	\left[\begin{array}{c}
		c_{1} \\
		\vdots \\
		c_{n}
	\end{array}\right]
	= \vb{x}(0).
\end{align*}
Detta har bara en lösning om $\abs{\vb{x}^{(1)}(0) \dots \vb{x}^{(n)}(0)} \neq 0$. Eftersom alla lösningarna är linjärt oberoende, är detta uppfylld. Konstanterna $c_{i}$ ges då unikt, och beviset är klart.

\paragraph{System av ODE med konstant matris}
Betrakta
\begin{align*}
	\deval{\vb{x}}{t}{t} = P\vb{x}(t),
\end{align*}
där $P$ är en konstant matris. Vi gör ansatsen $\vb{x}(t) = e^{rt}\vb*{\xi}$ och får
\begin{align*}
	\deval{\vb{x}}{t}{t} - P\vb{x}(t) = e^{rt}(rI - A)\vb*{\xi}.
\end{align*}
Eftersom exponentialfunktionen alltid är nollskild, kan detta bara bli noll om
\begin{align*}
	P\vb*{\xi} = r\vb*{\xi}.
\end{align*}
Alltså är $\vb{x}$ bara en lösning om $\vb*{\xi}$ är en egenvektor till $P$ och $r$ är det motsvarande egenvärdet.

\paragraph{Upprepande egenvärden}
Betrakta
\begin{align*}
	\deval{\vb{x}}{t}{t} = P\vb{x}(t),
\end{align*}
där $P$ är en konstant matris, låt $r$ vara ett egenvärde till $P$ med algebraisk multiplicitet $2$ och geometrisk multiplicitet $1$ och $\vb*{\xi}$ en motsvarande egenvektor. Då är en lösning
\begin{align*}
	\vb{x}^{(1)} = \vb*{\xi}e^{rt}
\end{align*}
och en ny lösning kan skrivas som
\begin{align*}
	\vb{x}^{(2)} = \vb*{\xi}te^{rt} + \vb*{\eta}e^{rt},
\end{align*}
där $\vb*{\eta}$ uppfyller
\begin{align*}
	(A - rI)\vb*{\eta} = \vb*{\xi}.
\end{align*}

\paragraph{Wronskianen för ett system med konstant matris}
Betrakta
\begin{align*}
	\deval{\vb{x}}{t}{t} = P\vb{x}(t),
\end{align*}
där $P$ är en konstant matris. Låt $\vb*{\xi}_{i}$ vara de olika egenvektorerna till $P$ motsvarande egenvärden $r_{i}$. Wronskianen till dessa ges av
\begin{align*}
	W(\vb*{\xi}_{1}, \dots, \vb*{\xi}_{n})(t) &= \mdet{e^{r_{1}t}\vb*{\xi}_{1} & \dots & e^{r_{n}t}\vb*{\xi}_{n}} \\
	                                          &= e^{(r_{1} + \dots + r_{n})}\mdet{\vb*{\xi}_{1} & \dots & \vb*{\xi}_{n}} \\
	                                          &= W(\vb*{\xi}_{1}, \dots, \vb*{\xi}_{n})(0)e^{\Tr{P}t},
\end{align*}
där vi har använt en sats för att få fram spåret. Det följer blant annat att Wronskianen är antingen $0$ eller nollskild överallt.

\paragraph{Diagonalisering}
Betrakta
\begin{align*}
	\deval{\vb{x}}{t}{t} = A\vb{x}(t),
\end{align*}
där $A$ är en konstant matris som kan skrivas som $A = PDP^{-1}$. Då kan vi införa $\vb{x}= P\vb{y}$, vilket ger
\begin{align*}
	\deval{\vb{x}}{t}{t} &= P\deval{\vb{y}}{t}{t} = PDP^{-1}\vb{y} = PD\vb{y}, \\
	\deval{\vb{y}}{t}{t} &= D\vb{y},
\end{align*}
vilket är en simplare variant av det ursprungliga problemet.

\paragraph{Partikulärlösningar}
Betrakta
\begin{align*}
	\deval{\vb{x}}{t}{t} = \vb{g}(t) + P\vb{x}(t).
\end{align*}
Det finns olika metoder att ta fram en partikulärlösning av detta.

\subparagraph{Diagonalisering}
Låt $P$ vara diagonaliserbar och konstant. Då får man vid diagonalisering att
\begin{align*}
	\deval{\vb{y}}{t}{t} = \vb{h}(t) + D\vb{y}(t)
\end{align*}
med $\vb{h} = T^{-1}\vb{g}$. Varje komponent kan då lösas som
\begin{align*}
	y_{j}(t) = c_{j}e^{r_{j}t} + e^{r_{j}t}\inteval{t_{0}}{t}{h_{j}(s)e^{-r_{j}s}}{s}.
\end{align*}

\subparagraph{Obestämda koefficienter}
Om $\vb{g}$ har en enkel form, kan man gissa på en lösning och bestämma koefficienterna baserad på ens gissning.

\subparagraph{Variation av parametrar}
Ansätt
\begin{align*}
	\vb{x}(t) = \Psi(t)\vb{u}(t).
\end{align*}
Då ger differentialekvationen
\begin{align*}
	\deval{\Psi}{t}{t}\vb{u}(t) + \Psi(t)\deval{\vb{u}}{t}{t} = P(t)\Psi(t)\vb{u}(t) + \vb{g}(t).
\end{align*}
Eftersom $\Psi$ är en fundamentalmatris för ekvationen, gäller att
\begin{align*}
	\deval{\Psi}{t}{t}P(t)\Psi(t),
\end{align*}
och vi får
\begin{align*}
	\Psi(t)\deval{\vb{u}}{t}{t} = \vb{g}(t).
\end{align*}
Vi löser för $\vb{u}$ och integrerar, vilket slutligen ger
\begin{align*}
	\vb{x}(t) = \Psi(t)\vb{c} + \Psi(t)\inteval{t_{0}}{t}{\Psi^{-1}(s)\vb{g}(s)}{s}.
\end{align*}