\section{Fourierserier}

\subsection{Definitioner}

\paragraph{Positiva summationskärnor}
En positiv summationskärna är en Riemannintegrerbar funktion $K_{n}$ på $(-a, a)$ som uppfyller
\begin{itemize}
	\item $K_{n}(x)\geq 0, x\in (-a, a)$.
	\item $\inteval{-a}{a}{K_{n}(x)}{x} = 1$.
	\item för varje $\delta > 0$ så gäller att $\lim\limits_{n\to\infty}\inteval{\delta < \abs{x} < a}{}{K_{n}(x)}{x} = 0$.
\end{itemize}

\paragraph{Cesaro-summerbarhet}
Låt
\begin{align*}
	s_{n} = \sum\limits_{k = 0}^{n}a_{k}
\end{align*}
och
\begin{align*}
	\sigma_{N} = \frac{1}{N}\sum\limits_{k = 0}^{N - 1}s_{k}.
\end{align*}
Då är $\sum\limits_{k = 0}^{n}a_{k}$ Cesarosummerbar om $\sigma_{N}$ konvergerar.

\paragraph{Summationskärnor}
Borde definieras

\paragraph{Feijerkärnan}
Vi definierar Feijerkärnan som
\begin{align*}
	F_{N}(t) = \frac{1}{N}\sum\limits_{k = 0}^{N - 1}D_{k}(t).
\end{align*}

\subsection{Satser}

\paragraph{Formel för Fourierkoefficienter}
Antag att en funktion $f$ kan skrivas som
\begin{align*}
	f(x) = \sum\limits_{n\in\Z}c_{n}e^{inx},
\end{align*}
där$\sum\limits_{n\in\Z}\abs{c_{n}}$ är begränsad. Då gäller det att
\begin{align*}
	c_{n} = \frac{1}{2\pi}\inteval{-\pi}{\pi}{f(t)e^{-int}}{t}.
\end{align*}

\subparagraph{Bevis}
Om antagandet är sant, skulle det gälla att
\begin{align*}
	c_{n} = \frac{1}{2\pi}\inteval{-\pi}{\pi}{\sum\limits_{m\in\Z}c_{m}e^{imt}e^{-int}}{t}.
\end{align*}
Antag först att det finns ett $N$ så att $c_{n} = 0$ för $\abs{n} > N$. Detta ger
\begin{align*}
	\inteval{-\pi}{\pi}{\sum\limits_{m\in\Z}c_{m}e^{imt}e^{-int}}{t} = \sum\limits_{m\in\Z}c_{m}\inteval{-\pi}{\pi}{e^{i(m - n)t}}{t}.
\end{align*}
Den återstående integralen ges av
\begin{align*}
	\inteval{-\pi}{\pi}{e^{i(m - n)t}}{t} =
	\begin{cases}
		\inteval{-\pi}{\pi}{}{t} = 2\pi,                                       &m = n, \\
		\frac{1}{i(m - n)}\left(e^{i(m - n)\pi} - e^{-i(m - n)\pi}\right) = \frac{e^{i(m - n)\pi}}{i(m - n)}\left(1 - e^{-2\pi i(m - n)}\right) = 0, &m\neq 0,
	\end{cases}
\end{align*}
och satsen stämmer.

Låt nu $n$ vara givet och välj ett $N$ så att $n < N$ och $\sum\limits_{\abs{n} > N}\abs{c_{n}} < \varepsilon$. Detta ger
\begin{align*}
	\abs{\frac{1}{2\pi}\inteval{-\pi}{\pi}{f(t)e^{-int}}{t} - \frac{1}{2\pi}\inteval{-\pi}{\pi}{\sum\limits_{m\in\Z}c_{m}e^{imt}e^{-int}}{t}} &= \frac{1}{2\pi}\abs{\inteval{-\pi}{\pi}{\sum\limits_{\abs{m} > N}c_{m}e^{i(m - n)t}}{t}} \\
	                                      &\leq \frac{1}{2\pi}\inteval{-\pi}{\pi}{\sum\limits_{\abs{m} > N}\abs{c_{m}e^{i(m - n)t}}}{t} \\
	                                      &= \frac{1}{2\pi}\inteval{-\pi}{\pi}{\sum\limits_{\abs{m} > N}\abs{c_{m}}}{t} \\
	                                      &= \frac{1}{2\pi}\sum\limits_{\abs{m} > N}2\pi\abs{c_{m}} \\
	                                      &= \sum\limits_{\abs{m} > N}\abs{c_{m}} < \varepsilon.
\end{align*}
Enligt vårt tidigare argument gäller det även att
\begin{align*}
	\frac{1}{2\pi}\inteval{-\pi}{\pi}{\sum\limits_{m\in\Z}c_{m}e^{imt}e^{-int}}{t} = c_{n},
\end{align*}
och vi har alltså visat
\begin{align*}
	\abs{\frac{1}{2\pi}\inteval{-\pi}{\pi}{f(t)e^{-int}}{t} - \frac{1}{2\pi}\inteval{-\pi}{\pi}{\sum\limits_{m\in\Z}c_{m}e^{imt}e^{-int}}{t}} &= \frac{1}{2\pi}\abs{\inteval{-\pi}{\pi}{\sum\limits_{\abs{m} > N}c_{m}e^{i(m - n)t}}{t}} < \varepsilon,
\end{align*}
vilket ger att satsen stämmer.

\paragraph{Konvergens och Cesarosummerbarhet}
Låt
\begin{align*}
	\sum\limits_{i = 0}^{\infty}a_{i} = s.
\end{align*}
Då är $\sum\limits_{i = 0}^{\infty}a_{i}$ och har värdet $s$ även i denna mening.

\proof
Vi har
\begin{align*}
	\abs{\sigma_{n} - s} &= \abs{\frac{1}{n}\sum\limits_{i = 0}^{\infty}s_{i} - s} \\
	                     &= \abs{\frac{(s_{0} - s) + \dots + (s_{n - 1} - s)}{n}} \\
	                     &\leq  \abs{\frac{(s_{0} - s) + \dots + (s_{N} - s)}{n}} + \abs{\frac{(s_{N + 1} - s) + \dots + (s_{n - 1} - s)}{n}}.
\end{align*}
Eftersom $s_{n}$ konvergerar, finns det ett $N$ så att när $n > N$ är $\abs{s_{n} - s} < \varepsilon$ för något $\varepsilon > 0$. Låt nu $N$ i ekvationen över vara så att när $n > N$ är $\abs{s_{n} - s} < \frac{\varepsilon}{2}$. Detta ger
\begin{align*}
	\abs{\frac{(s_{N + 1} - s) + \dots + (s_{n - 1} - s)}{n}} < \frac{n - N - 1}{n}\frac{\varepsilon}{2} < \frac{\varepsilon}{2}.
\end{align*}
Olikheten över kan då skrivas som
\begin{align*}
	\abs{\sigma_{n} - s} \leq \frac{C_{N}}{n} + \frac{\varepsilon}{2}.
\end{align*}
Välj nu $N$ igen så att $N = \frac{2C_{N}}{\varepsilon}$. Då blir olikheten:
\begin{align*}
	\abs{\sigma_{n} - s} \leq \varepsilon,
\end{align*}
och beviset är klart.

\paragraph{Hjälpsats för konvergens av Fourierserier}
Låt $f$ vara en Riemannintegrerbar funktion på $(-a, a)$, specifikt så att $\abs{f(x)}\leq M, x\in (-a, a)$, och kontinuerlig i $x = 0$. Antag vidare att $K_{n}$ en positiv summationskärna på $(-a, a)$. Då gäller att
\begin{align*}
	\lim\limits_{n\to\infty}\inteval{-a}{a}{K_{n}(x)f(x)}{x} = f(0).
\end{align*}

\proof
Vi vill visa att för alla $\varepsilon > 0$ existerar ett $M > 0$ så att om $n > M$ så är
\begin{align*}
	\abs{\inteval{-a}{a}{K_{n}(x)f(x)}{x} - f(0)} < \varepsilon.
\end{align*}

Vi har att
\begin{align*}
         &\abs{\inteval{-a}{a}{K_{n}(x)f(x)}{x} - f(0)} \\
	=    &\abs{\inteval{-a}{a}{K_{n}(x)f(x)}{x} - f(0)\inteval{-a}{a}{K_{n}(x)}{x}} \\
	=    & \abs{\inteval{-a}{a}{K_{n}(x)(f(x) - f(0))}{x}} \\
	=    & \abs{\inteval{-\delta}{\delta}{K_{n}(x)(f(x) - f(0))}{x} + \inteval{\delta < \abs{x} < a}{}{K_{n}(x)(f(x) - f(0))}{x}} \\
	\leq & \abs{\inteval{-\delta}{\delta}{K_{n}(x)(f(x) - f(0))}{x}} + \abs{\inteval{\delta < \abs{x} < a}{}{K_{n}(x)(f(x) - f(0))}{x}}.
\end{align*}
Eftersom $f$ är kontinuerlig, finns det ett $J > 0$ sådant att $\abs{f(x) - f(0)} < J$ när $\abs{x} < a$. Då kan vi skriva
\begin{align*}
	     &\abs{\inteval{-\delta}{\delta}{K_{n}(x)(f(x) - f(0))}{x}} + \abs{\inteval{\delta < \abs{x} < a}{}{K_{n}(x)(f(x) - f(0))}{x}} \\
	\leq & \abs{\inteval{-\delta}{\delta}{K_{n}(x)(f(x) - f(0))}{x}} + J\abs{\inteval{\delta < \abs{x} < a}{}{K_{n}(x)}{x}}.
\end{align*}
Tills nu har vi inte specifierat vårat $\delta$. Välj nu det så att $\abs{x} < \delta\implies\abs{f(x) - f(0)} < \frac{1}{2}\varepsilon$. Använd vidare att eftersom $\lim\limits_{n\to\infty}\inteval{\delta < \abs{x} < a}{}{K_{n}(x)}{x} = 0$ finns det ett $M > 0$ så att $n > M\implies \abs{\inteval{\delta < \abs{x} < a}{}{K_{n}(x)}{x}} < \frac{\varepsilon}{2J}$. Alltså har vi för ett tillräckligt stort $n$ att
\begin{align*}
	\abs{\inteval{-\delta}{\delta}{K_{n}(x)(f(x) - f(0))}{x}} + J\abs{\inteval{\delta < \abs{x} < a}{}{K_{n}(x)}{x}} < \frac{1}{2}\varepsilon\abs{\inteval{-\delta}{\delta}{K_{n}(x)}{x}} + J\frac{\varepsilon}{2J} = \varepsilon,
\end{align*}
och beviset är klart.

\paragraph{Följdsats}
Låt $K_{n}$ vara en positiv summationskärna, $f$ integrerbar och $f$ kontinuerlig i $x$. Då gäller det att
\begin{align*}
	\lim\limits_{n\to\infty}\inteval{-\pi}{\pi}{K_{n}(t)f(x - t)}{t} = f(x).
\end{align*}

\proof
Tillämpa satsen ovan på $g(t) = f(x - t)$.

\paragraph{Hjälpsats för Feijerkärnan}
$F_{N}$ är en positiv summationskärna på $[-\pi, \pi]$.

\proof
$F_{N}$ är en summa av Riemannintegrerbara funktioner, och är därmed Riemannintegrerbar.

För att visa att $F_{N} \geq 0$, skriv
\begin{align*}
	F_{N}(x) &= \frac{1}{2\pi(N + 1}\sum\limits_{n = 0}^{N}\frac{\sin{\left(\frac{n + 1}{2}t\right)}}{\sin{\left(\frac{1}{2}t\right)}} \\
	         &= \frac{1}{2\pi(N + 1}\sum\limits_{n = 0}^{N}\frac{\lambda^{n + 1} - \lambda^{-n - 1}}{\lambda - \lambda^{-1}},
\end{align*}
där $\lambda = e^{\frac{1}{2}ix}$. Det kan visas att detta ger
\begin{align*}
	F_{N}(t) = \frac{1}{2\pi (N + 1)}\left(\frac{\sin{\left(\frac{N + 1}{2}t\right)}}{\sin{\left(\frac{1}{2}t\right)}}\right)^{2}.
\end{align*}

För att visa det andra påståendet, använder vi att
\begin{align*}
	\inteval{-\pi}{\pi}{D_{n}(x)}{x} = \frac{1}{2\pi}\inteval{-\pi}{\pi}{\sum\limits_{}^{}e^{ikx}}{x} = 1 + \frac{1}{2\pi}\sum\limits_{}^{}e^{ikx}... = 1,
\end{align*}
vilket ger
\begin{align*}
	\inteval{-\pi}{\pi}{F_{N}(t)}{t} = \frac{1}{N + 1}\inteval{-\pi}{\pi}{\sum\limits_{}^{}D_{n}(x)}{x} = \frac{1}{N + 1}\sum\limits_{}^{}1.
\end{align*}

För att visa det tredje påståendet, skriv
\begin{align*}
	\inteval{\delta < \abs{x} < \pi}{}{F_{N}(t)}{t} &= \inteval{\delta < \abs{x} < \pi}{}{\frac{1}{2\pi (N + 1)}\left(\frac{\sin{\left(\frac{N + 1}{2}t\right)}}{\sin{\left(\frac{1}{2}t\right)}}\right)^{2}}{t} \\
	                                                &\leq \inteval{\delta < \abs{x} < \pi}{}{\frac{1}{2\pi (N + 1)}\left(\frac{1}{\sin{\left(\frac{1}{2}\delta\right)}}\right)^{2}}{t} \\
	                                                &\leq \frac{1}{(N + 1)}\left(\frac{1}{\sin{\left(\frac{1}{2}\delta\right)}}\right)^{2} \\
	                                                &\to 0. 
\end{align*}

\paragraph{Följdsats}
\begin{align*}
	\inteval{-\pi}{\pi}{F_{N}(t)f(x - t)}{t} = f(x).
\end{align*}

\proof

\paragraph{Annan följdsats}
Antag att $f$ är integrerbar på enhetscirkeln och att alla dens Fourierkoefficienter är $0$. Då är $f(x) = 0$ överallt där $f$ är kontinuerlig.

\proof

\paragraph{Riemann-Lebesgues lemma}
Antag att $f$ är Riemannintegrerbar på $(-\pi, pi]$. Då är
\begin{align*}
	\lim\limits_{\lambda\to\infty}\inteval{-\pi}{\pi}{f(t)e^{i\lambda t}}{t} = 0.
\end{align*}

\proof
Låt $s(x)$ vara en undertrappa till $f$. Eftersom $f$ är Riemannintegrerbar så finns det ett $s$ så att
\begin{align*}
	\inteval{-\pi}{\pi}{\abs{f(x) - s(x)}}{x} < \frac{\varepsilon}{2}.
\end{align*}
Detta ger
\begin{align*}
	\abs{\inteval{-\pi}{\pi}{f(t)e^{i\lambda t}}{t}} &\leq \abs{\inteval{-\pi}{\pi}{s(t)e^{i\lambda t}}{t}} + \abs{\inteval{-\pi}{\pi}{(f(t) - s(t))e^{i\lambda t}}{t}} \\
	                                                 &\leq \abs{\inteval{-\pi}{\pi}{s(t)e^{i\lambda t}}{t}} + \inteval{-\pi}{\pi}{\abs{f(t) - s(t)}}{t} \\
	                                                 &\leq \abs{\inteval{-\pi}{\pi}{s(t)e^{i\lambda t}}{t}} + \frac{\varepsilon}{2}.
\end{align*}
Vi har vidare
\begin{align*}
	\abs{\inteval{-\pi}{\pi}{s(t)e^{i\lambda t}}{t}} &= \abs{\sum\limits_{j = 1}^{n}\inteval{x_{j - 1}}{x_{j}}{m_{j}e^{i\lambda t}}{t}} \\
	                                                 &= \abs{\sum\limits_{j = 1}^{n}\frac{m_{j}}{i\lambda}(e^{i\lambda x_{j}} - e^{i\lambda x_{j - 1}})} \\
	                                                 &\leq \sum\limits_{j = 1}^{n}\frac{2\abs{m_{j}}}{\lambda} \\
	                                                 &\leq \frac{2nM}{\lambda},
\end{align*}
där $M = \sup{\{\abs{m_{1}}, \dots, \abs{m_{n}}\}}$. För $lambda > \frac{4nM}{\varepsilon}$ fås
\begin{align*}
	\abs{\inteval{-\pi}{\pi}{s(t)e^{i\lambda t}}{t}} \leq \varepsilon,
\end{align*}
och beviset är klart.

%Gör en uppdelning $P = \{x_{0}, \dots, x_{n}\}$ av ett intervall $[a, b]$. Då fås undertrapporna av
%\begin{align*}
%	L(f, P) = \sum\limits_{i = 1)^{n}(x_{i} - x_{i - 1})m_{i}
%\end{align*}
%och undertrapporna av
%\begin{align*}
%	U(f, P) = \sum\limits_{i = 1)^{n}(x_{i} - x_{i - 1})M_{i}
%\end{align*}
%där alla $m_{i}$ och $M_{i}$

\paragraph{Konvergens av Fourierserier}

\proof
Vi vill hitta ett villkor på $f$ så att för alla $\varepsilon > 0$ finns det ett $K > 0$ så att
\begin{align*}
	N > K \implies \abs{f(x) - \sum\limits_{n = -N}^{N}c_{n}e^{inx}} < \varepsilon,
\end{align*}
där
\begin{align*}
	c_{n} = \frac{1}{2\pi}\inteval{-\pi}{\pi}{f(t)e^{-int}}{t}.
\end{align*}
Vi definierar
\begin{align*}
	S_{N}(x) = \sum\limits_{n = -N}^{N}c_{n}e^{inx}
\end{align*}
och beräknar den som
\begin{align*}
	S_{N}(x) &= \frac{1}{2\pi}\sum\limits_{n = -N}^{N}\inteval{-\pi}{\pi}{f(t)e^{in(x - t)}}{t} \\
	         &= \frac{1}{2\pi}\inteval{-\pi}{\pi}{f(t)\frac{1 - e^{-i(2N + 1)(x - t)}}{1 - e^{-i(x - t)}}e^{-iN(x - t)}}{t}.
\end{align*}
Vi definierar Dirichletkärnan
\begin{align*}
	D_{N}(\alpha) &= \frac{1}{2\pi}\frac{1 - e^{i(2N + 1)\alpha}}{1 - e^{i\alpha}}e^{-iN\alpha} \\
	              &= \frac{1}{2\pi}e^{-iN\alpha}\frac{e^{i\left(N + \frac{1}{2}\right)\alpha}}{e^{\frac{1}{2}i\alpha}}\frac{e^{-i\left(N + \frac{1}{2}\right)\alpha} - e^{i\left(N + \frac{1}{2}\right)\alpha}}{e^{-\frac{1}{2}i\alpha} - e^{\frac{1}{2}i\alpha}} \\
	              &= \frac{1}{2\pi}\frac{\sin{\left(N + \frac{1}{2}\right)\alpha}}{\sin{\frac{1}{2}\alpha}}.
\end{align*}
Då kan vi skriva
\begin{align*}
	S_{N}(x) = \inteval{-\pi}{\pi}{f(t)D_{N}(x - t)}{t}.
\end{align*}