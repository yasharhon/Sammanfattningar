\section{Funktioner, vektorrum och dylikt}

\subsection{Definitioner}

\paragraph{Inreprodukt}
På kontinuerliga funktioner $[-\pi, \pi] \to \C$ definierar vi inreprodukten
\begin{align*}
	\inprod{u}{v} = \frac{1}{2\pi}\inteval{-\pi}{\pi}{u(t)\cc{v}(t)}{t}.
\end{align*}

\paragraph{Fullständiga mängder}
Låt $\left\{e_{i}\right\}_{i\in I}$ vara en ortonormal mängd vektorer i $V$. Då är $\left\{e_{i}\right\}_{i\in I}$ en fullständig mängd i $V$ om det för varje $u\in V$ och $\varepsilon >0 \exists N >0$ så att
\begin{align*}
	n > N\implies \abs{u - \sum\limits_{i\in I}\inprod{u}{e_{i}}e_{i}} < \varepsilon.
\end{align*}

\paragraph{$\ell^{2}$}
Vi definierar $\ell^{2}(\Z)$ som mängden av alla följder av komplexa tal med inreprodukt
\begin{align*}
	\inprod{u}{v} = \sum\limits_{i = 0}^{n}a_{i}\cc{b_{i}}.
\end{align*}

\paragraph{$L^{2}$}
Vi definierar $L^{2}([-\pi, \pi], \C)$ som mängden av alla komplexvärda Riemannintegrerbara funktioner på $[-\pi, \pi]$ med inreprodukten
\begin{align*}
	\inprod{u}{v} = \frac{1}{2\pi}\inteval{-\pi}{\pi}{u(x)\cc{v}(x)}{x}.
\end{align*}
I denna klassen definieras två funktioner som lika om $\norm{u - v} = 0$.

\paragraph{Legendrepolynomen}
Legendrepolynomen är en följd av polynom så att $p_{0}(x) = 1$, $p_{i}(1) = 1$ och $p_{n}$ är ortogonal på $p_{0}, \dots, p_{n - 1}$ med inreprodukten
\begin{align*}
	\inprod{u}{v} = \inteval{-1}{1}{u(x)\cc{v}(x)}{x}.
\end{align*}

\subsection{Satser}

\paragraph{Projektion och minsta avstånd}
Låt $e_{1}, \dots, e_{N}$ vara ortonormala basvektorer i inreproduktrummet $V$, och låt
\begin{align*}
	V_{N} = \left\{\sum\limits_{i = 1}^{N}a_{i}e_{i}, i\in\C\right\}.
\end{align*}
Då ges
\begin{align*}
	\inf\limits_{\Phi\in V_{N}}\abs{u - \Phi}
\end{align*}
av
\begin{align*}
	\Phi = \sum\limits_{i = 1}^{N}\inprod{u}{e_{i}}e_{i}.
\end{align*}

\proof

\subparagraph{Följdsats}
De komplexa tal $c_{-N}, \dots, c_{N}$ som minimerar
\begin{align*}
	\norm{u - \sum\limits_{n = -N}^{N}c_{n}e^{inx}}
\end{align*}
är
\begin{align*}
	c_{n} = \inprod{u}{e^{inx}}.
\end{align*}

\proof
Betrakta
\begin{align*}
	\norm{u - \sum\limits_{i = 1}^{N}\gamma_{i}e_{i}}.
\end{align*}
Vi utvecklar inreprodukten och får
\begin{align*}
	\norm{u - \sum\limits_{i = 1}^{N}\gamma_{i}e_{i}} = \norm{u}^{2} - \inprod{u}{\sum\limits_{i = 1}^{N}\gamma_{i}e_{i}} - \cc{\inprod{u}{\sum\limits_{i = 1}^{N}\gamma_{i}e_{i}}} + \norm{\sum\limits_{i = 1}^{N}\gamma_{i}e_{i}}^{2}.
\end{align*}
Vid att skriva $u$ som sin Fourierserie fås
\begin{align*}
	\norm{u - \sum\limits_{i = 1}^{N}\gamma_{i}e_{i}} = \norm{u}^{2} - \sum\limits_{i = 1}^{N}c_{i}\gamma_{i} - \sum\limits_{i = 1}^{N}\cc{c_{i}\gamma_{i}} + \sum\limits_{i = 1}^{N}\abs{\gamma_{i}}^{2}.
\end{align*}
Vid att lägga till och ta bort $\sum\limits_{i = 1}^{N}\abs{c_{i}}^{2}$ fås
\begin{align*}
	\norm{u - \sum\limits_{i = 1}^{N}\gamma_{i}e_{i}} &= \norm{u}^{2} - \sum\limits_{i = 1}^{N}\abs{c_{i}}^{2} + \sum\limits_{i = 1}^{N}\abs{c_{i}}^{2} - \sum\limits_{i = 1}^{N}c_{i}\gamma_{i} - \sum\limits_{i = 1}^{N}\cc{c_{i}\gamma_{i}} + \sum\limits_{i = 1}^{N}\abs{\gamma_{i}}^{2} \\
	                                                 &= \norm{u}^{2} - \sum\limits_{i = 1}^{N}\abs{c_{i}}^{2} + \sum\limits_{i = 1}^{N}\abs{c_{i} - \gamma_{i}}^{2}.
\end{align*}
Vid att välja $c_{i} = \gamma_{i}$ fås det efterfrågade minimumet.

\subparagraph{Mindre följdsats}
\begin{align*}
	\norm{u}^{2} = \sum\limits_{i = 1}^{N}\abs{c_{i}}^{2}.
\end{align*}
%Stämmer detta?

\paragraph{Parsevals sats}
Låt $\{\phi_{i}\}_{i = -\infty}^{\infty}$ vara ett ortonormalt system i $V$ och det för varje $u$ gälla att
\begin{align*}
	\norm{u}^{2} = \sum\limits_{i = -\infty}^{\infty}\abs{\inprod{\phi_{i}}{u}}^{2}.
\end{align*}
Då är $\{\phi_{i}\}_{i = 0}^{N}$ fullständig. Det omvända gäller även.

\proof
Om $\{e_{i}\}_{i = -\infty}^{\infty}$ är ett fullständigt ortonormalt system, kan vi välja $c_{i}$ så att för $N > M$ gäller det att
\begin{align*}
	\norm{u - \sum\limits_{i = -N}^{N}c_{i}\phi_{i}} < \varepsilon.
\end{align*}
Vänstersidan kan med satsen ovan skrivas som
\begin{align*}
	\norm{u - \sum\limits_{i = -N}^{N}c_{i}\phi_{i}}^{2} &= \norm{u}^{2} - \norm{\sum\limits_{i = -n}^{n}c_{i}\phi_{i}}^{2} \\
	                                                    &= \norm{u}^{2} - \norm{\sum\limits_{i = -N}^{N}\inprod{\phi_{i}}{u}\phi_{i}}^{2}.
\end{align*}
Vi får
\begin{align*}
	\norm{u}^{2} - \norm{\sum\limits_{i = -N}^{N}\inprod{\phi_{i}}{u}\phi_{i}}^{2} < \varepsilon,
\end{align*}
och eftersom $\varepsilon$ kan väljas godtyckligt, ger detta
\begin{align*}
	\norm{u}^{2} = \sum\limits_{i = -\infty}^{\infty}\abs{\inprod{\phi_{i}}{u}}^{2}.
\end{align*}

Om vi vidare antar att
\begin{align*}
	\norm{u}^{2} = \sum\limits_{i = -\infty}^{\infty}\abs{\inprod{\phi_{i}}{u}}^{2},
\end{align*}
följer motsatsen från definitionen.

\paragraph{Inreprodukt och ortonormal bas}
Låt $u$ och $v$ vara element i $V$, vars bas är ett fullständigt ortonormalt system $\{\phi_{i}\}_{i = 0}^{N}$. Då gäller att
\begin{align*}
	\inprod{u}{v} = \sum\limits_{i = 0}^{N}\inprod{u}{\phi_{i}}\cc{\inprod{v}{\phi_{i}}}.
\end{align*}

\proof
Med projektionen på $V_{N}$ med bas $\{\phi_{i}\}_{i = -\infty}^{\infty}$ får man
\begin{align*}
	\inprod{\proj{V_{N}}{u}}{\proj{V_{N}}{v}} = 
\end{align*}
%Gör klart

\paragraph{Komplexa exponentialfunktionen och $L^{2}$}
$e^{inx}, n\in\Z$ är ett fullständigt system av ortonormala vektorer i $L^{2}([-\pi, \pi], \C)$.

\proof
Låt $u\in L^{2}([-\pi, \pi], \C)$. Då finns det för alla $\varepsilon > 0$ en styckvist konstant funktion $s$ så att
\begin{align*}
	\norm{u(x) - s(x)} < \frac{1}{3}\varepsilon.
\end{align*}
Vi kan skriva
\begin{align*}
	\norm{u - \sum\limits_{n = -N}^{N}c_{n}e^{inx}}\leq \norm{s - \sum\limits_{n = -N}^{N}c_{n}e^{inx}} + \norm{s - u},
\end{align*}
så beviset är klart om vi kan visa att det finns ett $N$ så att
\begin{align*}
	\norm{s - \sum\limits_{n = -N}^{N}c_{n}e^{inx}} < \frac{2}{3}\varepsilon.
\end{align*}
Vi approximerar $s$ med en kontinuerlig funktion $h$ så att $\norm{s - h} < \frac{1}{3}\varepsilon$. Enligt Fejers sats konvergerar $\conv{F_{N}}{h}$ mot $h$ likformigt. Alltså är
\begin{align*}
	\norm{\conv{F_{N}}{h} - h} < \frac{1}{3}\varepsilon
\end{align*}
för tillräckligt stora $N$. Eftersom $\conv{F_{N}}{h}$ är ett trigonometriskt polynom, kan vi välja det och skriva
\begin{align*}
	\norm{s - \conv{F_{N}}{h}} < \norm{\conv{F_{N}}{h} - h} + \norm{s - h} < \frac{2}{3}\varepsilon,
\end{align*}
och därmed är beviset klart.

\paragraph{Polynomapproximation av funktioner}
Låt $f: [a,b]\to\C$. Då finns det för varje $\varepsilon > 0$ ett polynom $p$ så att
\begin{align*}
	\sup\limits_{[a, b]}\abs{f(x) - p(x)} < \varepsilon.
\end{align*}

\proof
Gör en Taylorapproximation av Fourierserien till $f$. Detta ger
\begin{align*}
	\abs{f(x) - p(x)} \leq \abs{f(x) - \sum\limits_{n = -N}^{N}c_{n}e^{inx}} + \abs{\sum\limits_{n = -N}^{N}c_{n}e^{inx} - p(x)} < \varepsilon. 
\end{align*}

\paragraph{Legendrepolynomens fullständighet}
Legendrepolynomen är ett fullständigt system i $L^{2}([-1, 1], \C)$.

\proof
Ideen är att det finns ett polynom $p$ av grad $n$ så att en funktion $f$ kan approximeras godtyckligt väl av detta polynomet i norm-mening. Detta ger
\begin{align*}
	\norm{f - p}^{2} = \inteval{-1}{1}{\abs{f(x) - p(x)}^{2}}{x} \leq 2\varepsilon^{2}.
\end{align*}
Det räcker därmed att visa att alla polynom kan skrivas som en linjärkombination av Legendrepolynom. Detta stämmer eftersom man kan hitta $n + 1$ linjärt oberoende Legendrepolynom.

\paragraph{Jämna och udda Legendrepolynom}
$p_{n}$ är jämn om $n$ är jämn och udda om $n$ är udda.

\proof
Följer från Gram-Schmidt.

\paragraph{Rodrigues' formel}
\begin{align*}
	p_{n}(x) = \frac{1}{2^{n}n!}\dv[n]{(x^{2} - 1) ^{n}}{x}.
\end{align*}

\proof
