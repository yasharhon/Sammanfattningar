\section{Distributioner}

\subsection{Definitioner}

\paragraph{Funktioners stöd}
Stödet av en funktion $f: \R\to\R$ är
\begin{align*}
	\supp{f} = \{x\in\R: f(x)\neq 0\}.
\end{align*}

\paragraph{Kompakt stöd}
$f: \R\to\R$ har kompakt stöd om $\supp{f}\subset [a, b]$.

\paragraph{Testfunktionerna}
Testfunktionerna $D$ är mängden av alla oändligt deriverbara funktioner från $\R$ till $\C$ med kompakt stöd.

\paragraph{Konvergens av testfunktioner}
Följden $\{\phi_{j}\}_{j = 1}^{\infty}$ av testfunktioner definieras som att den konvergerar mot $\phi_{0}$ i $D$ om det finns ett intervall $[a, b]$ så att $\supp{\phi_{j}}, \supp{\phi_{0}}\subset [a, b]$ för alla $j$ och
\begin{align*}
	\max\limits_{x\in\R}\abs{\deval[n]{\phi_{j}}{x}{x} - \deval[n]{\phi_{0}}{x}{x}}\to 0
\end{align*}
då $j\to\infty$ för alla $n\in\Z{+}$.

\paragraph{Distributioner}
Låt $D$ vara mängden av alla testfunktioner. Då är $\phi$ en distribution om den är en kontinuerlig linjär avbildning från $D$ till $\R$ evt. $\C$, och skrivs som $\phi\in D'$. Kontinuitet betyder i detta sammanhanget att $\phi\to\psi$ i $D$ implicerar $f[\phi]\to f[\psi]$ som (komplexa) tal.

\paragraph{Schwarzklassen}
Schwarzklassen är mängden av alla oändligt deriverbara funktioner från $\R$ till $\C$ så att det för alla $k, n\in\Z^{+}$ finns ett $C_{k, n}$ så att
\begin{align*}
	\max\limits_{x\in\R}(1 + \abs{x})^{k}\abs{\deval[n]{\phi}{x}{x}} \leq C_{k, n}.
\end{align*}

\paragraph{Konvergens i Schwarzklassen}
Följden $\{\phi_{j}\}_{j = 1}^{\infty}$ av funktioner i Schwarzklassen definieras som att den konvergerar mot $\phi_{0}$ i $S$ om
\begin{align*}
	\max\limits_{x\in\R}(1 + \abs{x})^{k}\abs{\deval[n]{\phi_{j}}{x}{x} - \deval[n]{\phi_{0}}{x}{x}}\to 0
\end{align*}
då $j\to\infty$ för alla $k, n\in\Z{+}$.

\paragraph{Tempererade distributioner}
De tempererade distributionerna $S'$ är mängden av alla kontinuerliga linjära avbildningar från $S$ till $\R$, evt. $\C$. Kontinuitet betyder i detta sammanhanget att $\phi\to\psi$ i $S$ implicerar $f[\phi]\to f[\psi]$ som (komplexa) tal.

\paragraph{Derivatan av en distribution}
Vi definierar derivatan av en distribution $f$ som
\begin{align*}
	f'[\phi] = - f\left[\dv{\phi}{x}\right].
\end{align*}

\paragraph{Fouriertransformen av en distribution}
Vi definierar Fouriertransformen av en distribution $f$ som
\begin{align*}
	\fou{f}[\phi] = f\left[\fou{f}\right].
\end{align*}

\subsection{Satser}

\paragraph{•}
Låt $\phi\in D'$. Då finns det funktioner $g_{k}\in L^{1}(\R, \R), h_{k}\in C^{\infty}(\R, \R)$ och tal $n_{k}\in\N$ så att för varje $f\in D$ finns det ett $K$ så att
\begin{align*}
	g(f) = \sum\limits_{k = 1}^{K}\inteval{\R}{}{(-1)^{n_{k}}\deval[n_{k}]{h_{k}f}{x}{x}g_{k}(x)}{x}.
\end{align*}

\proof

\paragraph{Fouriertransformens bijektivitet}
Fouriertransformen är en bijektiv avbildning från $S$ till $S$.

\proof