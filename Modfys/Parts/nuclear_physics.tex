\section{Kärnfysik}

\paragraph{Radiokativitet}
Radioaktivitet är ett begrepp för processer där atomkärnan förändras. För att en radioaktiv process skall ske (spontant), måste
\begin{itemize}
	\item processen ge från sig energi.
	\item konserveringslagarna för laddning, rörelsemängd, rörelsemängdsmoment, nukleontalet (inte i allmänna kärnprocesser), leptontalet och universums totala energi måste vara uppfyllda.
\end{itemize}

\paragraph{Aktivitet}
Den radioaktiva aktiviteten definieras som
\begin{align*}
	A = -\dv{N}{t}
\end{align*}
där $N$ är antal radioaktiva partiklar i systemet. Experimentellt avtar exponentiellt med tiden, vilket enkelt kan förklaras om aktiviteten är proportionell med antal radioaktiva partiklar.

\paragraph{Halveringstid}
Om man gör ansatsen $A = \lambda N$ får man $A\propto e^{-\lambda t}$. Halveringstiden $t_{\frac{1}{2}}$ definieras som tiden det tar för aktiviteten (eller, ekvivalent, antal radioaktiva partiklar) att halveras relativt ursprungsaktiviteten. Från detta relateras konstanten till halveringstiden med
\begin{align*}
	\lambda = \frac{\ln{2}}{t_{\frac{1}{2}}}.
\end{align*}

\paragraph{$Q$-värde}
$Q$-värdet för en radioaktiv process är den totala ändringen i energi för processen.

\paragraph{$\alpha$-sönderfall}
I $\alpha$-sönderfall ger kärnan från sig en He-kärna med $2$ protoner och $2$ neutroner. Energispektrumet för denna processen är diskret.

Man kan räkna fram att $\alpha$-partikelns energi ges av
\begin{align*}
	E_{\alpha} = \frac{Q}{1 + \frac{m_{\alpha}}{m_{\text{ny}}}}.
\end{align*}

\paragraph{$\beta^{-}$-sönderfall}
I $\beta^{-}$-sönderfall omdannas en neutron till en proton, och det skapas i processen en elektron och en elektron-antineutrino med symbolen $\bar{\nu}_{e}$. $\bar{\nu}_{e}$ är i princip masslös. Eftersom två partiklar skapas, är energifördelningen för $\beta^{-}$-partikeln (elektronen) kontinuerlig och stannar vid processens $Q$-värde.

Reaktionens $Q$-värde ges av
\begin{align*}
	Q_{\beta^{-}} &= (m_{N, Z} - m_{N - 1, Z + 1} - m_{e^{-}})c^{2} \\
	              &= (M_{N, Z} - M_{N - 1, Z + 1})c^{2},
\end{align*}
där $m$ refererar till kärnmassan och $M$ den totala atommassan. Villkoret för att reaktionen skall ske är
\begin{align*}
	M_{N, Z} > M_{N - 1, Z + 1}.
\end{align*}

\paragraph{$\beta^{+}$-sönderfall}
I $\beta^{+}$-sönderfall omdannas en proton till en neutron, och det skapas i processen en positron, med symbol $e^{+}$ och en elektronneutrino med symbolen $\nu_{e}$. Eftersom två partiklar skapas, är energifördelningen för $\beta^{+}$-partikeln (positronen) kontinuerlig och stannar vid processens $Q$-värde.

Reaktionens $Q$-värde ges av
\begin{align*}
	Q_{\beta^{-}} = (M_{N, Z} - M_{N - 1, Z + 1} - 2m_{e})c^{2},
\end{align*}
där vi nu ser att den nya atomen har plats till en mindre elektron och att vi har skapat en positron (som har massan $m_{e}$). Villkoret för att reaktionen skall ske är
\begin{align*}
	M_{N, Z} > M_{N - 1, Z + 1} + 2m_{e}.
\end{align*}

\paragraph{$\gamma$-strålning}
Atomkärnans energinivåer är även kvantiserade. När kärnan hoppar ned i sina energinivåer, skickas det ut en foton. Gammastrålning kan förekomma efter andra radioaktiva processer, när atomkärnan efter processen är kvar i ett högt energinivå och hoppar ned. Sådana processer har ofta kort halveringstid.

\paragraph{Elektroninfångning}
I elektroninfångning fångas en elektron in i atomkärnan, och bildar till sammans med protonen en neutron. Från detta skickas även ut en $\nu_{e}$. Denna processen har ett diskret energispektrum.

Under processen dyker även upp ett ookkuperat elektrontillstånd. När en elektron hoppar ned till detta tillståndet, skickas karakteristisk röntgenstrålning ut.

\paragraph{Inre konversion}
I stället för att göra sig av med energi genom $\gamma$-sönderfall, kan kärnan ge energin till en elektron som exiteras.