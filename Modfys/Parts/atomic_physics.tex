\section{Atomfysik}

\paragraph{Bohrs modell}
För att förklara väteatomens kvantiserade emissionsspektrum, postulerade Bohr att
\begin{itemize}
	\item elektronerna kretsar i stabila banor runt kärnan.
	\item en övergång mellan två banor ger en foton med energi $h\nu = \Delta E$.
	\item elektronens integrerade rörelsemängd längs med banan kring kärnan är en heltalsmultipel av $h$.
\end{itemize}
Om elektronen har konstant rörelsemängd i banan, ger detta
\begin{align*}
	2\pi rp = nh \implies v = \frac{nh}{2\pi m_{e}r},
\end{align*}
där $r$ är banans radius. Accelerationen i radiell riktning ger vidare
\begin{align*}
	\frac{e^{2}}{4\pi\varepsilon_{0}r^{2}} = \frac{m_{e}v^{2}}{r}
\end{align*}
eftersom atomkärnan har samma laddning som elektronen, med motsatt tecken. Enligt Bohrs postulat är de tillåtna banradierna
\begin{align*}
	r_{n} = \frac{\varepsilon_{0}h^{2}}{\pi e^{2}m_{e}}n^{2}.
\end{align*}
Den totala energin ges av
\begin{align*}
	E &= \frac{1}{2}m_{e}v^{2} - \frac{e^{2}}{4\pi\varepsilon_{0}r} \\
	  &= \frac{n^{2}h^{2}}{8\pi^{2}m_{e}r^{2}} - \frac{m_{e}e^{4}}{4\varepsilon_{0}^{2}h^{2}n^{2}} \\
	  &= \frac{m_{e}e^{4}}{8\varepsilon_{0}^{2}h^{2}n^{2}} - \frac{m_{e}e^{4}}{4\varepsilon_{0}^{2}h^{2}n^{2}} \\
	  &= -\frac{m_{e}e^{4}}{8\varepsilon_{0}^{2}h^{2}}\frac{1}{n^{2}}.
\end{align*}
Detta stämde med vätespektrumet.

\paragraph{Väteatomen}
En viktig experimentell kontroll av kvantmekaniken är väteatomens spekter. Bohr lyckades förklara detta med en modell där kvantiseringen av energi introducerades ad hoc, men vi vill nu se om vi får samma spektrumet med hjälp av kvantmekanik.

För att göra detta, vill vi beskriva ett system av en (fix) proton med en elektron i bana runt sig. I sfäriska koordinater med protonen i centrum blir Hamiltonoperatorn
\begin{align*}
	\hat{H} = -\frac{\hbar^{2}}{2m}\left(\frac{1}{r^{2}}\pdv{r}\left(r^{2}\pdv{r}\right) + \frac{1}{r^{2}\sin(\theta)}\pdv{\theta}\left(\sin(\theta)\pdv{\theta}\right) + \frac{1}{r^{2}\sin^{2}(\theta)}\pdv[2]{\phi}\right) - \frac{e^{2}}{4\pi\varepsilon_{0}r}.
\end{align*}
Att ens skriva upp Schrödingerekvationen är extremt jobbigt, så vi försöker i stället att separera den. Detta ger
\begin{align*}
	\frac{\sin^{2}(\theta)}{R}\pdv{r}\left(r^{2}\pdv{R}{r}\right) + \frac{\sin(\theta)}{\Theta}\pdv{\theta}\left(\sin{\theta}\pdv{\Theta}{\theta}\right) + \frac{1}{\Phi}\pdv[2]{\Phi}{\phi} + \frac{2mr^{2}\sin^{2}(\theta)}{\hbar^{2}}\left(E + \frac{e^{2}}{4\pi\varepsilon_{0}r}\right) = 0.
\end{align*}
Vi separerar först bort $\phi$-termen, och setter den lika med en konstant som vi kallar $-m_{l}^{2}$ (som kan vara vad som helst om vi inte exkluderar komplexa tal). Vidare får man
\begin{align*}
	\frac{1}{R}\pdv{r}\left(r^{2}\pdv{R}{r}\right) + \frac{1}{\sin(\theta)\Theta}\pdv{\theta}\left(\sin{\theta}\pdv{\Theta}{\theta}\right) - \frac{m_{l}^{2}}{\sin^{2}(\theta)} + \frac{2mr^{2}}{\hbar^{2}}\left(E + \frac{e^{2}}{4\pi\varepsilon_{0}r}\right) = 0
\end{align*}
Nu kan vi separera igen, med separationskonstant $l(l + 1)$. Vi får alltså tre ekvationer
\begin{align*}
	\frac{1}{\Phi}\pdv[2]{\Phi}{\phi}     &= -m_{l}^{2}, \\
	\frac{1}{R}\pdv{r}\left(r^{2}\pdv{R}{r}\right) + \frac{2mr^{2}}{\hbar^{2}}\left(E + \frac{e^{2}}{4\pi\varepsilon_{0}r}\right) &= l(l + 1), \\
	\frac{1}{\sin(\theta)\Theta}\pdv{\theta}\left(\sin{\theta}\pdv{\Theta}{\theta}\right) - \frac{m_{l}^{2}}{\sin^{2}(\theta)}      &= l(l + 1).
\end{align*}
Vi förenklar till
\begin{align*}
	\pdv[2]{\Phi}{\phi} + m_{l}^{2}\Phi                          &= 0, \\
	\pdv{r}\left(r^{2}\pdv{R}{r}\right) + \frac{2mr^{2}}{\hbar^{2}}\left(E + \frac{e^{2}}{4\pi\varepsilon_{0}r} - l(l + 1)\right)R                        &= 0, \\
	\frac{1}{\sin(\theta)}\pdv{\theta}\left(\sin{\theta}\pdv{\Theta}{\theta}\right) - \left(\frac{m_{l}^{2}}{\sin^{2}(\theta)} + l(l + 1)\right)\Theta &= 0.
\end{align*}
Det här är svårt att lösa, och lösningarna är så speciella att de är namngivna efter olika personer, men randvillkoren kommer att ge
\begin{align*}
	l     &= 0, 1, \dots, n - 1, \\
	m_{l} &= 0, \pm 1, \dots, \pm l.
\end{align*}
$l$ kallas för bankvanttalet och $m_{l}$ kallas för det magnetiska kvanttalet. Det sista kvanttalet $n$ indikerar energin.

Det återstår nu att lösa den radiella ekvationen. Vi gör heller inte det, men de tillåtna energierna är
\begin{align*}
	E = -\frac{m_{e}e^{4}}{8\varepsilon_{0}^{2}h^{2}}\frac{1}{n^{2}} = -\frac{m_{e}e^{4}}{2(4\pi\varepsilon_{0})^{2}\hbar^{2}}\frac{1}{n^{2}}.
\end{align*}
Från detta kan vi definiera Bohrradien
\begin{align*}
	a_{0} = \frac{4\pi\varepsilon_{0}\hbar^{2}}{m_{e}e^{2}}.
\end{align*}

\paragraph{Rörelsemängdsmoment i atomen}
Det visar sig att $Y = \Theta\Phi$ är egentillstånd till båda $\hat{L^{2}}$ och $\vb{L}_{z}$ med egenvärden $\hbar l(l + 1)$ respektiva $\hbar m_{l}$.

\paragraph{Magnetiskt moment i atomen}
Om vi betraktar en elektron i en planär cirkulär bana kring kärnan, ges det magnetiska momentet av
\begin{align*}
	\mu = IA = \frac{e}{\tau}\pi r^{2} = \frac{ev}{2\pi r}\pi r^{2} = \frac{evr}{2} = \frac{e}{2m_{e}}\abs{\vb{L}}.
\end{align*}
Från detta får vi en vektor
\begin{align*}
	\vb{mu} = -\frac{e}{2m_{e}}\vb{L},
\end{align*}
och speciellt
\begin{align*}
	\mu_{z} = -\frac{e}{2m_{e}}L_{z} = -\frac{e\hbar}{2m_{e}}m_{l} = -\mu_{\text{B}}m_{l},
\end{align*}
där vi har definierat Bohrmagnetonen
\begin{align*}
	\mu_{\text{B}} = \frac{e\hbar}{2m_{e}}.
\end{align*}

Om vi nu applicerar ett externt magnetiskt fält i $z$-riktningen, ger detta upphov till en potential
\begin{align*}
	U = -\vb{\mu}\times\vb{B} = \mu_{\text{B}}m_{l}B.
\end{align*}

Även spinnet ger upphov till ett magnetiskt moment
\begin{align*}
	\vb{\mu}_{S} = -g{S}\frac{e}{2m_{e}}\vb{S},
\end{align*}
där $g_{S}$ är det gyromagnetiska förhållandet som (nästan) alltid är lika med $2$.

\paragraph{Zeeman-effekten}
När en atom är i ett externt magnetiskt fält, splittas atomspektrumet därför att olika rörelsemängdsmoment har olika energier på grund av den uppkomna potentialen.

\paragraph{Stern-Gerlachs experiment}
Potentialen som uppkommer av ett externt magnetiskt fält ger upphov till en kraft
\begin{align*}
	\vb{F} &= -\grad{-\vb{\mu}\times\vb{B}}, \\
	F_{z}  &= \mu_{z}\del{z}{B}\vb{e}_{z} = -\frac{e\hbar}{2m_{e}}m_{l}\del{z}{B}\vb{e}_{z}.
\end{align*}

Det gjordes därför ett experiment där elektroner sköts genom ett magnetfält som ökade i styrka uppåt och fångades upp av en skärm. Baserad på resultaten från rörelsemängdsmomentet, skulle elektronerna fångas upp i ett udda antal olika punkter, men detta var inte fallet. Specifikt, då de förväntade att alla elektroner skulle hamna i samma punkt ($l = 0$), hamnade de i stället i två olika punkter. Detta måste uppkomma från spinnet.