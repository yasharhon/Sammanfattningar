\section{Allmän relativitetsteori}

\paragraph{Gravitationell tidsdilatation}
Betrakta två observatorer i ett tyngdfält med (svag) uniform fältstyrka $g$ som befinner sig på olika positioner med en höjdskillnad $h$. Då upplever observatören högre upp i fältet en tidsdilatation
\begin{align*}
	\Delta t' = (1 + \frac{gh}{c^{2}})\Delta t.
\end{align*}
Om man har ett icke-uniformt tyngdfält, kan man i stället använda att för små $\Delta h$ gäller att
\begin{align*}
	\frac{\Delta t(h + Delta{h}}{\Delta t(h)}       &= (1 + \frac{g(h)}{c^{2}}\Delta{h}),
	\prod\frac{\Delta t(h_{i}}{\Delta t(h_{i - 1})} &= \frac{\Delta t(h)}{\Delta t(h_{0})} &= (1 + \sum\Delta{h}\frac{g(h_{i})}{c^{2}})
\end{align*}
med $h_{i} = h_{0} + i\Delta{h}$, och i gränsen $\Delta h\to 0$
\begin{align*}
	\frac{\Delta t(h)}{\Delta t(h_{0})} = 1 + \inteval{h_{0}}{h}{x}{\frac{g(x)}{c^{2}}}.
\end{align*}