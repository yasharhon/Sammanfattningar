\section{Partikelsystem}
Ett partikelsystem är en samling av $N$ partiklar med (konstanta) massor $m_i$ och total massa $m$ som samverkar. Varje partikel påverkas av yttre krafter med summa $\vb{F}_{i}$ samt inre krafter $\vb{f}_{ij}$ med alla andra partikler i systemet.

Vi antar att alla inre krafter verkar parallellt med linjen mellan partiklerna. Newtons andra lag ger $\vb{f}_{ij} = -\vb{f}_{ji}$, vilket även implicerar $\vb{f}_{ii}= \vb{0}$.

Vi definierar kraftsummorna
\begin{align*}
	\vb{F}     &= \sum \vb{F}_{i}, \\
	\vb{f}     &= \sum\limits_{i}\sum\limits_{j}\vb{f}_{ij}.
\end{align*}
Vi får
\begin{align*}
	\vb{f} = \sum\limits_{i}\sum\limits_{j}\vb{f}_{ij} = \sum\limits_{j}\sum\limits_{i}\vb{f}_{ij} = -\sum\limits_{j}\sum\limits_{i}\vb{f}_{ji} = -\vb{f},
\end{align*}
och därmed $\vb{f} = \vb{0}$.

\paragraph{Masscentrum}
Vi kommer ihåg att masscentrum för ett partikelsystem definieras som
\begin{align*}
	\vb{r}_{G} = \frac{1}{\sum m_i}\sum m_i\vb{r}_i.
\end{align*}

\paragraph{Rörelsemängd}
Systemets totala rörelsemängd ges av
\begin{align*}
	\vb{p} = \sum m_i\vb{v}_{i} = \dv{t}\left(\sum m_i\vb{r}_{i}\right) = \dv{m\vb{r}_{G}}{t} = m\vb{v}_{G}.
\end{align*}

\paragraph{Kraftekvationen för ett partikelsystem}
Kraftekvationen för en enda partikel ger
\begin{align*}
	m_{i}\dv[2]{\vb{r}_{i}}{t} = \vb{F}_{i} + \sum\limits_{j}\vb{f}_{ij}.
\end{align*}
Om vi adderar alla dessa ekvationer, får man
\begin{align*}
	\sum m_{i}\dv[2]{\vb{r}_{i}}{t} = \sum \vb{F}_{i} + \sum\limits_{i}\sum\limits_{j}\vb{f}_{ij}, \\
	\dv[2]{t}\left(\sum m_{i}\vb{r}_{i}\right) = \vb{F} + \vb{f}, \\
	\dv{t}\left(m\vb{v}_{G}\right) = \dv{\vb{p}}{t} = \vb{F},
\end{align*}
vilket är kraftekvationen som vi känner den. Med konstant massa kan detta även skrivas som
\begin{align*}
	m\vb{a}_{G} = \vb{F}.
\end{align*}