\section{Analytisk mekanik}
Betrakta ett system av $n$ partiklar. Dessa beskrivs av $3n$ koordinater i $3$ dimensioner. Systemet kan även begränsas av
\begin{itemize}
	\item holonoma tvång på formen $f(\vb{r}_{1}, \dots, \vb{r}_{n}, t) = 0.$
	\item icke-holonoma tvång, som ej kan uttryckas som holonoma tvång och endast ger samband mellan differentialerna.
\end{itemize}
Vi kommer endast betrakta konservativa system.

\paragraph{Frihetsgrader}
Systemets antal frihetsgrader är det minsta antalet parametrar som entydigt bestämmer systemets läge.

Om systemet begränsas av $k$ holonoma tvång, har systemet $s = 3n - k$ frihetsgrader.

\paragraph{Generaliserade koordinater}
Systemets generaliserade koordinater är ett val av parametrar som beskriver systemets läge. Dessa betecknas som $q_{i}$. Det finns lika många av dessa som antal frihetsgrader i systemet.

Med dessa inför vi även tidsderivatorna $\dot{q}_{i}$.

\paragraph{Virtuell förflyttning}
Vi introducerar nu den virtuella förflyttningen $\delta\vb{r}$. Denna är förenlig med villkoren (som ecentuellt är frysta i tid) och i övrigt godtycklig.

\paragraph{Variation av generaliserad koordinat}
Vi introducerar variationen av en generaliserad koordinat som
\begin{align*}
	\delta q_{i} = \varepsilon\eta_{i}(t),
\end{align*}
där $\eta$ är $0$ i ändpunkterna.

\paragraph{Konfigurationsrum}
Konfigurationsrummet är ett rum vars koordinater är de generaliserade koordinaterna. Rörelse i detta rummet representerar alltså ändring av systemets tillstånd.

\paragraph{Hastighet}
Vi kan nu skriva hastigheten för en given partikel som
\begin{align*}
	\vb{v}_{k} = \pdv{\vb{r}_{k}}{t} + \sum\limits_{i = 1}^{s}\pdv{\vb{r}_{k}}{q_{i}}\dot{q}_{i}.
\end{align*}

\paragraph{Lagrangefunktionen}
Vi inför nu Lagrangefunktionen
\begin{align*}
	L = T - V.
\end{align*}

\paragraph{Verkningsfunktionalen}
Vi inför även verkningsfunktionalen
\begin{align*}
	S = \int\limits_{t_{1}}^{t_{2}}\dd{t}L.
\end{align*}

\paragraph{Hamiltons variationsprincip}
Hamiltons variationsprincip kan behandlas som en naturlag, och representerar ett alternativt sätt till Newtons lagar att beräkna partikelsystemets beteende på. Variationsprincipen säjer att $S$ har ett extremum längs den verkliga banan i konfigurationsrummet. Med variationskalkylen kan detta skrivas som
\begin{align*}
	\delta S = 0.
\end{align*}

\paragraph{Lagranges ekvationer}
Vi tillämpar nu vår kunnskap om variationskalkyl på verkningsintegralen och får
\begin{align*}
	\delta S = \int\limits_{t_{1}}^{t_{2}}\dd{t}\sum\limits_{i = 1}^{k}\pdv{L}{q_{i}}\Delta q_{i} + \pdv{L}{\dot{q}_{i}}\Delta \dot{q}_{i} = 0.
\end{align*}
Den sista familjen av termer ges av
\begin{align*}
	\int\limits_{t_{1}}^{t_{2}}\dd{t}\pdv{L}{\dot{q}_{i}}\delta \dot{q}_{i} = -\int\limits_{t_{1}}^{t_{2}}\dd{t}\dv{t}\left(\pdv{L}{\dot{q}_{i}}\right)\delta q_{i}
\end{align*}
då variationen är $0$ i ändpunkterna. Då kan variationsprincipen skrivas som
\begin{align*}
	\int\limits_{t_{1}}^{t_{2}}\dd{t}\sum\limits_{i = 1}^{k}\Delta q_{i}\left(\pdv{L}{q_{i}} - \dv{t}\left(\pdv{L}{\dot{q}_{i}}\right)\right) = 0.
\end{align*}
Om detta skall gälla för alla val av tidsintervall, ger detta Lagranges ekvationer
\begin{align*}
	\pdv{L}{q_{i}} - \dv{t}\left(\pdv{L}{\dot{q}_{i}}\right) = 0.
\end{align*}

\paragraph{Generaliserade rörelsemängder}
Vi inför nu generaliserade rörelsemängder
\begin{align*}
	p_{i} = \pdv{L}{\dot{q}_{i}}.
\end{align*}

\paragraph{Hamiltonfunktionen}
Lagranges ekvationer är ett system av andra ordningens differentialekvationer. För att försöka att reducera det till ett system av första ordningens ekvationer, inför vi Hamiltonfunktionen
\begin{align*}
	H = \sum p_{i}\dot{q}_{i} - L.
\end{align*}

\paragraph{Hamiltons ekvationer}
Variationen av Hamiltonfunktionen ges av
\begin{align*}
	\delta H &= \sum p_{i}\delta\dot{q}_{i} + \dot{q}_{i}\delta p_{i} - \pdv{L}{q_{i}}\delta q_{i} - \pdv{L}{\dot{q}_{i}}\delta\dot{q}_{i} \\
	         &= \sum p_{i}\delta\dot{q}_{i} + \dot{q}_{i}\delta p_{i} - \pdv{L}{q_{i}}\delta q_{i} - p_{i}\delta\dot{q}_{i} \\
	         &= \sum\dot{q}_{i}\delta p_{i} - \pdv{L}{q_{i}}\delta q_{i}.
\end{align*}
Med Lagranges ekvationer kan vi skriva
\begin{align*}
	\pdv{L}{q_{i}} = \dot{p}_{i},
\end{align*}
och Hamiltonfunktionens variation ges av
\begin{align*}
	\delta H = \sum\dot{q}_{i}\delta p_{i} - \dot{p}_{i}\delta q_{i}.
\end{align*}

Om vi jämför detta med variationen som fås vid att betrakta Hamiltonfunktionen som en funktion av de generaliserade koordinaterna och rörelsemängdsmomenterna, fås
\begin{align*}
	\delta H = \sum\pdv{H}{q_{i}}\delta q_{i} + \pdv{H}{p_{i}}\delta p_{i}.
\end{align*}
Detta ger Hamiltons ekvationer:
\begin{align*}
	\dot{q}_{i} = \pdv{H}{p_{i}}, \dot{p}_{i} = \pdv{H}{q_{i}},
\end{align*}
som är ett system med $2s$ ekvationer, där $s$ är antal frihetsgrader, av första ordningens differentialekvationer.

\paragraph{Koppling till energien}
Vi har att
\begin{align*}
	\sum p_{i}\dot{q}_{i} = \sum\pdv{L}{\dot{q}_{i}}\dot{q}_{i} = \sum\pdv{T}{\dot{q}_{i}}\dot{q}_{i} = 2T,
\end{align*}
där den sista likheten kommer av Eulers sats om homogena funktioner och att den kinetiska energin är en homogen funktion av ordning $2$. Detta ger
\begin{align*}
	H = 2T - L = T + V = E,
\end{align*}
och är alltså lika med den totala energin.