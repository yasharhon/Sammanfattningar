\section{Stela kroppar}
En stel kropp är en massbelagd domän så att avståndet mellan två godtyckliga punkter är konstant.

\subsection{Kinematik}
En stel kropp kan ha translationshastighet eller rotationshastighet. Translationshastighet karakteriseras av att $\vb{v}_{A} = \vb{v}_{B}$ för alla $A, B$. Rotationshastighet karakteriseras av att det finns ett $C$ som är stelt förenad med kroppen så att $\vb{v}_{C} = \vb{0}$ momentant. 

För att beskriva rörelsen till en stel kropp, bilda en referensram med axlerna fixa relativt kroppen. betrakta två punkter $A, B$ i kroppen, där origo i den nya referensramen är $A$. Då gäller det att
\begin{align*}
	\vb{v}_{B} = \vb{v}_{B, \text{sp}} + \vb{v}_{B, \text{rel}}.
\end{align*}
Eftersom axlerna är fixa relativt kroppen, ger andra termen inget bidrag, vilket ger
\begin{align*}
	\vb{v}_{B} = \vb{v}_{A} + \vb*{\omega}\times\vb{r}_{AB}
\end{align*}
och bekräftar vårt påstående om att all rörelse för en stel kropp är antingen translation eller rotation.

Betrakta vidare kroppens acceleration, som ges av
\begin{align*}
	\vb{a}_{B} = \vb{a}_{B, \text{sp}} + \vb{a}_{B, \text{cor}} + \vb{a}_{B, \text{rel}}.
\end{align*}
Fixa axler relativt kroppen ger att de två sista termerna ej bidrar och
\begin{align*}
	\vb{a}_{B} = \vb{a}_{A} + \dv{\vb*{\omega}}{t}\times\vb{r}_{AB} + \vb*{\omega}\times(\vb*{\omega}\times\vb{r}_{AB}),
\end{align*}
där den första termen är ett translatoriskt bidrag och de två andra är rotationsbidrag.

\paragraph{Plan rörelse}
Plan rörelse för en stel kropp karakteriseras av att hastigheten i alla punkter är parallellt med ett och samma fixa plan. Om rörelsen är i $xy$-planet, kommer $\vb*{\omega}$ peka längs med $z$-axeln.

Om en stel kropp roterar under plan rörelse, finns det alltid en punkt $C$ med $\vb{v}_{C} = \vb{0}$, som kallas momentancentrum. Denna punkt uppfyller $\vb{v}_{A} = -\vb*{\omega}\times\vb{r}_{AC}$. För att hitta den, multiplicera med $\vb*{\omega}$ på båda sidor för att få
\begin{align*}
	\vb*{\omega}\times\vb{v}_{A} &= -\vb*{\omega}\times(\vb*{\omega}\times\vb{r}_{AC}) \\
	                             &= -(\vb*{\omega}\cdot\vb{r}_{AC})\vb*{\omega} + \omega^2\vb{r}_{AC}.
\end{align*}
Eftersom rörelsen är plan, behöver vi bara betrakta ett snitt av kroppen i rörelsesplanet, vilket gör att den första skalärprodukten blir $0$. Detta ger då positionen till momentancentrumet enligt
\begin{align*}
	\vb{r}_{AC} = \frac{1}{\omega^2}\vb*{\omega}\times\vb{v}_{A}.
\end{align*}

Vi studerar vidare accelerationssambandet för plan rörelse, som ger
\begin{align*}
	\vb{a}_{B} = \vb{a}_{A} + \dv{\vb*{\omega}}{t}\times\vb{r}_{AB} + \vb*{\omega}\times(\vb*{\omega}\times\vb{r}_{AB}).
\end{align*}
Vi skriver ut termerna och får
\begin{align*}
	\vb{a}_{B} = \vb{a}_{A} + \alpha\vu{e}_{z}\times\vb{r}_{AB} + (\vb*{\omega}\cdot\vb{r}_{AB})\vb*{\omega} - \omega^2\vb{r}_{AB}.
\end{align*}
Eftersom rörelsen är plan, blir skalärprodukten $0$, och man får slutligen
\begin{align*}
	\vb{a}_{B} = \vb{a}_{A} + \alpha\vu{e}_{z}\times\vb{r}_{AB} - \omega^2\vb{r}_{AB}.
\end{align*}

\subsection{Dynamik}

\paragraph{Energilagen}
Definiera effekten
\begin{align*}
	P_{ij} = \vb{f}_{ij}\cdot\vb{v}_{i} + \vb{f}_{ji}\cdot\vb{v}_{j} = \vb{f}_{ij}\cdot(\vb{v}_{i} - \vb{v}_{j}).
\end{align*}
Vi använder att $\vb{v}_{j} = \vb{v}_{i} + \vb*{\omega}\times(\vb{r}_{j} - \vb{v}_{i})$ och får
\begin{align*}
	P_{ij} = -\vb{f}_{ij}\cdot\vb*{\omega}\times(\vb{r}_{j} - \vb{v}_{i}) = 0
\end{align*}
eftersom $\vb{f}_{ij}$ verkar längs linjen mellan partikel $i$ och $j$. Därmed gör de indre krafterna inget arbete, och
\begin{align*}
	U_{0 - 1}^{\text{(e)}} = T_{1} - T_{0}.
\end{align*}

\paragraph{Rotation kring fix axel}
Låt punkten $O$ vara på rotationsaxeln och välg cylindriska koordinater så att kroppen roterar kring $z$-axeln. För någon partikel i den stela kroppen har man
\begin{align*}
	\vb{v}_{i} = \vb{v}_{O} + \vb*{\omega}\times\vb{r}_{i} = \rho_{i}\omega\vu{e}_{\theta}.
\end{align*}
Den kinetiska energin ges nu av
\begin{align*}
	T = \frac{1}{2}\omega^2\sum m_{k}\rho_{i}^2
\end{align*}
och rörelsemängdsmomentet kring $O$ ges av
\begin{align*}
	\vb{H}_{O} &= \sum \vb{r}_{i}\times m_{i}\rho_{i}\omega\vu{e}_{\theta} \\
	           &= \sum m_{i}\rho_{i}^2\omega\vu{e}_{z} - \sum m_{i}\rho_{i}z_{i}\omega\vu{e}_{r}.
\end{align*}
Vi inför nu tröghetsmomentet kring $z$-axeln
\begin{align*}
	I_{z} = \sum m_{k}\rho_{i}^2.
\end{align*}
Då gäller att
\begin{align*}
	T     &= \frac{1}{2}I_{z}\omega^2, \\
	H_{z} &= I_{z}\omega.
\end{align*}

Kraftekvationens komponenter ger
\begin{align*}
	F_{r}      &= -ml\left(\dv{\theta}{t}\right)^2, \\
	F_{\theta} &= ml\dv[2]{\theta}{t}.
\end{align*}

Momentekvationen ger
\begin{align*}
	\dv{t}\left(I_{z}\dv{\theta}{t}\right) = M_{z}.
\end{align*}

Arbetet som görs på kroppen ges av
\begin{align*}
	\dd{U_{i}} = \vb{F}_{i}\cdot\dd{\vb{r}_{i}} = F_{i, \theta}\rho_{i}\omega\dd{t} = M_{i, z}\dd{\theta}.
\end{align*}