\section{Kvantmekanik i tre dimensioner}

\paragraph{Separabla problem}
Betrakta ett problem med en Hamiltonoperator som kan skrivas som $H = \sum H_{i}$, där varje $H_{i}$ endast verkar på $x_{i}$-koordinaten. Då kan Schrödingerekvationen skrivas som
\begin{align*}
	\sum H_{i}\Psi = i\hbar\del{t}{\Psi}.
\end{align*}
Vid att ansätta
\begin{align*}
	\Psi(\vb{x}) = \prod \Psi_{i}(x_{i})
\end{align*}
kan Schrödingerekvationen separeras till $d$ problem
\begin{align*}
	H_{i}\Psi_{i} = i\hbar\del{t}{\Psi_{i}}.
\end{align*}
Den totala lösningen fås vid att multiplicera alla lösningar, och den totala energin fås vid att summera upp energin för varje tillstånd. Om varje lösning har energin $E_{i, n_{i}}$, är den totala energin $E_{n_{1}\dots n_{d}} = \sum E_{i, n_{i}}$.

\paragraph{Sfäriskt symmetriska problem}
Betrakta ett enpartikelproblem med en sfäriskt symmetrisk potential, kan den tidsoberoende Schrödingerekvationen
\begin{align*}
	\left(-\frac{\hbar^{2}}{2m}\left(\frac{1}{r^{2}}\pdv{r}\left(r^{2}\pdv{r}\right) + \frac{1}{r^{2}\sin(\theta)}\pdv{\theta}\left(\sin(\theta)\pdv{\theta}\right) + \frac{1}{r^{2}\sin^{2}(\theta)}\pdv[2]{\phi}\right) + V(r)\right)\Psi = E\Psi
\end{align*}
separeras. Separera ut $r$-beroendet i funktionen $R$ och vinkelberoendet i funktionen $Y$. De två sidorna är då lika med konstanten som vi döper $l(l + 1)$. Detta ger följande ekvationer:
\begin{align*}
	\dv{r}\left(r^{2}\dv{R}{r}\right) -\frac{2mr^{2}}{\hbar^{2}}(V(r) - E)R &= l(l + 1)R, \\
	- \left(\frac{1}{\sin(\theta)}\pdv{\theta}\left(\sin(\theta)\pdv{Y}{\theta}\right) + \frac{1}{\sin^{2}(\theta)}\pdv[2]{Y}{\phi}\right)                           &= l(l + 1)Y.
\end{align*}

Vi kan även separera vinkelberoendet, och de två sidorna blir lika med en konstant $m^{2}$. Detta ger två ekvationer:
\begin{align*}
	\sin(\theta)\dv{\theta}\left(\sin(\theta)\dv{\Theta}{\theta}\right) + l(l + 1)\sin[2](\theta)\Theta          &= m^{2}\Theta, \\
	-\dv[2]{\Phi}{\phi} &= m^{2}\Phi.
\end{align*}

Första ekvationen har lösning
\begin{align*}
	\Phi = e^{im\phi}.
\end{align*}
Denna lösningen måste vara $2\pi$-periodisk, vilket implicerar att $m$ är ett heltal. Lösningarna är ortogonala under inreprodukten
\begin{align*}
	\braket{f}{g} = \integ{0}{2\pi}{\phi}{\cc{f}g}.
\end{align*}

När vi nu känner $m$, skriver vi om $\theta$-ekvationen som
\begin{align*}
	-\frac{1}{\sin(\theta)}\left(\dv{\theta}\left(\sin(\theta)\dv{\Theta}{\theta}\right) - \frac{m^{2}}{\sin{\theta}}\Theta\right) &= l(l + 1)\Theta.
\end{align*}
Efter lite mer algebra fås lösningen
\begin{align*}
	P_{l}^{m}(x) = (1 - x^{2})^{\frac{\abs{m}}{2}}\dv[\abs{m}]{P_{l}}{x},
\end{align*}
där $x = \cos{\theta}$ och $P_{l}$ är Legendrepolynomen av grad $l$. Legendrepolynomens gradtal ger att $\abs{m}\leq l$. Lösningarna är ortogonala under inreprodukten
\begin{align*}
	\braket{f}{g} = \integ{0}{\pi}{\theta}{\sin(\theta)\cc{f}g}.
\end{align*}
Produkten av dessa två blir klotytefunktionerna $Y_{l}^{m}$.

Kvar står den radiella ekvationen
\begin{align*}
	\dv{r}\left(r^{2}\dv{R}{r}\right) - \frac{2mr^{2}}{\hbar^{2}}(V(r) - E)R &= l(l + 1)R,
\end{align*}
som vi kan skriva om till
\begin{align*}
	\frac{1}{r^{2}}\dv{r}\left(r^{2}\dv{R}{r}\right) - \frac{2m}{\hbar^{2}}(V(r) - E)R &= \frac{l(l + 1)}{r^{2}}R, \\
	-\frac{1}{r^{2}}\left(\dv{r}\left(r^{2}\dv{R}{r}\right) + \frac{2mr^{2}}{\hbar^{2}}\left(V(r) + \frac{\hbar^{2}l(l + 1)}{2m}\right)R\right) &= \frac{2mE}{\hbar^{2}}R.
\end{align*}
Detta är ett Sturm-Liouville-problem, och vi vet därmed att det finns oändligt många lösningar för positiva energier. Lösningarna är ortogonala under inreprodukten
\begin{align*}
	\braket{f}{g} = \integ{0}{\infty}{r}{r^{2}\cc{f}g}.
\end{align*}
Den totala inreprodukten av lösningarna är nu
\begin{align*}
	\braket{f}{g} &= \integ{0}{\infty}{r}{r^{2}\integ{0}{\pi}{\theta}{\sin(\theta)\integ{0}{2\pi}{\phi}{\cc{f}g}}} \\
	              &= \integ{0}{\infty}{r}{\integ{0}{\pi}{\theta}{\integ{0}{2\pi}{\phi}{r^{2}\sin(\theta)\cc{f}g}}},
\end{align*}
vilket vi förväntade. Normeringsvillkoret är nu $\braket{\Psi} = 1$.

Vi inför substitutionen $u = Rr$ på den ursprungliga formen av ekvationen, vilket efter lite algebra ger
\begin{align*}
	-\frac{\hbar^{2}}{2m}\dv[2]{u}{r} + \left(V(r) + \frac{\hbar^{2}l(l + 1)}{2mr^{2}}\right)u = Eu.
\end{align*}
Detta är en endimensionell Schrödingerekvation där vi har lagt till en centrifugalbarriär $\frac{\hbar^{2}l(l + 1)}{2mr^{2}}$ till potentialen. Med denna substitutionen kan normeringsvillkoret skrivas som
\begin{align*}
	\integ{0}{\infty}{r}{\integ{0}{\pi}{\theta}{\integ{0}{2\pi}{\phi}{\sin(\theta)\abs{uY_{l}^{m}}^{2}}}} = 1.
\end{align*}

\paragraph{Väteatomen}
Väteatomen består av en proton och en neutron. Protonen är ungefär $2000$ gånger tyngre än elektronen, vilket betyder att atomens masscentrum till god approximation ligger i protonens centrum. Vid att lösa problemet i atomens masscentrum, fås protonen till att vara statisk och elektronen att röra sig i en potential $V(r) = -\frac{e^{2}}{4\pi\varepsilon_{0}r}$. Den radiella ekvationen blir då
\begin{align*}
	-\frac{\hbar^{2}}{2m}\dv[2]{u}{r} + \left(-\frac{e^{2}}{4\pi\varepsilon_{0}r} + \frac{\hbar^{2}l(l + 1)}{2mr^{2}}\right)u = Eu.
\end{align*}
Vi söker först bundna tillstånd, med negativ energi. Inför $E = -\frac{\hbar^{2}k^{2}}{2m},\ \rho = kr$ och $\rho_{0} = \frac{me^{2}}{2\pi\varepsilon_{0}\hbar^{2}k}$. Då fås
\begin{align*}
	\dv[2]{u}{\rho} = \left(1 - \frac{\rho_{0}}{\rho} + \frac{l(l + 1)}{\rho^{2}}\right)u.
\end{align*}
Vi vill lösa detta med en potensserieansats. För stora $\rho$ är de två sista termerna försumbara, och vi får i gränsen $u = Ae^{-\rho}$. För små $\rho$ dominerar sista termen, och vi får gränsen $u = C\rho^{l + 1}$. Vi gör därför ansatsen $u = v(\rho)e^{-\rho}$. Insatt i ekvationen ger detta
\begin{align*}
	\dv[2]{v}{\rho} - 2\dv{v}{\rho} + \frac{\rho_{0}}{\rho}v - \frac{l(l + 1)}{\rho^{2}}v = 0.
\end{align*}
Vi löser detta med potensserieansatsen
\begin{align*}
	v = \sum\limits_{i = 0}^{\infty}c_{i}\rho^{i + l + 1}.
\end{align*}
Vi får rekursionsformeln
\begin{align*}
	c_{i + 1} = \frac{2(j + l + 1) - \rho_{0}}{(j + 1)(j + 2l + 2)}c_{j}.
\end{align*}
Rekursionen måste terminera, då man annars skulle få en lösning som beter sig som $e^{\rho}$ för stora $\rho$. Detta ger villkor för värdena av $\rho$, alltså $k$, alltså $E$. Mer specifikt, låt $n$ vara det största värdet av $j + l + 1$. Detta ger $\rho_{0} = 2n$. Ur detta fås
\begin{align*}
	k = \frac{me^{2}}{4\pi\varepsilon_{0}\hbar^{2}n}
\end{align*}
och slutligen
\begin{align*}
	E_{n} &= -\frac{\hbar^{2}}{2m}\left(\frac{me^{2}}{4\pi\varepsilon_{0}\hbar^{2}n}\right)^{2} \\
	      &= -\frac{me^{4}}{2\hbar^{2}(4\pi\varepsilon_{0})^{2}n^{2}}.
\end{align*}
Den karakteristiska längdskalan i lösningen är Bohr-radien $a$ och ges av
\begin{align*}
	a = \frac{4\pi\varepsilon_{0}\hbar^{2}}{me^{2}}.
\end{align*}
De stationära tillstånden är
\begin{align*}
	\psi_{nlm} = A_{nl}e^{-\frac{r}{na}}\left(\frac{2r}{na}\right)^{l}L_{n - l - 1}^{2l + 1}\left(\frac{2r}{na}\right)Y_{l}^{m}(\theta, \phi),
\end{align*}
där $L$ är Laguerrepolynomen.

Vi ser att för ett givet $n$ kan $l$, vara olikt - mer specifikt är $l = 0, 1, \dots, n - 1$ och de olika egentillstånden är då degenererade med degenerationsgrad
\begin{align*}
	\sum\limits_{l = 0}^{n - 1}2l + 1 = n^{2}.
\end{align*}

\paragraph{Rörelsemängdsmoment}
Klassiskt har vi för rörelsemängdsmomentet för en enda partikel att
\begin{align*}
	\dot{\vb{L}} = \vb{r}\times\vb{F}.
\end{align*}
Speciellt, om det inte finns några yttre kraftmoment, är rörelsemängdsmomentet konstant. Vid att beskriva systemet i sfäriska koordinater och uttrycka hastigheten tangentiellt på ytan med konstant $r$ i termer av rörelsemängdsmomentet, kan Hamiltonianen skrivas som
\begin{align*}
	H = \frac{1}{2}mv_{r}^{2} + \frac{L^{2}}{2mr^{2}} + V.
\end{align*}

När vi gör övergången till kvantmekanik, får vi följande operator:
\begin{align*}
	\vb{L} = \vb{r}\times\vb{p}.
\end{align*}
För att få dens belopp, kan vi titta på Hamiltonoperatorn i sfäriska koordinater och jämföra termvis för att få en operator. Vi får
\begin{align*}
	L^{2} = -\hbar^{2}\left(\frac{1}{\sin(\theta)}\pdv{\theta}\left(\sin(\theta)\pdv{\theta}\right) + \frac{1}{\sin^{2}(\theta)}\pdv[2]{\phi}\right).
\end{align*}
Det finns även argument för detta som görs med hjälp av kryssprodukten, som jag kanske borde lägga till. På detta sättet kan vi även få
\begin{align*}
	L_{z} = \frac{\hbar}{i}\pdv{\phi}.
\end{align*}

Det visar sig att
\begin{align*}
	L^{2}Y_{l}^{m} = \hbar^{2}l(l + 1)Y_{l}^{m},\ L_{z}Y_{l}^{m} = \hbar mY_{l}^{m}.
\end{align*}
Eftersom de ingående operatorerna konstruerades från Hermiteska operatorer, är även dessa Hermiteska, och vi får direkt att klotytefunktionerna är ortogonala med inreprodukten som gavs ovan.

Mellan de olika komponenterna av rörelsemängdsmomentet finns följande kommutationsrelationer:
\begin{align*}
	\commut{L_{x}, L_{y}} &= \commut{yp_{z} - zp_{y}}{zp_{x} - xp_{z}} \\
	                      &= \commut{yp_{z}}{zp_{x}} - \commut{yp_{z}}{xp_{z}} - \commut{zp_{y}}{zp_{x}} + \commut{zp_{y}}{xp_{z}} \\
	                      &= y\commut{p_{z}}{zp_{x}} + x\commut{zp_{y}}{p_{z}} \\
	                      &= y(z\commut{p_{z}}{p_{x}} + \commut{p_{z}}{p_{x}}z) + x(p_{y}\commut{z}{p_{z}} + \commut{z}{p_{z}}p_{y}) \\
	                      &= y(\commut{p_{z}}{z})p_{x} + x(\commut{z}{p_{z}})p_{y} \\
	                      &= y(-i\hbar)p_{x} + x(i\hbar)p_{y} \\
	                      &= i\hbar L_{z}.
\end{align*}
På samma sätt fås
\begin{align*}
	\commut{L_{z}}{L_{x}} = i\hbar L_{y}, \\
	\commut{L_{y}}{L_{z}} = i\hbar L_{x}, \\
	\commut{L^{2}}{\vb{L}} = \vb{0}.
\end{align*}
Vi ser att det inte är något speciellt med någon komponent, så vi väljer att konstruera de gemensamme egenfunktionerna till $L^{2}$ och $L_{z}$.

För att konstruera dessa egenfunktionerna, definierar vi stegoperatorerna
\begin{align*}
	L_{+} = L_{x} + iL_{y},\ L_{-} = \adj{L_{+}} = L_{x} - iL_{y}.
\end{align*}
Dessa är inte Hermiteska. Kommutationsrelationerna är
\begin{align*}
	\commut{L^{2}}{L_{\pm}} &= 0, \\
	\commut{L_{z}}{L_{\pm}} &= \pm\hbar L_{z}.
\end{align*}
Produkten av de två operatorerna är
\begin{align*}
	L_{+}L_{-} = L_{x}^{2} + L_{y}^{2} - i\commut{L_{x}}{L_{y}} = L^{2} - L_{z}^{2} + \hbar L_{z}, \\
	L_{-}L_{+} = L^{2} - L_{z}^{2} - \hbar L_{z}.
\end{align*}
Egenfunktionerna uppfyller
\begin{align*}
	L^{2}L_{+}\ket{\Psi} &= L_{+}L^{2}\ket{\Psi} = \hbar^{2}l(l + 1)L_{+}\ket{\Psi}, \\
	L_{z}L_{+}\ket{\Psi} &= L_{+}L_{z}\ket{\Psi} + \hbar L_{+}\ket{\Psi} = \hbar(m + 1)L_{+}\ket{\Psi}, \\
	L^{2}L_{-}\ket{\Psi} &= L_{-}L^{2}\ket{\Psi} = \hbar^{2}l(l + 1)L_{-}\ket{\Psi}, \\
	L_{z}L_{-}\ket{\Psi} &= L_{-}L_{z}\ket{\Psi} - \hbar L_{-}\ket{\Psi} = \hbar(m - 1)L_{-}\ket{\Psi}.
\end{align*}
Vi ser alltså att stegoperatorerna skapar nya egentillstånd med olika väntevärden för $L_{z}$. Däremot vet vi att väntevärdena av $L^{2}$ och $L_{z}^{2}$ är strikt positiva, och $L^{2} \leq L_{z}^{2}$, så följden måste terminera någon gång. Med andra ord finns det två tillstånd så att
\begin{align*}
	L_{+}\ket{m_{\text{max}}} = 0,\ L_{-}\ket{m_{\text{min}}} = 0.
\end{align*}

Vi får vidare
\begin{align*}
	L^{2}\ket{m_{\text{max}}} &= (L_{-}L_{+} + L_{z}^{2} + \hbar L_{z})\ket{m_{\text{max}}} \\
	                          &= (0 + m^{2}\hbar^{2} + \hbar^{2}m_{\text{max}})\ket{m_{\text{max}}},
\end{align*}
vilket implicerar
\begin{align*}
	l(l + 1) = m^{2}\hbar^{2} + \hbar^{2}m_{\text{max}},
\end{align*}
med lösning
\begin{align*}
	m_{\text{max}} = l.
\end{align*}
På samma sätt fås
\begin{align*}
	m_{\text{min}} = -l.
\end{align*}
Vi noterar nu att det finns $2l$ steg mellan det maximala och minimala värdet. Eftersom antal steg även måste vara ett heltal, så kan $l$ vara alla multipler av $\frac{1}{2}$.

För att få normering, betraktar vi
\begin{align*}
	L_{\pm}\ket{m} = A_{\pm}\ket{m\pm 1}.
\end{align*}
Dens inreprodukt med sig själv är
\begin{align*}
	\braket{L_{-}L_{+}}{m} = \abs{A_{+}}^{2}\braket{m}
\end{align*}
å ena sidan och
\begin{align*}
	\braket{L^{2} - L_{z}^{2} - \hbar L_{z}}{m} = \hbar^{2}(l(l + 1) - m^{2} - m)\braket{m},
\end{align*}
vilket ger
\begin{align*}
	A_{+} = \sqrt{l(l + 1)- m(m + 1)}\hbar
\end{align*}
och på samma sätt
\begin{align*}
	A_{+} = \sqrt{l(l + 1)- m(m - 1)}\hbar.
\end{align*}

Osäkerhetsrelationen ger
\begin{align*}
	\Delta L_{x}\Delta L_{y} \geq \frac{1}{2}\abs{\expval{\commut{L_{x}}{L_{y}}}} = \frac{1}{2}\hbar\abs{\expval{L_{z}}} = \frac{1}{2}\abs{m}\hbar^{2}.
\end{align*}
Alltså är inte tillstånden vi har hittat egenfunktioner till $L_{x}$ eller $L_{y}$ om inte $m = 0$.

\paragraph{Partiklar med spinn i magnetfält}
En laddning $q$ med den givna spinnen får det magnetiska momentet $\vb{\mu} = \gamma\vb{S}$, där $\gamma = \frac{gq}{2m}$ och $g$ är en gyroskopisk faktor. För elektroner är den ungefär $2$ och för protoner är den ungefär $5$.

Välj nu koordinatsystem så att $\vb{B}$ pekar i $z$-riktningen. Hamiltonianen ges då av
\begin{align*}
	H = -\vb{\mu}\cdot\vb{B} = -\frac{gqB}{2m}S_{z} = \omega_{0}S_{z} = \frac{1}{2}\hbar\omega_{0}
	\mqty[
		1 & 0 \\
		0 & -1
	].
\end{align*}
Dens egenvärden är $E_{m}\hbar\omega_{0}$.

\paragraph{Schrödingerekvatinen för en partikel i ett magnetfält}
Den tidsberoende Schrödingerekvationen för en partikel i ett magnetfält blir
\begin{align*}
	i\hbar\dv{\chi}{t} = H\chi.
\end{align*}
En allmän spinor för en $s = \frac{1}{2}$-partikel ges då av
\begin{align*}
	\chi = a\chi_{+}e^{-i\frac{E_{+}t}{\hbar}} + b\chi_{-}e^{-i\frac{E_{-}t}{\hbar}}.
\end{align*}
Normeringsvillkoret för såna tillstånd uppmanar oss att skriva $a = \cos{\frac{\theta}{2}}$ och $b = \sin{\frac{\theta}{2}}$. Vi får med detta
\begin{align*}
	\expval{S_{x}} &= \frac{1}{2}\hbar\sin{\theta}\cos{\omega_{0}t}, \\
	\expval{S_{y}} &= \frac{1}{2}\hbar\sin{\theta}\sin{\omega_{0}t}, \\
	\expval{S_{z}} &= \frac{1}{2}\hbar\cos{\theta}.
\end{align*}
Alltså precesserar spinnet kring $z$-axeln med frekvens $\omega_{0}$ i ett plan som ges av initialtillståndet.