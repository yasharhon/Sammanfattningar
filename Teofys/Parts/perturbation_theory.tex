\section{Störningsteori}

\paragraph{Störningsräkningarnas principer}
Vi vill lösa problemet $H\Psi_{n} = E_{n}\Psi_{n}$. Detta görs på följande sätt:
\begin{itemize}
	\item Antag att vi har ett löst problem $H^{0}\Psi_{n}^{0} = E_{n}^{0}\Psi_{n}^{0}$.
	\item Inför den störda hamiltonianen $H = H^{0} + V$, vars egenfunktioner vi vill hitta.
	\item Inför styrkeparametern $\lambda$ genom $H = H^{0} + \lambda V$.
	\item Sök en potensserielösning $\Psi = \Psi_{n}^{0} + \lambda\Psi_{n}^{1} + \lambda^{2}\Psi_{n}^{2}+ \dots$. Denna har energi $E_{n} = E_{n}^{0} + \lambda E_{n}^{1} + \lambda^{2}E_{n}^{2} + \dots$. De olika termerna är koorektioner av olik ordning, och superskriptet indikerar vad ordningen är för varje term.
\end{itemize}

Störningarna av olik ordning fås vid att identifiera termer från Schrödingerekvationen
\begin{align*}
	(H^{0} + \lambda V)(\Psi_{n}^{0} + \lambda\Psi_{n}^{1} + \lambda^{2}\Psi_{n}^{2}+ \dots) = (E_{n}^{0} + \lambda E_{n}^{1} + \lambda^{2}E_{n}^{2} + \dots)(\Psi_{n}^{0} + \lambda\Psi_{n}^{1} + \lambda^{2}\Psi_{n}^{2}+ \dots).
\end{align*}
Till exempel ges första ordningens störning av
\begin{align*}
	H^{0}\Psi_{n}^{1} + V\Psi_{n}^{0} = E_{n}^{0}\Psi_{n}^{1} + E_{n}^{1}\Psi_{n}^{0}
\end{align*}
Inreprodukten med $\Psi_{n}^{0}$ på båda sidor ger
\begin{align*}
	\mel{\Psi_{n}^{0}}{H^{0}}{\Psi_{n}^{1}} + \mel{\Psi_{n}^{0}}{V}{\Psi_{n}^{0}} = E_{n}^{0}\braket{\Psi_{n}^{0}}{\Psi_{n}^{1}} + E_{n}^{1}\braket{\Psi_{n}^{0}}{\Psi_{n}^{0}}.
\end{align*}
Eftersom hamiltonoperatorn är Hermitesk, kommer första term på båda sidor vara lika, vilket ger
\begin{align*}
	 E_{n}^{1} = \mel{\Psi_{n}^{0}}{V}{\Psi_{n}^{0}}.
\end{align*}
Vi kan även visa att
\begin{align*}
	\Psi_{n}^{1} = \sum\limits_{m \neq n}\frac{\mel{\Psi_{m}^{0}}{V}{\Psi_{n}^{0}}}{E_{n}^{0} - E_{m}^{0}}\Psi_{m}^{0},
\end{align*}
som ökar i amplitud med $V$. Det gäller även att
\begin{align*}
	E_{n}^{2} = \sum\limits_{m \neq n}\frac{\abs{\mel{\Psi_{m}^{0}}{V}{\Psi_{n}^{0}}}^{2}}{E_{n}^{0} - E_{m}^{0}},
\end{align*}
som ökar i styrka med $V^{2}$. Observera att nämnaren i dessa uttryck gör det helt klart att denna sortens räkning endast gäller för icke-degenererade energier.

\paragraph{Degenererade energier}
För att få en indikation på hur det går till, betrakta ett tvåfalt degenererad egentillstånd med två egentillstånd $\Psi_{a}^{0}$ och $\Psi_{b}^{0}$. Dessa kan väljas ortogonala. För att få störningen av första ordningen, gör vi ansatsen $\Psi = \alpha\Psi_{a}^{0} + \beta\Psi_{b}^{0} + \Psi^{1}$. Schrödingerekvationen ger enligt samma argument som ovan att
\begin{align*}
	H^{0}\Psi^{1} + V(\alpha\Psi_{a}^{0} + \beta\Psi_{b}^{0}) = E^{0}\Psi^{1} + E^{1}(\alpha\Psi_{a}^{0} + \beta\Psi_{b}^{0}).
\end{align*}
Inreprodukten med $\Psi_{a}^{0}$ ger igen
\begin{align*}
	\mel{\Psi_{a}^{0}}{H^{0}}{\Psi^{1}} + \alpha\mel{\Psi_{a}^{0}}{V}{\Psi_{a}^{0}} + \beta\mel{\Psi_{a}^{0}}{V}{\Psi_{b}^{0}} = E_{n}^{0}\braket{\Psi_{a}^{0}}{\Psi^{1}} + E^{0}(\alpha\braket{\Psi_{a}^{0}}{\Psi_{a}^{0}} + \beta\braket{\Psi_{a}^{0}}{\Psi_{b}^{0}}),
\end{align*}
där första termen på varje sida igen är lika. Vid att definiera $W_{ab} = \mel{\Psi_{a}^{0}}{V}{\Psi_{a}^{0}}$ och göra motsvarande fast för $\Psi_{b}^{0}$ fås ekvationssystemet
\begin{align*}
	\alpha W_{aa} + \beta W_{ab} = E^{1}\alpha,\ \alpha W_{ba} + \beta W_{bb} = E^{1}\beta,
\end{align*}
som är en egenvärdesekvation i energin. Vid att välja basen på ett sånt sätt att $W_{ab} = W_{ba} = 0$ kan de möjliga egenvärdena läsas av direkt.

\paragraph{Störningar i väte}
Vi vill introducera störningar i väteatomens energier för att se vad som händer med energierna. Vid att jämföra detta med experiment kan man testa kvantmekanikens giltighet.

\paragraph{Störning med elektriskt fält}
Vi lägger först på ett elektriskt fält i $z$-riktning, med en potential på formen $V = e\mathcal{E}z_{\text{p}} - e\mathcal{E}z_{\text{e}}$ till totala systemet. Vi betraktar väteatomen i masscentrumssystemet, och detta kan då skrivas om i masscentrumssystemet som $V = e\mathcal{E}z$. Alternativt kan vi införa dipolmomentet $\vb{p} = -e\vb{r}_{e} + e\vb{r}_{p} = -e\vb{r}$, och skriva $V = -\vb{p}\cdot\vb{E}$. Första ordningens energikorrektion till grundtillståndet blir
\begin{align*}
	E_{100}^{1} = \expval{e\mathcal{E}z}{100} = 0,
\end{align*}
då grundtillståndet är sfäriskt symmetriskt.

Andra ordningens korrektion blir
\begin{align*}
	E_{100}^{2} = e^{2}\mathcal{E}^{2}\sum\limits_{nlm \neq 100}\frac{\abs{\mel{nlm}{z}{100}}^{2}}{E_{1}^{0} - E_{n}^{0}} = -\frac{9}{4}4\pi\varepsilon_{0}a^{3}\mathcal{E}^{2}.
\end{align*}
Att göra denna beräkningen analytiskt kan göras, men vi vill i stället göra en uppskattning vid att låta alla $E_{n}^{0}$ vara $E_{2}^{0}$ för $n > 2$. Vi kan då ta ut energifaktorn från summationen ovan och få
\begin{align*}
	E_{100}^{2} = \frac{e^{2}\mathcal{E}^{2}}{E_{1}^{0} - E_{2}^{0}}\sum\limits_{nlm \neq 100}\abs{\mel{nlm}{z}{100}}^{2}.
\end{align*}
Enligt argumentet från första ordningen kan vi utvidga summan till
\begin{align*}
	E_{100}^{2} = \frac{e^{2}\mathcal{E}^{2}}{E_{1}^{0} - E_{2}^{0}}\sum\abs{\mel{nlm}{z}{100}}^{2} = \frac{e^{2}\mathcal{E}^{2}}{E_{1}^{0} - E_{2}^{0}}\sum\mel{100}{z}{nlm}\mel{nlm}{z}{100}.
\end{align*}
Detta kan förenklas med hjälp av identitetsoperatorn till
\begin{align*}
	E_{100}^{2} = \frac{e^{2}\mathcal{E}^{2}}{E_{1}^{0} - E_{2}^{0}}\expval{z^{2}}{100} = -\frac{8}{3}4\pi\varepsilon_{0}a^{3}\mathcal{E}^{2}.
\end{align*}

De andra energitillstånden är degenererade med degenerationsgrad $2n^{2}$. Vi vill nu välja en god bas så att störningsmatrisen blir diagonal. Då blir diagonalelementen i störningsmatrisen korrektionerna av första ordning. Vi kan utnyttja att om $A$ är Hermitesk med icke-degenererade egenvärden och störningen $V$ kommuterar med både $A$ och $H^{0}$, är $V$ diagonal i egenbasen till $A$.

\paragraph{Relativistisk korrektion}
Den relativistiska kinetiska energin ges av
\begin{align*}
	\sqrt{p^{2}c^{2} + (mc^{2})^{2}} - mc^{2},
\end{align*}
med en första korrektionsterm
\begin{align*}
	V = -\frac{p^{4}}{8m^{3}c^{2}}.
\end{align*}
Denna kommuterar med den klassiska hamiltonoperatorn, och vi kan därför använda icke-degenererad störningsräkning. Man kan visa att
\begin{align*}
	E_{n}^{1} = -\frac{(E_{n}^{0})^{2}}{2mc^{2}}\left(\frac{4n}{l + \frac{1}{2}} - 3\right),
\end{align*}
och
\begin{align*}
	\frac{E_{n}^{1}}{E_{n}^{0}}\propto E_{n}^{1}\propto\alpha^{2},
\end{align*}
där vi har infört finstrukturkonstanten
\begin{align*}
	\alpha = \frac{e^{2}}{4\pi\varepsilon_{0}\hbar c}.
\end{align*}

\paragraph{Spinn-bankoppling}
Vi betraktar nu väteatomen i elektronens vilosystem. Då snurrar protonen kring elektronen, vilket ger upphov till en ström, som igen skapar ett magnetfält
\begin{align*}
	\vb{B} = \mu_{0}\frac{I}{2r}\vb{e}_{z} = \frac{e}{4\pi\varepsilon_{0}mc^{2}r^{3}}\vb{L}.
\end{align*}
Detta ger då en energikorrektion
\begin{align*}
	V = \frac{e^{2}}{4\pi\varepsilon_{0}m^{2}c^{2}r^{3}}\vb{L}\cdot\vb{S}.
\end{align*}
Störningarna som beräknades från detta var fel med en faktor $2$. Detta kom av att beräkningarna inte tog med i betraktningen att elektronens vilosystem inte är ett inertialsystem. Det faktiska energibidraget skall vara
\begin{align*}
	V = \frac{e^{2}}{8\pi\varepsilon_{0}m^{2}c^{2}r^{3}}\vb{L}\cdot\vb{S}.
\end{align*}
Vi får
\begin{align*}
	E_{n}^{1} &= \frac{e ^{2}}{8\pi\varepsilon_{0}}\frac{1}{m^{2}c^{2}}\frac{1}{2}\expval{J^{2} - L^{2} - S^{2}}\expval{\frac{1}{r^{3}}} \\
	          &= \frac{(E_{n}^{0})^{2}}{mc^{2}}\frac{n(j(j + 1) - l(l + 1) - \frac{3}{4})}{l(l + \frac{1}{2})(l + 1)},
\end{align*}
där vi har användt att alla ingående operatorer opererar på olika koordinater, varför vi kan splitta upp väntevärdena på detta sättet.

\paragraph{Hyperfinstruktur}
Koppling mellan elektronens och protonens magnetiska moment ger en korrektionsterm
\begin{align*}
	E_{100}^{1} = \frac{\mu_{0}g_{p}e^{2}}{3m_{e}m_{p}}\expval{\vb{S}_{p}\cdot\vb{S}_{e}}\abs{\eval{\Psi_{100}}_{\vb{r} = \vb{0}}}^{2}.
\end{align*}
Detta ger en energisplittring lika med energiskillnaden mellan en spinntriplett och en spinnsinglett.

\paragraph{Zeeman-effekt}
Zeeman-effekten är energiskift hos elektroner i ett homogent magnetiskt fält. Störningstermen är
\begin{align*}
	V = -(\vb{\mu}_{l} + \vb{\mu}_{s})\cdot\vb{B} = \frac{e}{2m}\vb{B}\cdot(\vb{L} + 2\vb{S}) = \frac{e}{2m}\vb{B}\cdot(\vb{J} + \vb{S}).
\end{align*}

\subparagraph{Svaga fält}
Här görs endast korrektion av finstrukturenergier. Eftersom $S_{z}$ är osäker i den kopplade basen, fås dock en icke-diagonal störningsmatris. Tidsmedelvärdet av $\vb{S}$ kan dock användas. Man får
\begin{align*}
	E^{1} = \frac{e}{2m}B\left(1 + \frac{j(j + 1) + s(s + 1) - l(l + 1)}{2j(j + 1)}\right)m_{j}\hbar.
\end{align*}

\subparagraph{Starka fält}
För starka fält är Zeeman-effekten dominant, och man kan göra räkningen i produktbasen. Man får
\begin{align*}
	E^{1} = \mu_{B}B(m_{l} + 2m_{s}).
\end{align*}

\subparagraph{Mellanstarka fält}
I detta fallet finns helt enkelt ingen bra bas. Det är dock inte mycket nytt som händer här.

\paragraph{Tidsberoende störningar}
För att göra tidsberoende störningar, utgår vi från ett känt ostört problem $H^{0}\ket{n} = E_{n}\ket{n}$. Vi vill nu titta på en störning $H = H^{0} + \lambda V(t)$. Om systemet startar i tillstånd $\ket{a}$, får vi
\begin{align*}
	\ket{\Psi} = c_{n}\ket{n},
\end{align*}
där
\begin{align*}
	c_{n} = \frac{1}{i\hbar}\integ{0}{t}{t'}{e^{i\omega_{0}t'}\mel{n}{V(t')}{a}},\ n \neq a,\ \omega_{0} = \frac{E_{n} - E_{a}}{\hbar}.
\end{align*}
Vi kan även beräkna övergångssannolikheten
\begin{align*}
	p_{a\to n} = \frac{1}{\hbar^{2}}\abs{\integ{0}{t}{t'}{e^{i\omega_{0}t'}\mel{n}{V(t')}{a}}}^{2}.
\end{align*}

\paragraph{Oscillerande störning}
Ett viktigt fall är en oscillerande störning
\begin{align*}
	V = V(\vb{r})\cos{\omega t}.
\end{align*}
De nya koefficienterna ges av
\begin{align*}
	c_{n} &= \frac{1}{2i\hbar}\mel{n}{V(\vb{r})}{a}\integ{0}{t}{t'}{e^{i\omega_{0}t'}(e^{i\omega t'} + e^{-i\omega t'})} \\
	      &= \frac{V_{ba}}{i\hbar}\left(e^{i\frac{\omega + \omega_{0}}{2}}\frac{\sin{(\omega + \omega_{0})t}}{\omega_{0} + \omega} + e^{i\frac{\omega_{0} - \omega}{2}}\frac{\sin{(\omega_{0} - \omega)t}}{\omega_{0} - \omega}\right).
\end{align*}
Denna har resonans för $\omega = \pm\omega_{0}$, vilket motsvarar en övergång $a\to n$ eller $n\to a$.

\paragraph{Emission och absorption av strålning}
Elektroner växelverkar i huvudsak med det elektriska fältet i en elektromagnetisk våg. Betrakta därför ett infallande monokromatiskt elektriskt fält polariserat i $z$-riktningen. Det ger upphov till en potential
\begin{align*}
	V = eE_{0}z\cos{\omega t}.
\end{align*}
Vi får
\begin{align*}
	V_{ba} = eE_{0}\mel{b}{z}{a} = -PE_{0},
\end{align*}
där $P$ är polarisationens matriselement mellan $a$ och $b$.

\paragraph{Inkoherent strålning}
Antag att man har en frekvensfördelning $\rho$ så att den termiska energin ges av
\begin{align*}
	\dd{U} = \rho\dd{\omega}.
\end{align*}
Inkoherent strålning är strålning där det inte finns interferens mellan olika moder. För såna fall kan man beräkna en övergångshastighet
\begin{align*}
	R_{ab} = \frac{\pi}{3\varepsilon_{0}\hbar^{2}}\abs{P}^{2}\rho^{2}.
\end{align*}
Detta är Fermis gyllene regel. Koefficienten
\begin{align*}
	\frac{\pi}{3\varepsilon_{0}\hbar^{2}}\abs{P}^{2}
\end{align*}
är Einsteins $B$-koefficient.

\paragraph{•}
Antag att vi har $N_{a}$ atomer i $a$ och $N_{b}$ atomer i $b$. Vi vill gärna veta hastigheten för spontan emission. En detaljerad balans ger
\begin{align*}
	\dv{N_{b}}{t} = -N_{a}A - N_{b}B_{ba}\rho(\omega_{0}) + N_{a}B_{ab}\rho(\omega_{0}).
\end{align*}
Maxwell-Boltzmann-statistik ger
\begin{align*}
	\frac{N_{a}}{N_{b}} = e^{\beta\hbar\omega},
\end{align*}
vilket implicerar
\begin{align*}
	\rho(\omega_{0}) = \frac{A}{e^{\beta\hbar\omega}B_{ab} - B_{ba}}.
\end{align*}
Detta skall vara lika med Plancks fördelning
\begin{align*}
	\frac{\hbar}{\pi^{2}c^{3}}\frac{\omega^{3}}{e^{\beta\hbar\omega - 1}},
\end{align*}
vilket implicerar
\begin{align*}
	A = \frac{\omega_{0}^{3}\abs{P}^{2}}{3\pi\varepsilon_{0}\hbar c^{3}}.
\end{align*}