\section{Grunderna i kvantmekanik}

\paragraph{Newtonsk mekanik}
I newtonsk mekanik beskrivs en partikel i ett intertialsystem av
\begin{align*}
	\vb{F} = m\vb{a},
\end{align*}
där $\vb{F}$ är summan av alla krafter på partikeln. Med givna initialvillkor kan partikelns bana beskrivas exakt.

Om partikeln endast påverkas av konservativa krafter, är dens energi konstant. För en sådan partikel är dens energi alltid större än potentialens minimum.

\paragraph{Analytisk mekanik}
Alternativt till newtonsk mekanik kan man formulera mekaniken med Lagranges och Hamiltons formaliser. Dessa utgår från att beskriva partikelns bana med generaliserade koordinater $q_{i}$ och hastigheter $\dot{q}_{i}$ (totala tidsderivator av koordinaterna), och hitta en bana som minimerar verkansintegralen
\begin{align*}
	S = \integ{}{}{t}{L}
\end{align*}
där $L = T - V$ och $T$ är kinetiska energin. Dessa leder till ekvationer på formen
\begin{align*}
	\del{q_{i}}{L} - \dv{t}\del{\dot{q}_{i}}{L} = 0.
\end{align*}
Detta är Lagranges formalism.

Alternativt kan man införa generaliserade rörelsemängder $p_{i} = \del{\dot{q}_{i}}{L}$ och Hamiltonfunktionen $H = \dot{q}_{i}p_{i} - L = T + V$. Denna transformationen ger dig
\begin{align*}
	\dot{q}_{i} = \del{p_{i}}{H},\ \dot{p}_{i} = -\del{q_{i}}{H}.
\end{align*}
Detta är Hamiltons formalism.

Om systemet är symmetriskt i $q_{i}$-riktningen, beror $H$ ej av denna koordinaten, vilket ger att $\dot{p}_{i} = 0$ och $p_{i}$ är konstant. En sådan koordinat kallas för en cyklisk koordinat. Detta är ett specialfall av Nöethers sats, som säjer att till varje symmetri hör en konserverad storhet.

En fråga är om man kan göra ett variabelbyte så att alla variabler är cykliska. För att diskutera detta, måste vi prata om kanoniska transformationer. En kanonisk transformation är ett variabelbyte som bevarar formen på Hamiltons ekvationer. Efter ett sådant variabelbyte fås en ny Hamiltonfunktion
\begin{align*}
	H(Q, P, t) = H(q, p, t) + \del{t}{S}(q, P, t).
\end{align*}
Vi vill nu gärna veta om denna nya storheten $S$ är verkan. Dens totala tidsderivata ges av
\begin{align*}
	\dot{S} = \del{q_{i}}{S}\dot{q}_{i} + \del{P_{i}}{S}\dot{P}_{i} + \del{t}{S}.
\end{align*}
Om detta är en transformation så att alla koordinater är cykliska, fås
\begin{align*}
	\dot{S} = \del{q_{i}}{S}\dot{q}_{i} + \del{t}{S}.
\end{align*}
Vi noterar enligt ovan att $\del{t}{S} = - H(q, p, t)$. Om nu $S$ skall vara verkan, måste denna totala tidsderivatan vara lika med Lagrangefunktionen. Definitionen av Hamiltonfunktionen ger då
\begin{align*}
	\del{q_{i}}{S} = p_{i},\ \del{P_{i}}{S} = Q_{i}.
\end{align*}
Vi får då Hamilton-Jacobis ekvation
\begin{align*}
	H(q, \del{p}{S}, t) + \del{t}{S} = 0.
\end{align*}

\paragraph{Behov för ny beskrivning}
I början av 1900-talet upptäcktes det via olika experiment att partiklar på små längdskalor visade beteende som ej samsvarade med klassiska teorir. Detta skapade ett behov för en ny teori.

\paragraph{Våg-partikel-dualitet}
Det första steget togs av Max Planck år 1900. I ett desperat försök på att beskriva svartkroppsstrålning antog han att elektromagnetisk strålning upptas och avges i diskreta kvanta av energin $\hbar\omega$, där $\hbar$ är en helt ny konstant. Försöket lyckades.

Senare förstod Einstein att ljus bestod av diskreta enheter med energi $E = pc$, där $p$ är enhetens rörelsemängd.

de Broglie kombinerade dessa två resultaten till sin hypotes: Att alla partiklar har vågegenskaper, och kan beskrivas av en våglängd som ges av
\begin{align*}
	p = \hbar k = \frac{h}{\lambda}.
\end{align*}
Vi har använt vågtalet $k = \frac{2\pi}{\lambda}$ och introducerat Plancks konstant $h = 2\pi\hbar$.

de Broglie antog även att alla partiklar kan beskrivas av en vågfunktion som ges av $\Psi = Ae^{i(kx - \omega t)}$ för en fri partikel. Vid att sätta in resultaten ovan kan denna vågfunktionen skrivas som
\begin{align*}
	\Psi = Ae^{i(\frac{p}{\hbar}x - \frac{E}{\hbar}t)}.
\end{align*}

\paragraph{Motivation av Schrödingerekvationen}
Givet resultaten ovan, noterar vi
\begin{align*}
	i\hbar\del{t}{\Psi} = E\Psi, -\frac{\hbar^{2}}{2m}\del{x}{\Psi} = \frac{p^{2}}{2m}\Psi.
\end{align*}
Vi kommer ihåg den klassiska Hamiltonfunktionen $H = T + V$. För en fri partikel ges denna av $H = \frac{p^{2}}{2m} + V$. Vi ser nu att de två derivationsoperatorerna ovan ersätter totala respektiva kinetiska energin, och Hamiltonsk mekanik inspirerar oss nu att skriva
\begin{align*}
	i\hbar\del{t}{\Psi} = -\frac{\hbar^{2}}{2m}\del{x}{\Psi} + V\Psi.
\end{align*}
Efter detta kommer Schrödingerekvationen att behandlas som en naturlag. I mer allmänna fall kommer vi även skriva detta som
\begin{align*}
	i\hbar\del{t}{\Psi} = H\Psi,
\end{align*}
där $H$ är en operator som representerar systemets Hamiltonfunktion.

\paragraph{Sannolikhet}
Den fysikaliska tolkningen av vågfunktionen är att $\abs{\Psi}^{2}\dd{V} = \cc{\Psi}\Psi\dd{V}$ är sannolikheten för att hitta partikeln i en mycket liten volym. Vi kräver då
\begin{align*}
	\integ[3]{}{}{x}{\abs{\Psi}^{2}} = 1.
\end{align*}
Med denna tolkningen refereras $\Psi$ till som sannolikhetsamplituden.

\paragraph{Operatorer}
Vi har redan sett att det finns en koppling mellan operatorer på vågfunktionen och fysikaliska storheter. Baserad på detta definierar vi väntevärdet av en observabel $q$ som
\begin{align*}
	\expval{q} = \integ[3]{}{}{x}{\cc{\Psi}\hat{q}\Psi}.
\end{align*}
Här är $\hat{q}$ operatorn som motsvarar $q$. Denna distinktionen kommer inte göras mycket vidare i sammanfattningen.

Vi kan tolka denna integralen som en inreprodukt, och med den tolkningen kräver vi att alla operatorer som representerar fysikaliska storheter är självadjungerade eller Hermiteska.

\paragraph{Osäkerhetsprincipen}
Bandbräddsteori ger oss för $x$ och $p$ att
\begin{align*}
	\Delta x\Delta p \geq \frac{1}{2}\hbar.
\end{align*}
Här är $\Delta x$ standardavvikelsen för $x$.

\paragraph{Kontinuitetsekvationen för sannolikhet}
Man kan visa att
\begin{align*}
	\del{t}{\abs{\Psi}^{2}} + \div{j} = 0,
\end{align*}
där $j$ är sannolikhetsströmmen
\begin{align*}
	j = -\frac{i\hbar}{2m}(\cc{\Psi}\grad{\Psi} - \Psi\grad{\cc{\Psi}}).
\end{align*}

\paragraph{Klassiskt förbjudna områden}
Områden där potentialen är större än ett tillstånds energi kallas klassiskt förbjudna områden. Klassiska system kan ej existera i såna områden, men vi kommer se att kvantmekaniska system har nollskild sannolikhet att finnas i klassiskt förbjudna områden.

\paragraph{Ehrenfests sats}
Man kan visa att
\begin{align*}
	\dv{\expval{x}}{t} = \frac{1}{m}\expval{p},\ \dv{\expval{p}}{t} = -\expval{\grad{V}}.
\end{align*}

\paragraph{Vägintegraler}
Även kvantmekaniken kan formuleras med hjälp av verkansintegralen. I denna formalismen ger sannolikheten för att en partikel väljer en bana mellan två punkter ges av
\begin{align*}
	\sum e^{i\frac{S}{\hbar}},
\end{align*}
där summationen görs över alla möjliga banor mellan de två punkterna.

\paragraph{Koppling till klassisk mekanik}
För att se kopplingen till klassisk fysik, är vi intresserade av att se vad som händer när $\hbar\to 0$. I detta fallet förväntar vi att de Broglie-våglängden blir liten, och att vågfunktionen därmed kommer oscillera mycket. Vi gör då ansatsen $\Psi = Ae^{i\frac{S}{\hbar}}$, där $S$ är någon funktion. Vi antar att $A$ varierar försumbart i rummet jämförd med $S$.

Kinetisk energi-operatorn tillämpad på detta ger
\begin{align*}
	-\frac{\hbar^{2}}{2m}\del[2]{x}{\Psi} &= -\frac{\hbar^{2}}{2m}\del{x}{\left(\del{x}{A}e^{i\frac{S}{\hbar}} + iA\frac{1}{\hbar}\del{x}{S}e^{i\frac{S}{\hbar}}\right)} \\
	                                      &=  -\frac{\hbar^{2}}{2m}\left(\del[2]{x}{A}e^{i\frac{S}{\hbar}} + i\del{x}{A}\frac{1}{\hbar}\del{x}{S}e^{i\frac{S}{\hbar}} + i\frac{1}{\hbar}\del{x}{A}\del{x}{S}e^{i\frac{S}{\hbar}} + iA\frac{1}{\hbar}\del[2]{x}{S}e^{i\frac{S}{\hbar}} - A\frac{1}{\hbar^{2}}(\del{x}{S})^{2}e^{i\frac{S}{\hbar}}\right).
\end{align*}
Observera att i gränsen kommer endast en term att överleva. Vi får även
\begin{align*}
	i\hbar\del{t}{\Psi} = i\hbar\left(\del{t}{A}e^{i\frac{S}{\hbar}} + iA\frac{1}{\hbar}\del{t}{S}e^{i\frac{S}{\hbar}}\right).
\end{align*}
Även här kommer endast en term överleva i gränsen. Insatt i Schrödingerekvationen och evaluerad i gränsen fås
\begin{align*}
	\left(\frac{1}{2m}(\del{x}{S})^{2} + V\right)\Psi = -\del{t}{S}\Psi,
\end{align*}
vilket implicerar
\begin{align*}
	\frac{1}{2m}(\del{x}{S})^{2} + V + \del{t}{S} = 0.
\end{align*}
De två första termerna motsvarar Hamiltonfunktionen i klassisk mekanik, där i har gjort transformationen $p\to\del{x}{S}$. Med denna tolkningen ser vi att detta motsvarar Hamilton-Jacobis ekvation.

\paragraph{Separation av Schrödingerekvationen}
Betrakta Schrödingerekvationen för en statisk potential. Då verkar Hamiltonoperatorn endast på rumliga koordinater. Vi gör då produktansatsen $\Psi = \psi\phi$, där $\psi$ endast beror av rumliga koordinater och $\phi$ endast av tiden. Detta ger
\begin{align*}
	H\Psi = \phi H\psi = i\hbar\del{t}{\Psi} = i\hbar \psi\dv{\phi}{t}.
\end{align*}
Vi kan nu dividera med $\Psi$ för att få
\begin{align*}
	\frac{1}{\psi}H\psi = \frac{i\hbar}{\phi}\dv{\phi}{t}.
\end{align*}
Eftersom varje sida beror av olika koordinater, måste de vara lika med en konstant. Av ingen anledning alls väljer vi att kalla denna konstanten för $E$. Vi löser först tidsdelen för att få
\begin{align*}
	\dv{\phi}{t} = -\frac{iE}{\hbar}\phi,
\end{align*}
med lösning
\begin{align*}
	\phi = Ce^{-i\frac{E}{\hbar}t}.
\end{align*}
Vi ser att $E$ måste motsvara systemets totala energi. Insatt i den andra ekvationen ger detta
\begin{align*}
	H\psi = E\psi.
\end{align*}
För stationära fall är alltså vågfunktionen ett egentillstånd till Hamiltonoperatorn.

Betrakta vidare väntevärdet av en observabel som representeras av en operator som ej verkar på tidskoordinaten. Detta ger
\begin{align*}
	\expval{q} &= \integ[3]{}{}{x}{\cc{\Psi}q\Psi} \\
	           &= \integ[3]{}{}{x}{\cc{\psi}\cc{phi}q(\psi\phi)} \\
	           &= \integ[3]{}{}{x}{\cc{\psi}\cc{phi}\phi q\psi} \\
	           &= \integ[3]{}{}{x}{\cc{\psi}\abs{phi}^{2} q\psi} \\
	           &= \integ[3]{}{}{x}{\cc{\psi} q\psi}.
\end{align*}
Alltså är väntevärdet fullständigt tidsoberoende. Speciellt gäller för energin att
\begin{align*}
	\expval{q} &= \integ[3]{}{}{x}{\cc{\Psi} H\Psi} \\
	           &= \integ[3]{}{}{x}{\cc{\psi} H\psi} \\
	           &= \integ[3]{}{}{x}{\cc{\psi} E\psi} \\
	           &= E\integ[3]{}{}{x}{\cc{\psi}\psi} \\
	           &= E.
\end{align*}
På samma sättet fås $\expval{E^{2}} = E^{2}$ och $\Delta E = \sqrt{\expval{E^{2}} - \expval{E}^{2}} = 0.$

Vi kan även skriva en allmän lösning som en superposition
\begin{align*}
	\Psi = \sum c_{n}\psi_{n}e^{-i\frac{E_{n}}{\hbar}t}
\end{align*}
där $H\psi_{n} = E_{n}\psi_{n}$ och summationen görs över hela systemets energispektrum.

\paragraph{Möjliga energier}
Om en partikel finns i någon potential, måste alla energiegenvärden vara större än potentialens minimum. För att visa detta, kan vi observera att
\begin{align*}
	\expval{p^{2}} = \integ{}{}{x}{\abs{p\Psi}^{2}} > 0
\end{align*}
och
\begin{align*}
	\expval{V} = \integ{}{}{x}{V\abs{\Psi}^{2}} > V_\text{min},
\end{align*}
och därmed måste energins väntevärde vara större än $V_\text{min}$.

\paragraph{Partikel i oändlig låda}
För att få en känsla för vilken sorts fysik som kommer ut av kvantmekaniken, betraktar vi en partikel i en oändlig lådpotential, dvs.
\begin{align*}
	V = 
	\begin{cases}
		0,      &0 < x < a, \\
		\infty, &\text{annars.}
	\end{cases}
\end{align*}
Detta motsvarar en låda med oändligt starka väggar.

Vi noterar först att Schrödingerekvationen ger
\begin{align*}
	\del[2]{x}{\Psi} = -\frac{2m(E - V)}{\hbar^{2}}\psi.
\end{align*}
Alltså är vågfunktionens krökning proportionell mot $\sqrt{E - V}$. Detta implicerar om att $E - V < 0$ avtar beloppet av vågfunktionen exponentiellt i detta området. Om detta gäller överallt, kan det ej existera lösningar för $E < V_{\text{min}}$. Vi kommer därför anta att detta är sant härifrån.

Vi observerar även att argumentet implicerar att eftersom potentialen är oändlig utanför lådan, måste vågfunktionen endast vara nollskild inuti lådan, och det återstår att lösa Schrödingerekvationen i denna regionen. Här fås.
\begin{align*}
	-\frac{\hbar^{2}}{2m}\del{x}{\del{x}{\psi}} &= E\psi, \\
	\del{x}{\del{x}{\psi}}                      &= -\frac{2mE}{\hbar^{2}}\psi.
\end{align*}
Vi definierar nu
\begin{align*}
	k^{2} = \frac{2mE}{\hbar^{2}}
\end{align*}
och får
\begin{align*}
	\del{x}{\del{x}{\psi}} = -k^{2}\psi.
\end{align*}
Detta har lösningar
\begin{align*}
	\psi = Ae^{ikx} + Be^{-ikx}.
\end{align*}

För att få mer information, behövs randvillkor. Vi kräver att vågfunktionen är kontinuerlig, vilket ger $\psi(0) = \psi(a) = 0$. Första randvillkoret ger
\begin{align*}
	A + B &= 0, \\
	\psi  &= A\sin{kx}.
\end{align*}
Observera att detta inte kan uppfyllas för $k = 0$, varför denna möjligheten kan försummas.

Andra randvillkoret ger
\begin{align*}
	ka = n\pi.
\end{align*}
Nu kan energin bestämmas enligt
\begin{align*}
	\frac{n^{2}\pi^{2}}{a^{2}} &= \frac{2mE}{\hbar^{2}}, \\
	E                          &= \frac{\hbar^{2}n^{2}\pi^{2}}{2ma^{2}}.
\end{align*}

Slutligen ger normaliseringsvillkoret
\begin{align*}
	\integ{0}{a}{x}{\abs{B}^{2}\sin^{2}{kx}} = 1.
\end{align*}
Integralen på vänstersidan är
\begin{align*}
	\abs{B}^{2}\integ{0}{a}{x}{\frac{1 - \cos{2kx}}{2}}.
\end{align*}
Den andra termen ger inget bidrag eftersom den har period $a$, och detta ger
\begin{align*}
	\abs{B}^{2}\frac{a}{2} &= 1, \\
	\abs{B}                &= \sqrt{\frac{2}{a}}.
\end{align*}
Observera att vi endast skriver absolutbeloppet eftersom $B$ kan innehålla en komplex fas utan att det ändrar fysiken.

\paragraph{Harmoniska oscillatorn}
Harmoniska oscillatorn är ett viktigt problem i kvantmekaniken.

Klassiskt dyker det i bland upp problem som involverar partiklar i en potential som är kvadratisk. Ett typiskt exempel är en partikel som är fast i en fjäder. I dessa sammanhang kommer vi skriva potentialen som $V = \frac{1}{2}m\omega^{2}x^{2}$. Vi vet att partikelns rörelse är periodisk med en viss amplitud. Dens totala energi ges av $E = \frac{1}{2}m\omega^{2}A^{2}$, där $A$ är dens amplitude.

Både klassiskt och kvantmekaniskt kommer vi Taylorutveckla olika potentialer och få
\begin{align*}
	V \approx V(0) + \deval{V}{x}{0}x + \frac{1}{2}\deval[2]{V}{x}{0}x^{2}.
\end{align*}
Vi antar att $x = 0$ motsvarar jämvikt, varför den andra termen måste vara $0$. Vidare är det konstanta bidraget ointressant, då det ej ger något bidrag till fysiken. Kvar står en harmonisk oscillatorterm vars styrka ges av potentialens krökning i jämviktsläget.