\section{Diracnotation}

\paragraph{Kvantmekaniska tillstånd och tillståndspostulatet}
Vi utvidgar nu konceptet vågfunktion till ett allmänt kvantmekaniskt tillstånd $\ket{\Psi}$. Detta är en så kallad ketvektor i ett Hilbertrum. Ett fundamentalt postulat i kvantmekaniken är att $\ket{\Psi}$ beskriver systemet fullständigt.

Till ketvektorerna hör även bravektorerna $\bra{\Psi}$, som bildar dualrummet till Hilbertrummet som alla ketvektorer finns i.

\paragraph{Inreprodukt och ortogonalitet}
Till dessa hör en inreprodukt
\begin{align*}
	\braket{\Psi}{\Phi} = \integ[3]{}{}{\vb{x}}{\cc{\Psi}\Phi}.
\end{align*}
Denna inreprodukten uppfyller $\braket{\Psi}{\Phi} = \cc{\braket{\Psi}{\Phi}}$.

Baserad på detta kan vi införa ortogonalitet och normerbarhet. Vi säjer att en vektor är normerad om $\norm{\Psi}^{2} = \braket{\Psi} = 1$. Vi kommer förutsätta att vektorerna vi ser är normerade. Vi säjer även att $\ket{\Psi_{1}}$ och $\ket{\Psi_{2}}$ är ortonormala om $\braket{\Psi_{i}}{\Psi_{j}} = \delta{ij}$.

\paragraph{Operatorer och operatorpostulatet}
Ett annat fundamentalt postulat i kvantmekaniken är att observabler representeras av Hermiteska operatorer med en fullständig mängd av egenvektorer. Och vad betyder detta?

Att representera en observabel med en operator innebär att det till varje klassiska observabel $A$ som kan skrivas som $A(x, p)$ tillhör en operator $\hat{A}(\hat{x}, \hat{p})$. Om det ej finns en klassisk motsvarighet till $A$, fås operatorn från experiment.

Med Diracnotation kan vi även skriva en operatorn som $A = \op{\Psi_{1}}{\Psi_{2}}$. Vi använder i bland begreppet c-nummer om $\braket{}$ och q-nummer om $\op{}{}$. Med detta kan vi även skriva $\expval{A} = \expval{A}{\Psi}$.

För att prata om Hermiteska operatorer, måste vi först prata om adjungerade operatorer. Den adjungerade operatorn $\adj{A}$ till $A$ definieras som den operatorn så att
\begin{align*}
	\integ[3]{}{}{\vb{x}}{\cc{\Psi}A\Phi} = \integ[3]{}{}{\vb{x}}{\cc{A\Psi}\Phi}.
\end{align*}
Med Diracnotation kan detta skrivas som $\mel{\Phi}{A}{\Psi} = \cc{\mel{\Psi}{\adj{A}}{\Phi}}$. En Hermitesk operator är en operator som uppfyller $\adj{A} = A$. Det visar sig att operatorer som $x, p$ och $H$ uppfyller detta, och det är ju trevligt.

Några viktiga räkneregler för adjungerade operatorer är:
\begin{itemize}
	\item $\adj{(c_{1}A + c_{2}B)} = \cc{c_{1}}\adj{A} + \cc{c_{2}}\adj{B}$.
	\item $\adj{(\adj{A})} = A$.
	\item $\adj{(AB)} = \adj{B}\adj{A}$.
	\item Normen i kvadrat av $A\ket{\Psi}$ ges av $\braket{\adj{A}A}{\Psi}$.
\end{itemize}

Det kommer visa sig att vi är intresserade av att studera egenvärdesproblem för självadjungerade operatorer. Därför vill vi veta lite om egenvärdena och egenfunktionerna till såna operatorer.

Betrakta först ett egentillstånd $\ket{\Psi_{n}}$ med egenvärde $a_{n}$. Vi får:
\begin{align*}
	\expval{A} = \expval{A}{\Psi_{n}} = \cc{(\expval{A}{\Psi_{n}})},
\end{align*}
vilket implicerar att $a_{n} = \cc{a_{n}}$, och vi drar slutsatsen att självadjungerade operatorer har reella egenvärden.

Betrakta vidare två olika egentillstånd. Vi får
\begin{align*}
	\mel{\Psi_{n}}{A}{\Psi_{m}} = a_{m}\braket{\Psi_{n}}{\Psi_{m}} = \cc{\mel{\Psi_{m}}{A}{\Psi_{n}}} = \cc{a_{n}}\cc{\braket{\Psi_{n}}{\Psi_{m}}} = \cc{a_{n}}\braket{\Psi_{n}}{\Psi_{m}},
\end{align*}
och drar slutsatsen att att $\braket{\Psi_{n}}{\Psi_{m}} = \delta_{nm}$. Observera att om samma egenvärde har flera egenfunktioner, kan man använda Gram-Schmidts metod för att få ortogonala egenfunktioner.

Att bevisa att dessa egenvektorerna är fullständinga är allmänt bara möjligt för ändligdimensionella vektorrum. Vi kommer baka detta in i postulatet, och postulera att operatorerna vi vill betrakta är konstruerade så att detta är sant.

\paragraph{Diracnotation och operatorer}
Med fullständighetspostulatet kan vi nu skriva ett allmänt tillstånd som $\ket{\Psi} = c_{i}\ket{\Psi_{i}}$, och vi ser även att $c_{i} = \braket{\Psi_{i}}{\Psi}$. Vilken bas har vi utvecklat $\ket{\Psi}$ i? Jo, det är godtyckligt. Oftast kommer det vara egenbasen till en operator.

För att en utgångspunkt, startar vi med identitetsoperatorn $I$. Med resonnemanget ovan kan vi skriva
\begin{align*}
	I\ket{\Psi} = \ket{\Psi} = \op{\Psi_{i}}{\Psi_{i}}\ket{\Psi},
\end{align*}
vilket implicerar $I = \op{\Psi_{i}}{\Psi_{i}}$.

Nu kan vi betrakta en godtycklig operator $Q$ och skriva
\begin{align*}
	Q = IQI = \op{\Psi_{i}}{\Psi_{i}}Q\op{\Psi_{j}}{\Psi_{j}}.
\end{align*}
Vi kan definiera $Q_{ij} = \mel{\Psi_{i}}{Q}{\Psi_{j}}$, och får då
\begin{align*}
	Q = Q_{ij}\op{\Psi_{i}}{\Psi_{j}}.
\end{align*}

\paragraph{Sannolikhetstolkning}
Med hjälp av detta kan vi skriva
\begin{align*}
	\braket{\Psi} = \ev{I}{\Psi} = \braket{\Psi}{\Psi_{i}}\braket{\Psi_{i}}{\Psi} = \cc{c_{i}}c_{i} = 1.
\end{align*}
Vi tolkar detta som att $\abs{c_{i}}^{2}$ är sannolikheten att systemet är i tillståndet $\ket{\Psi{i}}$.

Vi kan även skriva väntevärdet av en observabel
\begin{align*}
	\expval{A} = \ev{A}{\Psi} = \ev{AI}{\Psi} = \mel{\Psi}{A}{\Psi_{i}}\braket{\Psi_{i}}{\Psi} = a_{i}\braket{\Psi}{\Psi_{i}}\braket{\Psi_{i}}{\Psi} = a_{i}\abs{c_{i}}^{2}.
\end{align*}
Notera hur snyggt detta blir när vi använder operatorpostulatet. Detta resultatet tolkar vi som att väntevärdet ges av en summa av produkter av egenvärden och sannolikheten för att systemet är i det motsvarande egentillståndet.

\paragraph{Mätpostulatet}
Ett annat fundamentalt postulat i kvantmekaniken är följande:

Om ett system befinner sig i ett tillstånd $\ket{\Psi}$ ger en mätning av storheten $A$ något av egenvärdena $a_{i}$ till $A$ som resultat med sannolikhet $\braket{\Psi_{i}}{\Psi}$. Mätningen ändrar även systemets tillstånd från $\ket{\Psi}$ till $\ket{Psi_{i}}$, och vi säjer att tillståndet kollapsar.

\paragraph{Schrödingerekvationen}
Det sista postulatet i kvantmekaniken är att tillståndets tidsutvekling ges av
\begin{align*}
	H\ket{\Psi} = i\hbar\dv{t}\ket{\Psi}.
\end{align*}

\paragraph{Kommutatorer}
Vi definierar kommutatorn mellan två operatorer som
\begin{align*}
	\commut{A}{B} = AB - BA.
\end{align*}
Om denna är $0$ säjs $A$ och $B$ att kommutera. Om två operatorer kommuterar spelar det ingen roll i vilken ordning man mäter de motsvarande observablerna - man kommer få samma resultat.

Kommutatorn uppfyller följande relationer:
\begin{itemize}
	\item $\commut{A}{A} = 0$.
	\item $\commut{A}{B} = -\commut{B}{A}$.
	\item $\commut{AB}{C} = ABC - CAB = ABC - ACB + ACB - CAB = A\commut{B}{C} + \commut{A}{C}B$.
	\item $\commut{A}{BC} = ABC - BCA = ABC - BAC + BAC - BCA = B\commut{A}{C} + \commut{A}{B}C$.
\end{itemize}

\paragraph{Kvantmekanisk harmonisk oscillator}
Den kvantmekaniska harmoniska oscillatorn beskrivs av
\begin{align*}
	H = \frac{1}{2m}p^{2} + \frac{1}{2}m\omega^{2}x^{2}.
\end{align*}
vi vill skriva Hamiltonoperatorn i termer av nya operatorer, så kallade stegoperatorer. Den första är sänkningsoperatorn
\begin{align*}
	a = \frac{1}{\sqrt{2\hbar m\omega}}(ip + m\omega x)
\end{align*}
och höjningsoperatorn
\begin{align*}
	\adj{a} = \frac{1}{\sqrt{2\hbar m\omega}}(-ip + m\omega x).
\end{align*}
Vi noterar att $\adj{a} \neq a$, och därför representerar dessa inte i sig själv observabler. Däremot är
\begin{align*}
	x = \sqrt{\frac{\hbar}{2m\omega}}(\adj{a} + a),\ p = i\sqrt{\frac{\hbar m\omega}{2}}(\adj{a} - a).
\end{align*}
Kommutatorn mellan dessa är
\begin{align*}
	\commut{a}{\adj{a}} &= \frac{1}{2\hbar m\omega}\commut{ip + m\omega x}{-ip + m\omega x} \\
	                    &= \frac{1}{2\hbar m\omega}(\commut{p}{p} + m^{2}\omega^{2}\commut{x}{x} + im\omega\commut{p}{x} - im\omega\commut{x}{p}) \\
	                    &= \frac{1}{2\hbar m\omega}(im\omega(-i\hbar) - im\omega(i\hbar)) \\
	                    &= 1.
\end{align*}
Detta implicerar
\begin{align*}
	a\adj{a} = \adj{a}a + 1.
\end{align*}

Vi kan nu skriva Hamiltonoperatorn som
\begin{align*}
	H &= -\frac{1}{2m}\frac{\hbar m\omega}{2}(\adj{a} - a)^{2} + \frac{1}{2}m\omega^{2}\frac{\hbar}{2m\omega}(\adj{a} + a)^{2} \\
	  &= -\frac{\hbar\omega}{4}((\adj{a})^{2} + a^{2} - \adj{a}a - a\adj{a}) + \frac{\hbar\omega}{4}((\adj{a})^{2} + a^{2} + \adj{a}a + a\adj{a})^{2} \\
	  &= \frac{\hbar\omega}{2}(\adj{a}a + a\adj{a}) \\
	  &= \frac{\hbar\omega}{2}(2\adj{a}a + 1) \\
	  &= \hbar\omega\left(\adj{a}a + \frac{1}{2}\right).
\end{align*}
Vi kan definiera nummeroperatorn $N = \adj{a}a$, och då skriva $H = \hbar\omega\left(N + \frac{1}{2}\right)$. Nummeroperatorn uppfyller 
\begin{align*}
	\commut{N}{a}       &= \adj{a}\commut{a}{a} + \commut{\adj{a}}{a}a = -a, \\
	\commut{N}{\adj{a}} &= \adj{a}\commut{a}{\adj{a}} + \commut{a}{a}a = \adj{a}.
\end{align*}
Därmed uppfyller Hamiltonoperatorn (eftersom multipler av identiteten kommuterar med allt):
\begin{align*}
	\commut{H}{a}       &= -\hbar\omega a, \\
	\commut{H}{\adj{a}} &= \hbar\omega\adj{a}.
\end{align*}

Antag nu att vi känner en egenfunktion $\Psi$. Då får vi
\begin{align*}
	Ha\Psi = aH\Psi - \hbar\omega a\Psi = (E - \hbar\omega)a\Psi,
\end{align*}
och $a\Psi$ är en ny egenfunktion med egenvärde $E - \hbar\omega$. Vi får även
\begin{align*}
	H\adj{a}\Psi = \adj{a}H\Psi + \hbar\omega\adj{a}\Psi = (E + \hbar\omega)\adj{a}\Psi,
\end{align*}
och vi ser nu varför vi döpte operatorena som vi gjorde.

Vi vet samtidigt att vi kan inte fortsätta att sänka potentialen och hitta nya egenfuktioner för alla möjliga egenvärden. Därför måste det finnas ett $\Psi_{0}$ så att $\Psi_{0}$ är en egenfunktion med egenvärde $E_{0}$, men $a\Psi_{0}$ ej är en egenfunktion. Eftersom operatoralgebran funkar som den gör, är då den enda möjligheten att $a\Psi_{0} = 0$. Detta ger en ordinarie differentialekvation med lösning
\begin{align*}
	\Psi_{0} = Ae^{-\frac{m\omega x^{2}}{2\hbar}},\ A = \left(\frac{m\omega}{\pi\hbar}\right)^{\frac{1}{4}}.
\end{align*}
Vi ser vidare att grunntilståndsenergin ges av $E = \frac{1}{2}\hbar\omega$, eftersom $a\Psi_{0} = 0$ och $H = \hbar\omega\left(\adj{a}a + \frac{1}{2}\right)$. Vidare ges då det fullständiga energispektret av $E_{n} = \left(\frac{1}{2} + n\right)\hbar\omega, n = 0, 1, \dots$.

De nästa tillstånden kan fås enligt
\begin{align*}
	\Psi_{n} = A_{n}(\adj{a})^{n}\Psi_{0} \propto H_{n}e^{-\frac{m\omega x^{2}}{2\hbar}},
\end{align*}
där $H_{n}$ är Hermitepolynomen. Detta hade man också kunnat få med en potensserielösning, men detta är mycket snyggare.

Vi tittar lite på nummeroperatorn igen. Den är självadjungerad, så man skulle kunna tro att den representerar en observabel. Det visar sig att den gör det, ty om vi definierar $\ket{n}$ som egentillståndet till Hamiltonoperatorn med energi $E_{n}$, ger egenvärdesekvationen att $N\ket{n} = n\ket{n}$, och $N$ ger alltså ett mått på tillståndets energi.

Till sist kommer lite om normering. Vi antar att det förra tillståndet var normerad, och vill ha en konstant $A$ så att om $a\ket{n} = A\ket{n - 1}$, är även det nya tillståndet normerad. Vi får
\begin{align*}
	\abs{A}^{2}\braket{n - 1} = \braket{\adj{a}a}{n} = n\braket{n},
\end{align*}
där vi har utnyttjat att om $\ket{\Phi} = a\ket{\Psi}$ är $\braket{\Phi} = \braket{\adj{a}a}{\Psi}$. Eftersom de två tillstånden är normerade, måste $A = \sqrt{n}$. Alltså är $a\ket{n} = \sqrt{n}\ket{n - 1}$. På samma sätt fås $\adj{a}\ket{n} = \sqrt{n + 1}\ket{n + 1}$.

\paragraph{Tillstånd som vektorer}
Om vi utvecklar ett tillstånd i ortonormala basfunktioner, kan utvecklingskoefficienterna vara element i vektorer. Mer specifikt, om vi utvec Med denna formalismen kan till exempel inreprodukten skrivas som en skalärprodukt.

\paragraph{Operatorer som matriser}
Om man skriver tillstånden som vektorer, ser vi att operatorer blir matriser, ty vi kommer ihåg att identitetsoperatorn kunne skrivas som $I = \ket{m}\bra{m}$. Detta ger skriva
\begin{align*}
	A = IAI = \ket{m}\mel{m}{A}{n}\ket{n} = \ket{m}A_{mn}\ket{n},
\end{align*}
där de olika $A_{mn}$ är element i en matris som sedan kan operera på tillståndsvektorerna. Vi ser nu att en självadjungerad operator måste uppfylla
\begin{align*}
	A_{mn} = \mel{m}{A}{n} = \mel{n}{A}{m} = \cc{A_{mn}},
\end{align*}
och Hermiteska operatorer representeras av självadjungerade matriser. Sammansättningen av två operatorer ges av
\begin{align*}
	AB = IAIBI = \ket{m}\mel{m}{A}{k}\mel{k}{B}{n}\ket{n} = \ket{m}A_{mk}B_{kn}\ket{n},
\end{align*}
och motsvaras av matrismultiplikation.

Om man utvecklar tillstånden i egenfunktioner till $A$, ges matriselementen av
\begin{align*}
	A_{mn} = \mel{m}{A}{n} = a_{n}\braket{m}{n} = a_{n}\delta_{mn}.
\end{align*}

\paragraph{Diracnotation i diskreta och kontinuerliga fall}
I kontinuerliga fall ersätts summationer över egentillstånd med integraler. Med denna övergången kommer en övergång från att sannolikheten ges av $\abs{\braket{n}{\Psi}}^{2}$ till att sannolikheten för att hitta partikeln i ett litet intervall ges av $\dd{x}\abs{\braket{x}{\Psi}}^{2}$, där $\ket{x}$ är ett egentillstånd till positionsoperatorn, och vi kan alltså behandla vågfunktionen som en expansionskoefficient av det allmänna tillståndet $\ket{\Psi}$. Utifrån detta identifierar vi $\Psi = \braket{x}{\Psi}$. Identiteten blir
\begin{align*}
	I = \integ{}{}{x}{\ket{x}\bra{x}}.
\end{align*}

\paragraph{Ortonormering}
I diskreta fall är ortonormeringsvillkoret $\braket{m}{n} = \delta_{mn}$. I kontinuerliga fall vill vi använda Fouriertransformen för att hitta ett ortonormeringsvillkor.

Vi vet från teori om Fouriertransformen att
\begin{align*}
	\delta(x) &= \frac{1}{2\pi}\integ{-\infty}{\infty}{k}{e^{ikx}}, \\
	\delta(k) &= \frac{1}{2\pi}\integ{-\infty}{\infty}{x}{e^{-ikx}}.
\end{align*}
Betrakta nu två olika egentillstånd till rörelsemängdsoperatorn, dvs. plana vågor. Ortonormering av dessa ger
\begin{align*}
	\braket{p'}{p} &= \integ{}{}{x}{\braket{p'}{x}\braket{x}{p'}} \\
	               &= \abs{A}^{2}\integ{}{}{x}{e^{-i\frac{p'}{\hbar}x}e^{i\frac{p}{\hbar}x}} \\
	               &= \abs{A}^{2}\hbar\integ{}{}{y}{e^{i(p - p')y}} \\
	               &= \abs{A}^{2}2\pi\hbar\delta(p - p').
\end{align*}
Egenfunktionerna är inte fullständigt specifierade än, så vid att välja $A = \frac{1}{\sqrt{2\pi\hbar}}$ fås ett ortonormeringsvillkor $\braket{p'}{p} = \delta(p - p')$, vilket är ganska analogt med ortonormeringsvillkoret i diskreta fall.

\paragraph{Positionsegenfunktioner}
Betrakta nu positionsoperatorn i rörelsemängdsbasen. Egenfunktionsvillkoret ger
\begin{align*}
	\hat{x}\braket{p}{x} = x\braket{p}{x} = x\cc{braket{x}{p}},
\end{align*}
vilket implicerar $\hat{x} = -\frac{\hbar}{i}\del{x}{}$ (detta kommer även fram från Fouriertransformens egenskaper). Deltafunktionsnormering, evt. Fouriertransformering, ger $\braket{x}{y} = \delta(x - y)$.

\paragraph{Fullständighet för operatorer}
Två Hermiteska operatorer har en fullständig mängd gemensamma egenfunktioner om och endast om och endast om $\commut{A}{B} = 0$. Det finns ett bevis för detta i Mats' anteckningar.

\paragraph{Utvidgad osäkerhetsprincip}
För två självadjungerade operatorer $A$ och $B$ är
\begin{align*}
	\Delta A\Delta B = \frac{1}{2}\abs{\expval{\commut{A}{B}}}.
\end{align*}

För att visa detta, bilda operatorerna $a = A - \expval{A}$ och $b = B - \expval{B}$. Vi får
\begin{align*}
	\expval{a^{2}} = \expval{A^{2} - 2\expval{A}A + \expval{A}^{2}} = \expval{A^{2}} - \expval{A}^{2}
\end{align*}
och motsvarande för $b$. Cauchy-Schwarz' olikhet ger
\begin{align*}
	\expval{a^{2}}\expval{b^{2}}\geq\abs{\expval{ab}}^{2}.
\end{align*}
Vi kan skriva
\begin{align*}
	ab = \frac{1}{2}(ab + ba) + \frac{1}{2}\commut{a}{b}.
\end{align*}
Den första termen är självadjungerad, medan den andra byter tecken vid adjungering. Vi får därmed
\begin{align*}
	\expval{ab} = \expval{\frac{1}{2}(ab + ba)} + \expval{\frac{1}{2}\commut{a}{b}}.
\end{align*}
Vi noterar att
\begin{align*}
	\cc{\expval{\commut{a}{b}}} = -\expval{\commut{a}{b}},
\end{align*}
så denna termen måste vara helt imaginär. Detta implicerar
\begin{align*}
	\expval{ab} \geq \expval{\frac{1}{2}\commut{a}{b}}^{2}.
\end{align*}
Kommutatorn ges av
\begin{align*}
	\commut{a}{b} &= (A - \expval{A})(B - \expval{B}) - (B - \expval{B})(A - \expval{A}) \\
	              &= AB - \expval{B}A - \expval{A}B + \expval{A}\expval{B} - (BA - \expval{A}B - \expval{B}A + \expval{A}\expval{B}) \\
	              &= \commut{A}{B},
\end{align*}
och beviset är klart.

Osäkerhetsprincipen kan även formuleras lite annorlunda, vid att skriva imaginärtermen ovan som
\begin{align*}
	-i\expval{\frac{1}{2}i\commut{a}{b}}.
\end{align*}
med den fördel att operatorn vars väntevärde beräknas nu är självadjungerad. Osäkerhetsprincipen kan då skrivas som
\begin{align*}
	\Delta A\Delta B = \frac{1}{2}\abs{\expval{i\commut{A}{B}}}
\end{align*}

Mats har ett bevis i sina anteckningar, formulerad med en något konstigare operator. Detta beviset tyckte jag varken var förtydligande eller till hjälp, så jag struntar i att visa det.

Tolkningen av detta är att om två operatorer kommuterar, är de samtidigt mätbara. Vi säjer att de är kompatibla. Om två operatorer inte kommuterar, säjer vi att de är inkompatibla. Då är de inte allmänt samtidigt mätbara.

\paragraph{Tidsändring av väntevärden}
Vi får
\begin{align*}
	i\hbar\dv{\expval{Q}}{t} &= i\hbar\dv{t}\expval{Q}{\Psi} \\
	                         &= i\hbar\mel{\Psi}{Q}{\del{t}{\Psi}} + i\hbar\mel{\del{t}{\Psi}}{Q}{\Psi} + i\hbar\braket{\dot{Q}}{\Psi} \\
	                         &= \mel{\Psi}{Q}{i\hbar\del{t}{\Psi}} - \mel{i\hbar\del{t}{\Psi}}{Q}{\Psi} + i\hbar\braket{\dot{Q}}{\Psi} \\
	                         &= \mel{\Psi}{Q}{H\Psi} - \mel{H\Psi}{Q}{\Psi} + i\hbar\braket{\dot{Q}}{\Psi} \\
	                         &= \braket{\commut{Q}{H}}{\Psi} + i\hbar\braket{\dot{Q}}{\Psi},
\end{align*}
vilket ger (Heisenbergs ekvation?)
\begin{align*}
	\dv{\expval{Q}}{t} = \expval{\dot{Q}} + \frac{i}{\hbar}\braket{\commut{H}{Q}}{\Psi}.
\end{align*}
Vi noterar att observabler motsvarande tidsoberoende operatorer som kommuterar med Hamiltonoperatorn är bevarade.

\paragraph{Osäkerhetsrelation för tid}
Detta är sånt som får det att vrida sig i magen på mig.

Osäkerhetsrelationen ger
\begin{align*}
	\Delta H\Delta Q\geq\frac{1}{2}\abs{\expval{\commut{H}{Q}}}.
\end{align*}
Om $Q$ är tidsoberoende, ger Heisenbergs ekvation
\begin{align*}
	\Delta H\Delta Q\geq\frac{\hbar}{2}\abs{\dv{\expval{Q}}{t}}.
\end{align*}
Låt oss nu definiera tiden $\Delta t$ det tar för $\expval{Q}$ att ändra sig med $\Delta Q$ genom
\begin{align*}
	\Delta Q = \abs{\dv{\expval{Q}}{t}}\Delta t.
\end{align*}
Detta ger
\begin{align*}
	\Delta H\Delta t\geq\frac{\hbar}{2}.
\end{align*}

\paragraph{Basvektorer för separabla lösningarn}
Om man har ett separabelt problem, kan den allmänna lösningen skrivas med Diracnotation som
\begin{align*}
	\ket{\Psi_{n_{1}\dots n_{d}}} = \bigotimes\limits_{i = 1}^{d}\ket{\Psi_{i, n_{i}}},
\end{align*}
där $\otimes$ är tensorprodukten. Denna kommer oftast inte skrivas ut.