\section{Diracnotation}

\paragraph{Kvantmekaniska tillstånd och tillståndspostulatet}
Vi utvidgar nu konceptet vågfunktion till ett allmänt kvantmekaniskt tillstånd $\ket{\Psi}$. Detta är en så kallad ketvektor i ett Hilbertrum. Ett fundamentalt postulat i kvantmekaniken är att $\ket{\Psi}$ beskriver systemet fullständigt.

Till ketvektorerna hör även bravektorerna $\bra{\Psi}$, som bildar dualrummet till Hilbertrummet som alla ketvektorer finns i.

\paragraph{Inreprodukt och ortogonalitet}
Till dessa hör en inreprodukt
\begin{align*}
	\braket{\Psi}{\Phi} = \integ[3]{}{}{\vb{x}}{\cc{\Psi}\Phi}.
\end{align*}
Denna inreprodukten uppfyller $\braket{\Psi}{\Phi} = \cc{\braket{\Psi}{\Phi}}$.

Baserad på detta kan vi införa ortogonalitet och normerbarhet. Vi säjer att en vektor är normerad om $\norm{\Psi}^{2} = \braket{\Psi} = 1$. Vi kommer förutsätta att vektorerna vi ser är normerade. Vi säjer även att $\ket{\Psi_{1}}$ och $\ket{\Psi_{2}}$ är ortonormala om $\braket{\Psi_{i}}{\Psi_{j}} = \delta{ij}$.

\paragraph{Operatorer och operatorpostulatet}
Ett annat fundamentalt postulat i kvantmekaniken är att observabler representeras av Hermiteska operatorer med en fullständig mängd av egenvektorer. Och vad betyder detta?

Att representera en observabel med en operator innebär att det till varje klassiska observabel $A$ som kan skrivas som $A(x, p)$ tillhör en operator $\hat{A}(\hat{x}, \hat{p})$. Om det ej finns en klassisk motsvarighet till $A$, fås operatorn från experiment.

Med Diracnotation kan vi även skriva en operatorn som $A = \op{\Psi_{1}}{\Psi_{2}}$. Vi använder i bland begreppet c-nummer om $\braket{}$ och q-nummer om $\op{}{}$. Med detta kan vi även skriva $\expval{A} = \expval{A}{\Psi}$.

För att prata om Hermiteska operatorer, måste vi först prata om adjungerade operatorer. Den adjungerade operatorn $\adj{A}$ till $A$ definieras som den operatorn så att
\begin{align*}
	\integ[3]{}{}{\vb{x}}{\cc{\Psi}A\Phi} = \integ[3]{}{}{\vb{x}}{\cc{A\Psi}\Phi}.
\end{align*}
Med Diracnotation kan detta skrivas som $\mel{\Phi}{A}{\Psi} = \cc{\mel{\Psi}{\adj{A}}{\Phi}}$. En Hermitesk operator är en operator som uppfyller $\adj{A} = A$. Det visar sig att operatorer som $x, p$ och $H$ uppfyller detta, och det är ju trevligt.

Några viktiga räkneregler för adjungerade operatorer är:
\begin{itemize}
	\item $\adj{(c_{1}A + c_{2}B)} = \cc{c_{1}}\adj{A} + \cc{c_{2}}\adj{B}$.
	\item $\adj{(\adj{A})} = A$.
	\item $\adj{(AB)} = \adj{B}\adj{A}$.
	\item Normen i kvadrat av $A\ket{\Psi}$ ges av $\braket{\adj{A}A}{\Psi}$.
\end{itemize}

Det kommer visa sig att vi är intresserade av att studera egenvärdesproblem för självadjungerade operatorer. Därför vill vi veta lite om egenvärdena och egenfunktionerna till såna operatorer.

Betrakta först ett egentillstånd $\ket{\Psi_{n}}$ med egenvärde $a_{n}$. Vi får:
\begin{align*}
	\expval{A} = \expval{A}{\Psi_{n}} = \cc{(\expval{A}{\Psi_{n}})},
\end{align*}
vilket implicerar att $a_{n} = \cc{a_{n}}$, och vi drar slutsatsen att självadjungerade operatorer har reella egenvärden.

Betrakta vidare två olika egentillstånd. Vi får
\begin{align*}
	\mel{\Psi_{n}}{A}{\Psi_{m}} = a_{m}\braket{\Psi_{n}}{\Psi_{m}} = \cc{\mel{\Psi_{m}}{A}{\Psi_{n}}} = \cc{a_{n}}\cc{\braket{\Psi_{n}}{\Psi_{m}}} = \cc{a_{n}}\braket{\Psi_{n}}{\Psi_{m}},
\end{align*}
och drar slutsatsen att att $\braket{\Psi_{n}}{\Psi_{m}} = \delta_{nm}$. Observera att om samma egenvärde har flera egenfunktioner, kan man använda Gram-Schmidts metod för att få ortogonala egenfunktioner.

Att bevisa att dessa egenvektorerna är fullständinga är allmänt bara möjligt för ändligdimensionella vektorrum. Vi kommer baka detta in i postulatet, och postulera att operatorerna vi vill betrakta är konstruerade så att detta är sant.

\paragraph{Diracnotation och operatorer}
Med fullständighetspostulatet kan vi nu skriva ett allmänt tillstånd som $\ket{\Psi} = c_{i}\ket{\Psi_{i}}$, och vi ser även att $c_{i} = \braket{\Psi_{i}}{\Psi}$. Vilken bas har vi utvecklat $\ket{\Psi}$ i? Jo, det är godtyckligt. Oftast kommer det vara egenbasen till en operator.

För att en utgångspunkt, startar vi med identitetsoperatorn $I$. Med resonnemanget ovan kan vi skriva
\begin{align*}
	I\ket{\Psi} = \ket{\Psi} = \op{\Psi_{i}}{\Psi_{i}}\ket{\Psi},
\end{align*}
vilket implicerar $I = \op{\Psi_{i}}{\Psi_{i}}$.

Nu kan vi betrakta en godtycklig operator $Q$ och skriva
\begin{align*}
	Q = IQI = \op{\Psi_{i}}{\Psi_{i}}Q\op{\Psi_{j}}{\Psi_{j}}.
\end{align*}
Vi kan definiera $Q_{ij} = \mel{\Psi_{i}}{Q}{\Psi_{j}}$, och får då
\begin{align*}
	Q = Q_{ij}\op{\Psi_{i}}{\Psi_{j}}.
\end{align*}

\paragraph{Sannolikhetstolkning}
Med hjälp av detta kan vi skriva
\begin{align*}
	\braket{\Psi} = \ev{I}{\Psi} = \braket{\Psi}{\Psi_{i}}\braket{\Psi_{i}}{\Psi} = \cc{c_{i}}c_{i} = 1.
\end{align*}
Vi tolkar detta som att $\abs{c_{i}}^{2}$ är sannolikheten att systemet är i tillståndet $\ket{\Psi{i}}$.

Vi kan även skriva väntevärdet av en observabel
\begin{align*}
	\expval{A} = \ev{A}{\Psi} = \ev{AI}{\Psi} = \mel{\Psi}{A}{\Psi_{i}}\braket{\Psi_{i}}{\Psi} = a_{i}\braket{\Psi}{\Psi_{i}}\braket{\Psi_{i}}{\Psi} = a_{i}\abs{c_{i}}^{2}.
\end{align*}
Notera hur snyggt detta blir när vi använder operatorpostulatet. Detta resultatet tolkar vi som att väntevärdet ges av en summa av produkter av egenvärden och sannolikheten för att systemet är i det motsvarande egentillståndet.

\paragraph{Mätpostulatet}
Ett annat fundamentalt postulat i kvantmekaniken är följande:

Om ett system befinner sig i ett tillstånd $\ket{\Psi}$ ger en mätning av storheten $A$ något av egenvärdena $a_{i}$ till $A$ som resultat med sannolikhet $\braket{\Psi_{i}}{\Psi}$. Mätningen ändrar även systemets tillstånd från $\ket{\Psi}$ till $\ket{Psi_{i}}$, och vi säjer att tillståndet kollapsar.

\paragraph{Schrödingerekvationen}
Det sista postulatet i kvantmekaniken är att tillståndets tidsutvekling ges av
\begin{align*}
	H\ket{\Psi} = i\hbar\dv{t}\ket{\Psi}.
\end{align*}