\section{Statistisk mekanik}

\paragraph{Klassisk statistisk mekanik}
I den klassiska statistiska mekaniken är vi intresserade av att ange tillståndet till ett system med $N$ partiklar. Med tillstånd avses punkter i fasrummet på formen $x = (\vb{r}_{1}, \dots, \vb{r}_{N}, \vb{p}_{1}, \dots, \vb{p}_{N}) = (\vb{r}, \vb{p})$. Vi vill veta observablernas medelvärden
\begin{align*}
	\expval{A} = \frac{1}{\tau}\integ{0}{\tau}{t}{A(x(t))}.
\end{align*}
Att beräkna sådana tidsmedelvärden är opratiskt, då det skulle innebära att lösa för systemets fullständiga bana i fasrummet. Därför medlar vi i stället över ensembler, som är identiska kopior av systemet.

\paragraph{Statistiska mekanikens grundpostulat}
För våra behov kommer vi postulera att två tillstånd med samma energi antas med samma sannolikhet

\paragraph{Mikrokanonisk ensemble}
Det mikrokanoniska ensemblet motsvarar ett slutet system där energien är lik för alla tillstånd.

Antalet tillstånd är proportionellt mot volymen i fasrummet
\begin{align*}
	\Omega(E) = \frac{1}{h^{3N}N!}\int\limits_{E < H < E + \delta E}\dd[3N]{\vb{r}}\dd[3N]{\vb{p}}.
\end{align*}
Då fås ensemblemedelvärdet vid en given energi som
\begin{align*}
	\expval{A} = \frac{1}{\Omega N!}\int\limits_{E < H < E + \delta E}\dd[3N]{\vb{r}}\dd[3N]{\vb{p}}A = \frac{1}{\Omega N!}\int\limits_{E < H < E + \delta E}\dd{x}A.
\end{align*}

\paragraph{Entropi}
Boltzmann definierade den termodynamiska entropin som
\begin{align*}
	S = k\ln{\Omega}.
\end{align*}
Denna definitionen kopplar ihop mikroskopiska tendenser i systemet med dets makroskopiska beteende genom termodynamik.

\paragraph{Kanoniska ensemblem}
I det kanoniska ensemblet betraktar vi ett system som omges av ett energireservoir. Om reservoiren har energi $E_{2}$ och systemet energi $E_{1}$, är kombinationen av dessa ett mikrokanoniskt ensemble. Kombinatoriska argument ger
\begin{align*}
	\Omega(E_{1}, E_{2}) = \Omega_{1}(E_{1})\Omega_{2}(E_{2}),
\end{align*}
då $\Omega$ är ett mått på antal tillstånd. Om systemet har total energi $E$, kan detta skrivas som
\begin{align*}
	\Omega(E_{1}, E_{2}) = \Omega_{1}(E_{1})\Omega_{2}(E - E_{1}).
\end{align*}
Tillståndet med störst sannolikhet motsvarar maximalt $\Omega(E_{1}, E_{2})$ (ett argument kanske behövs här), vilket fås vid entropins maximum. Vi får
\begin{align*}
	S = k(\ln{\Omega_{1}(E_{1})} + \ln{\Omega_{2}(E - E_{1})}) = S_{1} + S_{2}.
\end{align*}
Maximum fås vid $\del{E_{1}}{S} = 0$. Vi får
\begin{align*}
	\del{E_{1}}{S} = \del{E_{1}}{S_{1}} + \del{E_{1}}{S_{2}} = \del{E_{1}}{S_{1}} - \del{E_{2}}{S_{2}}.
\end{align*}
Vi inför temperaturen genom
\begin{align*}
	\frac{1}{T} = \left(\pdv{S_{i}}{E_{i}}\right)_{N, V},
\end{align*}
enligt vad vi vet från termodynamiken, och får att sannolikheten maximeras då $T_{1} = T_{2}$.

Maximumet motsvarande detta kriteriet är mycket skarpt, så makroskopiska system har sällant avvikelser från detta. För att inse detta, skriv
\begin{align*}
	\Omega = e^{\frac{S}{k}}.
\end{align*}
Exponenten beror bara av $E_{1}$, så vi Taylorutvecklar den kring jämviktsenergin $E_{1}^{*}$ för att få
\begin{align*}
	\Omega = e^{\frac{1}{k}\left(S(E_{1}^{*}) + \frac{1}{2}(E_{1} - E_{1}^{*})^{2}\del[2]{E_{1}}{S}\right)}.
\end{align*}
Eftersom vi har ett maximum för sannolikheten, måste andraderivatan i exponenten vara negativ. Vidare, eftersom både energin och entropin är extensiva, kan den uppskattas som
\begin{align*}
	\del[2]{E_{1}}{S}\propto N^{2}\frac{N}{N^{2}} = N,
\end{align*}
vilket indikerar det givna beteendet.

\paragraph{Sannolikhetsfördelning i kanoniska ensemblet}
Vi har att sannolikheten för att systemet är i ett visst tillstånd ges av
\begin{align*}
	p(E_{1})\dd{E_{1}} = \frac{1}{\integ{}{}{E_{1}}{\Omega_{1}(E_{1})\Omega_{2}(E - E_{1})}}\Omega_{1}(E_{1})\Omega_{2}(E - E_{1})\dd{E_{1}}.
\end{align*}
Andra faktorn kan skrivas som
\begin{align*}
	\Omega_{2}(E - E_{1}) = e^{\frac{1}{k}S_{2}(E - E_{1})}.
\end{align*}
Om vi antar att reservoiren är mycket större än systemet, fås
\begin{align*}
	S_{2}(E - E_{1}) \approx S_{2}(E) - E_{1}\del{E}{S_{2}} = S_{2}(E) - \frac{E_{1}}{T_{2}}.
\end{align*}
Detta kan kombineras med jämviktskriteriet för att ge
\begin{align*}
	p(E_{1})\dd{E_{1}} &= \frac{1}{\integ{}{}{E_{1}}{\Omega_{1}(E_{1})e^{\frac{1}{k}\left(S_{2}(E) - \frac{E_{1}}{T_{1}}\right)}}}\Omega_{1}(E_{1})e^{\frac{1}{k}\left(S_{2}(E) - \frac{E_{1}}{T_{1}}\right)}\dd{E_{1}} \\
	                   &= \frac{1}{\integ{}{}{E_{1}}{\Omega_{1}(E_{1})e^{-\frac{E_{1}}{kT_{1}}}}}\Omega_{1}(E_{1})e^{-\frac{E_{1}}{kT_{1}}}\dd{E_{1}},
\end{align*}
och vid att byta beteckning till systemets egna storheter
\begin{align*}
	p(E)\dd{E} = \frac{1}{\integ{}{}{E}{\Omega_{1}(E)e^{-\frac{E}{kT}}}}\Omega_{1}(E)e^{-\frac{E}{kT}}\dd{E}.
\end{align*}
Vid att införa partitionsfunktionen
\begin{align*}
	Z = \integ{}{}{E}{\Omega_{1}(E)e^{-\frac{E}{kT}}}
\end{align*}
fås
\begin{align*}
	p(E)\dd{E} = \frac{1}{Z}\Omega_{1}(E)e^{-\frac{E}{kT}}\dd{E}.
\end{align*}

Vid jämvikt ges fria energin av $F = \expval{E} - TS(\expval{E})$. Då ges partitionsfunktionen av
\begin{align*}
	Z = \integ{}{}{E}{\Omega_{1}(E)e^{-\beta E}} = Z = \integ{}{}{E}{e^{\frac{S}{k}}Omega_{1}(E)e^{-\beta E}}.
\end{align*}
Eftersom volymen i fasrummet har en skarp topp vid $\expval{E}$ fås
\begin{align*}
	Z = e^{-\beta(\expval{E} - TS(\expval{E}))} = e^{-\beta F},
\end{align*}
där vi har infört $\beta = \frac{1}{kT}$.

Nu kan vi skriva energins väntevärde som
\begin{align*}
	\expval{E} &= \frac{1}{Z}\integ{}{}{E}{E\Omega_{1}(E)e^{-\beta E}} \\
	           &= \frac{1}{Zadsfsad}\integ{}{}{x}{\integ{}{}{E}{E\delta(H - E)e^{-\beta E}}} \\
	           &= \frac{1}{Z}\integ{}{}{x}{He^{-\beta H}},
\end{align*}
alltså en summation över alla tillstånd. Detta kommer vara sättet vi gör det på sedan.

\paragraph{Storkanonisk ensemble}
Detta motsvarar system som kan utväxla både energi och materia med reservoiren. Vi får jämvikt vid $T_{1} = T_{2}$ och $\mu_{1} = \mu_{2}$, där
\begin{align*}
	\frac{\mu}{T} = -\left(\pdv{S}{N}\right)_{E, V}.
\end{align*}
Sannolikhetsfördelningen ges av
\begin{align*}
	p_{G}(E, N) = \frac{1}{Z_{G}}\Omega(E, N)e^{-\beta(E - \mu N)},
\end{align*}
där vi har infört den storkanoniska partitionsfunktionen
\begin{align*}
	Z_{G} = \sum\limits_{N = 0}^{\infty}e^{\beta\mu N}Z(T, N),
\end{align*}
där $Z(T, N)$ är den kanoniska partitionsfunktionen.

\paragraph{Kvantmekanisk statistisk mekanik}
I kvantmekanisk statistisk mekanik kommer integralerna över fasrummet att ersättas med summationer över kvantiserade tillstånd.

\paragraph{Ockupationsrepresentation}
För att förstå hur partiklarna fördelas i kvanttillstånd, kan man använda ockupationsrepresentation. Då skriver man ned en serie siffror som anger hur många partiklar som ockuperar varje tillstånd.

\paragraph{Kvantmekaniska fria partiklar}
För att titta på dessa vill vi göra betraktningen i det storkanoniska ensemblet. Detta gör vi för att kombinationen av summation över antal partiklar och integration över tillstånd kan övergå i en summa över alla ockupationstal, som täcker kombinationen av dessa summationer. Vi får
\begin{align*}
	Z_{G} = \sum\limits_{n_1}\dots e^{-\beta\sum\limits_{k}(E_{k} - \mu)n_{k}} = \prod\limits_{k}\sum\limits_{n_{k}}e^{-\beta(E_{k} - \mu)n_{k}} = \prod\limits_{k}Z_{k} = e^{-\beta F}.
\end{align*}

Väntevärdet av antal partiklar i något tillstånd ges av
\begin{align*}
	\expval{n_{k}} = \frac{\sum ne^{-\beta(E_{k} - \mu)n}}{Z_{k}},
\end{align*}
där
\begin{align*}
	Z_{k} = \sum e^{-\beta(E_{k} - \mu)n}.
\end{align*}
Vi ser att vi kan skriva
\begin{align*}
	\expval{n_{k}} = \frac{1}{\beta}\del{\mu}{Z_{k}}.
\end{align*}

För fermioner vet vi att $n_{k} = 0, 1$ på grund av Pauliprincipen. Därmed fås
\begin{align*}
	\expval{n_{k}} = \frac{0 + e^{-\beta(E_{k} - \mu)}}{1 + e^{-\beta(E_{k} - \mu)}} = \frac{1}{e^{\beta(E_{k} - \mu)} + 1},
\end{align*}
vilket kallas Fermi-Dirac-fördelningen.

För bosoner fås
\begin{align*}
	Z_{k} = \sum e^{-\beta(E_{k} - \mu)n} = \frac{1}{1 - e^{-\beta(E_{k} - \mu)}},
\end{align*}
och
\begin{align*}
	\expval{n_{k}} = \frac{1}{e^{\beta(E_{k} - \mu)} - 1},
\end{align*}
vilket kallas Bose-Einstein-fördelningen.

Vi kan jämföra detta med resultatet från särskiljbara identiska partiklar, som ger
\begin{align*}
	\expval{n_{k}} = e^{-\beta(E_{k} - \mu)}.
\end{align*}
Detta är gränsen för båda fördelningarna när energierna blir stora jämförd med temperaturen.

\paragraph{Svartkroppsstrålning}
Fotoner är bosoner med spinn $1$, och följer därmed Bose-Einstein-statistik. Antalet fotoner kommer variera med temperaturen, och detta implicerar $\mu = 0$. Vi får då
\begin{align*}
	\expval{n_{k}} = \frac{1}{e^{\beta\hbar\omega} - 1}.
\end{align*}