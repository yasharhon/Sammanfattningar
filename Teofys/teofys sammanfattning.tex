\documentclass[a4paper, 11pt]{article}
\usepackage[utf8]{inputenc}
\usepackage[swedish]{babel}
\usepackage{hyperref}
\usepackage[margin=0.5in]{geometry}

\usepackage{amssymb}
\usepackage[arrowdel]{physics}

\newcommand{\del}[3][]{\partial_{#2}^{#1}#3}
\newcommand{\deval}[4][]{\eval{\dv[#1]{#2}{#3}}_{#4}}
\newcommand{\pdeval}[4][]{\eval{\del[#1]{#2}{#3}}_{#4}}
\newcommand{\cc}[1]{#1^{*}}
\newcommand{\adj}[1]{#1^{\dagger}}
\newcommand{\commut}[2]{[#1, #2]}
\newcommand{\integ}[5][]{\int\limits_{#2}^{#3}\dd[#1]{#4}#5}

\title{Sammanfattning av SI1155 Teoretisk fysik}
\author{Yashar Honarmandi \\ yasharh@kth.se}
\date{\today}

\begin{document}

\maketitle

\begin{abstract}
	Detta ær en sammanfattning av kursen SH1014 Modern fysik.
\end{abstract}

\pagenumbering{roman}
\thispagestyle{empty}

\newpage

\tableofcontents

\newpage

\pagenumbering{arabic}

\section{Teoretisk fysik}

\paragraph{Teorir}
En teori i fysik är ett matematiskt ramverk av naturlagar som ger samband mellan krafter och rörelse. Dessa baseras på resultat från experiment.

\paragraph{Modeller}
En modell är en beskrivning av ett system som innehåller vissa antaganden om vilka frihetsgrader systemet har och vilka krafter som verkar på dets komponenter.

\section{Grunderna i kvantmekanik}

\paragraph{Newtonsk mekanik}
I newtonsk mekanik beskrivs en partikel i ett intertialsystem av
\begin{align*}
	\vb{F} = m\vb{a},
\end{align*}
där $\vb{F}$ är summan av alla krafter på partikeln. Med givna initialvillkor kan partikelns bana beskrivas exakt.

Om partikeln endast påverkas av konservativa krafter, är dens energi konstant. För en sådan partikel är dens energi alltid större än potentialens minimum.

\paragraph{Analytisk mekanik}
Alternativt till newtonsk mekanik kan man formulera mekaniken med Lagranges och Hamiltons formaliser. Dessa utgår från att beskriva partikelns bana med generaliserade koordinater $q_{i}$ och hastigheter $\dot{q}_{i}$ (totala tidsderivator av koordinaterna), och hitta en bana som minimerar verkansintegralen
\begin{align*}
	S = \integ{}{}{t}{L}
\end{align*}
där $L = T - V$ och $T$ är kinetiska energin. Dessa leder till ekvationer på formen
\begin{align*}
	\del{q_{i}}{L} - \dv{t}\del{\dot{q}_{i}}{L} = 0.
\end{align*}
Detta är Lagranges formalism.

Alternativt kan man införa generaliserade rörelsemängder $p_{i} = \del{\dot{q}_{i}}{L}$ och Hamiltonfunktionen $H = \dot{q}_{i}p_{i} - L = T + V$. Denna transformationen ger dig
\begin{align*}
	\dot{q}_{i} = \del{p_{i}}{H},\ \dot{p}_{i} = -\del{q_{i}}{H}.
\end{align*}
Detta är Hamiltons formalism.

Om systemet är symmetriskt i $q_{i}$-riktningen, beror $H$ ej av denna koordinaten, vilket ger att $\dot{p}_{i} = 0$ och $p_{i}$ är konstant. En sådan koordinat kallas för en cyklisk koordinat. Detta är ett specialfall av Nöethers sats, som säjer att till varje symmetri hör en konserverad storhet.

En fråga är om man kan göra ett variabelbyte så att alla variabler är cykliska. För att diskutera detta, måste vi prata om kanoniska transformationer. En kanonisk transformation är ett variabelbyte som bevarar formen på Hamiltons ekvationer. Efter ett sådant variabelbyte fås en ny Hamiltonfunktion
\begin{align*}
	H(Q, P, t) = H(q, p, t) + \del{t}{S}(q, P, t).
\end{align*}
Vi vill nu gärna veta om denna nya storheten $S$ är verkan. Dens totala tidsderivata ges av
\begin{align*}
	\dot{S} = \del{q_{i}}{S}\dot{q}_{i} + \del{P_{i}}{S}\dot{P}_{i} + \del{t}{S}.
\end{align*}
Om detta är en transformation så att alla koordinater är cykliska, fås
\begin{align*}
	\dot{S} = \del{q_{i}}{S}\dot{q}_{i} + \del{t}{S}.
\end{align*}
Vi noterar enligt ovan att $\del{t}{S} = - H(q, p, t)$. Om nu $S$ skall vara verkan, måste denna totala tidsderivatan vara lika med Lagrangefunktionen. Definitionen av Hamiltonfunktionen ger då
\begin{align*}
	\del{q_{i}}{S} = p_{i},\ \del{P_{i}}{S} = Q_{i}.
\end{align*}
Vi får då Hamilton-Jacobis ekvation
\begin{align*}
	H(q, \del{p}{S}, t) + \del{t}{S} = 0.
\end{align*}

\paragraph{Behov för ny beskrivning}
I början av 1900-talet upptäcktes det via olika experiment att partiklar på små längdskalor visade beteende som ej samsvarade med klassiska teorir. Detta skapade ett behov för en ny teori.

\paragraph{Våg-partikel-dualitet}
Det första steget togs av Max Planck år 1900. I ett desperat försök på att beskriva svartkroppsstrålning antog han att elektromagnetisk strålning upptas och avges i diskreta kvanta av energin $\hbar\omega$, där $\hbar$ är en helt ny konstant. Försöket lyckades.

Senare förstod Einstein att ljus bestod av diskreta enheter med energi $E = pc$, där $p$ är enhetens rörelsemängd.

de Broglie kombinerade dessa två resultaten till sin hypotes: Att alla partiklar har vågegenskaper, och kan beskrivas av en våglängd som ges av
\begin{align*}
	p = \hbar k = \frac{h}{\lambda}.
\end{align*}
Vi har använt vågtalet $k = \frac{2\pi}{\lambda}$ och introducerat Plancks konstant $h = 2\pi\hbar$.

de Broglie antog även att alla partiklar kan beskrivas av en vågfunktion som ges av $\Psi = Ae^{i(kx - \omega t)}$ för en fri partikel. Vid att sätta in resultaten ovan kan denna vågfunktionen skrivas som
\begin{align*}
	\Psi = Ae^{i(\frac{p}{\hbar}x - \frac{E}{\hbar}t)}.
\end{align*}

\paragraph{Motivation av Schrödingerekvationen}
Givet resultaten ovan, noterar vi
\begin{align*}
	i\hbar\del{t}{\Psi} = E\Psi, -\frac{\hbar^{2}}{2m}\del{x}{\Psi} = \frac{p^{2}}{2m}\Psi.
\end{align*}
Vi kommer ihåg den klassiska Hamiltonfunktionen $H = T + V$. För en fri partikel ges denna av $H = \frac{p^{2}}{2m} + V$. Vi ser nu att de två derivationsoperatorerna ovan ersätter totala respektiva kinetiska energin, och Hamiltonsk mekanik inspirerar oss nu att skriva
\begin{align*}
	i\hbar\del{t}{\Psi} = -\frac{\hbar^{2}}{2m}\del{x}{\Psi} + V\Psi.
\end{align*}
Efter detta kommer Schrödingerekvationen att behandlas som en naturlag. I mer allmänna fall kommer vi även skriva detta som
\begin{align*}
	i\hbar\del{t}{\Psi} = H\Psi,
\end{align*}
där $H$ är en operator som representerar systemets Hamiltonfunktion.

\paragraph{Sannolikhet}
Den fysikaliska tolkningen av vågfunktionen är att $\abs{\Psi}^{2}\dd{V} = \cc{\Psi}\Psi\dd{V}$ är sannolikheten för att hitta partikeln i en mycket liten volym. Vi kräver då
\begin{align*}
	\integ[3]{}{}{x}{\abs{\Psi}^{2}} = 1.
\end{align*}
Med denna tolkningen refereras $\Psi$ till som sannolikhetsamplituden.

\paragraph{Operatorer}
Vi har redan sett att det finns en koppling mellan operatorer på vågfunktionen och fysikaliska storheter. Baserad på detta definierar vi väntevärdet av en observabel $q$ som
\begin{align*}
	\expval{q} = \integ[3]{}{}{x}{\cc{\Psi}\hat{q}\Psi}.
\end{align*}
Här är $\hat{q}$ operatorn som motsvarar $q$. Denna distinktionen kommer inte göras mycket vidare i sammanfattningen.

Vi kan tolka denna integralen som en inreprodukt, och med den tolkningen kräver vi att alla operatorer som representerar fysikaliska storheter är självadjungerade eller Hermiteska.

\paragraph{Osäkerhetsprincipen}
Bandbräddsteori ger oss för $x$ och $p$ att
\begin{align*}
	\Delta x\Delta p \geq \frac{1}{2}\hbar.
\end{align*}
Här är $\Delta x$ standardavvikelsen för $x$.

\paragraph{Kontinuitetsekvationen för sannolikhet}
Man kan visa att
\begin{align*}
	\del{t}{\abs{\Psi}^{2}} + \div{j} = 0,
\end{align*}
där $j$ är sannolikhetsströmmen
\begin{align*}
	j = -\frac{i\hbar}{2m}(\cc{\Psi}\grad{\Psi} - \Psi\grad{\cc{\Psi}}).
\end{align*}

\paragraph{Klassiskt förbjudna områden}
Områden där potentialen är större än ett tillstånds energi kallas klassiskt förbjudna områden. Klassiska system kan ej existera i såna områden, men vi kommer se att kvantmekaniska system har nollskild sannolikhet att finnas i klassiskt förbjudna områden.

\paragraph{Ehrenfests sats}
Man kan visa att
\begin{align*}
	\dv{\expval{x}}{t} = \frac{1}{m}\expval{p},\ \dv{\expval{p}}{t} = -\expval{\grad{V}}.
\end{align*}

\paragraph{Vägintegraler}
Även kvantmekaniken kan formuleras med hjälp av verkansintegralen. I denna formalismen ger sannolikheten för att en partikel väljer en bana mellan två punkter ges av
\begin{align*}
	\sum e^{i\frac{S}{\hbar}},
\end{align*}
där summationen görs över alla möjliga banor mellan de två punkterna.

\paragraph{Koppling till klassisk mekanik}
För att se kopplingen till klassisk fysik, är vi intresserade av att se vad som händer när $\hbar\to 0$. I detta fallet förväntar vi att de Broglie-våglängden blir liten, och att vågfunktionen därmed kommer oscillera mycket. Vi gör då ansatsen $\Psi = Ae^{i\frac{S}{\hbar}}$, där $S$ är någon funktion. Vi antar att $A$ varierar försumbart i rummet jämförd med $S$.

Kinetisk energi-operatorn tillämpad på detta ger
\begin{align*}
	-\frac{\hbar^{2}}{2m}\del[2]{x}{\Psi} &= -\frac{\hbar^{2}}{2m}\del{x}{\left(\del{x}{A}e^{i\frac{S}{\hbar}} + iA\frac{1}{\hbar}\del{x}{S}e^{i\frac{S}{\hbar}}\right)} \\
	                                      &=  -\frac{\hbar^{2}}{2m}\left(\del[2]{x}{A}e^{i\frac{S}{\hbar}} + i\del{x}{A}\frac{1}{\hbar}\del{x}{S}e^{i\frac{S}{\hbar}} + i\frac{1}{\hbar}\del{x}{A}\del{x}{S}e^{i\frac{S}{\hbar}} + iA\frac{1}{\hbar}\del[2]{x}{S}e^{i\frac{S}{\hbar}} - A\frac{1}{\hbar^{2}}(\del{x}{S})^{2}e^{i\frac{S}{\hbar}}\right).
\end{align*}
Observera att i gränsen kommer endast en term att överleva. Vi får även
\begin{align*}
	i\hbar\del{t}{\Psi} = i\hbar\left(\del{t}{A}e^{i\frac{S}{\hbar}} + iA\frac{1}{\hbar}\del{t}{S}e^{i\frac{S}{\hbar}}\right).
\end{align*}
Även här kommer endast en term överleva i gränsen. Insatt i Schrödingerekvationen och evaluerad i gränsen fås
\begin{align*}
	\left(\frac{1}{2m}(\del{x}{S})^{2} + V\right)\Psi = -\del{t}{S}\Psi,
\end{align*}
vilket implicerar
\begin{align*}
	\frac{1}{2m}(\del{x}{S})^{2} + V + \del{t}{S} = 0.
\end{align*}
De två första termerna motsvarar Hamiltonfunktionen i klassisk mekanik, där i har gjort transformationen $p\to\del{x}{S}$. Med denna tolkningen ser vi att detta motsvarar Hamilton-Jacobis ekvation.

\paragraph{Separation av Schrödingerekvationen}
Betrakta Schrödingerekvationen för en statisk potential. Då verkar Hamiltonoperatorn endast på rumliga koordinater. Vi gör då produktansatsen $\Psi = \psi\phi$, där $\psi$ endast beror av rumliga koordinater och $\phi$ endast av tiden. Detta ger
\begin{align*}
	H\Psi = \phi H\psi = i\hbar\del{t}{\Psi} = i\hbar \psi\dv{\phi}{t}.
\end{align*}
Vi kan nu dividera med $\Psi$ för att få
\begin{align*}
	\frac{1}{\psi}H\psi = \frac{i\hbar}{\phi}\dv{\phi}{t}.
\end{align*}
Eftersom varje sida beror av olika koordinater, måste de vara lika med en konstant. Av ingen anledning alls väljer vi att kalla denna konstanten för $E$. Vi löser först tidsdelen för att få
\begin{align*}
	\dv{\phi}{t} = -\frac{iE}{\hbar}\phi,
\end{align*}
med lösning
\begin{align*}
	\phi = Ce^{-i\frac{E}{\hbar}t}.
\end{align*}
Vi ser att $E$ måste motsvara systemets totala energi. Insatt i den andra ekvationen ger detta
\begin{align*}
	H\psi = E\psi.
\end{align*}
För stationära fall är alltså vågfunktionen ett egentillstånd till Hamiltonoperatorn.

Betrakta vidare väntevärdet av en observabel som representeras av en operator som ej verkar på tidskoordinaten. Detta ger
\begin{align*}
	\expval{q} &= \integ[3]{}{}{x}{\cc{\Psi}q\Psi} \\
	           &= \integ[3]{}{}{x}{\cc{\psi}\cc{phi}q(\psi\phi)} \\
	           &= \integ[3]{}{}{x}{\cc{\psi}\cc{phi}\phi q\psi} \\
	           &= \integ[3]{}{}{x}{\cc{\psi}\abs{phi}^{2} q\psi} \\
	           &= \integ[3]{}{}{x}{\cc{\psi} q\psi}.
\end{align*}
Alltså är väntevärdet fullständigt tidsoberoende. Speciellt gäller för energin att
\begin{align*}
	\expval{q} &= \integ[3]{}{}{x}{\cc{\Psi} H\Psi} \\
	           &= \integ[3]{}{}{x}{\cc{\psi} H\psi} \\
	           &= \integ[3]{}{}{x}{\cc{\psi} E\psi} \\
	           &= E\integ[3]{}{}{x}{\cc{\psi}\psi} \\
	           &= E.
\end{align*}
På samma sättet fås $\expval{E^{2}} = E^{2}$ och $\Delta E = \sqrt{\expval{E^{2}} - \expval{E}^{2}} = 0.$

Vi kan även skriva en allmän lösning som en superposition
\begin{align*}
	\Psi = \sum c_{n}\psi_{n}e^{-i\frac{E_{n}}{\hbar}t}
\end{align*}
där $H\psi_{n} = E_{n}\psi_{n}$ och summationen görs över hela systemets energispektrum.

\paragraph{Möjliga energier}
Om en partikel finns i någon potential, måste alla energiegenvärden vara större än potentialens minimum. För att visa detta, kan vi observera att
\begin{align*}
	\expval{p^{2}} = \integ{}{}{x}{\abs{p\Psi}^{2}} > 0
\end{align*}
och
\begin{align*}
	\expval{V} = \integ{}{}{x}{V\abs{\Psi}^{2}} > V_\text{min},
\end{align*}
och därmed måste energins väntevärde vara större än $V_\text{min}$.

\paragraph{Partikel i oändlig låda}
För att få en känsla för vilken sorts fysik som kommer ut av kvantmekaniken, betraktar vi en partikel i en oändlig lådpotential, dvs.
\begin{align*}
	V = 
	\begin{cases}
		0,      &0 < x < a, \\
		\infty, &\text{annars.}
	\end{cases}
\end{align*}
Detta motsvarar en låda med oändligt starka väggar.

Vi noterar först att Schrödingerekvationen ger
\begin{align*}
	\del[2]{x}{\Psi} = -\frac{2m(E - V)}{\hbar^{2}}\psi.
\end{align*}
Alltså är vågfunktionens krökning proportionell mot $\sqrt{E - V}$. Detta implicerar om att $E - V < 0$ avtar beloppet av vågfunktionen exponentiellt i detta området. Om detta gäller överallt, kan det ej existera lösningar för $E < V_{\text{min}}$. Vi kommer därför anta att detta är sant härifrån.

Vi observerar även att argumentet implicerar att eftersom potentialen är oändlig utanför lådan, måste vågfunktionen endast vara nollskild inuti lådan, och det återstår att lösa Schrödingerekvationen i denna regionen. Här fås.
\begin{align*}
	-\frac{\hbar^{2}}{2m}\del{x}{\del{x}{\psi}} &= E\psi, \\
	\del{x}{\del{x}{\psi}}                      &= -\frac{2mE}{\hbar^{2}}\psi.
\end{align*}
Vi definierar nu
\begin{align*}
	k^{2} = \frac{2mE}{\hbar^{2}}
\end{align*}
och får
\begin{align*}
	\del{x}{\del{x}{\psi}} = -k^{2}\psi.
\end{align*}
Detta har lösningar
\begin{align*}
	\psi = Ae^{ikx} + Be^{-ikx}.
\end{align*}

För att få mer information, behövs randvillkor. Vi kräver att vågfunktionen är kontinuerlig, vilket ger $\psi(0) = \psi(a) = 0$. Första randvillkoret ger
\begin{align*}
	A + B &= 0, \\
	\psi  &= A\sin{kx}.
\end{align*}
Observera att detta inte kan uppfyllas för $k = 0$, varför denna möjligheten kan försummas.

Andra randvillkoret ger
\begin{align*}
	ka = n\pi.
\end{align*}
Nu kan energin bestämmas enligt
\begin{align*}
	\frac{n^{2}\pi^{2}}{a^{2}} &= \frac{2mE}{\hbar^{2}}, \\
	E                          &= \frac{\hbar^{2}n^{2}\pi^{2}}{2ma^{2}}.
\end{align*}

Slutligen ger normaliseringsvillkoret
\begin{align*}
	\integ{0}{a}{x}{\abs{B}^{2}\sin^{2}{kx}} = 1.
\end{align*}
Integralen på vänstersidan är
\begin{align*}
	\abs{B}^{2}\integ{0}{a}{x}{\frac{1 - \cos{2kx}}{2}}.
\end{align*}
Den andra termen ger inget bidrag eftersom den har period $a$, och detta ger
\begin{align*}
	\abs{B}^{2}\frac{a}{2} &= 1, \\
	\abs{B}                &= \sqrt{\frac{2}{a}}.
\end{align*}
Observera att vi endast skriver absolutbeloppet eftersom $B$ kan innehålla en komplex fas utan att det ändrar fysiken.

\paragraph{Harmoniska oscillatorn}
Harmoniska oscillatorn är ett viktigt problem i kvantmekaniken.

Klassiskt dyker det i bland upp problem som involverar partiklar i en potential som är kvadratisk. Ett typiskt exempel är en partikel som är fast i en fjäder. I dessa sammanhang kommer vi skriva potentialen som $V = \frac{1}{2}m\omega^{2}x^{2}$. Vi vet att partikelns rörelse är periodisk med en viss amplitud. Dens totala energi ges av $E = \frac{1}{2}m\omega^{2}A^{2}$, där $A$ är dens amplitude.

Både klassiskt och kvantmekaniskt kommer vi Taylorutveckla olika potentialer och få
\begin{align*}
	V \approx V(0) + \deval{V}{x}{0}x + \frac{1}{2}\deval[2]{V}{x}{0}x^{2}.
\end{align*}
Vi antar att $x = 0$ motsvarar jämvikt, varför den andra termen måste vara $0$. Vidare är det konstanta bidraget ointressant, då det ej ger något bidrag till fysiken. Kvar står en harmonisk oscillatorterm vars styrka ges av potentialens krökning i jämviktsläget.

\section{Endimensionella problem}

\paragraph{Partikel i oändlig låda}
För att få en känsla för vilken sorts fysik som kommer ut av kvantmekaniken, betraktar vi en partikel i en oändlig lådpotential, dvs.
\begin{align*}
	V = 
	\begin{cases}
		0,      &0 < x < a, \\
		\infty, &\text{annars.}
	\end{cases}
\end{align*}
Detta motsvarar en låda med oändligt starka väggar.

Vi noterar först att Schrödingerekvationen ger
\begin{align*}
	\del[2]{x}{\Psi} = -\frac{2m(E - V)}{\hbar^{2}}\psi.
\end{align*}
Alltså är vågfunktionens krökning proportionell mot $\sqrt{E - V}$. Detta implicerar om att $E - V < 0$ avtar beloppet av vågfunktionen exponentiellt i detta området. Om detta gäller överallt, kan det ej existera lösningar för $E < V_{\text{min}}$. Vi kommer därför anta att detta är sant härifrån.

Vi observerar även att argumentet implicerar att eftersom potentialen är oändlig utanför lådan, måste vågfunktionen endast vara nollskild inuti lådan, och det återstår att lösa Schrödingerekvationen i denna regionen. Här fås.
\begin{align*}
	-\frac{\hbar^{2}}{2m}\del{x}{\del{x}{\psi}} &= E\psi, \\
	\del{x}{\del{x}{\psi}}                      &= -\frac{2mE}{\hbar^{2}}\psi.
\end{align*}
Vi definierar nu
\begin{align*}
	k^{2} = \frac{2mE}{\hbar^{2}}
\end{align*}
och får
\begin{align*}
	\del{x}{\del{x}{\psi}} = -k^{2}\psi.
\end{align*}
Detta har lösningar
\begin{align*}
	\psi = Ae^{ikx} + Be^{-ikx}.
\end{align*}

För att få mer information, behövs randvillkor. Vi kräver att vågfunktionen är kontinuerlig, vilket ger $\psi(0) = \psi(a) = 0$. Första randvillkoret ger
\begin{align*}
 	A + B &= 0, \\
 	\psi  &= A\sin{kx}.
\end{align*}
Observera att detta inte kan uppfyllas för $k = 0$, varför denna möjligheten kan försummas.

Andra randvillkoret ger
\begin{align*}
	ka = n\pi.
\end{align*}
Nu kan energin bestämmas enligt
\begin{align*}
	\frac{n^{2}\pi^{2}}{a^{2}} &= \frac{2mE}{\hbar^{2}}, \\
	E                          &= \frac{\hbar^{2}n^{2}\pi^{2}}{2ma^{2}}.
\end{align*}

Slutligen ger normaliseringsvillkoret
\begin{align*}
	\integ{0}{a}{x}{\abs{B}^{2}\sin^{2}{kx}} = 1.
\end{align*}
Integralen på vänstersidan är
\begin{align*}
	\abs{B}^{2}\integ{0}{a}{x}{\frac{1 - \cos{2kx}}{2}}.
\end{align*}
Den andra termen ger inget bidrag eftersom den har period $a$, och detta ger
\begin{align*}
	\abs{B}^{2}\frac{a}{2} &= 1, \\
	\abs{B}                &= \sqrt{\frac{2}{a}}.
\end{align*}
Observera att vi endast skriver absolutbeloppet eftersom $B$ kan innehålla en komplex fas utan att det ändrar fysiken.

\paragraph{Fria partiklar}
För en fri partikel, dvs. en partikel som inte känner någon potential, är egenfunktionerna till Hamiltonoperatorn plana vågor. Om en given lösning har vågtal $k$, är lösningen
\begin{align*}
	\Psi = Ae^{i(kx - \omega t)}.
\end{align*}
Detta är en egenfunktion till rörelsemängdsoperatorn med egenvärde $p = \hbar k$. Energin ges då av
\begin{align*}
	E = \frac{\hbar^{2}k^{2}}{2m} = \hbar\omega(k),
\end{align*}
där sista likheten kommer av energioperatorn som en partiell derivata med avseende på tiden. En sådan relation mellan $\omega$ och $k$ kallas för en dispersionsrelation.

Detta tillståndet har en konstant sannolikhetstäthet överallt. Därmed är tillståndet ej normerbara, och såna tillstånd kan i sig själv ej vara fysikaliska.

För att få ett normerbart tillstånd för en fri partikel, kan man superponera egentillstånden enligt
\begin{align*}
	\eval{\Psi(x, t)}_{t = 0} = \frac{1}{\sqrt{2\pi}}\integ{-\infty}{\infty}{k}{\eval{\Psi(k, t)}_{t = 0}e^{ikx}}
\end{align*}
Detta är Fouriertransformen, där $\eval{\Psi(k)}_{t = 0}$ är Fouriertransformen vid $t = 0$, och man har att
\begin{align*}
	\eval{\Psi(k, t)}_{t = 0} = \frac{1}{\sqrt{2\pi}}\integ{-\infty}{\infty}{x}{\eval{\Psi(x, t)}_{t = 0}e^{-ikx}}.
\end{align*}
Vi kan nu skriva vågfunktionens tidsutveckling som
\begin{align*}
	\Psi(x, t) = \frac{1}{\sqrt{2\pi}}\integ{-\infty}{\infty}{k}{\Psi(k, t)e^{i(kx - \omega t)}}.
\end{align*}
Det visar sig att sådana tillstånd kan vara normerbara.

\paragraph{Fashastighet och grupphastighet}
Våghastigheten hos en plan våg kallas fashastigheten och ges av $v_{\text{f}} = \frac{\omega}{k}$. För en plan våg ges den av $v_{\text{f}} = \sqrt{\frac{E}{2m}}$. Den klassiska hastigheten för en partikel med energi $E$ ges av $v_{\text{g}} = \sqrt{\frac{2E}{m}}$. Det är ju inte kul, varför beter inte kvantmekaniska partiklar sig likadant? Det kommer av att ett vågpaket inte rör sig med fashastigheten till någon av vågorna den är uppbygd av, men med grupphastigheten $v_{\text{g}} = \dv{\omega}{k}$.

För att se hur den uppkommer, antag att $\Psi(k)$ har ett maximum vid $k_{0}$ och har så liten spridning att expansionen
\begin{align*}
	\omega = \omega_{0} + v_{\text{g}}(k - k_{0})
\end{align*}
för den givna definitionen av $v_{\text{g}}$ är en bra approximation öveallt där $\Psi(k)$ inte är försumbar. Detta ger
\begin{align*}
	\Psi(x, t) &= \frac{1}{\sqrt{2\pi}}\integ{-\infty}{\infty}{k}{\Psi(k, t)e^{i(kx - (\omega_{0} + v_{\text{g}}(k - k_{0}))t}} \\
	           &= \frac{1}{\sqrt{2\pi}}e^{i(v_{\text{g}}k_{0} - \omega_{0})t}\integ{-\infty}{\infty}{k}{\Psi(k, t)e^{ik(x - v_{\text{g}}t)}} \\
	           &= e^{i(v_{\text{g}}k_{0} - \omega_{0})t}\Psi(x - v_{\text{g}}t, 0).
\end{align*}
Detta är alltså ett vågpaket som rör sig med hastighet $v_{\text{g}}$.

\paragraph{Deltapotentialen}
Betrakta en potential $V = -\alpha\delta(x)$. Vi vill studera både bundna tillstånd till dena potentialen och spridning på den.

Bundna tillstånd fås för $E < 0$, och ges av
\begin{align*}
	\psi = Ae^{\kappa x} + Be^{-\kappa x},\ \kappa = \sqrt{-\frac{2mE}{\hbar^{2}}}
\end{align*}
där konstanterna är olika på varje sida av potentialen. Mer specifikt ger normeringskrav och kontinuitet att
\begin{align*}
	\psi =
	\begin{cases}
		Ae^{-\kappa x}, &x > 0, \\
		Ae^{\kappa x},  &x < 0.
	\end{cases}
\end{align*}
Derivatan av vågfunktionen är diskontinuerlig. Integration av Schrödingerekvationen ger
\begin{align*}
	\integ{-\varepsilon}{\varepsilon}{x}{\del[2]{x}{\Psi}} = \deval{\Psi}{x}{\varepsilon} - \deval{\Psi}{x}{-\varepsilon} = \frac{2m}{\hbar^{2}}\integ{-\varepsilon}{\varepsilon}{x}{V\Psi}.
\end{align*}
Med den givna potentialen blir högersidan $-\frac{2m\alpha A}{\hbar^{2}}$. Å andra sidan har derivatan värdet $-\kappa Ae^{-\kappa\varepsilon}$ för $x > 0$ och $\kappa Ae^{-\kappa\varepsilon}$ för $x < 0$, fås
\begin{align*}
	-2\kappa A = -\frac{2m\alpha A}{\hbar^{2}},
\end{align*}
vilket har lösningen $E = -\frac{m\alpha^{2}}{2\hbar^{2}}$. Alltså finns endast ett bundet tillstånd.

Vi studerar vidare spridningstillstånden, som har energi $E > 0$. Om vi antar att det kommer in en våg från vänster, ges vågfunktionen av
\begin{align*}
	\Psi =
	\begin{cases}
		 Ae^{ikx} + Be^{-ikx}, &x < 0, \\
		 Ce^{ikx},             &x > 0,
	\end{cases}
\end{align*}
där $k = \sqrt{\frac{2mE}{\hbar^{2}}}$. Normerbarheten av lösningen är ointressang, då vi endast är intresserade av hur stor sannolikheten är för att partikeln transmitteras eller reflekteras.

Kontinuitetsvillkoret ger $A + B = C$. Derivatans diskontinuitet ger
\begin{align*}
	Cik - Aik + Bik = -\frac{2m\alpha}{\hbar^{2}}\Psi(0) = -\frac{2m\alpha}{\hbar^{2}}(A + B).
\end{align*}
Vi definierar $\beta = \frac{m\alpha}{\hbar^{2}k}$. Då har detta systemet lösningar
\begin{align*}
	B = \frac{i\beta}{1 - i\beta}A,\ F = \frac{1}{1 - i\beta}A.
\end{align*}
Vi vill använda detta för att definiera transmissions- och reflektionskoefficienter i termer av sannolikhetsström. För en plan våg är sannolikhetsströmmen proportionell mot $\abs{\Psi}^{2}$, och övriga faktorer kommer vara lika i vårat fall eftersom alla ingående plana vågor har samma vågtal. Det visar sig också att sannolikhetsströmmen på vänstersidan ges av $\abs{A}^{2} - \abs{B}^{2}$, alltså av ett bidrag från den inkommande vågen och ett från den reflekterade vågen. Då kan vi definiera
\begin{align*}
	T &= \frac{\abs{B}^{2}}{\abs{A}^{2}} = \frac{1}{1 + \frac{m\alpha^{2}}{2\hbar^{2}E}}, \\
	R &= \frac{\abs{C}^{2}}{\abs{A}^{2}} = \frac{1}{1 + \frac{2\hbar^{2}E}{m\alpha^{2}}}.
\end{align*}
Dessa uppfyller $T + R = 1$, och vi noterar att transmissionssannolikheten ökar med $E$. Vi noterar även att dessa ej beror på om potentialen är en deltabrunn eller en deltavägg. Anledningen till att partiklar kan koma sig genom deltaväggar är tunneling.

\paragraph{Ändlig potentialbrunn}
Tillkommer kanske.

\paragraph{Rektangulär potentialbarriär}
Tillkommer kanske.

\section{Diracnotation}

\paragraph{Kvantmekaniska tillstånd och tillståndspostulatet}
Vi utvidgar nu konceptet vågfunktion till ett allmänt kvantmekaniskt tillstånd $\ket{\Psi}$. Detta är en så kallad ketvektor i ett Hilbertrum. Ett fundamentalt postulat i kvantmekaniken är att $\ket{\Psi}$ beskriver systemet fullständigt.

Till ketvektorerna hör även bravektorerna $\bra{\Psi}$, som bildar dualrummet till Hilbertrummet som alla ketvektorer finns i.

\paragraph{Inreprodukt och ortogonalitet}
Till dessa hör en inreprodukt
\begin{align*}
	\braket{\Psi}{\Phi} = \integ[3]{}{}{\vb{x}}{\cc{\Psi}\Phi}.
\end{align*}
Denna inreprodukten uppfyller $\braket{\Psi}{\Phi} = \cc{\braket{\Psi}{\Phi}}$.

Baserad på detta kan vi införa ortogonalitet och normerbarhet. Vi säjer att en vektor är normerad om $\norm{\Psi}^{2} = \braket{\Psi} = 1$. Vi kommer förutsätta att vektorerna vi ser är normerade. Vi säjer även att $\ket{\Psi_{1}}$ och $\ket{\Psi_{2}}$ är ortonormala om $\braket{\Psi_{i}}{\Psi_{j}} = \delta{ij}$.

\paragraph{Operatorer och operatorpostulatet}
Ett annat fundamentalt postulat i kvantmekaniken är att observabler representeras av Hermiteska operatorer med en fullständig mängd av egenvektorer. Och vad betyder detta?

Att representera en observabel med en operator innebär att det till varje klassiska observabel $A$ som kan skrivas som $A(x, p)$ tillhör en operator $\hat{A}(\hat{x}, \hat{p})$. Om det ej finns en klassisk motsvarighet till $A$, fås operatorn från experiment.

Med Diracnotation kan vi även skriva en operatorn som $A = \op{\Psi_{1}}{\Psi_{2}}$. Vi använder i bland begreppet c-nummer om $\braket{}$ och q-nummer om $\op{}{}$. Med detta kan vi även skriva $\expval{A} = \expval{A}{\Psi}$.

För att prata om Hermiteska operatorer, måste vi först prata om adjungerade operatorer. Den adjungerade operatorn $\adj{A}$ till $A$ definieras som den operatorn så att
\begin{align*}
	\integ[3]{}{}{\vb{x}}{\cc{\Psi}A\Phi} = \integ[3]{}{}{\vb{x}}{\cc{A\Psi}\Phi}.
\end{align*}
Med Diracnotation kan detta skrivas som $\mel{\Phi}{A}{\Psi} = \cc{\mel{\Psi}{\adj{A}}{\Phi}}$. En Hermitesk operator är en operator som uppfyller $\adj{A} = A$. Det visar sig att operatorer som $x, p$ och $H$ uppfyller detta, och det är ju trevligt.

Några viktiga räkneregler för adjungerade operatorer är:
\begin{itemize}
	\item $\adj{(c_{1}A + c_{2}B)} = \cc{c_{1}}\adj{A} + \cc{c_{2}}\adj{B}$.
	\item $\adj{(\adj{A})} = A$.
	\item $\adj{(AB)} = \adj{B}\adj{A}$.
	\item Normen i kvadrat av $A\ket{\Psi}$ ges av $\braket{\adj{A}A}{\Psi}$.
\end{itemize}

Det kommer visa sig att vi är intresserade av att studera egenvärdesproblem för självadjungerade operatorer. Därför vill vi veta lite om egenvärdena och egenfunktionerna till såna operatorer.

Betrakta först ett egentillstånd $\ket{\Psi_{n}}$ med egenvärde $a_{n}$. Vi får:
\begin{align*}
	\expval{A} = \expval{A}{\Psi_{n}} = \cc{(\expval{A}{\Psi_{n}})},
\end{align*}
vilket implicerar att $a_{n} = \cc{a_{n}}$, och vi drar slutsatsen att självadjungerade operatorer har reella egenvärden.

Betrakta vidare två olika egentillstånd. Vi får
\begin{align*}
	\mel{\Psi_{n}}{A}{\Psi_{m}} = a_{m}\braket{\Psi_{n}}{\Psi_{m}} = \cc{\mel{\Psi_{m}}{A}{\Psi_{n}}} = \cc{a_{n}}\cc{\braket{\Psi_{n}}{\Psi_{m}}} = \cc{a_{n}}\braket{\Psi_{n}}{\Psi_{m}},
\end{align*}
och drar slutsatsen att att $\braket{\Psi_{n}}{\Psi_{m}} = \delta_{nm}$. Observera att om samma egenvärde har flera egenfunktioner, kan man använda Gram-Schmidts metod för att få ortogonala egenfunktioner.

Att bevisa att dessa egenvektorerna är fullständinga är allmänt bara möjligt för ändligdimensionella vektorrum. Vi kommer baka detta in i postulatet, och postulera att operatorerna vi vill betrakta är konstruerade så att detta är sant.

\paragraph{Diracnotation och operatorer}
Med fullständighetspostulatet kan vi nu skriva ett allmänt tillstånd som $\ket{\Psi} = c_{i}\ket{\Psi_{i}}$, och vi ser även att $c_{i} = \braket{\Psi_{i}}{\Psi}$. Vilken bas har vi utvecklat $\ket{\Psi}$ i? Jo, det är godtyckligt. Oftast kommer det vara egenbasen till en operator.

För att en utgångspunkt, startar vi med identitetsoperatorn $I$. Med resonnemanget ovan kan vi skriva
\begin{align*}
	I\ket{\Psi} = \ket{\Psi} = \op{\Psi_{i}}{\Psi_{i}}\ket{\Psi},
\end{align*}
vilket implicerar $I = \op{\Psi_{i}}{\Psi_{i}}$.

Nu kan vi betrakta en godtycklig operator $Q$ och skriva
\begin{align*}
	Q = IQI = \op{\Psi_{i}}{\Psi_{i}}Q\op{\Psi_{j}}{\Psi_{j}}.
\end{align*}
Vi kan definiera $Q_{ij} = \mel{\Psi_{i}}{Q}{\Psi_{j}}$, och får då
\begin{align*}
	Q = Q_{ij}\op{\Psi_{i}}{\Psi_{j}}.
\end{align*}

\paragraph{Sannolikhetstolkning}
Med hjälp av detta kan vi skriva
\begin{align*}
	\braket{\Psi} = \ev{I}{\Psi} = \braket{\Psi}{\Psi_{i}}\braket{\Psi_{i}}{\Psi} = \cc{c_{i}}c_{i} = 1.
\end{align*}
Vi tolkar detta som att $\abs{c_{i}}^{2}$ är sannolikheten att systemet är i tillståndet $\ket{\Psi{i}}$.

Vi kan även skriva väntevärdet av en observabel
\begin{align*}
	\expval{A} = \ev{A}{\Psi} = \ev{AI}{\Psi} = \mel{\Psi}{A}{\Psi_{i}}\braket{\Psi_{i}}{\Psi} = a_{i}\braket{\Psi}{\Psi_{i}}\braket{\Psi_{i}}{\Psi} = a_{i}\abs{c_{i}}^{2}.
\end{align*}
Notera hur snyggt detta blir när vi använder operatorpostulatet. Detta resultatet tolkar vi som att väntevärdet ges av en summa av produkter av egenvärden och sannolikheten för att systemet är i det motsvarande egentillståndet.

\paragraph{Mätpostulatet}
Ett annat fundamentalt postulat i kvantmekaniken är följande:

Om ett system befinner sig i ett tillstånd $\ket{\Psi}$ ger en mätning av storheten $A$ något av egenvärdena $a_{i}$ till $A$ som resultat med sannolikhet $\braket{\Psi_{i}}{\Psi}$. Mätningen ändrar även systemets tillstånd från $\ket{\Psi}$ till $\ket{Psi_{i}}$, och vi säjer att tillståndet kollapsar.

\paragraph{Schrödingerekvationen}
Det sista postulatet i kvantmekaniken är att tillståndets tidsutvekling ges av
\begin{align*}
	H\ket{\Psi} = i\hbar\dv{t}\ket{\Psi}.
\end{align*}

\paragraph{Kommutatorer}
Vi definierar kommutatorn mellan två operatorer som
\begin{align*}
	\commut{A}{B} = AB - BA.
\end{align*}
Om denna är $0$ säjs $A$ och $B$ att kommutera. Om två operatorer kommuterar spelar det ingen roll i vilken ordning man mäter de motsvarande observablerna - man kommer få samma resultat.

Kommutatorn uppfyller följande relationer:
\begin{itemize}
	\item $\commut{A}{A} = 0$.
	\item $\commut{A}{B} = -\commut{B}{A}$.
	\item $\commut{AB}{C} = ABC - CAB = ABC - ACB + ACB - CAB = A\commut{B}{C} + \commut{A}{C}B$.
	\item $\commut{A}{BC} = ABC - BCA = ABC - BAC + BAC - BCA = B\commut{A}{C} + \commut{A}{B}C$.
\end{itemize}

\paragraph{Kvantmekanisk harmonisk oscillator}
Den kvantmekaniska harmoniska oscillatorn beskrivs av
\begin{align*}
	H = \frac{1}{2m}p^{2} + \frac{1}{2}m\omega^{2}x^{2}.
\end{align*}
vi vill skriva Hamiltonoperatorn i termer av nya operatorer, så kallade stegoperatorer. Den första är sänkningsoperatorn
\begin{align*}
	a = \frac{1}{\sqrt{2\hbar m\omega}}(ip + m\omega x)
\end{align*}
och höjningsoperatorn
\begin{align*}
	\adj{a} = \frac{1}{\sqrt{2\hbar m\omega}}(-ip + m\omega x).
\end{align*}
Vi noterar att $\adj{a} \neq a$, och därför representerar dessa inte i sig själv observabler. Däremot är
\begin{align*}
	x = \sqrt{\frac{\hbar}{2m\omega}}(\adj{a} + a),\ p = i\sqrt{\frac{\hbar m\omega}{2}}(\adj{a} - a).
\end{align*}
Kommutatorn mellan dessa är
\begin{align*}
	\commut{a}{\adj{a}} &= \frac{1}{2\hbar m\omega}\commut{ip + m\omega x}{-ip + m\omega x} \\
	                    &= \frac{1}{2\hbar m\omega}(\commut{p}{p} + m^{2}\omega^{2}\commut{x}{x} + im\omega\commut{p}{x} - im\omega\commut{x}{p}) \\
	                    &= \frac{1}{2\hbar m\omega}(im\omega(-i\hbar) - im\omega(i\hbar)) \\
	                    &= 1.
\end{align*}
Detta implicerar
\begin{align*}
	a\adj{a} = \adj{a}a + 1.
\end{align*}

Vi kan nu skriva Hamiltonoperatorn som
\begin{align*}
	H &= -\frac{1}{2m}\frac{\hbar m\omega}{2}(\adj{a} - a)^{2} + \frac{1}{2}m\omega^{2}\frac{\hbar}{2m\omega}(\adj{a} + a)^{2} \\
	  &= -\frac{\hbar\omega}{4}((\adj{a})^{2} + a^{2} - \adj{a}a - a\adj{a}) + \frac{\hbar\omega}{4}((\adj{a})^{2} + a^{2} + \adj{a}a + a\adj{a})^{2} \\
	  &= \frac{\hbar\omega}{2}(\adj{a}a + a\adj{a}) \\
	  &= \frac{\hbar\omega}{2}(2\adj{a}a + 1) \\
	  &= \hbar\omega\left(\adj{a}a + \frac{1}{2}\right).
\end{align*}
Vi kan definiera nummeroperatorn $N = \adj{a}a$, och då skriva $H = \hbar\omega\left(N + \frac{1}{2}\right)$. Nummeroperatorn uppfyller 
\begin{align*}
	\commut{N}{a}       &= \adj{a}\commut{a}{a} + \commut{\adj{a}}{a}a = -a, \\
	\commut{N}{\adj{a}} &= \adj{a}\commut{a}{\adj{a}} + \commut{a}{a}a = \adj{a}.
\end{align*}
Därmed uppfyller Hamiltonoperatorn (eftersom multipler av identiteten kommuterar med allt):
\begin{align*}
	\commut{H}{a}       &= -\hbar\omega a, \\
	\commut{H}{\adj{a}} &= \hbar\omega\adj{a}.
\end{align*}

Antag nu att vi känner en egenfunktion $\Psi$. Då får vi
\begin{align*}
	Ha\Psi = aH\Psi - \hbar\omega a\Psi = (E - \hbar\omega)a\Psi,
\end{align*}
och $a\Psi$ är en ny egenfunktion med egenvärde $E - \hbar\omega$. Vi får även
\begin{align*}
	H\adj{a}\Psi = \adj{a}H\Psi + \hbar\omega\adj{a}\Psi = (E + \hbar\omega)\adj{a}\Psi,
\end{align*}
och vi ser nu varför vi döpte operatorena som vi gjorde.

Vi vet samtidigt att vi kan inte fortsätta att sänka potentialen och hitta nya egenfuktioner för alla möjliga egenvärden. Därför måste det finnas ett $\Psi_{0}$ så att $\Psi_{0}$ är en egenfunktion med egenvärde $E_{0}$, men $a\Psi_{0}$ ej är en egenfunktion. Eftersom operatoralgebran funkar som den gör, är då den enda möjligheten att $a\Psi_{0} = 0$. Detta ger en ordinarie differentialekvation med lösning
\begin{align*}
	\Psi_{0} = Ae^{-\frac{m\omega x^{2}}{2\hbar}},\ A = \left(\frac{m\omega}{\pi\hbar}\right)^{\frac{1}{4}}.
\end{align*}
Vi ser vidare att grunntilståndsenergin ges av $E = \frac{1}{2}\hbar\omega$, eftersom $a\Psi_{0} = 0$ och $H = \hbar\omega\left(\adj{a}a + \frac{1}{2}\right)$. Vidare ges då det fullständiga energispektret av $E_{n} = \left(\frac{1}{2} + n\right)\hbar\omega, n = 0, 1, \dots$.

De nästa tillstånden kan fås enligt
\begin{align*}
	\Psi_{n} = A_{n}(\adj{a})^{n}\Psi_{0} \propto H_{n}e^{-\frac{m\omega x^{2}}{2\hbar}},
\end{align*}
där $H_{n}$ är Hermitepolynomen. Detta hade man också kunnat få med en potensserielösning, men detta är mycket snyggare.

Vi tittar lite på nummeroperatorn igen. Den är självadjungerad, så man skulle kunna tro att den representerar en observabel. Det visar sig att den gör det, ty om vi definierar $\ket{n}$ som egentillståndet till Hamiltonoperatorn med energi $E_{n}$, ger egenvärdesekvationen att $N\ket{n} = n\ket{n}$, och $N$ ger alltså ett mått på tillståndets energi.

Till sist kommer lite om normering. Vi antar att det förra tillståndet var normerad, och vill ha en konstant $A$ så att om $a\ket{n} = A\ket{n - 1}$, är även det nya tillståndet normerad. Vi får
\begin{align*}
	\abs{A}^{2}\braket{n - 1} = \braket{\adj{a}a}{n} = n\braket{n},
\end{align*}
där vi har utnyttjat att om $\ket{\Phi} = a\ket{\Psi}$ är $\braket{\Phi} = \braket{\adj{a}a}{\Psi}$. Eftersom de två tillstånden är normerade, måste $A = \sqrt{n}$. Alltså är $a\ket{n} = \sqrt{n}\ket{n - 1}$. På samma sätt fås $\adj{a}\ket{n} = \sqrt{n + 1}\ket{n + 1}$.

\paragraph{Tillstånd som vektorer}
Om vi utvecklar ett tillstånd i ortonormala basfunktioner, kan utvecklingskoefficienterna vara element i vektorer. Mer specifikt, om vi utvec Med denna formalismen kan till exempel inreprodukten skrivas som en skalärprodukt.

\paragraph{Operatorer som matriser}
Om man skriver tillstånden som vektorer, ser vi att operatorer blir matriser, ty vi kommer ihåg att identitetsoperatorn kunne skrivas som $I = \ket{m}\bra{m}$. Detta ger skriva
\begin{align*}
	A = IAI = \ket{m}\mel{m}{A}{n}\ket{n} = \ket{m}A_{mn}\ket{n},
\end{align*}
där de olika $A_{mn}$ är element i en matris som sedan kan operera på tillståndsvektorerna. Vi ser nu att en självadjungerad operator måste uppfylla
\begin{align*}
	A_{mn} = \mel{m}{A}{n} = \mel{n}{A}{m} = \cc{A_{mn}},
\end{align*}
och Hermiteska operatorer representeras av självadjungerade matriser. Sammansättningen av två operatorer ges av
\begin{align*}
	AB = IAIBI = \ket{m}\mel{m}{A}{k}\mel{k}{B}{n}\ket{n} = \ket{m}A_{mk}B_{kn}\ket{n},
\end{align*}
och motsvaras av matrismultiplikation.

Om man utvecklar tillstånden i egenfunktioner till $A$, ges matriselementen av
\begin{align*}
	A_{mn} = \mel{m}{A}{n} = a_{n}\braket{m}{n} = a_{n}\delta_{mn}.
\end{align*}

\paragraph{Diracnotation i diskreta och kontinuerliga fall}
I kontinuerliga fall ersätts summationer över egentillstånd med integraler. Med denna övergången kommer en övergång från att sannolikheten ges av $\abs{\braket{n}{\Psi}}^{2}$ till att sannolikheten för att hitta partikeln i ett litet intervall ges av $\dd{x}\abs{\braket{x}{\Psi}}^{2}$, där $\ket{x}$ är ett egentillstånd till positionsoperatorn, och vi kan alltså behandla vågfunktionen som en expansionskoefficient av det allmänna tillståndet $\ket{\Psi}$. Utifrån detta identifierar vi $\Psi = \braket{x}{\Psi}$. Identiteten blir
\begin{align*}
	I = \integ{}{}{x}{\ket{x}\bra{x}}.
\end{align*}

\paragraph{Ortonormering}
I diskreta fall är ortonormeringsvillkoret $\braket{m}{n} = \delta_{mn}$. I kontinuerliga fall vill vi använda Fouriertransformen för att hitta ett ortonormeringsvillkor.

Vi vet från teori om Fouriertransformen att
\begin{align*}
	\delta(x) &= \frac{1}{2\pi}\integ{-\infty}{\infty}{k}{e^{ikx}}, \\
	\delta(k) &= \frac{1}{2\pi}\integ{-\infty}{\infty}{x}{e^{-ikx}}.
\end{align*}
Betrakta nu två olika egentillstånd till rörelsemängdsoperatorn, dvs. plana vågor. Ortonormering av dessa ger
\begin{align*}
	\braket{p'}{p} &= \integ{}{}{x}{\braket{p'}{x}\braket{x}{p'}} \\
	               &= \abs{A}^{2}\integ{}{}{x}{e^{-i\frac{p'}{\hbar}x}e^{i\frac{p}{\hbar}x}} \\
	               &= \abs{A}^{2}\hbar\integ{}{}{y}{e^{i(p - p')y}} \\
	               &= \abs{A}^{2}2\pi\hbar\delta(p - p').
\end{align*}
Egenfunktionerna är inte fullständigt specifierade än, så vid att välja $A = \frac{1}{\sqrt{2\pi\hbar}}$ fås ett ortonormeringsvillkor $\braket{p'}{p} = \delta(p - p')$, vilket är ganska analogt med ortonormeringsvillkoret i diskreta fall.

\paragraph{Positionsegenfunktioner}
Betrakta nu positionsoperatorn i rörelsemängdsbasen. Egenfunktionsvillkoret ger
\begin{align*}
	\hat{x}\braket{p}{x} = x\braket{p}{x} = x\cc{braket{x}{p}},
\end{align*}
vilket implicerar $\hat{x} = -\frac{\hbar}{i}\del{x}{}$ (detta kommer även fram från Fouriertransformens egenskaper). Deltafunktionsnormering, evt. Fouriertransformering, ger $\braket{x}{y} = \delta(x - y)$.

\paragraph{Fullständighet för operatorer}
Två Hermiteska operatorer har en fullständig mängd gemensamma egenfunktioner om och endast om och endast om $\commut{A}{B} = 0$. Det finns ett bevis för detta i Mats' anteckningar.

\paragraph{Utvidgad osäkerhetsprincip}
\begin{align*}
	\Delta A\Delta B = \frac{1}{2}\abs{\expval{\commut{A}{B}}}.
\end{align*}
Det finns ett bevis för detta i Mats' anteckningar, och följer från Cauchy-Schwarz' olikhet.

Tolkningen av detta är att om två operatorer kommuterar, är de samtidigt mätbara. Vi säjer att de är kompatibla. Om två operatorer inte kommuterar, säjer vi att de är inkompatibla. Då är de inte allmänt samtidigt mätbara.

\paragraph{Tidsändring av väntevärden}
Vi får
\begin{align*}
	i\hbar\dv{\expval{Q}}{t} &= i\hbar\dv{t}\expval{Q}{\Psi} \\
	                         &= i\hbar\mel{\Psi}{Q}{\del{t}{\Psi}} + i\hbar\mel{\del{t}{\Psi}}{Q}{\Psi} + i\hbar\braket{\dot{Q}}{\Psi} \\
	                         &= \mel{\Psi}{Q}{i\hbar\del{t}{\Psi}} - \mel{i\hbar\del{t}{\Psi}}{Q}{\Psi} + i\hbar\braket{\dot{Q}}{\Psi} \\
	                         &= \mel{\Psi}{Q}{H\Psi} - \mel{H\Psi}{Q}{\Psi} + i\hbar\braket{\dot{Q}}{\Psi} \\
	                         &= \braket{\commut{Q}{H}}{\Psi} + i\hbar\braket{\dot{Q}}{\Psi},
\end{align*}
vilket ger (Heisenbergs ekvation?)
\begin{align*}
	\dv{\expval{Q}}{t} = \expval{\dot{Q}} + \frac{i}{\hbar}\braket{\commut{H}{Q}}{\Psi}.
\end{align*}
Vi noterar att observabler motsvarande tidsoberoende operatorer som kommuterar med Hamiltonoperatorn är bevarade.

\paragraph{Osäkerhetsrelation för tid}
Detta är sånt som får det att vrida sig i magen på mig.

Osäkerhetsrelationen ger
\begin{align*}
	\Delta H\Delta Q\geq\frac{1}{2}\abs{\expval{\commut{H}{Q}}}.
\end{align*}
Om $Q$ är tidsoberoende, ger Heisenbergs ekvation
\begin{align*}
	\Delta H\Delta Q\geq\frac{\hbar}{2}\abs{\dv{\expval{Q}}{t}}.
\end{align*}
Låt oss nu definiera tiden $\Delta t$ det tar för $\expval{Q}$ att ändra sig med $\Delta Q$ genom
\begin{align*}
	\Delta Q = \abs{\dv{\expval{Q}}{t}}\Delta t.
\end{align*}
Detta ger
\begin{align*}
	\Delta H\Delta t\geq\frac{\hbar}{2}.
\end{align*}

\section{Kvantmekanik i tre dimensioner}

\paragraph{Separabla problem}
Betrakta ett problem med en Hamiltonoperator som kan skrivas som $H = \sum H_{i}$, där varje $H_{i}$ endast verkar på $x_{i}$-koordinaten. Då kan Schrödingerekvationen skrivas som
\begin{align*}
	\sum H_{i}\Psi = i\hbar\del{t}{\Psi}.
\end{align*}
Vid att ansätta
\begin{align*}
	\Psi(\vb{x}) = \prod \Psi_{i}(x_{i})
\end{align*}
kan Schrödingerekvationen separeras till $d$ problem
\begin{align*}
	H_{i}\Psi_{i} = i\hbar\del{t}{\Psi_{i}}.
\end{align*}
Den totala lösningen fås vid att multiplicera alla lösningar, och den totala energin fås vid att summera upp energin för varje tillstånd. Om varje lösning har energin $E_{i, n_{i}}$, är den totala energin $E_{n_{1}\dots n_{d}} = \sum E_{i, n_{i}}$.

\paragraph{Sfäriskt symmetriska problem}
Betrakta ett enpartikelproblem med en sfäriskt symmetrisk potential, kan den tidsoberoende Schrödingerekvationen
\begin{align*}
	\left(-\frac{\hbar^{2}}{2m}\left(\frac{1}{r^{2}}\pdv{r}\left(r^{2}\pdv{r}\right) + \frac{1}{r^{2}\sin(\theta)}\pdv{\theta}\left(\sin(\theta)\pdv{\theta}\right) + \frac{1}{r^{2}\sin^{2}(\theta)}\pdv[2]{\phi}\right) + V(r)\right)\Psi = E\Psi
\end{align*}
separeras. Separera ut $r$-beroendet i funktionen $R$ och vinkelberoendet i funktionen $Y$. De två sidorna är då lika med konstanten som vi döper $l(l + 1)$. Detta ger följande ekvationer:
\begin{align*}
	\dv{r}\left(r^{2}\dv{R}{r}\right) -\frac{2mr^{2}}{\hbar^{2}}(V(r) - E)R &= l(l + 1)R, \\
	- \left(\frac{1}{\sin(\theta)}\pdv{\theta}\left(\sin(\theta)\pdv{Y}{\theta}\right) + \frac{1}{\sin^{2}(\theta)}\pdv[2]{Y}{\phi}\right)                           &= l(l + 1)Y.
\end{align*}

Vi kan även separera vinkelberoendet, och de två sidorna blir lika med en konstant $m^{2}$. Detta ger två ekvationer:
\begin{align*}
	\sin(\theta)\dv{\theta}\left(\sin(\theta)\dv{\Theta}{\theta}\right) + l(l + 1)\sin[2](\theta)\Theta          &= m^{2}\Theta, \\
	-\dv[2]{\Phi}{\phi} &= m^{2}\Phi.
\end{align*}

Första ekvationen har lösning
\begin{align*}
	\Phi = e^{im\phi}.
\end{align*}
Denna lösningen måste vara $2\pi$-periodisk, vilket implicerar att $m$ är ett heltal. Lösningarna är ortogonala under inreprodukten
\begin{align*}
	\braket{f}{g} = \integ{0}{2\pi}{\phi}{\cc{f}g}.
\end{align*}

När vi nu känner $m$, skriver vi om $\theta$-ekvationen som
\begin{align*}
	-\frac{1}{\sin(\theta)}\left(\dv{\theta}\left(\sin(\theta)\dv{\Theta}{\theta}\right) - \frac{m^{2}}{\sin{\theta}}\Theta\right) &= l(l + 1)\Theta.
\end{align*}
Efter lite mer algebra fås lösningen
\begin{align*}
	P_{l}^{m}(x) = (1 - x^{2})^{\frac{\abs{m}}{2}}\dv[\abs{m}]{P_{l}}{x},
\end{align*}
där $x = \cos{\theta}$ och $P_{l}$ är Legendrepolynomen av grad $l$. Legendrepolynomens gradtal ger att $\abs{m}\leq l$. Lösningarna är ortogonala under inreprodukten
\begin{align*}
	\braket{f}{g} = \integ{0}{\pi}{\theta}{\sin(\theta)\cc{f}g}.
\end{align*}
Produkten av dessa två blir klotytefunktionerna $Y_{l}^{m}$.

Kvar står den radiella ekvationen
\begin{align*}
	\dv{r}\left(r^{2}\dv{R}{r}\right) - \frac{2mr^{2}}{\hbar^{2}}(V(r) - E)R &= l(l + 1)R,
\end{align*}
som vi kan skriva om till
\begin{align*}
	\frac{1}{r^{2}}\dv{r}\left(r^{2}\dv{R}{r}\right) - \frac{2m}{\hbar^{2}}(V(r) - E)R &= \frac{l(l + 1)}{r^{2}}R, \\
	-\frac{1}{r^{2}}\left(\dv{r}\left(r^{2}\dv{R}{r}\right) + \frac{2mr^{2}}{\hbar^{2}}\left(V(r) + \frac{\hbar^{2}l(l + 1)}{2m}\right)R\right) &= \frac{2mE}{\hbar^{2}}R.
\end{align*}
Detta är ett Sturm-Liouville-problem, och vi vet därmed att det finns oändligt många lösningar för positiva energier. Lösningarna är ortogonala under inreprodukten
\begin{align*}
	\braket{f}{g} = \integ{0}{\infty}{r}{r^{2}\cc{f}g}.
\end{align*}
Den totala inreprodukten av lösningarna är nu
\begin{align*}
	\braket{f}{g} &= \integ{0}{\infty}{r}{r^{2}\integ{0}{\pi}{\theta}{\sin(\theta)\integ{0}{2\pi}{\phi}{\cc{f}g}}} \\
	              &= \integ{0}{\infty}{r}{\integ{0}{\pi}{\theta}{\integ{0}{2\pi}{\phi}{r^{2}\sin(\theta)\cc{f}g}}},
\end{align*}
vilket vi förväntade. Normeringsvillkoret är nu $\braket{\Psi} = 1$.

Vi inför substitutionen $u = Rr$ på den ursprungliga formen av ekvationen, vilket efter lite algebra ger
\begin{align*}
	-\frac{\hbar^{2}}{2m}\dv[2]{u}{r} + \left(V(r) + \frac{\hbar^{2}l(l + 1)}{2mr^{2}}\right)u = Eu.
\end{align*}
Detta är en endimensionell Schrödingerekvation där vi har lagt till en centrifugalbarriär $\frac{\hbar^{2}l(l + 1)}{2mr^{2}}$ till potentialen. Med denna substitutionen kan normeringsvillkoret skrivas som
\begin{align*}
	\integ{0}{\infty}{r}{\integ{0}{\pi}{\theta}{\integ{0}{2\pi}{\phi}{\sin(\theta)\abs{uY_{l}^{m}}^{2}}}} = 1.
\end{align*}

\paragraph{Väteatomen}
Väteatomen består av en proton och en neutron. Protonen är ungefär $2000$ gånger tyngre än elektronen, vilket betyder att atomens masscentrum till god approximation ligger i protonens centrum. Vid att lösa problemet i atomens masscentrum, fås protonen till att vara statisk och elektronen att röra sig i en potential $V(r) = -\frac{e^{2}}{4\pi\varepsilon_{0}r}$. Den radiella ekvationen blir då
\begin{align*}
	-\frac{\hbar^{2}}{2m}\dv[2]{u}{r} + \left(-\frac{e^{2}}{4\pi\varepsilon_{0}r} + \frac{\hbar^{2}l(l + 1)}{2mr^{2}}\right)u = Eu.
\end{align*}
Vi söker först bundna tillstånd, med negativ energi. Inför $E = -\frac{\hbar^{2}k^{2}}{2m},\ \rho = kr$ och $\rho_{0} = \frac{me^{2}}{2\pi\varepsilon_{0}\hbar^{2}k}$. Då fås
\begin{align*}
	\dv[2]{u}{\rho} = \left(1 - \frac{\rho_{0}}{\rho} + \frac{l(l + 1)}{\rho^{2}}\right)u.
\end{align*}
Vi vill lösa detta med en potensserieansats. För stora $\rho$ är de två sista termerna försumbara, och vi får i gränsen $u = Ae^{-\rho}$. För små $\rho$ dominerar sista termen, och vi får gränsen $u = C\rho^{l + 1}$. Vi gör därför ansatsen $u = v(\rho)e^{-\rho}$. Insatt i ekvationen ger detta
\begin{align*}
	\dv[2]{v}{\rho} - 2\dv{v}{\rho} + \frac{\rho_{0}}{\rho}v - \frac{l(l + 1)}{\rho^{2}}v = 0.
\end{align*}
Vi löser detta med potensserieansatsen
\begin{align*}
	v = \sum\limits_{i = 0}^{\infty}c_{i}\rho^{i + l + 1}.
\end{align*}
Vi får rekursionsformeln
\begin{align*}
	c_{i + 1} = \frac{2(j + l + 1) - \rho_{0}}{(j + 1)(j + 2l + 2)}c_{j}.
\end{align*}
Rekursionen måste terminera, då man annars skulle få en lösning som beter sig som $e^{\rho}$ för stora $\rho$. Detta ger villkor för värdena av $\rho$, alltså $k$, alltså $E$. Mer specifikt, låt $n$ vara det största värdet av $j + l + 1$. Detta ger $\rho_{0} = 2n$. Ur detta fås
\begin{align*}
	k = \frac{me^{2}}{4\pi\varepsilon_{0}\hbar^{2}n}
\end{align*}
och slutligen
\begin{align*}
	E_{n} &= -\frac{\hbar^{2}}{2m}\left(\frac{me^{2}}{4\pi\varepsilon_{0}\hbar^{2}n}\right)^{2} \\
	      &= -\frac{me^{4}}{2\hbar^{2}(4\pi\varepsilon_{0})^{2}n^{2}}.
\end{align*}
Den karakteristiska längdskalan i lösningen är Bohr-radien $a$ och ges av
\begin{align*}
	a = \frac{4\pi\varepsilon_{0}\hbar^{2}}{me^{2}}.
\end{align*}
De stationära tillstånden är
\begin{align*}
	\psi_{nlm} = A_{nl}e^{-\frac{r}{na}}\left(\frac{2r}{na}\right)^{l}L_{n - l - 1}^{2l + 1}\left(\frac{2r}{na}\right)Y_{l}^{m}(\theta, \phi),
\end{align*}
där $L$ är Laguerrepolynomen.

Vi ser att för ett givet $n$ kan $l$, vara olikt - mer specifikt är $l = 0, 1, \dots, n - 1$ och de olika egentillstånden är då degenererade med degenerationsgrad
\begin{align*}
	\sum\limits_{l = 0}^{n - 1}2l + 1 = n^{2}.
\end{align*}

\paragraph{Rörelsemängdsmoment}
Klassiskt har vi för rörelsemängdsmomentet för en enda partikel att
\begin{align*}
	\dot{\vb{L}} = \vb{r}\times\vb{F}.
\end{align*}
Speciellt, om det inte finns några yttre kraftmoment, är rörelsemängdsmomentet konstant. Vid att beskriva systemet i sfäriska koordinater och uttrycka hastigheten tangentiellt på ytan med konstant $r$ i termer av rörelsemängdsmomentet, kan Hamiltonianen skrivas som
\begin{align*}
	H = \frac{1}{2}mv_{r}^{2} + \frac{L^{2}}{2mr^{2}} + V.
\end{align*}

När vi gör övergången till kvantmekanik, får vi följande operator:
\begin{align*}
	\vb{L} = \vb{r}\times\vb{p}.
\end{align*}
För att få dens belopp, kan vi titta på Hamiltonoperatorn i sfäriska koordinater och jämföra termvis för att få en operator. Vi får
\begin{align*}
	L^{2} = -\hbar^{2}\left(\frac{1}{\sin(\theta)}\pdv{\theta}\left(\sin(\theta)\pdv{\theta}\right) + \frac{1}{\sin^{2}(\theta)}\pdv[2]{\phi}\right).
\end{align*}
Det finns även argument för detta som görs med hjälp av kryssprodukten, som jag kanske borde lägga till. På detta sättet kan vi även få
\begin{align*}
	L_{z} = \frac{\hbar}{i}\pdv{\phi}.
\end{align*}

Det visar sig att
\begin{align*}
	L^{2}Y_{l}^{m} = \hbar^{2}l(l + 1)Y_{l}^{m},\ L_{z}Y_{l}^{m} = \hbar mY_{l}^{m}.
\end{align*}
Eftersom de ingående operatorerna konstruerades från Hermiteska operatorer, är även dessa Hermiteska, och vi får direkt att klotytefunktionerna är ortogonala med inreprodukten som gavs ovan.

Mellan de olika komponenterna av rörelsemängdsmomentet finns följande kommutationsrelationer:
\begin{align*}
	\commut{L_{x}, L_{y}} &= \commut{yp_{z} - zp_{y}}{zp_{x} - xp_{z}} \\
	                      &= \commut{yp_{z}}{zp_{x}} - \commut{yp_{z}}{xp_{z}} - \commut{zp_{y}}{zp_{x}} + \commut{zp_{y}}{xp_{z}} \\
	                      &= y\commut{p_{z}}{zp_{x}} + x\commut{zp_{y}}{p_{z}} \\
	                      &= y(z\commut{p_{z}}{p_{x}} + \commut{p_{z}}{p_{x}}z) + x(p_{y}\commut{z}{p_{z}} + \commut{z}{p_{z}}p_{y}) \\
	                      &= y(\commut{p_{z}}{z})p_{x} + x(\commut{z}{p_{z}})p_{y} \\
	                      &= y(-i\hbar)p_{x} + x(i\hbar)p_{y} \\
	                      &= i\hbar L_{z}.
\end{align*}
På samma sätt fås
\begin{align*}
	\commut{L_{z}}{L_{x}} = i\hbar L_{y}, \\
	\commut{L_{y}}{L_{z}} = i\hbar L_{x}, \\
	\commut{L^{2}}{\vb{L}} = \vb{0}.
\end{align*}
Vi ser att det inte är något speciellt med någon komponent, så vi väljer att konstruera de gemensamme egenfunktionerna till $L^{2}$ och $L_{z}$.

För att konstruera dessa egenfunktionerna, definierar vi stegoperatorerna
\begin{align*}
	L_{+} = L_{x} + iL_{y},\ L_{-} = \adj{L_{+}} = L_{x} - iL_{y}.
\end{align*}
Dessa är inte Hermiteska. Kommutationsrelationerna är
\begin{align*}
	\commut{L^{2}}{L_{\pm}} &= 0, \\
	\commut{L_{z}}{L_{\pm}} &= \pm\hbar L_{z}.
\end{align*}
Produkten av de två operatorerna är
\begin{align*}
	L_{+}L_{-} = L_{x}^{2} + L_{y}^{2} - i\commut{L_{x}}{L_{y}} = L^{2} - L_{z}^{2} + \hbar L_{z}, \\
	L_{-}L_{+} = L^{2} - L_{z}^{2} - \hbar L_{z}.
\end{align*}
Egenfunktionerna uppfyller
\begin{align*}
	L^{2}L_{+}\ket{\Psi} &= L_{+}L^{2}\ket{\Psi} = \hbar^{2}l(l + 1)L_{+}\ket{\Psi}, \\
	L_{z}L_{+}\ket{\Psi} &= L_{+}L_{z}\ket{\Psi} + \hbar L_{+}\ket{\Psi} = \hbar(m + 1)L_{+}\ket{\Psi}, \\
	L^{2}L_{-}\ket{\Psi} &= L_{-}L^{2}\ket{\Psi} = \hbar^{2}l(l + 1)L_{-}\ket{\Psi}, \\
	L_{z}L_{-}\ket{\Psi} &= L_{-}L_{z}\ket{\Psi} - \hbar L_{-}\ket{\Psi} = \hbar(m - 1)L_{-}\ket{\Psi}.
\end{align*}
Vi ser alltså att stegoperatorerna skapar nya egentillstånd med olika väntevärden för $L_{z}$. Däremot vet vi att väntevärdena av $L^{2}$ och $L_{z}^{2}$ är strikt positiva, och $L^{2} \leq L_{z}^{2}$, så följden måste terminera någon gång. Med andra ord finns det två tillstånd så att
\begin{align*}
	L_{+}\ket{m_{\text{max}}} = 0,\ L_{-}\ket{m_{\text{min}}} = 0.
\end{align*}

Vi får vidare
\begin{align*}
	L^{2}\ket{m_{\text{max}}} &= (L_{-}L_{+} + L_{z}^{2} + \hbar L_{z})\ket{m_{\text{max}}} \\
	                          &= (0 + m^{2}\hbar^{2} + \hbar^{2}m_{\text{max}})\ket{m_{\text{max}}},
\end{align*}
vilket implicerar
\begin{align*}
	l(l + 1) = m^{2}\hbar^{2} + \hbar^{2}m_{\text{max}},
\end{align*}
med lösning
\begin{align*}
	m_{\text{max}} = l.
\end{align*}
På samma sätt fås
\begin{align*}
	m_{\text{min}} = -l.
\end{align*}
Vi noterar nu att det finns $2l$ steg mellan det maximala och minimala värdet. Eftersom antal steg även måste vara ett heltal, så kan $l$ vara alla multipler av $\frac{1}{2}$.

För att få normering, betraktar vi
\begin{align*}
	L_{\pm}\ket{m} = A_{\pm}\ket{m\pm 1}.
\end{align*}
Dens inreprodukt med sig själv är
\begin{align*}
	\braket{L_{-}L_{+}}{m} = \abs{A_{+}}^{2}\braket{m}
\end{align*}
å ena sidan och
\begin{align*}
	\braket{L^{2} - L_{z}^{2} - \hbar L_{z}}{m} = \hbar^{2}(l(l + 1) - m^{2} - m)\braket{m},
\end{align*}
vilket ger
\begin{align*}
	A_{+} = \sqrt{l(l + 1)- m(m + 1)}\hbar
\end{align*}
och på samma sätt
\begin{align*}
	A_{+} = \sqrt{l(l + 1)- m(m - 1)}\hbar.
\end{align*}

Osäkerhetsrelationen ger
\begin{align*}
	\Delta L_{x}\Delta L_{y} \geq \frac{1}{2}\abs{\expval{\commut{L_{x}}{L_{y}}}} = \frac{1}{2}\hbar\abs{\expval{L_{z}}} = \frac{1}{2}\abs{m}\hbar^{2}.
\end{align*}
Alltså är inte tillstånden vi har hittat egenfunktioner till $L_{x}$ eller $L_{y}$ om inte $m = 0$.

\paragraph{Partiklar med spinn i magnetfält}
En laddning $q$ med den givna spinnen får det magnetiska momentet $\vb{\mu} = \gamma\vb{S}$, där $\gamma = \frac{gq}{2m}$ och $g$ är en gyroskopisk faktor. För elektroner är den ungefär $2$ och för protoner är den ungefär $5$.

Välj nu koordinatsystem så att $\vb{B}$ pekar i $z$-riktningen. Hamiltonianen ges då av
\begin{align*}
	H = -\vb{\mu}\cdot\vb{B} = -\frac{gqB}{2m}S_{z} = \omega_{0}S_{z} = \frac{1}{2}\hbar\omega_{0}
	\mqty[
		1 & 0 \\
		0 & -1
	].
\end{align*}
Dens egenvärden är $E_{m}\hbar\omega_{0}$.

\paragraph{Schrödingerekvatinen för en partikel i ett magnetfält}
Den tidsberoende Schrödingerekvationen för en partikel i ett magnetfält blir
\begin{align*}
	i\hbar\dv{\chi}{t} = H\chi.
\end{align*}
En allmän spinor för en $s = \frac{1}{2}$-partikel ges då av
\begin{align*}
	\chi = a\chi_{+}e^{-i\frac{E_{+}t}{\hbar}} + b\chi_{-}e^{-i\frac{E_{-}t}{\hbar}}.
\end{align*}
Normeringsvillkoret för såna tillstånd uppmanar oss att skriva $a = \cos{\frac{\theta}{2}}$ och $b = \sin{\frac{\theta}{2}}$. Vi får med detta
\begin{align*}
	\expval{S_{x}} &= \frac{1}{2}\hbar\sin{\theta}\cos{\omega_{0}t}, \\
	\expval{S_{y}} &= \frac{1}{2}\hbar\sin{\theta}\sin{\omega_{0}t}, \\
	\expval{S_{z}} &= \frac{1}{2}\hbar\cos{\theta}.
\end{align*}
Alltså precesserar spinnet kring $z$-axeln med frekvens $\omega_{0}$ i ett plan som ges av initialtillståndet.

\section{Identiska partiklar}

\paragraph{Kvanttillstånd för särskiljbara partiklar}
Om vi har två särskiljbara partiklar, kan tillståndet skrivas som en produkt, dvs. $\ket{\Psi} = \ket{a}\ket{b}$. Bakom scenen är detta en tensorprodukt, men det skriver vi typiskt inte explicit.

\paragraph{Tillstånd för identiska partiklar}
För identiska partiklar finns det inte exakta banor som det gör i klassisk fysik. Därför går det inte att avgöra vilken partikel som är vilken om de inte är separerade. Därmed kan man inte använda produkttillstånd. Det behövs därför tillstånd som respekterar byte av partiklar.

\paragraph{Utbytesoperatorn}
Vi definierar utbytesoperatorn
\begin{align*}
	P\Psi(\vb{r}_{1}, \vb{r}_{2}) = P\Psi_{12} = \Psi_{21}.
\end{align*}
För lika tillstånd kräver vi $P\Psi_{12} = c\Psi_{12}$. Vid att använda utbytesoperatorn två gånger fås kravet $\abs{c} = 1$, med lösningar $c = \pm 1$. Första fallet motsvarar symmetri under utbyte, och andra fallet motsvarar antisymmetri.

\paragraph{Tillstånd för identiska partiklar med produkttillstånd}
Det visar sig att vi kan konstruera tillåtna tillstånd för identiska partiklar med produkttillstånd ändå. Vafan. Det gör vi så här:
\begin{align*}
	\ket{\Psi} = \frac{1}{\sqrt{2}}(\ket{a}\ket{b} \pm \ket{b}\ket{a}).
\end{align*}

\paragraph{Symmetriseringspostulatet}
Identiska partiklar med heltaliga spinn har symmetriska tillstånd under utbyte. Såna kallas för bosoner.

Identiska partiklar med halvtaliga spinn har antisymmetriska tillstånd under utbyte. Såna kallas för fermioner.

\paragraph{Utbyteskrafter}
För särskiljbara partiklar kan vi betrakta väntevärdet
\begin{align*}
	\expval{(x_{1} - x_{2})^{2}} &= \bra{a}\bra{b}(x_{1}^{2} + x_{2}^{2} - 2x_{1}x_{2})\ket{a}\ket{b} \\
	                             &= \expval{x_{1}^{2}}{a} + \expval{x_{2}^{2}}{b} - 2\expval{x_{1}}{a}\expval{x_{2}}{b} \\
	                             &= \expval{x^{2}}{a} + \expval{x^{2}}{b} - 2\expval{x}{a}\expval{x}{b}.
\end{align*}
För identiska partiklar fås
\begin{align*}
	\expval{(x_{1} - x_{2})^{2}} &= \frac{1}{2}(\bra{a}\bra{b} \pm \bra{b}\bra{a})(x_{1}^{2} + x_{2}^{2} - 2x_{1}x_{2})(\ket{a}\ket{b} \pm \ket{b}\ket{a}) \\
	                             &= \expval{x^{2}}{a} + \expval{x^{2}}{b} - 2\expval{x}{a}\expval{x}{b} \pm \bra{a}\bra{b}(x_{1}^{2} + x_{2}^{2} - 2x_{1}x_{2})\ket{b}\ket{a} \\
	                             &= \expval{x^{2}}{a} + \expval{x^{2}}{b} - 2\expval{x}{a}\expval{x}{b} \mp 2\bra{a}\bra{b}x_{1}x_{2}\ket{b}\ket{a} \\
	                             &= \expval{x^{2}}{a} + \expval{x^{2}}{b} - 2\expval{x}{a}\expval{x}{b} \mp 2\bra{a}x_{1}\ket{b}\bra{b}x_{2}\ket{a} \\
	                             &= \expval{x^{2}}{a} + \expval{x^{2}}{b} - 2\expval{x}{a}\expval{x}{b} \mp 2\abs{\mel{a}{x}{b}}^{2}.
\end{align*}
Vi har utnyttjar det faktum att medelkvadratavståndet är en observabel för att förenkla lite.

\section{Störningsteori}

\paragraph{Störningsräkningarnas principer}
Vi vill lösa problemet $H\Psi_{n} = E_{n}\Psi_{n}$. Detta görs på följande sätt:
\begin{itemize}
	\item Antag att vi har ett löst problem $H^{0}\Psi_{n}^{0} = E_{n}^{0}\Psi_{n}^{0}$.
	\item Inför den störda hamiltonianen $H = H^{0} + V$, vars egenfunktioner vi vill hitta.
	\item Inför styrkeparametern $\lambda$ genom $H = H^{0} + \lambda V$.
	\item Sök en potensserielösning $\Psi = \Psi_{n}^{0} + \lambda\Psi_{n}^{1} + \lambda^{2}\Psi_{n}^{2}+ \dots$. Denna har energi $E_{n} = E_{n}^{0} + \lambda E_{n}^{1} + \lambda^{2}E_{n}^{2} + \dots$. De olika termerna är koorektioner av olik ordning, och superskriptet indikerar vad ordningen är för varje term.
\end{itemize}

Störningarna av olik ordning fås vid att identifiera termer från Schrödingerekvationen
\begin{align*}
	(H^{0} + \lambda V)(\Psi_{n}^{0} + \lambda\Psi_{n}^{1} + \lambda^{2}\Psi_{n}^{2}+ \dots) = (E_{n}^{0} + \lambda E_{n}^{1} + \lambda^{2}E_{n}^{2} + \dots)(\Psi_{n}^{0} + \lambda\Psi_{n}^{1} + \lambda^{2}\Psi_{n}^{2}+ \dots).
\end{align*}
Till exempel ges första ordningens störning av
\begin{align*}
	H^{0}\Psi_{n}^{1} + V\Psi_{n}^{0} = E_{n}^{0}\Psi_{n}^{1} + E_{n}^{1}\Psi_{n}^{0}
\end{align*}
Inreprodukten med $\Psi_{n}^{0}$ på båda sidor ger
\begin{align*}
	\mel{\Psi_{n}^{0}}{H^{0}}{\Psi_{n}^{1}} + \mel{\Psi_{n}^{0}}{V}{\Psi_{n}^{0}} = E_{n}^{0}\braket{\Psi_{n}^{0}}{\Psi_{n}^{1}} + E_{n}^{1}\braket{\Psi_{n}^{0}}{\Psi_{n}^{0}}.
\end{align*}
Eftersom hamiltonoperatorn är Hermitesk, kommer första term på båda sidor vara lika, vilket ger
\begin{align*}
	 E_{n}^{1} = \mel{\Psi_{n}^{0}}{V}{\Psi_{n}^{0}}.
\end{align*}
Vi kan även visa att
\begin{align*}
	\Psi_{n}^{1} = \sum\limits_{m \neq n}\frac{\mel{\Psi_{m}^{0}}{V}{\Psi_{n}^{0}}}{E_{n}^{0} - E_{m}^{0}}\Psi_{m}^{0},
\end{align*}
som ökar i amplitud med $V$. Det gäller även att
\begin{align*}
	E_{n}^{2} = \sum\limits_{m \neq n}\frac{\abs{\mel{\Psi_{m}^{0}}{V}{\Psi_{n}^{0}}}^{2}}{E_{n}^{0} - E_{m}^{0}},
\end{align*}
som ökar i styrka med $V^{2}$. Observera att nämnaren i dessa uttryck gör det helt klart att denna sortens räkning endast gäller för icke-degenererade energier.

\paragraph{Degenererade energier}
För att få en indikation på hur det går till, betrakta ett tvåfalt degenererad egentillstånd med två egentillstånd $\Psi_{a}^{0}$ och $\Psi_{b}^{0}$. Dessa kan väljas ortogonala. För att få störningen av första ordningen, gör vi ansatsen $\Psi = \alpha\Psi_{a}^{0} + \beta\Psi_{b}^{0} + \Psi^{1}$. Schrödingerekvationen ger enligt samma argument som ovan att
\begin{align*}
	H^{0}\Psi^{1} + V(\alpha\Psi_{a}^{0} + \beta\Psi_{b}^{0}) = E^{0}\Psi^{1} + E^{1}(\alpha\Psi_{a}^{0} + \beta\Psi_{b}^{0}).
\end{align*}
Inreprodukten med $\Psi_{a}^{0}$ ger igen
\begin{align*}
	\mel{\Psi_{a}^{0}}{H^{0}}{\Psi^{1}} + \alpha\mel{\Psi_{a}^{0}}{V}{\Psi_{a}^{0}} + \beta\mel{\Psi_{a}^{0}}{V}{\Psi_{b}^{0}} = E_{n}^{0}\braket{\Psi_{a}^{0}}{\Psi^{1}} + E^{0}(\alpha\braket{\Psi_{a}^{0}}{\Psi_{a}^{0}} + \beta\braket{\Psi_{a}^{0}}{\Psi_{b}^{0}}),
\end{align*}
där första termen på varje sida igen är lika. Vid att definiera $W_{ab} = \mel{\Psi_{a}^{0}}{V}{\Psi_{a}^{0}}$ och göra motsvarande fast för $\Psi_{b}^{0}$ fås ekvationssystemet
\begin{align*}
	\alpha W_{aa} + \beta W_{ab} = E^{1}\alpha,\ \alpha W_{ba} + \beta W_{bb} = E^{1}\beta,
\end{align*}
som är en egenvärdesekvation i energin. Vid att välja basen på ett sånt sätt att $W_{ab} = W_{ba} = 0$ kan de möjliga egenvärdena läsas av direkt.

\paragraph{Störningar i väte}
Vi vill introducera störningar i väteatomens energier för att se vad som händer med energierna. Vid att jämföra detta med experiment kan man testa kvantmekanikens giltighet.

\paragraph{Störning med elektriskt fält}
Vi lägger först på ett elektriskt fält i $z$-riktning, med en potential på formen $V = e\mathcal{E}z_{\text{p}} - e\mathcal{E}z_{\text{e}}$ till totala systemet. Vi betraktar väteatomen i masscentrumssystemet, och detta kan då skrivas om i masscentrumssystemet som $V = e\mathcal{E}z$. Alternativt kan vi införa dipolmomentet $\vb{p} = -e\vb{r}_{e} + e\vb{r}_{p} = -e\vb{r}$, och skriva $V = -\vb{p}\cdot\vb{E}$. Första ordningens energikorrektion till grundtillståndet blir
\begin{align*}
	E_{100}^{1} = \expval{e\mathcal{E}z}{100} = 0,
\end{align*}
då grundtillståndet är sfäriskt symmetriskt.

Andra ordningens korrektion blir
\begin{align*}
	E_{100}^{2} = e^{2}\mathcal{E}^{2}\sum\limits_{nlm \neq 100}\frac{\abs{\mel{nlm}{z}{100}}^{2}}{E_{1}^{0} - E_{n}^{0}} = -\frac{9}{4}4\pi\varepsilon_{0}a^{3}\mathcal{E}^{2}.
\end{align*}
Att göra denna beräkningen analytiskt kan göras, men vi vill i stället göra en uppskattning vid att låta alla $E_{n}^{0}$ vara $E_{2}^{0}$ för $n > 2$. Vi kan då ta ut energifaktorn från summationen ovan och få
\begin{align*}
	E_{100}^{2} = \frac{e^{2}\mathcal{E}^{2}}{E_{1}^{0} - E_{2}^{0}}\sum\limits_{nlm \neq 100}\abs{\mel{nlm}{z}{100}}^{2}.
\end{align*}
Enligt argumentet från första ordningen kan vi utvidga summan till
\begin{align*}
	E_{100}^{2} = \frac{e^{2}\mathcal{E}^{2}}{E_{1}^{0} - E_{2}^{0}}\sum\abs{\mel{nlm}{z}{100}}^{2} = \frac{e^{2}\mathcal{E}^{2}}{E_{1}^{0} - E_{2}^{0}}\sum\mel{100}{z}{nlm}\mel{nlm}{z}{100}.
\end{align*}
Detta kan förenklas med hjälp av identitetsoperatorn till
\begin{align*}
	E_{100}^{2} = \frac{e^{2}\mathcal{E}^{2}}{E_{1}^{0} - E_{2}^{0}}\expval{z^{2}}{100} = -\frac{8}{3}4\pi\varepsilon_{0}a^{3}\mathcal{E}^{2}.
\end{align*}

De andra energitillstånden är degenererade med degenerationsgrad $2n^{2}$. Vi vill nu välja en god bas så att störningsmatrisen blir diagonal. Då blir diagonalelementen i störningsmatrisen korrektionerna av första ordning. Vi kan utnyttja att om $A$ är Hermitesk med icke-degenererade egenvärden och störningen $V$ kommuterar med både $A$ och $H^{0}$, är $V$ diagonal i egenbasen till $A$.

\paragraph{Relativistisk korrektion}
Den relativistiska kinetiska energin ges av
\begin{align*}
	\sqrt{p^{2}c^{2} + (mc^{2})^{2}} - mc^{2},
\end{align*}
med en första korrektionsterm
\begin{align*}
	V = -\frac{p^{4}}{8m^{3}c^{2}}.
\end{align*}
Denna kommuterar med den klassiska hamiltonoperatorn, och vi kan därför använda icke-degenererad störningsräkning. Man kan visa att
\begin{align*}
	E_{n}^{1} = -\frac{(E_{n}^{0})^{2}}{2mc^{2}}\left(\frac{4n}{l + \frac{1}{2}} - 3\right),
\end{align*}
och
\begin{align*}
	\frac{E_{n}^{1}}{E_{n}^{0}}\propto E_{n}^{1}\propto\alpha^{2},
\end{align*}
där vi har infört finstrukturkonstanten
\begin{align*}
	\alpha = \frac{e^{2}}{4\pi\varepsilon_{0}\hbar c}.
\end{align*}

\paragraph{Spinn-bankoppling}
Vi betraktar nu väteatomen i elektronens vilosystem. Då snurrar protonen kring elektronen, vilket ger upphov till en ström, som igen skapar ett magnetfält
\begin{align*}
	\vb{B} = \mu_{0}\frac{I}{2r}\vb{e}_{z} = \frac{e}{4\pi\varepsilon_{0}mc^{2}r^{3}}\vb{L}.
\end{align*}
Detta ger då en energikorrektion
\begin{align*}
	V = \frac{e^{2}}{4\pi\varepsilon_{0}m^{2}c^{2}r^{3}}\vb{L}\cdot\vb{S}.
\end{align*}
Störningarna som beräknades från detta var fel med en faktor $2$. Detta kom av att beräkningarna inte tog med i betraktningen att elektronens vilosystem inte är ett inertialsystem. Det faktiska energibidraget skall vara
\begin{align*}
	V = \frac{e^{2}}{8\pi\varepsilon_{0}m^{2}c^{2}r^{3}}\vb{L}\cdot\vb{S}.
\end{align*}
Vi får
\begin{align*}
	E_{n}^{1} &= \frac{e ^{2}}{8\pi\varepsilon_{0}}\frac{1}{m^{2}c^{2}}\frac{1}{2}\expval{J^{2} - L^{2} - S^{2}}\expval{\frac{1}{r^{3}}} \\
	          &= \frac{(E_{n}^{0})^{2}}{mc^{2}}\frac{n(j(j + 1) - l(l + 1) - \frac{3}{4})}{l(l + \frac{1}{2})(l + 1)},
\end{align*}
där vi har användt att alla ingående operatorer opererar på olika koordinater, varför vi kan splitta upp väntevärdena på detta sättet.

\paragraph{Hyperfinstruktur}
Koppling mellan elektronens och protonens magnetiska moment ger en korrektionsterm
\begin{align*}
	E_{100}^{1} = \frac{\mu_{0}g_{p}e^{2}}{3m_{e}m_{p}}\expval{\vb{S}_{p}\cdot\vb{S}_{e}}\abs{\eval{\Psi_{100}}_{\vb{r} = \vb{0}}}^{2}.
\end{align*}
Detta ger en energisplittring lika med energiskillnaden mellan en spinntriplett och en spinnsinglett.

\paragraph{Zeeman-effekt}
Zeeman-effekten är energiskift hos elektroner i ett homogent magnetiskt fält. Störningstermen är
\begin{align*}
	V = -(\vb{\mu}_{l} + \vb{\mu}_{s})\cdot\vb{B} = \frac{e}{2m}\vb{B}\cdot(\vb{L} + 2\vb{S}) = \frac{e}{2m}\vb{B}\cdot(\vb{J} + \vb{S}).
\end{align*}

\subparagraph{Svaga fält}
Här görs endast korrektion av finstrukturenergier. Eftersom $S_{z}$ är osäker i den kopplade basen, fås dock en icke-diagonal störningsmatris. Tidsmedelvärdet av $\vb{S}$ kan dock användas. Man får
\begin{align*}
	E^{1} = \frac{e}{2m}B\left(1 + \frac{j(j + 1) + s(s + 1) - l(l + 1)}{2j(j + 1)}\right)m_{j}\hbar.
\end{align*}

\subparagraph{Starka fält}
För starka fält är Zeeman-effekten dominant, och man kan göra räkningen i produktbasen. Man får
\begin{align*}
	E^{1} = \mu_{B}B(m_{l} + 2m_{s}).
\end{align*}

\subparagraph{Mellanstarka fält}
I detta fallet finns helt enkelt ingen bra bas. Det är dock inte mycket nytt som händer här.

\paragraph{Tidsberoende störningar}
För att göra tidsberoende störningar, utgår vi från ett känt ostört problem $H^{0}\ket{n} = E_{n}\ket{n}$. Vi vill nu titta på en störning $H = H^{0} + \lambda V(t)$. Om systemet startar i tillstånd $\ket{a}$, får vi
\begin{align*}
	\ket{\Psi} = c_{n}\ket{n},
\end{align*}
där
\begin{align*}
	c_{n} = \frac{1}{i\hbar}\integ{0}{t}{t'}{e^{i\omega_{0}t'}\mel{n}{V(t')}{a}},\ n \neq a,\ \omega_{0} = \frac{E_{n} - E_{a}}{\hbar}.
\end{align*}
Vi kan även beräkna övergångssannolikheten
\begin{align*}
	p_{a\to n} = \frac{1}{\hbar^{2}}\abs{\integ{0}{t}{t'}{e^{i\omega_{0}t'}\mel{n}{V(t')}{a}}}^{2}.
\end{align*}

\paragraph{Oscillerande störning}
Ett viktigt fall är en oscillerande störning
\begin{align*}
	V = V(\vb{r})\cos{\omega t}.
\end{align*}
De nya koefficienterna ges av
\begin{align*}
	c_{n} &= \frac{1}{2i\hbar}\mel{n}{V(\vb{r})}{a}\integ{0}{t}{t'}{e^{i\omega_{0}t'}(e^{i\omega t'} + e^{-i\omega t'})} \\
	      &= \frac{V_{ba}}{i\hbar}\left(e^{i\frac{\omega + \omega_{0}}{2}}\frac{\sin{(\omega + \omega_{0})t}}{\omega_{0} + \omega} + e^{i\frac{\omega_{0} - \omega}{2}}\frac{\sin{(\omega_{0} - \omega)t}}{\omega_{0} - \omega}\right).
\end{align*}
Denna har resonans för $\omega = \pm\omega_{0}$, vilket motsvarar en övergång $a\to n$ eller $n\to a$.

\paragraph{Emission och absorption av strålning}
Elektroner växelverkar i huvudsak med det elektriska fältet i en elektromagnetisk våg. Betrakta därför ett infallande monokromatiskt elektriskt fält polariserat i $z$-riktningen. Det ger upphov till en potential
\begin{align*}
	V = eE_{0}z\cos{\omega t}.
\end{align*}
Vi får
\begin{align*}
	V_{ba} = eE_{0}\mel{b}{z}{a} = -PE_{0},
\end{align*}
där $P$ är polarisationens matriselement mellan $a$ och $b$.

\paragraph{Inkoherent strålning}
Antag att man har en frekvensfördelning $\rho$ så att den termiska energin ges av
\begin{align*}
	\dd{U} = \rho\dd{\omega}.
\end{align*}
Inkoherent strålning är strålning där det inte finns interferens mellan olika moder. För såna fall kan man beräkna en övergångshastighet
\begin{align*}
	R_{ab} = \frac{\pi}{3\varepsilon_{0}\hbar^{2}}\abs{P}^{2}\rho^{2}.
\end{align*}
Detta är Fermis gyllene regel. Koefficienten
\begin{align*}
	\frac{\pi}{3\varepsilon_{0}\hbar^{2}}\abs{P}^{2}
\end{align*}
är Einsteins $B$-koefficient.

\paragraph{•}
Antag att vi har $N_{a}$ atomer i $a$ och $N_{b}$ atomer i $b$. Vi vill gärna veta hastigheten för spontan emission. En detaljerad balans ger
\begin{align*}
	\dv{N_{b}}{t} = -N_{a}A - N_{b}B_{ba}\rho(\omega_{0}) + N_{a}B_{ab}\rho(\omega_{0}).
\end{align*}
Maxwell-Boltzmann-statistik ger
\begin{align*}
	\frac{N_{a}}{N_{b}} = e^{\beta\hbar\omega},
\end{align*}
vilket implicerar
\begin{align*}
	\rho(\omega_{0}) = \frac{A}{e^{\beta\hbar\omega}B_{ab} - B_{ba}}.
\end{align*}
Detta skall vara lika med Plancks fördelning
\begin{align*}
	\frac{\hbar}{\pi^{2}c^{3}}\frac{\omega^{3}}{e^{\beta\hbar\omega - 1}},
\end{align*}
vilket implicerar
\begin{align*}
	A = \frac{\omega_{0}^{3}\abs{P}^{2}}{3\pi\varepsilon_{0}\hbar c^{3}}.
\end{align*}

\end{document}
