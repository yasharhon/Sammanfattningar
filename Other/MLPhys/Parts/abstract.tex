This is a summary of FK7068 Machine Learning for Physicists and Astronomers, a course taught at Stockholm's university.

If you are reading this, you are probably a student at KTH and wondering why the author decided to take this course at all when KTH also offers an introductory course in machine learning. The answer is that at the time of writing, the introductory machine learning course offered at the computer science department has a poor reputation, being known for only superficially introducing concepts and giving little hands-on experience. Courses in applied mathematics could be taken as an alternative route to machine learning competence, but there are several such courses stretched out over different periods. FK7068 is a 7.5 credit course introducing machine learning in a way suited to physicists and the mathematical background we possess. It also features a project, giving some hands-on experience. Lastly, it is entirely confined to period 2, making it fit very well with the schedules of theoretical physics students who opt to focus on high-energy physics. The course is highly recommended.