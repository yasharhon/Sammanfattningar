\section{Basic Concepts}

\paragraph{Machine Learning}
Machine learning is the field of study that gives computers the ability to learn without explicit programming. A computer is said to learn from some experience $E$ with respekt to a task $T$ and a performance measure $P$ if its performance at $T$ as measured by $P$ increases with $E$.

\paragraph{Supervised Learning}
Supervised learning deals with labelled data, and the task is to correctly label unfamiliar data. The two main types of supervised learning problems are classification and regression.

\paragraph{Unsupervised Learning}
Unsupervised learning, bu contrast, deals with data that does not have labels. The task is generally to identify some patterns within the data. Examples of such problems include clustering, anomaly detection, pattern recognition and association mining.

\paragraph{Reinforcement Learning}
Reinforcement learning is, loosely speaking, based on performing random explorations of prediction processes and learning from the output.

\paragraph{A Note on Hardware}
Machine learning benefits strongly from parallelization. This is one of the reasons why, despite CPUs having high clock speeds, GPUs are the preferred hardware for training algorithms.

\paragraph{Using Machine Learning}
Machine learning is notoriously full of pitfalls. Great care must be taken in the preparatory work and the process of creating and training the model. The general steps in the procedure are
\begin{enumerate}
	\item \textbf{Define the problem and plan.} Formulate good questions and assess whether a machine learning approach is adequate and necessary.
	\item \textbf{Estimate the required computational resources.} Combined with economic aspects, this will determine the limits of the complexity of your model.
	\item \textbf{Prepare data.} This step is crucial and will include identifying sources, data collection and preprocessing. Relevant questions to ask include whether you have enough data (a quintessential point) and whether the data is sufficiently diverse.
	\item \textbf{Construct an appropriate model.}
	\item \textbf{Test the model.}
	\item \textbf{Deploy.}
\end{enumerate}

See also \href{https://karpathy.github.io/2019/04/25/recipe/}{this article} for a well-known and detailed example of how one should think.

\paragraph{Statistical Learning}
Statistical learning is a framework in which the tools of statistics are used to describe and evaluate a machine learning algorithm.

A useful starting point is the limit of large numbers, for which we can us the Markov density approximation
\begin{align*}
	P(X = x) \approx \frac{1}{N}\sum\limits_{l}\delta(x - x_{l}),
\end{align*}
which implies
\begin{align*}
	\expval{f(X)} \approx \frac{1}{N}\sum\limits_{l}f(x_{l}).
\end{align*}
This holds for identically distributed, independent realisation $x_{l}$ of $P(X = x)$.

Given this, the input to the learner is the domain set $X$, points in which are numerical representations of the data, the label space $Y$, which contains the possible labels that can be ascribed to a point in $X$, and a set $S$ of pairs $(x, y)\in X\times Y$, which is a realization of the joint probability distribution $P(X = x, Y = y) = f(x, y)$. The output is a prediction rule $h: X\to Y$.

We will need some way to assess the quality of $h$. This is done by introducing a loss function $L(h(x), y)$, which measures the ability of $h$ to predict $y$ given $x$. Next, we introduce the risk functional
\begin{align*}
	R(h) = \inte{}{}\dd{x}\dd{y}f(x, y)L(h(x), y) = \inte{}{}\dd{x}P(X = x)\inte{}{}\dd{y}P(Y = y\given X = x)L(h(x), y),
\end{align*}
which measures the expected loss. Assuming a good choice of the loss function, the best choice of $h$ is taken to be the one that minimizes the risk. Such a predictor is called Bayes optimal.

We take $f$ to be unknown, but assuming the training data to be sufficiently large we can construct a Markov density approximation for $f$. We then find
\begin{align*}
	R(h)\approx R_{S}(h) = \frac{1}{\abs{S}}\sum\limits_{i}\inte{}{}\dd{x}\dd{y}\delta(x - x_{i})\delta(y - y_{i})L(h(x), y) = \frac{1}{\abs{S}}\sum\limits_{i}L(h(x_{i}), y_{i}),
\end{align*}
which is the so-called empirical risk. We can now approximately identify an optimal choice of $h$ by minimizing this quantity. This process is called empirical risk minimization. The assumption that the class of hypotheses has a member that minimizes $R_{S}$ is the realizability assumption.

\paragraph{Probability Terms of Relevance}
A few terms of relevance are
\begin{itemize}
	\item accuracy, which is the fraction of predictions that is correct.
	\item precision, which is the fraction of positive predictions that corresponds to true positives.
	\item sensitivity, which is the fraction of true positive cases that is identified.
	\item specificity, which is the fraction of negative predictions that corresponds to true negatives.
\end{itemize}

\paragraph{Probably Approximately Correct Learning}
How wrong is the empirical approximation? It is clear that $R_{S}$ approaches $R$ in the limit of infinite sample sizes. For finite sample sizes, however, we can only make probabilistic statements. We introduce probably approximately correct (PAC) learning as follows: A hypothesis class $H$ is PAC learnable if there exists a number $m_{H}$ depending on two probabilities such that for every pair of probabilities $\epsilon, \delta$ and under the realizability assumption, when running the learning algorithm with $m\geq m_{H}(\epsilon, \delta)$ iid samples, the algorithm returns a hypothesis that satisfies
\begin{align*}
	P(\abs{R(h) - R_{s}(h)} > \epsilon) < \delta.
\end{align*}
It turns out that in the case of a finite hypothesis class we have
\begin{align*}
	m_{H}(\epsilon, \delta) = \frac{1}{\epsilon}\left(\ln(\abs{H}) + \ln(\frac{1}{\delta})\right).
\end{align*}
Relaxing the realizability assumption in the so-called agnostic case we instead have
\begin{align*}
	m_{H}(\epsilon, \delta) = \frac{1}{2\epsilon^{2}}\left(\ln(\abs{H}) + \ln(\frac{2}{\delta})\right).
\end{align*}

\paragraph{The Confusion Matrix}
The confusion matrix is a $2\times 2$ matrix for which the diagonal elements contain the true positives and negatives, the upper right contains the false positives and the lower left contains the false negatives.