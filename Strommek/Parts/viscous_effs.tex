\section{Viskösa effekter}

\paragraph{Bildningen av gränsskikt}
Vidhäftningsvillkoret gör att nära ytor ändras hastigheten extremt snabbt i ett tunt skikt nära ytan, det så kallade gränsskiktet. Vi kommer studera beteendet i och utanför såna gränsskikt i fall som är symmetriska i $z$-riktning.

\paragraph{Formulering av problem}
Vi kommer betrakta strömning av en fluid i $x$-riktning med friströmshastighet $U$. Fluiden strömmar förbi en platta som är parallell med $x$-riktningen och har längd $L$. Över ytan finns ett gränsskikt med tjocklek $\delta$

\paragraph{Reynoldstalet}
Navier-Stokes' ekvation ger
\begin{align*}
	u_{x}\del{x}{u_{x}} + u_{y}\del{y}{u_{x}} = -\frac{1}{\rho}\del{x}{p} + \nu(\del[2]{x}{u_{x}} + \del[2]{y}{u_{x}}).
\end{align*}
Vi studerar först problemet i gränsskiktet. Här är viskösa krafter viktiga, så $y$-derivatan kommer vara stor här. Om vi av någon oklar anledning antar att $x$-derivatan är av ledande ordning på vänstersidan, blir detta
\begin{align*}
	u_{x}\del{x}{u_{x}} = \nu\del[2]{y}{u_{x}}.
\end{align*}
Detta låter oss göra en grov storleksuppskattning
\begin{align*}
	\frac{U^{2}}{L} = \nu\frac{U}{\delta^{2}},\ \frac{\delta}{L} = \frac{1}{\sqrt{\rey}},
\end{align*}
där $\rey = \frac{UL}{\nu}$ är Reynoldstalet. I de flesta tillämpningar är Reynoldstalet stort, så gränsskiktet är tunt.

\paragraph{Dimensionslösa variabler}
Vi vill nu förenkla Navier-Stokes' ekvationer för stora Reynoldstal. För att göra detta låter vi $U$ och $p_{\infty}$ vara hastigheten och trycket långt uppströms. Vi antar att $u_{x}$ är av storleksordning $U$ och att $\del{x}{p} = \rho u_{x}\del{x}{u_{x}}$. Detta ger storleksordningsuppskattningen
\begin{align*}
	p_{\infty} - p \approx \rho U^{2}.
\end{align*}
Vi kan även skatta vertikala hastigheten med hjälp av inkompressibilitetsvillkoret som
\begin{align*}
	u_{y} \approx \frac{\delta}{L}U = \frac{1}{\sqrt{\rey}}U.
\end{align*}
Detta motiverar oss att införa de dimensionslösa variablerna
\begin{align*}
	x' = \frac{x}{L},\ y' = \frac{y}{\delta} = \sqrt{\rey}\frac{y}{L},\ u_{x}' = \frac{u_{x}}{U},\ u_{y}' = \sqrt{\rey}\frac{u_{y}}{U},\ p' = \frac{p_{\infty} - p}{\rho U^{2}}.
\end{align*}
I termer av dessa variabler är Navier-Stokes' ekvationer
\begin{align*}
	u_{x}'\del{x'}{u_{x}'} + u_{y}'\del{y'}{u_{x}'}                 &= -\del{y'}{p'} + \frac{1}{\rey}\del[2]{x'}{u_{x}'} + \del[2]{y'}{u_{x}'}, \\
	\frac{1}{\rey}(u_{x}'\del{x'}{u_{y}'} + u_{y}'\del{y'}{u_{y}'}) &= -\del{y'}{p'} + \frac{1}{\rey^{2}}\del[2]{x'}{u_{x}'} + \frac{1}{\rey}\del[2]{y'}{u_{x}'}, \\
	\del{x'}{u_{x}'} + \del{y'}{u_{y}'} = 0.
\end{align*}
För stora Reynoldstal kan vi nu bortse från många termer här. I våra ursprungliga variabler fås då
\begin{align*}
	u_{x}\del{x}{u_{x}} + u_{y}\del{y}{u_{x}}                 &= -\del{y}{p} + \del[2]{y}{u_{x}}, \\
	0                                                         &= -\del{y}{p}, \\
	\del{x}{u_{x}} + \del{y}{u_{y}} = 0.
\end{align*}
Man kan tydligen även använda Bernoullis ekvation (fast inte) längsmed en strömlinje som följer gränsskiktets rand. Vi får då
\begin{align*}
	\frac{p}{\rho} + \frac{1}{2}U^{2} = c,\ -\frac{1}{\rho}\del{x}{p} = U\del{x}{U}.
\end{align*}
Nu har vi ställt upp alla ekvationer, och vi har randvillkoren
\begin{align*}
	u_{x}(x, 0) = u_{y}(x, 0) = 0,\ u_{x}(x, \infty) = U,
\end{align*}

\paragraph{Mått på tjocklek}
Härifrån måste vi välja någon definition av gränsskiktets tjocklek. Detta är tre olika sätt att göra det på.

\paragraph{Delta-nittinio}
Delta-nittinio-måttet definieras av
\begin{align*}
	u(x, \delta_{99}) = 0.99U.
\end{align*}

\paragraph{Förträngningstjockleken}
Förträngningstjockleken definieras som avståndet i vertikal riktning en strömlinje långt från väggen förflyttas på grund av väggen.

För att beräkna den observerar vi att masskonserveringen ger
\begin{align*}
	\integ{0}{h}{y}{U} = \integ{0}{h + \delta_{\star}}{y}{u_{x}},
\end{align*}
där vi gör andra integralen på ett (långt?) avstånd från där strömningen möter plattan. Eftersom vi pratar om strömlinjer långt från väggen, gör vi skattningen $u_{x} = U$ långt borta, vilket ger
\begin{align*}
	\delta_{\star} = \integ{0}{h}{y}{1 - \frac{u_{x}}{U}}.
\end{align*}

\paragraph{Rörelsemängdtjocklek}
Om tryckgradienten är noll fås för kontrollvolymen vi studerade ovan
\begin{align*}
	-f = -\integ{0}{h}{y}{\rho U^{2}} + \integ{0}{h + \delta_{\star}}{y}{\rho u_{x}^{2}}
\end{align*}
där
\begin{align*}
	f = \integ{0}{x}{x}{\mu\del{y}{u_{x}}(x, 0)}
\end{align*}
är friktionskraften i $x$-riktning på den delen av plattan som är innanför kontrollvolymen. Vi definierar nu rörelsemängdstjockleken
\begin{align*}
	\theta &= \frac{f}{\rho U^{2}} \\
	       &= \integ{0}{h}{y}{1 - \left(\frac{u_{x}}{U}\right)^{2}} + \frac{1}{U^{2}}\integ{h}{h + \delta_{\star}}{y}{u_{x}^{2}} \\
	       &\approx \integ{0}{h}{y}{1 - \left(\frac{u_{x}}{U}\right)^{2}} + \delta_{\star} \\
	       &= \integ{0}{h}{y}{\frac{u_{x}}{U}\left( 1 - \frac{u_{x}}{U}\right)}.
\end{align*}
Vi kan tydligen skriva detta som
\begin{align*}
	\theta = \integ{0}{\infty}{y}{\frac{u_{x}}{U}\left( 1 - \frac{u_{x}}{U}\right)}.
\end{align*}

\paragraph{Blasius' gränsskikt}
Blasius' lösning för gränsskiktet är en något annorlunda lösning av samma problemet vi har studerat. I detta fall är tryckgradienten lika med noll och strömningen möter plattan i origo. Inkompressibilitetsvillkoret låter oss införa strömfunktionen $\Psi$, och Navier-Stokes' ekvationer ger
\begin{align*}
	\del{y}{\Psi}\del{y}{\del{x}{\Psi}} - \del{x}{\Psi}\del[2]{y}{\Psi} = \nu\del[3]{y}{\Psi}.
\end{align*}
Eftersom systemet ej ändras under addition av en konstant till strömfunktionen, kan vi välja $\Psi(x, 0) = 0$. De övriga randvillkoren blir
\begin{align*}
	\del{y}{\Psi}(x, 0) = 0,\ \del{y}{\Psi}(x, y) \to U.
\end{align*}
Vi kommer lösa detta problemet med en likformighetsansats
\begin{align*}
	u_{x} = Uf'(\eta),\ \eta = \frac{y}{\delta(x)}.
\end{align*}
Detta ger
\begin{align*}
	\Psi = \integ{0}{y}{y}{u_{x}} = U\delta\integ{0}{\eta}{\eta}{f'(\eta)} = U\delta f(\eta).
\end{align*}
Insatt i Navier-Stokes' ekvation ger detta
\begin{align*}
	f''' = -\left(\frac{U\delta}{\nu}\dv{\delta}{x}\right)ff''.
\end{align*}
Om detta skall gälla, måste prefaktorn vara en konstant. Vi kan sätta den till $\frac{1}{2}$, vilket ger
\begin{align*}
	\delta = \sqrt{\frac{\nu x}{U}}.
\end{align*}
Att ändra den konstanten skulle bara motsvara en omskalning av enheterna, så vi ser att valet av konstant är godtyckligt. Den återstående ekvationen är
\begin{align*}
	\frac{1}{2}ff'' + f''' = 0,
\end{align*}
med randvillkor
\begin{align*}
	f(0) = 0,\ f'(0) = 0,\ f'(\eta) \to 1.
\end{align*}

Vi kan även beräkna väggskjuvspänningen
\begin{align*}
	\tau = \mu\del{y}{u_{x}}(x, 0) = \frac{0.332\rho U^{2}}{\sqrt{\rey_{x}}}
\end{align*}
där $\rey_{x}$ är Reynoldstalet beräknad med $x$. Det totala aerodynamiska moståndet på plattan ges då av
\begin{align*}
	F = \integ{0}{L}{x}{\tau} = \frac{0.664\rho U^{2}}{\sqrt{\rey_{L}}}.
\end{align*}
Vi kan även definiera en motståndskoefficient
\begin{align*}
	C = \frac{F}{\frac{1}{2}\rho U^{2}} = \frac{1.33}{\sqrt{\rey_{L}}}.
\end{align*}

\paragraph{Exempel: Blasius' gränsskikt med känd asymptot}
Betrakta viskös strömning över en platta med längd $L$. Vätskan har friströmshastighet $U$, täthet $\rho$ och kinematisk viskositet $\nu$. De fyra punkterna $A$, $B$, $A'$ och $B'$ definieras av koordinaterna $(0, 0)$, $(L, 0)$, $(0, h)$ och $(L, h)$, där $h$ är ett avstand som är mycket större än gränsskiktets tjocklek. För stora $\eta$ gäller att $f(\eta) \approx \eta - 1.721$.

Vi söker först skillnaden i massflöde per längd i $z$ genom $AA'$ och $BB'$. Första ges av
\begin{align*}
	\Phi = \integ{0}{h}{y}{\rho u_{x}} = \integ{0}{h}{y}{\rho U},
\end{align*}
och andra ges av
\begin{align*}
	\Phi = \integ{0}{h}{y}{\rho u_{x}} = \integ{0}{h}{y}{\rho U\dv{f}{\eta}},
\end{align*}
där vi har använt likformighetslösningen. Skillnaden ges av
\begin{align*}
	\Delta\Phi &= \integ{0}{h}{y}{\rho U\left(1 - \frac{u_{x}}{U}\right)} \\
	           &= \rho\delta U\integ{0}{\frac{h}{\delta}}{\eta}{1 - \dv{f}{\eta}} \\
	           &= \rho\delta U\left[\eta - f(\eta)\right]_{0}^{\frac{h}{\delta}}.
\end{align*}
Unre gränsen ger inget bidrag, och i övre gränsen kan asymptotiska lösningen användas, vilket ger
\begin{align*}
	\Delta\Phi = 1.721\rho\delta U.
\end{align*}

Vi söker vidare rörelsemängdsflödet i $x$-riktning per längd i $z$-riktning genom $A'B'$, som ges av
\begin{align*}
	\Phi_{p} = \integ{0}{L}{x}{\rho u_{y}u_{x}}.
\end{align*}
Med likformighetslösningen fås
\begin{align*}
	u_{y} &= -U\dv{(f\delta)}{x} \\
	      &= -U\left(\dv{\delta}{x}f + \delta\dv{f}{\eta}\dv{\eta}{x}\right) \\
	      &= -U\left(\dv{\delta}{x}f - \frac{y}{\delta^{2}}\delta\dv{f}{\eta}\dv{\delta}{x}\right) \\
	      &= U\dv{\delta}{x}\left(\frac{y}{\delta}\dv{f}{\eta} - f\right) \\
	      &= U\dv{\delta}{x}\left(\eta\dv{f}{\eta} - f\right).
\end{align*}
Därmed fås
\begin{align*}
	\Phi_{p} = \integ{0}{L}{x}{\rho U^{2}\dv{\delta}{x}\left(\eta\dv{f}{\eta} - f\right)\dv{f}{\eta}}.
\end{align*}
Vi integrerar i det asymptotiska området, varför den asymptotiska lösningen kan användas för att ge
\begin{align*}
	\Phi_{p} &= \integ{0}{L}{x}{\rho U^{2}\dv{\delta}{x}\left(\eta - \eta + 1.721\right)} \\
	         &= 1.721\rho U^{2}\sqrt{\frac{\nu L}{U}}.
\end{align*}

Om man antar att den asymptotiska lösningen gäller överallt mellan de fyra punkterna, kan man beräkna motståndskraften på plattan. Skjuvspänningen på plattan i $x$-riktning ges av
\begin{align*}
	\tau_{xy} &= \mu\left(\dv{u_{x}}{y} + \dv{u_{y}}{y}\right) \\
	          &= \mu U\dv{y}\left(\frac{f}{\eta}\right) \\
	          &= \mu\frac{U}{\delta}\dv[2]{f}{\eta}.
\end{align*}
Motståndskraften per längd i $z$-riktning ges då av
\begin{align*}
	F_{x} &= \integ{0}{L}{x}{\mu\frac{U}{\delta}\dv[2]{f}{\eta}} \\
	      &= \mu U\integ{0}{L}{x}{\frac{1}{\delta}\dv[2]{f}{\eta}}
\end{align*}

\paragraph{Avlösning}
Kring kroppar som omges av strömning kommer det bildas gränsskikt där vorticiteten är hög. Detta kommer påverka strömningen med olik karaktär beroende på ett Reynoldstal som beskriver strömningen.

\paragraph{Motståndskraft på kropp i fluid}
Betrakta en statisk kropp i en fluid med friströmshastighet $U\ub{x}$. På denna är motståndskraften
\begin{align*}
	F = \frac{1}{2}\rho U^{2}C_{D}S
\end{align*}
där $S$ är den projicerade arean i strömriktningen och $C_{D}$ är en motståndskoefficient som beror på Reynoldstalet.

\paragraph{Exempel: Terminalhastighet för en regndroppe}
Betrakta en vattendroppe med diametern $d$ med tätheten $\rho_{\text{s}}$ i tyngdfältet som faller genom atmosfären, som har täthet $\rho$ och kinematisk viskositet $\nu$. Antag att motståndskoefficienten för vattendroppen ges av
\begin{align*}
	C_{D} = \frac{24}{R}\left(1 + \frac{1}{6}R^{\frac{2}{3}}\right),
\end{align*}
där $R$ är Reynoldstalet $\frac{Ud}{\nu}$. Det antas att detta är lägre än $1000$. Vi vill bestämma droppens terminalhastighet.

För att göra detta, ställ upp kraftjämvikten i vertikal riktning, som vid terminalhastigheten ger
\begin{align*}
	D + \frac{\pi}{6}d^{3}\rho g - mg = 0.
\end{align*}
Genom att använda motståndskoefficienten fås
\begin{align*}
	\frac{1}{2}\rho U^{2}C_{D}\frac{\pi}{4}d^{2} = mg - \frac{\pi}{6}d^{3}\rho g = \frac{\pi}{6}d^{3}\rho_{\text{s}}g\left(1 - \frac{\rho}{\rho_{\text{s}}}\right).
\end{align*}
Vi förenklar till
\begin{align*}
	\rho U^{2}C_{D} &= \frac{4}{3}d\rho_{\text{s}}g\left(1 - \frac{\rho}{\rho_{\text{s}}}\right), \\
	U               &= \sqrt{\frac{4}{3C_{D}}\frac{\rho_{\text{s}}}{\rho}dg\left(1 - \frac{\rho}{\rho_{\text{s}}}\right)}.
\end{align*}
Detta är en implicit ekvation i terminalhastigheten, då dämpningskoefficienten beror av denna. Vi kan lösa den numeriskt genom att gissa en dämpningskoefficient, beräkna den motsvarande terminalhastigheten, beräkna det motsvarande Reynoldstalet och dämpningskoefficienten och upprepa proceduren tills du får konvergens.