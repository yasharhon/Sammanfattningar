\section{Viskösa effekter}

\paragraph{Exempel: Couetteströmning av två vätskor}
Betrakta två Newtonska fluider mellan två plattor, med kinematisk viskositeter $\nu_{1}$ $\nu_{2}$ (nummererade från botten till topp), täthet $\rho$ och konstant tryck $p$. Vätskorna tar upp lika stor plats i $y$-led. Strömningen är stationär och fullt utbildad. Plattorna är på ett avstånd $h$ i $y$-riktning. Ena plattan är fäst, och andra plattan rör sig med en hastighet $U$ i $x$-riktning. Vi vill nu bestämma stationära hastighetsfältet och spänningen mellan de två fluiderna.

Vi vill nu lösa Navier-Stokes' ekvation, med samma randvillkor som för Couetteströmning. Det stationära villkoret och konstanta trycket tar bort två termer, och symmetrin kombinerad med att flödet bara drivs i $x$-riktning tar bort en till term. Kvar står termen med Laplaceoperatorn. Att lösa detta problemet innehåller en lite subtil distinktion för icke-homogena vätskor. Denna termen kommer från en derivata av töjningstensorn. Symmetrin ger då att differentialekvationen som beskriver hastighetsfältet är
\begin{align*}
	\frac{1}{\rho}\del{y}{(\mu\del{y}{u_{x}})} = 0.
\end{align*}
Tätheten kan tas in i derivationen eftersom den är konstant för att ge
\begin{align*}
	\del{y}{(\nu\del{y}{u_{x}})} = 0.
\end{align*}
Vi ser då att lösningen kommer vara olik i de två vätskorna - mer specifikt
\begin{align*}
	u_{x} =
	\begin{cases}
		\frac{A_{1}}{\nu_{1}}y,     & y < \frac{1}{2}h, \\
		\frac{A_{2}}{\nu_{2}}y + B, & y > \frac{1}{2}h,
	\end{cases}
\end{align*}
där ena konstanttermen har tagits bort direkt för att uppfylla ena randvillkoret. Andra randbillkoret ger
\begin{align*}
	B = U - \frac{A_{2}h}{\nu_{2}},\ u_{x} =
	\begin{cases}
		\frac{A_{1}}{\nu_{1}}y,                                 & y < \frac{1}{2}h, \\
		U + \frac{A_{2}h}{\nu_{2}}\left(\frac{y}{h} - 1\right), & y > \frac{1}{2}h.
	\end{cases}
\end{align*}
Det står kvar två konstanter, så två till villkor behövs. Ena är att hastighetsfältet måste vara kontinuerligt, då man annars skulle ha att vätskorna glider längsmed varandra, vilket inte kan hända eftersom flödet är visköst (argumentet är det samma som för vidhäftningsvillkoret). Detta ger
\begin{align*}
	\frac{A_{1}h}{2\nu_{1}} = U - \frac{A_{2}h}{2\nu_{2}}.
\end{align*}
För att få det sista villkoret kan man frilägga de två vätskorna. Dessa kommer verka med en skjuvkraft på varandra som enligt Newtons tredje lag måste vara lika. Om $\tau_{i}$ är skjuvkraften på vätska $i$ i $x$-riktning, blir villkoret $\tau_{1} = -\tau_{2}$. Vi har
\begin{align*}
	\tau_{1} = \mu_{1}\del{y}{u_{x}},\ \tau_{2} = -\mu_{2}\del{y}{u_{x}},
\end{align*}
vilket ger
\begin{align*}
	\mu_{1}\frac{A_{1}}{\nu_{1}} = \mu_{1}\frac{A_{2}}{\nu_{2}},\ A_{1} = A_{2} = A.
\end{align*}
Därmed fås
\begin{align*}
	\frac{Ah}{2\nu_{1}}                       &= U - \frac{Ah}{2\nu_{2}}, \\
	A\frac{\nu_{1} + \nu_{2}}{\nu_{1}\nu_{2}} &= \frac{2U}{h}, \\
	A                                         &= \frac{2U}{h}\frac{\nu_{1}\nu_{2}}{\nu_{1} + \nu_{2}}.
\end{align*}
Hastighetsfältet blir slutligen
\begin{align*}
	u_{x} =
	\begin{cases}
		2U\frac{\nu_{2}}{\nu_{1} + \nu_{2}}\frac{y}{h},                                   & y < \frac{1}{2}h, \\
		U\left(1 + 2\frac{\nu_{1}}{\nu_{1} + \nu_{2}}\left(\frac{y}{h} - 1\right)\right), & y > \frac{1}{2}h.
	\end{cases}
\end{align*}

Nu återstår bara att beräkna skjuvspänningen. Vi får
\begin{align*}
	\tau_{1} = \mu_{1}\frac{2U}{h}\frac{\nu_{2}}{\nu_{1} + \nu_{2}} = \frac{2U\rho}{h}\frac{\nu_{1}\nu_{2}}{\nu_{1} + \nu_{2}}, \\
	\tau_{2} = -\mu_{2}\frac{2U}{h}\frac{\nu_{1}}{\nu_{1} + \nu_{2}} = -\frac{2U\rho}{h}\frac{\nu_{1}\nu_{2}}{\nu_{1} + \nu_{2}}.
\end{align*}

\paragraph{Bildningen av gränsskikt}
Vidhäftningsvillkoret gör att nära ytor ändras hastigheten extremt snabbt i ett tunt skikt nära ytan, det så kallade gränsskiktet. Vi kommer studera beteendet i och utanför såna gränsskikt i fall som är symmetriska i $z$-riktning.

\paragraph{Formulering av problem}
Vi kommer betrakta strömning av en fluid i $x$-riktning med friströmshastighet $U$. Fluiden strömmar förbi en platta som är parallell med $x$-riktningen och har längd $L$. Över ytan finns ett gränsskikt med tjocklek $\delta$

\paragraph{Reynoldstalet}
Navier-Stokes' ekvation ger
\begin{align*}
	u_{x}\del{x}{u_{x}} + u_{y}\del{y}{u_{x}} = -\frac{1}{\rho}\del{x}{p} + \nu(\del[2]{x}{u_{x}} + \del[2]{y}{u_{x}}).
\end{align*}
Vi studerar först problemet i gränsskiktet. Här är viskösa krafter viktiga, så $y$-derivatan kommer vara stor här. Om vi av någon oklar anledning antar att $x$-derivatan är av ledande ordning på vänstersidan, blir detta
\begin{align*}
	u_{x}\del{x}{u_{x}} = \nu\del[2]{y}{u_{x}}.
\end{align*}
Detta låter oss göra en grov storleksuppskattning
\begin{align*}
	\frac{U^{2}}{L} = \nu\frac{U}{\delta^{2}},\ \frac{\delta}{L} = \frac{1}{\sqrt{\rey}},
\end{align*}
där $\rey = \frac{UL}{\nu}$ är Reynoldstalet. I de flesta tillämpningar är Reynoldstalet stort, så gränsskiktet är tunt.

\paragraph{Dimensionslösa variabler}
Vi vill nu förenkla Navier-Stokes' ekvationer för stora Reynoldstal. För att göra detta låter vi $U$ och $p_{\infty}$ vara hastigheten och trycket långt uppströms. Vi antar att $u_{x}$ är av storleksordning $U$ och att $\del{x}{p} = \rho u_{x}\del{x}{u_{x}}$. Detta ger storleksordningsuppskattningen
\begin{align*}
	p_{\infty} - p \approx \rho U^{2}.
\end{align*}
Vi kan även skatta vertikala hastigheten med hjälp av inkompressibilitetsvillkoret som
\begin{align*}
	u_{y} \approx \frac{\delta}{L}U = \frac{1}{\sqrt{\rey}}U.
\end{align*}
Detta motiverar oss att införa de dimensionslösa variablerna
\begin{align*}
	x' = \frac{x}{L},\ y' = \frac{y}{\delta} = \sqrt{\rey}\frac{y}{L},\ u_{x}' = \frac{u_{x}}{U},\ u_{y}' = \sqrt{\rey}\frac{u_{y}}{U},\ p' = \frac{p_{\infty} - p}{\rho U^{2}}.
\end{align*}
I termer av dessa variabler är Navier-Stokes' ekvationer
\begin{align*}
	u_{x}'\del{x'}{u_{x}'} + u_{y}'\del{y'}{u_{x}'}                 &= -\del{y'}{p'} + \frac{1}{\rey}\del[2]{x'}{u_{x}'} + \del[2]{y'}{u_{x}'}, \\
	\frac{1}{\rey}(u_{x}'\del{x'}{u_{y}'} + u_{y}'\del{y'}{u_{y}'}) &= -\del{y'}{p'} + \frac{1}{\rey^{2}}\del[2]{x'}{u_{x}'} + \frac{1}{\rey}\del[2]{y'}{u_{x}'}, \\
	\del{x'}{u_{x}'} + \del{y'}{u_{y}'} = 0.
\end{align*}
För stora Reynoldstal kan vi nu bortse från många termer här. I våra ursprungliga variabler fås då
\begin{align*}
	u_{x}\del{x}{u_{x}} + u_{y}\del{y}{u_{x}}                 &= -\del{y}{p} + \del[2]{y}{u_{x}}, \\
	0                                                         &= -\del{y}{p}, \\
	\del{x}{u_{x}} + \del{y}{u_{y}} = 0.
\end{align*}
Man kan tydligen även använda Bernoullis ekvation (fast inte) längsmed en strömlinje som följer gränsskiktets rand. Vi får då
\begin{align*}
	\frac{p}{\rho} + \frac{1}{2}U^{2} = c,\ -\frac{1}{\rho}\del{x}{p} = U\del{x}{U}.
\end{align*}
Nu har vi ställt upp alla ekvationer, och vi har randvillkoren
\begin{align*}
	u_{x}(x, 0) = u_{y}(x, 0) = 0,\ u_{x}(x, \infty) = U,
\end{align*}

\paragraph{Mått på tjocklek}
Härifrån måste vi välja någon definition av gränsskiktets tjocklek. Detta är tre olika sätt att göra det på.

\paragraph{Delta-nittinio}
Delta-nittinio-måttet definieras av
\begin{align*}
	u(x, \delta_{99}) = 0.99U.
\end{align*}

\paragraph{Förträngningstjockleken}
Förträngningstjockleken definieras som avståndet i vertikal riktning en strömlinje långt från väggen förflyttas på grund av väggen.

För att beräkna den observerar vi att masskonserveringen ger
\begin{align*}
	\integ{0}{h}{y}{U} = \integ{0}{h + \delta_{\star}}{y}{u_{x}},
\end{align*}
där vi gör andra integralen på ett (långt?) avstånd från där strömningen möter plattan. Eftersom vi pratar om strömlinjer långt från väggen, gör vi skattningen $u_{x} = U$ långt borta, vilket ger
\begin{align*}
	\delta_{\star} = \integ{0}{h}{y}{1 - \frac{u_{x}}{U}}.
\end{align*}

\paragraph{Rörelsemängdtjocklek}
Om tryckgradienten är noll fås för kontrollvolymen vi studerade ovan
\begin{align*}
	-f = -\integ{0}{h}{y}{\rho U^{2}} + \integ{0}{h + \delta_{\star}}{y}{\rho u_{x}^{2}}
\end{align*}
där
\begin{align*}
	f = \integ{0}{x}{x}{\mu\del{y}{u_{x}}(x, 0)}
\end{align*}
är friktionskraften i $x$-riktning på den delen av plattan som är innanför kontrollvolymen. Vi definierar nu rörelsemängdstjockleken
\begin{align*}
	\theta &= \frac{f}{\rho U^{2}} \\
	       &= \integ{0}{h}{y}{1 - \left(\frac{u_{x}}{U}\right)^{2}} + \frac{1}{U^{2}}\integ{h}{h + \delta_{\star}}{y}{u_{x}^{2}} \\
	       &\approx \integ{0}{h}{y}{1 - \left(\frac{u_{x}}{U}\right)^{2}} + \delta_{\star} \\
	       &= \integ{0}{h}{y}{\frac{u_{x}}{U}\left( 1 - \frac{u_{x}}{U}\right)}.
\end{align*}
Vi kan tydligen skriva detta som
\begin{align*}
	\theta = \integ{0}{\infty}{y}{\frac{u_{x}}{U}\left( 1 - \frac{u_{x}}{U}\right)}.
\end{align*}

\paragraph{Blasius' gränsskikt}
Blasius' lösning för gränsskiktet är en något annorlunda lösning av samma problemet vi har studerat. I detta fall är tryckgradienten lika med noll och strömningen möter plattan i origo. Inkompressibilitetsvillkoret låter oss införa strömfunktionen $\Psi$, och Navier-Stokes' ekvationer ger
\begin{align*}
	\del{y}{\Psi}\del{y}{\del{x}{\Psi}} - \del{x}{\Psi}\del[2]{y}{\Psi} = \nu\del[3]{y}{\Psi}.
\end{align*}
Eftersom systemet ej ändras under addition av en konstant till strömfunktionen, kan vi välja $\Psi(x, 0) = 0$. De övriga randvillkoren blir
\begin{align*}
	\del{y}{\Psi}(x, 0) = 0,\ \del{y}{\Psi}(x, y) \to U.
\end{align*}
Vi kommer lösa detta problemet med en likformighetsansats
\begin{align*}
	u_{x} = Uf'(\eta),\ \eta = \frac{y}{\delta(x)}.
\end{align*}
Detta ger
\begin{align*}
	\Psi = \integ{0}{y}{y}{u_{x}} = U\delta\integ{0}{\eta}{\eta}{f'(\eta)} = U\delta f(\eta).
\end{align*}
Insatt i Navier-Stokes' ekvation ger detta
\begin{align*}
	f''' = -\left(\frac{U\delta}{\nu}\dv{\delta}{x}\right)ff''.
\end{align*}
Om detta skall gälla, måste prefaktorn vara en konstant. Vi kan sätta den till $\frac{1}{2}$, vilket ger
\begin{align*}
	\delta = \sqrt{\frac{\nu x}{U}}.
\end{align*}
Att ändra den konstanten skulle bara motsvara en omskalning av enheterna, så vi ser att valet av konstant är godtyckligt. Den återstående ekvationen är
\begin{align*}
	\frac{1}{2}ff'' + f''' = 0,
\end{align*}
med randvillkor
\begin{align*}
	f(0) = 0,\ f'(0) = 0,\ f'(\eta) \to 1.
\end{align*}

Vi kan även beräkna väggskjuvspänningen
\begin{align*}
	\tau = \mu\del{y}{u_{x}}(x, 0) = \frac{0.332\rho U^{2}}{\sqrt{\rey_{x}}}
\end{align*}
där $\rey_{x}$ är Reynoldstalet beräknad med $x$. Det totala aerodynamiska moståndet på plattan ges då av
\begin{align*}
	F = \integ{0}{L}{x}{\tau} = \frac{0.664\rho U^{2}}{\sqrt{\rey_{L}}}.
\end{align*}
Vi kan även definiera en motståndskoefficient
\begin{align*}
	C = \frac{F}{\frac{1}{2}\rho U^{2}} = \frac{1.33}{\sqrt{\rey_{L}}}.
\end{align*}

\paragraph{Exempel: Blasius' gränsskikt med känd asymptot}
Betrakta viskös strömning över en lång platta. Vätskan har friströmshastighet $U$, täthet $\rho$ och kinematisk viskositet $\nu$. De fyra punkterna $A$, $B$, $A'$ och $B'$ definieras av koordinaterna $(0, 0)$, $(x, 0)$, $(0, h)$ och $(x, h)$, där $h$ är ett avstand som är mycket större än gränsskiktets tjocklek. För stora $\eta$ gäller att $f(\eta) \approx \eta - 1.721$.

Vi söker först skillnaden i massflöde per längd i $z$ genom $AA'$ och $BB'$. Första ges av
\begin{align*}
	\Phi = \integ{0}{h}{y}{\rho u_{x}} = \integ{0}{h}{y}{\rho U},
\end{align*}
och andra ges av
\begin{align*}
	\Phi = \integ{0}{h}{y}{\rho u_{x}} = \integ{0}{h}{y}{\rho U\dv{f}{\eta}},
\end{align*}
där vi har använt likformighetslösningen. Skillnaden ges av
\begin{align*}
	\Delta\Phi &= \integ{0}{h}{y}{\rho U\left(1 - \frac{u_{x}}{U}\right)} \\
	           &= \rho\delta U\integ{0}{\frac{h}{\delta}}{\eta}{1 - \dv{f}{\eta}} \\
	           &= \rho\delta U\left[\eta - f(\eta)\right]_{0}^{\frac{h}{\delta}}.
\end{align*}
Unre gränsen ger inget bidrag, och i övre gränsen kan asymptotiska lösningen användas, vilket ger
\begin{align*}
	\Delta\Phi = 1.721\rho\delta U = 1.721\rho\sqrt{\nu Ux}.
\end{align*}

Vi söker vidare skillnaden i rörelsemängdsflödet i $x$-riktning per längd i $z$-riktning genom $AA'$ och $BB'$. Första ges av
\begin{align*}
	\Phi_{p} = \integ{0}{h}{y}{\rho u_{x}u_{x}} = \integ{0}{h}{y}{\rho U^{2}}.
\end{align*}
Andra ges av
\begin{align*}
	\Phi_{p} = \integ{0}{h}{y}{\rho u_{x}^{2}}.
\end{align*}
Skillnaden ges av
\begin{align*}
	\Delta\Phi_{p} &= \rho U^{2}\integ{0}{h}{y}{1 - \left(\frac{u_{x}}{U}\right)^{2}} \\
	               &= \rho U^{2}\integ{0}{h}{y}{1 - \left(\dv{f}{\eta}\right)^{2}} \\
	               &= \rho\delta U^{2}\integ{0}{\frac{h}{\delta}}{\eta}{1 - \left(\dv{f}{\eta}\right)^{2}} \\
	               &= \rho\delta U^{2}\left(\frac{h}{\delta} - f\left(\frac{h}{\delta}\right)\dv{f}{\eta}\eval_{\eta = \frac{h}{\delta}} + f(0)\dv{f}{\eta}\eval_{\eta = 0} +  \integ{0}{\frac{h}{\delta}}{\eta}{f\dv[2]{f}{\eta}}\right) \\
	               &= \rho\delta U^{2}\left(1.721 - 2\integ{0}{\frac{h}{\delta}}{\eta}{f\dv[3]{f}{\eta}}\right) \\
	               &= \rho\delta U^{2}\left(1.721 + 2\dv[2]{f}{\eta}\eval_{\eta = 0}\right) \\
	               &= \sqrt{\nu Ux}\rho U\left(1.721 + 2\dv[2]{f}{\eta}\eval_{\eta = 0}\right).
\end{align*}
Om man antar att $\dv[2]{f}{\eta}\eval_{\eta = 0} = 0.332$ fås
\begin{align*}
	\Delta\Phi_{p} = 2.385\rho\delta U^{2} = 2.385\rho\sqrt{\nu U^{3}x}.
\end{align*}

Om man antar att den asymptotiska lösningen gäller överallt mellan de fyra punkterna, kan man beräkna motståndskraften på plattan. Skjuvspänningen på plattan i $x$-riktning ges av
\begin{align*}
	\tau_{xy} &= \mu\left(\dv{u_{x}}{y} + \dv{u_{y}}{y}\right) \\
	          &= \mu U\dv{y}\left(\dv{f}{\eta}\right) \\
	          &= \mu\frac{U}{\delta}\dv[2]{f}{\eta}.
\end{align*}
Detta skall självklart evalueras vid plattan, där $y = 0$. Motståndskraften per längd i $z$-riktning ges då av
\begin{align*}
	F_{x} &= \integ{0}{x}{x'}{\mu\frac{U}{\delta}\dv[2]{f}{\eta}\eval_{\eta = 0}} \\
	      &= \dv[2]{f}{\eta}\eval_{\eta = 0}\mu U\integ{0}{x}{x'}{\frac{1}{\delta}} \\
	      &= \dv[2]{f}{\eta}\eval_{\eta = 0}\mu U\integ{0}{x}{x'}{\sqrt{\frac{U}{\nu x'}}} \\
	      &= 2\dv[2]{f}{\eta}\eval_{\eta = 0}\sqrt{\frac{Ux}{\nu}}\mu U \\
	      &= 2\dv[2]{f}{\eta}\eval_{\eta = 0}\sqrt{\nu Ux}\rho U
\end{align*}

Vi vill vidare beräkna rörelsemängdsflödet i $x$-led genom $A'B'$. Friläggning av kontrollvolymen ger att
\begin{align*}
	\Phi_{p} &= -F_{x} + \Delta\Phi_{p} \\
	         &= 1.721\rho\delta U^{2} \\
	         &= 1.721\sqrt{\nu Ux}\rho U.
\end{align*}

Slutligen kontrollerar vi detta upp mot Blasiuslösningens uttryck för $y$-komponenten av hastighetsfältet som
\begin{align*}
	u_{y} = \frac{1}{2}U\sqrt{\frac{\nu}{Ux}}\left(\eta\dv{f}{\eta} - f\right).
\end{align*}
Massflödet genom $A'B'$ ges av
\begin{align*}
	\Phi &= \integ{0}{x}{x'}{\frac{1}{2}\rho U\sqrt{\frac{\nu}{Ux'}}\left(\eta\dv{f}{\eta} - f\right)} \\
	     &= \frac{1.721}{2}\rho U\integ{0}{x}{x'}{\sqrt{\frac{\nu}{Ux'}}} \\
	     &= 1.721\rho\sqrt{\nu Ux},
\end{align*}
som väntat. Rörelsemängdsflödet ges av
\begin{align*}
	\Phi_{p} &= \integ{0}{x}{x'}{\frac{1}{2}\rho U\sqrt{\frac{\nu}{Ux'}}\left(\eta\dv{f}{\eta} - f\right)U\dv{f}{\eta}} \\
	         &= \frac{1.721}{2}\rho U^{2}\integ{0}{x}{x'}{\sqrt{\frac{\nu}{Ux'}}} \\
	         &= 1.721\sqrt{\nu Ux}\rho U,
\end{align*}
som väntat.

\paragraph{Avlösning}
Kring kroppar som omges av strömning kommer det bildas gränsskikt där vorticiteten är hög. Detta kommer påverka strömningen med olik karaktär beroende på ett Reynoldstal som beskriver strömningen.

\paragraph{Motståndskraft på kropp i fluid}
Betrakta en statisk kropp i en fluid med friströmshastighet $U\ub{x}$. På denna är motståndskraften
\begin{align*}
	F = \frac{1}{2}\rho U^{2}C_{D}S
\end{align*}
där $S$ är den projicerade arean i strömriktningen och $C_{D}$ är en motståndskoefficient som beror på Reynoldstalet.

\paragraph{Exempel: Terminalhastighet för en regndroppe}
Betrakta en vattendroppe med diametern $d$ med tätheten $\rho_{\text{s}}$ i tyngdfältet som faller genom atmosfären, som har täthet $\rho$ och kinematisk viskositet $\nu$. Antag att motståndskoefficienten för vattendroppen ges av
\begin{align*}
	C_{D} = \frac{24}{R}\left(1 + \frac{1}{6}R^{\frac{2}{3}}\right),
\end{align*}
där $R$ är Reynoldstalet $\frac{Ud}{\nu}$. Det antas att detta är lägre än $1000$. Vi vill bestämma droppens terminalhastighet.

För att göra detta, ställ upp kraftjämvikten i vertikal riktning, som vid terminalhastigheten ger
\begin{align*}
	D + \frac{\pi}{6}d^{3}\rho g - mg = 0.
\end{align*}
Genom att använda motståndskoefficienten fås
\begin{align*}
	\frac{1}{2}\rho U^{2}C_{D}\frac{\pi}{4}d^{2} = mg - \frac{\pi}{6}d^{3}\rho g = \frac{\pi}{6}d^{3}\rho_{\text{s}}g\left(1 - \frac{\rho}{\rho_{\text{s}}}\right).
\end{align*}
Vi förenklar till
\begin{align*}
	\rho U^{2}C_{D} &= \frac{4}{3}d\rho_{\text{s}}g\left(1 - \frac{\rho}{\rho_{\text{s}}}\right), \\
	U               &= \sqrt{\frac{4}{3C_{D}}\frac{\rho_{\text{s}}}{\rho}dg\left(1 - \frac{\rho}{\rho_{\text{s}}}\right)}.
\end{align*}
Detta är en implicit ekvation i terminalhastigheten, då dämpningskoefficienten beror av denna. Vi kan lösa den numeriskt genom att gissa en dämpningskoefficient, beräkna den motsvarande terminalhastigheten, beräkna det motsvarande Reynoldstalet och dämpningskoefficienten och upprepa proceduren tills du får konvergens.