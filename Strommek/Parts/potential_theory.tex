\section{Potentialteori}
Potentialteorin som betraktas här är för fall med symmetri i $z$-riktning, tror jag.

\paragraph{Strömfunktionen}
Kontinuitetsekvationen för en inkompressibel fluid ger
\begin{align*}
	\del{x}{u_{x}} + \del{y}{u_{y}} = 0.
\end{align*}
Lösningen av detta ges av strömfunktionen $\Psi$. Den definieras så att
\begin{align*}
	\del{y}{\Psi} = u_{x},\ \del{x}{\Psi} = -u_{y}.
\end{align*}
En viktig karakteristik för strömfunktionen fås genom att betrakta en liten ändring
\begin{align*}
	\dd{\Psi} = \grad{\Psi}\cdot\dd{\vb{r}} = -u_{y}\dd{x} + u_{x}\dd{y}.
\end{align*}
Om vi rör oss längsmed en strömlinje, är denna lilla ändringen lika med $0$.

\paragraph{Potentiallösning för vorticitetsfria fluider}
Om en fluid är vorticitetsfri, gäller det att
\begin{align*}
	\vb{u} = \grad{\Phi}.
\end{align*}
Inkompressibilitetsvillkoret är då
\begin{align*}
	\laplacian{\Phi} = 0.
\end{align*}

\paragraph{Potential för strömning kring en cylinder}
I cylinderkoordinater är gradientoperatorn $\grad = \ub{r}\del{r} + \frac{1}{r}\ub{\phi}\del{\phi}$ och laplaceoperatorn $\laplacian = \del[2]{r} + \frac{1}{r}\del{r} + \frac{1}{r^{2}}\del[2]{\phi}$. För att studera strömningen av en inviskös fluid kring en cylinder med radien $a$ centrerad i origo, söker vi då lösningen till
\begin{align*}
	\del[2]{r}{\Phi} + \frac{1}{r}\del{r}{\Phi} + \frac{1}{r^{2}}\del[2]{\phi}{\Phi} = 0,\ \del{r}{\Phi}\eval_{r = a} = 0,\ \left(\ub{r}\del{r}{\Phi} + \frac{1}{r}\ub{\phi}\del{\phi}{\Phi}\right)\eval_{r = \infty} = U\ub{x}.
\end{align*}
Vi löser detta med en separabel ansats - mer specifikt $\Phi = R(r)\cos{\phi}$. För att se att detta är en bra ansats, titta på det andra randvillkoret för $\theta = 0$. Givet detta fås
\begin{align*}
	\cos{\phi}\dv[2]{R}{r} + \frac{\cos{\phi}}{r}\dv{R}{r} - \frac{\cos{\phi}}{r^{2}}R &= 0, \\
	r^{2}\dv[2]{R}{r} + r\dv{R}{r} - R                                                 &= 0.
\end{align*}
Genom att ansätta $R = r^{n}$ för något $n$ fås
\begin{align*}
	(n(n - 1) + n - 1)r^{n} = (n^{2} - 1)r^{n} = 0.
\end{align*}
Om detta skall gälla för alla $r$, måste $n = \pm 1$ och $R = Ar + \frac{B}{r}$. Första randvillkoret ger
\begin{align*}
	A - \frac{B}{a^{2}} = 0,\ R = A\left(r + \frac{a}{r}a\right).
\end{align*}
Andra randvillkoret ger för $\phi = 0$
\begin{align*}
	\left(\ub{r}A\left(1 - \frac{a^{2}}{r^{2}}\right) - \frac{\sin{\phi}}{r}\ub{\phi}A\left(r + \frac{a}{r}a\right)\right)\eval_{\phi = 0,\ r = \infty} = \ub{x}A,
\end{align*}
och den slutgiltiga lösningen är
\begin{align*}
	\Phi = U\cos{\phi}\left(r + \frac{a}{r}a\right).
\end{align*}
Vi noterar till senare att detta kan skrivas
\begin{align*}
	\Phi = U\cos{\phi}r + U\cos(-\phi)\frac{a}{r}a = \Re\left(Uz + \frac{Ua^{2}}{z}\right).
\end{align*}
Hastighetskomponenterna är
\begin{align*}
	u_{r} = U\cos{\phi}\left(1 - \frac{a^{2}}{r^{2}}\right),\ u_{\phi} = -U\sin{\phi}\left(1 + \frac{a^{2}}{r^{2}}\right).
\end{align*}

\paragraph{d'Alemberts paradox}
d'Alemberts paradox är att teorin för potentialströmning ger att luftmotståndet på en godtycklig kropp är $0$. Detta är rimligt eftersom vi inte tar med viskösa effekter, som kommer introduceras.

\paragraph{Komplex potential}
Vi inför nu den komplexa strömfunktionen $W = \Phi + i\Psi$. Om detta ska vara en analytisk funktion (alltså kunna skrivas som en funktion av ett komplext tal $w = x + iy$ så att realdelen och imaginärdelen behandlas likadant), måste den uppfylla Cauchy-Riemanns ekvationer
\begin{align*}
	\del{x}{\Phi} = \del{y}{\Psi},\ \del{y}{\Phi} = -\del{x}{\Psi}.
\end{align*}
Genom att välja realdelen till att vara hastighetsfältets fås att imaginärdelen är strömfunktionen, och motsatt. Cauchy-Riemanns ekvationer implicerar nu direkt att både realdelen och imaginärdelen uppfyller Laplace' ekvation.

\paragraph{Hastighetsfältet från potentialen}
Hastighetsfältet kan beräknas på tre olika sätt:
\begin{itemize}
	\item Beräkna $\dv{W}{w}$ längsmed en vilken som helst riktning. Om man exempelvis fixerar $y$ fås $\dv{W}{w} = u_{x} - iu_{y}$.
	\item Beräkna $\grad{\Re(W)}$.
	\item Beräkna $\curl(\Im(W)\ub{z})$.
\end{itemize}

\paragraph{Potential för friström}
\begin{align*}
	W = Uw.
\end{align*}

\paragraph{Strömning kring ett hörn}
Betrakta funktionen
\begin{align*}
	W = Aw^{n} = Ar^{n}e^{in\theta}
\end{align*}
där $A$ är en konstant och $n > \frac{1}{2}$. Strömlinjerna ges av att imaginärdelen av $W$ är konstant. Två såna, som motsvarar $\Im(W) = 0$, är $\theta = 0$ och $\theta = \frac{\pi}{n}$. Därmed kan detta tolkas som potentialen för strömning kring ett hörn med öppningsvinkel $\frac{\pi}{n}$.

\paragraph{Potential för en källa och en sänka}
Betrakta funktionen
\begin{align*}
	W = \frac{m}{2\pi}\ln{w} = \frac{m}{2\pi}(\ln{r} + i\theta).
\end{align*}
De motsvarande hastighetskomponenterna är
\begin{align*}
	u_{r} = \frac{m}{2\pi r},\ u_{\theta} = 0.
\end{align*}
Flödet ut ur en kurva kring origo ges av
\begin{align*}
	q = \integ{C}{}{l}{u_{r}} = m.
\end{align*}
Tecknet på $m$ ger alltså om det är en källa eller sänka, och beloppet ger styrkan.

\paragraph{Potential för virvel}
Betrakta funktionen
\begin{align*}
	W = \frac{i\Gamma}{2\pi}\ln{w} = \frac{\Gamma}{2\pi}(-\theta + i\ln{r}).
\end{align*}
De motsvarande hastighetskomponenterna är
\begin{align*}
	u_{r} = 0,\ u_{\theta} = -\frac{\Gamma}{2\pi r}.
\end{align*}
Detta är en virvel med cirkulationen $\Gamma$ i medurs riktning.

\paragraph{Dipol}
Betrakta funktionen
\begin{align*}
	W = \frac{\mu}{z}.
\end{align*}
Genom att sätta $\frac{\mu}{2\Im{W}} = -C$ kan man visa att strömlinjerna ges av
\begin{align*}
	x^{2} + (y - C)^{2} = C^{2}.
\end{align*}
Detta är cirklar med radius $C$ och centrum i $(0, C)$, vilket motsvarar strömningar från origo i cirklar som tangerar origo.

\paragraph{Kutta-Jukowskis sats}
Kutta-Jukowskis sats säjer att lyftkraften på en kropp ges av $\rho U\Gamma$, där $\Gamma$ är cirkulationen kring kroppen. Detta är en god approximation för många olika kroppar, speciellt tunna, strömlinjeformade kroppar.