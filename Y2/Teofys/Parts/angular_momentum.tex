\section{Rörelsemängdsmoment}

\paragraph{Rörelsemängdsmoment}
Klassiskt har vi för rörelsemängdsmomentet för en enda partikel att
\begin{align*}
	\dot{\vb{L}} = \vb{r}\times\vb{F}.
\end{align*}
Speciellt, om det inte finns några yttre kraftmoment, är rörelsemängdsmomentet konstant. Vid att beskriva systemet i sfäriska koordinater och uttrycka hastigheten tangentiellt på ytan med konstant $r$ i termer av rörelsemängdsmomentet, kan Hamiltonianen skrivas som
\begin{align*}
	H = \frac{1}{2}mv_{r}^{2} + \frac{L^{2}}{2mr^{2}} + V.
\end{align*}

När vi gör övergången till kvantmekanik, får vi följande operator:
\begin{align*}
	\vb{L} = \vb{r}\times\vb{p}.
\end{align*}
För att få dens belopp, kan vi titta på Hamiltonoperatorn i sfäriska koordinater och jämföra termvis för att få en operator. Vi får
\begin{align*}
	L^{2} = -\hbar^{2}\left(\frac{1}{\sin(\theta)}\pdv{\theta}\left(\sin(\theta)\pdv{\theta}\right) + \frac{1}{\sin^{2}(\theta)}\pdv[2]{\phi}\right).
\end{align*}
Det finns även argument för detta som görs med hjälp av kryssprodukten, som jag kanske borde lägga till. På detta sättet kan vi även få
\begin{align*}
	L_{z} = \frac{\hbar}{i}\pdv{\phi}.
\end{align*}

Det visar sig att
\begin{align*}
	L^{2}Y_{l}^{m} = \hbar^{2}l(l + 1)Y_{l}^{m},\ L_{z}Y_{l}^{m} = \hbar mY_{l}^{m}.
\end{align*}
Eftersom de ingående operatorerna konstruerades från Hermiteska operatorer, är även dessa Hermiteska, och vi får direkt att klotytefunktionerna är ortogonala med inreprodukten som gavs ovan.

Mellan de olika komponenterna av rörelsemängdsmomentet finns följande kommutationsrelationer:
\begin{align*}
	\commut{L_{x}, L_{y}} &= \commut{yp_{z} - zp_{y}}{zp_{x} - xp_{z}} \\
	                      &= \commut{yp_{z}}{zp_{x}} - \commut{yp_{z}}{xp_{z}} - \commut{zp_{y}}{zp_{x}} + \commut{zp_{y}}{xp_{z}} \\
	                      &= y\commut{p_{z}}{zp_{x}} + x\commut{zp_{y}}{p_{z}} \\
	                      &= y(z\commut{p_{z}}{p_{x}} + \commut{p_{z}}{p_{x}}z) + x(p_{y}\commut{z}{p_{z}} + \commut{z}{p_{z}}p_{y}) \\
	                      &= y(\commut{p_{z}}{z})p_{x} + x(\commut{z}{p_{z}})p_{y} \\
	                      &= y(-i\hbar)p_{x} + x(i\hbar)p_{y} \\
	                      &= i\hbar L_{z}.
\end{align*}
På samma sätt fås
\begin{align*}
	\commut{L_{z}}{L_{x}} = i\hbar L_{y}, \\
	\commut{L_{y}}{L_{z}} = i\hbar L_{x}, \\
	\commut{L^{2}}{\vb{L}} = \vb{0}.
\end{align*}
Vi ser att det inte är något speciellt med någon komponent, så vi väljer att konstruera de gemensamme egenfunktionerna till $L^{2}$ och $L_{z}$.

För att konstruera dessa egenfunktionerna, definierar vi stegoperatorerna
\begin{align*}
	L_{+} = L_{x} + iL_{y},\ L_{-} = \adj{L_{+}} = L_{x} - iL_{y}.
\end{align*}
Dessa är inte Hermiteska. Kommutationsrelationerna är
\begin{align*}
	\commut{L^{2}}{L_{\pm}} &= 0, \\
	\commut{L_{z}}{L_{\pm}} &= \pm\hbar L_{z}.
\end{align*}
Produkten av de två operatorerna är
\begin{align*}
	L_{+}L_{-} = L_{x}^{2} + L_{y}^{2} - i\commut{L_{x}}{L_{y}} = L^{2} - L_{z}^{2} + \hbar L_{z}, \\
	L_{-}L_{+} = L^{2} - L_{z}^{2} - \hbar L_{z}.
\end{align*}
Egenfunktionerna uppfyller
\begin{align*}
	L^{2}L_{+}\ket{\Psi} &= L_{+}L^{2}\ket{\Psi} = \hbar^{2}l(l + 1)L_{+}\ket{\Psi}, \\
	L_{z}L_{+}\ket{\Psi} &= L_{+}L_{z}\ket{\Psi} + \hbar L_{+}\ket{\Psi} = \hbar(m + 1)L_{+}\ket{\Psi}, \\
	L^{2}L_{-}\ket{\Psi} &= L_{-}L^{2}\ket{\Psi} = \hbar^{2}l(l + 1)L_{-}\ket{\Psi}, \\
	L_{z}L_{-}\ket{\Psi} &= L_{-}L_{z}\ket{\Psi} - \hbar L_{-}\ket{\Psi} = \hbar(m - 1)L_{-}\ket{\Psi}.
\end{align*}
Vi ser alltså att stegoperatorerna skapar nya egentillstånd med olika väntevärden för $L_{z}$. Däremot vet vi att väntevärdena av $L^{2}$ och $L_{z}^{2}$ är strikt positiva, och $L^{2} \leq L_{z}^{2}$, så följden måste terminera någon gång. Med andra ord finns det två tillstånd så att
\begin{align*}
	L_{+}\ket{m_{\text{max}}} = 0,\ L_{-}\ket{m_{\text{min}}} = 0.
\end{align*}

Vi får vidare
\begin{align*}
	L^{2}\ket{m_{\text{max}}} &= (L_{-}L_{+} + L_{z}^{2} + \hbar L_{z})\ket{m_{\text{max}}} \\
	                          &= (0 + m^{2}\hbar^{2} + \hbar^{2}m_{\text{max}})\ket{m_{\text{max}}},
\end{align*}
vilket implicerar
\begin{align*}
	l(l + 1) = m^{2}\hbar^{2} + \hbar^{2}m_{\text{max}},
\end{align*}
med lösning
\begin{align*}
	m_{\text{max}} = l.
\end{align*}
På samma sätt fås
\begin{align*}
	m_{\text{min}} = -l.
\end{align*}
Vi noterar nu att det finns $2l$ steg mellan det maximala och minimala värdet. Eftersom antal steg även måste vara ett heltal, så kan $l$ vara alla multipler av $\frac{1}{2}$.

För att få normering, betraktar vi
\begin{align*}
	L_{\pm}\ket{m} = A_{\pm}\ket{m\pm 1}.
\end{align*}
Dens inreprodukt med sig själv är
\begin{align*}
	\expval{L_{-}L_{+}}{m} = \abs{A_{+}}^{2}\braket{m \pm 1}
\end{align*}
å ena sidan, eftersom stegoperatorerna är varandras adjungerade operatorer, och
\begin{align*}
	\expval{L^{2} - L_{z}^{2} - \hbar L_{z}}{m} = \hbar^{2}(l(l + 1) - m^{2} - m)\braket{m},
\end{align*}
vilket ger
\begin{align*}
	A_{+} = \sqrt{l(l + 1)- m(m + 1)}\hbar
\end{align*}
och på samma sätt
\begin{align*}
	A_{-} = \sqrt{l(l + 1)- m(m - 1)}\hbar.
\end{align*}

Osäkerhetsrelationen ger
\begin{align*}
	\Delta L_{x}\Delta L_{y} \geq \frac{1}{2}\abs{\expval{\commut{L_{x}}{L_{y}}}} = \frac{1}{2}\hbar\abs{\expval{L_{z}}} = \frac{1}{2}\abs{m}\hbar^{2}.
\end{align*}
Alltså är inte tillstånden vi har hittat egenfunktioner till $L_{x}$ eller $L_{y}$ om inte $m = 0$.

\paragraph{Stern-Gerlachs experiment}
I Stern-Gerlachs experiment sköts en stråla av silveratomer genom ett magnetfält, och observerade strålen på andra sidan. Resultatet blev två distinkta träffpunkter.

Atomernas energi i magnetfältet ges av $H = -\vb{\mu}\cdot\vb{B}$. Om det magnetiska momentet $\vb{\mu}$ är intrinsikt, ges $z$-komponenten av kraften på atomen av $F_{z} = -\vb{\mu}\cdot\dv{\vb{B}}{z}$.

Stern och Gerlach tolkade sitt experiment som att de hade hittat en rumskvantisering som Bohr hade postulerat. Detta var en feltolkning, då uppsplittringen orsakas av elektronspinnen. Denna tolkningen föreslogs först i 1925 av Pauli och Kramers. Uhlenbeck och Goudsmit publicerade en artikel om detta i 1926.

\paragraph{Generalisering av rörelsemängdsmoment}
Vi ser att vi kommer behöva en utvidgad beskrivning av rörelsemängdsmoment, och definierar då ett rörelsemängdsmoment som en operator $\vb{L}$ med kommutationsrelationer
\begin{align*}
	\commut{S_{i}}{S_{j}} = \varepsilon_{ijk}S_{k}.
\end{align*}
Från detta följer de andra egenskaperna som vi fick från banrörelsemängdsmomentet. Någon borde kanske visa detta.

\paragraph{Spinn}
Vi har nu hittat halvtaliga lösningar av $l$ för rörelsemängdsmomentet. Vi tror att dessa kommer beskriva det som kallas för spinn. För att införa spinn, postulerar vi nu att varje partikel har ett intrinsikt rörelsemängdsmoment $\vb{S}$ som uppfyller
\begin{align*}
	\commut{S_{x}}{S_{y}} = i\hbar S_{z},\ \commut{S^{2}}{S_{z}} = 0,
\end{align*}
alltså samma egenskaper som banrörelsemängdsmomentet hade.

Egenvärdet för detta rörelsemängdsmomentet är $\hbar^{2}s(s + 1)$. Värdet av $s$ är fixt för en given sorts partikel, och är:
\begin{itemize}
	\item $\frac{1}{2}$ för materiepartiklar som elektroner, kvarkar, protoner, neutroner osv.
	\item $1$ för kraftpartiklar som fotoner, gluoner och $Z$- och $W$-bosoner.
	\item $0$ för Higgsbosonen.
	\item $2$ för den hypotetiska gravitonen.
\end{itemize}

För att konstruera operatorn, tittar vi på experimenten för $s = \frac{1}{2}$-partiklar. För dessa ersätter $s$ kvanttalet $l$ i halvtaliga fall. Det visar sig att det magnetiska momentet i $z$-riktning ges av $\mu_{z} = -\frac{e}{m}S_{z}$, där $S_{z} = \pm\frac{\hbar}{2}$. Spinnoperatorn kan då i egenbasen konstrueras som
\begin{align*}
	S_{z} = \frac{1}{2}\hbar
	\mqty[
		1 & 0 \\
		0 & -1
	],\
	S^{2} = s(s + 1)\hbar^{2}
	\mqty[
		1 & 0 \\
		0 & 1
	].
\end{align*}
Egentillstånden kallas för spinorer, och betecknas med $\chi_{\pm}$ eller $\ket{\frac{1}{2}, \pm 1}$.

Det motsvarande Hilbertrummet är tvådimensionellt, och vi kan genom vårat postulat införa stegoperatorerna
\begin{align*}
	S_{\pm} = S_{x} \pm iS_{y}.
\end{align*}
De verkar enligt
\begin{align*}
	S_{\pm}\ket{s, m} = \sqrt{s(s + 1) - m(m \pm 1)}\hbar\ket{s, m \pm 1}.
\end{align*}
Vi kan då konstruera matriserna
\begin{align*}
	S_{+} = \hbar
	\mqty[
		0 & 1 \\
		0 & 0
	],\
	S_{-} = \hbar
	\mqty[
		0 & 0 \\
		1 & 0
	].
\end{align*}
Detta ger oss
\begin{align*}
	S_{x} = \frac{1}{2}\hbar
	\mqty[
		0 & 1 \\
		1 & 0
	],\ 
	S_{y} = \frac{1}{2}\hbar
	\mqty[
		0 & -i \\
		i & 0
	].
\end{align*}

\paragraph{Partiklar med spinn i magnetfält}
En laddning $q$ med den givna spinnen får det magnetiska momentet $\vb{\mu} = \gamma\vb{S}$, där $\gamma = \frac{gq}{2m}$ och $g$ är en gyroskopisk faktor. För elektroner är den ungefär $2$ och för protoner är den ungefär $5$.

Välj nu koordinatsystem så att $\vb{B}$ pekar i $z$-riktningen. Hamiltonianen ges då av
\begin{align*}
	H = -\vb{\mu}\cdot\vb{B} = -\frac{gqB}{2m}S_{z} = \omega_{0}S_{z} = \frac{1}{2}\hbar\omega_{0}
	\mqty[
		1 & 0 \\
		0 & -1
	].
\end{align*}
Dens egenvärden är $E_{m}\hbar\omega_{0}$.

\paragraph{Schrödingerekvatinen för en partikel i ett magnetfält}
Den tidsberoende Schrödingerekvationen för en partikel i ett magnetfält blir
\begin{align*}
	i\hbar\dv{\chi}{t} = H\chi.
\end{align*}
En allmän spinor för en $s = \frac{1}{2}$-partikel ges då av
\begin{align*}
	\chi = a\chi_{+}e^{-i\frac{E_{+}t}{\hbar}} + b\chi_{-}e^{-i\frac{E_{-}t}{\hbar}}.
\end{align*}
Normeringsvillkoret för såna tillstånd uppmanar oss att skriva $a = \cos{\frac{\theta}{2}}$ och $b = \sin{\frac{\theta}{2}}$. Vi får med detta
\begin{align*}
	\expval{S_{x}} &= \frac{1}{2}\hbar\sin{\theta}\cos{\omega_{0}t}, \\
	\expval{S_{y}} &= \frac{1}{2}\hbar\sin{\theta}\sin{\omega_{0}t}, \\
	\expval{S_{z}} &= \frac{1}{2}\hbar\cos{\theta}.
\end{align*}
Alltså precesserar spinnet kring $z$-axeln med frekvens $\omega_{0}$ i ett plan som ges av initialtillståndet.

\paragraph{Addition av rörelsemängdsmoment}
Betrakta det totala spinnet $\vb{S} = \vb{S}_{1} + \vb{S}_{2}$ för ett system. Eftersom de två termerna opererar på olika koordinater, gäller det att
\begin{align*}
	\commut{S_{x}}{S_{y}} = i\hbar S_{z}.
\end{align*}
Vi har nu två val av bas för Hilbertrummet:
\begin{itemize}
	\item Produktbasen $\ket{s_{1}, s_{2}, m_{1}, m_{2}} = \ket{s_{1},m_{1}}\otimes\ket{s_{2}, m_{2}}$. Detta visar sig inte vara en bra bas.
	\item Den kopplade basen $\ket{s_{1}, s_{2}, s, m}$, som även skrivs $\ket{s, m}$.
\end{itemize}
För att göra basbytet, kan vi använda fullständighetsrelationen
\begin{align*}
	\ket{s, m} = \sum\limits_{m_{1}, m_{2}}\ket{s_{1}, s_{2}, m_{1}, m_{2}}\braket{s_{1}, s_{2}, m_{1}, m_{2}}{s, m}.
\end{align*}
Vi definierar då Clebsh-Gordankoefficienterna
\begin{align*}
	C_{m_{1}, m_{2}, m}^{s_{1}, s_{2}, m} = \braket{s_{1}, s_{2}, m_{1}, m_{2}}{s, m}.
\end{align*}
Dessa är endast nollskilda om $m = m_{1} + m_{2}$. Detta ses eftersom $S_{z} = S_{1z} + S_{2z}$. Vidare kan $s$ anta värden från $\abs{s_{1} - s_{2}}$ till $s_{1} + s_{2}$. Detta ses eftersom produktbasen innehåller $(2s_{1} + 1)(2s_{2} + 1)$ basvektorer. Om vi antar $s_{1} > s_{2}$, kan vi testa påståendet i den kopplade basen. Det totala antalet basvektorer är då
\begin{align*}
	\sum\limits_{s_{1} - s_{2}}^{s_{1} + s_{2}}2s + 1 = (2s_{1} + 1)(2s_{2} + 1).
\end{align*}