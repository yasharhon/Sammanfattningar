\subsection{Andra ordningen}

\paragraph{Entydighet av lösning}
Betrakta den andra ordningens ODE
\begin{align*}
	\deval[2]{y}{t}{t} + p(t)\deval{y}{t}{t} + q(t)y(t) &= g(t),\ y > t_0, \\
	y(t_0)                                              &= y_0, \\
	\deval{y}{t}{t_{0}}                                 &= y'_{0}.
\end{align*}
Den har en entydig lösning om $p, q$ är Lipschitzkontinuerliga.

\paragraph{Form på lösning av andra ordningens ODE}
Betrakta den andra ordningens ODE
\begin{align*}
	\deval[2]{y}{t}{t} + p(t)\deval{y}{t}{t} + q(t)y(t) = L(t, y) = g(t).
\end{align*}
Låt $y_{\text{P}}$ vara en partikulär lösning till denna. Då är $y$ en lösning om och endast om
\begin{align*}
	y = y_{\text{H}} + y_{\text{P}},
\end{align*}
där $y_{\text{H}}$ löser den homogena ekvationen.

\subparagraph{Bevis}
Vi har
\begin{align*}
	L(t, y) = L(t, y_{\text{P}} + y_{\text{H}}) = L(t, y_{\text{P}}) + L(t, y_{\text{H}}) = g(t) + 0 = g(t),
\end{align*}
och därmed löser $y$ differentialekvationen. Vi har även
\begin{align*}
	L(t, y - y_{\text{P}}) = g(t) - g(t) = 0,
\end{align*}
och $y - y_{\text{P}}$ löser den homogena ekvationen. Eftersom detta är sant, har vi visat ekvivalens.

\paragraph{Fundamentala lösningar}
Betrakta
\begin{align*}
	\deval[2]{y}{t}{t} + p(t)\deval{y}{t}{t} + q(t)y(t) &= g(t),\ t\in I,
\end{align*}
där $p, q, g$ är kontinuerliga på $I$. Låt $y_{1}$ uppfylla
\begin{align*}
	y_{1}(t_0) = 1, \deval{y_{1}}{t}{t_0} = 0
\end{align*}
och $y_{2}$ uppfylla
\begin{align*}
	y_{2}(t_0) = 0, \deval{y_{2}}{t}{t_0} = 1.
\end{align*}
Då definieras $y_{1}, y_{2}$ som mängden av fundamentala lösningar av differentialekvationen.

\paragraph{Linjär kombination av lösningar}
Betrakta
\begin{align*}
	\deval[2]{y}{t}{t} + p(t)\deval{y}{t}{t} + q(t)y(t) &= g(t),\ t > t_0, \\
	y(t_0)                                              &= y_0, \\
	\deval{y}{t}{t_{0}}                                 &= y'_{0}
\end{align*}
och anta att $y_{1}, y_{2}$ är lösningar. Då finns det $c_{1}, c_{2}$ så att $y = c_{1}y_{1} + c_{2}y_{2}$ är en lösning om $W(y_{1}, y_{2})(t_{0}) \neq 0$.

\paragraph{Abels sats}
Betrakta
\begin{align*}
	\deval[2]{y}{t}{t} + p(t)\deval{y}{t}{t} + q(t)y(t) &= g(t),\ t\in I, \\
	y(t_0)                                              &= y_0, \\
	\deval{y}{t}{t_{0}}                                 &= y'_{0}
\end{align*}
och anta att $y_{1}, y_{2}$ är lösningar. Då gäller att
\begin{align*}
	W(y_{1}, y_{2})(t) = W(y_{1}, y_{2})(t_{0})e^{-\inteval{t_{0}}{t}{p(s)}{s}}.
\end{align*}

\subparagraph{Bevis}
\begin{align*}
	\deval{W}{t}{t} &= \deval{y_{1}}{t}{t}\deval{y_{2}}{t}{t} - \deval{y_{1}}{t}{t}\deval{y_{2}}{t}{t} + y_{1}\deval[2]{y_{2}}{t}{t} - y_{2}\deval[2]{y_{1}}{t}{t} \\
	                &= y_{1}\left(-p(t)\deval{y_{2}}{t}{t} + q(t)y_{2}(t)\right) - y_{2}\left(-p(t)\deval{y_{1}}{t}{t} + q(t)y_{1}(t)\right) \\
	                &= -p(t)W(y_{1}, y_{2})(t).
\end{align*}
Denna differentialekvationen har lösning
\begin{align*}
	W(y_{1}, y_{2})(t) = W(y_{1}, y_{2})(t_{0})e^{-\inteval{t_{0}}{t}{p(s)}{s}},
\end{align*}
vilket skulle visas.

\paragraph{Linjärt beroende av lösningar}
Betrakta
\begin{align*}
	\deval[2]{y}{t}{t} + p(t)\deval{y}{t}{t} + q(t)y(t) &= g(t),\ t\in I, \\
	y(t_0)                                              &= y_0, \\
	\deval{y}{t}{t_{0}}                                 &= y'_{0}
\end{align*}
och anta att $y_{1}, y_{2}$ är lösningar. Då är dessa linjärt beroende på $I$ om och endast om $W(y_{1}, y_{2})(t) = 0$.

\subparagraph{Bevis}
Om dessa är linjärt beroende, ser man att Wronskianen blir lika med $0$, då kolumnerna i matrisen vars determinant ger Wronskianen kommer vara multipler av varandra.

\paragraph{Lösning av andra ordningens ODE med konstanta koefficienter}
Låt $r_1, r_2$ vara lösningar till
\begin{align*}
	r^2 + pr + q = 0.
\end{align*}
Då ges lösningarna till
\begin{align*}
	\deval[2]{y}{t}{t} + p\deval{y}{t}{t} + qy(t) = L(t, y) = 0
\end{align*}
av
\begin{align*}
	y(t) = 
	\begin{cases}
		c_1e^{r_1t} + c_2e^{r_2t},\ &r_1\neq r_2, \\
		(c_1t + c_2)e^{r_1t},\      &r_1 = r_2.
	\end{cases}
\end{align*}

\paragraph{Variation av parametrar}
Betrakta
\begin{align*}
	\deval[2]{y}{t}{t} + p(t)\deval{y}{t}{t} + q(t)y(t) &= g(t),\ t\in I
\end{align*}
där $p, q, g$ är kontinuerliga på $I$ och $y_{1}, y_{2}$ är lösningar av den motsvarande homogena ekvationen, ges en partikulär lösning av ekvationen av
\begin{align*}
	y_{\text{p}} = -y_{1}\inteval{t_{0}}{t}{\frac{y_{2}(s)g(s)}{W(y_{1}, y_{2})(s)}}{s} + y_{2}\inteval{t_{0}}{t}{\frac{y_{1}(s)g(s)}{W(y_{1}, y_{2})(s)}}{s}
\end{align*}
där $t_{0}\in I$.

\paragraph{Eulerekvationer}
Betrakta en ekvation på formen
\begin{align*}
	x^{2}\deval[2]{y}{x}{x} + ax\deval{y}{x}{x} + by = 0.
\end{align*}
För att hitta lösningar, gör ansatsen $y(x) = x^{r}$. Om detta är en lösning, uppfyller $r$
\begin{align*}
	r(r - 1) + ar + b = 0.
\end{align*}
I fallet att ekvationen över har en dubbelrot, är den andra lösningen $y_{2}(x) = x^{r}\ln{\abs{x}}$.