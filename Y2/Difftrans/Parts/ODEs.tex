\section{Ordinarie differentialekvationer (ODE)}

\subsection{Användbara defitioner och satser}

\paragraph{Lipschitzkontinuitet}
En funktion $f$ är Lipschitzkontinuerlig om det finns ett $K$ så att det för varje $x_1, x_2$ gäller att
\begin{align*}
	\abs{f(x_1) - f(x_2)} \leq K\abs{x_1 - x_2}.
\end{align*}

\paragraph{Lipschitzkontinuitet och deriverbarhet}
Låt $f\in C^{1}$. Då är $f$ Lipschitzkontinuerlig.

\paragraph{Grönwalls lemma}
Antag att det finns positiva $A, K$ så att $h: [0, T\to\R]$ uppfyller
\begin{align*}
	h(t) \leq K\inteval{0}{t}{h(s)}{s} + A.
\end{align*}
Då gäller att
\begin{align*}
	h(t) \leq Ae^{Kt}.
\end{align*}

\subparagraph{Bevis}
Definiera
\begin{align*}
	I(t) = \inteval{0}{t}{h(s)}{s}.
\end{align*}
Då gäller att
\begin{align*}
	\deval{I}{t}{t} = h(t) \leq KI(t) + A.
\end{align*}
Denna differentialolikheten kan vi lösa vid att tillämpa integrerande faktor. Detta kommer att ge
\begin{align*}
	\dv{t}\left(e^{-Kt}I(t)\right) \leq Ae^{-Kt}.
\end{align*}
Vi integrerar från $0$ till $r$ och använder att $I(0) = 0$ för att få
\begin{align*}
	I(r) \leq \frac{A}{K}(e^{Kr} - 1).
\end{align*}
Derivation på båda sidor ger
\begin{align*}
	h(r) \leq Ae^{Kr},
\end{align*}
vilket skulle visas.

\paragraph{Positivt definitiva funktioner}
Låt $D$ vara en öppen omgivning av $\vb{0}$. Funktionen $V$ är positivt definitiv om $V(\vb{0}) = 0$ och $V(\vb{x}) > 0,\ \vb{x}\neq\vb{0}$. Definitionen är analog för negativt definitiva funktioner. Vid att inkludera likheten i olikhetstecknet fås också definitionen av positivt och negativt semidefinitiva funktioner.

\paragraph{Analytiska funktioner}
En funktion är analytisk om den lokalt beskrivs av en potensserie.

\paragraph{Potenser av matriser}
Vi definierar
\begin{align*}
	e^{At} = I + \sum\limits_{n = 1}^{\infty}\frac{A^{n}t^{n}}{n!}.
\end{align*}

\paragraph{Eulers metod}
Betrakta differentialekvationen
\begin{align*}
	\deval{y}{t}{t} &= f(t, y),\ 0 < t < T, \\
	y(0)             &= y_0.
\end{align*}
Vi gör indelningen $t_n = n\Delta t, n = 0, 1, \dots, N$ så att $\Delta t = \frac{T}{N}$ och inför $y_n = y(t_n)$. Vidare gör vi approximationen
\begin{align*}
	\frac{y_{n + 1} - y_{n}}{\Delta t} = f(t_n, y).
\end{align*}
Vi utvidgar nu Eulerapproximationen $\bar{y}$ till en styckvis linjär funktion som definieras enligt
\begin{align*}
	y(t) - y(t_{n}) = f(t_n, y)(t - t_{n}),\ t_n \leq t < t_{n + 1}.
\end{align*}
Denna uppfyller
\begin{align*}
	\dv{\bar{y}}{t} = f(\bar{y}_{n}) = \bar{f}(t, \bar{y}),\ t_n \leq t < t_{n + 1}.
\end{align*}

\paragraph{Konvergens av Eulers metod}
Betrakta differentialekvationen
\begin{align*}
	\deval{y}{t}{t} &= f(t, y),\ 0 < t < T, \\
	y(0)             &= y_0,
\end{align*}
där $f$ är Lipschitzkontinuerlig, låt $\bar{y}, \bar{\bar{y}}$ vara två Eulerapproximationer av denna, med indelningar $\bar{t}_n = n\frac{T}{N}, n = 0, 1, \dots, N$ respektive $\bar{\bar{t}}_{m} = m\frac{T}{M}, n = 0, 1, \dots, M$ och inför $\Delta t = \max\left(\frac{T}{N}, \frac{T}{M}\right)$. Antag vidare att det finns ett $C$ så att
\begin{align*}
	\max\left(\abs{f(0)}, \abs{y(0)}\right) &\leq C, \\
	\abs{f(a) - f(b)}                       &\leq C\abs{a - b}.
\end{align*}
Då finns det $B_{1}, B_{2}$ så att
\begin{align*}
	\max\limits_{t\in [0, T]}\left(\abs{\bar{y}(t)}, \abs{\bar{\bar{y}}(t)}\right) &\leq B_{1}, \\
	\max\limits_{t\in [0, T]}\abs{\bar{y}(t) - \bar{\bar{y}}(t)}                   &\leq B_{2}\Delta t.
\end{align*}

\subparagraph{Bevis}
Vi beviser det första påståendet först.

Lipschitzkontinuitet av $f$ ger
\begin{align*}
	\abs{f(z)} \leq C\abs{z} + \abs{f(0)} \leq C(1 + \abs{z}).
\end{align*}
Eulers metod ger
\begin{align*}
	\bar{y}(\bar{t}_{n}) = \bar{y}(\bar{t}_{n - 1}) + \frac{T}{N}f(\bar{y}(\bar{t}_{n - 1})).
\end{align*}
Dessa två ger till sammans
\begin{align*}
	\abs{\bar{y}(\bar{t}_{n})} &\leq \abs{\bar{y}(\bar{t}_{n - 1})} + \frac{T}{N}\abs{f(\bar{y}(\bar{t}_{n - 1}))} \\
	                           &\leq \abs{\bar{y}(\bar{t}_{n - 1})} + C\frac{T}{N}(1 + \abs{\bar{y}(\bar{t}_{n - 1})}) \\
	                           &= (1 + C\frac{T}{N})\abs{\bar{y}(\bar{t}_{n - 1})} + C\Delta t.
\end{align*}
Vi använder induktion på detta resultatet och får
\begin{align*}
	\abs{\bar{y}(\bar{t}_{n})} &\leq (1 + C\frac{T}{N})^{n}\abs{\bar{y}(0)} + C\frac{T}{N}\frac{(1 + C\frac{T}{N})^{n} - 1}{C\frac{T}{N}} \\
	                           &= (1 + C\frac{T}{N})^{n}\abs{\bar{y}(0)} + (1 + C\frac{T}{N})^{n} - 1.
\end{align*}
Vi vet även att
\begin{align*}
	(1 + C\frac{T}{N})^{n} \leq e^{Cn\frac{T}{N}} = e^{C\bar{t}_{n}},
\end{align*}
vilket slutligen ger
\begin{align*}
	\abs{\bar{y}(\bar{t}_{n})} = e^{C\bar{t}_{n}}\abs{\bar{y}(0)} + e^{C\bar{t}_{n}} - 1.
\end{align*}
En motsvarande gräns kan fås för $\bar{\bar{y}}$, vilket slutför beviset.

Vidare bevisar vi det andra påståendet. Skillnaden mellan de två approximationerna ges av
\begin{align*}
	\bar{y}(t) - \bar{\bar{y}}(t) &= \bar{y}(0) - \bar{\bar{y}}(0) + \inteval{0}{t}{\bar{f}(t, \bar{y}) - \bar{\bar{f}}(t, \bar{\bar{y}})}{t} \\
	                              &= \inteval{0}{t}{\bar{f}(t, \bar{y}) - \bar{\bar{f}}(t, \bar{\bar{y}})}{t}.
\end{align*}
Betrakta ett $t\in [\bar{t}_{n}, \bar{t}_{n + 1})\cup [\bar{\bar{t}}_{m}, \bar{\bar{t}}_{m + 1})$. Vi adderar och subtraherar $f(\bar{y}(t))$ och $f(\bar{\bar{y}}(t))$ från integranden och får
\begin{align*}
	\bar{f}(t, \bar{y}) - \bar{\bar{f}}(t, \bar{\bar{y}} &= f(\bar{y}(\bar{t}_{n})) - f(\bar{\bar{y}}(\bar{\bar{t}}_{m})) \\
	                                                     &= (f(\bar{y}(\bar{t}_{n})) - f(\bar{y}(t))) + (f(\bar{\bar{y}}(t))) - f(\bar{\bar{y}}(\bar{\bar{t}}_{m}))) + (f(\bar{y}(t)) - f(\bar{\bar{y}}(t))) \\
	                                                     &= R_{1} + R_{2} + R_{3}.
\end{align*}
Lipschitzantagandet för $f$ ger
\begin{align*}
	\abs{f(\bar{y}(\bar{t}_{n})) - f(\bar{y}(t)} \leq C\abs{\bar{y}(\bar{t}_{n}) - \bar{y}(t)}.
\end{align*}
Med hjälpresultatet för $\abs{f(z)}$ kan vi skriva
\begin{align*}
	\abs{\bar{y}(\bar{t}_{n}) - \bar{y}(t)} = (t - t_{n})\abs{f(\bar{y}(\bar{t}_{n}))} \leq C(1 + \abs{\bar{y}(\bar{t}_{n})})(t - t_{n})
\end{align*}
och slutligen
\begin{align*}
	\abs{R_{1}} &\leq C^{2}(1 + \abs{\bar{y}(\bar{t}_{n})})(t - t_{n}), \\
	\abs{R_{2}} &\leq C^{2}(1 + \abs{\bar{\bar{y}}(\bar{\bar{t}}_{m})})(t - t_{m}), \\
	\abs{R_{3}} &\leq C\abs{\bar{y}(t) - \bar{\bar{y}}(t)},
\end{align*}
där antaganden igen har användts.

Integranden kan nu skrivas som
\begin{align*}
	\abs{\bar{f}(t, \bar{y}) - \bar{\bar{f}}(t, \bar{\bar{y}})} &\leq C^{2}(1 + \abs{\bar{y}(\bar{t}_{n})})(t - t_{n}) + C^{2}(1 + \abs{\bar{\bar{y}}(\bar{\bar{t}}_{m})})(t - t_{m}) + \leq C\abs{\bar{y}(t) - \bar{\bar{y}}(t)}.
\end{align*}
Om vi antar att det första påsåtåendet i satsen stämmer, fås
\begin{align*}
	\abs{\bar{f}(t, \bar{y}) - \bar{\bar{f}}(t, \bar{\bar{y}})} &\leq C^{2}(1 + B_{1})\Delta t + C\abs{\bar{y}(t) - \bar{\bar{y}}(t)},
\end{align*}
och integralen kan skrivas som
\begin{align*}
	\bar{y}(t) - \bar{\bar{y}}(t) &\leq \inteval{0}{t}{C^{2}(1 + B_{1})\Delta t + C\abs{\bar{y}(t) - \bar{\bar{y}}(t)}}{t} \\
	                              &\leq \inteval{0}{t}{C^{2}(1 + B_{1})\Delta t + C\abs{\bar{y}(t) - \bar{\bar{y}}(t)}}{t} \\
	                              &\leq \inteval{0}{t}{C\abs{\bar{y}(t) - \bar{\bar{y}}(t)}}{t} + C^{2}(1 + B_{1})T\Delta t \\
	                              &= \inteval{0}{t}{C\abs{\bar{y}(t) - \bar{\bar{y}}(t)}}{t} + C_{1}T\Delta t.
\end{align*}
Grönwalls lemma ger slutligen
\begin{align*}
	\bar{y}(t) - \bar{\bar{y}}(t) \leq C_{1}T\Delta te^{CT}.
\end{align*}

\paragraph{Linjära differentialekvationer}
Om en differentialekvation kan skrivas på formen $F(t, y, \dv{y}{x}, \dots) = 0$, är den linjär om $F$ är linjär i alla sina argument förutom $t$.

\paragraph{Wronskianen}
Wronskianen definieras som
\begin{align*}
	W(y_1, y_2)(t) = \mdet{y_{1}(t) & y_{2}(t) \\ \deval{y_{1}}{t}{t} & \deval{y_{2}}{t}{t}}.
\end{align*}
För vektorvärda funktioner definieras den som determinanten av matrisen vars kolumner är de olika funktionerna.

\paragraph{Linjärt beroende funktioner}
$f: I\to\R, g: I\to\R$ är linjärt beroende om det finns $k_{1}, k_{2}$ så att
\begin{align*}
	k_{1}f(t) + k_{2}g(t) = 0\ \forall\ t\in I.
\end{align*}

\paragraph{Fundamentalt sätt av lösningar}
Betrakta någon ODE och ett sätt lösningar. Detta sättet är ett fundamentalt sätt av lösningar om och endast om deras wronskian är nollskild överallt i lösningsintervallet.

\paragraph{Ordinarie punkter}
Betrakta differentialekvationen
\begin{align*}
	P(x)\deval[2]{y}{x}{x} + Q(x)\deval{y}{x}{x} + R(x)y(x) = 0.
\end{align*}
Vi skriver denna om till
\begin{align*}
	\deval[2]{y}{x}{x} + p(x)\deval{y}{x}{x} + q(x)y(x) = 0.
\end{align*}
Om både $p$ och $q$ är analytiske kring punkten $x_{0}$, är $x_{0}$ en ordinarie punkt till differentialekvationen.

\paragraph{Singulära punkter}
Betrakta differentialekvationen
\begin{align*}
	P(x)\deval[2]{y}{x}{x} + Q(x)\deval{y}{x}{x} + R(x)y(x) = 0.
\end{align*}
Vi skriver denna om till
\begin{align*}
	\deval[2]{y}{x}{x} + p(x)\deval{y}{x}{x} + q(x)y(x) = 0.
\end{align*}
Om antingen $Q$ eller $R$ är nollskilda i $x_{0}$, medan $P(x_{0}) = 0$, är $x_{0}$ en singulär punkt till differentialekvationen.

\paragraph{Regulära singulära punkter}
Betrakta differentialekvationen
\begin{align*}
	P(x)\deval[2]{y}{x}{x} + Q(x)\deval{y}{x}{x} + R(x)y(x) = 0.
\end{align*}
Vi skriver denna om till
\begin{align*}
	x^{2}\deval[2]{y}{x}{x} + x(xp(x))\deval{y}{x}{x} + x^{2}q(x)y(x) = 0.
\end{align*}
Om antingen $p$ eller $q$ ej är analytiska kring $x_{0}$, men $xp$ och $x^{2}q$ är det, är $x_{0}$ en regulär singulär punkt till differentialekvationen.

\subsection{Första ordningen}

\paragraph{Entydighet av lösning}
Betrakta differentialekvationen
\begin{align*}
	\deval{y}{t}{t} &= f(y), \\
	y(0)            &= y_0.
\end{align*}
Detta har en unik lösning om $f$ är Lipschitzkontinuerlig.

Observera att beviset kan även göras för en funktion $f(t, y)$ vid att skriva differentialekvationen som ett system och komma med en motsvarande sats för system av differentialekvationer.

\subparagraph{Bevis}
Betrakta två lösningar $y, z$ av differentialekvationen. Vi får
\begin{align*}
	y(\tau) - y_{0} = \inteval{0}{\tau}{f(y)}{t}
\end{align*}
och samma för $z$. Vi subtraherar dessa två resultat och får
\begin{align*}
	y(\tau) - z(\tau) = y_{0} - z_{0} + \inteval{0}{\tau}{f(y) - f(z)}{t}.
\end{align*}
Vid att beräkna absolutbeloppet av båda sidor och använda Cauchy-Schwarz' oliket får man vidare
\begin{align*}
	\abs{y(\tau) - z(\tau)} \leq \abs{y_{0} - z_{0}} + \inteval{0}{\tau}{\abs{f(y) - f(z)}}{t}.
\end{align*}
Kravet om Lipschitzkontinuitet av $f$ ger vidare
\begin{align*}
	\abs{y(\tau) - z(\tau)} \leq \abs{y_{0} - z_{0}} + \inteval{0}{\tau}{K\abs{y(t) - z(t)}}{t}.
\end{align*}
Grönwalls lemma ger slutligen
\begin{align*}
	\abs{y(\tau) - z(\tau)} \leq \abs{y_{0} - z_{0}}e^{K\tau}.
\end{align*}
Om $y_{0} = z_{0}$ är $y = z$, och beviset är klart.

\paragraph{Lösning av linjära ODE av första ordning}
Antag att vi har en differentialekvation på formen
\begin{align*}
	\deval{y}{t}{t} + p(t)y(t) = g(t).
\end{align*}
Beräkna
\begin{align*}
	P(t) = \inteval{a}{t}{p}{x}
\end{align*}
och inför den integrerande faktorn $e^{P(t)}$. Multiplicera med den på båda sidor för att få
\begin{align*}
	e^{P(t)}\deval{y}{t}{t} + p(t)e^{P(t)}y(t) = e^{P(t)}g(t).
\end{align*}
Detta kan skrivas om till
\begin{align*}
	\dv{t}\left(ye^{P}\right)(t) = e^{P(t)}g(t) = \deval{H}{t}{t}.
\end{align*}
Analysens huvudsats ger då
\begin{align*}
	y(t)e^{P(t)} = H(t) + c
\end{align*}
och slutligen
\begin{align*}
	y(t) = ce^{-P(t)} + e^{-P(t)}H(t).
\end{align*}

Låt oss lägga till bivillkoret $y(a) = y_0$. Man kan då visa att lösningen kan skrivas som
\begin{align*}
	y(t) = y_0e^{-\inteval{a}{t}{p}{x}} + \inteval{a}{t}{g(x)e^{-\inteval{x}{t}{p}{s}}}{x}.
\end{align*}

\paragraph{Separabla ODE av första ordning}
Antag att vi har en differentialekvation som kan skrivas på formen
\begin{align*}
	m(x) + n(y(x))\deval{y}{x}{x} = 0.
\end{align*}
Denna betecknas som en separabel ODE av första ordning.

För att lösa den, beräkna primitiv funktion på båda sidor, vilket ger
\begin{align*}
	M(x) + N(y(x)) = c,\ c\in\R.
\end{align*}
Om $N$ är inverterbar, får man då $y$ enligt
\begin{align*}
	y(x) = N^{-1}(c - M(x)).
\end{align*}

\subsection{Andra ordningen}

\paragraph{Entydighet av lösning}
Betrakta den andra ordningens ODE
\begin{align*}
	\deval[2]{y}{t}{t} + p(t)\deval{y}{t}{t} + q(t)y(t) &= g(t),\ y > t_0, \\
	y(t_0)                                              &= y_0, \\
	\deval{y}{t}{t_{0}}                                 &= y'_{0}.
\end{align*}
Den har en entydig lösning om $p, q$ är Lipschitzkontinuerliga.

\paragraph{Form på lösning av andra ordningens ODE}
Betrakta den andra ordningens ODE
\begin{align*}
	\deval[2]{y}{t}{t} + p(t)\deval{y}{t}{t} + q(t)y(t) = L(t, y) = g(t).
\end{align*}
Låt $y_{\text{P}}$ vara en partikulär lösning till denna. Då är $y$ en lösning om och endast om
\begin{align*}
	y = y_{\text{H}} + y_{\text{P}},
\end{align*}
där $y_{\text{H}}$ löser den homogena ekvationen.

\subparagraph{Bevis}
Vi har
\begin{align*}
	L(t, y) = L(t, y_{\text{P}} + y_{\text{H}}) = L(t, y_{\text{P}}) + L(t, y_{\text{H}}) = g(t) + 0 = g(t),
\end{align*}
och därmed löser $y$ differentialekvationen. Vi har även
\begin{align*}
	L(t, y - y_{\text{P}}) = g(t) - g(t) = 0,
\end{align*}
och $y - y_{\text{P}}$ löser den homogena ekvationen. Eftersom detta är sant, har vi visat ekvivalens.

\paragraph{Fundamentala lösningar}
Betrakta
\begin{align*}
	\deval[2]{y}{t}{t} + p(t)\deval{y}{t}{t} + q(t)y(t) &= g(t),\ t\in I,
\end{align*}
där $p, q, g$ är kontinuerliga på $I$. Låt $y_{1}$ uppfylla
\begin{align*}
	y_{1}(t_0) = 1, \deval{y_{1}}{t}{t_0} = 0
\end{align*}
och $y_{2}$ uppfylla
\begin{align*}
	y_{2}(t_0) = 0, \deval{y_{2}}{t}{t_0} = 1.
\end{align*}
Då definieras $y_{1}, y_{2}$ som mängden av fundamentala lösningar av differentialekvationen.

\paragraph{Linjär kombination av lösningar}
Betrakta
\begin{align*}
	\deval[2]{y}{t}{t} + p(t)\deval{y}{t}{t} + q(t)y(t) &= g(t),\ t > t_0, \\
	y(t_0)                                              &= y_0, \\
	\deval{y}{t}{t_{0}}                                 &= y'_{0}
\end{align*}
och anta att $y_{1}, y_{2}$ är lösningar. Då finns det $c_{1}, c_{2}$ så att $y = c_{1}y_{1} + c_{2}y_{2}$ är en lösning om $W(y_{1}, y_{2})(t_{0}) \neq 0$.

\paragraph{Abels sats}
Betrakta
\begin{align*}
	\deval[2]{y}{t}{t} + p(t)\deval{y}{t}{t} + q(t)y(t) &= g(t),\ t\in I, \\
	y(t_0)                                              &= y_0, \\
	\deval{y}{t}{t_{0}}                                 &= y'_{0}
\end{align*}
och anta att $y_{1}, y_{2}$ är lösningar. Då gäller att
\begin{align*}
	W(y_{1}, y_{2})(t) = W(y_{1}, y_{2})(t_{0})e^{-\inteval{t_{0}}{t}{p(s)}{s}}.
\end{align*}

\subparagraph{Bevis}
\begin{align*}
	\deval{W}{t}{t} &= \deval{y_{1}}{t}{t}\deval{y_{2}}{t}{t} - \deval{y_{1}}{t}{t}\deval{y_{2}}{t}{t} + y_{1}\deval[2]{y_{2}}{t}{t} - y_{2}\deval[2]{y_{1}}{t}{t} \\
	                &= y_{1}\left(-p(t)\deval{y_{2}}{t}{t} + q(t)y_{2}(t)\right) - y_{2}\left(-p(t)\deval{y_{1}}{t}{t} + q(t)y_{1}(t)\right) \\
	                &= -p(t)W(y_{1}, y_{2})(t).
\end{align*}
Denna differentialekvationen har lösning
\begin{align*}
	W(y_{1}, y_{2})(t) = W(y_{1}, y_{2})(t_{0})e^{-\inteval{t_{0}}{t}{p(s)}{s}},
\end{align*}
vilket skulle visas.

\paragraph{Linjärt beroende av lösningar}
Betrakta
\begin{align*}
	\deval[2]{y}{t}{t} + p(t)\deval{y}{t}{t} + q(t)y(t) &= g(t),\ t\in I, \\
	y(t_0)                                              &= y_0, \\
	\deval{y}{t}{t_{0}}                                 &= y'_{0}
\end{align*}
och anta att $y_{1}, y_{2}$ är lösningar. Då är dessa linjärt beroende på $I$ om och endast om $W(y_{1}, y_{2})(t) = 0$.

\subparagraph{Bevis}
Om dessa är linjärt beroende, ser man att Wronskianen blir lika med $0$, då kolumnerna i matrisen vars determinant ger Wronskianen kommer vara multipler av varandra.

\paragraph{Lösning av andra ordningens ODE med konstanta koefficienter}
Låt $r_1, r_2$ vara lösningar till
\begin{align*}
	r^2 + pr + q = 0.
\end{align*}
Då ges lösningarna till
\begin{align*}
	\deval[2]{y}{t}{t} + p\deval{y}{t}{t} + qy(t) = L(t, y) = 0
\end{align*}
av
\begin{align*}
	y(t) = 
	\begin{cases}
		c_1e^{r_1t} + c_2e^{r_2t},\ &r_1\neq r_2, \\
		(c_1t + c_2)e^{r_1t},\      &r_1 = r_2.
	\end{cases}
\end{align*}

\paragraph{Variation av parametrar}
Betrakta
\begin{align*}
	\deval[2]{y}{t}{t} + p(t)\deval{y}{t}{t} + q(t)y(t) &= g(t),\ t\in I
\end{align*}
där $p, q, g$ är kontinuerliga på $I$ och $y_{1}, y_{2}$ är lösningar av den motsvarande homogena ekvationen, ges en partikulär lösning av ekvationen av
\begin{align*}
	y_{\text{p}} = -y_{1}\inteval{t_{0}}{t}{\frac{y_{2}(s)g(s)}{W(y_{1}, y_{2})(s)}}{s} + y_{2}\inteval{t_{0}}{t}{\frac{y_{1}(s)g(s)}{W(y_{1}, y_{2})(s)}}{s}
\end{align*}
där $t_{0}\in I$.

\paragraph{Eulerekvationer}
Betrakta en ekvation på formen
\begin{align*}
	x^{2}\deval[2]{y}{x}{x} + ax\deval{y}{x}{x} + by = 0.
\end{align*}
För att hitta lösningar, gör ansatsen $y(x) = x^{r}$. Om detta är en lösning, uppfyller $r$
\begin{align*}
	r(r - 1) + ar + b = 0.
\end{align*}
I fallet att ekvationen över har en dubbelrot, är den andra lösningen $y_{2}(x) = x^{r}\ln{\abs{x}}$.

\subsection{System av ODE}

\paragraph{Formulering}
Betrakta ett system av funktioner $x_{1}, x_{2}, \dots$ som beskrivs av systemet
\begin{align*}
	\deval{x_{1}}{t}{t} &= g_{1}(t) + \sum p_{1i}(t)x_{i}, \\
	\deval{x_{2}}{t}{t} &= g_{2}(t) + \sum p_{2i}(t)x_{i}, \\
	\vdots
\end{align*}
av differentialekvationer. Definiera
\begin{align*}
	\vb{x}(t) =\left[
	\begin{array}{c}
		x_{1}(t) \\
		x_{2}(t) \\
		\vdots
	\end{array}\right],
	\vb{g}(t) =\left[
	\begin{array}{c}
		g_{1}(t) \\
		g_{2}(t) \\
		\vdots
	\end{array}\right],
	P(t) =\left[
	\begin{array}{ccc}
		p_{11}(t) & p_{12}(t) & \dots \\
		p_{21}(t) & p_{22}(t) & \dots \\
		\vdots    & \vdots    & \ddots
	\end{array}\right].
\end{align*}
Då kan systemet skrivas som
\begin{align*}
	\deval{\vb{x}}{t}{t} = \vb{g}(t) + P\vb{x}(t).
\end{align*}
Detta kan även generaliseras till
\begin{align*}
	\deval{\vb{x}}{t}{t} = \vb{f}(\vb{x}(t)) + \vb{g}(t).
\end{align*}

\paragraph{Autonoma system}
Ett autonomt system är på formen
\begin{align*}
	\deval{\vb{x}}{t}{t} = \vb{f}(\vb{x}(t)).
\end{align*}

\paragraph{Form på lösning av system av ODE}
Låt $\vb{x}_{\text{p}}$ lösa
\begin{align*}
	\deval{\vb{x}}{t}{t} = \vb{g}(t) + P\vb{x}(t).
\end{align*}
Då är alla lösningar på formen
\begin{align*}
	\vb{x} = \vb{x}_{\text{p}} + \vb{x}_{\text{h}}
\end{align*}
där $\vb{x}_{\text{h}}$ löser det motsvarande homogena systemet.

\paragraph{Fundamentalmatris}
Betrakta
\begin{align*}
	\deval{\vb{x}}{t}{t} = P(t)\vb{x}(t)
\end{align*}
med fundamentalt sätt av lösningar $\vb{x}^{(1)}(t), \dots, \vb{x}^{(n)}(t)$. Då definieras systemets fundamentalmatris som
\begin{align*}
	\Psi = \left[\vb{x}^{(1)}(t) \dots \vb{x}^{(n)}(t)\right].
\end{align*}
Vi definierar även den speciella fundamentalmatrisen $\Phi$, vars kolumner satisfierar begynnelsesvillkoret
\begin{align*}
	\vb{x}^{(1)}(t_{0}) =
	\left[\begin{array}{c}
		1      \\
		0      \\
		\vdots \\
		0
	\end{array}\right],
	\dots,
	\vb{x}^{(n)}(t_{0}) =
	\left[\begin{array}{c}
		0      \\
		0      \\
		\vdots \\
		1
	\end{array}\right].
\end{align*}
Det kan visas att denna ges av
\begin{align*}
	\Phi(t) = e^{A(t)t}.
\end{align*}

\paragraph{Linjär kombination av lösningar}
Låt $\vb{x}^{(1)}, \dots, \vb{x}^{(n)}\in\R,\ 0 < t < T$ vara ett fundamentalt sätt av lösningar till
\begin{align*}
	\deval{\vb{x}}{t}{t} = P(t)\vb{x}(t),\ t > 0,
\end{align*}
där $P$ är kontinuerlig. Då kan varje lösning till ekvationen skrivas som
\begin{align*}
	\vb{x} = \sum c_{i}\vb{x}^{(i)}
\end{align*}
på precis ett sätt. Med fundamentalmatrisen kan detta uttryckas som
\begin{align*}
	\vb{x} = \Psi\vb{c},
\end{align*}
där $\vb{c}$ är en vektor med koefficienter.

\subparagraph{Bevis}
Begynnelsesvärdeproblemet implicerar att vår lösning måste uppfylla
\begin{align*}
	\left[\vb{x}^{(1)}(0) \dots \vb{x}^{(n)}(0)\right]
	\left[\begin{array}{c}
		c_{1} \\
		\vdots \\
		c_{n}
	\end{array}\right]
	= \vb{x}(0).
\end{align*}
Detta har bara en lösning om $\abs{\vb{x}^{(1)}(0) \dots \vb{x}^{(n)}(0)} \neq 0$. Eftersom alla lösningarna är linjärt oberoende, är detta uppfylld. Konstanterna $c_{i}$ ges då unikt, och beviset är klart.

\paragraph{System av ODE med konstant matris}
Betrakta
\begin{align*}
	\deval{\vb{x}}{t}{t} = P\vb{x}(t),
\end{align*}
där $P$ är en konstant matris. Vi gör ansatsen $\vb{x}(t) = e^{rt}\vb*{\xi}$ och får
\begin{align*}
	\deval{\vb{x}}{t}{t} - P\vb{x}(t) = e^{rt}(rI - A)\vb*{\xi}.
\end{align*}
Eftersom exponentialfunktionen alltid är nollskild, kan detta bara bli noll om
\begin{align*}
	P\vb*{\xi} = r\vb*{\xi}.
\end{align*}
Alltså är $\vb{x}$ bara en lösning om $\vb*{\xi}$ är en egenvektor till $P$ och $r$ är det motsvarande egenvärdet.

\paragraph{Upprepande egenvärden}
Betrakta
\begin{align*}
	\deval{\vb{x}}{t}{t} = P\vb{x}(t),
\end{align*}
där $P$ är en konstant matris, låt $r$ vara ett egenvärde till $P$ med algebraisk multiplicitet $2$ och geometrisk multiplicitet $1$ och $\vb*{\xi}$ en motsvarande egenvektor. Då är en lösning
\begin{align*}
	\vb{x}^{(1)} = \vb*{\xi}e^{rt}
\end{align*}
och en ny lösning kan skrivas som
\begin{align*}
	\vb{x}^{(2)} = \vb*{\xi}te^{rt} + \vb*{\eta}e^{rt},
\end{align*}
där $\vb*{\eta}$ uppfyller
\begin{align*}
	(A - rI)\vb*{\eta} = \vb*{\xi}.
\end{align*}

\paragraph{Wronskianen för ett system med konstant matris}
Betrakta
\begin{align*}
	\deval{\vb{x}}{t}{t} = P\vb{x}(t),
\end{align*}
där $P$ är en konstant matris. Låt $\vb*{\xi}_{i}$ vara de olika egenvektorerna till $P$ motsvarande egenvärden $r_{i}$. Wronskianen till dessa ges av
\begin{align*}
	W(\vb*{\xi}_{1}, \dots, \vb*{\xi}_{n})(t) &= \mdet{e^{r_{1}t}\vb*{\xi}_{1} & \dots & e^{r_{n}t}\vb*{\xi}_{n}} \\
	                                          &= e^{(r_{1} + \dots + r_{n})}\mdet{\vb*{\xi}_{1} & \dots & \vb*{\xi}_{n}} \\
	                                          &= W(\vb*{\xi}_{1}, \dots, \vb*{\xi}_{n})(0)e^{\Tr{P}t},
\end{align*}
där vi har använt en sats för att få fram spåret. Det följer blant annat att Wronskianen är antingen $0$ eller nollskild överallt.

\paragraph{Diagonalisering}
Betrakta
\begin{align*}
	\deval{\vb{x}}{t}{t} = A\vb{x}(t),
\end{align*}
där $A$ är en konstant matris som kan skrivas som $A = PDP^{-1}$. Då kan vi införa $\vb{x}= P\vb{y}$, vilket ger
\begin{align*}
	\deval{\vb{x}}{t}{t} &= P\deval{\vb{y}}{t}{t} = PDP^{-1}\vb{y} = PD\vb{y}, \\
	\deval{\vb{y}}{t}{t} &= D\vb{y},
\end{align*}
vilket är en simplare variant av det ursprungliga problemet.

\paragraph{Partikulärlösningar}
Betrakta
\begin{align*}
	\deval{\vb{x}}{t}{t} = \vb{g}(t) + P\vb{x}(t).
\end{align*}
Det finns olika metoder att ta fram en partikulärlösning av detta.

\subparagraph{Diagonalisering}
Låt $P$ vara diagonaliserbar och konstant. Då får man vid diagonalisering att
\begin{align*}
	\deval{\vb{y}}{t}{t} = \vb{h}(t) + D\vb{y}(t)
\end{align*}
med $\vb{h} = T^{-1}\vb{g}$. Varje komponent kan då lösas som
\begin{align*}
	y_{j}(t) = c_{j}e^{r_{j}t} + e^{r_{j}t}\inteval{t_{0}}{t}{h_{j}(s)e^{-r_{j}s}}{s}.
\end{align*}

\subparagraph{Obestämda koefficienter}
Om $\vb{g}$ har en enkel form, kan man gissa på en lösning och bestämma koefficienterna baserad på ens gissning.

\subparagraph{Variation av parametrar}
Ansätt
\begin{align*}
	\vb{x}(t) = \Psi(t)\vb{u}(t).
\end{align*}
Då ger differentialekvationen
\begin{align*}
	\deval{\Psi}{t}{t}\vb{u}(t) + \Psi(t)\deval{\vb{u}}{t}{t} = P(t)\Psi(t)\vb{u}(t) + \vb{g}(t).
\end{align*}
Eftersom $\Psi$ är en fundamentalmatris för ekvationen, gäller att
\begin{align*}
	\deval{\Psi}{t}{t}P(t)\Psi(t),
\end{align*}
och vi får
\begin{align*}
	\Psi(t)\deval{\vb{u}}{t}{t} = \vb{g}(t).
\end{align*}
Vi löser för $\vb{u}$ och integrerar, vilket slutligen ger
\begin{align*}
	\vb{x}(t) = \Psi(t)\vb{c} + \Psi(t)\inteval{t_{0}}{t}{\Psi^{-1}(s)\vb{g}(s)}{s}.
\end{align*}

\subsection{Exakta differentialekvationer}

\paragraph{Formulering}
Betrakta ekvationen
\begin{align*}
	M(x, y(x)) + N(x, y(x))\deval{y}{x}{x} = 0.
\end{align*}
Denna är exakt om den kan skrivas på formen
\begin{align*}
	\deval{\psi}{x}{x, y(x)} = 0.
\end{align*}
Det gåller då att
\begin{align*}
	\pdeval{\psi}{x}{x, y(x)} = M(x, y(x)),\ \pdeval{\psi}{y}{x, y(x)} = N(x, y(x)),
\end{align*}
och lösningarna ges implicit av
\begin{align*}
	\psi(x, y(x)) = c.
\end{align*}

\paragraph{Exakthet av differentialekvationer}
Differentialekvationen
\begin{align*}
	M(x, y(x)) + N(x, y(x))\deval{y}{x}{x} = 0
\end{align*}
är exakt om
\begin{align*}
	\pdeval{M}{y}{x, y(x)} = \pdeval{N}{x}{x, y(x)}.
\end{align*}

\subsection{Potensserier}

\paragraph{Kriterier för potensserielösning}
I vissa fall kan man ansätta
\begin{align*}
	y(x) = \sum\limits_{i = 1}^{\infty}a_{x}x^{n}
\end{align*}
som en lösning av en differentialekvation. Detta kan endast göras om alla involverade koefficienter är analytiska.

\paragraph{Singulära punkter och Euler-liknande ekvationer}
Betrakta differentialekvationen
\begin{align*}
	P(x)\deval[2]{y}{x}{x} + Q(x)\deval{y}{x}{x} + R(x)y(x) = 0.
\end{align*}
Vi skriver denna om till
\begin{align*}
	x^{2}\deval[2]{y}{x}{x} + x(xp(x))\deval{y}{x}{x} + x^{2}q(x)y(x) = 0.
\end{align*}
Antag att antingen $p$ eller $q$ ej är analytiska kring $0$, men $xp$ och $x^{2}q$ är det. Då kan man med hjälp av ansatsen
\begin{align*}
	y(x) = \sum\limits_{n = 0}^{\infty}a_{x}x^{n + r}
\end{align*}
få att detta är en lösning om $r$ uppfyller
\begin{align*}
	r(r - 1) + p_{0}r + q_{0} = 0,
\end{align*}
där $p_{0} = \lim_{x\to 0}xp(x)$ och $q_{0} = \lim_{x\to 0}x^{2}q(x)$. Från detta fås vidare en rekursionsrelation för koefficienterna $a_{n}$.

Låt nu $r_{1}, r_{2}$ vara värden av $r$ som ger lösningar, med $r_{1} > r_{2}$, och antag att $y_{1}$ är lösningen som fås vid att använda $r_{1}$ i ansatsen. Då kan följande ansatser göras för att hitta en ny lösning:
\begin{itemize}
	\item Om $r_{1} - r_{2}$ inte är ett heltal, kommer man få två olika rekursionsrelationer med hjälp av ansatsen.
	\item Om $r_{1} = r_{2}$, gör man ansatsen
	\begin{align*}
		y_{2}(x) = y_{1}(x)\ln{x} + x^{r_{1}}\sum\limits_{n = 1}^{\infty}b_{n}x^{n},
	\end{align*}
	där koefficienterna $b_{n}$ måste bestämmas.
	\item Om $r_{1} - r_{2}$ är ett positivt heltal, gör man ansatsen
	\begin{align*}
		y_{2}(x) = ay_{1}(x)\ln{x} + x^{r_{2}}\left(1 + \sum\limits_{n = 1}^{\infty}b_{n}x^{n}\right),
	\end{align*}
	för några tal $a, b_{n}$.
\end{itemize}

\subsection{Stabilitet}

\paragraph{Jämviktspunkter}
Betrakta
\begin{align*}
	\deval{\vb{x}}{t}{t} = \vb{f}(\vb{x}(t)).
\end{align*}
En jämviktspunkt för detta systemet är en punkt $\vb{x}(t_{0})$ så att $\vb{f}(\vb{x}(t_{0})) = \vb{0}$, med implikationen att $\vb{x}(t)$ är konstant för $t > t_{0}$.

\paragraph{Stabila jämviktspunkter}
En jämviktspunkt $\vb{x}_{0}$ är stabil om det för varje $\varepsilon > 0$ finns ett $\delta > 0$ så att alla lösningar $\vb{x}$ som uppfyller $\abs{\vb{x}(t_{0}) - \vb{x}_{0}} < \delta$, existerar för $t > t_{0}$ och uppfyller $\abs{\vb{x}(t) - \vb{x}_{0}} < \varepsilon,\ t > t_{0}$. En jämviktspunkt som ej uppfyller detta är instabil.

\paragraph{Asymptotiskt stabila jämviktspunkter}
En jämviktspunkt $\vb{x}_{0}$ är asymptotiskt stabil om den är stabil och det finns ett $\delta_{0} > 0$ så att om $\abs{\vb{x}(t_{0}) - \vb{x}_{0}} < \delta_{0}$, gäller det att
\begin{align*}
	\lim\limits_{t\to\infty}\vb{x}(t) = \vb{x}_{0}.
\end{align*}

\paragraph{Stabilitet av autonom ODE}
Betrakta
\begin{align*}
\deval{y}{t}{t} = g(y(t)),\ g(y_{0}) = 0.
\end{align*}
Då gäller att
\begin{itemize}
	\item om $\deval{g}{y}{y_{0}} < 0$, är $y_{0}$ asymptotiskt stabil.
	\item om $\deval{g}{y}{y_{0}} > 0$, är $y_{0}$ instabil.
\end{itemize}

\subparagraph{Bevis}
Betrakta $(y - y_{0})^{2}$. Nära $y_{0}$ gäller att
\begin{align*}
	\dv{t}(y(t) - y_{0})^{2} &= 2(y(t) - y_{0})g(y(t)) \\
	                         &\approx 2(y(t) - y_{0})\left(g(y_{0}) + \deval{g}{y}{y_{0}}(y(t) - y_{0}) + o((y(t) - y_{0})^{2})\right) \\
	                         &= 2(y(t) - y_{0})\left(\deval{g}{y}{y_{0}}(y(t) - y_{0}) + o((y(t) - y_{0})^{2})\right)
\end{align*}
Vi får på något sätt
\begin{align*}
	\dv{t}(y(t) - y_{0})^{2} < \deval{g}{y}{y_{0}}(y(t) - y_{0})^{2},
\end{align*}
som kan lösas för att ge
\begin{align*}
	(y(t) - y_{0})^{2} < e^{\deval{g}{y}{y_{0}}t}(y(0) - y_{0})^{2},
\end{align*}
som går mot $0$ för stora $t$ enligt vårt antagande om $g$:s derivata.

\paragraph{Karakterisering av jämviktspunkter för system}
Betrakta systemet
\begin{align*}
	\deval{\vb{x}}{t}{t} = P\vb{x}(t),
\end{align*}
där $P$ är konstant och reellvärd. För enkelhetens skull kommer vi här att låta systemet vara ett system i två variabler. Låt även $P$ ha egenvärden $r_{1}, r_{2}\neq 0$. Då gäller att $\vb{0}$ är en kritisk punkt. Lösningarnas banor kan nu beskrivas på följande sätt:
\begin{itemize}
	\item Om $r_{1}, r_{2} < 0$ går alla lösningar in mot origo, och origo kallas en stabil nod.
	\item Om $r_{1}, r_{2} > 0$ går alla lösningar ut från origo, och origo kallas en instabil nod.
	\item Om egenvärderna har olika tecken går lösningarna in mot origo parallellt med en egenvektor och ut parallellt med den andra, och origo kallas en instabil sadelpunkt.
	\item Om $r_{1} = \alpha + i\beta, r_{2} = \alpha - i\beta$ gäller att:
	\begin{itemize}
		\item Om $\alpha > 0$ går lösningarna i spiraler ut från origo, och origo kallas en instabil spiralpunkt.
		\item Om $\alpha > 0$ går lösningarna i spiraler in mot origo, och origo kallas en stabil spiralpunkt.
		\item Om $\alpha = 0$ går lösningarna i bana kring origo, och origo kallas ett centrum.
	\end{itemize}
	\item Om $r_{1} = r_{2} = r$ och det finns två egenvektorer motsvarande egenvärdet $r$ går banorna i linjer från eller till origo, beroende på tecknet till $r$, och origo är en instabil eller stabil nod.
	\item Om $r_{1} = r_{2} = r$ och det bara finns en egenvektor motsvarande egenvärdet $r$ går lösningarna i kurvade banor ut från eller in mot origo, där dessa banorna blir parallella med egenvektorn långt borta från origo, och origo är en stabil eller instabil degenererad nod.
\end{itemize}

\subparagraph{Slutsats}
Det gäller alltså att
\begin{itemize}
	\item Om alla $P$s egenvärden har negativ realdel, är origo en stabil jämviktspunkt.
	\item Om något av $P$:s egenvärden har positiv realdel, är origo en instabil jämviktspunkt.
\end{itemize}

\paragraph{Stabilitet av jämviktspunkter för icke-linjära system av ODE}
Betrakta
\begin{align*}
	\deval{\vb{x}}{t}{t} = \vb{f}(\vb{x}(t)),
\end{align*}
Låt detta ha en kritisk punkt $\vb{x}_{0}$ och låt $\vb{g}\in C^{1}$ i en öppen mängd kring $\vb{x}_{0}$. Vi linjariserar kring $\vb{x}_{0}$, vilket går om
\begin{align*}
	\lim\limits_{\vb{x}\to\vb{x}_{0}}\frac{\abs{\vb{f}(\vb{x}(t))}}{\abs{\vb{x}(t)}} = 0,
\end{align*}
vilket uppfylls om $\vb{f}\in C^{2}$. Inför funktionalmatrisen aka Jacobimatrisen
\begin{align*}
	J(\vb{x}) = 
	\left[\begin{array}{ccc}
		\deval{f_1}{x_1}{\vb{x}} & \dots  & \deval{f_1}{x_n}{\vb{x}} \\
		\vdots                     & \ddots & \vdots \\
		\deval{f_p}{x_1}{\vb{x}} & \dots  & \deval{f_p}{x_n}{\vb{x}}
	\end{array}\right]
\end{align*}
och betrakta $J(\vb{x}_{0})$. Då gäller att
\begin{itemize}
	\item Om alla $J(\vb{x}_{0})$s egenvärden har negativ realdel, är $\vb{x}_{0}$ en stabil jämviktspunkt.
	\item Om något av $J(\vb{x}_{0})$ egenvärden har positiv realdel, är $\vb{x}_{0}$ en instabil jämviktspunkt.
\end{itemize}

\paragraph{Lyapunovfunktioner}
Betrakta
\begin{align*}
	\deval{\vb{x}}{t}{t} = \vb{f}(\vb{x}(t)).
\end{align*}
Antag att systemet har en kritisk punkt $\vb{0}$. Om det finns en positivt definitiv funktion $V\in C^{1}$ och en negativt definitiv funktion
\begin{align*}
	V' = \pdv{V}{x}f_{1} + \pdv{V}{y}f_{2}
\end{align*}
på någon omgivning av $\vb{0}$, är $\vb{0}$ en stabil jämviktspunkt. Om $V'$ är negativt semidefinitiv, är $\vb{0}$ en stabil jämviktspunkt.