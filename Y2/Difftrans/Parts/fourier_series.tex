\section{Fourierserier}

\subsection{Definitioner}

\paragraph{Cesàro-summerbarhet}
Låt
\begin{align*}
	s_{n} = \sum\limits_{k = 0}^{n}a_{k}
\end{align*}
och
\begin{align*}
	\sigma_{N} = \frac{1}{N + 1}\sum\limits_{k = 0}^{N}s_{k}.
\end{align*}
Då är $\sum\limits_{k = 0}^{n}a_{k}$ Cesàrosummerbar om $\sigma_{N}$ konvergerar när $N\to\infty$.

\paragraph{Summationskärnor}
En summationskärna är en Riemannintegrerbar funktion $K_{n}$ på $(-a, a)$ som uppfyller
\begin{itemize}
	\item $\inteval{-a}{a}{K_{n}(x)}{x} = 1$.
	\item $\inteval{-a}{a}{\abs{K_{n}(x)}}{x} \leq M$ för något $M$.
	\item för varje $\delta > 0$ så gäller att $\lim\limits_{n\to\infty}\inteval{\delta < \abs{x} < a}{}{K_{n}(x)}{x} = 0$.
\end{itemize}

\paragraph{Positiva summationskärnor}
En positiv summationskärna är en summationskärna med positiva funktioner.

\paragraph{Dirichletkärnan}
Dirichletkärnan är summationskärnan vars element ges av
\begin{align*}
	D_{n}(\alpha) = \sum\limits_{k = -n}^{n}e^{ik\alpha} = \frac{1}{2\pi}\frac{1 - e^{i(2n + 1)\alpha}}{1 - e^{i\alpha}}e^{-in\alpha}.
\end{align*}

\paragraph{Fejérkärnan}
Vi definierar Fejérkärnan som
\begin{align*}
	F_{N}(t) = \frac{1}{N + 1}\sum\limits_{k = 0}^{N}D_{k}(t).
\end{align*}

\subsection{Satser}

\paragraph{Konvergens och Cesarosummerbarhet}
Låt
\begin{align*}
	\sum\limits_{i = 0}^{\infty}a_{i} = s.
\end{align*}
Då är $\sum\limits_{i = 0}^{\infty}a_{i}$ Cesarosummerbar och har värdet $s$ även i denna mening.

\proof
Vi har
\begin{align*}
	\abs{\sigma_{n} - s} &= \abs{\frac{1}{n}\sum\limits_{i = 0}^{n - 1}s_{i} - s} \\
	                     &= \abs{\frac{(s_{0} - s) + \dots + (s_{n - 1} - s)}{n}} \\
	                     &\leq  \abs{\frac{(s_{0} - s) + \dots + (s_{M} - s)}{n}} + \abs{\frac{(s_{M + 1} - s) + \dots + (s_{n - 1} - s)}{n}}.
\end{align*}
Vi vill visa att det finns ett $N$ så att $n > N\implies \abs{\sigma_{n} - s} < \varepsilon$ för varje $\varepsilon > 0$.

Låt nu $\varepsilon > 0$ vara givet. Eftersom $s_{n}$ konvergerar, finns det ett $M$ så att när $n > M$ är $\abs{s_{n} - s} < \chi$ för varje $\chi > 0$. Låt nu $M$ i ekvationen över vara så att när $n > M$ är $\abs{s_{n} - s} < \frac{\varepsilon}{2}$. Detta ger
\begin{align*}
	\abs{\frac{(s_{M + 1} - s) + \dots + (s_{n - 1} - s)}{n}} < \frac{n - M}{n}\frac{\varepsilon}{2} < \frac{\varepsilon}{2}.
\end{align*}
Olikheten över kan då skrivas som
\begin{align*}
	\abs{\sigma_{n} - s} \leq \frac{C_{M}}{n} + \frac{\varepsilon}{2}.
\end{align*}
Låt nu $N$ vara lika med $\frac{2C_{M}}{\varepsilon}$, och välj $n > N$. Då blir olikheten
\begin{align*}
	\abs{\sigma_{n} - s} \leq \varepsilon,
\end{align*}
och beviset är klart.

\paragraph{Formel för Fourierkoefficienter}
Antag att en funktion $f$ kan skrivas som
\begin{align*}
	f(x) = \sum\limits_{n\in\Z}c_{n}e^{inx},
\end{align*}
där$\sum\limits_{n\in\Z}\abs{c_{n}}$ är begränsad. Då gäller det att
\begin{align*}
	c_{n} = \frac{1}{2\pi}\inteval{-\pi}{\pi}{f(t)e^{-int}}{t}.
\end{align*}

\proof
Om antagandet är sant, vill vi visa att
\begin{align*}
	c_{n} = \frac{1}{2\pi}\inteval{-\pi}{\pi}{\sum\limits_{m\in\Z}c_{m}e^{imt}e^{-int}}{t}.
\end{align*}

Vi får först
\begin{align*}
	\inteval{-\pi}{\pi}{\sum\limits_{m\in\Z}c_{m}e^{imt}e^{-int}}{t} = \sum\limits_{m\in\Z}c_{m}\inteval{-\pi}{\pi}{e^{i(m - n)t}}{t},
\end{align*}
eftersom summan är absolutkonvergent. Den återstående integralen ges av
\begin{align*}
	\inteval{-\pi}{\pi}{e^{i(m - n)t}}{t} =
	\begin{cases}
		\inteval{-\pi}{\pi}{}{t} = 2\pi,                                       &m = n, \\
		\frac{1}{i(m - n)}\left(e^{i(m - n)\pi} - e^{-i(m - n)\pi}\right) = \frac{e^{i(m - n)\pi}}{i(m - n)}\left(1 - e^{-2\pi i(m - n)}\right) = 0, &m\neq 0.
	\end{cases}
\end{align*}

Låt nu $n$ vara givet och välj ett $N$ så att $\abs{n} < N$ och $\sum\limits_{\abs{n} > N}\abs{c_{n}} < \varepsilon$. Detta ger
\begin{align*}
	\abs{\frac{1}{2\pi}\inteval{-\pi}{\pi}{f(t)e^{-int}}{t} - \frac{1}{2\pi}\inteval{-\pi}{\pi}{\sum\limits_{m\in\Z}c_{m}e^{imt}e^{-int}}{t}} &= \frac{1}{2\pi}\abs{\inteval{-\pi}{\pi}{\sum\limits_{\abs{m} > N}c_{m}e^{i(m - n)t}}{t}} \\
	                                      &\leq \frac{1}{2\pi}\inteval{-\pi}{\pi}{\sum\limits_{\abs{m} > N}\abs{c_{m}e^{i(m - n)t}}}{t} \\
	                                      &= \frac{1}{2\pi}\inteval{-\pi}{\pi}{\sum\limits_{\abs{m} > N}\abs{c_{m}}}{t} \\
	                                      &= \frac{1}{2\pi}\sum\limits_{\abs{m} > N}2\pi\abs{c_{m}} \\
	                                      &= \sum\limits_{\abs{m} > N}\abs{c_{m}} < \varepsilon.
\end{align*}
Enligt vårt tidigare argument gäller det även att
\begin{align*}
	\frac{1}{2\pi}\inteval{-\pi}{\pi}{\sum\limits_{m\in\Z}c_{m}e^{imt}e^{-int}}{t} = c_{n},
\end{align*}
och vi har alltså visat
\begin{align*}
	\abs{\frac{1}{2\pi}\inteval{-\pi}{\pi}{f(t)e^{-int}}{t} - c_{n}} < \varepsilon,
\end{align*}
vilket ger att satsen stämmer.

\paragraph{Riemann-Lebesgues lemma}
Antag att $f$ är Riemannintegrerbar på $(-\pi, \pi]$. Då är
\begin{align*}
	\lim\limits_{\lambda\to\pm\infty}\inteval{-\pi}{\pi}{f(t)e^{i\lambda t}}{t} = 0.
\end{align*}

\proof
Låt $s$ vara en undertrappa till $f$. Eftersom $f$ är Riemannintegrerbar så finns det ett $s$ så att
\begin{align*}
	\inteval{-\pi}{\pi}{\abs{f(x) - s(x)}}{x} < \frac{\varepsilon}{2}.
\end{align*}
Detta ger
\begin{align*}
	\abs{\inteval{-\pi}{\pi}{f(t)e^{i\lambda t}}{t}} &\leq \abs{\inteval{-\pi}{\pi}{s(t)e^{i\lambda t}}{t}} + \abs{\inteval{-\pi}{\pi}{(f(t) - s(t))e^{i\lambda t}}{t}} \\
	                                                 &\leq \abs{\inteval{-\pi}{\pi}{s(t)e^{i\lambda t}}{t}} + \inteval{-\pi}{\pi}{\abs{f(t) - s(t)}}{t} \\
	                                                 &\leq \abs{\inteval{-\pi}{\pi}{s(t)e^{i\lambda t}}{t}} + \frac{\varepsilon}{2}.
\end{align*}
Vi har vidare
\begin{align*}
	\abs{\inteval{-\pi}{\pi}{s(t)e^{i\lambda t}}{t}} &= \abs{\sum\limits_{j = 1}^{n}\inteval{x_{j - 1}}{x_{j}}{m_{j}e^{i\lambda t}}{t}} \\
	                                                 &= \abs{\sum\limits_{j = 1}^{n}\frac{m_{j}}{i\lambda}(e^{i\lambda x_{j}} - e^{i\lambda x_{j - 1}})} \\
	                                                 &\leq \sum\limits_{j = 1}^{n}\frac{2\abs{m_{j}}}{\lambda} \\
	                                                 &\leq \frac{2nM}{\lambda},
\end{align*}
där $M = \sup{\{\abs{m_{1}}, \dots, \abs{m_{n}}\}}$. För $\lambda > \frac{4nM}{\varepsilon}$ fås
\begin{align*}
	\abs{\inteval{-\pi}{\pi}{s(t)e^{i\lambda t}}{t}} \leq \frac{1}{2}\varepsilon,
\end{align*}
och beviset är klart.

\paragraph{Integraler med positiva summationskärnor}
Låt $f$ vara en Riemannintegrerbar funktion på $(-a, a)$, specifikt så att $\abs{f(x)}\leq M, x\in (-a, a)$, och kontinuerlig i $x = 0$. Antag vidare att $K_{n}$ är en positiv summationskärna på $(-a, a)$. Då gäller att
\begin{align*}
	\lim\limits_{n\to\infty}\inteval{-a}{a}{K_{n}(x)f(x)}{x} = f(0).
\end{align*}

\proof
Vi vill visa att för alla $\varepsilon > 0$ existerar ett $M > 0$ så att om $n > M$ så är
\begin{align*}
	\abs{\inteval{-a}{a}{K_{n}(x)f(x)}{x} - f(0)} < \varepsilon.
\end{align*}

Vi har att
\begin{align*}
         &\abs{\inteval{-a}{a}{K_{n}(x)f(x)}{x} - f(0)} \\
	=    &\abs{\inteval{-a}{a}{K_{n}(x)f(x)}{x} - f(0)\inteval{-a}{a}{K_{n}(x)}{x}} \\
	=    & \abs{\inteval{-a}{a}{K_{n}(x)(f(x) - f(0))}{x}} \\
	=    & \abs{\inteval{-\delta}{\delta}{K_{n}(x)(f(x) - f(0))}{x} + \inteval{\delta < \abs{x} < a}{}{K_{n}(x)(f(x) - f(0))}{x}} \\
	\leq & \abs{\inteval{-\delta}{\delta}{K_{n}(x)(f(x) - f(0))}{x}} + \abs{\inteval{\delta < \abs{x} < a}{}{K_{n}(x)(f(x) - f(0))}{x}}.
\end{align*}
Eftersom $f$ är kontinuerlig, finns det ett $J > 0$ sådant att $\abs{f(x) - f(0)} < J$ när $\abs{x} < a$. Då kan vi skriva
\begin{align*}
	     &\abs{\inteval{-\delta}{\delta}{K_{n}(x)(f(x) - f(0))}{x}} + \abs{\inteval{\delta < \abs{x} < a}{}{K_{n}(x)(f(x) - f(0))}{x}} \\
	\leq & \abs{\inteval{-\delta}{\delta}{K_{n}(x)(f(x) - f(0))}{x}} + J\abs{\inteval{\delta < \abs{x} < a}{}{K_{n}(x)}{x}}.
\end{align*}
Tills nu har vi inte specifierat vårat $\delta$. Välj nu det så att $\abs{x} < \delta\implies\abs{f(x) - f(0)} < \frac{1}{2}\varepsilon$. Använd vidare att eftersom $\lim\limits_{n\to\infty}\inteval{\delta < \abs{x} < a}{}{K_{n}(x)}{x} = 0$ finns det ett $M > 0$ så att $n > M\implies \abs{\inteval{\delta < \abs{x} < a}{}{K_{n}(x)}{x}} < \frac{\varepsilon}{2J}$. Alltså har vi för ett tillräckligt stort $n$ att
\begin{align*}
	\abs{\inteval{-\delta}{\delta}{K_{n}(x)(f(x) - f(0))}{x}} + J\abs{\inteval{\delta < \abs{x} < a}{}{K_{n}(x)}{x}} < \frac{1}{2}\varepsilon\abs{\inteval{-\delta}{\delta}{K_{n}(x)}{x}} + J\frac{\varepsilon}{2J} = \varepsilon,
\end{align*}
och beviset är klart.

\paragraph{Följdsats}
Låt $K_{n}$ vara en positiv summationskärna och $f$ vara kontinuerlig och integrerbar i $x$. Då gäller det att
\begin{align*}
	\lim\limits_{n\to\infty}\inteval{-\pi}{\pi}{K_{n}(t)f(x - t)}{t} = f(x).
\end{align*}

\proof
Tillämpa satsen ovan på $g(t) = f(x - t)$.

\paragraph{Omskrivning av Dirichletkärnan}
Dirichletkärnan kan skrivas som
\begin{align*}
	D_{N}(\alpha) = \frac{1}{2\pi}\frac{\sin{\left(N + \frac{1}{2}\right)\alpha}}{\sin{\frac{1}{2}\alpha}}.
\end{align*}

\proof
\begin{align*}
	D_{N}(\alpha) &= \frac{1}{2\pi}\frac{1 - e^{i(2N + 1)\alpha}}{1 - e^{i\alpha}}e^{-iN\alpha} \\
	              &= \frac{1}{2\pi}e^{-iN\alpha}\frac{e^{i\left(N + \frac{1}{2}\right)\alpha}}{e^{\frac{1}{2}i\alpha}}\frac{e^{-i\left(N + \frac{1}{2}\right)\alpha} - e^{i\left(N + \frac{1}{2}\right)\alpha}}{e^{-\frac{1}{2}i\alpha} - e^{\frac{1}{2}i\alpha}} \\
	              &= \frac{1}{2\pi}\frac{\sin{\left(N + \frac{1}{2}\right)\alpha}}{\sin{\frac{1}{2}\alpha}}.
\end{align*}

\paragraph{Fejérkärnans positivitet}
$F_{N}$ är en positiv summationskärna på $[-\pi, \pi]$.

\proof
$F_{N}$ är en summa av Riemannintegrerbara funktioner, och är därmed Riemannintegrerbar.

För att visa att $F_{N} \geq 0$, skriv
\begin{align*}
	F_{N}(x) = \frac{1}{2\pi(N + 1)}\sum\limits_{n = 0}^{N}\frac{\lambda^{2n + 1} - \lambda^{-2n - 1}}{\lambda - \lambda^{-1}},
\end{align*}
där $\lambda = e^{\frac{1}{2}ix}$. Detta kan vidare skrivas om till
\begin{align*}
	F_{N}(x) &= \frac{1}{2\pi(N + 1)}\frac{1}{\lambda - \lambda^{-1}}\sum\limits_{n = 0}^{N}\lambda^{2n + 1} - \lambda^{-2n - 1} \\
	         &= \frac{1}{2\pi(N + 1)}\frac{1}{2i\sin{\frac{1}{2}t}}\sum\limits_{n = 0}^{N}\lambda^{2n + 1} - \lambda^{-2n - 1} \\
	         &= \frac{1}{2\pi(N + 1)}\frac{1}{(2i\sin{\frac{1}{2}t)^{2}}}\sum\limits_{n = 0}^{N}\lambda^{2n + 2} - \lambda^{2n} - \lambda^{-2n} + \lambda^{-2n - 2} \\
	         &= \frac{1}{2\pi(N + 1)}\frac{1}{(2i\sin{\frac{1}{2}t})^{2}}\sum\limits_{n = 0}^{N}\lambda^{2n}(\lambda^{2} - 1) + \lambda^{-2n}(\lambda^{-2} - 1) \\
	         &= \frac{1}{2\pi(N + 1)}\frac{1}{(2i\sin{\frac{1}{2}t})^{2}}\left(\frac{1 - \lambda^{2N + 2}}{1 - \lambda^{2}}(\lambda^{2} - 1) + \frac{1 - \lambda^{-2N - 2}}{1 - \lambda^{-2}}(\lambda^{-2} - 1)\right) \\
	         &= \frac{1}{2\pi(N + 1)}\frac{1}{(2i\sin{\frac{1}{2}t})^{2}}\left(- 1 + \lambda^{2N + 2} - 1 + \lambda^{-2N - 2}\right) \\
	         &= \frac{1}{2\pi(N + 1)}\frac{1}{(2i\sin{\frac{1}{2}t})^{2}}(\lambda^{N + 1} - \lambda^{-N - 1})^{2} \\
	         &= \frac{1}{2\pi(N + 1)}\frac{(2i\sin{\frac{N + 1}{2}t})^{2}}{(2i\sin{\frac{1}{2}t})^{2}} \\
	         &= \frac{1}{2\pi(N + 1)}\left(\frac{\sin{\frac{N + 1}{2}t}}{\sin{\frac{1}{2}t}}\right)^{2}.
\end{align*}

För att visa det andra påståendet, använder vi att
\begin{align*}
	\inteval{-\pi}{\pi}{D_{n}(x)}{x} = \frac{1}{2\pi}\inteval{-\pi}{\pi}{\sum\limits_{k = -n}^{n}e^{ikx}}{x} = 1,
\end{align*}
vilket ger
\begin{align*}
	\inteval{-\pi}{\pi}{F_{N}(t)}{t} = \frac{1}{N + 1}\inteval{-\pi}{\pi}{\sum\limits_{n = 0}^{N}D_{n}(x)}{x} = \frac{1}{N + 1}\sum\limits_{n = 0}^{N}1 = 1.
\end{align*}

För att visa det tredje påståendet, skriv
\begin{align*}
	\inteval{\delta < \abs{x} < \pi}{}{F_{N}(t)}{t} &= \inteval{\delta < \abs{x} < \pi}{}{\frac{1}{2\pi (N + 1)}\left(\frac{\sin{\left(\frac{N + 1}{2}t\right)}}{\sin{\left(\frac{1}{2}t\right)}}\right)^{2}}{t} \\
	                                                &\leq \inteval{\delta < \abs{x} < \pi}{}{\frac{1}{2\pi (N + 1)}\left(\frac{1}{\sin{\left(\frac{1}{2}\delta\right)}}\right)^{2}}{t} \\
	                                                &\leq \frac{1}{(N + 1)}\left(\frac{1}{\sin{\left(\frac{1}{2}\delta\right)}}\right)^{2} \\
	                                                &\to 0. 
\end{align*}

\paragraph{Omskrivning av Fourierserier}
\begin{align*}
	\sum\limits_{n = -N}^{N}c_{n}e^{int} = \inteval{-\pi}{\pi}{f(x - u)D_{N}(u)}{u}.
\end{align*}

\proof
\begin{align*}
	s_{N}(x) &= \frac{1}{2\pi}\sum\limits_{n = -N}^{N}\inteval{-\pi}{\pi}{f(t)e^{in(x - t)}}{t} \\
	         &= \frac{1}{2\pi}\inteval{-\pi}{\pi}{f(t)\frac{1 - e^{-i(2N + 1)(x - t)}}{1 - e^{-i(x - t)}}e^{-iN(x - t)}}{t}.
\end{align*}
Vi känner igen ena faktorn i integranden som Dirichletkärnan, och därmed är beviset klart.

\paragraph{Fejérs sats}
Låt $f$ vara styckvis kontinuerlig på $[-\pi, \pi]$ och kontinuerlig i $t$, och definiera
\begin{align*}
	s_{N}      &= \sum\limits_{n = -N}^{N}c_{n}e^{int}, \\
	\sigma_{N} &= \frac{1}{N + 1}\sum\limits_{n = 0}^{N}s_{n},
\end{align*}
där $c_{n}$ är Fourierkoefficienterna till $f$. Då är $\lim\limits_{N\to\infty}\sigma_{N} = f(t)$.

\proof
Vi utgår från förra satsen, och skriver
\begin{align*}
	s_{N}(x) &= \inteval{-\pi}{\pi}{f(t)D_{N}(x - t)}{t} \\
	         &= -\inteval{x + \pi}{x - \pi}{f(x - u)D_{N}(u)}{u} \\
	         &= \inteval{-\pi}{\pi}{f(x - u)D_{N}(u)}{u},
\end{align*}
där vi har utnyttjat integrandens periodicitet. Cesàrosumman av detta är
\begin{align*}
	\sigma_{N} &= \frac{1}{N + 1}\sum\limits_{n = 0}^{N}\inteval{-\pi}{\pi}{f(x - t)D_{N}(t)}{t} \\
	           &= \inteval{-\pi}{\pi}{f(x - t)\frac{1}{N + 1}\sum\limits_{n = 0}^{N}D_{N}(t)}{t} \\
	           &= \inteval{-\pi}{\pi}{F_{N}(t)f(x - t)}{t}.
\end{align*}
Enligt satsen om integraler med positiva summationskärnor är detta lika med $f(t)$, och beviset är klart.

\paragraph{$0$-funktionen och dens Fourierkoefficienter}
Antag att $f$ är integrerbar på enhetscirkeln och att alla dens Fourierkoefficienter är $0$. Då är $f(x) = 0$ överallt där $f$ är kontinuerlig.

\proof

\paragraph{Yttersta konvergenssats för Fourierserier}
Antag att $f$ uppfyller
\begin{itemize}
	\item $f$ är styckvis $C^{1}$ på $(-\pi, \pi]$.
	\item Höger- och vänstergränsvärdet existerar även mellan de olika intervallen där $f$ är $C^{1}$.
\end{itemize}
Då konvergerar Fourierserien $S_{N}$ till
\begin{itemize}
	\item $\lim\limits_{N\to\infty}S_{N}(x) = f(x)$ på något av intervallen.
	\item $\lim\limits_{N\to\infty}S_{N}(x) = \frac{f(x^{+}) + f(x^{-})}{2}$ på gränsen mellan två intervall, där $f(x^{\pm}) = \lim\limits_{t\to x^{\pm}}f(t)$.
\end{itemize}

\proof
Vi vill att
\begin{align*}
	\abs{\inteval{-\pi}{\pi}{f(x - t)D_{N}(t)}{t} - \frac{f(x^{+}) + f(x^{-})}{2}}\to 0.
\end{align*}
Detta är sant om
\begin{align*}
	\abs{\inteval{0}{\pi}{f(x - t)D_{N}(t)}{t} - \frac{f(x^{-})}{2}} + \abs{\inteval{-\pi}{0}{f(x - t)D_{N}(t)}{t} - \frac{f(x^{+})}{2}} \to 0.
\end{align*}
Vi betraktar ett av dessa uttrycken, då beviset är analogt för det andra.

Det gäller att Dirichletkärnan är jämn och integreras till $1$ på $[-\pi, \pi]$. Detta ger
\begin{align*}
	\frac{f(x^{-})}{2} = \inteval{0}{\pi}{f(x^{-})D_{N}(t)}{t},
\end{align*}
och vi får
\begin{align*}
	\abs{\inteval{0}{\pi}{f(x - t)D_{N}(t)}{t} - \frac{f(x^{-})}{2}} &= \abs{\inteval{0}{\pi}{(f(x - t) - f(x^{-}))D_{N}(t)}{t}} \\
	                                                                 &= \abs{\inteval{0}{\pi}{\frac{f(x - t) - f(x^{-})}{t}\frac{t}{\sin{\left(\frac{1}{2}t\right)}}\sin{\left(\frac{N + 1}{2}t\right)}}{t}}.
\end{align*}
Det första bråket är begränsad och kontinuerligt när $x - t$ ej är på gränsen mellan två intervall, och är därmed Riemannintegrerbar. Det samma är det andra bråket, och vi kan skriva detta som
\begin{align*}
	\abs{\inteval{0}{\pi}{f(x - t)D_{N}(t)}{t} - \frac{f(x^{-})}{2}} = \abs{\inteval{0}{\pi}{R(t)\sin{\left(\frac{N + 1}{2}t\right)}}{t}},
\end{align*}
där $R$ är Riemannintegrerbar. Enligt Riemann-Lebesgues lemma går detta mot $0$, och beviset är klart.

\paragraph{Begränsning av Fourierkoefficienter}
Låt $f\in C^{1}$ på enhetscirkeln. Då gäller att
\begin{align*}
	\abs{c_{n}} \leq \frac{C}{n},\ n \neq 0.
\end{align*}

\proof
Riemann-Lebesgues sats ger att $\dv{f}{x}$ har begränsade Fourierkoefficienter, vilket ger
\begin{align*}
	\abs{c_{n}\left(\dv{f}{x}\right)} &= \abs{\frac{1}{2\pi}\inteval{-\pi}{\pi}{\deval{f}{x}{t}e^{-int}}{t}} \\
	                                  &= \frac{1}{2\pi}\abs{f(\pi)e^{-in\pi} - f(-\pi)e^{in\pi} + in\inteval{-\pi}{\pi}{f(t)e^{-int}}{t}}.
\end{align*}
Eftersom $f$ är kontinuerlig på enhetscirkeln, ger detta
\begin{align*}
	f(\pi)e^{-in\pi} - f(-\pi)e^{in\pi} = f(\pi)(e^{-in\pi} - e^{in\pi}) = -2f(\pi)\sin{n\pi} = 0,
\end{align*}
och
\begin{align*}
	\abs{c_{n}\left(\dv{f}{x}\right)} &= \frac{1}{2\pi}\abs{in\inteval{-\pi}{\pi}{f(t)e^{-int}}{t}} \\
	                                  &= \frac{\abs{n}}{2\pi}\abs{\inteval{-\pi}{\pi}{f(t)e^{-int}}{t}} \\
	                                  &= \abs{n}\abs{c_{n}(f)}.
\end{align*}
Eftersom vänstersidan är begränsad av en konstant $C$, ger detta
\begin{align*}
	C \geq \abs{c_{n}\left(\dv{f}{x}\right)} = \abs{n}\abs{c_{n}(f)},
\end{align*}
och beviset är klart.

\subparagraph{Följdsats}
Låt $f\in C^{2}$ på enhetscirkeln. Då konvergerar dens Fourierserie mot funktionen.

\proof
Med samma resonnemang som i förra satsen får man
\begin{align*}
	\abs{c_{n}} \leq \frac{C}{n^{2}},
\end{align*}
och
\begin{align*}
	\sum\limits_{i = -N}^{N}\abs{c_{n}(f)e^{inx}} \leq \sum\limits_{i = -N}^{N}\abs{c_{n}(f)} \leq \sum\limits_{i = -N}^{N}\frac{C}{n^{2}},
\end{align*}
och därmed konvergerar summan.