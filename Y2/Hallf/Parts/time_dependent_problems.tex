\section{Tidsberoende problem}

\paragraph{Diskreta problem}
Betrakta ett problem där vi har $n$ punktmassor på en linje som rör sig normalt på linjens riktning, och har en förskjutning $w_{i}$ från jämviktsläget. Newtons andra lag ger
\begin{align*}
	m_{i}\ddot{w}_{i} = P_{i} - F_{i}.
\end{align*}
Här är $P_{i}$ den yttre kraften massa $i$ utsätts för och $F_{i}$ en sorts motståndsterm. Denna kan till exempel uppstå på grund av styvhet, och blir då på formen
\begin{align*}
	F_{i} = K_{ij}w_{j}.
\end{align*}
Problemet kan formuleras på matrisform som
\begin{align*}
	M\ddot{\vb{w}} + K\vb{w} = \vb{P}.
\end{align*}

Det homogena problemet kan lösas med en ansats $\vb{w} = \vb{a}\sin{\omega t}$. Detta ger upphov till ett egenvärdesproblem i egenfrekvenserna $\omega$.

Notera att 
\begin{align*}
	F_{i} = K_{ij}w_{j}.
\end{align*}
ger matrisrelationen
\begin{align*}
	\vb{F} = K\vb{w},
\end{align*}
där $K$ är styvhetsmatrisen. Vi hade alternativt kunnat ställa upp flexibilitetsmatrisen $\alpha$ för att få
\begin{align*}
	\vb{w} = \alpha\vb{F}.
\end{align*}
Det visar sig att både styvhetsmatrisen och flexibilitetsmatrisen är symmetriska.

\paragraph{Longitudinella svängningar i kontinuerliga kroppar}
Betrakta en kropp som påverkas av en normalkraft $N$ i varje ända. Snitta ut ett litet element. På ytan normal på svängningsriktningen får vi
\begin{align*}
	\rho A\dd{x}\del[2]{t}{u} = \dd{N}.
\end{align*}
Hookes lag ger
\begin{align*}
	N = EA\del{x}{u},
\end{align*}
vilket implicerar
\begin{align*}
	\rho A\del[2]{t}{u} = \del{x}{\left(EA\del{x}{u}\right)}.
\end{align*}
För en kropp med konstant styvhet $EA$ fås vågekvationen, med våghastighet
\begin{align*}
	c = \sqrt{\frac{EA}{\rho}}.
\end{align*}

\paragraph{Böjsvängningar i kontinuerliga kroppar}
Betrakta en kropp som böjs av någon utbredd last $q$. Snitta ut ett litet element. Kraftjämvikt i vertikal riktning ger
\begin{align*}
	T + \dd{T} - T + q\dd{x} = \dd{T} + q\dd{x} = \rho A\dd{x}\del[2]{t}{w} \implies \rho A\del[2]{t}{w} = q + \del{x}{T}.
\end{align*}
För låga frekvenser kan man betrakta rotationen av elementet som statisk, eventuellt inkludera en liten korrektion till egenfrekvenserna. Vi vet även från balkteori att
\begin{align*}
	\del{x}{M} = T,\ M = -EI\del[2]{x}{w}.
\end{align*}
Detta kan kombineras med resultaten ovan för att få
\begin{align*}
	\del[2]{x}{\left(EI\del[2]{x}{w}\right)} + \rho A\del[2]{t}{w} = q.
\end{align*}
Om det inte finns någon extern last och balken har konstant styvhet, reduceras detta till
\begin{align*}
	EI\del[4]{x}{w} + \rho A\del[2]{t}{w} = 0.
\end{align*}

\paragraph{Vågutbredning i kontinuerliga kroppar}
Om man gör en periodisk ansats för balkens utböjning, får man att vågtalet och frekvensen måste hänga ihop på något sätt. Detta kallas för dispersion.