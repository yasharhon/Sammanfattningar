\section{Avbildningar}

\subsection{Definitioner}

\paragraph{Isomorfir}
En isomorfi är en bijektiv avbildning mellan vektorrum.

\paragraph{Linjära avbildningar}
En avbildning $T$ är linjär om
\begin{align*}
	T(x + y) = T(x) + T(y), \\
	T(cx) = cT(x), c\in\R.
\end{align*}
Vi säjer att $T$ respekterar eller bevarar strukturen som vektorrum.

\paragraph{Matriser för linjära avbildningar}
Om $B = \{x_{i}\}_{i\in I}$ är en bas för $V$ och $D = \{y_{j}\}_{j\in J}$ är en bas för $W$ definieras matrisen för $L: V\to W$ i de givna baserna genom
\begin{align*}
	L(x_{i}) = \sum\limits_{j\in J}a_{ji}y_{j}.
\end{align*}
Linjära kombinationer är per definition ändliga, och därmed summeras det över ett ändligt antal termer även om $I$ är oändlig.

\paragraph{Analytiska funktioner av operatorer}
En analytisk funktion av en operator $L$ definieras som
\begin{align*}
	f(L) = \sum a_{i}L^{i}.
\end{align*}

\paragraph{Matrisnorm}
Normen av en matris definieras som
\begin{align*}
	\norm{A} = \sup\limits_{\norm{x} = 1}\norm{Ax}.
\end{align*}

\paragraph{Nilpotenta operatorer}
En operator $L$ är nilpotent om $L^{n} = 0$ för något $n$.

\paragraph{Duala avbildningar}
Om $L: V\to W$ är linjär, ges den duala avbildningen $L^{*}: W^{*}\to V^{*}$ av $L^{*}(\phi) = \comp{{\phi, L}}$.

\subsection{Satser}

\paragraph{Basbyte}
Låt $L$ vara en avbildning från $V$ till $W$. Låt $L_{B, D}$ vara en avbildning mellan vektorrum från basen $B$ i definitionsmängden till $D$ i målmängden, och låt $P_{A, B}$ vara avbildningen som byter bas från $A$ till $B$ i samma vektorrum. Då gäller det att
\begin{align*}
	L_{B, D} = P_{D', D}L_{B', D'}P_{B, B'}
\end{align*}

\proof
Kommutativt diagram

\paragraph{Koordinatavbildning}
Låt $B = \left\{x_{i}\right\}_{i\in I}$ vara en bas för vektorrummet $V$. Detta ger en isomorfi
\begin{align*}
	V                  &\to k^{I}\equiv \dirsum{i\in I}{}{k}, \\
	x = \sum a_{i}x{i} &\to \left\{a_{i}\right\}_{i\in I}.
\end{align*}

\proof
Avbildningen
\begin{align*}
	\left\{a_{i}\right\}_{i\in I} &\to \sum a_{i}x_{i}
\end{align*}
ger en avbildning $k^{I}\to V$ som är injektiv eftersom $B$ är linjärt oberoende och surjektiv eftersom $B$ spänner upp $V$. Eftersom denna avbildningen är bijektiv, måste även den inversa avbildningen vara bijektiv.

\paragraph{Kärna och injektivitet}
En linjär avbildning är injektiv om och endast om $\ker{L} = \{0\}$.

\proof
Antag att $L$ är injektiv. Det gäller att
\begin{align*}
	L(x) = L(y)\implies L(x - y) = 0\implies x - y\in\ker{L}.
\end{align*}
Alltså kan alla element i kärnan skrivas som differansen av två element som ej nödvändigtvis är i kärnan. Eftersom $L$ är injektiv, är $x - y = 0$, och kärnan innehåller endast $0$.

Antag nu att $\ker{L} = \{0\}$. Detta ger
\begin{align*}
	L(x) = L(y) \implies L(x - y) = 0 \implies x - y = 0,
\end{align*}
och beviset är klart.

\paragraph{Kvotavbildning}
Om $W\subseteq V$ är ett delrum , ger $x\to x + W$ en linjär kvotavbildning från $V$ till $\frac{V}{W}$.

\proof
Vi har
\begin{align*}
	x + y &\to x + y + W = x + W + y + W, \\
	ax    &\to ax + W = a(x + W),
\end{align*}
och beviset är klart.

\paragraph{Isomorfisatsen}
\begin{align*}
	\Im{L} \cong \frac{V}{\ker{L}}
\end{align*}

\proof
Avbildningen $\Phi(x + \ker{L}) = L(x)$ ger en väldefinierad avbildning från $\frac{V}{\ker{L}}$ till $\Im{L}$ eftersom $x + \ker{L} = y + \ker{L}$ implicerar $L(x) = L(y)$ ty $L$ är linjär. $\Phi$ är injektiv eftersom $\ker{\Phi} = \set{x + \ker{L}}{L(x) = 0} = \set{\ker{L}}{}$. Detta implicerar att om $\Phi(x + \ker{L}) = \Phi(y + \ker{L})$, är $x - y\in\ker{L}$, och de två är ekvivalenta sidoklasser. $\Phi$ är surjektiv eftersom $y = L(x)$ för något $x$ ger $y = \Phi(x + \ker{L})$, och alltså finns det för alla $y\in\Im{L}$ ett $x$ så att $y = \Phi(x + \ker{L})$.

\paragraph{Dimensionssatsen}
Om $V$ är ändligdimensionellt är $\rank{L} + \dim{\ker{L}} = \dim{V}$.

\proof

\paragraph{Faktorisering med kvotrum}
Om $U\subseteq\ker{L}$ finns det en unik avbildning $\Phi: \frac{V}{U}\to W$ sådan att $L = \Phi\circ\Psi$.

\proof
Definiera $\Phi(x + U) = L(x)$.

\paragraph{Norm av potenser av matriser}
\begin{align*}
	\norm{A^{i}}\leq \norm{A}^{i}
\end{align*}

\proof

\paragraph{Konvergens av funktioner av matriser}
En funktion $f$ av en matris konvergerar om
\begin{align*}
	f(\norm{A}) = \sum a_{i}\norm{A}^{i}
\end{align*}
konvergerar.

\proof

\paragraph{Matris för duala avbildningar}
Låt $L: V\to W$ ha matris $A$ för något val av baser för $V$ och $W$. Då har $L^{*}$ matris $A^{T}$ för ett motsvarande val av dual bas.

\proof
Vi tittar på avbildningar av baselement. Låt $\{e_{i}\}$ vara basen för $V$ och $\{f_{i}\}$ vara basen för $W$. Det motsvarande valet av bas för dualrummen är $\{e_{i}^{*}\}$ och $\{f_{i}^{*}\}$.

Definitionen ger att $L(e_{i}) = \sum\limits_{k}A_{ik}f_{k}$, vilket implicerar $L = \sum\limits_{k}\sum\limits_{l}A_{kl}f_{l}e_{k}^{*}$. Definitionen av $L^{*}$ ger vidare
\begin{align*}
	L^{*}(f_{i}^{*}) &= \comp{{f_{i}^{*}, \sum\limits_{k}\sum\limits_{l}A_{kl}f_{l}e_{k}^{*}}} \\
	                 &= \sum\limits_{k}\sum\limits_{l}A_{kl}f_{i}^{*}(f_{l})e_{k}^{*} \\
	                 &= \sum\limits_{k}\sum\limits_{l}A_{kl}\delta_{il}e_{k}^{*} \\
	                 &= \sum\limits_{k}A_{ki}e_{k}^{*} \\
	                 &= \sum\limits_{k}A_{ik}^{T}e_{k}^{*},
\end{align*}
vilket bevisar satsen.