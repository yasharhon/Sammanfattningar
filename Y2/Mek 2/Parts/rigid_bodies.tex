\section{Stela kroppar}
En stel kropp är en massbelagd domän så att avståndet mellan två godtyckliga punkter är konstant.

\subsection{Allmän kinematik}
En stel kropp kan ha translationshastighet eller rotationshastighet. Translationshastighet karakteriseras av att $\vb{v}_{A} = \vb{v}_{B}$ för alla $A, B$. Rotationshastighet karakteriseras av att det finns ett $C$ som är stelt förenad med kroppen så att $\vb{v}_{C} = \vb{0}$ momentant. 

\paragraph{Hastighet}
För att beskriva rörelsen till en stel kropp, bilda en referensram med axlerna fixa relativt kroppen. Betrakta två punkter $A, B$ i kroppen, där origo i den nya referensramen är $A$. Då gäller det att
\begin{align*}
	\vb{v}_{B} = \vb{v}_{B, \text{sp}} + \vb{v}_{B, \text{rel}}.
\end{align*}
Eftersom axlerna är fixa relativt kroppen, ger andra termen inget bidrag, vilket ger
\begin{align*}
	\vb{v}_{B} = \vb{v}_{A} + \vb*{\omega}\times\vb{r}_{AB}
\end{align*}
och bekräftar vårt påstående om att all rörelse för en stel kropp är antingen translation eller rotation.

\paragraph{Acceleration}
Kroppens acceleration ges av
\begin{align*}
	\vb{a}_{B} = \vb{a}_{B, \text{sp}} + \vb{a}_{B, \text{cor}} + \vb{a}_{B, \text{rel}}.
\end{align*}
Fixa axler relativt kroppen ger att de två sista termerna ej bidrar och
\begin{align*}
	\vb{a}_{B} = \vb{a}_{A} + \dv{\vb*{\omega}}{t}\times\vb{r}_{AB} + \vb*{\omega}\times(\vb*{\omega}\times\vb{r}_{AB}),
\end{align*}
där den första termen är ett translatoriskt bidrag och de två andra är rotationsbidrag.

\subsection{Allmän dynamik}

\paragraph{Allmän rotation kring fix axel}
Vi betraktar en stel kropps rotation kring en fix punkt $O$ parallellt med en axel $Q$. Vi inför vinklarna $\alpha, \beta, \gamma$ mellan $\vb*{\omega}$ och axlarna, enhetsvektorn
\begin{align*}
	\vu{e}_{Q} = \cos{\alpha}\vu{e}_{x} + \cos{\beta}\vu{e}_{y} + \cos{\gamma}\vu{e}_{z},
\end{align*}
vinkeln $\theta_k$ mellan $\vb*{\omega}$ och $\vb{r}_{k}$ och avståndet $\rho_{k}$ från rotationsaxeln till partikeln. Vi beräknar först
\begin{align*}
	\rho_{k}^2 =& r_{k}^2\sin^{2}{\theta} \\
	           =& \abs{\vb{r_{k}}\times\vu{e}_{Q}}^2 \\
	           =& \abs{(y_{k}\cos{\gamma} - z_{k}\cos{\beta})\vu{e}_{x} + (z_{k}\cos{\alpha} - x_{k}\cos{\gamma})\vu{e}_{y} + (x_{k}\cos{\beta} - y_{k}\cos{\alpha})\vu{e}_{z}}^2 \\
	           =& (y_{k}\cos{\gamma} - z_{k}\cos{\beta})^2 + (z_{k}\cos{\alpha} - x_{k}\cos{\gamma})^2 + (x_{k}\cos{\beta} - y_{k}\cos{\alpha})^2 \\
	           =& (y_{k}^2 + z_{k}^2)\cos^{2}{\alpha} + (x_{k}^2 + z_{k}^2)\cos^{2}{\beta} + (x_{k}^2 + y_{k}^2)\cos^{2}{\gamma} \\
	            &- 2y_{k}z_{k}\cos{\beta}\cos{\gamma} - 2x_{k}z_{k}\cos{\alpha}\cos{\gamma} - 2x_{k}y_{k}\cos{\alpha}\cos{\beta}.
\end{align*}

Vi inför nu tröghetsmomentet
\begin{align*}
	I_{Q} =& \sum m_{i}\rho_{i}^2 \\
	      =& \sum m_{i}(y_{k}^2 + z_{k}^2)\cos^{2}{\alpha} + \sum m_{i}(x_{k}^2 + z_{k}^2)\cos^{2}{\beta} + \sum m_{i}(x_{k}^2 + y_{k}^2)\cos^{2}{\gamma} \\
	       &- 2\sum m_{i}y_{k}z_{k}\cos{\beta}\cos{\gamma} - 2\sum m_{i}x_{k}z_{k}\cos{\alpha}\cos{\gamma} - 2\sum m_{i}x_{k}y_{k}\cos{\alpha}\cos{\beta}
\end{align*}
och definierar tröghetstensorn $I$ med komponenter
\begin{align*}
	I_{xx} = \sum m_{i}y_{i}^2 + z_{i}^2, I _{xy} = -\sum m_{i}x_{i}y_{i} 
\end{align*}
och motsvarande för andra subskript. Detta ger
\begin{align*}
	I_{Q} =& I_{xx}\cos^{2}{\alpha} + I_{yy}\cos^{2}{\beta} + I_{zz}\cos^{2}{\gamma} \\
	       &+ 2I_{yz}\cos{\beta}\cos{\gamma} + 2I_{xz}\cos{\alpha}\cos{\gamma} + 2I_{xy}\cos{\alpha}\cos{\beta} \\
	      =& \vu{e}_{Q}\cdot I\vu{e}_{Q}.
\end{align*}

\paragraph{Mer om tröghetstensorn}
Tröghetstensorn är symmetrisk, och kan därmed diagonaliseras. Axlarna så att tröghetsmatrisen är diagonal relativt dessa kallas huvudaxlar.

\subparagraph{Parallellförflyttningssatsen för diagonalelementer}
Betrakta en stel kropp. Inför två referensramer $S, S'$, där den primmade ramen är ett masscentrumssystem, kroppen roterar kring $z'$-axeln, $z$-axeln är parallell med $z'$-axeln och avståndet mellan dessa är $d$. Vi inför tröghetsmomenten
\begin{align*}
	I_{zz}   &= \sum m_{i}(x_{i}^2 + y_{i}^2), \\
	I_{z'z'} &= \sum m_{i}(x_{i}'^2 + y_{i}'^2).
\end{align*}
Vi skriver om första enligt
\begin{align*}
	I_{z} &= \sum m_{i}(x_{G}^2 + 2x_{i}'x_{G} + x_{i}'^{2} + y_{G}^2 + 2y_{i}'y_{G} + y_{i}'^{2}) \\
	      &= \sum m_{i}(x_{G}^2 + y_{G}^2) + \sum m_{i}(x_{i}'^{2} + y_{i}'^{2}) + 2x_{G}\sum m_{i}x_{i}' + 2y_{G}\sum m_{i}y_{i}' \\
	      &= md^2 + I_{z'z'},
\end{align*}
där de två sista summorna försvinner eftersom $S'$ är ett masscentrumssystem. Härledningen är analog för de andra diagonalelementen.

\subparagraph{Parallellförflyttningssatsen för tröghetsprodukter}
Betrakta en stel kropp. Inför två referensramer $S, S'$, där den primmade ramen är ett masscentrumssystem och enhetsvektorerna är parallella. Vi har
\begin{align*}
	I _{xy} &= -\sum m_{i}x_{i}y_{i} \\
	        &= -\sum m_{i}(x_{G}y_{G} + x_{i}'y_{G} + x_{G}y_{i}' + x_{i}'y_{i}') \\
	        &= -mx_{G}y_{G} - y_{G}\sum m_{i}x_{i}' - x_{G}\sum m_{i}y_{i}' - \sum m_{i}x_{i}'y_{i}' \\
	        &= I_{x'y'} - mx_{G}y_{G}.
\end{align*}

\subparagraph{Kinetisk energi}
Den relativa kinetiska energin ges av
\begin{align*}
	T' = \frac{1}{2}\sum m_{i}v_{i}'^{2}.
\end{align*}
Med $\rho_{i}$ som avståndet till rotationsaxeln ger hastighetssambandet
\begin{align*}
	T' &= \frac{1}{2}\sum m_{i}\rho_{i}^{2}\omega^{2} \\
	   &= \frac{1}{2}I_{G}\omega^{2}.
\end{align*}
Den totala kinetiska energin ges av
\begin{align*}
	T = \frac{1}{2}mv_{G}^{2} + \frac{1}{2}I_{G}\omega^{2},
\end{align*}
som innehåller en banterm och en spinnterm. Med hjälp av tröghetstensorn kan den andra termen skrivas som
\begin{align*}
	T' &= \frac{1}{2}I_{Q}\omega^2 \\
	   &= \frac{1}{2}\vu{e}_{Q}\cdot I\vu{e}_{Q}\omega^2 \\
	   &= \frac{1}{2}\omega\vu{e}_{Q}\cdot I\omega\vu{e}_{Q} \\
	   &= \frac{1}{2}\vb*{\omega}\cdot I\vb*{\omega}.
\end{align*}

\paragraph{Rörelsemängdsmoment}
Rörelsemängdsmomentet kan skrivas som
\begin{align*}
	\vb{H}_{O} &= \sum \vb{r}_{i}\times m_{i}\vb{v}_{i} \\
	           &= \sum m_{i}\vb{r}_{i}\times(\vb*{\omega}\times\vb{r}_{i}) \\
	           &= \sum m_{i}((\vb{r}_{i}\cdot\vb{r}_{i})\vb*{\omega} - (\vb{r}_{i}\cdot\vb*{\omega})\vb{r}_{i}).
\end{align*}
Vi betraktar nu $x$-komponenten, som ges av
\begin{align*}
	H_{O, x} &= \sum m_{i}((x_{i}^2 + y_{i}^2 + z_{i}^2)\omega_{x} - (x_{i}\omega_{x} + y_{i}\omega_{y} + z_{i}\omega_{z})x_{i}) \\
	         &= \sum m_{i}((y_{i}^2 + z_{i}^2)\omega_{x} - (y_{i}\omega_{y} + z_{i}\omega_{z})x_{i}) \\
	         &= I_{xx}\omega_{x} + I_{xy}\omega_{y} + I_{xz}\omega_{z}.
\end{align*}
Från detta fås
\begin{align*}
	\vb{H}_{O} = I\vb*{\omega}.
\end{align*}
Med detta kan den kinetiska energin i masscentrumssystemet skrivas som
\begin{align*}
	T' = \frac{1}{2}\vb*{\omega}\cdot I\vb*{\omega}.
\end{align*}

\paragraph{Energilagen}
Definiera effekten
\begin{align*}
	P_{ij} = \vb{f}_{ij}\cdot\vb{v}_{i} + \vb{f}_{ji}\cdot\vb{v}_{j} = \vb{f}_{ij}\cdot(\vb{v}_{i} - \vb{v}_{j}).
\end{align*}
Vi använder att $\vb{v}_{j} = \vb{v}_{i} + \vb*{\omega}\times(\vb{r}_{j} - \vb{v}_{i})$ och får
\begin{align*}
	P_{ij} = -\vb{f}_{ij}\cdot\vb*{\omega}\times(\vb{r}_{j} - \vb{v}_{i}) = 0
\end{align*}
eftersom $\vb{f}_{ij}$ verkar längs linjen mellan partikel $i$ och $j$. Därmed gör de indre krafterna inget arbete, och
\begin{align*}
	U_{0 - 1}^{\text{(e)}} = T_{1} - T_{0}.
\end{align*}

\paragraph{Arbete}
För att beräkna arbetet som görs på en stel kropp i allmän rörelse, betrakta
\begin{align*}
	T = \frac{1}{2}mv_{G}^{2} + \frac{1}{2}\vb*{\omega}\cdot I\vb*{\omega}.
\end{align*}
Med energilagen fås nettoeffekten $P$ på kroppen av
\begin{align*}
	P &= \dv{T}{t} \\
	  &= m\dv{\vb{v}_{G}}{t}\cdot\vb{v}_{G} + \vb*{\omega}\cdot I\vb*{\alpha} \\
	  &= \vb{F}\cdot\vb{v}_{G} + \vb{M}_{G}\cdot\vb*{\omega}.
\end{align*}
Arbetet ges då av
\begin{align*}
	\dd{U} = P\dd{t} = \vb{F}\cdot\dd{\vb{r}_{G}} + M_{G, n}\dd{\theta}
\end{align*}
där $M_{G, n}$ är momentets komponent normalt på rotationsplanet.

\paragraph{Eulers dynamiska ekvationer}
Betrakta en stel kropp som roterar kring en fix punkt $O$. Välj en rumsfixt inertialram $OXYZ$ och en kroppsfix ram $Oxyz$. Det gäller att
\begin{align*}
	\dv{\vb{H}_{O}}{t} = \relderiv{\vb{H}_{O}} + \vb*{\omega}\times\vb{H}_{O}.
\end{align*}
Eftersom kroppen är stel, har vi vidare
\begin{align*}
	\relderiv{\vb{H}_{O}} = I_{O}\relderiv{\vb*{\omega}}.
\end{align*}
Rörelsemängdsmomentlagen ger då Eulers dynamiska ekvation
\begin{align*}
	I_{O}\relderiv{\vb*{\omega}} + \vb*{\omega}\times\vb{H}_{O} = \dv{\vb{M}_{O}}{t}.
\end{align*}
Med den kroppsfixa ramen orienterat längs med kroppens huvudaxler blir detta på komponentform
\begin{align*}
	I_{xx}\dv{\omega_{x}}{t} + (I_{zz} - I_{yy})\omega_{y}\omega_{z} &= M_{O, x}, \\
	I_{yy}\dv{\omega_{y}}{t} + (I_{xx} - I_{zz})\omega_{z}\omega_{x} &= M_{O, y}, \\
	I_{zz}\dv{\omega_{z}}{t} + (I_{yy} - I_{xx})\omega_{x}\omega_{y} &= M_{O, z}.
\end{align*}

\paragraph{Eulers kinematiska ekvationer}
Betrakta en stel kropp som roterar kring en fix punkt $O$. Välj en rumsfixt inertialram $OXYZ$ och en kroppsfix ram $Oxyz$. Låt linjen där $xy$-planet skär $XY$-planet vara parallell med $\vb{n}$, som igen är parallell med $\vu{e}_{Z}\times\vu{e}_{z}$. Introducera vinkeln $\theta$ mellan $Z$- och $z$-axeln, vinkeln $\psi$ mellan $X$-axeln och $\vb{n}$ och vinkeln $\phi$ mellan $\vb{n}$ och $x$-axeln. Dessa kallas för Eulervinklarna. Vinkelhastigheten är en summa av vinkelhastigheter motsvarande ändring av varje vinkel, och ges av
\begin{align*}
	\vb*{\omega} = \dv{\psi}{t}\vu{e}_{Z} + \dv{\theta}{t}\vu{e}_{n} + \dv{\phi}{t}\vu{e}_{z}.
\end{align*}
Projektion på det kroppsfixa systemet ger
\begin{align*}
	\vu{e}_{Z} &= \cos{\theta}\vu{e}_{z} + \sin{\theta}(\sin{\phi}\vu{e}_{x} + \cos{\phi}\vu{e}_{y}), \\
	\vu{e}_{n} &= \cos{\phi}\vu{e}_{x} - \sin{\phi}\vu{e}_{y}.
\end{align*}
Från detta fås Eulers kinematiska ekvationer
\begin{align*}
	\omega_{x} &= \dv{\psi}{t}\sin{\theta}\sin{\phi} + \dv{\theta}{t}\cos{\phi}, \\
	\omega_{y} &= \dv{\psi}{t}\sin{\theta}\cos{\phi} - \dv{\theta}{t}\sin{\phi}, \\
	\omega_{z} &= \dv{\psi}{t}\cos{\theta} + \dv{\phi}{t}.
\end{align*}

\paragraph{3D-rotation av en axisymmetrisk kropp}
Betrakta en axisymmetrisk stel kropp som roterar kring en fix punkt $O$. Välj en rumsfixt inertialram $OXYZ$ och en ram $Oxyz$, där $z$-axeln pekar längs med kroppens symmetriaxel och kroppen kan rotera fritt kring detta. Denna sortens ram kallas en resalram. Nu kan $x$-axeln väljas parallellt med $\vb{n}$. Kroppens rotation kan nu delas i resalramens vinkelhastighet
\begin{align*}
	\vb*{\omega}_{S} = \dv{\psi}{t}\vu{e}_{Z} + \dv{\theta}{t}\vu{e}_{n}
\end{align*}
och kroppens vinkelhastighet relativt resalramen
\begin{align*}
	\vb*{\omega}_{0} = \dv{\phi}{t}\vu{e}_{z}.
\end{align*}
Nu får vi
\begin{align*}
	\dv{\vb{H}_{O}}{t} = \relderiv{\vb{H}_{O}} + \vb*{\omega}_{S}\times\vb{H}_{O}
\end{align*}
och
\begin{align*}
	\relderiv{\vb{H}_{O}} = I_{O}\relderiv{\vb*{\omega}_{0}},
\end{align*}
vilket slutligen ger
\begin{align*}
	I_{O}\relderiv{\vb*{\omega}_{0}} + \vb*{\omega}_{S}\times\vb{H}_{O} = \dv{\vb{M}_{O}}{t}.
\end{align*}

\subsection{Plan rörelse}
Plan rörelse för en stel kropp karakteriseras av att hastigheten i alla punkter är parallellt med ett och samma fixa plan. Om rörelsen är i $xy$-planet, kommer $\vb*{\omega}$ peka längs med $z$-axeln.

\paragraph{Momentancentrum}
Om en stel kropp roterar under plan rörelse, finns det alltid en punkt $C$ med $\vb{v}_{C} = \vb{0}$, som kallas momentancentrum. Denna punkt uppfyller $\vb{v}_{A} = -\vb*{\omega}\times\vb{r}_{AC}$. För att hitta den, multiplicera med $\vb*{\omega}$ på båda sidor för att få
\begin{align*}
	\vb*{\omega}\times\vb{v}_{A} &= -\vb*{\omega}\times(\vb*{\omega}\times\vb{r}_{AC}) \\
	                             &= -(\vb*{\omega}\cdot\vb{r}_{AC})\vb*{\omega} + \omega^2\vb{r}_{AC}.
\end{align*}
Eftersom rörelsen är plan, behöver vi bara betrakta ett snitt av kroppen i rörelsesplanet, vilket gör att den första skalärprodukten blir $0$. Detta ger då positionen till momentancentrumet enligt
\begin{align*}
	\vb{r}_{AC} = \frac{1}{\omega^2}\vb*{\omega}\times\vb{v}_{A}.
\end{align*}

\subparagraph{Hastighet}
Vid att införa cylindriska koordinater kring origo fås ortsvektorn till någon partikel i kroppen som $\vb{r}_{i}' = \rho_{i}\vu{e}_{\rho}$. Partikelns hastighet ges av
\begin{align*}
	\vb{v}_{i} &= \vb{v}_{G} + \vb*{\omega}\times\vb{r}_{i}' \\
	           &= \vb{v}_{G} + \omega\rho_{i}\vu{e}_{\theta}.
\end{align*}

\paragraph{Acceleration}
Accelerationssambandet för plan rörelse ger
\begin{align*}
	\vb{a}_{B} = \vb{a}_{A} + \dv{\vb*{\omega}}{t}\times\vb{r}_{AB} + \vb*{\omega}\times(\vb*{\omega}\times\vb{r}_{AB}).
\end{align*}
Vi skriver ut termerna och får
\begin{align*}
	\vb{a}_{B} = \vb{a}_{A} + \alpha\vu{e}_{z}\times\vb{r}_{AB} + (\vb*{\omega}\cdot\vb{r}_{AB})\vb*{\omega} - \omega^2\vb{r}_{AB}.
\end{align*}
Eftersom rörelsen är plan, blir skalärprodukten $0$, och man får slutligen
\begin{align*}
	\vb{a}_{B} = \vb{a}_{A} + \alpha\vu{e}_{z}\times\vb{r}_{AB} - \omega^2\vb{r}_{AB}.
\end{align*}

\paragraph{2D-rotation kring fix axel}
Låt punkten $O$ vara på rotationsaxeln och välg cylindriska koordinater så att kroppen roterar kring $z$-axeln. För någon partikel i den stela kroppen har man
\begin{align*}
	\vb{v}_{i} = \vb{v}_{O} + \vb*{\omega}\times\vb{r}_{i} = \rho_{i}\omega\vu{e}_{\theta}.
\end{align*}
Den kinetiska energin ges nu av
\begin{align*}
	T = \frac{1}{2}\omega^2\sum m_{k}\rho_{i}^2
\end{align*}
och rörelsemängdsmomentet kring $O$ ges av
\begin{align*}
	\vb{H}_{O} &= \sum \vb{r}_{i}\times m_{i}\rho_{i}\omega\vu{e}_{\theta} \\
	           &= \sum m_{i}\rho_{i}^2\omega\vu{e}_{z} - \sum m_{i}\rho_{i}z_{i}\omega\vu{e}_{r}.
\end{align*}
Vi inför nu tröghetsmomentet kring $z$-axeln
\begin{align*}
	I_{z} = \sum m_{k}\rho_{i}^2,
\end{align*}
med motsvarande definitioner kring andra axlar (högst analog till elementer i tröghetstensorn). Då gäller att
\begin{align*}
	T     &= \frac{1}{2}I_{z}\omega^2, \\
	H_{z} &= I_{z}\omega.
\end{align*}

Kraftekvationens komponenter ger
\begin{align*}
	F_{r}      &= -ml\left(\dv{\theta}{t}\right)^2, \\
	F_{\theta} &= ml\dv[2]{\theta}{t}.
\end{align*}

Momentekvationen ger
\begin{align*}
	\dv{t}\left(I_{z}\dv{\theta}{t}\right) = M_{z}.
\end{align*}

Arbetet som görs på kroppen ges av
\begin{align*}
	\dd{U_{i}} = \vb{F}_{i}\cdot\dd{\vb{r}_{i}} = F_{i, \theta}\rho_{i}\omega\dd{t} = M_{i, z}\dd{\theta}.
\end{align*}

\paragraph{Tröghetsmoment för en plan kropp}
Låt oss först betrakta en plan kropp som roterar i planet. Då kan det visas att
\begin{align*}
	I_{z} = I_{x} + I_{y}.
\end{align*}

\subparagraph{Rörelsemängdsmoment}
Det relativa rörelsemängdsmomentet ges av
\begin{align*}
	\vb{H}_{G}' &= \sum \vb{r}_{i}'\times m_{i}\vb{v}_{i}' \\
	            &= \sum \rho_{i}^{2} m_{i}\vu{e}_{r}\times\vu{e}_{\theta} \\
	            &= I_{G}\omega\vu{e}_{z}.
\end{align*}
Det totala rörelsemängdsmomentet är då
\begin{align*}
	\vb{H}_{O} = \vb{r}_{OG}\times m\vb{v}_{G} + I_{G}\omega\vu{e}_{z}
\end{align*}
som också består av en banterm och en spinnterm.

\subparagraph{Dynamiska lagar}
För en stel kropp i plan rörelse gäller att
\begin{align*}
	m\vb{a}_{G} = \vb{F}, I_{G}\alpha = M_{z}.
\end{align*}

\subparagraph{Arbete}
Effekten som utövs under en kropps rörelse ges av
\begin{align*}
	P = \dv{T}{t} &= m\vb{a}_{G}\cdot\vb{v}_{G} + I\vb*{\alpha}\cdot\vb*{\omega} \\
	              &= \vb{F}\cdot\vb{v}_{G} + \vb{M}_{G}\cdot\vb*{\omega}.
\end{align*}
Arbetet som görs ges då av
\begin{align*}
	\dd{U} &= P\dd{t} \\
	       &= \vb{F}\cdot\vb{v}_{G}\dd{t} + \vb{M}_{G}\cdot\vb*{\omega}\dd{t} \\
	       &= \vb{F}\cdot\vb{v}_{G}\dd{t} + \vb{M}_{G}\cdot\vb*{\omega}\dd{t} \\
	       &= \vb{F}\cdot\dd{\vb{r}_{G}} + M_{z}\dd{\theta},
\end{align*}
där vi har användt att $\vb{M}_{G}$ pekar i $z$-riktning.