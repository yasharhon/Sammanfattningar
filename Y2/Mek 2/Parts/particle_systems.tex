\section{Partikelsystem}
Ett partikelsystem är en samling av $N$ partiklar med (konstanta) massor $m_i$ och total massa $m$ som samverkar. Varje partikel påverkas av yttre krafter med summa $\vb{F}_{i}$ samt inre krafter $\vb{f}_{ij}$ med alla andra partikler i systemet.

Vi antar att alla inre krafter verkar parallellt med linjen mellan partiklerna. Newtons andra lag ger $\vb{f}_{ij} = -\vb{f}_{ji}$, vilket även implicerar $\vb{f}_{ii}= \vb{0}$.

Vi definierar kraftsummorna
\begin{align*}
	\vb{F}     &= \sum \vb{F}_{i}, \\
	\vb{f}     &= \sum\limits_{i}\sum\limits_{j}\vb{f}_{ij}.
\end{align*}
Vi får
\begin{align*}
	\vb{f} = \sum\limits_{i}\sum\limits_{j}\vb{f}_{ij} = \sum\limits_{j}\sum\limits_{i}\vb{f}_{ij} = -\sum\limits_{j}\sum\limits_{i}\vb{f}_{ji} = -\vb{f},
\end{align*}
och därmed $\vb{f} = \vb{0}$.

\paragraph{Masscentrum}
Vi kommer ihåg att masscentrum för ett partikelsystem definieras som
\begin{align*}
	\vb{r}_{G} = \frac{1}{\sum m_i}\sum m_i\vb{r}_i.
\end{align*}

\paragraph{Rörelsemängd}
Systemets totala rörelsemängd ges av
\begin{align*}
	\vb{p} = \sum m_i\vb{v}_{i} = \dv{t}\left(\sum m_i\vb{r}_{i}\right) = \dv{m\vb{r}_{G}}{t} = m\vb{v}_{G}.
\end{align*}

\paragraph{Kraftekvationen för ett partikelsystem}
Kraftekvationen för en enda partikel ger
\begin{align*}
	m_{i}\dv[2]{\vb{r}_{i}}{t} = \vb{F}_{i} + \sum\limits_{j}\vb{f}_{ij}.
\end{align*}
Om vi adderar alla dessa ekvationer, får man
\begin{align*}
	\sum m_{i}\dv[2]{\vb{r}_{i}}{t} = \sum \vb{F}_{i} + \sum\limits_{i}\sum\limits_{j}\vb{f}_{ij}, \\
	\dv[2]{t}\left(\sum m_{i}\vb{r}_{i}\right) = \vb{F} + \vb{f}, \\
	\dv{t}\left(m\vb{v}_{G}\right) = \dv{\vb{p}}{t} = \vb{F},
\end{align*}
vilket är kraftekvationen som vi känner den. Med konstant massa kan detta även skrivas som
\begin{align*}
	m\vb{a}_{G} = \vb{F}.
\end{align*}

\paragraph{Energilagen för ett partikelsystem}
Arbetet som görs på en partikel i ett parikelsystem under en infinitesimal rörelse ges av
\begin{align*}
	\dd{U}_{i} = \vb{f}_{i}\cdot\dd{\vb{r}_i} + \vb{F}_{i}\cdot\dd{\vb{r}_i} = \dd{U}_{i}^{\text{(i)}} + \dd{U}_{i}^{\text{(e)}}.
\end{align*}
Det totala arbetet som görs på partikelsystemet ges av
\begin{align*}
	\dd{U} = \sum \dd{U}_{i}^{\text{(i)}} + \dd{U}_{i}^{\text{(e)}} = \dd{U}^{\text{(i)}} + \dd{U}^{\text{(e)}},
\end{align*}
där vi har infört arbetet som görs av inre och yttre krafter. Vi kan från detta integrera för att få
\begin{align*}
	U_{0 - 1} = T_{1} - T_{0},
\end{align*}
där $T$ nu är hela systemets kinetiska energi och $U$ är det totala arbetet som görs av alla krafter.

\subparagraph{Tolkning av kinetisk energi}
Vi undersöker vidare partikelsystemets kinetiska energi. För att göra detta, introducerar vi en masscentrumsram med origo i masscentrum och axlar som inte ändrar riktning. Hastighetssambandet ger
\begin{align*}
	\vb{v}_{i} = \vb{v}_{G} + \vb{v}_{G}' + \vb*{\omega}\times\vb{r}_{i}',
\end{align*}
där apostrofen indikerar storheter i masscentrumsramen. Eftersom systemet inte roterar, förenklas detta till
\begin{align*}
	\vb{v}_{i} = \vb{v}_{G} + \vb{v}_{i}'.
\end{align*}
Den kinetiska energin ges då av
\begin{align*}
	T &= \frac{1}{2}\sum m_{i}v_{i}^2 \\
	  &= \frac{1}{2}\sum m_{i}v_{G}^2 + \sum m_{i}\vb{v}_{i}'\cdot\vb{v}_{G} + \frac{1}{2}\sum m_{i}v_{i}'^2 \\
	  &= \frac{1}{2}\sum m_{i}v_{G}^2 + \vb{v}_{G}\cdot\sum m_{i}\vb{v}_{i}' + \frac{1}{2}\sum m_{i}v_{i}'^2.
\end{align*}
Det gäller att
\begin{align*}
	\sum m_{i}\vb{r}_{i}' = \vb{0}
\end{align*}
eftersom systemets masscentrum är i origo. Derivation med avseende på tiden ger
\begin{align*}
	\sum m_{i}\vb{v}_{i}' = \vb{0},
\end{align*}
vilket ger
\begin{align*}
	T = \frac{1}{2}mv_{G}^2 + T'.
\end{align*}
Bidragen till den kinetiska energin är alltså masscentrumsrörelse och partiklernas rörelse relativt masscentrum.

\paragraph{Momentekvationen}
Systemets totala rörelsemängdsmoment med avseende på punkten $O$ ges av
\begin{align*}
	\vb{H}_{O} = \sum\vb{r}_{i}\times m_{i}\vb{v}_{i}.
\end{align*}
För att härleda kraftekvationen, utgår vi från kraftekvationen
\begin{align*}
	m_{i}\vb{a}_{i} = \vb{F}_{i} + \sum\vb{f}_{ij}.
\end{align*}
Multiplicera med ortsvektorn från vänster och summera över alla partikler för att få
\begin{align*}
	\sum\vb{r}_{i}\times m_{i}\vb{a}_{i} = \sum\vb{r}_{i}\times\vb{F}_{i} + \sum\sum\vb{r}_{i}\times\vb{f}_{ij}.
\end{align*}
Vi har att
\begin{align*}
	\dv{\vb{H}_{O}}{t} = \sum\dv{\vb{r}_{i}}{t}\times m_{i}\vb{v}_{i} + \sum\vb{r}_{i}\times m_{i}\dv{\vb{v}_{i}}{t} = \sum\vb{r}_{i}\times m_{i}\vb{a}_{i}
\end{align*}
eftersom vektorerna i första termen är lika varandra. Vi har vidare att
\begin{align*}
	\vb{r}_{i}\times\vb{f}_{ij} + \vb{r}_{j}\times\vb{f}_{ji} = \vb{f}_{ij}\times (\vb{r}_{i} - \vb{r}_{j}) = \vb{0},
\end{align*}
då den inre kraften är parallell med linjen mellan partiklerna. Den återstående termen är det totala momentet till de yttre krafterna, och vi får
\begin{align*}
	\dv{\vb{H}_{O}}{t} = \vb{M}_{O}.
\end{align*}

\subparagraph{Rörelsemängdsmoment med avseende på olika punkter}
Betrakta rörelsemängdsmomentet kring två punkter $A, B$. Det gäller att
\begin{align*}
	\vb{H}_{A} &= \sum\vb{r}_{A, i}\times m_{i}\vb{v}_{i} \\
	           &= \sum(\vb{r}_{AB} + \vb{r}_{B, i})\times m_{i}\vb{v}_{i} \\
	           &= \vb{r}_{AB}\times\sum m_{i}\vb{v}_{i} + \sum \vb{r}_{B, i}\times m_{i}\vb{v}_{i} \\
	           &= \vb{r}_{AB}\times m\vb{v}_{G} + \vb{H}_{B}.
\end{align*}

\paragraph{Tolkning av rörelsemängdsmomentet}
Betrakta rörelsemängdsmomentet med avseende på en fix punkt $O$ och masscentrum $G$ i ett masscentrumsystem. Sambandsformelen ger
\begin{align*}
	\vb{H}_{O} = \vb{r}_{G}\times m\vb{v}_{G} + \vb{H}_{G}.
\end{align*}
Rörelsemängdsmomentet med avseende på masscentrum ges av
\begin{align*}
	\vb{H}_{G} = \sum\vb{r}_{i}'\times m_{i}\vb{v}_{i}.
\end{align*}
För att skriva denna enbart med storheter i masscentrumsystemet, tidsderiverar man relationen
\begin{align*}
	\vb{r}_{i} = \vb{r}_{G} + \vb{r}_{i}'
\end{align*}
och får
\begin{align*}
	\dv{\vb{r}_{i}}{t} &= \dv{\vb{r}_{G}}{t} + \dv{\vb{r}_{i}'}{t}, \\
	\vb{v}_{i}         &= \vb{v}_{G} + \dv{\vb{r}_{i}'}{t}.
\end{align*}
För att derivera den sista termen, använder vi ekvation \ref{eq:relderiv} i fallet $\vb*{\omega} = \vb{0}$ för att få
\begin{align*}
	\vb{v}_{i} = \vb{v}_{G} + \vb{v}_{i}'.
\end{align*}
Detta ger
\begin{align*}
	\vb{H}_{G} &= \sum\vb{r}_{i}'\times m_{i}\vb{v}_{G} + \sum\vb{r}_{i}'\times m_{i}\vb{v}_{i}' \\
	           &= \left(\sum m_{i}\vb{r}_{i}'\right)\times \vb{v}_{G} + \sum\vb{r}_{i}'\times m_{i}\vb{v}_{i}' \\
	           &= \sum\vb{r}_{i}'\times m_{i}\vb{v}_{i}' \\
	           &= \vb{H}_{G}'
\end{align*}
enligt definitionen av masscentrum och dens ortsvektor i ett masscentrumssystem. Vi har nu explicit skrivit att rörelsemängdsmomentet i ett masscentrumssystem endast beror av storheter som är relativa det systemet. Detta ger slutligen relationen
\begin{align*}
	\vb{H}_{O} = \vb{H}_{G}' + \vb{r}_{G}\times m\vb{v}_{G}.
\end{align*}
Den första termen är rörelsemängdsmomentet relativt masscentrum, och den andra termen är banrörelsemängdsmomentet som uppstår från masscentrums rörelse.

\paragraph{Rörelsemängdsmomentlagen för en rörlig punkt}
Jämför rörelsemängdsmomenten relativt en fix punkt $O$ och relativt en annan punkt $A$. Vårt samband ger
\begin{align*}
	\vb{H}_{O} = \vb{H}_{A} + \vb{r}_{OA}\times m\vb{v}_{G}.
\end{align*}
Tidsderivation ger
\begin{align*}
	\dv{\vb{H}_{O}}{t} &= \dv{\vb{H}_{A}}{t} + \dv{\vb{r}_{OA}}{t}\times m\vb{v}_{G} + \vb{r}_{OA}\times m\dv{\vb{v}_{G}}{t} \\
	                   &= \dv{\vb{H}_{A}}{t} + \vb{v}_{OA}\times m\vb{v}_{G} + \vb{r}_{OA}\times m\vb{a}_{G} \\
	                   &= \dv{\vb{H}_{A}}{t} + \vb{v}_{OA}\times m\vb{v}_{G} + \vb{r}_{OA}\times\vb{F} \\
	                   &= \dv{\vb{H}_{A}}{t} + \vb{v}_{OA}\times m\vb{v}_{G} - \vb{r}_{AO}\times\vb{F}.
\end{align*}
Vi skriver om och använder rörelsemängdsmomentlagen för att få
\begin{align*}
	\dv{\vb{H}_{A}}{t} + \vb{v}_{OA}\times m\vb{v}_{G} = \vb{M}_{O} + \vb{r}_{AO}\times\vb{F}.
\end{align*}
Högersiden ger förflytningen av momentet till en ny punkt, som vi såg i grundkursen (ja, jag blev också chockad över att den fanns kvar), och vi får
\begin{align*}
	\dv{\vb{H}_{A}}{t} + \vb{v}_{OA}\times m\vb{v}_{G} = \vb{M}_{A}.
\end{align*}