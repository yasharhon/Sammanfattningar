\section{Fasta tillståndets fysik}

\paragraph{Kristaller}
Kristaller är en underkategori av fasta ämnen där atomerna i ämnet har en viss ordning. Mer specifikt är atomerna ordnade i ett mönster som upprepas perfekt i hela kristallen.

De flesta reella kristallinska ämnen är byggda upp av många små kristaller, där dessa gärna har olika orienteringar relativt varandra.

\paragraph{Bindning i kristaller}
Kristaller är oftast bundna av jonbindningar, kovalenta bindningar eller metallbindningar. Och vad är nu metallbindinigar?

\paragraph{Metallbindningar}
Atomerna i en kristall kan bindas av metallbindningar. Detta uppstår när atomerna i kristallen ger från sig sina elektroner. Elektronerna bildar en gas, medan de positiva metallatomerna hålls ihop av en balans mellan attraktion till elektrongasen och repulsion mellan andra atomer.

\paragraph{Materialers elektriska egenskaper}
Alla materialer kan delas i
\begin{itemize}
	\item isolatorer, som har väldigt hög resistivitet.
	\item ledare, som har väldigt låg resistivitet.
	\item halvledare, som har egenskaper någon stans mittemellan.
\end{itemize}

\paragraph{Bandstruktur}
Energinivåerna för en enda atom är kvantiserade, men i en kristalll kan elektronen leva överallt runt de olika atomerna. De tillåtna energinivåerna för elektronerna i en kristall konvergerar då mot intervall med tillåtna energier, så kallade energiband. Hur dessa banden ser ut för ett material kallas för materialets bandstruktur, och är närt kopplad till de elektriska egenskaperna.

\paragraph{Valensband och ledningsband}
Valensbandet är det bandet med högst energi som har elektroner i sig. Ledningsbandet är det bandet med lägst energi som har ookkuperade elektrontillstånd. Gapet mellan dessa kallas för bandgapet.

\paragraph{Ledares bandstruktur}
I en ledare krävs relativt låg energi för att exitera elektronerna med högst energi så att de kan röra på sig i kristallen. I termer av bandstruktur kan man säja att valensbandet och ledningsbandet är det samma, eller att bandgapet är litet. Eftersom denna processen kräver lite energi, är den spontan, och därmed ledar ledare elektricitet väl.

\paragraph{Isolatorers bandstruktur}
I en isolator är bandgapet stort, och det är svårt för elektroner att exiteras och börja röra på sig. Därmed ledar de elektricitet dåligt.

\paragraph{Halvledares bandstruktur}
Halvledare har ett bandgap som är någon stans imellan, vilket ger de vissa mer exotiska egenskaper.

\paragraph{Fyllning av band}
Elektroner fyller tillstånd i band med en given energi med en sannolikhet enligt Fermi-Dirac-fördelningen
\begin{align*}
	f(E) = \frac{1}{1 + e^{\frac{E - \mu}{kT}}}.
\end{align*}

\paragraph{Dopning av halvledare}
Dopning av en halvledare är att tillsätta olika andra ämnen till den. Detta gör att man kan ändra dens elektriska egenskaper, antingen vid att tillföra ookkuperade elektrontillstånd nära valensbandet eller fyllda elektrontillstånd nära ledningsbandet.

\paragraph{Supraledare}
Vissa materialer förlorar all sin resistivitet när de kyls ned till väldigt låga temperaturer. Dessa kalles supraledare.