\section{Interpolation}

Det fundamentala interpolationsproblemet går ut på att hitta en kurva som bäst möjligt passar med vissa datapunkter. Kom i håg satsen om möjlighet för polynominterpolation.

\paragraph{Polynominterpolation - första försök}
Vi gör först en naiv ansats
\begin{align*}
	p(x) = \sum\limits_{i = 0}^{n}c_{i}x^{i}
\end{align*}
och anpassar konstanterna så att $p$ antar rätt värden i datapunkterna. Detta ger oss ett linjärt system
\begin{align*}
	X\vb{c} = \vb{y},
\end{align*}
där $\vb{c}$ är en vektor med alla koefficienter, $\vb{y}$ är en vektor med alla $y$-värden och $X_{ij} = x_{j}^{i - 1}$. $X$ kallas för en Vandermonde-matris. Om man har många datapunkter, kan detta dock ge upphov till ett illakonditionerad system.

\paragraph{Lagrangeinterpolation}
Som vi så innan är ett möjligt interpolationspolynom
\begin{align*}
	p(x) = \sum\limits_{i = 1}^{n + 1}y_{i}\prod\limits_{j\neq i}\frac{x - x_{j}}{x_{i} - x_{j}}.
\end{align*}

\paragraph{Newtons interpolationsmetod}
Vi gör en ny ansats
\begin{align*}
	p(x) &= d_{0} + d_{1}(x - x_{0}) + d_{2}(x - x_{0})(x - x_{1}) + \dots \\
	     &= \sum\limits_{i = 0}^{n}d_{i}\prod\limits_{j < i}(x - x_{j}).
\end{align*}
Vi ser att $p(x_{0}) = y_{0}$, och man kan från detta få de nästa koefficienterna.

\paragraph{Minsta kvadratmetoden}
Minsta kvadratmetoden är en metod för approximation av överbestämda ekvationssystem, dvs. system med fler ekvationer än obekanta. Sådana system har inget interpolationspolynom.

Sådana ekvationssystem kan formuleras som $A\vb{x} = \vb{b}$. Minsta kvadratlösningen är den lösningen som minimerar $\norm{A\vb{x} - \vb{b}}$. Mer specifikt, om vi söker en funktion $f(x) = \sum\limits_{i = 1}^{n}c_{i}\phi{i}(x)$, där funktionerna $\phi_{i}$ kan väljas som vi vill, är $A_{ij} = \phi{j}(x_{i})$, $\vb{x}_{i} = c_{i}$ och $\vb{b}_{i} = y_{i}$.

Minstakvadratlösningen till detta systemet löser normalekvationerna
\begin{align*}
	A^{T}A\vb{x} = A^{T}\vb{b}.
\end{align*}
Om kolumnerna i $A$ är linjärt oberoende, har detta en lösning.