\section{Grundläggande koncept för numeriska metoder}

\paragraph{Lokal konvergens}
En numerisk metod säjs vara lokalt konvergent om den konvergerar mot ett givet numeriskt värde för startgissningar som är tillräckligt nära det rätta värdet.

\paragraph{Linjär konvergens}
En numerisk metod vars feltermer uppfyller
\begin{align*}
	\lim\limits_{n\to\infty}\frac{e_{n + 1}}{e_{n}} = S
\end{align*}
är linjärt konvergent med hastighet $S$.

\paragraph{Kvadratisk konvergens}
En numerisk metod vars feltermer uppfyller
\begin{align*}
	\lim\limits_{n\to\infty}\frac{e_{n + 1}}{e_{n}^{2}} = M
\end{align*}
är kvadratiskt konvergent.

\paragraph{Noggrannhetsordning}
Om $u_{h}$ är en approximation av en storhet $u$, där $h$ är positionsparametern, och det finns tall $p, C$ och $h_{0}$ sådana att
\begin{align*}
	\abs{u_{h} - u} \leq Ch^{p}, h\leq h_{0},
\end{align*}
sägs approximationen ha noggrannhetsordning $p$.

\paragraph{Absolut och relativt fel}
Låt $\tilde{x}$ vara ett approximativt värde av storheten $x$. Då definieras det absoluta felet som
\begin{align*}
	e_{x} = \tilde{x} - x
\end{align*}
och det relativa felet som
\begin{align*}
	r_{x} = \frac{e_{x}}{x}.
\end{align*}

\paragraph{Felgränser}
Felgränserna definieras som
\begin{align*}
	\abs{e_{x}} \leq E_{x}, \abs{r_{x}} \leq R_{x}.
\end{align*}

\paragraph{Felförplantning}
Låt $y = f(x)$ och $\tilde{y} = f(\tilde{x})$ vara en approximation av $y$. Detta ger
\begin{align*}
	e_{y} &= \tilde{y} - y \\
	      &= f(\tilde{x}) - f(x) \\
	      &\approx f(\tilde{x}) - \left(f(\tilde{x}) + \deval{f}{x}{\tilde{x}}(x - \tilde{x})\right) \\
	      &= -\deval{f}{x}{\tilde{x}}(x - \tilde{x})) \\
	      &= \deval{f}{x}{\tilde{x}}e_{x},
\end{align*}
och felgränsen ges av
\begin{align*}
	E_{y} \approx \abs{\deval{f}{x}{\tilde{x}}}E_{x}.
\end{align*}
Den motsvarande relationen i högre dimensioner är
\begin{align*}
	E_{y} \approx \sum\abs{\pdeval{f}{x_{i}}{\tilde{\vb{x}}}}E_{x_{i}}.
\end{align*}

Om vi nu utgår från ett approximativt värde $\tilde{y} = f(\tilde{\vb{x}})$ och approximerar de partiella derivatorna, till exempel genom mätningar, med
\begin{align*}
	\pdeval{f}{x_{i}}{\tilde{\vb{x}}} = \frac{f(\tilde{\vb{x}} + E_{x_{i}}\vb{e}_{x_{i}}) - f(\tilde{\vb{x}})}{E_{x_{i}}},
\end{align*}
kan man definiera
\begin{align*}
	\tilde{y}_{i} = f(\tilde{\vb{x}} + E_{x_{i}}\vb{e}_{x_{i}})
\end{align*}
och få
\begin{align*}
	E_{y} \approx \sum\abs{\tilde{y}_{i} - \tilde{y}}.
\end{align*}

\paragraph{Kancellation}
Om man på en dator subtraherar två nästan lika stora tal, förlorar man precision. Det suger.

\paragraph{Utskiftning}
Om man på en dator adderar två tal som är mycket olika stora, kan datorn ignorera information från den minre siffran.

\paragraph{Konditionstal}
Konditionstalet definieras som
\begin{align*}
	\kappa = \lim\limits_{\delta\to 0}\max\limits_{\abs{e_{x}} \leq \delta}\frac{\abs{\frac{e_{y}}{y}}}{\abs{\frac{e_{x}}{x}}}.
\end{align*}