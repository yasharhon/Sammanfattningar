\section{Inreprodukt}

\subsection{Definitioner}

\paragraph{Definitionen av inreprodukt}
Inreproduktet är en avbildning $V\times V\to\R$, var $V$ är ett vektorrum. Notationen är $\inprod{u}{v}$. Inreproduktet ska uppfylla
\begin{align*}
	\inprod{u + w}{v} &= \inprod{u}{v} + \inprod{w}{v}, \\
	\inprod{cu}{v}    &= c\inprod{u}{v}, c\in\R, \\
	\inprod{v}{u}     &= \inprod{u}{v}, \\
	\inprod{u}{u}     &\geq 0
\end{align*}
där inreproduktet av ett element med sig själv är $0$ om och endast om elementet är nollelementet.

\paragraph{Norm}
Normen till ett element, starkt analogt med längd, definieras som
\begin{align*}
	\norm{x} = \sqrt{\inprod{x}{x}}.
\end{align*}

\paragraph{Ortogonalitet}
Två elementer $u, v$ är ortogonala om
\begin{align*}
	\inprod{u}{v} = 0.
\end{align*}

\paragraph{Projektion}
Låt $L = \Span{x_1, \dots, x_n}$, var alla $x_i$ är ortogonala. Då ges projektionen av ett element $u$ på $L$ av
\begin{align*}
	\proj{L}{u} = \sum\limits_{i = 1}^{n}\frac{\inprod{u}{x_i}}{\norm{x_i}^2}x_i.
\end{align*}

\subsection{Satser}