\section{Vektorrum}

\subsection{Definitioner}

\paragraph{Grupper}
En grupp $G$ definieras av en mängd $X$ och en binär operation $\cdot$ på två elementer i $X$ (kommer ej skrivas ut). Denna operationen ska uppfylla
\begin{itemize}
	\item operationen är assosiativ, dvs. $a(bc) = (ab)c$.
	\item grupen är stängd under operationen, dvs. för $a, b\in G$ är $ab\in G$.
	\item det finns ett enhetselement $e$ så att $ae = ea = a$.
	\item det för varje element finns en invers så att $aa^{-1} = a^{-1}a = e$.
\end{itemize}

\paragraph{Abelska grupper}
En grupp är abelsk om den uppfyllar $ab = ba$ för alla $b, a\in X$.

\paragraph{Vektorrum}
Om man definierar skalärmultiplikation med elementer i en abelsk grupp, bildar grupen ett vektorrum $V$ under addition om
\begin{itemize}
	\item $cx\in V,~c\in\R,~x\in V$.
	\item $c(x + y) = cx + cy,~c\in\R,~x,y\in V$.
	\item $(c + d)x = cx + dx,~c,d\in\R$.
	\item $c(dx) = (cd)x$.
	\item $1x = x$.
\end{itemize}

\paragraph{Delrum}
En delmängd $V$ av ett vektorrum är ett delrum om
\begin{itemize}
	\item $e\in V$.
	\item $x, y\in V\implies x + y\in V$.
	\item $cx\in V$ för alla $c\in\R$.
\end{itemize}

\paragraph{Bas}
$v_1, \dots, v_k\in V$ är en bas för $V$ om
\begin{itemize}
	\item $\Span{v_1, \dots, v_k} = V$.
	\item vektorerna i basen är linjärt oberoende.
\end{itemize}
Om vi kallar basen till $V$ för $\beta$ kan alla vektorer i $V$ skrivas som basvektorer på följande sätt:
\begin{align*}
	x = \sum\limits_{i = 0}^{k}c_iv_i, \\
	[x]_\beta =
	\left[\begin{array}{c}
		c_1 \\
		c_2 \\
		\vdots \\
		c_k
	\end{array}\right].
\end{align*}
Detta är en vektor i $\R^k$.

\paragraph{Ortogonal bas}
En ortogonal bas för ett vektorrum $W$ är en ortogonal mängd av vektorer som bildar en bas för $W$.

\paragraph{Dimension}
Dimensionen till ett vektorrum är antalet vektorer i basis.

\subsection{Bevis}

\paragraph{Delrum i $\R^n$}
Om $V$ är ett delrum i $R^n$ kan det skrivas som $\Span{v_1, \dots, v_k}$.

\subparagraph{Bevis}
2 ez.

\paragraph{Linjärt beroende och basstorlek}
Låt $\beta = \{v_1, \dots, v_n\}$ vara en bas för $V$ och $\{w_1, \dots, w_p\}$ vara vektorer i $V$ med $p > n$. Då är $\{w_1, \dots, w_p\}$ linjärt beroende.

\subparagraph{Bevis}
Basvektorer.

\paragraph{Antal vektorer i en bas}
Antalet vektorer i basen för ett delrum $V$ är oberoende av valet av bas.

\subparagraph{Bevis}
Typ det samma.

\paragraph{Val av bas}
För ett delrum med dimension $n$ är vilken som helst mängd av $n$ linjärt oberoende vektorer i $V$ en bas för $V$.

\subparagraph{Bevis}
Mer av det här?!

\paragraph{Avbildningar med val av bas}
Låt $B = \{b_1, \dots, b_n\}$ vara en bas för delrummet $V$, $C$ en bas för delrummet $W$ och $T$ en linjär avbildning från $V$ till $W$. Då ges avbildningen i koordinatvektorer av
\begin{align*}
	[T(x)]_C = A[x]_B,
\end{align*}
där $A$ ges av
\begin{align*}
	A = [[T(b_1)]_C \dots [T(b_n)]_C].
\end{align*}

\paragraph{Vektorer och ortogonala baser}
Låt $\{u_1, \dots, u_n\}$ vara en ortogonal bas för $W$. Alla vektorer $w\in W$ kan skrivas som
\begin{align*}
	w = \sum\limits_{i = 1}^n \frac{\inprod{w}{u_i}}{\inprod{u_i}{u_i}}u_i.
\end{align*}

\subparagraph{Bevis}
Oklart.

\paragraph{Projektion}
Låt $\{u_1, \dots, u_n\}$ vara en ortogonal bas för $W$. Då gäller
\begin{align*}
	\proj{W}{x} = \sum\limits_{i = 1}^n\frac{\inprod{x}{u_i}}{\inprod{u_i}{u_i}}u_i.
\end{align*}

\subparagraph{Bevis}
Varför behövs det?

\paragraph{Bästa approximationers sats}
Låt $W$ vara ett delrum till $R^n$ och $\vect{x}\in\R^n$. Då gäller
\begin{align*}
	\abs{\vect{x} - \proj{W}{\vect{x}}} \leq \abs{\vect{x} - \vect{w}}, \vect{w}\in W.
\end{align*}

\subparagraph{Bevis}
Schmart.