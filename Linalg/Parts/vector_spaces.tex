\section{Vektorrum}

\subsection{Definitioner}

\paragraph{Gruper}
En grupp definieras av en mängd $X$ och en binär operation $\cdot$ på två elementer i $X$ (kommer ej skrivas ut). Denna operationen ska uppfylla
\begin{itemize}
	\item operationen är assosiativ, dvs. $a(bc) = (ab)c$.
	\item det finns ett enhetselement $e$ så att $ae = ea = a$.
	\item det för varje element finns en invers så att $aa^{-1} = a^{-1}a = e$.
\end{itemize}

\paragraph{Abelska gruper}
En grupp är abelsk om den uppfyllar $ab = ba$ för alla $b, a\in X$.

\paragraph{Vektorrum}

\paragraph{Delrum}
En delmängd $V$ av ett vektorrum är ett delrum om
\begin{itemize}
	\item $e\in V$.
	\item $x, y\in V\implies x + y\in V$.
	\item $cx\in V$ för alla $c\in\R$.
\end{itemize}

\paragraph{Bas i $R^n$}
$\vect{v}_1, \vect{v}_2, \dots, \vect{v}_k\in V$ är en bas för $V$ om
\begin{itemize}
	\item $\Span(\vect{v}_1, \vect{v}_2, \dots, \vect{v}_k) = V$.
	\item vektorerna i basen är linjärt oberoende.
\end{itemize}
Alla vektorer i $V$ kan skrivas som basvektorer på följande sätt:
\begin{align*}
	\vect{x} = \sum\limits_{i = 0}^{k}c_i\vect{v}_i, \\
	\vect{x}_\beta =
	\left[\begin{array}{c}
		c_1 \\
		c_2 \\
		\vdots \\
		c_k
	\end{array}\right].
\end{align*}

\paragraph{Dimension}
Dimensionen till ett vektorrum är antalet vektorer i basis.

\subsection{Bevis}

\paragraph{Delrum i $\R^n$}
Om $V$ är ett delrum i $R^n$ kan det skrivas som $\Span(\vect{v}_1, \vect{v}_2, \dots, \vect{v}_k)$.

\paragraph{Bevis}
2 ez.