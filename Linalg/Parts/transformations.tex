\section{Avbildningar}

\subsection{Definitioner}

\paragraph{Avbildningar}
En avbildning är en funktion som avbildar elementer från ett vektorrum till ett annat.

\paragraph{Bilden till en avbildning}
För en avbildning $T(\vect{x}): V\to W$ definierar man bildet till $T$ som
\begin{align*}
	\Im{T} = \{\vect{y} : \vect{y} = T(\vect{x})\}\subset W.
\end{align*}
Detta är ett delrum.

\paragraph{Kärnan till en avbildning}
För en avbildning $T(\vect{x}): V\to W$ definierar man kärnan till $T$ som
\begin{align*}
	\ker{T} = \{\vect{x} : T(\vect{x})= \vect{0}\}\subset V.
\end{align*}
Detta är ett delrum.

\paragraph{Linjära avbildningar}
En avbildning $T$ är linjär om
\begin{align*}
	T(\vect{x} + \vect{y}) = T(\vect{x}) + T(\vect{y}), \\
	T(c\vect{x}) = cT(\vect{x}), c\in\R.
\end{align*}

\paragraph{Egenvektorer och egenvärden}\label{par:eigen_values}
Låt $T: V\to V$ vara en linjär avbildning. Egenvektorerna till $T$ är alla nollskilda vektorer $\vect{x}\in V$ så att
\begin{align*}
	T(\vect{x}) = \lambda\vect{x}.
\end{align*}
$\lambda$ är det motsvarande egenvärdet till $\vect{x}$.

\paragraph{Egenrum}
Låt $\vect{x}$ vara en egenvektor med motsvarande egenvärde $\lambda$ till avbildningen $T$. Då definieras egenrummet motsvarande $\lambda$ enligt
\begin{align*}
	E_\lambda = \{\vect{x}\in V: T(\vect{x}) = \lambda\vect{x}\}.
\end{align*}
Merk att nollvektoren även inkluderas i $E_\lambda$, så att egenrummet är ett delrum.

\paragraph{Karakteristisk polynom}
Låt $T: \R^n\to\R^n$ med standardmatris $A$. Då är det karakteristiska polynomet (av grad $n$) till $T$
\begin{align*}
	C(\lambda) = \det{A - \lambda I},
\end{align*}
där $I$ är identitetsmatrisen.

\paragraph{Multiplicitet för egenvärden}
Ett egenvärdes algebraiska multipliciteten är multipliciteten till detta egenvärdet i det karakteristiska polynomet. Ett egenvärdes geometriska multiplicitet är dimensionen till egenrummet motsvarande detta egenvärdet. Det kommer ej göras en distinktion i denna sammanfattning.

\paragraph{Kvadratiska former}
En kvadratisk form är en avbildning på formen
\begin{align*}
	Q: \R^n\to\R, Q(\vect{x}) = \sum\limits_{1\leq i\leq j\leq n}d_{i, j}x_ix_j.
\end{align*}

\paragraph{Positiv och negativ definitet}
En kvadratisk form $Q$ är positivt definit om $Q(\vect{x}) > 0$ för alla $\vect{x}\neq\vect{0}$, och negativt definit om $Q(\vect{x}) < 0$ för alla $\vect{x}\neq\vect{0}$. Om det finns ett nollställe, är $Q$ antingen positivt eller negativt semidefinit. Om den ej hamnar i några av dessa kategorier, är $Q$ indefinit.

\paragraph{Kvadratiska kurvor}
En kvadratisk kurva är lösningsmängden till ett andragradspolynom i $n$ variabler.

\subsection{Satser}

\paragraph{Avildningar och enhetsvektorer}
För en linjär avbildning $T(\vect{x})$ har man att
\begin{align*}
	T(\vect{x}) = \sum\limits_{i = 1}^{n} x_iT(\vect{e}_i)
\end{align*}
där $x_i$ är komponenterna av $\vect{x}$.

\proof

\paragraph{Avbildningar och matriser}
För en avbildning $T(\vect{x}): \R^n\to\R^m$ kan man definiera matrisen
\begin{align*}
	A = \left[ T(\vect{e}_1) ~ T(\vect{e}_2) ~ \dots ~ T(\vect{e}_n) \right]
\end{align*}
var $\vect{e}_i$ är enhetsvektorerna i $\R^n$. Då kan avbildningen skrivas som
\begin{align*}
	T(\vect{x}) = A\vect{x}.
\end{align*}

\proof

\paragraph{Samansättning av linjära avbildningar}
För två linjära avbildningar $S, T$ är avbildningen $S\circ T$ linjär.

\proof

\paragraph{Dimensionalitet och avbildningar}
För en avbildning $T: \R^n\to\R^m$ är $\dim{\Im T} + \dim{\ker{T}} = n$.

\proof

\paragraph{Avbildningar med val av bas}
Låt $B = \{b_1, \dots, b_n\}$ vara en bas för delrummet $V$, $C$ en bas för delrummet $W$ och $T$ en linjär avbildning från $V$ till $W$. Då ges avbildningen i koordinatvektorer av
\begin{align*}
	[T(x)]_C = A[x]_B,
\end{align*}
där $A$ ges av
\begin{align*}
	A = [[T(b_1)]_C \dots [T(b_n)]_C].
\end{align*}

\proof

\paragraph{Karakteristisk polynom och egenvärden}
Låt $T:\R^n\to\R^n$. Egenvärden till $T$ är lösningerna till $C(\lambda) = 0$, där $C$ är det karakteristiska polynomet till $T$.

\proof

\paragraph{Egenrum och dimensionalitet}
Låt $T$ vara en avbildning med $p$ distinkta egenvärden. Då gäller:
\begin{itemize}
	\item Dimensionen till egenrummet motsvarande något egenvärde är mindre än eller lika med multipliciteten till detta egenvärdet.
	\item Standardmatrisen $A$, givet en avbildning mellan rum med samma dimensionalitet $n$, är diagonaliserbar om och endast om summan av egenrummens dimension är $n$.
	\item Om standardmatrisen är diagoniserbar och $B_k$ är en bas för egenrummet till egenvärdet $\lambda_k$, bilder samlingen av basvektorerna till alla $B_i$ en bas för $\R^n$.
\end{itemize}

\proof

\paragraph{Diagonalisering och avbildningar}
Låt $A = PDP^{-1}$ vara standardmatrisen för en avbildning, och $B$ en basis för $\R^n$ bildat av kolumnerna i $P$. Då är $D$ matrisen för avbildningen i bas $B$.

\proof

\paragraph{Definitet och egenvärden}
Låt $Q$ vara en kvadratisk form. Då gäller:
\begin{itemize}
	\item Q är positivt definit om och endast om alla egenvärden år positiva.
	\item Q är negativt definit om och endast om alla egenvärden år negativa.
	\item Q är indefinit om det finns båda negativa och positiva egenvärden.
\end{itemize}