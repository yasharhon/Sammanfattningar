\section{Linjära avbildningar}

\subsection{Definitioner}

\paragraph{Linjära avbildningar}
En avbildning $T(\vect{x})$ är linjär om
\begin{itemize}
	\item $T(\vect{x} + \vect{y}) = T(\vect{x}) + T(\vect{y})$ och
	\item $T(c\vect{x}) = cT(\vect{x})$ för något $c$.
\end{itemize}

\paragraph{Bilden till en avbildning}
För en avbildning $T(\vect{x})$ definierar man bildet till $T$ som
\begin{align*}
	Im(T) = \{\vect{y} : \vect{y} = T(\vect{x})\}.
\end{align*}

\paragraph{Nollrummet till en avbildning}
För en avbildning $T(\vect{x})$ definierar man nollrummet till $T$ som
\begin{align*}
	\text{Null}(T) = \{\vect{x} : T(\vect{x})= \vect{0}\}.
\end{align*}

\paragraph{Linjära avbildningar}
En avbildning $T$ är linjär om
\begin{align*}
	T(\vect{x} + \vect{y}) = T(\vect{x}) + T(\vect{y}), \\
	T(c\vect{x}) = cT(\vect{x}), c\in\R.
\end{align*}

\subsection{Satser}

\paragraph{Avildningar och enhetsvektorer}
För en linjär avbildning $T(\vect{x})$ har man att
\begin{align*}
	T(\vect{x}) = \sum\limits_{i = 1}^{n} x_iT(\vect{e}_i)
\end{align*}
var $x_i$ är komponenterna av $\vect{x}$.

\paragraph{Bevis}
Borde gå.

\paragraph{Avbildningar och matriser}
För en avbildning $T(\vect{x}): \R^n\to\R^m$ kan man definiera matrisen
\begin{align*}
	A = \left[ T(\vect{e}_1) ~ T(\vect{e}_2) ~ \dots ~ T(\vect{e}_n) \right]
\end{align*}
var $\vect{e}_i$ är enhetsvektorerna i $\R^n$. Då kan avbildningen skrivas som
\begin{align*}
	T(\vect{x}) = A\vect{x}.
\end{align*}

\subparagraph{Bevis}
Inte svårt alls.

\paragraph{Samansättning av linjära avbildningar}
För två linjära avbildningar $S, T$ är avbildningen $S\circ T$ linjär.

\subparagraph{Bevis}
Vi använder oss av definitionen.

\paragraph{Dimensionalitet och avbildningar}
För en avbildning $T: \R^n\to\R^m$ är $\dim{\Im T} + \dim{\Null T} = n$.