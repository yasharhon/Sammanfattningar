\section{Vektorer}
Begreppet vektor används här om elementer i vektorrum (eller inreproduktrum).

\subsection{Definitioner}

\paragraph{Linjärt hölje}
Det linjära höljet av vektorerna $v_1, \dots, v_n$ är
\begin{align*}
	\Span{v_1, \dots, v_n} = \left\{\sum\limits_{i = 1}^{n} t_iv_i\ |\ t_i\in\R \right\}
\end{align*}

\paragraph{Linjärt oberoende vektorer}
Vektorerna $v_1, \dots, v_n$ är linjärt oberoende om ekvationen
\begin{align*}
	\sum\limits_{i = 1}^{n}t_iv_i = 0
\end{align*}
endast har lösningen $t_i = 0$ för $i = 1, \dots, n$. $0$ refererar här till vektorrummets nollelement.

\paragraph{Ortogonala mängder}
Låt $\{u_1, \dots, u_n\}$ vara en mängd elementer så att $\inprod{u_i}{u_j} = 0$ för alla $i\neq j$. En slik mängd kallas en ortogonal mängd.

\paragraph{Enhetsvektorer i $\R^m$}
Vektorerna
\begin{align*}
	\vect{e}_1 =
	\left[\begin{array}{c}
    	1    \\
    	0    \\
    	\vdots \\
	    0
	\end{array}\right],
	\vect{e}_2 =
	\left[\begin{array}{c}
    	0    \\
    	1    \\
    	\vdots \\
	    0
	\end{array}\right],
	\dots, \vect{e}_m =
	\left[\begin{array}{c}
    	0    \\
    	0    \\
    	\vdots \\
	    1
	\end{array}\right]
\end{align*}
i $\R^m$ kallas enhetsvektorer. Man har att $\Span{\vect{e}_1, \vect{e}_2, \dots, \vect{e}_m} = \R^{m}$.

\paragraph{Skalärprodukt}
För två vektorer $\vect{u}, \vect{v}\in\R^n$ med koefficienter $u_i$ och $v_i$ definierar man
\begin{align*}
	\vect{u}\cdot\vect{v} = \sum\limits_{i = 1}^n u_iv_i.
\end{align*}
Detta är inreproduktet i $\R^n$.

\subsection{Satser}

\paragraph{Ortogonalitet och linjärt beroende}
Vektorerna i en ortogonal mängd är linjärt oberoende.

\subparagraph{Bevis}
Många multiplikationer.