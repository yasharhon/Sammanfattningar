\section{Geometri}

\subsection{Definitioner}

\paragraph{Linjer i $\R^2$}
En linje i $R^2$ definieras av en ekvation på formen $ax + by + c = 0$. Linjen är normal på
\begin{align*}
	\vect{n} = 
	\left[\begin{array}{c}
		a \\
		b
	\end{array}\right].
\end{align*}

\paragraph{Plan i $R^3$}
Ett plan i $R^3$ definieras av en ekvation på formen $ax + by + cz + d = 0$. Planet är normalt på
\begin{align*}
	\vect{n} = 
	\left[\begin{array}{c}
		a \\
		b \\
		c
	\end{array}\right].
\end{align*}

\paragraph{Kryssprodukt i $\R^3$}
Kryssproduktet av två vektorer ges av
\begin{align*}
	\vect{u}\times \vect{v} &=
	\left[\begin{array}{c}
		u_1 \\
		u_2 \\
		u_3
	\end{array}\right]
	\times
	\left[\begin{array}{c}
		v_1 \\
		v_2 \\
		v_3
	\end{array}\right] \\
	                        &=
	\left[\begin{array}{c}
		u_2v_3 - u_3v_2 \\
		-(u_1v_3 - u_3v_1) \\
		(u_1v_2 - u_2v_1)
	\end{array}\right]
\end{align*}

\subsection{Satser}

\paragraph{Distanser till hyperplan i $\R^n$}
Låt $H$ vara ett hyperplan definierad av ekvationen
\begin{align*}
	d + \sum\limits_{i = 1}^{n} a_ix_i = 0
\end{align*}
och
\begin{align*}
	N = \Span{
	\left[\begin{array}{c}
		a_1 \\
		a_2 \\
		\vdots \\
		a_n
	\end{array}\right]}
\end{align*}
vara dess ortogonala komplement. Då ges avstandet från en punkt $Q$ med koordinater $q_i, i = 1, 2, \dots, n$ av
\begin{align*}
	s = \frac{\sum\limits_{i = 1}^{n}a_iq_i}{\sqrt{\sum\limits_{i = 1}^{n}a_i^2}}
\end{align*}

\subparagraph{Bevis}
Aa.

\paragraph{Kryssproduktens egenskaper}
\begin{align*}
	(\vect{u}\times\vect{v})\cdot\vect{u} &= (\vect{u}\times\vect{v})\cdot\vect{v} = 0 \\
	\vect{u}\times\vect{v}                &= -\vect{v}\times\vect{u}
\end{align*}

\subparagraph{Bevis}

\paragraph{Plan med punkter}
Givet tre punkter $P, Q, R$ är vektorn
\begin{align*}
	\vect{n} = (Q - P)\times (R - P)
\end{align*}
normal på planet som innehåller de tre punkterna.

\subparagraph{Bevis}

\paragraph{Area av parallellogrammer i $\R^3$}
Arean av ett parallellogram i $\R^3$ definierad av två vektorer $\vect{u}, \vect{v}$ ges av
\begin{align*}
	A = \abs{\vect{u}\times\vect{v}}.
\end{align*}

\subparagraph{Bevis}

\paragraph{Avstånd mellan linjer}
Låt $L_1, L_2$ vara två linjer. \\
Om linjerna är parallella är avståndet
\begin{align*}
	s = \abs{(\vect{l}_1 - \vect{l}_2) - \proj{L_2'}{\vect{l}_1 - \vect{l}_2}}
\end{align*}
där $L_2'$ är parallell med $L_2$ och går genom origo. \\
Om linjerna är i parallella plan $H_1, H_2$ definierar man först
\begin{align*}
	P_1' = \{\vect{p}_1 + t\vect{n}_1, \vect{p}_1\in H_1\}\cap H_2
\end{align*}
som är parallell med $L_1$ och ligger i $H_2$ för ett smart val av $t$. Avståndet är då
\begin{align*}
	s = t\abs{\vect{n}}.
\end{align*}