\section{Speciella funktioner}

\paragraph{Trigonometriska och hyperbolskap funktioner}
Trigonometriska funktioner kan utvidgas via deras Taylorpolynom till att även ta komplexa argument. Det samma kan hyperbolska funktioner, vilket ger relationen
\begin{align*}
	\cos(ix) &= \cosh{x}, \\
	\sin(ix) &= i\sinh{x}.
\end{align*}

\paragraph{Besselfunktioner}
I lösning av Laplaces ekvation på enhetsskivan dyker det upp två funktioner $J_{n}$ och $Y_{n}$. Dessa har följande egenskaper:
\begin{itemize}
	\item $J_{n}$ är begränsad när $r\to\infty$.
	\item $J_{n}\propto r^{\abs{n}}$ då $r\to 0$.
	\item $J_{n}\propto r^{-\abs{n}}$ då $r\to 0$.
\end{itemize}
Dessa definieras, för heltaliga $n$ och positiva argument, av
\begin{align*}
	J_{n}(\theta) &= \frac{1}{2\pi}\inteval{0}{2\pi}{r}{re^{-in\theta + ir\sin{\theta}}}, \\
	Y_{n}(\theta) &= \frac{1}{\pi}\left(\inteval{0}{\pi}{r}{\sin(\theta\sin{r} - nr)} - \inteval{0}{\infty}{r}{(e^{nr} + (-1)^{n}e^{-nr})e^{-\theta\sinh{r}}}\right).
\end{align*}

\paragraph{Legendrepolynom}

\paragraph{Klotytefunktioner}
Laplaceoperatorn i sfäriska koordinater innehåller en del som endast beror på $r$ och en del som beror av vinklarna. Dens egenfunktioner är klotytefunktionerna
\begin{align*}
	Y_{l, m}(\theta, \phi) = Ne^{im\phi}P_{m}^{l}(\cos{\theta}).
\end{align*}
För ett fixt $l$ finns det $2l + 1$ möjliga värden av $m$. Dessa är alla heltal som uppfyller $\abs{m} \leq l$.