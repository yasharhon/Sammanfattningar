\section{Variationsanalys}

\paragraph{Funktionaler}
Variationsanalys handlar om funktionaler. Detta är avbildningar från funktioner till skalärer.

\paragraph{Extrempunkter och variationer}
Att analytiskt hitta en funktion som är en extrempunkt för en given funktional är allmänt inte enkelt. Den typiska strategin är att i stället anta att man har hittat en funktion som minimerar funktionalen, och introducera en variationsfunktion och en parameter $\varepsilon$ multiplicerad med den. Då vet man enligt antagandet att $\varepsilon = 0$ motsvarar en extrempunkt.

\paragraph{Tillväxt av funktionalen}
För att få en ide om hur en funktional beter sig beroende på $\varepsilon$, antag att $y$ minimerar funktionalen $J$, diskretisera $y$ i $N$ punkter och introducera variationen $\eta$. Detta ger
\begin{align*}
	J(y + \eta) - J(y) = \sum\limits_{i = 1}^{N}\pdv{J}{y_{i}} + O(\varepsilon^{2}).
\end{align*}
De partiella derivatorna

\paragraph{Variationsproblem typ 1}
Vi har en funktional $J$ som är extremal och funktionen $y$ är fixerad i randpunkterna.

\paragraph{Variationsproblem typ 2}
Vi har en funktional som är extremal och funktionen $y$ är fix i en punkt. Andra villkoret kommer från funktionalen.