\section{Kraftsystem}
Ett kraftsystem är ett system av krafter som verkar på en kropp och deras angrepspunkter.

\paragraph{Ekvimomenta kraftsystem}
Två kraftsystem är ekvimomenta om
\begin{itemize}
	\item $\sum(\vect{F}_i)_1 = \sum(\vect{F}_i)_2$, där subskriptet utanför parentesen bestämmer vilket kraftsystem kraften är i.
	\item $(\vect{M}_A)_1 = (\vect{M}_A)_2$, där $\vect{M}_A$ anger summan av alla kraftmoment med avseende på $A$.
\end{itemize}

Två ekvimomenta kraftsystem har samma moment i alla punkter eftersom
\begin{align*}
	(\vect{M}_B)_1 &= (\vect{M}_A)_1 + \vect{r}_{AB}\times (\vect{F})_1, \\
	(\vect{M}_B)_2 &= (\vect{M}_A)_2 + \vect{r}_{AB}\times (\vect{F})_2 \\
	               &= (\vect{M}_A)_1 + \vect{r}_{AB}\times (\vect{F})_1 \\
	               &= (\vect{M}_B)_1.
\end{align*}

\paragraph{Kraftpar}
Ett kraftpar består av två lika stora och motsatt riktade krafter som ej ligger på samma linje. Kraftsumman är $\vect{0}$ och beloppet av det totala momentet är
\begin{align*}
	M_O = dF,
\end{align*}
där $F$ är kraftens belopp och $d$ är avståndet mellan linjerna de två krafterna ligger på. Man kan visa att momentvektorn ej beror på val av $O$, och därmed kan placeras var som hälst i kroppen.

\paragraph{Förflyttning av krafter}
För ett kraftsystem av $n$ krafter kan man flytta alla dessa till punkten $A$ vid att för varje kraft $\vect{F}_i$ lägga till $\vect{F}_i, -\vect{F}_i$ i punkten $A$. Kraften i $A$ och kraften i $P_i$ bildar då ett kraftpar med moment $\vect{M}_i = \vect{r}_{AP_i}\times \vect{F}_i$ i punkten $A$. Momentet kan placeras i $A$ eftersom beloppet av momentet för ett kraftpar ej beror av valet av punkt. Då finns även en $\vect{F}_i$ kvar. Resultatet blir att kraftsystemet är ekvivalent med en enkelt kraft och ett enkelt moment som ges av
\begin{align*}
	&\vect{F} = \sum\vect{F}_i, \\
	&\vect{M} = \sum\vect{r}_{AP_i}\times\vect{F}_i.
\end{align*}

\paragraph{Förflyttning till ny punkt}
Låt oss försöka flytta krafterna till en ny punkt. Det är klart att kraftsumman är den samma, och
\begin{align*}
	\vect{M}_A &= \sum\vect{r}_{AP_i}\times\vect{F}_i \\
	           &= \sum(\vect{r}_{AB} + \vect{r}_{BP_i})\times\vect{F}_i \\
	           &= \sum\vect{r}_{AB}\times\vect{F}_i + \vect{r}_{BP_i}\times\vect{F}_i \\
	           &= \vect{r}_{AB}\times\sum\vect{F}_i + \sum\vect{r}_{BP_i}\times\vect{F}_i \\
	           &= \vect{r}_{AB}+times\vect{F} + \vect{M}_{B}
\end{align*}