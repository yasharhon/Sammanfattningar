\section{Kraftsystem}
Ett kraftsystem är ett system av krafter som verkar på en kropp samt deras angrepspunkter.

\paragraph{Ekvimomenta kraftsystem}
Två kraftsystem är ekvimomenta om
\begin{itemize}
	\item $\sum(\vect{F}_i)_1 = \sum(\vect{F}_i)_2$, där subskriptet utanför parentesen bestämmer vilket kraftsystem kraften är i.
	\item $(\vect{M}_A)_1 = (\vect{M}_A)_2$, där $\vect{M}_A$ anger summan av alla kraftmoment med avseende på $A$.
\end{itemize}

Två ekvimomenta kraftsystem har samma moment i alla punkter eftersom
\begin{align*}
	(\vect{M}_B)_1 &= (\vect{M}_A)_1 + \vect{r}_{AB}\times (\vect{F})_1, \\
	(\vect{M}_B)_2 &= (\vect{M}_A)_2 + \vect{r}_{AB}\times (\vect{F})_2 \\
	               &= (\vect{M}_A)_1 + \vect{r}_{AB}\times (\vect{F})_1 \\
	               &= (\vect{M}_B)_1.
\end{align*}

\paragraph{Kraftpar}
Ett kraftpar består av två lika stora och motsatt riktade krafter som ej ligger på samma linje. Kraftsumman är $\vect{0}$ och momentet till ett kraftpar med avseende på en godtycklig punkt $O$ ges av
\begin{align*}
	\vect{M}_O &= \vect{r}_{OA}\times\vect{F} + \vect{r}_{OB}\times (-\vect{F}) \\
	           &= (\vect{r}_{OA}- \vect{r}_{OB})\times\vect{F} \\
	           &= \vect{r}_{BA}\times\vect{F},
\end{align*}
och beror ej av $O$. Kraftmomentets belopp är $M = dF$, där $d$ är avståndet mellan krafternas verkningslinjer.

\paragraph{Förflyttning av krafter}
För ett kraftsystem av $n$ krafter kan man flytta alla dessa till punkten $A$ vid att för varje kraft $\vect{F}_i$ lägga till $\vect{F}_i, -\vect{F}_i$ i punkten $A$. Kraften i $P_i$ och en av krafterna i $A$ bildar då ett kraftpar med moment $\vect{M}_i = \vect{r}_{AP_i}\times\vect{F}_i$ m.a.p. punkten $A$ (eller någon annan punkt). Då finns även en $\vect{F}_i$ kvar i $A$. Efter att alla krafterna är flyttade blir resultatet en enkelt kraft och ett enkelt moment, båda i $A$, som ges av
\begin{align*}
	&\vect{F} = \sum\vect{F}_i, \\
	&\vect{M} = \sum\vect{r}_{AP_i}\times\vect{F}_i.
\end{align*}
Detta kallas kraftsystemets reduktionsresultat.

\paragraph{Förflyttning till ny punkt}
Låt oss jämföra reduktionsresultatet för två olika punkter. Det är klart att kraftsumman är den samma, och
\begin{align*}
	\vect{M}_A &= \sum\vect{r}_{AP_i}\times\vect{F}_i \\
	           &= \sum(\vect{r}_{AB} + \vect{r}_{BP_i})\times\vect{F}_i \\
	           &= \sum\vect{r}_{AB}\times\vect{F}_i + \vect{r}_{BP_i}\times\vect{F}_i \\
	           &= \vect{r}_{AB}\times\sum\vect{F}_i + \sum\vect{r}_{BP_i}\times\vect{F}_i \\
	           &= \vect{r}_{AB}\times\vect{F} + \vect{M}_{B}
\end{align*}

\paragraph{Kraftskruven}
Om man vill förenkla ett system mest möjligt, dekomponera $\vect{M}$ i dens komponenter parallelt och vinkelrät på $\vect{F}$. Ersätt den vinkelräta komponenten med ett krafpar ett avstånd $d = \frac{M_\perp}{F}$ från den förra punkten. Då blir resultatet en kraft och ett kraftmoment på samma linje, och detta kallas en kraftskruv.

\paragraph{Enkraftsresultant}
Om systemet kan reduceras till en kraftskruv utan kraftmoment, är detta systemets enkraftsresultant.

Om systemet har en enkraftsresultant, är detta ekvivalent med att det ursprungliga totala kraftmomentet och kraftsumman är ortogonala. Tillräckligheten kommer direkt från härledningen. Nödvendigheten visas om vi antar att det finns en enkraftsresultant i $B$, men $\vect{F}\cdot\vect{M}_A\neq 0$ i något $A$. Detta ger
\begin{align*}
	\vect{M}_A              &= \vect{M}_B + \vect{r}_{AB}\times\vect{F} = \vect{r}_{AB}\times\vect{F} \\
	\vect{F}\cdot\vect{M}_A &= \vect{F}\cdot(\vect{r}_{AB}\times\vect{F}) = 0.
\end{align*}
Detta ger en motsägelse, och ekvivalensen är bevisat.