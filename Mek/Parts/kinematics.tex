\section{Kinematik}

I kinematiken beskrivs rörelser utan att diskutera deras uppkomst. Här kommer vi diskutera kinematik för partiklar.

\paragraph{Essensiella storheter}
Ortsvektorn $\vect{r}$ beskriver positionen till partikeln. Dens derivata
\begin{align*}
	\vect{v} = \dv{\vect{r}}{t}
\end{align*}
kallas hastighet och dens andraderivata
\begin{align*}
	\vect{a} = \dv{\vect{v}}{t} = \dv[2]{\vect{r}}{t}
\end{align*}
kallas acceleration.

\paragraph{Cirkelrörelse}
För cirkelrörelse med konstant radius är hastigheten vinkelrät på ortsvektorn, och har storlek $v = R\dv{\theta}{t}$. Den tangentiala accelerationen har storlek $a_{\parallel} = \dv{v}{t}$ och den centripetala accelerationen, som är riktad in mot centrum av rörelsen, har storlek $a_{\perp} = \frac{v^2}{R}$.