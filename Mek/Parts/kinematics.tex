\section{Kinematik}

I kinematiken beskrivs rörelser utan att diskutera deras uppkomst. Här kommer vi diskutera kinematik för partiklar.

\paragraph{Essensiella storheter}
Ortsvektorn $\vect{r}$ beskriver positionen till partikeln. Dens derivata
\begin{align*}
	\vect{v} = \dv{\vect{r}}{t}
\end{align*}
kallas hastighet och dens andraderivata
\begin{align*}
	\vect{a} = \dv{\vect{v}}{t} = \dv[2]{\vect{r}}{t}
\end{align*}
kallas acceleration.

\paragraph{Naturliga komponenter}
En lämplig beskrivning av en partikels bana är att skriva den som beroende av sträckan $s$ partikeln har förflyttat sig. Med $\vect{r}(t) = \vect{r}(s(t))$ får vi
\begin{align*}
	\vect{v} = \dv{\vect{r}}{t} = \dv{\vect{r}}{s}\dv{s}{t}.
\end{align*}
Enligt definitionen är $\vect{e}_\text{t} = \dv{\vect{r}}{s}$ en enhetsvektor parallell med partikelns bana. Vi får även
\begin{align*}
	\vect{a} = \dv{\vect{v}}{t} = \dv{\vect{r}}{s}\dv[2]{s}{t} + \dv{\vect{e}_\text{t}}{t}\dv{s}{t} = \dv{\vect{v}}{t} = \dv{\vect{r}}{s}\dv[2]{s}{t} + \dv{\vect{e}_\text{t}}{s}\left(\dv{s}{t}\right)^2.
\end{align*}
Den nya vektorn $\dv{\vect{e}_\text{t}}{s}$ är av intresse. Vi har att
\begin{align*}
	\vect{e}_\text{t}\cdot\vect{e}_\text{t} = 0 \implies 2\vect{e}_\text{t}\cdot\dv{\vect{e}_\text{t}}{s} = 0,
\end{align*}
så denna vektorn är normal på banan. Dens belopp ges av $\abs{\dv{\vect{e}_\text{t}}{t}} = \frac{1}{\rho}$, där $\rho$ är kurvans krökningsradie i någon given punkt. Vi använder definitionen av farten och skriver accelerationen som
\begin{align*}
	\vect{a} = \dv{v}{t}\vect{e}_\text{t} + \frac{v^2}{\rho}\vect{e}_\text{n}.
\end{align*}
Vid att betrakta $\vect{v}\times\vect{a}$ och använda kedjeregeln kan banans krökningsradie skrivas som
\begin{align*}
	\rho = \frac{\abs{\dv{s}{u}}^3}{\abs{\dv{\vect{r}}{u}\times\dv[2]{\vect{r}}{u}}}
\end{align*}
där $u$ är någon godtycklig parameter.

\paragraph{Cirkelrörelse}
För cirkelrörelse med konstant radius är hastigheten vinkelrät på ortsvektorn, och har storlek $v = R\dv{\theta}{t}$. Den tangentiala accelerationen har storlek $a_{\parallel} = \dv{v}{t}$ och den centripetala accelerationen, som är riktad in mot centrum av rörelsen, har storlek $a_{\perp} = \frac{v^2}{R}$.

\paragraph{Cylinderkoordinater}
I cylinderkoordinater beskriver vi en partikels position med polära koordinater i $xy$-planet och en extra $z$-komponent. Vektorerna
\begin{align*}
	\vect{e}_{r} = 
	\left[\begin{array}{c}
		\cos{\phi} \\
		\sin{\phi} \\
		0
	\end{array}\right],
	\vect{e}_{\phi} = 
	\left[\begin{array}{c}
		-\sin{\phi} \\
		\cos{\phi} \\
		0
	\end{array}\right],
	\vect{e}_{z} = 
	\left[\begin{array}{c}
		0 \\
		0 \\
		0
	\end{array}\right]
\end{align*}
kommer användas för att beskriva partikelns rörelse. Deras derivator ges av
\begin{align*}
	&\dv{\vect{e}_{r}}{t} =
	\left[\begin{array}{c}
		-\dv{\theta}{t}\sin{\phi} \\
		\dv{\theta}{t}\cos{\phi} \\
		0
	\end{array}\right]
	= \dv{\theta}{t}\vect{e}_{\theta}, \\
	&\dv{\vect{e}_{\theta}}{t} =
	\left[\begin{array}{c}
		-\dv{\theta}{t}\cos{\phi} \\
		-\dv{\theta}{t}\sin{\phi} \\
		0
	\end{array}\right]
	= -\dv{\theta}{t}\vect{e}_{r}, \\
	&\dv{\vect{e}_{z}}{t} = \vect{0}.
\end{align*}

$\vect{e}_{r}$ pekar från origo mot partikelns position, $\vect{e}_{\theta}$ pekar tangentiellt på den cirkeln i $xy$-planet partikeln ligger på, pekande moturs, och $\vect{e}_{z}$ pekar rakt uppåt. Med detta kan vi skriva partikelns position som
\begin{align*}
	\vect{r} = r\vect{e}_{r} + z\vect{e}_{z}.
\end{align*}
Då ges hastigheten av
\begin{align*}
	\vect{v} = \dv{\vect{r}}{t} = \dv{r}{t}\vect{e}_{r} + r\dv{\phi}{t}\vect{e}_{\phi} + \dv{z}{t}\vect{e}_{z}
\end{align*}
och accelerationen ges av
\begin{align*}
	\vect{a} = \dv[2]{\vect{r}}{t} = \left(\dv[2]{r}{t} - r\left(\dv{\phi}{t}\right)^2\right)\vect{e}_{r} + \left(r\dv[2]{\phi}{t} + 2\dv{r}{t}\dv{\phi}{t}\right)\vect{e}_{\theta} + \dv[2]{z}{t}\vect{e}_{z}.
\end{align*}

\paragraph{Rörelsemängd}
Rörelsemängden för en partikel definieras som
\begin{align*}
	\vect{p} = m\vect{v}.
\end{align*}

\paragraph{Rörelsemängdsmoment}
Rörelsesmängdmsmomentet för en partikel med avseende på punkten $O$ definieras som
\begin{align*}
	\vb{H}_{O} = \vb{r}_{O}\times\vb{p}.
\end{align*}