\section{Kinematik}

I kinematiken beskrivs rörelser utan att diskutera deras uppkomst. Här kommer vi diskutera kinematik för partiklar.

\paragraph{Essensiella storheter}
Ortsvektorn $\vect{r}$ beskriver positionen till partikeln. Dens derivata
\begin{align*}
	\vect{v} = \dv{\vect{r}}{t}
\end{align*}
kallas hastighet och dens andraderivata
\begin{align*}
	\vect{a} = \dv{\vect{v}}{t} = \dv[2]{\vect{r}}{t}
\end{align*}
kallas acceleration.

\paragraph{Cirkelrörelse}
För cirkelrörelse med konstant radius är hastigheten vinkelrät på ortsvektorn, och har storlek $v = R\dv{\theta}{t}$. Den tangentiala accelerationen har storlek $a_{\parallel} = \dv{v}{t}$ och den centripetala accelerationen, som är riktad in mot centrum av rörelsen, har storlek $a_{\perp} = \frac{v^2}{R}$.

\paragraph{Naturliga komponenter}
En lämplig beskrivning av en partikels bana är att skriva den som beroende av sträckan partikeln har förflyttat sig. Med $\vect{r}(t) = \vect{r}(s(t))$ får vi
\begin{align*}
	\vect{v} = \dv{\vect{r}}{t} = \dv{\vect{r}}{s}\dv{s}{t}.
\end{align*}
Enligt definitionen är $\vect{e}_\text{t} = \dv{\vect{r}}{s}$ en enhetsvektor parallell med partikelns bana. Vi får även
\begin{align*}
	\vect{a} = \dv{\vect{v}}{t} = \dv{\vect{r}}{s}\dv[2]{s}{t} + \dv{\vect{e}_\text{t}}{t}\dv{s}{t} = \dv{\vect{v}}{t} = \dv{\vect{r}}{s}\dv[2]{s}{t} + \dv{\vect{e}_\text{t}}{s}\left(\dv{s}{t}\right)^2.
\end{align*}
Den nya vektorn $\dv{\vect{e}_\text{t}}{s}$ är av intresse. Vi har att
\begin{align*}
	&\vect{e}_\text{t}\cdot\vect{e}_\text{t} = 0 \\
	&2\vect{e}_\text{t}\cdot\dv{\vect{e}_\text{t}}{s} = 0,
\end{align*}
så denna vektorn är normal på banan. Dens belopp ges av $\abs{\dv{\vect{e}_\text{t}}{t}} = \frac{1}{\rho}$, där $\rho$ är kurvans krökningsradie i någon given punkt. Vi använder definitionen av farten och skriver accelerationen som
\begin{align*}
	\vect{a} = \dv{v}{t}\vect{e}_\text{t} + \frac{v^2}{\rho}\vect{e}_\text{n}.
\end{align*}
Vid att betrakta $\vect{v}\times\vect{a}$ och använda kedjeregeln kan banans krökningsradie skrivas som
\begin{align*}
	\rho = \frac{\abs{\dv{s}{u}}^3}{\abs{\dv{\vect{r}}{u}\times\dv[2]{\vect{r}}{u}}}
\end{align*}
där $u$ är någon godtycklig parameter.