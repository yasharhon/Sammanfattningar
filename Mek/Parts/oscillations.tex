\section{Svängningar}

\paragraph{Fri, odämpad svängning}
Även betecknad som den harmoniska oskillatorn är detta det enklaste systemet som beskriver svängningsfenomen.

Grundideen är att en partikel utsätts för en kraft proportionell till dens förskjutning från någon punkt och motsatt riktad förskjutningen. Formulerat med Newtons andra lag blir detta
\begin{align*}
	m\dv[2]{x}{t} = -kx.
\end{align*}
Vi definierar den naturliga frekvensen $\omega_n$ enligt $\omega_n^2 = \sqrt{\frac{k}{m}}$ och får
\begin{align*}
	\dv[2]{x}{t} + \omega_n^2x = 0.
\end{align*}
Denna ekvationen har allmän lösning
\begin{align*}
	x = A\sin{\omega_nt} + B\cos{\omega_nt}.
\end{align*}
Vi skriver den alternativt som
\begin{align*}
	x = C\sin{(\omega_nt + \alpha)},
\end{align*}
där det gäller att
\begin{align*}
	A = C\cos{\alpha},\ B = C\sin{\alpha}.
\end{align*}