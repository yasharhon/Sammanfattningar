\section{Svängningar}

\paragraph{Fri, odämpad svängning}
Även betecknad som den harmoniska oskillatorn är detta det enklaste systemet som beskriver svängningsfenomen.

Grundideen är att en partikel utsätts för en kraft proportionell till dens förskjutning från någon punkt och motsatt riktad förskjutningen. Projicerat ned på en axel ger Newtons andra lag
\begin{align*}
	m\dv[2]{x}{t} = -kx.
\end{align*}
Vi definierar den naturliga frekvensen $\omega_n$ enligt $\omega_n^2 = \sqrt{\frac{k}{m}}$ och får
\begin{align*}
	\dv[2]{x}{t} + \omega_n^2x = 0.
\end{align*}
Denna ekvationen har allmän lösning
\begin{align*}
	x = A\sin{\omega_nt} + B\cos{\omega_nt}.
\end{align*}
Vi skriver den alternativt som
\begin{align*}
	x = C\sin{(\omega_nt + \alpha)},
\end{align*}
där det gäller att
\begin{align*}
	A = C\cos{\alpha},\ B = C\sin{\alpha}.
\end{align*}

\paragraph{Fri, dämpad svängning}
Vi vill nu addera en dämpande kraft $F = -c\dv{x}{t}$ till vår modell. Newtons andra lag blir
\begin{align*}
	\dv[2]{x}{t} + \frac{c}{m}\dv{x}{t} + \frac{k}{m}x = 0.
\end{align*}
Vi definierar dämpningskoefficienten $\zeta$ genom $2\omega_n\zeta = \frac{c}{m}$ och får
\begin{align*}
	\dv[2]{x}{t} + 2\omega_n\zeta\dv{x}{t} + \omega_n^2x = 0.
\end{align*}
Formen på lösningarna beror på hur stor $\zeta$ är.

\subparagraph{Stark dämpning}
Man har stark dämpning när $\zeta > 1$, och den allmänna lösningen är då
\begin{align*}
	x = Ae^{-\omega_n\left(\zeta + \sqrt{\zeta^2 - 1}\right)t} + Be^{-\omega_n\left(\zeta - \sqrt{\zeta^2 - 1}\right)t}.
\end{align*}

\subparagraph{Kritisk dämpning}
Man har kritisk dämpning när $\zeta = 1$, och den allmänna lösningen är då
\begin{align*}
	x = (At + B)e^{-\omega_nt}.
\end{align*}

\subparagraph{Svag dämpning}
Man har svag dämpning när $\zeta < 1$, och den allmänna lösningen är då
\begin{align*}
	x = \left(Ae^{i\omega_n\sqrt{1 - \zeta^2}t} + Be^{-i\omega_n\sqrt{1 - \zeta^2}t}\right)e^{-\omega_n\zeta t}.
\end{align*}
Vi definierar $\omega_d = \omega_n\sqrt{1 - \zeta^2}$ och får
\begin{align*}
	x = \left(Ae^{i\omega_dt} + Be^{-i\omega_dt}\right)e^{-\omega_n\zeta t}.
\end{align*}
Vi hoppas verkligen att detta kan skrivas som $x = Ce^{-\omega_n\zeta t}\sin{\left(\omega_dt + \alpha\right)}$. För att visa detta, skriv först om $x$ till
\begin{align*}
	x = \left((A + B)\cos{\omega_dt} + (A - B)i\sin{i\omega_dt}\right)e^{-\omega_n\zeta t}.
\end{align*}
Då har man som innan
\begin{align*}
	A = C\cos{\alpha},\ B = C\sin{\alpha}.
\end{align*}

Om man betraktar maxima nummer $n$ och $n - 1$ får man
\begin{align*}
	\frac{x_{n}}{x_{n + 1}} = \frac{Ce^{-\zeta\omega_nt}\sin{\omega_dt + \alpha}}{Ce^{-\zeta\omega_n(t + \tau_d}\sin{\omega_d(t + \tau_d) + \alpha}}.
\end{align*}
Vi använder rörelsens periodicitet och får
\begin{align*}
	\frac{x_{n}}{x_{n + 1}} = e^{\zeta\omega_n\tau_{d}}.
\end{align*}
Vidare får man
\begin{align*}
	\delta = \ln{\frac{x_n}{x_{n + 1}}} = \zeta\omega_n\frac{2\pi}{\omega_d} = \zeta\omega_n\frac{2\pi}{\omega_n\sqrt{1 - \zeta^2}}.
\end{align*}
Vi kan lösa detta och få
\begin{align*}
	\zeta = \frac{\delta}{\sqrt{\delta^2 + 4\pi^2}}.
\end{align*}

\paragraph{Påtvungnen odämpat svänging}
Vi vill nu undersöka odämpat svängning under påverkan av en harmonisk kraft $F = F_0\sin{\omega t}$. Newtons andra lag ger
\begin{align}
	m\dv[2]{x}{t} = -kx + F_0\sin{\omega t}.
\end{align}
Vi förenklar och får
\begin{align*}
	\dv[2]{x}{t} + \omega_n^2 = \frac{F_0}{m}\sin{\omega t},
\end{align*}
med samma definition av $\omega_n$. Denna har lösning $x = x_h + x_p$, alltså en kombination av en homogen och en partikulär lösning. Den homogena lösningen är redan känd. Vi gör ansatsen $x_p = X\sin{\omega t}$. Insatt i differentialekvationen får man att ansatsen är en lösning om
\begin{align*}
	X = \frac{\frac{F_0}{k}}{1 - \left(\frac{\omega}{\omega_n}\right)^2}.
\end{align*}
Vi definierar förstorningsfaktorn $M$ som $M = \frac{X}{X_0}$, där $X_0$ är amplituden motsvarande ingen påtvingningskraft, alltså $\omega = 0$. I detta enkla fallet får man
\begin{align*}
	M = \frac{1}{1 - \left(\frac{\omega}{\omega_n}\right)^2}.
\end{align*}

\paragraph{Påtvungen dämpat svängning}
Vi vill nu undersöka dämpat svängning under påverkan av en dämpningskraft $F = -cdv{x}{t}$ och en harmonisk kraft $F = F_0\sin{\omega t}$. Newtons andra lag ger
\begin{align}
	m\dv[2]{x}{t} = -kx - c\dv{x}{t} + F_0\sin{\omega t}.
\end{align}
Vi förenklar och får
\begin{align*}
	\dv[2]{x}{t} + 2\zeta\omega_n\dv{x}{t} + \omega_n^2 = \frac{F_0}{m}\sin{\omega t},
\end{align*}
med samma definition av $\omega_n$ och $\zeta$. Denna har lösning $x = x_h + x_p$, alltså en kombination av en homogen och en partikulär lösning.

Den homogena lösningen är redan känd, och har den trevliga egenskapen att den försvinner över tid. Vi refererar till denna som den transienta lösningen.

Vi gör ansatsen $x_p = X\sin{(\omega t - \alpha)} = X(\sin{\omega t}\cos{\alpha} - \cos{\omega t}\sin{\alpha})$. Insatt i differentialekvationen får man att ansatsen är en lösning om
\begin{align*}
	X\left((\omega_n^2 - \omega^2)\cos{\alpha} + 2\zeta\omega_n\omega\sin{\alpha}\right) &= \frac{F_0}{m}, \\
	X\left(-(\omega_n^2 - \omega^2)\sin{\alpha} + 2\zeta\omega_n\omega\cos{\alpha}\right) &= 0.
\end{align*}
Detta ger
\begin{align*}
	\tan{\alpha} &= \frac{2\zeta\frac{\omega}{\omega_n}}{1 - \left(\frac{\omega}{\omega_n}\right)^2}, \\
	X            &= \frac{\frac{F_0}{k}}{\sqrt{\left(1 - \left(\frac{\omega}{\omega_n}\right)^2\right)^2 + 4\zeta^2\left(\frac{\omega}{\omega_n}\right)^2}}.
\end{align*}
Förstorningsfaktorn blir
\begin{align*}
	M = \frac{1}{\sqrt{\left(1 - \left(\frac{\omega}{\omega_n}\right)^2\right)^2 + 4\zeta^2\left(\frac{\omega}{\omega_n}\right)^2}}.
\end{align*}