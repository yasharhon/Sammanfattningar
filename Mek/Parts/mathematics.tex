\section{Nyttig matematik}

\paragraph{Pappus 1:a sats}
Låt $C$ vara en plan kurva som ej skär $x$-axeln, och rotera denna kring $x$-axeln. Arean som då alstras ges av produktet av kurvans längd och sträckan som kurvans centroid har förflyttat sig under rotationen.

\subparagraph{Bevis}
Om kurvan roteras en vinkel $\theta\in [0, 2\pi]$ ges arean av
\begin{align*}
	A = \int\limits_{C}\theta y\dd{s}.
\end{align*}
$y$-koordinaten till kurvans centroid ges av
\begin{align*}
	y_{G} = \frac{1}{L}\int\limits_{C}y\dd{s}
\end{align*}
där $L$ är kurvans längd. Centroiden kommer förflytta sig en sträcka $\theta y_{G}$ under rotationen, och vi ser då att
\begin{align*}
	\theta y_{G}L = \theta\int\limits_{C}y\dd{s},
\end{align*}
vilket skulle visas.

\paragraph{Pappus 1:a sats}
Låt $S$ vara en yta i planet som ej skär $x$-axeln, och rotera denna kring $x$-axeln. Volymen som då alstras ges av produktet av ytans area och sträckan som ytans centroid har förflyttat sig under rotationen.

\subparagraph{Bevis}
Om ytan roteras en vinkel $\theta\in [0, 2\pi]$ ges arean av
\begin{align*}
	V = \int\limits_{S}\theta y\dd{A}.
\end{align*}
$y$-koordinaten till kurvans centroid ges av
\begin{align*}
	y_{G} = \frac{1}{A}\int\limits_{S}y\dd{A}
\end{align*}
där $A$ är ytans area. Centroiden kommer förflytta sig en sträcka $\theta y_{G}$ under rotationen, och vi ser då att
\begin{align*}
	\theta y_{G}A = \theta\int\limits_{S}y\dd{A},
\end{align*}
vilket skulle visas.