\section{Jämvikt}
Jämviktsläget är ett läge som är konstant i tiden relativt någon referensram. Två nödvändiga jämviktsvillkor är att
\begin{align*}
	\sum\vect{F} = 0, \sum\vect{M}_A = 0\ \forall\ A.
\end{align*}
Villkoret är även tillräckligt om det gäller för alla delsystem av det totala systemet.

\paragraph{Ekvivalenta jämviktsekvationer i 2D}
I 2D ger jämviktsekvationerna
\begin{align*}
	F_x = 0, F_y = 0, M_A = 0.
\end{align*}
Vid att translatera $x$-komponenten av $\vect{F}$ till punkten $B$ får vi
\begin{align*}
	\vect{M}_{B} = \vect{M}_A + \vect{r}_{AB}\times\vect{F} = 0,
\end{align*}
så dessa jämviktsekvationerna implicerar det alternativa jämviktsvillkoret
\begin{align*}
	F_x = 0, M_A = 0, M_B = 0
\end{align*}
eller
\begin{align*}
	M_A = 0, M_B = 0, M_C = 0.
\end{align*}
Alternativt, givet
\begin{align*}
	F_x = 0, M_A = 0, M_B = 0
\end{align*}
har man
\begin{align*}
	\vect{M}_A = \vect{M}_B + \vect{r}_{BA}\times\vect{F}.
\end{align*}
Om $\vect{r}_{AB}$ ej är parallell på $y$-axeln, ger detta $\vect{F} = \vect{0}$. Den sista implikationen visas på motsvarande sätt, och detta bevisar ekvivalensen.

\paragraph{Statisk obestämthet}
Jämviktsvillkoren ger ett visst antal oberoende ekvationer som är uppfyllda vid jämvikt. Om kraftsystemet involverar flera okända än antalet ekvationer, kan inte krafterna bestämmas med stelkroppsmekanik, och systemet är statiskt obestämt.