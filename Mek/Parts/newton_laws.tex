\section{Newtons lagar}

\paragraph{Första lagen}
Att kraftsumman på en partikel är lika med noll är ekvivalent med att partikeln har konstant fart.

\paragraph{Andra lagen}
\begin{align*}
	\vect{F} = \dv{\vect{p}}{t}
\end{align*}

\paragraph{Tredje lagen}
Om en kropp utövar en kraft på en annan, komemr denna utöva en lika stor och motsatt riktad kraft på den första kroppen.

\paragraph{Galileitransformationer}
Vi söker ett samband mellan två referensramer. Låt en ram $S$ vara intertial och låt $S'$ röra sig relativt $S$ så att
\begin{itemize}
	\item hastigheten till origo i $S'$, betecknad $\vb{v}_{O'}$, är konstant.
	\item $S'$ ej roterar relativt $S$.
\end{itemize}

Enligt definitionen har man
\begin{align*}
	\vb{r} = \vb{r}_{O'} + \vb{r}'.
\end{align*}
Derivation med avseende på tid ger
\begin{align*}
	\vb{v} &= \vb{v}_{O'} + \dv{t}\left(x'\vb{e}_{x'} + y'\vb{e}_{y'} + z'\vb{e}_{z'}\right) \\
	       &= \vb{v}_{O'} + \dv{x'}{t}\vb{e}_{x'} + \dv{y'}{t}\vb{e}_{y'} + \dv{z'}{t}\vb{e}_{z'} \\
	       &= \vb{v}_{O'} + \vb{v}'.
\end{align*}
Derivation en gång till ger
\begin{align*}
	\vb{a} = \vb{a}'.
\end{align*}

Konsekvensen är att Newtons andra lag blir på exakt samma sätt i $S'$ som i $S$.