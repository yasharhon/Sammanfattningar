\section{Energi och arbete}

\paragraph{Arbete och effekt}
Arbetet $U$ definieras som
\begin{align*}
	\dd{U} = \vb{F}\cdot\dd{\vb{r}}.
\end{align*}
Effekten $P$ definieras som
\begin{align*}
	\dd{P} = \vb{F}\cdot\dd{\vb{v}}.
\end{align*}
Detta ger
\begin{align*}
	\dd{U} = \vb{F}\cdot\dd{\vb{r}} = \vb{F}\cdot\vb{v}\dd{t} = P\dd{t}
\end{align*}
och därmed
\begin{align*}
	P = \dv{U}{t}.
\end{align*}

\paragraph{Kinetisk energi}
Kinetisk energi definieras som
\begin{align*}
	T = \frac{1}{2}mv^2.
\end{align*}
Om vi jämför detta med definitionen av arbete får man
\begin{align*}
	U = \int\vb{F}\cdot\dd{\vb{r}} = \int m\vb{a}\cdot\vb{v}\dd{t}.
\end{align*}
Vi använder uttrycket för hastighet och acceleration i naturliga komponenter och får
\begin{align*}
	\int m\vb{a}\cdot\vb{v}\dd{t} = \int mv\dv{v}{t}\dd{t} = \int mv\dd{v} = \frac{1}{2}mv_2^2 - \frac{1}{2}mv_1^2 = \Delta T.
\end{align*}