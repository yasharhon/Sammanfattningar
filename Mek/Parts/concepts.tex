\section{Fundamentala koncepter}

\paragraph{Krafter}
En kraft $\vect{F}$ beskrivs av en vektor med belopp och rikting, samt en angrepspunkt.

\paragraph{Kraftmoment}
En kraft kan ha en viss vridningsförmåga med avseende på en punkt. Detta är kraftens kraftmoment. Dens storhet ges av
\begin{align*}
	\vect{M}_O = \vect{r}\times\vect{F},
\end{align*}
där $O$ är punkten vi tänker oss att kraften vrider kring, $\vect{r}$ är vektorn från $O$ till $\vect{F}$:s angrepspunkt och $\vect{F}$ är själva kraften.

Riktingen till kraftmomentet anger den positiva rotationsriktningen. Vad betyder detta? Jo, låt en linje gå genom $O$ och parallellt med $\vect{M}$. Då skapar $\vect{M}$ en vridning mot klockan kring denna linjen.

Kraftmomentet ändras inte av att kraften förskjutas längs med dens verkningslinje. Detta ser man vid att låta den angripa i två punkter $A, B$ på verkningslinjen.
\begin{align*}
	\vect{M}_O  &= \vect{r}_{OA}\times\vect{F} \\
	\vect{M}'_O &= \vect{r}_{OB}\times\vect{F} \\
	            &= (\vect{r}_{OA} + \vect{r}_{AB})\times\vect{F} \\
	            &= \vect{r}_{OA}\times\vect{F} + \vect{r}_{AB}\times\vect{F}\\
	            &= \vect{M},
\end{align*}
då den andra vektoren är parallell med $\vect{F}$.

Detta kan utvidgas till kraftmomentet kring en axel vid att välja en punkt $P$ på axeln och beräkna kraftmomentet med avseende på denna punkten. Projektionen på axeln av detta kraftmomentet är oberoende av valet av $P$. Detta ser man vid att välja en ny punkt $Q$ och beräkna
\begin{align*}
	\vect{M}_P &= \vect{r}_{PA}\times\vect{F} \\
	\vect{M}_Q &= \vect{r}_{QA}\times\vect{F} \\
	           &= (\vect{r}_{QP} + \vect{r}_{PA})\times\vect{F} \\
	           &= \vect{r}_{QP}\times\vect{F} + \vect{r}_{PA}\times\vect{F}
\end{align*}
Man projicerar sen på axeln.
\begin{align*}
	\vect{M}_Q\cdot\vect{e}_\lambda &= \vect{r}_{QP}\times\vect{F}\cdot\vect{e}_\lambda + \vect{r}_{PA}\times\vect{F}\cdot\vect{e}_\lambda \\
	                                &= \vect{r}_{QP}\times\vect{F}\cdot\vect{e}_\lambda \\
	                                &= \vect{M}_Q\cdot\vect{e}_\lambda,
\end{align*}
där $\vect{e}_\lambda$ är parallell med axeln. Detta är eftersom $\vect{r}_{QP}$ är parallell med $\vect{e}_\lambda$, och kryssprodukten vi beräknar då måste vara normal på båda dessa.