\documentclass[a4paper, 11pt]{article}
\usepackage[utf8]{inputenc}
\usepackage[swedish]{babel}
\usepackage{amsmath}
\usepackage{amssymb}
\usepackage[arrowdel]{physics}
\usepackage{hyperref}
\usepackage{ifthen}
\usepackage{pgffor}
\usepackage[margin=0.5in]{geometry}

\newcommand{\N}{\mathbb{N}}
\newcommand{\Z}{\mathbb{Z}}
\newcommand{\Q}{\mathbb{Q}}
\newcommand{\R}{\mathbb{R}}
\newcommand{\C}{\mathbb{C}}
\newcommand{\set}[2]{
	\left\{#1
	\ifx&#2&
	\else
   	: #2
	\fi
	\right\}}
\newcommand{\comp}[1]
{
	\foreach \i [count=\xi] in #1
	{
		\ifnum\xi>1
			\circ
		\fi
		\i
	}
}
\newcommand{\Span}[1]{\text{Span}\left(#1\right)}
\renewcommand{\det}[1]{\text{det}(#1)}
\renewcommand{\ker}[1]{\text{ker}(#1)}
\renewcommand{\dim}[1]{\text{dim}(#1)}
\renewcommand{\Im}[1]{\text{Im}(#1)}
\newcommand{\proj}[2]{\text{proj}_{#1}(#2)}
\newcommand{\inprod}[2]{\expval{#1, #2}}
\newcommand{\proof}{\subparagraph{Bevis}}
\newcommand{\dirsum}[3]{\bigoplus\limits_{#1}^{#2}#3}

\title{Sammanfattning av SF1681 Linjär algebra, fortsättningskurs}
\author{Yashar Honarmandi \\ yasharh@kth.se}
\date{\today}

\begin{document}

\maketitle

\begin{abstract}
	Detta ær en sammanfattning av kursen SH1014 Modern fysik.
\end{abstract}

\pagenumbering{roman}
\thispagestyle{empty}

\newpage

\tableofcontents

\newpage

\pagenumbering{arabic}

%\twocolumn

\section{Vektorrum}

\subsection{Definitioner}

\paragraph{Kroppar}
En kropp är något som har definierat multiplikation och addition, och som fungerar som (är isomorft med) $\R, \C$ osv.

\paragraph{Vektorrum}
Ett vektorrum är en mängd med en operation som gör $V$ till en abelsk grupp och för vilken det finns en kropp $k$ med skalärer och en operation med skalären som uppfyller
\begin{itemize}
	\item $c(x + y) = cx + cy,~c\in\R,~x,y\in V$.
	\item $(c + d)x = cx + dx,~c,d\in\R$.
	\item $c(dx) = (cd)x$.
	\item $1x = x$.
\end{itemize}

\paragraph{Delrum}
En delmängd $V$ av ett vektorrum är ett delrum om
\begin{itemize}
	\item $0\in V$, där $0$ är nollelementet.
	\item $x, y\in V\implies x + y\in V$.
	\item $cx\in V$ för alla $c\in\R$.
\end{itemize}

\paragraph{Inre direkt summa}
Vi definierar den inre direkta summan
\begin{align*}
	\dirsum{i = 0}{\infty}{V_{i}} = \left\{\sum\limits_{i = 0}^{\infty}a_{i}, a_{i}\in V_{i}\right\}.
\end{align*}

\paragraph{Yttre direkt summa}
Den yttre direkte summan av två vektorrum definieras som
\begin{align*}
	V\oplus W = \left\{(x, y), x\in V, y\in W\right\}.
\end{align*}

\paragraph{Kvotrum}
Om $W\subseteq V$ är delrum, kan vi bilda
\begin{align*}
	\frac{V}{W} = \left\{x + W, x\in V\right\},
\end{align*}
där vi har användt summan
\begin{align*}
	x + W = \left\{x + y, y\in W\right\}.
\end{align*}
Dessa kallas för sidoklasser.

\paragraph{Operationer på sidoklasser}
Till sidoklasser hör operationer
\begin{align*}
	(x + W) + (y + W) &= x + y + W, \\
	a(x + W)          &= (ax) + W.
\end{align*}

\paragraph{Linjärt oberoende mängder}
$S$ är en linjärt oberoende mängd om
\begin{align*}
	\sum a_{i}x_{i} = 0\implies a_{i} = 0 \forall i,
\end{align*}
där alla $x_{i}$ är elementer i $S$.

\paragraph{Linjärt hölje}
Det linjära höljet $\Span{S}$ av mängden $S$ är
\begin{itemize}
	\item mängden av alla linjärkombinationer av vektorer i $S$.
	\item det minsta delrummet som innehåller $S$.
	\item $\bigcap\limits_{S\subset W}W$.
	\item $\sum\limits_{x\in S} \Span{x}$.
\end{itemize}

\paragraph{Bas}
En bas $B$ för vektorrummet $W$ är en linjärt oberoende mängd så att $V = \Span{B}$, dvs. att alla vektorer i $V$ är linjärkombinationer av vektorer i $B$ på ett unikt sätt.

\paragraph{Duala rum}
För ett vektorrum $V$ över kropp $k$ är duala rummet $V^{*}$ mängden av alla linjära former på $V$, dvs. alla linjära avbildningar $V\to k$.

\paragraph{Dual bas}
Givet en bas $\{e_{i}\}_{i\in I}$ för $V$, väljer vi en bas $\{e_{i}^{*}\}_{i\in I}$ fför $V^{*}$ som uppfyller
\begin{align*}
	e_{i}^{*}(e_{j}) = \delta_{ij}.
\end{align*}

\subsection{Satser}

\paragraph{Operationer på sidoklasser}
Operationer på sidoklasser är väldefinierade.

\proof

\section{Linjära avbildningar}

\subsection{Definitioner}

\paragraph{Linjära avbildningar}
En avbildning $T(\vect{x})$ är linjär om
\begin{itemize}
	\item $T(\vect{x} + \vect{y}) = T(\vect{x}) + T(\vect{y})$ och
	\item $T(c\vect{x}) = cT(\vect{x})$ för något $c$.
\end{itemize}

\paragraph{Bilden till en avbildning}
För en avbildning $T(\vect{x})$ definierar man bildet till $T$ som
\begin{align*}
	Im(T) = \{\vect{y} : \vect{y} = T(\vect{x})\}.
\end{align*}

\paragraph{Nollrummet till en avbildning}
För en avbildning $T(\vect{x})$ definierar man nollrummet till $T$ som
\begin{align*}
	\text{Null}(T) = \{\vect{x} : T(\vect{x})= \vect{0}\}.
\end{align*}

\subsection{Satser}

\paragraph{Avildningar och enhetsvektorer}
För en avbildning $T(\vect{x}): \R^n\to\R^m$ har man att
\begin{align*}
	T(\vect{x}) = \sum\limits_{i = 1}^{n} x_iT(\vect{e}_i)
\end{align*}
var $x_i$ är komponenterna av $\vect{x}$.

\paragraph{Bevis}
Borde gå.

\paragraph{Avbildningar och matriser}
För en avbildning $T(\vect{x}): \R^n\to\R^m$ kan man definiera matrisen
\begin{align*}
	A = \left[ T(\vect{e}_1) ~ T(\vect{e}_2) ~ \dots ~ T(\vect{e}_n) \right]
\end{align*}
var $\vect{e}_i$ är enhetsvektorerna i $\R^n$. Då kan avbildningen skrivas som
\begin{align*}
	T(\vect{x}) = A\vect{x}.
\end{align*}

\subparagraph{Bevis}
Inte svårt alls.

\section{Egenvärden och olika polynom}

\subsection{Definitioner}

\paragraph{Egenvektorer}
$x$ är en egenvektor till $L$ om det finns ett $\lambda\in k$ så att
\begin{align*}
	Lx = \lambda x.
\end{align*}
$\lambda$ kallas det motsvarande egenvärdet.

\paragraph{Karakteristiskt polynom}
Om $V$ är ändligdimensionellt ges det karakteristiska polynomet av
\begin{align*}
	p_{L}(x) = \det{xI - L}\in k[x],
\end{align*}
där $I$ är identitetsavbildningen.

\paragraph{Minimalpolynom}
Om $A$ är en matris, är minimalpolynomet $q_{A}(x)\in k[x]$ det moniska polynomet av lägst grad så att $q_{A}(A) = 0$.

\paragraph{Diagonaliserbarhet}
En operator är diagonaliserbar om det finns en bas så att operatorns matris i den basen är diagonal.

\paragraph{Samtidig diagonaliserbarhet}
Två operatorer $L_{1}$ och $L_{2}$ är samtidigt diagonaliserbara om båda är diagonaliserbara och det finns en gemensam bas av egenvektorer.

\subsection{Satser}

\paragraph{Karakteristiska polynom och egenvärden}
Om $lambda$ är ett egenvärde till $L$ så är $p_{L}(\lambda) = 0$.

\proof
Ez

\paragraph{Existens av minimalpolynom}
Om $V$ är ändligdimensionellt, har $L$ ett karakteristiskt polynom.

\proof
Betrakta matrisen $A$ för $L$ i någon bas. Det gäller att mängden $\{A^{0}, A^{1}, \dots, A^{n^2}\}$ är linjärt beroende, och därmed finns det koefficienter $a_{0}, \dots, a_{n}$ så att
\begin{align*}
	\sum a_{i}A^{i} = 0.
\end{align*}

\paragraph{Cayley-Hamiltons sats}
$p_{L}(L) = 0$.

\proof
Om matrisen för $L$ är diagonal så är det uppenbart, ty
\begin{align*}
	A^{i} =
	\left[\begin{array}{ccc}
		\lambda_{1} & \dots  & 0 \\
		\vdots      & \ddots & \vdots \\
		0           & \dots  & \lambda_{n}
	\end{array}\right]
	\implies
	p_{A}(A) =
	\left[\begin{array}{ccc}
		p_{A}(\lambda_{1}) & \dots  & 0 \\
		\vdots             & \ddots & \vdots \\
		0                  & \dots  & p_{A}(\lambda_{n})
	\end{array}\right].
\end{align*}
I övrigt oklart.

\subparagraph{Korrolar}
$q_{L}$ är en faktor i $p_{L}$.

\paragraph{Multipliciteter och diagonaliserbarhet}
Om $L$ är diagonaliserbar, är den geometriska multipliciteten lika med den algebraiska multipliciteten för alla $L$:s egenvärden.

\proof

\paragraph{Konjugerade matriser}
Alla matriser är konjugerade med en övertriangulär matris med matrisens egenvärden på diagonalen.

\paragraph{Samtidig diagonaliserbarhet och kommutativitet}
Låt $V$ vara ett ändligdimensionellt vektorrum och $L_{1}, L_{2}$ två operatorer på detta. Då går det att diagonalisera $L_{1}$ och $L_{2}$ om de kommuterar.

\proof

\paragraph{Kommutativitet och egenrum}
Låt $L_{1}$ och $L_{2}$ kommutera och $E_{1}$ vara egenrum till $L_{1}$. Då är $L_{2}(E_{1})\subset E_{1}$.

\proof

\paragraph{Nilpotens och blockdiagonalitet}
Om $L$ är nilpotent finns det en bas för $V$ så att matrisen för $L$ blir blockdiagonal, där varje block är på formen
\begin{align*}
	\mqty[
		0      & 1      & 0      & \dots  & 0 \\
		0      & 0      & 1      & \dots  & \vdots \\
		0      & 0      & 0      & \ddots & \vdots \\
		\vdots & \vdots & \vdots & \ddots & \vdots \\
		0      & \dots  & \dots  & \dots  & 0
	].
\end{align*}

\proof
Här kommer endast en bevisidé presenteras.

Det finns ett $s$ så att $L^{s} = 0$ och $L^{s - 1} \neq 0$. Vi väljer då ett delrum $W_{s}$ så att
\begin{align*}
	V = \ker{L^{s}} = W_{s} \oplus \ker{L^{s - 1}}.
\end{align*}
Vi väljer vidare $W_{s - 1}\subseteq\ker{L^{s - 1}}$ så att
\begin{align*}
	\ker{L^{s - 1}} = W_{s - 1} \oplus L(W_{s}) \oplus \ker{L^{s - 2}}.
\end{align*}
Detta går eftersom $L(W_{s}) \subseteq \ker{L^{s - 2}}$ och $L(W_{s}) \cap \ker{L^{s - 2}} = \{0\}$. Upprepa prosedyren tills man får
\begin{align*}
	\ker{L} = \dirsum{i = 0}{s - 1}{L^{i}(W_{i + 1})}.
\end{align*}
Välj nu baser för alla $W_{i}$ och bilderna av alla potenser av $L$. Dissa bildar en bas för $V$. Med en lämplig ordning på basen fås $\dim{W_{i}}$ block på formen ovan i matrisen, varje med storlek $i\times i$.

\paragraph{Jordans normalform}
Om en operator har karakteristiskt polynom
\begin{align*}
	p_{L}(x) = \prod (x - \lambda_{i}),
\end{align*}
finns det en bas så att matrisen för $L$ är på formen
\begin{align*}
	\left[\begin{array}{cccc}
		\Lambda_{1} & 0           & \dots  & 0 \\
		0           & \Lambda_{2} & \dots  & \vdots \\
		\vdots      & \vdots      & \ddots & \vdots \\
		0           & \dots       & \dots  & \Lambda_{i}
	\end{array}\right],
\end{align*}
med 
\begin{align*}
	\Lambda_{i} =
	\left[\begin{array}{cccc}
		\lambda_{i} & 1           & \dots       & 0 \\
		0           & \lambda_{i} & \dots       & \vdots \\
		\vdots      & \dots       & \ddots      & 1 \\
		0           & \dots       & 0           & \lambda_{i}
	\end{array}\right].
\end{align*}
En sådan matris är på Jordans normalform. Vi noterar att $\Lambda_{i} = \lambda_{i}I + N$, där $N$ är nilpotent.

\proof

\section{Inreprodukt}

\subsection{Definitioner}

\paragraph{Inreprodukt över $\R$}
En inreprodukt $\inprod{x}{y}$ på ett vektorrum $V$ över $\R$ är en avbildning $V\times V\to \R$ som är
\begin{itemize}
	\item bilinjär, dvs.
	\begin{itemize}
		\item $\inprod{x + y}{z} = \inprod{x}{z} + \inprod{y}{z}$.
		\item $\inprod{ax}{y} = a\inprod{x}{y}$.
		\item $\inprod{x}{y + z} = \inprod{x}{y} + \inprod{x}{z}$.
		\item $\inprod{x}{ay} = a\inprod{x}{y}$.
	\end{itemize}
	\item symmetrisk, dvs. $\inprod{x}{y} = \inprod{y}{x}$.
	\item positivt definit, dvs. $\inprod{x}{x} > 0$ om $x \neq 0$.
\end{itemize}

\paragraph{Inreprodukt över $\C$}
En inreprodukt $\inprod{x}{y}$ på ett vektorrum $V$ över $\C$ är en avbildning $V\times V\to \C$ som är
\begin{itemize}
	\item seskvilinjär, dvs. bilinjär, men $\inprod{ax}{y} = \cc{a}\inprod{x}{y}$.
	\item konjugatsymmetrisk, dvs. $\inprod{x}{y} = \cc{\inprod{y}{x}}$.
	\item positivt definit, dvs. $\inprod{x}{x} > 0$ om $x \neq 0$. Notera att detta och konjugatsymmetrin implicerar att $\inprod{x}{x}$ har ingen imaginärdel.
\end{itemize}

\paragraph{Inreprodukt från matris}
Vi kan få en inreprodukt i $\C^{n}$ från en matris genom
\begin{align*}
	\inprod{x}{y} = \sum a_{ij}\cc{x_{i}}{y_{j}} = (\cc{x})^{T}Ay.
\end{align*}

Om detta skall uppfylla konjugatsymmetri, ger det
\begin{align*}
	(\cc{x})^{T}Ay = \cc{(\cc{y})^{T}Ax} = y^{T}\cc{A}\cc{x}.
\end{align*}

Transponering av högersidan ger
\begin{align*}
	(\cc{x})^{T}Ay = (\cc{x})^{T}(\cc{A})^{T}y,
\end{align*}
och därmed uppfylls konjugatsymmetrin om
\begin{align*}
	A = (\cc{A})^{T}.
\end{align*}
Om matrisen uppfyller detta, säjs den vara konjugatsymmetrisk eller Hermitesk.

\paragraph{Norm}
Normen eller längden av en vektor definieras som
\begin{align*}
	\abs{x} = \sqrt{\inprod{x}{x}}.
\end{align*}

\paragraph{Vinkel}
Vinkeln $\theta$ mellan två vektorer definieras som
\begin{align*}
	\cos{\theta} = \frac{\inprod{x}{x}}{\abs{x}\abs{y}}.
\end{align*}

\paragraph{Ortogonalitet}
$x$ och $y$ är ortogonala om
\begin{align*}
	\inprod{x}{y} = 0.
\end{align*}

\paragraph{Ortogonalt komplement}
Om $W\subseteq V$ är ett delrum så finns det ett ortogonalt komplement
\begin{align*}
	W^{\perp} = \left\{x\in V: \inprod{x}{y} = 0\forall y\in W\right\} \subseteq V.
\end{align*}

\paragraph{Projektion}
Låt $V$ vara ett delrum med bas bas $B = \{e_{i}\}_{i= 0}^{n}$. Då definieras projektionen som
\begin{align*}
	\proj{V}{x} = \sum\limits_{i = 0}^{n}\frac{\inprod{x_{i}}{e_{j}}}{\norm{e_{i}}^{2}}e_{i}.
\end{align*}

\paragraph{Adjungerade operatorer}
På ett inreproduktrum $V$ är $\adj{L}$ den adjungerade operatorn till $L$ om
\begin{align*}
	\inprod{\adj{L}(x)}{y} = \inprod{x}{L(y)}\forall x, y\in V.
\end{align*}

Om $L: V\to W$ definieras den adjungerade operatorn som $\adj{L}: W\to V$ som uppfyller
\begin{align*}
	\inprod{y}{L(x)} = \inprod{x}{\adj{L}(y)}\forall x \in V, y\in W.
\end{align*}

\paragraph{Självadjungerade operatorer}
På ett inreproduktrum $V$ är $L$ självadjungerad om
\begin{align*}
	\inprod{L(x)}{y} = \inprod{x}{L(y)}\forall x, y\in V.
\end{align*}
Matrisen för en sådan operator sägs vara Hermitesk.

\paragraph{Cauchyföljder}
En Cauchyföljd är en följd som indexeras med naturliga talen och som uppfyller att för varje $\varepsilon > 0$ finns det ett $N$ så att
\begin{align*}
	i, j > N\implies \norm{x_{i} - x_{j}} < \varepsilon.
\end{align*}

\paragraph{Fullständiga rum}
Ett rum $V$ är fullständigt om alla Cauchy-följder konvergerar.

\paragraph{Hilbertrum}
Ett Hilbertrum är ett inreproduktrum som är fullständigt.

\paragraph{$\ell^{2}$}
Vi definierar $\ell^{2}(\C)$ som mängden av alla följder av tal i $\C$ så att
\begin{align*}
	\sum\limits_{i = 0}^{N}\abs{a_{i}}^{2}
\end{align*}
är begränsad, med inreprodukten
\begin{align*}
	\inprod{A}{B} = \sum\limits_{i = 0}^{N}\cc{a_{i}}b_{i}.
\end{align*}

\paragraph{$L^{2}$}
Vi definierar $L^{2}([0, 1], \C)$ som mängden av alla komplexvärda funktioner på $[0, 1]$ med inreprodukt
\begin{align*}
	\inprod{f}{g} = \inteval{0}{1}{\cc{f}(t)g(t)}{t}.
\end{align*}

\paragraph{Ortogonala och unitära operatorer}
En ortogonal operator över ett reellt vektorrum $V$ är en inverterbar operator som uppfyller $\inprod{Lx}{Ly} = \inprod{x}{y}\forall x, y\in V$.

En unitär operator över ett komplext vektorrum $V$ är en inverterbar operator som uppfyller $\inprod{Lx}{Ly} = \inprod{x}{y}\forall x, y\in V$.

\subsection{Satser}

\paragraph{Cauchy-Schwarz olikhet}

\begin{align*}
	\abs{\inprod{x}{y}} \leq \abs{x}\abs{y}.
\end{align*}

\proof

\paragraph{Triangelolikheten}

\begin{align*}
	\abs{x + y} \leq \abs{x} + \abs{y}.
\end{align*}

\paragraph{Ortogonalt komplement och vektorrum}
Om $V$ är ett ändligdimensionellt vektorrum, är
\begin{align*}
	W = W \oplus W^{\perp}.
\end{align*}

\proof
Det gäller att
\begin{align*}
	W \cap W^{\perp} = \{0\}.
\end{align*}

\paragraph{Inreprodukt och minsta norm}
Låt $e_{1}, \dots, e_{N}$ vara ortonormala basvektorer i inreproduktrummet $V$, och låt
\begin{align*}
	V_{N} = \left\{\sum\limits_{i = 1}^{N}a_{i}e_{i}\right\}.
\end{align*}
Då ges
\begin{align*}
	\inf\limits_{\Phi\in V_{N}}\norm{u - \Phi}
\end{align*}
av
\begin{align*}
	\Phi = \sum\limits_{i = 1}^{N}\inprod{u}{e_{i}}e_{i}.
\end{align*}

\proof

\paragraph{Bra och dualrum}
Om $V$ är ett inreproduktum, definierar båd $x\to\bra{x}$ och $x\to\bra{\cc{x}}$ en injektiv avbildning $V\to V^{*}$. Om $V$ är ändligdimensionellt, är detta dessutom en isomorfi.

\proof

\paragraph{Inreproduktrum och ortogonal bas}
Ett ändligdimensionellt vektorrum har en ortogonal bas.

\proof

\paragraph{Gram-Schmidts metod}
Låt $V$ vara ett vektorrum med ändlig dimension eller en uppräknelig bas $B = \{x_{i}\}_{i= 0}^{n}$. Då bildar vektorerna
\begin{align*}
	e_{i} = x_{i} - \sum\limits_{j = 0}^{i - 1}\frac{\inprod{e_{j}}{x_{i}}}{\norm{e_{j}}^{2}}e_{j}
\end{align*}
en ortogonal bas för $V$.

\proof

\paragraph{Matrisen för en adjungerad operator}
Låt $L$ beskrivas av matrisen $A$ i någon ortonormal bas. Då beskrivs $\adj{L}$ av matrisen $\cc{(A^{T})}$, även om operatorn är mellan två olika vektorrum.

\proof
Vi har
\begin{align*}
	L(e_{i}) = \sum\limits_{j}a_{ij}e_{j},
\end{align*}
vilket ger
\begin{align*}
	\inprod{e_{i}}{L(e_{j})} = \inprod{e_{i}}{\sum\limits_{k}a_{jk}e_{k}} = \sum\limits_{k}\inprod{e_{i}}{a_{jk}e_{k}} = \sum\limits_{k}a_{jk}\inprod{e_{i}}{e_{k}} = \sum\limits_{k}a_{jk}\delta_{ik} = a_{ji}.
\end{align*}
Om $B$ är matrisen för $\adj{L}$, så vi nu att dens komponenter kan skrivas som $b_{ji} = \inprod{e_{i}}{\adj{L}(e_{j})}$. Vi utvecklar detta och får
\begin{align*}
	b_{ji} = \inprod{e_{i}}{\adj{L}(e_{j})} = \cc{\inprod{\adj{L}(e_{j})}{e_{i}}} = \cc{\inprod{e_{j}}{L(e_{i})}} = \cc{a_{ij}},
\end{align*}
och därmed är beviset klart.

\paragraph{Egenvärden för självadjungerade operatorer}
Självadjungerade operatorer har bara reella egenvärden.

\proof
\begin{align*}
	\inprod{L(x)}{x} = \inprod{x}{L(x)}.
\end{align*}
Detta kan utvecklas om $x$ är en egenvektor för att ge
\begin{align*}
	\cc{\lambda}\inprod{x}{x} = \lambda\inprod{x}{x},
\end{align*}
och beviset är klart.

\paragraph{Diagonalisering av självadjungerade operatorer}
Alla självadjungerade operatorer på ändligdimensionella inreproduktrum kan diagonaliseras.

\proof
Vi gör induktion över $\dim{V}$.

Det är trivialt för dimension $1$.

Annars, om vi har en egenvektor $x$ motsvarande egenvärdet $\lambda$, bilda $W = \Span{x}$. Då har man $L(W)\subseteq W$. Vidare, om $y\in W^{\perp}$, har man
\begin{align*}
	\inprod{x}{y} = 0\implies \inprod{\lambda x}{y} = \inprod{L(x)}{y} = \inprod{x}{L(y)} = 0,
\end{align*}
och $L(W^{\perp})\subseteq W^{\perp}$. Man kan vidare bilda en bas för $V$ med basen för $W^{\perp}$ och $x$. Matrisen för $L$ med avseende på denna basen är
\begin{align*}
	\mqty[
		\lambda & 0 \\
		0       & X
	].
\end{align*}
Per induktion finns det då en ortogonal bas för $W^{\perp}$ så att matrisen för $L$ på $W^{\perp}$ blir diagonal.

\paragraph{Ortogonal bas och självadjungerade operatorer}
Om $L$ är en självadjungerad operator på ett ändligdimensionellt vektorrum $V$, finns det en ortogonal bas av egenvektorer till $L$.

\proof

\paragraph{Självadjungerade operatorer och ortogonala egenvektorer}
Låt $x$ vara en egenvektor till $L$ med egenvärdet $\lambda $och $y$ vara en egenvektor med egenvärde $\mu$. Om $\lambda - \mu \neq 0$ är $\inprod{x}{y} = 0$.

\proof
\begin{align*}
	\inprod{L(x)}{y} = \inprod{x}{L(y)}.
\end{align*}
Vi använder att $x$ och $y$ är egenvektorer och får
\begin{align*}
	\cc{\lambda}\inprod{x}{y} = \mu\inprod{x}{y}.
\end{align*}
Enligt antagandet måste då $\inprod{x}{y} = 0$.

\paragraph{Längdbevarande operatorer}
För en operator $L$ på ett reellt inreproduktrum $V$ är följande ekvivalent:
\begin{itemize}
	\item $\norm{Lx} = \norm{x}\forall x\in V$.
	\item $\inprod{x}{y} = \inprod{Lx}{Ly}\forall x, y\in V$.
\end{itemize}
Om vektorrummet är ändligdimensionellt, är påståenden även ekvivalenta med
\begin{itemize}
	\item $L$ avbildar ortonormala baser på ortonormala baser.
\end{itemize}

\proof

\paragraph{Längdbevarande operatorer som bijektioner}
Längdbevarande operatorer på ändligdimensionella vektorrum är bijektiva.

\proof
Sådana operatorer är injektiva ty
\begin{align*}
	\norm{Lx} = 0 \implies x = 0.
\end{align*}

De är surjektiva ty matrisen för avbildningen i någon bas måste ha linjärt oberoende kolumner för att vara injektiv. Detta implicerar att avbildningen är surjektiv.

\paragraph{Ortogonala grupper}
\begin{itemize}
	\item Mängden $O(V) = \{L: V\to V: L\text{ är ortogonal}\}$ är en grupp under sammansättning om $V$ är ett reellt inreproduktrum med en ortogonal bas.
	\item Mängden $O_{n}(V) = \{A\in M_{n}(\R): A\text{ är ortogonal}\}$ är en grupp under matrismultiplikation.
	\item Mängden $O_{n}(V) = \{A\in M_{n}(\R): \det{A} = 1, A\text{ är ortogonal}\}$ är en grupp under matrismultiplikation.
\end{itemize}

Satsen stämmer även för de komplexa motsvarigheterna.

\proof

\paragraph{Egenvärden och egenvektorer till ortogonala och unitära operatorer}
Om $L$ är en unitär operator på ett ändligdimensionellt vektorrum, finns det en ortogonal bas av egenvektorer till $L$ och alla egenvärden till $L$ har belopp $1$.

\proof
Det finns (möjligtvis) minst ett egenvärde $\lambda$. Välj en motsvarande egenvektor $x$. Detta ger
\begin{align*}
	\norm{x} = \norm{Lx} \implies \abs{\lambda} = 1.
\end{align*}
Låt nu $W = \Span{x}$. Vi har att $L(W^{\perp})\to W^{\perp}$ eftersom $L$ bevarar inreprodukten. Detta ger (?) per induktion att det finns en bas för $W^{\perp}$ av egenvektorer till $L$. Unionen av $x$ och denna basen är därmed en bas för hela vektorrummet.

\paragraph{Exponentialavbidlningen och Hermiteska operatorer}
Låt $H$ vara Hermitesk. Då är $e^{iH}$ unitär.

\proof
Eftersom $H$ är Hermitesk, finns det en ortogonal bas av egenvektorer. I denna basen är matrisen för $H$ en diagonalmatris. Därmed är matrisen för $e^{iH}$ en diagonalmatris med $e^{i\lambda}$ på diagonalen, där $\lambda$ är något egenvärde. Alltså finns det en ortogonal bas av egenvektorer till $e^{iH}$, där varje egenvärde har belopp $1$, och $e^{iH}$ är unitär.

\end{document}
