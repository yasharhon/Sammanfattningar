\section{Inreprodukt}

\subsection{Definitioner}

\paragraph{Inreprodukt över $\R$}
En inreprodukt $\inprod{x}{y}$ på ett vektorrum $V$ över $\R$ är en avbildning $V\times V\to \R$ som är
\begin{itemize}
	\item bilinjär, dvs.
	\begin{itemize}
		\item $\inprod{x + y}{z} = \inprod{x}{z} + \inprod{y}{z}$.
		\item $\inprod{ax}{y} = a\inprod{x}{y}$.
		\item $\inprod{x}{y + z} = \inprod{x}{y} + \inprod{x}{z}$.
		\item $\inprod{x}{ay} = a\inprod{x}{y}$.
	\end{itemize}
	\item symmetrisk, dvs. $\inprod{x}{y} = \inprod{y}{x}$.
	\item positivt definit, dvs. $\inprod{x}{x} > 0$ om $x \neq 0$.
\end{itemize}

\paragraph{Inreprodukt över $\C$}
En inreprodukt $\inprod{x}{y}$ på ett vektorrum $V$ över $\C$ är en avbildning $V\times V\to \C$ som är
\begin{itemize}
	\item seskvilinjär, dvs. bilinjär, men $\inprod{ax}{y} = \cc{a}\inprod{x}{y}$.
	\item konjugatsymmetrisk, dvs. $\inprod{x}{y} = \cc{\inprod{y}{x}}$.
	\item positivt definit, dvs. $\inprod{x}{x} > 0$ om $x \neq 0$. Notera att detta och konjugatsymmetrin implicerar att $\inprod{x}{x}$ har ingen imaginärdel.
\end{itemize}

\paragraph{Inreprodukt från matris}
Vi kan få en inreprodukt i $\C^{n}$ från en matris genom
\begin{align*}
	\inprod{x}{y} = \sum a_{ij}\cc{x_{i}}{y_{j}} = (\cc{x})^{T}Ay.
\end{align*}

Om detta skall uppfylla konjugatsymmetri, ger det
\begin{align*}
	(\cc{x})^{T}Ay = \cc{(\cc{y})^{T}Ax} = y^{T}\cc{A}\cc{x}.
\end{align*}

Transponering av högersidan ger
\begin{align*}
	(\cc{x})^{T}Ay = (\cc{x})^{T}(\cc{A})^{T}y,
\end{align*}
och därmed uppfylls konjugatsymmetrin om
\begin{align*}
	A = (\cc{A})^{T}.
\end{align*}
Om matrisen uppfyller detta, säjs den vara konjugatsymmetrisk eller Hermitesk.

\paragraph{Norm}
Normen eller längden av en vektor definieras som
\begin{align*}
	\abs{x} = \sqrt{\inprod{x}{x}}.
\end{align*}

\paragraph{Vinkel}
Vinkeln $\theta$ mellan två vektorer definieras som
\begin{align*}
	\cos{\theta} = \frac{\inprod{x}{x}}{\abs{x}\abs{y}}.
\end{align*}

\paragraph{Ortogonalitet}
$x$ och $y$ är ortogonala om
\begin{align*}
	\inprod{x}{y} = 0.
\end{align*}

\paragraph{Ortogonalt komplement}
Om $W\subseteq V$ är ett delrum så finns det ett ortogonalt komplement
\begin{align*}
	W^{\perp} = \left\{x\in V: \inprod{x}{y} = 0\forall y\in W\right\} \subseteq V.
\end{align*}

\subsection{Satser}

\paragraph{Cauchy-Schwarz olikhet}

\begin{align*}
	\abs{\inprod{x}{y}} \leq \abs{x}\abs{y}.
\end{align*}

\proof

\paragraph{Triangelolikheten}

\begin{align*}
	\abs{x + y} \leq \abs{x} + \abs{y}.
\end{align*}

\paragraph{Ortogonalt komplement och vektorrum}
Om $V$ är ett ändligdimensionellt vektorrum, är
\begin{align*}
	W = W \oplus W^{\perp}.
\end{align*}

\proof
Det gäller att
\begin{align*}
	W \cap W^{\perp} = \{0\}.
\end{align*}