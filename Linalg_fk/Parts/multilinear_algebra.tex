\section{Multilinjär algebra}

Observera att Einsteinnotation kommer användas ofta i denna del.

Olika definitioner

\subsection{Definitioner}

\paragraph{Tensorprodukt av vektorrum från produkter}
Tensorprodukten $V\otimes W$ av vektorrummen $V$ och $W$ definieras som vektorrummet med bas $\{e_{i}\otimes f_{j}\}_{i\in I, j\in J}$ om $\{e_{i}\}_{i\in I}$ och $\{f_{j}\}_{j\in J}$ är baser för $V$ respektiva $W$.

\paragraph{Basbyten under tensorprodukt}
Betrakta $V\otimes W$, med baser om $\{e_{i}\}_{i\in I}$ och $\{f_{j}\}_{j\in J}$ för $V$ respektiva $W$. Vid basbytet
\begin{align*}
	e_{i}' = p{ik}e_{k}, f_{j}' = q{jl}f_{l}
\end{align*}
ges basvektorerna för $V\otimes W$ av
\begin{align*}
	e_{i}'\otimes f_{j}' = (p_{ik}e_{k})\otimes (q_{jl}f_{l}).
\end{align*}
Detta definieras som att det uppfyllar distributiva lagen, vilket ger
\begin{align*}
	e_{i}'\otimes f_{j}' = p_{ik}q_{jl}e_{k}\otimes f_{l}.
\end{align*}

\paragraph{Tensorprodukt av vektorrum från dualer}
Tensorprodukten $V\otimes W$ av vektorrummen $V$ och $W$ som verkas på av kroppen $k$ definieras som $\{L: V^{*}\times W^{*}\to k:\ L\text{ är bilinjär}\}$.

\paragraph{Bas för tensorprodukt}
Vid att använda bilinjariteten till ett godtyckligt $L$ kan det skrivas som
\begin{align*}
	L(a_{i}e_{i}^{*}, b_{j}f_{j}^{*}) = a_{i}b_{j}L(e_{i}^{*}, f_{j}^{*}).
\end{align*}
Eftersom en linjär avbildning ges unikt av dens verkan på basvektorerna, tar vi varje $L(e_{i}^{*}, f_{j}^{*})$ som ett element i en bas för tensorprodukten. Vi låter även denna vara
\begin{align*}
	e_{i}\otimes f_{j} = L(e_{i}^{*}, f_{j}^{*}) = e_{i}^{*}(e_{k})f_{j}^{*}(f_{l}) = \delta_{ik}\delta_{jl}.
\end{align*}

\subsection{Satser}