\section{Multilinjär algebra}

Observera att Einsteinnotation kommer användas i denna del.

\subsection{Definitioner}

\paragraph{Tensorprodukt av vektorrum från produkter}
Tensorprodukten $V\otimes W$ av vektorrummen $V$ och $W$ definieras här som vektorrummet med bas $\{e_{i}\otimes f_{j}\}_{i\in I, j\in J}$ om $\{e_{i}\}_{i\in I}$ och $\{f_{j}\}_{j\in J}$ är baser för $V$ respektiva $W$.

\paragraph{Basbyten med tensorprodukt från produkter}
Vi utgår från definitionen av tensorprodukt ovan. Betrakta $V\otimes W$, med baser om $\{e_{i}\}_{i\in I}$ och $\{f_{j}\}_{j\in J}$ för $V$ respektiva $W$. Vid basbytet
\begin{align*}
	e_{i}' = p_{ik}e_{k}, f_{j}' = q_{jl}f_{l}
\end{align*}
ges basvektorerna för $V\otimes W$ av
\begin{align*}
	e_{i}'\otimes f_{j}' = (p_{ik}e_{k})\otimes (q_{jl}f_{l}).
\end{align*}
Tensorprodukten definieras här som att det uppfyllar distributiva lagen, vilket ger
\begin{align*}
	e_{i}'\otimes f_{j}' = p_{ik}q_{jl}e_{k}\otimes f_{l}.
\end{align*}

\paragraph{Tensorprodukt av vektorrum från dualer}
Tensorprodukten $V\otimes W$ av vektorrummen $V$ och $W$ som verkas på av kroppen $k$ definieras här som $\{L: V^{*}\times W^{*}\to k:\ L\text{ är bilinjär}\}$. Elementen $x\otimes y$ i $V\otimes W$ definieras som $x\otimes y(\phi, psi) = \phi(x)\psi(y)$.

\paragraph{Tensorprodukt av avbildningar}
Låt $L_{1}$ och $L_{2}$ vara linjära avbildningar på $V_{1}$ respektiva $V_{2}$. Givet universella egenskapen finns det en unik avbildning $L_{1}\otimes L_{2}: V_{1}\otimes V_{2}\to W_{1}\otimes W_{2}$ som definieras av $f(x, y) = L_{1}(x)\otimes L_{2}(y)$ och ges av $L_{1}\otimes L_{2}(x, y) = L_{1}(x)\otimes L_{2}(y)$. Denna definieras som tensorprodukten av $L_{1}$ och $L_{2}$.

\paragraph{Spåravbildningen}
Givet universella egenskapen definierar vi $\Tr: V^{*}\otimes V\to k$ som den avbildningen som kommer från evalueringsavbildningen $V^{*}\times V\to k, (\phi, x)\to \phi(x)$.

\paragraph{Symmetriska tensorer}
Delrummet av symmetriska tensorer definieras som
\begin{align*}
	\Sym[2]{V} = \Span{x\otimes x,\ x\in V}.
\end{align*}

\paragraph{Alternerande tensorer}
Delrummet av alternerande tensorer definieras som
\begin{align*}
	\Alt[2]{V} = \Span{x\otimes y - y\otimes x,\ x, y\in V}.
\end{align*}

\subsection{Satser}

\paragraph{Bas för tensorprodukt från dualer}
Om $B = \{e_{1}, \dots, e_{n}\}$ är en bas för $V$ och $V = \{f_{1}, \dots, f_{m}\}$ är en bas för $W$ ger $B = \{e_{i}\otimes f_{j}\}_{i = 1, j = 1}^{m, n}$ en bas för $V\otimes W$.

\proof
Alla avbildningar i $V\otimes W$ bestäms entydigt av hur de verkar på $(e_{i}*, f_{j}*)$ eftersom dessa är bilinjära. Låt $e_{k}\otimes f_{l}$ vara så att $e_{k}\otimes f_{l}(e_{i}*, f_{j}*) = \delta_{ki}\delta_{lj}$. Då kan en ny linjär avbildning $f$ som uppfyller $f(e_{i}*, f_{j}*) = a_{ij}$ skrivas som
\begin{align*}
	f = a_{kl}e_{k}\otimes f_{l}.
\end{align*}
Eftersom $f$ nu kan vara godtycklig, är beviset klart.

\paragraph{Homomorfisats}
Låt $\text{Hom}_{k}(V, W)$ vara mängden av alla linjära avbildningar från $V$ till $W$. Då är $\text{Hom}_{k}(V, W)\cong V^{*}\otimes W$.

\proof
$V^{*}\otimes W$ har bas med element $e_{i}^{*}\otimes f_{j}$. Varje sådant element motsvarar en linjär avbildning $V\to W$ genom $L_{ij}(x) = e_{i}^{*}(x) f_{j}$.

Om $V^{*}$ har dimension $m$ och $W$ dimension $n$, skulle man nu kunna ställa upp en matris för en godtycklig avbildning i $\text{Hom}_{k}(V, W)$. Denna skulle haft storlek $n\times m$. Matrisen för $L_{ij}$ är av samma storlek, och har nollor i alla element förutom element $(i, j)$, som är en etta. Om matrisen för en godtycklig avbildning har element $a_{ij}$, kan denna avbildningen därmed skrivas som $a_{ji}L_{ij}$ (transponeringen kommer av att andra indexet i $L_{ij}$ motsvarar element i basen för $W$, vilket motsvarar radindex i matrisen för avbildningen). Denna summan innehåller lika många termer som det är i basen för $V^{*}\otimes W$, och därmed är beviset klart.

\paragraph{Universella egenskapen}
Om $f: V\times W\to U$ är bilinjär finns en unik bilinjär avbilding $\phi: V\otimes W\to U$ så att $f = \comp{{\phi, \psi}}$, där $\psi: V\times W\to V\otimes W$, för alla $f: V\times W\to U$.

\proof
Avbildningen $\psi: V\times W\to V\otimes W, (x, y)\to x\otimes y$ är bilinjär. Om nu $f$ är bilinjär kan vi definiera en bilinjär avbildning $\phi: V\otimes W\to U$ som $\phi(e_{i}\otimes f_{j}) = f(e_{i}, f_{j})$, där $e_{i}$ och $f_{j}$ är i basen för $V$ respektiva $W$. Vi ser nu att $f = \comp{{\psi, \phi}}$.

För att visa att $\phi$ är unik, anta att $F$ uppfyller $F(e_{i}\otimes f_{j}) = f(e_{i}, f_{j})$. Då är $\phi - F$ noll på hela basen för $V\otimes W$, och måste därmed vara noll som linjär avbildning.

\paragraph{Tensorprodukt av avbildningar}
Låt $L_{1}$ och $L_{2}$ vara linjära avbildningar på $V_{1}$ respektiva $V_{2}$ med matriser $A_{1}$ respektiva $A_{2}$ för några val av baser för $V_{1}$ och $V_{2}$. Då ges matrisen för $L_{1}\otimes L_{2}$ av
\begin{align*}
	\mqty[
		A_{1, 11}A_{2}             & A_{1, 12}A_{2}             & \dots  & A_{1, 1, \dim{V_{1}}}A_{2} \\
		A_{1, 21}A_{2}             & A_{1, 22}A_{2}             & \dots  & A_{1, 2, \dim{V_{1}}}A_{2} \\
		\vdots                     & \vdots                     & \ddots & \vdots \\
		A_{1, \dim{V_{1}}, 1}A_{2} & A_{1, \dim{V_{1}}, 2}A_{2} & \dots  & A_{1, \dim{V_{1}}, \dim{V_{1}}}A_{2}
	].
\end{align*}

\proof
Låt $V_{1}$ och $V_{2}$ ha baser $\set{e_{i}}{}_{i\in I}$ respektiva $\set{f_{j}}{}_{j\in J}$, och $W_{1}$ och $W_{2}$ ha baser $\set{g_{k}}{}_{k\in K}$ respektiva $\set{h_{m}}{}_{m\in M}$. De linjära avbildningarna ges av
\begin{align*}
	L_{1}(e_{i}) &= A_{1, ki}g_{k}, \\
	L_{2}(f_{j}) &= A_{2, mj}h_{m}.
\end{align*}
Med definitionen av tensorprodukten av $L_{1}$ och $L_{2}$ fås
\begin{align*}
	L_{1}\otimes L_{2}(e_{i}, f_{j}) &= L_{1}(e_{i})\otimes L_{2}(f_{j}) \\
	                                 &= A_{1, ki}g_{k}\otimes A_{2, mj}h_{m} \\
	                                 &= A_{1, ki}A_{2, mj}g_{k}\otimes h_{m}
\end{align*}
För att kunna fortsätta, behöver vi en idé om ordning i båda baserna. Vi ordnar dessa först efter indexet till vänster och därefter efter indexet till höger, så att ordningen blir $e_{1}\otimes f_{1}, e_{1}\otimes f_{2}, \dots, e_{2}\otimes f_{1}, e_{2}\otimes f_{2}, \dots$ i $V_{1}\otimes V_{2}$ och motsvarande i $W_{1}\otimes W_{2}$. Lite fundering ger då att $e_{i}\otimes f_{j}$ är element nummer $\dim{V_{2}}(i - 1) + j$ i basen. Om nu $L_{1}\otimes L_{2}$ har matris $B$ med det givna valet av baser, ger detta
\begin{align*}
	B_{\dim{W_{2}}(k - 1) + m, \dim{V_{2}}(i - 1) + j} = A_{1, ki}A_{2, mj}.
\end{align*}
Det här kanske säjer inte så mycket, men fixera nu alla index förutom $m$. Koefficienten framför nästa basvektor fås vid att röra sig ned rätt kolumn i $A_{2}$. På samma sättet ser vi att fixering av alla index förutom $j$ motsvarar att röra sig längsmed rätt rad i $A_{2}$. I båda fall multipliceras det med ett element från $A_{1}$, vilket ger att matrisen för $L_{1}\otimes L_{2}$ med det givna valet av bas blir
\begin{align*}
	B =
	\mqty[
		A_{1, 11}A_{2}             & A_{1, 12}A_{2}             & \dots  & A_{1, 1, \dim{V_{1}}}A_{2} \\
		A_{1, 21}A_{2}             & A_{1, 22}A_{2}             & \dots  & A_{1, 2, \dim{V_{1}}}A_{2} \\
		\vdots                     & \vdots                     & \ddots & \vdots \\
		A_{1, \dim{V_{1}}, 1}A_{2} & A_{1, \dim{V_{1}}, 2}A_{2} & \dots  & A_{1, \dim{V_{1}}, \dim{V_{1}}}A_{2}
	].
\end{align*}

\paragraph{Tensorprodukt som direkt summa}
Om $1 + 1\neq 0$ i kroppen $k$ som verkar på $V$ är
\begin{align*}
	V\otimes V = \Sym[2]{V} \oplus \Alt[2]{V}.
\end{align*}

\proof
Vi vill först visa att ett godtyckligt $x\otimes y$ kan skrivas som en linjär kombination av element från de två delrummen. Definiera $a = x + y$ och $b = x - y$. Detta ger
\begin{align*}
	x\otimes y = \frac{a + b}{2}\otimes\frac{a - b}{2} = \frac{1}{4}(a\otimes a - a\otimes b + b\otimes a - b\otimes b).
\end{align*}
Det är klart att de två termerna i mitten till sammans är från $\Alt[2]{V}$, medan den första och sista är från $\Sym[2]{V}$, vilket visar första delen av påståendet.

För att visa att linjärkombinationen är unik, visar vi att skärningen mellan delrummen endast är $0$. Vi kan utveckla termerna från de olika delrummen och få
\begin{align*}
	a\otimes a - b\otimes b &= (x + y)\otimes (x + y) - (x - y)\otimes (x - y) \\
	                        &= x\otimes x + x\otimes y + y\otimes x + y\otimes y - x\otimes x + x\otimes y + y\otimes x - y\otimes y \\
	                        &= (1 + 1)x\otimes y + (1 + 1)y\otimes x, \\
	b\otimes a - a\otimes b &= (x - y)\otimes (x + y) - (x + y)\otimes (x - y) \\
	                        &= x\otimes x + x\otimes y - y\otimes x - y\otimes y - x\otimes x + x\otimes y - y\otimes x + y\otimes y \\
	                        &= (1 + 1)x\otimes y - (1 + 1)y\otimes x.
\end{align*}
Om $x\otimes y\in\Alt[2]{V}$ är termerna från $\Sym[2]{V}$ lika med noll, vilket medför $x\otimes y = -y\otimes x$, och om $x\otimes y\in\Sym[2]{V}$ är termerna från $\Alt[2]{V}$ lika med noll, vilket medför $x\otimes y = y\otimes x$. Om båda dessa skall uppfyllas samtidigt, måste $x\otimes y = 0$. Därmed är beviset klart.