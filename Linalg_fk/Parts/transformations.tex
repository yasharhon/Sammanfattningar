\section{Avbildningar}

\subsection{Definitioner}

\paragraph{Isomorfir}
En isomorfi är en bijektiv avbildning mellan vektorrum.

\paragraph{Linjära avbildningar}
En avbildning $T$ är linjär om
\begin{align*}
	T(x + y) = T(x) + T(y), \\
	T(cx) = cT(x), c\in\R.
\end{align*}
Vi säjer att $T$ respekterar eller bevarar strukturen som vektorrum.

\paragraph{Matriser för linjära avbildningar}
Om $B$ är en bas för $V$ och $D$ är en bas för $W$ kan vi ordna en matris för $L: V\to W$ genom
\begin{align*}
	L(x_{i}) = \sum\limits_{j\in I}a_{ji}y_{j},
\end{align*}
där alla $x_{i}\in B$, alla $y_{i}\in D$ och $I$ är en mängd av index som det skall summeras över.

\paragraph{Koordinater}
Låt $B = \left\{x_{i}\right\}_{i\in I}$ vara en bas för vektorrummet $V$. Vi definierar då en avbildning
\begin{align*}
	V                  &\to k^{I}\equiv \dirsum{i\in I}{}{k}, \\
	x = \sum a_{i}x{i} &\to \left\{a_{i}\right\}_{i\in I}.
\end{align*}
Detta är en isomorfi.

\subsection{Satser}

\paragraph{Basbyte}
Låt $L$ vara en avbildning från $V$ till $W$. Låt $_{D}[L]_{B}$ vara representationen av avbildningen $L$ från basen $B$ till $D$, där $B$ är en bas för $V$ och $D$ en bas för $W$. Då gäller det att
\begin{align*}
	
\end{align*}

\paragraph{Isomorfisatsen}
Låt $L: V\to W$ vara linjär. Då gäller det att
\begin{align*}
	\Im{L} \cong \frac{V}{\ker{L}}.
\end{align*}