\section{Egenvärden och olika polynom}

\subsection{Definitioner}

\paragraph{Egenvektorer}
$x$ är en egenvektor till $L$ om det finns ett $\lambda\in k$ så att
\begin{align*}
	Lx = \lambda x.
\end{align*}
$\lambda$ kallas det motsvarande egenvärdet.

\paragraph{Karakteristiskt polynom}
Om $V$ är ändligdimensionellt ges det karakteristiska polynomet av
\begin{align*}
	p_{L}(x) = \det{xI - L}\in k[x],
\end{align*}
där $I$ är identitetsavbildningen.

\paragraph{Minimalpolynom}
Om $A$ är en matris, är minimalpolynomet $q_{A}(x)\in k[x]$ det moniska polynomet av lägst grad så att $q_{A}(A) = 0$.

\paragraph{Diagonaliserbarhet}
En operator är diagonaliserbar om det finns en bas så att operatorns matris i den basen är diagonal.

\paragraph{Samtidig diagonaliserbarhet}
Två operatorer $L_{1}$ och $L_{2}$ är samtidigt diagonaliserbara om båda är diagonaliserbara och det finns en gemensam bas av egenvektorer.

\subsection{Satser}

\paragraph{Karakteristiska polynom och egenvärden}
Om $lambda$ är ett egenvärde till $L$ så är $p_{L}(\lambda) = 0$.

\proof
Ez

\paragraph{Existens av minimalpolynom}
Om $V$ är ändligdimensionellt, har $L$ ett karakteristiskt polynom.

\proof
Betrakta matrisen $A$ för $L$ i någon bas. Det gäller att mängden $\{A^{0}, A^{1}, \dots, A^{n^2}\}$ är linjärt beroende, och därmed finns det koefficienter $a_{0}, \dots, a_{n}$ så att
\begin{align*}
	\sum a_{i}A^{i} = 0.
\end{align*}

\paragraph{Cayley-Hamiltons sats}
$p_{L}(L) = 0$.

\proof
Om matrisen för $L$ är diagonal så är det uppenbart, ty
\begin{align*}
	A^{i} =
	\left[\begin{array}{ccc}
		\lambda_{1} & \dots  & 0 \\
		\vdots      & \ddots & \vdots \\
		0           & \dots  & \lambda_{n}
	\end{array}\right]
	\implies
	p_{A}(A) =
	\left[\begin{array}{ccc}
		p_{A}(\lambda_{1}) & \dots  & 0 \\
		\vdots             & \ddots & \vdots \\
		0                  & \dots  & p_{A}(\lambda_{n})
	\end{array}\right].
\end{align*}
I övrigt oklart.

\subparagraph{Korrolar}
$q_{L}$ är en faktor i $p_{L}$.

\paragraph{Multipliciteter och diagonaliserbarhet}
Om $L$ är diagonaliserbar, är den geometriska multipliciteten lika med den algebraiska multipliciteten för alla $L$:s egenvärden.

\proof

\paragraph{Konjugerade matriser}
Alla matriser är konjugerade med en övertriangulär matris med matrisens egenvärden på diagonalen.

\paragraph{Samtidig diagonaliserbarhet och kommutativitet}
Låt $V$ vara ett ändligdimensionellt vektorrum och $L_{1}, L_{2}$ två operatorer på detta. Då går det att diagonalisera $L_{1}$ och $L_{2}$ om de kommuterar.

\proof

\paragraph{Kommutativitet och egenrum}
Låt $L_{1}$ och $L_{2}$ kommutera och $E_{1}$ vara egenrum till $L_{1}$. Då är $L_{2}(E_{1})\subset E_{1}$.

\proof

\paragraph{Nilpotens och blockdiagonalitet}
Om $L$ är nilpotent finns det en bas för $V$ så att matrisen för $L$ blir blockdiagonal, där varje block är på formen
\begin{align*}
	\mqty[
		0      & 1      & 0      & \dots  & 0 \\
		0      & 0      & 1      & \dots  & \vdots \\
		0      & 0      & 0      & \ddots & \vdots \\
		\vdots & \vdots & \vdots & \ddots & \vdots \\
		0      & \dots  & \dots  & \dots  & 0
	].
\end{align*}

\proof
Här kommer endast en bevisidé presenteras.

Det finns ett $s$ så att $L^{s} = 0$ och $L^{s - 1} \neq 0$. Vi väljer då ett delrum $W_{s}$ så att
\begin{align*}
	V = \ker{L^{s}} = W_{s} \oplus \ker{L^{s - 1}}.
\end{align*}
Vi väljer vidare $W_{s - 1}\subseteq\ker{L^{s - 1}}$ så att
\begin{align*}
	\ker{L^{s - 1}} = W_{s - 1} \oplus L(W_{s}) \oplus \ker{L^{s - 2}}.
\end{align*}
Detta går eftersom $L(W_{s}) \subseteq \ker{L^{s - 2}}$ och $L(W_{s}) \cap \ker{L^{s - 2}} = \{0\}$. Upprepa prosedyren tills man får
\begin{align*}
	\ker{L} = \dirsum{i = 0}{s - 1}{L^{i}(W_{i + 1})}.
\end{align*}
Välj nu baser för alla $W_{i}$ och bilderna av alla potenser av $L$. Dissa bildar en bas för $V$. Med en lämplig ordning på basen fås $\dim{W_{i}}$ block på formen ovan i matrisen, varje med storlek $i\times i$.

\paragraph{Jordans normalform}
Om en operator har karakteristiskt polynom
\begin{align*}
	p_{L}(x) = \prod (x - \lambda_{i}),
\end{align*}
finns det en bas så att matrisen för $L$ är på formen
\begin{align*}
	\left[\begin{array}{cccc}
		\Lambda_{1} & 0           & \dots  & 0 \\
		0           & \Lambda_{2} & \dots  & \vdots \\
		\vdots      & \vdots      & \ddots & \vdots \\
		0           & \dots       & \dots  & \Lambda_{i}
	\end{array}\right],
\end{align*}
med 
\begin{align*}
	\Lambda_{i} =
	\left[\begin{array}{cccc}
		\lambda_{i} & 1           & \dots       & 0 \\
		0           & \lambda_{i} & \dots       & \vdots \\
		\vdots      & \dots       & \ddots      & 1 \\
		0           & \dots       & 0           & \lambda_{i}
	\end{array}\right].
\end{align*}
En sådan matris är på Jordans normalform. Vi noterar att $\Lambda_{i} = \lambda_{i}I + N$, där $N$ är nilpotent.

\proof