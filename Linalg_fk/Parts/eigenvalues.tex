\section{Egenvärden och olika polynom}

\subsection{Definitioner}

\paragraph{Egenvektorer}
$x$ är en egenvektor till $L$ om det finns ett $\lambda\in k$ så att
\begin{align*}
	Lx = \lambda x.
\end{align*}
$\lambda$ kallas det motsvarande egenvärdet.

\paragraph{Karakteristiskt polynom}
Om $V$ är ändligdimensionellt ges det karakteristiska polynomet av
\begin{align*}
	p_{L}(x) = \det{xI - L}\in k[x],
\end{align*}
där $I$ är identitetsavbildningen.

\paragraph{Minimalpolynom}
Om $L$ är linjär, är minimalpolynomet $q_{L}(x)\in k[x]$ det moniska polynomet av lägst grad så att $q_{L}(x) = 0$.

\subsection{Satser}

\paragraph{Karakteristiska polynom och egenvärden}
Om $lambda$ är ett egenvärde till $L$ så är $p_{L}(\lambda) = 0$.

\proof
Ez

\paragraph{Existens av minimalpolynom}
Om $V$ är ändligdimensionellt, har $L$ ett karakteristiskt polynom.

\proof
Betrakta matrisen $A$ för $L$ i någon bas. Det gäller att mängden $\{A^{0}, A^{1}, \dots, A^{n^2}\}$ är linjärt beroende, och därmed finns det koefficienter $a_{0}, \dots, a_{n}$ så att
\begin{align*}
	\sum a_{i}A^{i} = 0.
\end{align*}

\paragraph{Cayley-Hamiltons sats}
$p_{L}(L) = 0$.

\proof
Om matrisen för $L$ är diagonal så är det uppenbart, ty
\begin{align*}
	A^{i} =
	\left[\begin{array}{ccc}
		\lambda_{1} & \dots  & 0 \\
		\vdots      & \ddots & \vdots \\
		0           & \dots  & \lambda_{n}
	\end{array}\right]
	\implies
	p_{A}(A) =
	\left[\begin{array}{ccc}
		p_{A}(\lambda_{1}) & \dots  & 0 \\
		\vdots             & \ddots & \vdots \\
		0                  & \dots  & p_{A}(\lambda_{n})
	\end{array}\right].
\end{align*}