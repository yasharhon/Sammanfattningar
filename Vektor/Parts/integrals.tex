\section{Integraler}

\paragraph{Linjeintegraler}
En linjeintegral skrivs på formen
\begin{align*}
	\integ{C}{\vb{v}}{\vb{r}}.
\end{align*}
Det representerar hur mycket av ett vektorfält som är parallellt med en bana i rummet. Om det låter oklart, tänk att vektorfältet $\vb{v}$ puttar på en partikel som rär sig längs med banan $C$.

\paragraph{Rotation}
Från en linjeintegral kan rotationen definieras som
\begin{align*}
	\rot{\vb{v}}\cdot\vb{n} = \lim\limits_{A\to 0}\frac{1}{A}\integ{C}{\vb{v}}{\vb{r}},
\end{align*}
där $A$ är arean som omslutas av kurvan $C$ och $\vb{n}$ är normal på $C$. Denna tolkas fysikalisk som tätheten av virvlar i fältet $\vb{v}$ som roterar normalt på $\vb{n}$.

\paragraph{Flödesintegraler}
En flödesintegral skrivs på formen
\begin{align*}
	\integ{S}{\vb{v}}{\vb{S}}.
\end{align*}
Den representerar hur mycket av ett vektorfält som flöder genom ytan $S$.

\paragraph{Divergens}
Från en flödesintegral kan divergensen definieras som
\begin{align*}
	\div{\vb{v}} = \lim\limits_{V\to 0}\frac{1}{V}\integ{V}{\vb{v}}{\vb{S}},
\end{align*}
där $V$ är volymen som omslutas av ytan $S$. Denna tolkas fysikalisk som tätheten av källor till fältet $\vb{v}$.