\section{Tensorer}

Låt oss betrakta två observatörer $A$ och $B$. Dessa beskriver någon storhet som vektorn $\vb{v}$. I de två observatörerna sina koordinatsystem skrivs $\vb{v}$ som $v_i\vb{e}_{i}$ respektiva $v_i'\vb{e}_{i}'$. Vi söker någon metod att transformera oss mellan dessa.

Vi kan skriva komponenten $v_i'$ som
\begin{align*}
	v_i' = \vb{e}_{i}'\cdot\vb{v} = \vb{e}_{i}\cdot (v_j\vb{e}_{j}) = L_{ij}v_{j},
\end{align*}
där vi har användt indexnotation och definierat vinkeln
\begin{align*}
	L_{ij} = \vb{e}_{i}'\cdot\vb{e}_{j}.
\end{align*}
Detta definierar en matris $L$ som låter oss transformera mellan koordinatsystemerna.

En matris $L$ kan representera en koordinattransform om och endast om
\begin{align*}
	LL^{T} = L^{T}L = I.
\end{align*}

För att visa detta, skriv
\begin{align*}
	(LL^{T})_{ij} = L_{ik}L^{T}_{kj} = L_{ij}L_{jk}.
\end{align*}
Om $L$ har samma struktur som vi förväntade för koordinattransformationer, har man
\begin{align*}
	L_{ij}L_{jk} = (\vb{e}_{i}'\cdot\vb{e}_{k})(\vb{e}_{j}'\cdot\vb{e}_{k}).
\end{align*}
Vi kommer ihåg att vi har
\begin{align*}
	\vb{e}_{i}' = (\vb{e}_{i}'\cdot\vb{e}_{j})\vb{e}_{j}
\end{align*}
och kan då skriva
\begin{align*}
	L_{ij}L_{jk} = \vb{e}_{i}'\cdot\vb{e}_{j}'.
\end{align*}
Vi jobbar med ortogonala koordinatsystemer, vilket ger $\vb{e}_{i}'\cdot\vb{e}_{j}' = \delta_{ij}$, vilket skulle visas.

Vi har även att $\det (L) = \pm 1$. För att visa detta, skriv
\begin{align*}
	1 = \det (I) = \det (LL^{T}) = \det (L)\det (L^{T}) = \det (L)^2.
\end{align*}

Mängden av alla koordinattransformationer som uppfyller dessa relationerna kallas $O(3)$, där $O$ står för ortogonal. Mängden av alla transformationer i $O(3)$ vars matriser har positiv determinant kallas $SO(3)$, där $S$ står för speciell. Vid att kombinera dessa med sammansättning av flera transformationer, definierar de en gruppstruktur. 

Gruper uppfyller
\begin{itemize}
	\item Om $g, g'$ är i grupen $G$, är $gg'$ i $G$.
	\item Det finns ett element $e$ i $G$ så att $eg = ge = g$ för alla $g$.
	\item För alla $g$ finns det ett $g^{-1}$ så att $gg^{-1} = g^{-1}g = e$.
	\item $a(bc) = (ab)c$ för $a, b, c$ i $G$.
\end{itemize}

Vi definierar rotationstransformen kring $\vb{e}_{i}$-axeln med en vinkel $\alpha$. Då gäller att alla transformationer i $SO(3)$ kan skrivas som $L = L_{\gamma}^{3}L_{\beta}^{1}L_{\alpha}^{3}$, där de tre vinklarna kallas Eulervinklar.

\paragraph{Tensorprodukt}
Tensorprodukten mellan två tensorer av ordning $1$ definieras som
\begin{align*}
	[\tenprod{\vb{a}}{\vb{b}}]_{ij} = a_ib_j.
\end{align*}
Vi tar det ssom ett axiom att tensorprodukten bevarar linjäritet.

\paragraph{Generering av tensorer}
Tensorer av ordning $2$ genereras med hjälp av tensorprodukten mellan två tensorer av ordning 1. Vi kan då skriva
\begin{align*}
	T = T_{ij}\tenprod{\vb{e}_i}{\vb{e}_j}.
\end{align*}
Vi låter alla $\tenprod{\vb{e}_i}{\vb{e}_j}$ vara en bas för vektorrummet $\tenprod{V}{V}$.

\paragraph{Transformation av tensorer}
Vi har redan definierat hur man transformerar vektorer. Vi definierar transformationen av en tensor av ordning $2$ som
\begin{align*}
	T_{ij}' = L_{ik}L_{jl}T_{kl}.
\end{align*}
För tensorer av ordning $3$ definieras det som
\begin{align*}
	T_{ijk}' = L_{il}L_{jm}L_{kn}T_{lmn}.
\end{align*}